\section{Week 40}

Several trial runs for integrating trajectories in parallel using adaptive
timestep integrators suggest that attempting steps with all particles each
time is generally more efficient in terms of computation time than the 
alternative of only stepping forwards in time with the trajectories which are 
yet to reach the end point. Although there is a benefit of approximately 
$20 \%$ for first and second order accurate adaptive integrators, there is no
significant difference for third order methods, whereas higher order methods
require about $30 \%$ \emph{more} computation time when only stepping forwards
with trajectories which are not at the end point. Alas, I see no real benefit
to applying masks to the integrated trajectories, taking into account the more
complex and less comprehensible code it requires.

\begin{table}[H]
    \centering
    \caption{Butcher tableau for the Dormand-Prince 5(4) adaptive timestep integrator.
        The first row of $b$-coefficients yield the \nth{5} order solution, 
    while the second row yields the \nth{4} order interpolant. See~\textcite{prince1981highorder}.}
    \label{tab:bucherdopri54}
    \(\renewcommand{\arraystretch}{2.5}
    \begin{array}{c|ccccccc}
        \toprule
        0 \\
        \dfrac{1}{5} & \dfrac{1}{5} \\
        \dfrac{3}{10} & \dfrac{3}{40} & \dfrac{9}{40} \\
        \dfrac{4}{5} & \dfrac{44}{45} & -\dfrac{56}{15} & \dfrac{32}{9} \\
        \dfrac{8}{9} & \dfrac{19372}{6561} & -\dfrac{25360}{2187} & \dfrac{64448}{6561} & -\dfrac{212}{769} \\
        1 & \dfrac{9017}{3168} & -\dfrac{355}{33} & \dfrac{46732}{5247} & \dfrac{49}{176} & -\dfrac{5103}{18656} \\
        1 & \dfrac{35}{384} & 0 & \dfrac{500}{1113} & \dfrac{125}{192} & -\dfrac{2187}{6784} & \dfrac{11}{84} \\ 
        \midrule
        & \dfrac{35}{384} & 0 & \dfrac{500}{1113} & \dfrac{125}{192} & -\dfrac{2187}{6784} & \dfrac{11}{84} \\
        & \dfrac{5179}{57600} & 0 & \dfrac{7571}{16695} & \dfrac{393}{640} & -\dfrac{92097}{339200} & \dfrac{187}{2100} & \dfrac{1}{40} \\ 
        \bottomrule
    \end{array}
    \)
\end{table}


Second order forward difference approximation of first derivative:

\begin{equation}
    \label{eq:secondorderforwarddifferencefirstderivative}
    \dv[]{f(x)}{x} \approx \frac{-f(x + 2h) + 4f(x + h) - 3f(x)}{2h}     
\end{equation}

Second order backward difference approximation of first derivative:

\begin{equation}
    \label{eq:secondorderbackwarddifferencefirstderivative}
    \dv[]{f(x)}{x} \approx \frac{3f(x) - 4f(x - h) + f(x - 2h)}{2h} 
\end{equation}

Second order forward difference approximation of second derivative:

\begin{equation}
    \label{eq:secondorderforwarddifferencesecondderivative}
    \dv[2]{f(x)}{x} \approx \frac{2f(x) - 5f(x+h) + 4f(x + 2h) - f(x + 3h)}{h^{2}} 
\end{equation}

Second order backward difference approximation of second derivative:

\begin{equation}
    \label{eq:secondorderbackwarddifferencesecondderivative}
    \dv[2]{f(x)}{x} \approx \frac{2f(x) - 5f(x-h) + 4f(x-2h) - f(x-3h)}{h^{2}} 
\end{equation}

