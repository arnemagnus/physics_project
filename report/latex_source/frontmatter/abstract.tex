A generic flow system can be described as a structure whose state depends on
flowing streams of energy, material or information. Traditional examples of
such systems are the transport of properties such as pressure, temperature or
matter in fluids, and the transport of charge in electrical currents. However,
valuable insight into various phenomena, including, but not limited to,
algal blooms in the ocean and crowd patterns formed by humans, can
be obtained by regarding them as flow systems.

Lagrangian coherent structures (henceforth abbreviated to LCSs) can be described
as `landscapes' within a multidimensional space, which exert a major influence
upon the flow patterns in dynamical systems.
Compared to the conventional approach of increasing the spatial and temporal
resolution of the model(s) involved in numerical simulations, LCSs provide a
simplified means of predicting the future states of complex systems. In this
context, a complex system is a system which exhibits sensitive dependence
to initial conditions.
%Such structures provide a
%simplified means of predicting the future states of complex systems, which
%exhibit sensitive dependence to initial conditions, over a time interval of
%interest, compared to the conventional approach of increasing the spatial and
%temporal resolution of the model(s) involved in numerical simulations.
From their variational theory, hyperbolic LCSs are identified as locally most
repelling or attractive material surfaces. Somewhat simplified, such surfaces
can be considered as generalized trajectories, created by the underlying
transport phenomenon. When investigating transport systems, hyperbolic LCSs are
of particular interest, as they form the skeleton of observable flow patterns.
%Hyperbolic LCSs are of particular interest when
%investigating transport systems, as they form the skeleton of observable flow
%patterns.

This project is centered around the dependence of the numerical identification
of such LCSs upon which numerical integration scheme was used in order to
simulate the underlying transport phenomenon.
%them.
LCSs were computed and compared for an analytical double gyre velocity
field, for which reference LCSs are documented in the literature. In particular,
four traditional singlestep and four embedded, adaptive stepsize Runge-Kutta
methods were investigated. A numerical solution obtained by means of a high
order embedded method with a very small numerical tolerance --- two orders of
magnitude smaller than the smallest tolerance level utilized otherwise ---
and thus a very small expected numerical error, was utilized as a reference
solution for comparing the performance of the different methods.
%LCSs were computed and
%compared for an analytical velocity field, for which reference LCSs are
%documented in the literature.

It turns out that the LCSs for the system under consideration are quite robust
numerically. Thus, none of the numerical integration schemes stand out as the
absolutely best-suited. However, based upon the errors of the computed flow
maps, high order integration methods are generally advisable for generic flow
systems, as they invariably yield more efficient calculations, which are less
susceptible to numerical round-off error. Moreover, there is no apparent reason
to expect the LCSs of a generic flow system to be as robust as they were for the
system considered for this project work.

Note that the numerical implementation of one of the LCS existence conditions
which arise from their variational theory has not yet (to the author's
knowledge) been described completely in the literature.
%This is a
%consequence of the study of LCSs being in it relative infancy, as a branch of
%physics.
The approach considered in this project is based on careful inspection
of properties of the underlying system, which might not be applicable to
generic systems, depending on their scale. In short, there is currently a
lot of room for research within the field.
