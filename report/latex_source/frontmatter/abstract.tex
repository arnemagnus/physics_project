Lagrangian coherent structures (henceforth abbreviated as LCSs) can be described
as `landscapes' within a multidimensional space, which exert a major influence
upon the flow patterns in dynamical systems. Such structures provide a
simplified means of predicting the future states of complex systems, such as
nonlinear manybody problems, over a time interval of interest, compared to the
conventional approach of performing simulations for an increasing number of
particles. From their variational theory, hyperbolic LCSs are identified as
locally most repelling or attractive material surfaces, which form the skeleton
of observable tracer patterns.

This project is centered around the dependence of the numerical identification
of such LCSs upon which numerical integration scheme was used in calculating
them. LCSs were computed and compared for an analytical double gyre velocity
field, for which reference LCSs are documented in the literature. In particular,
four traditional singlestep and four embedded, adaptive
stepsize Runge-Kutta methods were investigated.
A numerical solution obtained by means of a high order embedded method with
a very small numerical tolerance --- two orders of magnitude smaller than the
smallest tolerance level utilized otherwise --- was utilized as the basis
for comparing the performance of the different methods.
%LCSs were computed and
%compared for an analytical velocity field, for which reference LCSs are
%documented in the literature.

It turns out that the LCSs for the system under consideration are quite robust
numerically, at least for the chosen parameter values. Thus, none of the
numerical integration schemes stand out as the absolutely best-suited. However,
based upon the errors of the computed flow maps, high order integration
methods are generally advisable for generic flow systems, as they invariably
yield more efficient calculations, which are less susceptible to numerical
round-off error. Moreover, there is no apparent reason to expect the LCSs of
a generic flow system to be as robust as they were for the system considered
for this project work.

Note that the numerical implementation of one of the LCS existence conditions
which arise from their variational theory has not yet received a complete
description in the literature. This is a consequence of the study of LCSs
being in it relative infancy, as a branch of physics. The approach
considered in this project is based on careful inspection of properties of the
underlying system, which may or may not be applicable to generic systems,
depending on their scale. In other words, there is currently a lot of room for
research within the field.

%Generally, the time step or tolerance level involved in the numerical
%integration scheme should be made based on physical considerations of the
%system for which LCSs are sought. In particular, when analyzing discrete
%systems, for which \emp{interpolation} is necessary in general, the
%\emph{interaction} between the chosen integration and interpolation schemes
%could have severe ramifications regarding the accuracy of the computed LCSs.
%This effect was not investigated
