This thesis is submitted as part of the formal requirements of the subject
TFY4510 --- `Physics, Specialization Project' at NTNU, accounting for a total
of 15 ECTS. All of the underlying work was performed in Trondheim, during the
fall semester of 2017, that is, the \nth{9} semester of my enrollment in the
5 year study programme culminating in a M.Sc.\ in `Physics and Mathematics',
with a specialization in Applied Physics. The program code which was developed
for this project is available at
github\footnote{\url{https://github.com/arnemagnus/physics_project}}; guidance
can be provided upon request.
%
%\parencite{vansebille2018lagrangian}
%\parencite{gough2017persistent}

I would like to extend my gratitude to my supervisor, Assoc.\ Prof.\
Tor Nordam. Our many discussions have provided me with great insight, while
his great knowledge, and humour proved excellent tools in terms
of relaxing my early-semester nerves.
%The reason for which I sought to work with
%him for my specialization project, is how his main current research interests,
%in particular, numerical simulations on oceanic transport processes, coincide
%well with my own glowing interest for the subject of computational physics.
%Furthermore, in spite of my project work turning out to be somewhat more
%theoretically involved exercise than either of us might have anticipated in
%advance, there was never any doubts regarding the applicability of my studies to
%further our understanding of naturally occurring transport phenomena --- which
%has been a great motivational factor of mine.
Coupled with his expertise with regards to the use supercomputers for scientific
purposes --- without the use of which, by the
way, several of the computations constituting this thesis would not have been
practically feasible --- I could not be more satisfied. I am already looking
forwards to the continuation of our cooperation, for the upcoming spring term.

My dear friend and fellow student Simon Nordgreen, who has also performed his
project work under the guidance of Assoc.\ Prof.\ Nordam, definitely
deserves a mention. His and my main research topics are closely correlated,
which lead to the two of us to many laborious, yet fruitful, collaborations
regarding our understanding of the underlying variational principles. I think it
is safe to say that two heads think better than one. To all the friends and
acquaintances I have had the pleasure of getting to know during my
time in Trondheim, I extend my thanks.

Last, but most certainly not least, I gratefully thank my family for their
unconditional support and encouragement throughout my time as a university
student.


\begin{minipage}[t]{\textwidth}
    \begin{flushright}
    Trondheim, December 2017\\
    Arne Magnus Tveita Løken
    \end{flushright}
\end{minipage}
