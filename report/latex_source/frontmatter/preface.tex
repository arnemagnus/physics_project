This thesis is submitted as part of the formal requirements of the subject
TFY4510 --- `Physics, Specialization Project' at NTNU, accounting for a total
of 15 ECTS. All of the underlying work was performed in Trondheim, during the
fall semester of 2017, that is, the \nth{9} semester of my enrollment in the
5 year study programme culminating in a M.Sc.\ in `Physics and Mathematics',
with a specialization in Applied Physics. The program code which was developed
for this project is available at
github\footnote{\url{https://github.com/arnemagnus/physics_project}}; guidance
can be provided upon request.
%
%\parencite{vansebille2018lagrangian}
%\parencite{gough2017persistent}

I would like to extend my gratitude to my supervisor, Assoc.\ Prof.\
Tor Nordam. Our many discussions have provided me with great insight, while
his great knowledge, and humour proved excellent tools in terms
of relaxing my early-semester nerves.
%The reason for which I sought to work with
%him for my specialization project, is how his main current research interests,
%in particular, numerical simulations on oceanic transport processes, coincide
%well with my own glowing interest for the subject of computational physics.
%Furthermore, in spite of my project work turning out to be somewhat more
%theoretically involved exercise than either of us might have anticipated in
%advance, there was never any doubts regarding the applicability of my studies to
%further our understanding of naturally occurring transport phenomena --- which
%has been a great motivational factor of mine.
Coupled with his expertise with regards to the use supercomputers for scientific
purposes --- without the use of which, by the
way, several of the computations constituting this thesis would not have been
practically feasible --- I could not be more satisfied. I am already looking
forwards to the continuation of our cooperation, for the upcoming spring term.

My dear friend and fellow student Simon Nordgreen, who has also performed his
project work under the guidance of Assoc.\ Prof.\ Nordam, definitely
deserves a mention. His and my main research topics are closely correlated,
which lead to the two of us to many laborious, yet fruitful, collaborations
regarding our understanding of the underlying variational principles. I think it
is safe to say that two heads think better than one. To all the friends and
acquaintances I have had the pleasure of getting to know during my
time in Trondheim, I extend my thanks.

Last, but most certainly not least, I gratefully thank my family for their
unconditional support and encouragement throughout my time as a university
student.


\begin{minipage}[t]{\textwidth}
    \begin{flushright}
    Trondheim, December 2017\\
    Arne Magnus Tveita Løken
    \end{flushright}
\end{minipage}\\[2cm]

\begin{figure}[htpb]
    \centering
    %% Creator: Matplotlib, PGF backend
%%
%% To include the figure in your LaTeX document, write
%%   \input{<filename>.pgf}
%%
%% Make sure the required packages are loaded in your preamble
%%   \usepackage{pgf}
%%
%% Figures using additional raster images can only be included by \input if
%% they are in the same directory as the main LaTeX file. For loading figures
%% from other directories you can use the `import` package
%%   \usepackage{import}
%% and then include the figures with
%%   \import{<path to file>}{<filename>.pgf}
%%
%% Matplotlib used the following preamble
%%   \usepackage[utf8x]{inputenc}
%%   \usepackage[T1]{fontenc}
%%   \usepackage[]{libertine}\usepackage[libertine]{newtxmath}
%%
\begingroup%
\makeatletter%
\begin{pgfpicture}%
\pgfpathrectangle{\pgfpointorigin}{\pgfqpoint{5.050000in}{3.100000in}}%
\pgfusepath{use as bounding box, clip}%
\begin{pgfscope}%
\pgfsetbuttcap%
\pgfsetmiterjoin%
\definecolor{currentfill}{rgb}{1.000000,1.000000,1.000000}%
\pgfsetfillcolor{currentfill}%
\pgfsetlinewidth{0.000000pt}%
\definecolor{currentstroke}{rgb}{1.000000,1.000000,1.000000}%
\pgfsetstrokecolor{currentstroke}%
\pgfsetdash{}{0pt}%
\pgfpathmoveto{\pgfqpoint{0.000000in}{0.000000in}}%
\pgfpathlineto{\pgfqpoint{5.050000in}{0.000000in}}%
\pgfpathlineto{\pgfqpoint{5.050000in}{3.100000in}}%
\pgfpathlineto{\pgfqpoint{0.000000in}{3.100000in}}%
\pgfpathclose%
\pgfusepath{fill}%
\end{pgfscope}%
\begin{pgfscope}%
\pgfsetbuttcap%
\pgfsetmiterjoin%
\definecolor{currentfill}{rgb}{1.000000,1.000000,1.000000}%
\pgfsetfillcolor{currentfill}%
\pgfsetlinewidth{0.000000pt}%
\definecolor{currentstroke}{rgb}{0.000000,0.000000,0.000000}%
\pgfsetstrokecolor{currentstroke}%
\pgfsetstrokeopacity{0.000000}%
\pgfsetdash{}{0pt}%
\pgfpathmoveto{\pgfqpoint{0.404000in}{0.248000in}}%
\pgfpathlineto{\pgfqpoint{2.650634in}{0.248000in}}%
\pgfpathlineto{\pgfqpoint{2.650634in}{1.610506in}}%
\pgfpathlineto{\pgfqpoint{0.404000in}{1.610506in}}%
\pgfpathclose%
\pgfusepath{fill}%
\end{pgfscope}%
\begin{pgfscope}%
\pgfsetbuttcap%
\pgfsetroundjoin%
\definecolor{currentfill}{rgb}{0.000000,0.000000,0.000000}%
\pgfsetfillcolor{currentfill}%
\pgfsetlinewidth{0.803000pt}%
\definecolor{currentstroke}{rgb}{0.000000,0.000000,0.000000}%
\pgfsetstrokecolor{currentstroke}%
\pgfsetdash{}{0pt}%
\pgfsys@defobject{currentmarker}{\pgfqpoint{0.000000in}{-0.048611in}}{\pgfqpoint{0.000000in}{0.000000in}}{%
\pgfpathmoveto{\pgfqpoint{0.000000in}{0.000000in}}%
\pgfpathlineto{\pgfqpoint{0.000000in}{-0.048611in}}%
\pgfusepath{stroke,fill}%
}%
\begin{pgfscope}%
\pgfsys@transformshift{0.528813in}{0.248000in}%
\pgfsys@useobject{currentmarker}{}%
\end{pgfscope}%
\end{pgfscope}%
\begin{pgfscope}%
\pgftext[x=0.528813in,y=0.150778in,,top]{\rmfamily\fontsize{10.000000}{12.000000}\selectfont \(\displaystyle 0.30\)}%
\end{pgfscope}%
\begin{pgfscope}%
\pgfsetbuttcap%
\pgfsetroundjoin%
\definecolor{currentfill}{rgb}{0.000000,0.000000,0.000000}%
\pgfsetfillcolor{currentfill}%
\pgfsetlinewidth{0.803000pt}%
\definecolor{currentstroke}{rgb}{0.000000,0.000000,0.000000}%
\pgfsetstrokecolor{currentstroke}%
\pgfsetdash{}{0pt}%
\pgfsys@defobject{currentmarker}{\pgfqpoint{0.000000in}{-0.048611in}}{\pgfqpoint{0.000000in}{0.000000in}}{%
\pgfpathmoveto{\pgfqpoint{0.000000in}{0.000000in}}%
\pgfpathlineto{\pgfqpoint{0.000000in}{-0.048611in}}%
\pgfusepath{stroke,fill}%
}%
\begin{pgfscope}%
\pgfsys@transformshift{0.903252in}{0.248000in}%
\pgfsys@useobject{currentmarker}{}%
\end{pgfscope}%
\end{pgfscope}%
\begin{pgfscope}%
\pgftext[x=0.903252in,y=0.150778in,,top]{\rmfamily\fontsize{10.000000}{12.000000}\selectfont \(\displaystyle 0.45\)}%
\end{pgfscope}%
\begin{pgfscope}%
\pgfsetbuttcap%
\pgfsetroundjoin%
\definecolor{currentfill}{rgb}{0.000000,0.000000,0.000000}%
\pgfsetfillcolor{currentfill}%
\pgfsetlinewidth{0.803000pt}%
\definecolor{currentstroke}{rgb}{0.000000,0.000000,0.000000}%
\pgfsetstrokecolor{currentstroke}%
\pgfsetdash{}{0pt}%
\pgfsys@defobject{currentmarker}{\pgfqpoint{0.000000in}{-0.048611in}}{\pgfqpoint{0.000000in}{0.000000in}}{%
\pgfpathmoveto{\pgfqpoint{0.000000in}{0.000000in}}%
\pgfpathlineto{\pgfqpoint{0.000000in}{-0.048611in}}%
\pgfusepath{stroke,fill}%
}%
\begin{pgfscope}%
\pgfsys@transformshift{1.277691in}{0.248000in}%
\pgfsys@useobject{currentmarker}{}%
\end{pgfscope}%
\end{pgfscope}%
\begin{pgfscope}%
\pgftext[x=1.277691in,y=0.150778in,,top]{\rmfamily\fontsize{10.000000}{12.000000}\selectfont \(\displaystyle 0.60\)}%
\end{pgfscope}%
\begin{pgfscope}%
\pgfsetbuttcap%
\pgfsetroundjoin%
\definecolor{currentfill}{rgb}{0.000000,0.000000,0.000000}%
\pgfsetfillcolor{currentfill}%
\pgfsetlinewidth{0.803000pt}%
\definecolor{currentstroke}{rgb}{0.000000,0.000000,0.000000}%
\pgfsetstrokecolor{currentstroke}%
\pgfsetdash{}{0pt}%
\pgfsys@defobject{currentmarker}{\pgfqpoint{0.000000in}{-0.048611in}}{\pgfqpoint{0.000000in}{0.000000in}}{%
\pgfpathmoveto{\pgfqpoint{0.000000in}{0.000000in}}%
\pgfpathlineto{\pgfqpoint{0.000000in}{-0.048611in}}%
\pgfusepath{stroke,fill}%
}%
\begin{pgfscope}%
\pgfsys@transformshift{1.652130in}{0.248000in}%
\pgfsys@useobject{currentmarker}{}%
\end{pgfscope}%
\end{pgfscope}%
\begin{pgfscope}%
\pgftext[x=1.652130in,y=0.150778in,,top]{\rmfamily\fontsize{10.000000}{12.000000}\selectfont \(\displaystyle 0.75\)}%
\end{pgfscope}%
\begin{pgfscope}%
\pgfsetbuttcap%
\pgfsetroundjoin%
\definecolor{currentfill}{rgb}{0.000000,0.000000,0.000000}%
\pgfsetfillcolor{currentfill}%
\pgfsetlinewidth{0.803000pt}%
\definecolor{currentstroke}{rgb}{0.000000,0.000000,0.000000}%
\pgfsetstrokecolor{currentstroke}%
\pgfsetdash{}{0pt}%
\pgfsys@defobject{currentmarker}{\pgfqpoint{0.000000in}{-0.048611in}}{\pgfqpoint{0.000000in}{0.000000in}}{%
\pgfpathmoveto{\pgfqpoint{0.000000in}{0.000000in}}%
\pgfpathlineto{\pgfqpoint{0.000000in}{-0.048611in}}%
\pgfusepath{stroke,fill}%
}%
\begin{pgfscope}%
\pgfsys@transformshift{2.026569in}{0.248000in}%
\pgfsys@useobject{currentmarker}{}%
\end{pgfscope}%
\end{pgfscope}%
\begin{pgfscope}%
\pgftext[x=2.026569in,y=0.150778in,,top]{\rmfamily\fontsize{10.000000}{12.000000}\selectfont \(\displaystyle 0.90\)}%
\end{pgfscope}%
\begin{pgfscope}%
\pgfsetbuttcap%
\pgfsetroundjoin%
\definecolor{currentfill}{rgb}{0.000000,0.000000,0.000000}%
\pgfsetfillcolor{currentfill}%
\pgfsetlinewidth{0.803000pt}%
\definecolor{currentstroke}{rgb}{0.000000,0.000000,0.000000}%
\pgfsetstrokecolor{currentstroke}%
\pgfsetdash{}{0pt}%
\pgfsys@defobject{currentmarker}{\pgfqpoint{0.000000in}{-0.048611in}}{\pgfqpoint{0.000000in}{0.000000in}}{%
\pgfpathmoveto{\pgfqpoint{0.000000in}{0.000000in}}%
\pgfpathlineto{\pgfqpoint{0.000000in}{-0.048611in}}%
\pgfusepath{stroke,fill}%
}%
\begin{pgfscope}%
\pgfsys@transformshift{2.401008in}{0.248000in}%
\pgfsys@useobject{currentmarker}{}%
\end{pgfscope}%
\end{pgfscope}%
\begin{pgfscope}%
\pgftext[x=2.401008in,y=0.150778in,,top]{\rmfamily\fontsize{10.000000}{12.000000}\selectfont \(\displaystyle 1.05\)}%
\end{pgfscope}%
\begin{pgfscope}%
\pgfsetbuttcap%
\pgfsetroundjoin%
\definecolor{currentfill}{rgb}{0.000000,0.000000,0.000000}%
\pgfsetfillcolor{currentfill}%
\pgfsetlinewidth{0.803000pt}%
\definecolor{currentstroke}{rgb}{0.000000,0.000000,0.000000}%
\pgfsetstrokecolor{currentstroke}%
\pgfsetdash{}{0pt}%
\pgfsys@defobject{currentmarker}{\pgfqpoint{-0.048611in}{0.000000in}}{\pgfqpoint{0.000000in}{0.000000in}}{%
\pgfpathmoveto{\pgfqpoint{0.000000in}{0.000000in}}%
\pgfpathlineto{\pgfqpoint{-0.048611in}{0.000000in}}%
\pgfusepath{stroke,fill}%
}%
\begin{pgfscope}%
\pgfsys@transformshift{0.404000in}{0.495728in}%
\pgfsys@useobject{currentmarker}{}%
\end{pgfscope}%
\end{pgfscope}%
\begin{pgfscope}%
\pgftext[x=0.073723in,y=0.447777in,left,base]{\rmfamily\fontsize{10.000000}{12.000000}\selectfont \(\displaystyle 0.80\)}%
\end{pgfscope}%
\begin{pgfscope}%
\pgfsetbuttcap%
\pgfsetroundjoin%
\definecolor{currentfill}{rgb}{0.000000,0.000000,0.000000}%
\pgfsetfillcolor{currentfill}%
\pgfsetlinewidth{0.803000pt}%
\definecolor{currentstroke}{rgb}{0.000000,0.000000,0.000000}%
\pgfsetstrokecolor{currentstroke}%
\pgfsetdash{}{0pt}%
\pgfsys@defobject{currentmarker}{\pgfqpoint{-0.048611in}{0.000000in}}{\pgfqpoint{0.000000in}{0.000000in}}{%
\pgfpathmoveto{\pgfqpoint{0.000000in}{0.000000in}}%
\pgfpathlineto{\pgfqpoint{-0.048611in}{0.000000in}}%
\pgfusepath{stroke,fill}%
}%
\begin{pgfscope}%
\pgfsys@transformshift{0.404000in}{0.805389in}%
\pgfsys@useobject{currentmarker}{}%
\end{pgfscope}%
\end{pgfscope}%
\begin{pgfscope}%
\pgftext[x=0.073723in,y=0.757438in,left,base]{\rmfamily\fontsize{10.000000}{12.000000}\selectfont \(\displaystyle 0.85\)}%
\end{pgfscope}%
\begin{pgfscope}%
\pgfsetbuttcap%
\pgfsetroundjoin%
\definecolor{currentfill}{rgb}{0.000000,0.000000,0.000000}%
\pgfsetfillcolor{currentfill}%
\pgfsetlinewidth{0.803000pt}%
\definecolor{currentstroke}{rgb}{0.000000,0.000000,0.000000}%
\pgfsetstrokecolor{currentstroke}%
\pgfsetdash{}{0pt}%
\pgfsys@defobject{currentmarker}{\pgfqpoint{-0.048611in}{0.000000in}}{\pgfqpoint{0.000000in}{0.000000in}}{%
\pgfpathmoveto{\pgfqpoint{0.000000in}{0.000000in}}%
\pgfpathlineto{\pgfqpoint{-0.048611in}{0.000000in}}%
\pgfusepath{stroke,fill}%
}%
\begin{pgfscope}%
\pgfsys@transformshift{0.404000in}{1.115049in}%
\pgfsys@useobject{currentmarker}{}%
\end{pgfscope}%
\end{pgfscope}%
\begin{pgfscope}%
\pgftext[x=0.073723in,y=1.067098in,left,base]{\rmfamily\fontsize{10.000000}{12.000000}\selectfont \(\displaystyle 0.90\)}%
\end{pgfscope}%
\begin{pgfscope}%
\pgfsetbuttcap%
\pgfsetroundjoin%
\definecolor{currentfill}{rgb}{0.000000,0.000000,0.000000}%
\pgfsetfillcolor{currentfill}%
\pgfsetlinewidth{0.803000pt}%
\definecolor{currentstroke}{rgb}{0.000000,0.000000,0.000000}%
\pgfsetstrokecolor{currentstroke}%
\pgfsetdash{}{0pt}%
\pgfsys@defobject{currentmarker}{\pgfqpoint{-0.048611in}{0.000000in}}{\pgfqpoint{0.000000in}{0.000000in}}{%
\pgfpathmoveto{\pgfqpoint{0.000000in}{0.000000in}}%
\pgfpathlineto{\pgfqpoint{-0.048611in}{0.000000in}}%
\pgfusepath{stroke,fill}%
}%
\begin{pgfscope}%
\pgfsys@transformshift{0.404000in}{1.424710in}%
\pgfsys@useobject{currentmarker}{}%
\end{pgfscope}%
\end{pgfscope}%
\begin{pgfscope}%
\pgftext[x=0.073723in,y=1.376758in,left,base]{\rmfamily\fontsize{10.000000}{12.000000}\selectfont \(\displaystyle 0.95\)}%
\end{pgfscope}%
\begin{pgfscope}%
\pgfpathrectangle{\pgfqpoint{0.404000in}{0.248000in}}{\pgfqpoint{2.246634in}{1.362506in}} %
\pgfusepath{clip}%
\pgfsetrectcap%
\pgfsetroundjoin%
\pgfsetlinewidth{1.505625pt}%
\definecolor{currentstroke}{rgb}{0.121569,0.466667,0.705882}%
\pgfsetstrokecolor{currentstroke}%
\pgfsetdash{}{0pt}%
\pgfusepath{stroke}%
\end{pgfscope}%
\begin{pgfscope}%
\pgfpathrectangle{\pgfqpoint{0.404000in}{0.248000in}}{\pgfqpoint{2.246634in}{1.362506in}} %
\pgfusepath{clip}%
\pgfsetrectcap%
\pgfsetroundjoin%
\pgfsetlinewidth{1.505625pt}%
\definecolor{currentstroke}{rgb}{1.000000,0.498039,0.054902}%
\pgfsetstrokecolor{currentstroke}%
\pgfsetdash{}{0pt}%
\pgfusepath{stroke}%
\end{pgfscope}%
\begin{pgfscope}%
\pgfpathrectangle{\pgfqpoint{0.404000in}{0.248000in}}{\pgfqpoint{2.246634in}{1.362506in}} %
\pgfusepath{clip}%
\pgfsetrectcap%
\pgfsetroundjoin%
\pgfsetlinewidth{1.505625pt}%
\definecolor{currentstroke}{rgb}{0.172549,0.627451,0.172549}%
\pgfsetstrokecolor{currentstroke}%
\pgfsetdash{}{0pt}%
\pgfusepath{stroke}%
\end{pgfscope}%
\begin{pgfscope}%
\pgfpathrectangle{\pgfqpoint{0.404000in}{0.248000in}}{\pgfqpoint{2.246634in}{1.362506in}} %
\pgfusepath{clip}%
\pgfsetrectcap%
\pgfsetroundjoin%
\pgfsetlinewidth{1.003750pt}%
\definecolor{currentstroke}{rgb}{1.000000,0.388235,0.278431}%
\pgfsetstrokecolor{currentstroke}%
\pgfsetdash{}{0pt}%
\pgfpathmoveto{\pgfqpoint{0.558748in}{0.938981in}}%
\pgfpathlineto{\pgfqpoint{0.586949in}{0.978804in}}%
\pgfpathlineto{\pgfqpoint{0.615805in}{1.015625in}}%
\pgfpathlineto{\pgfqpoint{0.645222in}{1.049607in}}%
\pgfpathlineto{\pgfqpoint{0.675115in}{1.080930in}}%
\pgfpathlineto{\pgfqpoint{0.707755in}{1.111901in}}%
\pgfpathlineto{\pgfqpoint{0.740783in}{1.140232in}}%
\pgfpathlineto{\pgfqpoint{0.776525in}{1.167912in}}%
\pgfpathlineto{\pgfqpoint{0.812568in}{1.193078in}}%
\pgfpathlineto{\pgfqpoint{0.851283in}{1.217417in}}%
\pgfpathlineto{\pgfqpoint{0.892665in}{1.240734in}}%
\pgfpathlineto{\pgfqpoint{0.936702in}{1.262869in}}%
\pgfpathlineto{\pgfqpoint{0.983381in}{1.283701in}}%
\pgfpathlineto{\pgfqpoint{1.032686in}{1.303140in}}%
\pgfpathlineto{\pgfqpoint{1.084604in}{1.321119in}}%
\pgfpathlineto{\pgfqpoint{1.139118in}{1.337594in}}%
\pgfpathlineto{\pgfqpoint{1.196215in}{1.352531in}}%
\pgfpathlineto{\pgfqpoint{1.258369in}{1.366416in}}%
\pgfpathlineto{\pgfqpoint{1.323087in}{1.378544in}}%
\pgfpathlineto{\pgfqpoint{1.390356in}{1.388899in}}%
\pgfpathlineto{\pgfqpoint{1.460166in}{1.397450in}}%
\pgfpathlineto{\pgfqpoint{1.535002in}{1.404338in}}%
\pgfpathlineto{\pgfqpoint{1.609866in}{1.409003in}}%
\pgfpathlineto{\pgfqpoint{1.687242in}{1.411573in}}%
\pgfpathlineto{\pgfqpoint{1.764626in}{1.411865in}}%
\pgfpathlineto{\pgfqpoint{1.839509in}{1.409895in}}%
\pgfpathlineto{\pgfqpoint{1.911881in}{1.405732in}}%
\pgfpathlineto{\pgfqpoint{1.979235in}{1.399656in}}%
\pgfpathlineto{\pgfqpoint{2.041562in}{1.391867in}}%
\pgfpathlineto{\pgfqpoint{2.098853in}{1.382570in}}%
\pgfpathlineto{\pgfqpoint{2.151099in}{1.371968in}}%
\pgfpathlineto{\pgfqpoint{2.200775in}{1.359617in}}%
\pgfpathlineto{\pgfqpoint{2.245381in}{1.346216in}}%
\pgfpathlineto{\pgfqpoint{2.284911in}{1.332076in}}%
\pgfpathlineto{\pgfqpoint{2.321822in}{1.316473in}}%
\pgfpathlineto{\pgfqpoint{2.353642in}{1.300685in}}%
\pgfpathlineto{\pgfqpoint{2.382812in}{1.283802in}}%
\pgfpathlineto{\pgfqpoint{2.409307in}{1.265921in}}%
\pgfpathlineto{\pgfqpoint{2.433100in}{1.247202in}}%
\pgfpathlineto{\pgfqpoint{2.454172in}{1.227887in}}%
\pgfpathlineto{\pgfqpoint{2.472517in}{1.208329in}}%
\pgfpathlineto{\pgfqpoint{2.490351in}{1.186055in}}%
\pgfpathlineto{\pgfqpoint{2.505371in}{1.163918in}}%
\pgfpathlineto{\pgfqpoint{2.519635in}{1.138901in}}%
\pgfpathlineto{\pgfqpoint{2.532874in}{1.110636in}}%
\pgfpathlineto{\pgfqpoint{2.543155in}{1.083635in}}%
\pgfpathlineto{\pgfqpoint{2.552215in}{1.054070in}}%
\pgfpathlineto{\pgfqpoint{2.559861in}{1.022142in}}%
\pgfpathlineto{\pgfqpoint{2.566850in}{0.982443in}}%
\pgfpathlineto{\pgfqpoint{2.571775in}{0.940874in}}%
\pgfpathlineto{\pgfqpoint{2.575123in}{0.892057in}}%
\pgfpathlineto{\pgfqpoint{2.576499in}{0.836446in}}%
\pgfpathlineto{\pgfqpoint{2.575786in}{0.774556in}}%
\pgfpathlineto{\pgfqpoint{2.572428in}{0.694492in}}%
\pgfpathlineto{\pgfqpoint{2.565951in}{0.596723in}}%
\pgfpathlineto{\pgfqpoint{2.554317in}{0.457240in}}%
\pgfpathlineto{\pgfqpoint{2.534230in}{0.238000in}}%
\pgfpathlineto{\pgfqpoint{2.534230in}{0.238000in}}%
\pgfusepath{stroke}%
\end{pgfscope}%
\begin{pgfscope}%
\pgfpathrectangle{\pgfqpoint{0.404000in}{0.248000in}}{\pgfqpoint{2.246634in}{1.362506in}} %
\pgfusepath{clip}%
\pgfsetrectcap%
\pgfsetroundjoin%
\pgfsetlinewidth{1.003750pt}%
\definecolor{currentstroke}{rgb}{1.000000,0.388235,0.278431}%
\pgfsetstrokecolor{currentstroke}%
\pgfsetdash{}{0pt}%
\pgfpathmoveto{\pgfqpoint{2.350898in}{0.238000in}}%
\pgfpathlineto{\pgfqpoint{2.346074in}{0.331718in}}%
\pgfpathlineto{\pgfqpoint{2.339588in}{0.416901in}}%
\pgfpathlineto{\pgfqpoint{2.332042in}{0.488807in}}%
\pgfpathlineto{\pgfqpoint{2.323181in}{0.553270in}}%
\pgfpathlineto{\pgfqpoint{2.313276in}{0.610098in}}%
\pgfpathlineto{\pgfqpoint{2.301449in}{0.664615in}}%
\pgfpathlineto{\pgfqpoint{2.289108in}{0.711171in}}%
\pgfpathlineto{\pgfqpoint{2.275188in}{0.754899in}}%
\pgfpathlineto{\pgfqpoint{2.259796in}{0.795477in}}%
\pgfpathlineto{\pgfqpoint{2.243094in}{0.832734in}}%
\pgfpathlineto{\pgfqpoint{2.225274in}{0.866658in}}%
\pgfpathlineto{\pgfqpoint{2.206530in}{0.897367in}}%
\pgfpathlineto{\pgfqpoint{2.187039in}{0.925070in}}%
\pgfpathlineto{\pgfqpoint{2.166952in}{0.950019in}}%
\pgfpathlineto{\pgfqpoint{2.144084in}{0.974836in}}%
\pgfpathlineto{\pgfqpoint{2.120766in}{0.996934in}}%
\pgfpathlineto{\pgfqpoint{2.094720in}{1.018486in}}%
\pgfpathlineto{\pgfqpoint{2.068355in}{1.037513in}}%
\pgfpathlineto{\pgfqpoint{2.039319in}{1.055762in}}%
\pgfpathlineto{\pgfqpoint{2.007619in}{1.072969in}}%
\pgfpathlineto{\pgfqpoint{1.973269in}{1.088917in}}%
\pgfpathlineto{\pgfqpoint{1.938756in}{1.102539in}}%
\pgfpathlineto{\pgfqpoint{1.901645in}{1.114881in}}%
\pgfpathlineto{\pgfqpoint{1.861948in}{1.125783in}}%
\pgfpathlineto{\pgfqpoint{1.819679in}{1.135103in}}%
\pgfpathlineto{\pgfqpoint{1.774851in}{1.142695in}}%
\pgfpathlineto{\pgfqpoint{1.727479in}{1.148420in}}%
\pgfpathlineto{\pgfqpoint{1.677577in}{1.152108in}}%
\pgfpathlineto{\pgfqpoint{1.627656in}{1.153558in}}%
\pgfpathlineto{\pgfqpoint{1.575236in}{1.152791in}}%
\pgfpathlineto{\pgfqpoint{1.522829in}{1.149743in}}%
\pgfpathlineto{\pgfqpoint{1.470452in}{1.144441in}}%
\pgfpathlineto{\pgfqpoint{1.418121in}{1.136846in}}%
\pgfpathlineto{\pgfqpoint{1.368341in}{1.127426in}}%
\pgfpathlineto{\pgfqpoint{1.318638in}{1.115770in}}%
\pgfpathlineto{\pgfqpoint{1.271512in}{1.102505in}}%
\pgfpathlineto{\pgfqpoint{1.224500in}{1.086956in}}%
\pgfpathlineto{\pgfqpoint{1.180093in}{1.069958in}}%
\pgfpathlineto{\pgfqpoint{1.135847in}{1.050561in}}%
\pgfpathlineto{\pgfqpoint{1.094241in}{1.029846in}}%
\pgfpathlineto{\pgfqpoint{1.055282in}{1.008029in}}%
\pgfpathlineto{\pgfqpoint{1.016560in}{0.983757in}}%
\pgfpathlineto{\pgfqpoint{0.980520in}{0.958571in}}%
\pgfpathlineto{\pgfqpoint{0.944787in}{0.930821in}}%
\pgfpathlineto{\pgfqpoint{0.911770in}{0.902411in}}%
\pgfpathlineto{\pgfqpoint{0.879139in}{0.871378in}}%
\pgfpathlineto{\pgfqpoint{0.849250in}{0.840035in}}%
\pgfpathlineto{\pgfqpoint{0.819826in}{0.806089in}}%
\pgfpathlineto{\pgfqpoint{0.790946in}{0.769378in}}%
\pgfpathlineto{\pgfqpoint{0.762701in}{0.729750in}}%
\pgfpathlineto{\pgfqpoint{0.735188in}{0.687068in}}%
\pgfpathlineto{\pgfqpoint{0.710535in}{0.644864in}}%
\pgfpathlineto{\pgfqpoint{0.686686in}{0.599906in}}%
\pgfpathlineto{\pgfqpoint{0.663731in}{0.552170in}}%
\pgfpathlineto{\pgfqpoint{0.641759in}{0.501668in}}%
\pgfpathlineto{\pgfqpoint{0.619162in}{0.443898in}}%
\pgfpathlineto{\pgfqpoint{0.597910in}{0.383067in}}%
\pgfpathlineto{\pgfqpoint{0.578081in}{0.319347in}}%
\pgfpathlineto{\pgfqpoint{0.559733in}{0.252952in}}%
\pgfpathlineto{\pgfqpoint{0.555898in}{0.238000in}}%
\pgfpathmoveto{\pgfqpoint{0.430052in}{0.238000in}}%
\pgfpathlineto{\pgfqpoint{0.446814in}{0.312934in}}%
\pgfpathlineto{\pgfqpoint{0.465426in}{0.386306in}}%
\pgfpathlineto{\pgfqpoint{0.485704in}{0.456908in}}%
\pgfpathlineto{\pgfqpoint{0.506000in}{0.519716in}}%
\pgfpathlineto{\pgfqpoint{0.527663in}{0.579649in}}%
\pgfpathlineto{\pgfqpoint{0.550624in}{0.636529in}}%
\pgfpathlineto{\pgfqpoint{0.574797in}{0.690230in}}%
\pgfpathlineto{\pgfqpoint{0.600081in}{0.740684in}}%
\pgfpathlineto{\pgfqpoint{0.626367in}{0.787883in}}%
\pgfpathlineto{\pgfqpoint{0.653543in}{0.831871in}}%
\pgfpathlineto{\pgfqpoint{0.681500in}{0.872740in}}%
\pgfpathlineto{\pgfqpoint{0.710133in}{0.910616in}}%
\pgfpathlineto{\pgfqpoint{0.739349in}{0.945652in}}%
\pgfpathlineto{\pgfqpoint{0.769061in}{0.978015in}}%
\pgfpathlineto{\pgfqpoint{0.801530in}{1.010079in}}%
\pgfpathlineto{\pgfqpoint{0.834408in}{1.039464in}}%
\pgfpathlineto{\pgfqpoint{0.870011in}{1.068216in}}%
\pgfpathlineto{\pgfqpoint{0.905938in}{1.094382in}}%
\pgfpathlineto{\pgfqpoint{0.944551in}{1.119701in}}%
\pgfpathlineto{\pgfqpoint{0.983410in}{1.142591in}}%
\pgfpathlineto{\pgfqpoint{1.024916in}{1.164503in}}%
\pgfpathlineto{\pgfqpoint{1.069062in}{1.185263in}}%
\pgfpathlineto{\pgfqpoint{1.115837in}{1.204721in}}%
\pgfpathlineto{\pgfqpoint{1.165230in}{1.222754in}}%
\pgfpathlineto{\pgfqpoint{1.217227in}{1.239252in}}%
\pgfpathlineto{\pgfqpoint{1.271816in}{1.254121in}}%
\pgfpathlineto{\pgfqpoint{1.326497in}{1.266744in}}%
\pgfpathlineto{\pgfqpoint{1.383739in}{1.277739in}}%
\pgfpathlineto{\pgfqpoint{1.443533in}{1.286990in}}%
\pgfpathlineto{\pgfqpoint{1.505868in}{1.294358in}}%
\pgfpathlineto{\pgfqpoint{1.568239in}{1.299514in}}%
\pgfpathlineto{\pgfqpoint{1.630634in}{1.302509in}}%
\pgfpathlineto{\pgfqpoint{1.693039in}{1.303343in}}%
\pgfpathlineto{\pgfqpoint{1.755442in}{1.301955in}}%
\pgfpathlineto{\pgfqpoint{1.815335in}{1.298414in}}%
\pgfpathlineto{\pgfqpoint{1.872704in}{1.292821in}}%
\pgfpathlineto{\pgfqpoint{1.927536in}{1.285234in}}%
\pgfpathlineto{\pgfqpoint{1.977327in}{1.276188in}}%
\pgfpathlineto{\pgfqpoint{2.024556in}{1.265416in}}%
\pgfpathlineto{\pgfqpoint{2.069205in}{1.252926in}}%
\pgfpathlineto{\pgfqpoint{2.108783in}{1.239630in}}%
\pgfpathlineto{\pgfqpoint{2.145755in}{1.224941in}}%
\pgfpathlineto{\pgfqpoint{2.180100in}{1.208922in}}%
\pgfpathlineto{\pgfqpoint{2.211795in}{1.191661in}}%
\pgfpathlineto{\pgfqpoint{2.240819in}{1.173287in}}%
\pgfpathlineto{\pgfqpoint{2.267150in}{1.153979in}}%
\pgfpathlineto{\pgfqpoint{2.290775in}{1.133992in}}%
\pgfpathlineto{\pgfqpoint{2.313990in}{1.111251in}}%
\pgfpathlineto{\pgfqpoint{2.334408in}{1.088012in}}%
\pgfpathlineto{\pgfqpoint{2.352045in}{1.064791in}}%
\pgfpathlineto{\pgfqpoint{2.369047in}{1.038820in}}%
\pgfpathlineto{\pgfqpoint{2.385221in}{1.009781in}}%
\pgfpathlineto{\pgfqpoint{2.398511in}{0.981649in}}%
\pgfpathlineto{\pgfqpoint{2.410812in}{0.950879in}}%
\pgfpathlineto{\pgfqpoint{2.421952in}{0.917500in}}%
\pgfpathlineto{\pgfqpoint{2.431788in}{0.881687in}}%
\pgfpathlineto{\pgfqpoint{2.441323in}{0.838179in}}%
\pgfpathlineto{\pgfqpoint{2.449029in}{0.792493in}}%
\pgfpathlineto{\pgfqpoint{2.455648in}{0.739250in}}%
\pgfpathlineto{\pgfqpoint{2.460770in}{0.678656in}}%
\pgfpathlineto{\pgfqpoint{2.464187in}{0.611077in}}%
\pgfpathlineto{\pgfqpoint{2.465932in}{0.530696in}}%
\pgfpathlineto{\pgfqpoint{2.465658in}{0.431620in}}%
\pgfpathlineto{\pgfqpoint{2.462883in}{0.307956in}}%
\pgfpathlineto{\pgfqpoint{2.460541in}{0.238000in}}%
\pgfpathlineto{\pgfqpoint{2.460541in}{0.238000in}}%
\pgfusepath{stroke}%
\end{pgfscope}%
\begin{pgfscope}%
\pgfpathrectangle{\pgfqpoint{0.404000in}{0.248000in}}{\pgfqpoint{2.246634in}{1.362506in}} %
\pgfusepath{clip}%
\pgfsetrectcap%
\pgfsetroundjoin%
\pgfsetlinewidth{1.003750pt}%
\definecolor{currentstroke}{rgb}{1.000000,0.388235,0.278431}%
\pgfsetstrokecolor{currentstroke}%
\pgfsetdash{}{0pt}%
\pgfpathmoveto{\pgfqpoint{2.545400in}{0.238000in}}%
\pgfpathlineto{\pgfqpoint{2.577711in}{0.569416in}}%
\pgfpathlineto{\pgfqpoint{2.588619in}{0.696614in}}%
\pgfpathlineto{\pgfqpoint{2.593924in}{0.782299in}}%
\pgfpathlineto{\pgfqpoint{2.596073in}{0.850196in}}%
\pgfpathlineto{\pgfqpoint{2.595816in}{0.905910in}}%
\pgfpathlineto{\pgfqpoint{2.593479in}{0.955088in}}%
\pgfpathlineto{\pgfqpoint{2.589392in}{0.997210in}}%
\pgfpathlineto{\pgfqpoint{2.583121in}{1.037637in}}%
\pgfpathlineto{\pgfqpoint{2.575952in}{1.070237in}}%
\pgfpathlineto{\pgfqpoint{2.567232in}{1.100421in}}%
\pgfpathlineto{\pgfqpoint{2.557169in}{1.127918in}}%
\pgfpathlineto{\pgfqpoint{2.546017in}{1.152702in}}%
\pgfpathlineto{\pgfqpoint{2.531951in}{1.178397in}}%
\pgfpathlineto{\pgfqpoint{2.517041in}{1.200983in}}%
\pgfpathlineto{\pgfqpoint{2.499280in}{1.223610in}}%
\pgfpathlineto{\pgfqpoint{2.480971in}{1.243379in}}%
\pgfpathlineto{\pgfqpoint{2.459915in}{1.262797in}}%
\pgfpathlineto{\pgfqpoint{2.436127in}{1.281555in}}%
\pgfpathlineto{\pgfqpoint{2.409632in}{1.299427in}}%
\pgfpathlineto{\pgfqpoint{2.380457in}{1.316255in}}%
\pgfpathlineto{\pgfqpoint{2.348630in}{1.331960in}}%
\pgfpathlineto{\pgfqpoint{2.314179in}{1.346519in}}%
\pgfpathlineto{\pgfqpoint{2.274653in}{1.360725in}}%
\pgfpathlineto{\pgfqpoint{2.232531in}{1.373503in}}%
\pgfpathlineto{\pgfqpoint{2.185349in}{1.385477in}}%
\pgfpathlineto{\pgfqpoint{2.133112in}{1.396386in}}%
\pgfpathlineto{\pgfqpoint{2.075830in}{1.406017in}}%
\pgfpathlineto{\pgfqpoint{2.013511in}{1.414184in}}%
\pgfpathlineto{\pgfqpoint{1.946164in}{1.420715in}}%
\pgfpathlineto{\pgfqpoint{1.886274in}{1.424475in}}%
\pgfpathlineto{\pgfqpoint{1.811397in}{1.427455in}}%
\pgfpathlineto{\pgfqpoint{1.734014in}{1.428307in}}%
\pgfpathlineto{\pgfqpoint{1.654136in}{1.426916in}}%
\pgfpathlineto{\pgfqpoint{1.574270in}{1.423242in}}%
\pgfpathlineto{\pgfqpoint{1.486940in}{1.416863in}}%
\pgfpathlineto{\pgfqpoint{1.412123in}{1.408806in}}%
\pgfpathlineto{\pgfqpoint{1.339845in}{1.398796in}}%
\pgfpathlineto{\pgfqpoint{1.282550in}{1.389843in}}%
\pgfpathlineto{\pgfqpoint{1.217857in}{1.376932in}}%
\pgfpathlineto{\pgfqpoint{1.155732in}{1.362254in}}%
\pgfpathlineto{\pgfqpoint{1.096191in}{1.345796in}}%
\pgfpathlineto{\pgfqpoint{1.041723in}{1.328406in}}%
\pgfpathlineto{\pgfqpoint{1.029330in}{1.324958in}}%
\pgfpathlineto{\pgfqpoint{1.021883in}{1.323195in}}%
\pgfpathlineto{\pgfqpoint{0.972538in}{1.304382in}}%
\pgfpathlineto{\pgfqpoint{0.906125in}{1.276156in}}%
\pgfpathlineto{\pgfqpoint{0.862114in}{1.253711in}}%
\pgfpathlineto{\pgfqpoint{0.820763in}{1.230060in}}%
\pgfpathlineto{\pgfqpoint{0.782085in}{1.205359in}}%
\pgfpathlineto{\pgfqpoint{0.743696in}{1.178024in}}%
\pgfpathlineto{\pgfqpoint{0.710398in}{1.151709in}}%
\pgfpathlineto{\pgfqpoint{0.675064in}{1.120973in}}%
\pgfpathlineto{\pgfqpoint{0.642536in}{1.089287in}}%
\pgfpathlineto{\pgfqpoint{0.612769in}{1.057235in}}%
\pgfpathlineto{\pgfqpoint{0.583504in}{1.022454in}}%
\pgfpathlineto{\pgfqpoint{0.554829in}{0.984771in}}%
\pgfpathlineto{\pgfqpoint{0.526842in}{0.944031in}}%
\pgfpathlineto{\pgfqpoint{0.516633in}{0.926289in}}%
\pgfpathlineto{\pgfqpoint{0.471794in}{0.847683in}}%
\pgfpathlineto{\pgfqpoint{0.448514in}{0.800926in}}%
\pgfpathlineto{\pgfqpoint{0.426184in}{0.751398in}}%
\pgfpathlineto{\pgfqpoint{0.403146in}{0.694710in}}%
\pgfpathlineto{\pgfqpoint{0.394000in}{0.670413in}}%
\pgfpathlineto{\pgfqpoint{0.394000in}{0.670413in}}%
\pgfusepath{stroke}%
\end{pgfscope}%
\begin{pgfscope}%
\pgfpathrectangle{\pgfqpoint{0.404000in}{0.248000in}}{\pgfqpoint{2.246634in}{1.362506in}} %
\pgfusepath{clip}%
\pgfsetrectcap%
\pgfsetroundjoin%
\pgfsetlinewidth{1.003750pt}%
\definecolor{currentstroke}{rgb}{1.000000,0.388235,0.278431}%
\pgfsetstrokecolor{currentstroke}%
\pgfsetdash{}{0pt}%
\pgfpathmoveto{\pgfqpoint{2.544322in}{0.238000in}}%
\pgfpathlineto{\pgfqpoint{2.578787in}{0.593813in}}%
\pgfpathlineto{\pgfqpoint{2.588156in}{0.709153in}}%
\pgfpathlineto{\pgfqpoint{2.592952in}{0.795019in}}%
\pgfpathlineto{\pgfqpoint{2.594492in}{0.863014in}}%
\pgfpathlineto{\pgfqpoint{2.593550in}{0.918678in}}%
\pgfpathlineto{\pgfqpoint{2.590460in}{0.967596in}}%
\pgfpathlineto{\pgfqpoint{2.585607in}{1.009213in}}%
\pgfpathlineto{\pgfqpoint{2.578536in}{1.048818in}}%
\pgfpathlineto{\pgfqpoint{2.570719in}{1.080488in}}%
\pgfpathlineto{\pgfqpoint{2.561432in}{1.109615in}}%
\pgfpathlineto{\pgfqpoint{2.550908in}{1.136030in}}%
\pgfpathlineto{\pgfqpoint{2.537400in}{1.163503in}}%
\pgfpathlineto{\pgfqpoint{2.522902in}{1.187677in}}%
\pgfpathlineto{\pgfqpoint{2.507692in}{1.209000in}}%
\pgfpathlineto{\pgfqpoint{2.489684in}{1.230396in}}%
\pgfpathlineto{\pgfqpoint{2.468868in}{1.251342in}}%
\pgfpathlineto{\pgfqpoint{2.447638in}{1.269567in}}%
\pgfpathlineto{\pgfqpoint{2.423718in}{1.287258in}}%
\pgfpathlineto{\pgfqpoint{2.397121in}{1.304177in}}%
\pgfpathlineto{\pgfqpoint{2.367870in}{1.320180in}}%
\pgfpathlineto{\pgfqpoint{2.335987in}{1.335169in}}%
\pgfpathlineto{\pgfqpoint{2.299024in}{1.350009in}}%
\pgfpathlineto{\pgfqpoint{2.259456in}{1.363486in}}%
\pgfpathlineto{\pgfqpoint{2.214822in}{1.376285in}}%
\pgfpathlineto{\pgfqpoint{2.165125in}{1.388110in}}%
\pgfpathlineto{\pgfqpoint{2.110375in}{1.398726in}}%
\pgfpathlineto{\pgfqpoint{2.050580in}{1.407931in}}%
\pgfpathlineto{\pgfqpoint{1.988245in}{1.415296in}}%
\pgfpathlineto{\pgfqpoint{1.920886in}{1.421033in}}%
\pgfpathlineto{\pgfqpoint{1.883452in}{1.423154in}}%
\pgfpathlineto{\pgfqpoint{1.808574in}{1.426088in}}%
\pgfpathlineto{\pgfqpoint{1.731191in}{1.426884in}}%
\pgfpathlineto{\pgfqpoint{1.653810in}{1.425504in}}%
\pgfpathlineto{\pgfqpoint{1.576439in}{1.421968in}}%
\pgfpathlineto{\pgfqpoint{1.489109in}{1.415604in}}%
\pgfpathlineto{\pgfqpoint{1.414292in}{1.407584in}}%
\pgfpathlineto{\pgfqpoint{1.342013in}{1.397604in}}%
\pgfpathlineto{\pgfqpoint{1.257339in}{1.383389in}}%
\pgfpathlineto{\pgfqpoint{1.195164in}{1.370091in}}%
\pgfpathlineto{\pgfqpoint{1.135558in}{1.355137in}}%
\pgfpathlineto{\pgfqpoint{1.078537in}{1.338523in}}%
\pgfpathlineto{\pgfqpoint{0.999364in}{1.312619in}}%
\pgfpathlineto{\pgfqpoint{0.930370in}{1.285093in}}%
\pgfpathlineto{\pgfqpoint{0.886261in}{1.263863in}}%
\pgfpathlineto{\pgfqpoint{0.842361in}{1.240116in}}%
\pgfpathlineto{\pgfqpoint{0.801143in}{1.215079in}}%
\pgfpathlineto{\pgfqpoint{0.762622in}{1.188911in}}%
\pgfpathlineto{\pgfqpoint{0.726799in}{1.161884in}}%
\pgfpathlineto{\pgfqpoint{0.693686in}{1.134170in}}%
\pgfpathlineto{\pgfqpoint{0.658586in}{1.101848in}}%
\pgfpathlineto{\pgfqpoint{0.628614in}{1.070989in}}%
\pgfpathlineto{\pgfqpoint{0.599106in}{1.037494in}}%
\pgfpathlineto{\pgfqpoint{0.570147in}{1.001173in}}%
\pgfpathlineto{\pgfqpoint{0.541829in}{0.961865in}}%
\pgfpathlineto{\pgfqpoint{0.520676in}{0.929041in}}%
\pgfpathlineto{\pgfqpoint{0.514731in}{0.917784in}}%
\pgfpathlineto{\pgfqpoint{0.490051in}{0.875678in}}%
\pgfpathlineto{\pgfqpoint{0.466184in}{0.830778in}}%
\pgfpathlineto{\pgfqpoint{0.443216in}{0.783081in}}%
\pgfpathlineto{\pgfqpoint{0.419438in}{0.728306in}}%
\pgfpathlineto{\pgfqpoint{0.396886in}{0.670425in}}%
\pgfpathlineto{\pgfqpoint{0.394000in}{0.662544in}}%
\pgfpathlineto{\pgfqpoint{0.394000in}{0.662544in}}%
\pgfusepath{stroke}%
\end{pgfscope}%
\begin{pgfscope}%
\pgfpathrectangle{\pgfqpoint{0.404000in}{0.248000in}}{\pgfqpoint{2.246634in}{1.362506in}} %
\pgfusepath{clip}%
\pgfsetbuttcap%
\pgfsetroundjoin%
\pgfsetlinewidth{1.003750pt}%
\definecolor{currentstroke}{rgb}{0.000000,0.000000,0.000000}%
\pgfsetstrokecolor{currentstroke}%
\pgfsetdash{{1.000000pt}{1.650000pt}}{0.000000pt}%
\pgfpathmoveto{\pgfqpoint{0.558748in}{0.938981in}}%
\pgfpathlineto{\pgfqpoint{0.586949in}{0.978804in}}%
\pgfpathlineto{\pgfqpoint{0.615805in}{1.015625in}}%
\pgfpathlineto{\pgfqpoint{0.645222in}{1.049607in}}%
\pgfpathlineto{\pgfqpoint{0.675115in}{1.080930in}}%
\pgfpathlineto{\pgfqpoint{0.707755in}{1.111901in}}%
\pgfpathlineto{\pgfqpoint{0.740783in}{1.140232in}}%
\pgfpathlineto{\pgfqpoint{0.776525in}{1.167912in}}%
\pgfpathlineto{\pgfqpoint{0.812568in}{1.193078in}}%
\pgfpathlineto{\pgfqpoint{0.851283in}{1.217417in}}%
\pgfpathlineto{\pgfqpoint{0.892665in}{1.240734in}}%
\pgfpathlineto{\pgfqpoint{0.936702in}{1.262869in}}%
\pgfpathlineto{\pgfqpoint{0.983381in}{1.283701in}}%
\pgfpathlineto{\pgfqpoint{1.032686in}{1.303140in}}%
\pgfpathlineto{\pgfqpoint{1.084604in}{1.321119in}}%
\pgfpathlineto{\pgfqpoint{1.139118in}{1.337594in}}%
\pgfpathlineto{\pgfqpoint{1.196215in}{1.352531in}}%
\pgfpathlineto{\pgfqpoint{1.258369in}{1.366416in}}%
\pgfpathlineto{\pgfqpoint{1.323087in}{1.378544in}}%
\pgfpathlineto{\pgfqpoint{1.390356in}{1.388899in}}%
\pgfpathlineto{\pgfqpoint{1.460166in}{1.397450in}}%
\pgfpathlineto{\pgfqpoint{1.535002in}{1.404338in}}%
\pgfpathlineto{\pgfqpoint{1.609866in}{1.409003in}}%
\pgfpathlineto{\pgfqpoint{1.687242in}{1.411573in}}%
\pgfpathlineto{\pgfqpoint{1.764626in}{1.411865in}}%
\pgfpathlineto{\pgfqpoint{1.839509in}{1.409895in}}%
\pgfpathlineto{\pgfqpoint{1.911881in}{1.405732in}}%
\pgfpathlineto{\pgfqpoint{1.979235in}{1.399656in}}%
\pgfpathlineto{\pgfqpoint{2.041562in}{1.391867in}}%
\pgfpathlineto{\pgfqpoint{2.098853in}{1.382570in}}%
\pgfpathlineto{\pgfqpoint{2.151099in}{1.371968in}}%
\pgfpathlineto{\pgfqpoint{2.200775in}{1.359617in}}%
\pgfpathlineto{\pgfqpoint{2.245381in}{1.346216in}}%
\pgfpathlineto{\pgfqpoint{2.284911in}{1.332076in}}%
\pgfpathlineto{\pgfqpoint{2.321822in}{1.316473in}}%
\pgfpathlineto{\pgfqpoint{2.353642in}{1.300685in}}%
\pgfpathlineto{\pgfqpoint{2.382812in}{1.283802in}}%
\pgfpathlineto{\pgfqpoint{2.409307in}{1.265921in}}%
\pgfpathlineto{\pgfqpoint{2.433100in}{1.247202in}}%
\pgfpathlineto{\pgfqpoint{2.454172in}{1.227887in}}%
\pgfpathlineto{\pgfqpoint{2.472517in}{1.208329in}}%
\pgfpathlineto{\pgfqpoint{2.490351in}{1.186055in}}%
\pgfpathlineto{\pgfqpoint{2.505371in}{1.163918in}}%
\pgfpathlineto{\pgfqpoint{2.519635in}{1.138901in}}%
\pgfpathlineto{\pgfqpoint{2.532874in}{1.110636in}}%
\pgfpathlineto{\pgfqpoint{2.543155in}{1.083635in}}%
\pgfpathlineto{\pgfqpoint{2.552215in}{1.054070in}}%
\pgfpathlineto{\pgfqpoint{2.559861in}{1.022142in}}%
\pgfpathlineto{\pgfqpoint{2.566850in}{0.982443in}}%
\pgfpathlineto{\pgfqpoint{2.571775in}{0.940874in}}%
\pgfpathlineto{\pgfqpoint{2.575123in}{0.892057in}}%
\pgfpathlineto{\pgfqpoint{2.576499in}{0.836446in}}%
\pgfpathlineto{\pgfqpoint{2.575786in}{0.774556in}}%
\pgfpathlineto{\pgfqpoint{2.572428in}{0.694492in}}%
\pgfpathlineto{\pgfqpoint{2.565951in}{0.596723in}}%
\pgfpathlineto{\pgfqpoint{2.554317in}{0.457240in}}%
\pgfpathlineto{\pgfqpoint{2.534230in}{0.238000in}}%
\pgfpathlineto{\pgfqpoint{2.534230in}{0.238000in}}%
\pgfusepath{stroke}%
\end{pgfscope}%
\begin{pgfscope}%
\pgfpathrectangle{\pgfqpoint{0.404000in}{0.248000in}}{\pgfqpoint{2.246634in}{1.362506in}} %
\pgfusepath{clip}%
\pgfsetbuttcap%
\pgfsetroundjoin%
\pgfsetlinewidth{1.003750pt}%
\definecolor{currentstroke}{rgb}{0.000000,0.000000,0.000000}%
\pgfsetstrokecolor{currentstroke}%
\pgfsetdash{{1.000000pt}{1.650000pt}}{0.000000pt}%
\pgfpathmoveto{\pgfqpoint{0.000000in}{0.000000in}}%
\pgfusepath{stroke}%
\end{pgfscope}%
\begin{pgfscope}%
\pgfpathrectangle{\pgfqpoint{0.404000in}{0.248000in}}{\pgfqpoint{2.246634in}{1.362506in}} %
\pgfusepath{clip}%
\pgfsetbuttcap%
\pgfsetroundjoin%
\pgfsetlinewidth{1.003750pt}%
\definecolor{currentstroke}{rgb}{0.000000,0.000000,0.000000}%
\pgfsetstrokecolor{currentstroke}%
\pgfsetdash{{1.000000pt}{1.650000pt}}{0.000000pt}%
\pgfpathmoveto{\pgfqpoint{0.000000in}{0.000000in}}%
\pgfusepath{stroke}%
\end{pgfscope}%
\begin{pgfscope}%
\pgfpathrectangle{\pgfqpoint{0.404000in}{0.248000in}}{\pgfqpoint{2.246634in}{1.362506in}} %
\pgfusepath{clip}%
\pgfsetbuttcap%
\pgfsetroundjoin%
\pgfsetlinewidth{1.003750pt}%
\definecolor{currentstroke}{rgb}{0.000000,0.000000,0.000000}%
\pgfsetstrokecolor{currentstroke}%
\pgfsetdash{{1.000000pt}{1.650000pt}}{0.000000pt}%
\pgfpathmoveto{\pgfqpoint{2.350898in}{0.238000in}}%
\pgfpathlineto{\pgfqpoint{2.346074in}{0.331718in}}%
\pgfpathlineto{\pgfqpoint{2.339588in}{0.416901in}}%
\pgfpathlineto{\pgfqpoint{2.332042in}{0.488807in}}%
\pgfpathlineto{\pgfqpoint{2.323181in}{0.553270in}}%
\pgfpathlineto{\pgfqpoint{2.313276in}{0.610098in}}%
\pgfpathlineto{\pgfqpoint{2.301449in}{0.664615in}}%
\pgfpathlineto{\pgfqpoint{2.289108in}{0.711171in}}%
\pgfpathlineto{\pgfqpoint{2.275188in}{0.754899in}}%
\pgfpathlineto{\pgfqpoint{2.259796in}{0.795477in}}%
\pgfpathlineto{\pgfqpoint{2.243094in}{0.832734in}}%
\pgfpathlineto{\pgfqpoint{2.225274in}{0.866658in}}%
\pgfpathlineto{\pgfqpoint{2.206530in}{0.897367in}}%
\pgfpathlineto{\pgfqpoint{2.187039in}{0.925070in}}%
\pgfpathlineto{\pgfqpoint{2.166952in}{0.950019in}}%
\pgfpathlineto{\pgfqpoint{2.144084in}{0.974836in}}%
\pgfpathlineto{\pgfqpoint{2.120766in}{0.996934in}}%
\pgfpathlineto{\pgfqpoint{2.094720in}{1.018486in}}%
\pgfpathlineto{\pgfqpoint{2.068355in}{1.037513in}}%
\pgfpathlineto{\pgfqpoint{2.039319in}{1.055762in}}%
\pgfpathlineto{\pgfqpoint{2.007619in}{1.072969in}}%
\pgfpathlineto{\pgfqpoint{1.973269in}{1.088917in}}%
\pgfpathlineto{\pgfqpoint{1.938756in}{1.102539in}}%
\pgfpathlineto{\pgfqpoint{1.901645in}{1.114881in}}%
\pgfpathlineto{\pgfqpoint{1.861948in}{1.125783in}}%
\pgfpathlineto{\pgfqpoint{1.819679in}{1.135103in}}%
\pgfpathlineto{\pgfqpoint{1.774851in}{1.142695in}}%
\pgfpathlineto{\pgfqpoint{1.727479in}{1.148420in}}%
\pgfpathlineto{\pgfqpoint{1.677577in}{1.152108in}}%
\pgfpathlineto{\pgfqpoint{1.627656in}{1.153558in}}%
\pgfpathlineto{\pgfqpoint{1.575236in}{1.152791in}}%
\pgfpathlineto{\pgfqpoint{1.522829in}{1.149743in}}%
\pgfpathlineto{\pgfqpoint{1.470452in}{1.144441in}}%
\pgfpathlineto{\pgfqpoint{1.418121in}{1.136846in}}%
\pgfpathlineto{\pgfqpoint{1.368341in}{1.127426in}}%
\pgfpathlineto{\pgfqpoint{1.318638in}{1.115770in}}%
\pgfpathlineto{\pgfqpoint{1.271512in}{1.102505in}}%
\pgfpathlineto{\pgfqpoint{1.224500in}{1.086956in}}%
\pgfpathlineto{\pgfqpoint{1.180093in}{1.069958in}}%
\pgfpathlineto{\pgfqpoint{1.135847in}{1.050561in}}%
\pgfpathlineto{\pgfqpoint{1.094241in}{1.029846in}}%
\pgfpathlineto{\pgfqpoint{1.055282in}{1.008029in}}%
\pgfpathlineto{\pgfqpoint{1.016560in}{0.983757in}}%
\pgfpathlineto{\pgfqpoint{0.980520in}{0.958571in}}%
\pgfpathlineto{\pgfqpoint{0.944787in}{0.930821in}}%
\pgfpathlineto{\pgfqpoint{0.911770in}{0.902411in}}%
\pgfpathlineto{\pgfqpoint{0.879139in}{0.871378in}}%
\pgfpathlineto{\pgfqpoint{0.849250in}{0.840035in}}%
\pgfpathlineto{\pgfqpoint{0.819826in}{0.806089in}}%
\pgfpathlineto{\pgfqpoint{0.790946in}{0.769378in}}%
\pgfpathlineto{\pgfqpoint{0.762701in}{0.729750in}}%
\pgfpathlineto{\pgfqpoint{0.735188in}{0.687068in}}%
\pgfpathlineto{\pgfqpoint{0.710535in}{0.644864in}}%
\pgfpathlineto{\pgfqpoint{0.686686in}{0.599906in}}%
\pgfpathlineto{\pgfqpoint{0.663731in}{0.552170in}}%
\pgfpathlineto{\pgfqpoint{0.641759in}{0.501668in}}%
\pgfpathlineto{\pgfqpoint{0.619162in}{0.443898in}}%
\pgfpathlineto{\pgfqpoint{0.597910in}{0.383067in}}%
\pgfpathlineto{\pgfqpoint{0.578081in}{0.319347in}}%
\pgfpathlineto{\pgfqpoint{0.559733in}{0.252952in}}%
\pgfpathlineto{\pgfqpoint{0.555898in}{0.238000in}}%
\pgfpathmoveto{\pgfqpoint{0.430052in}{0.238000in}}%
\pgfpathlineto{\pgfqpoint{0.446814in}{0.312934in}}%
\pgfpathlineto{\pgfqpoint{0.465426in}{0.386306in}}%
\pgfpathlineto{\pgfqpoint{0.485704in}{0.456908in}}%
\pgfpathlineto{\pgfqpoint{0.506000in}{0.519716in}}%
\pgfpathlineto{\pgfqpoint{0.527663in}{0.579649in}}%
\pgfpathlineto{\pgfqpoint{0.550624in}{0.636529in}}%
\pgfpathlineto{\pgfqpoint{0.574797in}{0.690230in}}%
\pgfpathlineto{\pgfqpoint{0.600081in}{0.740684in}}%
\pgfpathlineto{\pgfqpoint{0.626367in}{0.787883in}}%
\pgfpathlineto{\pgfqpoint{0.653543in}{0.831871in}}%
\pgfpathlineto{\pgfqpoint{0.681500in}{0.872740in}}%
\pgfpathlineto{\pgfqpoint{0.710133in}{0.910616in}}%
\pgfpathlineto{\pgfqpoint{0.739349in}{0.945652in}}%
\pgfpathlineto{\pgfqpoint{0.769061in}{0.978015in}}%
\pgfpathlineto{\pgfqpoint{0.801530in}{1.010079in}}%
\pgfpathlineto{\pgfqpoint{0.834408in}{1.039464in}}%
\pgfpathlineto{\pgfqpoint{0.870011in}{1.068216in}}%
\pgfpathlineto{\pgfqpoint{0.905938in}{1.094382in}}%
\pgfpathlineto{\pgfqpoint{0.944551in}{1.119701in}}%
\pgfpathlineto{\pgfqpoint{0.983410in}{1.142591in}}%
\pgfpathlineto{\pgfqpoint{1.024916in}{1.164503in}}%
\pgfpathlineto{\pgfqpoint{1.069062in}{1.185263in}}%
\pgfpathlineto{\pgfqpoint{1.115837in}{1.204721in}}%
\pgfpathlineto{\pgfqpoint{1.165230in}{1.222754in}}%
\pgfpathlineto{\pgfqpoint{1.217227in}{1.239252in}}%
\pgfpathlineto{\pgfqpoint{1.271816in}{1.254121in}}%
\pgfpathlineto{\pgfqpoint{1.326497in}{1.266744in}}%
\pgfpathlineto{\pgfqpoint{1.383739in}{1.277739in}}%
\pgfpathlineto{\pgfqpoint{1.443533in}{1.286990in}}%
\pgfpathlineto{\pgfqpoint{1.505868in}{1.294358in}}%
\pgfpathlineto{\pgfqpoint{1.568239in}{1.299514in}}%
\pgfpathlineto{\pgfqpoint{1.630634in}{1.302509in}}%
\pgfpathlineto{\pgfqpoint{1.693039in}{1.303343in}}%
\pgfpathlineto{\pgfqpoint{1.755442in}{1.301955in}}%
\pgfpathlineto{\pgfqpoint{1.815335in}{1.298414in}}%
\pgfpathlineto{\pgfqpoint{1.872704in}{1.292821in}}%
\pgfpathlineto{\pgfqpoint{1.927536in}{1.285234in}}%
\pgfpathlineto{\pgfqpoint{1.977327in}{1.276188in}}%
\pgfpathlineto{\pgfqpoint{2.024556in}{1.265416in}}%
\pgfpathlineto{\pgfqpoint{2.069205in}{1.252926in}}%
\pgfpathlineto{\pgfqpoint{2.108783in}{1.239630in}}%
\pgfpathlineto{\pgfqpoint{2.145755in}{1.224941in}}%
\pgfpathlineto{\pgfqpoint{2.180100in}{1.208922in}}%
\pgfpathlineto{\pgfqpoint{2.211795in}{1.191661in}}%
\pgfpathlineto{\pgfqpoint{2.240819in}{1.173287in}}%
\pgfpathlineto{\pgfqpoint{2.267150in}{1.153979in}}%
\pgfpathlineto{\pgfqpoint{2.290775in}{1.133992in}}%
\pgfpathlineto{\pgfqpoint{2.313990in}{1.111251in}}%
\pgfpathlineto{\pgfqpoint{2.334408in}{1.088012in}}%
\pgfpathlineto{\pgfqpoint{2.352045in}{1.064791in}}%
\pgfpathlineto{\pgfqpoint{2.369047in}{1.038820in}}%
\pgfpathlineto{\pgfqpoint{2.385221in}{1.009781in}}%
\pgfpathlineto{\pgfqpoint{2.398511in}{0.981649in}}%
\pgfpathlineto{\pgfqpoint{2.410812in}{0.950879in}}%
\pgfpathlineto{\pgfqpoint{2.421952in}{0.917500in}}%
\pgfpathlineto{\pgfqpoint{2.431788in}{0.881687in}}%
\pgfpathlineto{\pgfqpoint{2.441323in}{0.838179in}}%
\pgfpathlineto{\pgfqpoint{2.449029in}{0.792493in}}%
\pgfpathlineto{\pgfqpoint{2.455648in}{0.739250in}}%
\pgfpathlineto{\pgfqpoint{2.460770in}{0.678656in}}%
\pgfpathlineto{\pgfqpoint{2.464187in}{0.611077in}}%
\pgfpathlineto{\pgfqpoint{2.465932in}{0.530696in}}%
\pgfpathlineto{\pgfqpoint{2.465658in}{0.431620in}}%
\pgfpathlineto{\pgfqpoint{2.462883in}{0.307956in}}%
\pgfpathlineto{\pgfqpoint{2.460541in}{0.238000in}}%
\pgfpathlineto{\pgfqpoint{2.460541in}{0.238000in}}%
\pgfusepath{stroke}%
\end{pgfscope}%
\begin{pgfscope}%
\pgfpathrectangle{\pgfqpoint{0.404000in}{0.248000in}}{\pgfqpoint{2.246634in}{1.362506in}} %
\pgfusepath{clip}%
\pgfsetbuttcap%
\pgfsetroundjoin%
\pgfsetlinewidth{1.003750pt}%
\definecolor{currentstroke}{rgb}{0.000000,0.000000,0.000000}%
\pgfsetstrokecolor{currentstroke}%
\pgfsetdash{{1.000000pt}{1.650000pt}}{0.000000pt}%
\pgfpathmoveto{\pgfqpoint{2.545400in}{0.238000in}}%
\pgfpathlineto{\pgfqpoint{2.577710in}{0.569415in}}%
\pgfpathlineto{\pgfqpoint{2.588618in}{0.696613in}}%
\pgfpathlineto{\pgfqpoint{2.593924in}{0.782298in}}%
\pgfpathlineto{\pgfqpoint{2.596073in}{0.850195in}}%
\pgfpathlineto{\pgfqpoint{2.595815in}{0.905908in}}%
\pgfpathlineto{\pgfqpoint{2.593478in}{0.955087in}}%
\pgfpathlineto{\pgfqpoint{2.589391in}{0.997209in}}%
\pgfpathlineto{\pgfqpoint{2.583120in}{1.037636in}}%
\pgfpathlineto{\pgfqpoint{2.575952in}{1.070236in}}%
\pgfpathlineto{\pgfqpoint{2.567232in}{1.100419in}}%
\pgfpathlineto{\pgfqpoint{2.557169in}{1.127917in}}%
\pgfpathlineto{\pgfqpoint{2.546017in}{1.152701in}}%
\pgfpathlineto{\pgfqpoint{2.531951in}{1.178396in}}%
\pgfpathlineto{\pgfqpoint{2.517041in}{1.200982in}}%
\pgfpathlineto{\pgfqpoint{2.499280in}{1.223609in}}%
\pgfpathlineto{\pgfqpoint{2.480971in}{1.243379in}}%
\pgfpathlineto{\pgfqpoint{2.459915in}{1.262797in}}%
\pgfpathlineto{\pgfqpoint{2.436127in}{1.281554in}}%
\pgfpathlineto{\pgfqpoint{2.409632in}{1.299426in}}%
\pgfpathlineto{\pgfqpoint{2.380456in}{1.316254in}}%
\pgfpathlineto{\pgfqpoint{2.348630in}{1.331960in}}%
\pgfpathlineto{\pgfqpoint{2.314179in}{1.346518in}}%
\pgfpathlineto{\pgfqpoint{2.274653in}{1.360725in}}%
\pgfpathlineto{\pgfqpoint{2.232531in}{1.373502in}}%
\pgfpathlineto{\pgfqpoint{2.185349in}{1.385476in}}%
\pgfpathlineto{\pgfqpoint{2.133112in}{1.396386in}}%
\pgfpathlineto{\pgfqpoint{2.075830in}{1.406017in}}%
\pgfpathlineto{\pgfqpoint{2.013511in}{1.414184in}}%
\pgfpathlineto{\pgfqpoint{1.946164in}{1.420715in}}%
\pgfpathlineto{\pgfqpoint{1.886274in}{1.424474in}}%
\pgfpathlineto{\pgfqpoint{1.811396in}{1.427454in}}%
\pgfpathlineto{\pgfqpoint{1.734014in}{1.428307in}}%
\pgfpathlineto{\pgfqpoint{1.654136in}{1.426915in}}%
\pgfpathlineto{\pgfqpoint{1.574269in}{1.423242in}}%
\pgfpathlineto{\pgfqpoint{1.486940in}{1.416862in}}%
\pgfpathlineto{\pgfqpoint{1.412123in}{1.408806in}}%
\pgfpathlineto{\pgfqpoint{1.339845in}{1.398795in}}%
\pgfpathlineto{\pgfqpoint{1.282550in}{1.389843in}}%
\pgfpathlineto{\pgfqpoint{1.217857in}{1.376932in}}%
\pgfpathlineto{\pgfqpoint{1.155732in}{1.362253in}}%
\pgfpathlineto{\pgfqpoint{1.096191in}{1.345795in}}%
\pgfpathlineto{\pgfqpoint{1.041723in}{1.328406in}}%
\pgfpathlineto{\pgfqpoint{1.029330in}{1.324957in}}%
\pgfpathlineto{\pgfqpoint{1.021883in}{1.323194in}}%
\pgfpathlineto{\pgfqpoint{0.972538in}{1.304381in}}%
\pgfpathlineto{\pgfqpoint{0.906125in}{1.276156in}}%
\pgfpathlineto{\pgfqpoint{0.862114in}{1.253711in}}%
\pgfpathlineto{\pgfqpoint{0.820763in}{1.230060in}}%
\pgfpathlineto{\pgfqpoint{0.782085in}{1.205359in}}%
\pgfpathlineto{\pgfqpoint{0.743696in}{1.178024in}}%
\pgfpathlineto{\pgfqpoint{0.710398in}{1.151710in}}%
\pgfpathlineto{\pgfqpoint{0.675064in}{1.120973in}}%
\pgfpathlineto{\pgfqpoint{0.642536in}{1.089287in}}%
\pgfpathlineto{\pgfqpoint{0.612769in}{1.057235in}}%
\pgfpathlineto{\pgfqpoint{0.583504in}{1.022454in}}%
\pgfpathlineto{\pgfqpoint{0.554829in}{0.984771in}}%
\pgfpathlineto{\pgfqpoint{0.526842in}{0.944031in}}%
\pgfpathlineto{\pgfqpoint{0.516633in}{0.926288in}}%
\pgfpathlineto{\pgfqpoint{0.471794in}{0.847683in}}%
\pgfpathlineto{\pgfqpoint{0.448514in}{0.800926in}}%
\pgfpathlineto{\pgfqpoint{0.426184in}{0.751398in}}%
\pgfpathlineto{\pgfqpoint{0.403146in}{0.694710in}}%
\pgfpathlineto{\pgfqpoint{0.394000in}{0.670413in}}%
\pgfpathlineto{\pgfqpoint{0.394000in}{0.670413in}}%
\pgfusepath{stroke}%
\end{pgfscope}%
\begin{pgfscope}%
\pgfpathrectangle{\pgfqpoint{0.404000in}{0.248000in}}{\pgfqpoint{2.246634in}{1.362506in}} %
\pgfusepath{clip}%
\pgfsetbuttcap%
\pgfsetroundjoin%
\pgfsetlinewidth{1.003750pt}%
\definecolor{currentstroke}{rgb}{0.000000,0.000000,0.000000}%
\pgfsetstrokecolor{currentstroke}%
\pgfsetdash{{1.000000pt}{1.650000pt}}{0.000000pt}%
\pgfpathmoveto{\pgfqpoint{0.000000in}{0.000000in}}%
\pgfusepath{stroke}%
\end{pgfscope}%
\begin{pgfscope}%
\pgfpathrectangle{\pgfqpoint{0.404000in}{0.248000in}}{\pgfqpoint{2.246634in}{1.362506in}} %
\pgfusepath{clip}%
\pgfsetbuttcap%
\pgfsetroundjoin%
\pgfsetlinewidth{1.003750pt}%
\definecolor{currentstroke}{rgb}{0.000000,0.000000,0.000000}%
\pgfsetstrokecolor{currentstroke}%
\pgfsetdash{{1.000000pt}{1.650000pt}}{0.000000pt}%
\pgfpathmoveto{\pgfqpoint{2.544322in}{0.238000in}}%
\pgfpathlineto{\pgfqpoint{2.578787in}{0.593812in}}%
\pgfpathlineto{\pgfqpoint{2.588156in}{0.709151in}}%
\pgfpathlineto{\pgfqpoint{2.592952in}{0.795018in}}%
\pgfpathlineto{\pgfqpoint{2.594492in}{0.863013in}}%
\pgfpathlineto{\pgfqpoint{2.593549in}{0.918677in}}%
\pgfpathlineto{\pgfqpoint{2.590459in}{0.967595in}}%
\pgfpathlineto{\pgfqpoint{2.585606in}{1.009212in}}%
\pgfpathlineto{\pgfqpoint{2.578535in}{1.048817in}}%
\pgfpathlineto{\pgfqpoint{2.570719in}{1.080487in}}%
\pgfpathlineto{\pgfqpoint{2.561432in}{1.109614in}}%
\pgfpathlineto{\pgfqpoint{2.550907in}{1.136029in}}%
\pgfpathlineto{\pgfqpoint{2.537400in}{1.163502in}}%
\pgfpathlineto{\pgfqpoint{2.522902in}{1.187676in}}%
\pgfpathlineto{\pgfqpoint{2.507692in}{1.208999in}}%
\pgfpathlineto{\pgfqpoint{2.489684in}{1.230396in}}%
\pgfpathlineto{\pgfqpoint{2.468868in}{1.251341in}}%
\pgfpathlineto{\pgfqpoint{2.447638in}{1.269567in}}%
\pgfpathlineto{\pgfqpoint{2.423718in}{1.287258in}}%
\pgfpathlineto{\pgfqpoint{2.397121in}{1.304177in}}%
\pgfpathlineto{\pgfqpoint{2.367870in}{1.320179in}}%
\pgfpathlineto{\pgfqpoint{2.335987in}{1.335169in}}%
\pgfpathlineto{\pgfqpoint{2.299024in}{1.350009in}}%
\pgfpathlineto{\pgfqpoint{2.259456in}{1.363485in}}%
\pgfpathlineto{\pgfqpoint{2.214822in}{1.376284in}}%
\pgfpathlineto{\pgfqpoint{2.165125in}{1.388110in}}%
\pgfpathlineto{\pgfqpoint{2.110375in}{1.398726in}}%
\pgfpathlineto{\pgfqpoint{2.050580in}{1.407931in}}%
\pgfpathlineto{\pgfqpoint{1.988245in}{1.415296in}}%
\pgfpathlineto{\pgfqpoint{1.920886in}{1.421033in}}%
\pgfpathlineto{\pgfqpoint{1.883452in}{1.423154in}}%
\pgfpathlineto{\pgfqpoint{1.808574in}{1.426088in}}%
\pgfpathlineto{\pgfqpoint{1.731191in}{1.426884in}}%
\pgfpathlineto{\pgfqpoint{1.653810in}{1.425503in}}%
\pgfpathlineto{\pgfqpoint{1.576439in}{1.421967in}}%
\pgfpathlineto{\pgfqpoint{1.489109in}{1.415603in}}%
\pgfpathlineto{\pgfqpoint{1.414292in}{1.407584in}}%
\pgfpathlineto{\pgfqpoint{1.342013in}{1.397603in}}%
\pgfpathlineto{\pgfqpoint{1.257339in}{1.383389in}}%
\pgfpathlineto{\pgfqpoint{1.195164in}{1.370090in}}%
\pgfpathlineto{\pgfqpoint{1.135558in}{1.355137in}}%
\pgfpathlineto{\pgfqpoint{1.078537in}{1.338523in}}%
\pgfpathlineto{\pgfqpoint{0.999364in}{1.312618in}}%
\pgfpathlineto{\pgfqpoint{0.930370in}{1.285093in}}%
\pgfpathlineto{\pgfqpoint{0.886261in}{1.263863in}}%
\pgfpathlineto{\pgfqpoint{0.842361in}{1.240116in}}%
\pgfpathlineto{\pgfqpoint{0.801143in}{1.215079in}}%
\pgfpathlineto{\pgfqpoint{0.762622in}{1.188911in}}%
\pgfpathlineto{\pgfqpoint{0.726799in}{1.161884in}}%
\pgfpathlineto{\pgfqpoint{0.693686in}{1.134170in}}%
\pgfpathlineto{\pgfqpoint{0.658586in}{1.101848in}}%
\pgfpathlineto{\pgfqpoint{0.628614in}{1.070989in}}%
\pgfpathlineto{\pgfqpoint{0.599106in}{1.037494in}}%
\pgfpathlineto{\pgfqpoint{0.570147in}{1.001173in}}%
\pgfpathlineto{\pgfqpoint{0.541829in}{0.961865in}}%
\pgfpathlineto{\pgfqpoint{0.520676in}{0.929041in}}%
\pgfpathlineto{\pgfqpoint{0.514731in}{0.917784in}}%
\pgfpathlineto{\pgfqpoint{0.490051in}{0.875677in}}%
\pgfpathlineto{\pgfqpoint{0.466184in}{0.830777in}}%
\pgfpathlineto{\pgfqpoint{0.443216in}{0.783081in}}%
\pgfpathlineto{\pgfqpoint{0.419438in}{0.728306in}}%
\pgfpathlineto{\pgfqpoint{0.396886in}{0.670425in}}%
\pgfpathlineto{\pgfqpoint{0.394000in}{0.662544in}}%
\pgfpathlineto{\pgfqpoint{0.394000in}{0.662544in}}%
\pgfusepath{stroke}%
\end{pgfscope}%
\begin{pgfscope}%
\pgfpathrectangle{\pgfqpoint{0.404000in}{0.248000in}}{\pgfqpoint{2.246634in}{1.362506in}} %
\pgfusepath{clip}%
\pgfsetbuttcap%
\pgfsetroundjoin%
\pgfsetlinewidth{1.003750pt}%
\definecolor{currentstroke}{rgb}{0.000000,0.000000,0.000000}%
\pgfsetstrokecolor{currentstroke}%
\pgfsetdash{{1.000000pt}{1.650000pt}}{0.000000pt}%
\pgfpathmoveto{\pgfqpoint{0.000000in}{0.000000in}}%
\pgfusepath{stroke}%
\end{pgfscope}%
\begin{pgfscope}%
\pgfsetrectcap%
\pgfsetmiterjoin%
\pgfsetlinewidth{0.803000pt}%
\definecolor{currentstroke}{rgb}{0.000000,0.000000,0.000000}%
\pgfsetstrokecolor{currentstroke}%
\pgfsetdash{}{0pt}%
\pgfpathmoveto{\pgfqpoint{0.404000in}{0.248000in}}%
\pgfpathlineto{\pgfqpoint{0.404000in}{1.610506in}}%
\pgfusepath{stroke}%
\end{pgfscope}%
\begin{pgfscope}%
\pgfsetrectcap%
\pgfsetmiterjoin%
\pgfsetlinewidth{0.803000pt}%
\definecolor{currentstroke}{rgb}{0.000000,0.000000,0.000000}%
\pgfsetstrokecolor{currentstroke}%
\pgfsetdash{}{0pt}%
\pgfpathmoveto{\pgfqpoint{2.650634in}{0.248000in}}%
\pgfpathlineto{\pgfqpoint{2.650634in}{1.610506in}}%
\pgfusepath{stroke}%
\end{pgfscope}%
\begin{pgfscope}%
\pgfsetrectcap%
\pgfsetmiterjoin%
\pgfsetlinewidth{0.803000pt}%
\definecolor{currentstroke}{rgb}{0.000000,0.000000,0.000000}%
\pgfsetstrokecolor{currentstroke}%
\pgfsetdash{}{0pt}%
\pgfpathmoveto{\pgfqpoint{0.404000in}{0.248000in}}%
\pgfpathlineto{\pgfqpoint{2.650634in}{0.248000in}}%
\pgfusepath{stroke}%
\end{pgfscope}%
\begin{pgfscope}%
\pgfsetrectcap%
\pgfsetmiterjoin%
\pgfsetlinewidth{0.803000pt}%
\definecolor{currentstroke}{rgb}{0.000000,0.000000,0.000000}%
\pgfsetstrokecolor{currentstroke}%
\pgfsetdash{}{0pt}%
\pgfpathmoveto{\pgfqpoint{0.404000in}{1.610506in}}%
\pgfpathlineto{\pgfqpoint{2.650634in}{1.610506in}}%
\pgfusepath{stroke}%
\end{pgfscope}%
\begin{pgfscope}%
\pgfsetbuttcap%
\pgfsetmiterjoin%
\definecolor{currentfill}{rgb}{1.000000,1.000000,1.000000}%
\pgfsetfillcolor{currentfill}%
\pgfsetfillopacity{0.800000}%
\pgfsetlinewidth{1.003750pt}%
\definecolor{currentstroke}{rgb}{0.800000,0.800000,0.800000}%
\pgfsetstrokecolor{currentstroke}%
\pgfsetstrokeopacity{0.800000}%
\pgfsetdash{}{0pt}%
\pgfpathmoveto{\pgfqpoint{1.127368in}{0.317444in}}%
\pgfpathlineto{\pgfqpoint{1.927266in}{0.317444in}}%
\pgfpathquadraticcurveto{\pgfqpoint{1.955044in}{0.317444in}}{\pgfqpoint{1.955044in}{0.345222in}}%
\pgfpathlineto{\pgfqpoint{1.955044in}{0.730222in}}%
\pgfpathquadraticcurveto{\pgfqpoint{1.955044in}{0.758000in}}{\pgfqpoint{1.927266in}{0.758000in}}%
\pgfpathlineto{\pgfqpoint{1.127368in}{0.758000in}}%
\pgfpathquadraticcurveto{\pgfqpoint{1.099590in}{0.758000in}}{\pgfqpoint{1.099590in}{0.730222in}}%
\pgfpathlineto{\pgfqpoint{1.099590in}{0.345222in}}%
\pgfpathquadraticcurveto{\pgfqpoint{1.099590in}{0.317444in}}{\pgfqpoint{1.127368in}{0.317444in}}%
\pgfpathclose%
\pgfusepath{stroke,fill}%
\end{pgfscope}%
\begin{pgfscope}%
\pgfsetrectcap%
\pgfsetroundjoin%
\pgfsetlinewidth{1.003750pt}%
\definecolor{currentstroke}{rgb}{1.000000,0.388235,0.278431}%
\pgfsetstrokecolor{currentstroke}%
\pgfsetdash{}{0pt}%
\pgfpathmoveto{\pgfqpoint{1.155146in}{0.653833in}}%
\pgfpathlineto{\pgfqpoint{1.224590in}{0.653833in}}%
\pgfusepath{stroke}%
\end{pgfscope}%
\begin{pgfscope}%
\pgftext[x=1.335701in,y=0.605222in,left,base]{\rmfamily\fontsize{10.000000}{12.000000}\selectfont \(\displaystyle \textnormal{tol}=10^{-7}\)}%
\end{pgfscope}%
\begin{pgfscope}%
\pgfsetbuttcap%
\pgfsetroundjoin%
\pgfsetlinewidth{1.003750pt}%
\definecolor{currentstroke}{rgb}{0.000000,0.000000,0.000000}%
\pgfsetstrokecolor{currentstroke}%
\pgfsetdash{{1.000000pt}{1.650000pt}}{0.000000pt}%
\pgfpathmoveto{\pgfqpoint{1.155146in}{0.454389in}}%
\pgfpathlineto{\pgfqpoint{1.224590in}{0.454389in}}%
\pgfusepath{stroke}%
\end{pgfscope}%
\begin{pgfscope}%
\pgftext[x=1.335701in,y=0.405778in,left,base]{\rmfamily\fontsize{10.000000}{12.000000}\selectfont Reference}%
\end{pgfscope}%
\begin{pgfscope}%
\pgfsetbuttcap%
\pgfsetmiterjoin%
\definecolor{currentfill}{rgb}{1.000000,1.000000,1.000000}%
\pgfsetfillcolor{currentfill}%
\pgfsetlinewidth{0.000000pt}%
\definecolor{currentstroke}{rgb}{0.000000,0.000000,0.000000}%
\pgfsetstrokecolor{currentstroke}%
\pgfsetstrokeopacity{0.000000}%
\pgfsetdash{}{0pt}%
\pgfpathmoveto{\pgfqpoint{2.762966in}{0.248000in}}%
\pgfpathlineto{\pgfqpoint{5.009600in}{0.248000in}}%
\pgfpathlineto{\pgfqpoint{5.009600in}{1.610506in}}%
\pgfpathlineto{\pgfqpoint{2.762966in}{1.610506in}}%
\pgfpathclose%
\pgfusepath{fill}%
\end{pgfscope}%
\begin{pgfscope}%
\pgfsetbuttcap%
\pgfsetroundjoin%
\definecolor{currentfill}{rgb}{0.000000,0.000000,0.000000}%
\pgfsetfillcolor{currentfill}%
\pgfsetlinewidth{0.803000pt}%
\definecolor{currentstroke}{rgb}{0.000000,0.000000,0.000000}%
\pgfsetstrokecolor{currentstroke}%
\pgfsetdash{}{0pt}%
\pgfsys@defobject{currentmarker}{\pgfqpoint{0.000000in}{-0.048611in}}{\pgfqpoint{0.000000in}{0.000000in}}{%
\pgfpathmoveto{\pgfqpoint{0.000000in}{0.000000in}}%
\pgfpathlineto{\pgfqpoint{0.000000in}{-0.048611in}}%
\pgfusepath{stroke,fill}%
}%
\begin{pgfscope}%
\pgfsys@transformshift{2.887779in}{0.248000in}%
\pgfsys@useobject{currentmarker}{}%
\end{pgfscope}%
\end{pgfscope}%
\begin{pgfscope}%
\pgftext[x=2.887779in,y=0.150778in,,top]{\rmfamily\fontsize{10.000000}{12.000000}\selectfont \(\displaystyle 0.30\)}%
\end{pgfscope}%
\begin{pgfscope}%
\pgfsetbuttcap%
\pgfsetroundjoin%
\definecolor{currentfill}{rgb}{0.000000,0.000000,0.000000}%
\pgfsetfillcolor{currentfill}%
\pgfsetlinewidth{0.803000pt}%
\definecolor{currentstroke}{rgb}{0.000000,0.000000,0.000000}%
\pgfsetstrokecolor{currentstroke}%
\pgfsetdash{}{0pt}%
\pgfsys@defobject{currentmarker}{\pgfqpoint{0.000000in}{-0.048611in}}{\pgfqpoint{0.000000in}{0.000000in}}{%
\pgfpathmoveto{\pgfqpoint{0.000000in}{0.000000in}}%
\pgfpathlineto{\pgfqpoint{0.000000in}{-0.048611in}}%
\pgfusepath{stroke,fill}%
}%
\begin{pgfscope}%
\pgfsys@transformshift{3.262218in}{0.248000in}%
\pgfsys@useobject{currentmarker}{}%
\end{pgfscope}%
\end{pgfscope}%
\begin{pgfscope}%
\pgftext[x=3.262218in,y=0.150778in,,top]{\rmfamily\fontsize{10.000000}{12.000000}\selectfont \(\displaystyle 0.45\)}%
\end{pgfscope}%
\begin{pgfscope}%
\pgfsetbuttcap%
\pgfsetroundjoin%
\definecolor{currentfill}{rgb}{0.000000,0.000000,0.000000}%
\pgfsetfillcolor{currentfill}%
\pgfsetlinewidth{0.803000pt}%
\definecolor{currentstroke}{rgb}{0.000000,0.000000,0.000000}%
\pgfsetstrokecolor{currentstroke}%
\pgfsetdash{}{0pt}%
\pgfsys@defobject{currentmarker}{\pgfqpoint{0.000000in}{-0.048611in}}{\pgfqpoint{0.000000in}{0.000000in}}{%
\pgfpathmoveto{\pgfqpoint{0.000000in}{0.000000in}}%
\pgfpathlineto{\pgfqpoint{0.000000in}{-0.048611in}}%
\pgfusepath{stroke,fill}%
}%
\begin{pgfscope}%
\pgfsys@transformshift{3.636657in}{0.248000in}%
\pgfsys@useobject{currentmarker}{}%
\end{pgfscope}%
\end{pgfscope}%
\begin{pgfscope}%
\pgftext[x=3.636657in,y=0.150778in,,top]{\rmfamily\fontsize{10.000000}{12.000000}\selectfont \(\displaystyle 0.60\)}%
\end{pgfscope}%
\begin{pgfscope}%
\pgfsetbuttcap%
\pgfsetroundjoin%
\definecolor{currentfill}{rgb}{0.000000,0.000000,0.000000}%
\pgfsetfillcolor{currentfill}%
\pgfsetlinewidth{0.803000pt}%
\definecolor{currentstroke}{rgb}{0.000000,0.000000,0.000000}%
\pgfsetstrokecolor{currentstroke}%
\pgfsetdash{}{0pt}%
\pgfsys@defobject{currentmarker}{\pgfqpoint{0.000000in}{-0.048611in}}{\pgfqpoint{0.000000in}{0.000000in}}{%
\pgfpathmoveto{\pgfqpoint{0.000000in}{0.000000in}}%
\pgfpathlineto{\pgfqpoint{0.000000in}{-0.048611in}}%
\pgfusepath{stroke,fill}%
}%
\begin{pgfscope}%
\pgfsys@transformshift{4.011096in}{0.248000in}%
\pgfsys@useobject{currentmarker}{}%
\end{pgfscope}%
\end{pgfscope}%
\begin{pgfscope}%
\pgftext[x=4.011096in,y=0.150778in,,top]{\rmfamily\fontsize{10.000000}{12.000000}\selectfont \(\displaystyle 0.75\)}%
\end{pgfscope}%
\begin{pgfscope}%
\pgfsetbuttcap%
\pgfsetroundjoin%
\definecolor{currentfill}{rgb}{0.000000,0.000000,0.000000}%
\pgfsetfillcolor{currentfill}%
\pgfsetlinewidth{0.803000pt}%
\definecolor{currentstroke}{rgb}{0.000000,0.000000,0.000000}%
\pgfsetstrokecolor{currentstroke}%
\pgfsetdash{}{0pt}%
\pgfsys@defobject{currentmarker}{\pgfqpoint{0.000000in}{-0.048611in}}{\pgfqpoint{0.000000in}{0.000000in}}{%
\pgfpathmoveto{\pgfqpoint{0.000000in}{0.000000in}}%
\pgfpathlineto{\pgfqpoint{0.000000in}{-0.048611in}}%
\pgfusepath{stroke,fill}%
}%
\begin{pgfscope}%
\pgfsys@transformshift{4.385535in}{0.248000in}%
\pgfsys@useobject{currentmarker}{}%
\end{pgfscope}%
\end{pgfscope}%
\begin{pgfscope}%
\pgftext[x=4.385535in,y=0.150778in,,top]{\rmfamily\fontsize{10.000000}{12.000000}\selectfont \(\displaystyle 0.90\)}%
\end{pgfscope}%
\begin{pgfscope}%
\pgfsetbuttcap%
\pgfsetroundjoin%
\definecolor{currentfill}{rgb}{0.000000,0.000000,0.000000}%
\pgfsetfillcolor{currentfill}%
\pgfsetlinewidth{0.803000pt}%
\definecolor{currentstroke}{rgb}{0.000000,0.000000,0.000000}%
\pgfsetstrokecolor{currentstroke}%
\pgfsetdash{}{0pt}%
\pgfsys@defobject{currentmarker}{\pgfqpoint{0.000000in}{-0.048611in}}{\pgfqpoint{0.000000in}{0.000000in}}{%
\pgfpathmoveto{\pgfqpoint{0.000000in}{0.000000in}}%
\pgfpathlineto{\pgfqpoint{0.000000in}{-0.048611in}}%
\pgfusepath{stroke,fill}%
}%
\begin{pgfscope}%
\pgfsys@transformshift{4.759974in}{0.248000in}%
\pgfsys@useobject{currentmarker}{}%
\end{pgfscope}%
\end{pgfscope}%
\begin{pgfscope}%
\pgftext[x=4.759974in,y=0.150778in,,top]{\rmfamily\fontsize{10.000000}{12.000000}\selectfont \(\displaystyle 1.05\)}%
\end{pgfscope}%
\begin{pgfscope}%
\pgfsetbuttcap%
\pgfsetroundjoin%
\definecolor{currentfill}{rgb}{0.000000,0.000000,0.000000}%
\pgfsetfillcolor{currentfill}%
\pgfsetlinewidth{0.803000pt}%
\definecolor{currentstroke}{rgb}{0.000000,0.000000,0.000000}%
\pgfsetstrokecolor{currentstroke}%
\pgfsetdash{}{0pt}%
\pgfsys@defobject{currentmarker}{\pgfqpoint{-0.048611in}{0.000000in}}{\pgfqpoint{0.000000in}{0.000000in}}{%
\pgfpathmoveto{\pgfqpoint{0.000000in}{0.000000in}}%
\pgfpathlineto{\pgfqpoint{-0.048611in}{0.000000in}}%
\pgfusepath{stroke,fill}%
}%
\begin{pgfscope}%
\pgfsys@transformshift{2.762966in}{0.495728in}%
\pgfsys@useobject{currentmarker}{}%
\end{pgfscope}%
\end{pgfscope}%
\begin{pgfscope}%
\pgfsetbuttcap%
\pgfsetroundjoin%
\definecolor{currentfill}{rgb}{0.000000,0.000000,0.000000}%
\pgfsetfillcolor{currentfill}%
\pgfsetlinewidth{0.803000pt}%
\definecolor{currentstroke}{rgb}{0.000000,0.000000,0.000000}%
\pgfsetstrokecolor{currentstroke}%
\pgfsetdash{}{0pt}%
\pgfsys@defobject{currentmarker}{\pgfqpoint{-0.048611in}{0.000000in}}{\pgfqpoint{0.000000in}{0.000000in}}{%
\pgfpathmoveto{\pgfqpoint{0.000000in}{0.000000in}}%
\pgfpathlineto{\pgfqpoint{-0.048611in}{0.000000in}}%
\pgfusepath{stroke,fill}%
}%
\begin{pgfscope}%
\pgfsys@transformshift{2.762966in}{0.805389in}%
\pgfsys@useobject{currentmarker}{}%
\end{pgfscope}%
\end{pgfscope}%
\begin{pgfscope}%
\pgfsetbuttcap%
\pgfsetroundjoin%
\definecolor{currentfill}{rgb}{0.000000,0.000000,0.000000}%
\pgfsetfillcolor{currentfill}%
\pgfsetlinewidth{0.803000pt}%
\definecolor{currentstroke}{rgb}{0.000000,0.000000,0.000000}%
\pgfsetstrokecolor{currentstroke}%
\pgfsetdash{}{0pt}%
\pgfsys@defobject{currentmarker}{\pgfqpoint{-0.048611in}{0.000000in}}{\pgfqpoint{0.000000in}{0.000000in}}{%
\pgfpathmoveto{\pgfqpoint{0.000000in}{0.000000in}}%
\pgfpathlineto{\pgfqpoint{-0.048611in}{0.000000in}}%
\pgfusepath{stroke,fill}%
}%
\begin{pgfscope}%
\pgfsys@transformshift{2.762966in}{1.115049in}%
\pgfsys@useobject{currentmarker}{}%
\end{pgfscope}%
\end{pgfscope}%
\begin{pgfscope}%
\pgfsetbuttcap%
\pgfsetroundjoin%
\definecolor{currentfill}{rgb}{0.000000,0.000000,0.000000}%
\pgfsetfillcolor{currentfill}%
\pgfsetlinewidth{0.803000pt}%
\definecolor{currentstroke}{rgb}{0.000000,0.000000,0.000000}%
\pgfsetstrokecolor{currentstroke}%
\pgfsetdash{}{0pt}%
\pgfsys@defobject{currentmarker}{\pgfqpoint{-0.048611in}{0.000000in}}{\pgfqpoint{0.000000in}{0.000000in}}{%
\pgfpathmoveto{\pgfqpoint{0.000000in}{0.000000in}}%
\pgfpathlineto{\pgfqpoint{-0.048611in}{0.000000in}}%
\pgfusepath{stroke,fill}%
}%
\begin{pgfscope}%
\pgfsys@transformshift{2.762966in}{1.424710in}%
\pgfsys@useobject{currentmarker}{}%
\end{pgfscope}%
\end{pgfscope}%
\begin{pgfscope}%
\pgfpathrectangle{\pgfqpoint{2.762966in}{0.248000in}}{\pgfqpoint{2.246634in}{1.362506in}} %
\pgfusepath{clip}%
\pgfsetrectcap%
\pgfsetroundjoin%
\pgfsetlinewidth{1.505625pt}%
\definecolor{currentstroke}{rgb}{0.121569,0.466667,0.705882}%
\pgfsetstrokecolor{currentstroke}%
\pgfsetdash{}{0pt}%
\pgfusepath{stroke}%
\end{pgfscope}%
\begin{pgfscope}%
\pgfpathrectangle{\pgfqpoint{2.762966in}{0.248000in}}{\pgfqpoint{2.246634in}{1.362506in}} %
\pgfusepath{clip}%
\pgfsetrectcap%
\pgfsetroundjoin%
\pgfsetlinewidth{1.505625pt}%
\definecolor{currentstroke}{rgb}{1.000000,0.498039,0.054902}%
\pgfsetstrokecolor{currentstroke}%
\pgfsetdash{}{0pt}%
\pgfusepath{stroke}%
\end{pgfscope}%
\begin{pgfscope}%
\pgfpathrectangle{\pgfqpoint{2.762966in}{0.248000in}}{\pgfqpoint{2.246634in}{1.362506in}} %
\pgfusepath{clip}%
\pgfsetrectcap%
\pgfsetroundjoin%
\pgfsetlinewidth{1.003750pt}%
\definecolor{currentstroke}{rgb}{1.000000,0.388235,0.278431}%
\pgfsetstrokecolor{currentstroke}%
\pgfsetdash{}{0pt}%
\pgfpathmoveto{\pgfqpoint{4.709864in}{0.238000in}}%
\pgfpathlineto{\pgfqpoint{4.705039in}{0.331717in}}%
\pgfpathlineto{\pgfqpoint{4.698554in}{0.416900in}}%
\pgfpathlineto{\pgfqpoint{4.691008in}{0.488807in}}%
\pgfpathlineto{\pgfqpoint{4.682147in}{0.553269in}}%
\pgfpathlineto{\pgfqpoint{4.672242in}{0.610098in}}%
\pgfpathlineto{\pgfqpoint{4.660415in}{0.664615in}}%
\pgfpathlineto{\pgfqpoint{4.648074in}{0.711170in}}%
\pgfpathlineto{\pgfqpoint{4.634154in}{0.754899in}}%
\pgfpathlineto{\pgfqpoint{4.618762in}{0.795477in}}%
\pgfpathlineto{\pgfqpoint{4.602060in}{0.832734in}}%
\pgfpathlineto{\pgfqpoint{4.584240in}{0.866658in}}%
\pgfpathlineto{\pgfqpoint{4.565496in}{0.897367in}}%
\pgfpathlineto{\pgfqpoint{4.546005in}{0.925070in}}%
\pgfpathlineto{\pgfqpoint{4.525918in}{0.950019in}}%
\pgfpathlineto{\pgfqpoint{4.503050in}{0.974836in}}%
\pgfpathlineto{\pgfqpoint{4.479732in}{0.996934in}}%
\pgfpathlineto{\pgfqpoint{4.453686in}{1.018486in}}%
\pgfpathlineto{\pgfqpoint{4.427321in}{1.037513in}}%
\pgfpathlineto{\pgfqpoint{4.398285in}{1.055762in}}%
\pgfpathlineto{\pgfqpoint{4.366585in}{1.072969in}}%
\pgfpathlineto{\pgfqpoint{4.332235in}{1.088917in}}%
\pgfpathlineto{\pgfqpoint{4.297722in}{1.102539in}}%
\pgfpathlineto{\pgfqpoint{4.260611in}{1.114881in}}%
\pgfpathlineto{\pgfqpoint{4.220914in}{1.125783in}}%
\pgfpathlineto{\pgfqpoint{4.178645in}{1.135103in}}%
\pgfpathlineto{\pgfqpoint{4.133817in}{1.142695in}}%
\pgfpathlineto{\pgfqpoint{4.086445in}{1.148420in}}%
\pgfpathlineto{\pgfqpoint{4.036543in}{1.152108in}}%
\pgfpathlineto{\pgfqpoint{3.986622in}{1.153558in}}%
\pgfpathlineto{\pgfqpoint{3.934202in}{1.152791in}}%
\pgfpathlineto{\pgfqpoint{3.881795in}{1.149743in}}%
\pgfpathlineto{\pgfqpoint{3.829418in}{1.144441in}}%
\pgfpathlineto{\pgfqpoint{3.777087in}{1.136846in}}%
\pgfpathlineto{\pgfqpoint{3.727307in}{1.127426in}}%
\pgfpathlineto{\pgfqpoint{3.677604in}{1.115770in}}%
\pgfpathlineto{\pgfqpoint{3.630478in}{1.102505in}}%
\pgfpathlineto{\pgfqpoint{3.583466in}{1.086956in}}%
\pgfpathlineto{\pgfqpoint{3.539059in}{1.069958in}}%
\pgfpathlineto{\pgfqpoint{3.494813in}{1.050561in}}%
\pgfpathlineto{\pgfqpoint{3.453207in}{1.029846in}}%
\pgfpathlineto{\pgfqpoint{3.414248in}{1.008029in}}%
\pgfpathlineto{\pgfqpoint{3.375526in}{0.983758in}}%
\pgfpathlineto{\pgfqpoint{3.339486in}{0.958571in}}%
\pgfpathlineto{\pgfqpoint{3.303753in}{0.930821in}}%
\pgfpathlineto{\pgfqpoint{3.270736in}{0.902411in}}%
\pgfpathlineto{\pgfqpoint{3.238105in}{0.871378in}}%
\pgfpathlineto{\pgfqpoint{3.208216in}{0.840035in}}%
\pgfpathlineto{\pgfqpoint{3.178792in}{0.806089in}}%
\pgfpathlineto{\pgfqpoint{3.149912in}{0.769379in}}%
\pgfpathlineto{\pgfqpoint{3.121667in}{0.729750in}}%
\pgfpathlineto{\pgfqpoint{3.094154in}{0.687068in}}%
\pgfpathlineto{\pgfqpoint{3.069501in}{0.644864in}}%
\pgfpathlineto{\pgfqpoint{3.045652in}{0.599906in}}%
\pgfpathlineto{\pgfqpoint{3.022697in}{0.552170in}}%
\pgfpathlineto{\pgfqpoint{3.000725in}{0.501668in}}%
\pgfpathlineto{\pgfqpoint{2.978128in}{0.443898in}}%
\pgfpathlineto{\pgfqpoint{2.956876in}{0.383067in}}%
\pgfpathlineto{\pgfqpoint{2.937047in}{0.319347in}}%
\pgfpathlineto{\pgfqpoint{2.918699in}{0.252952in}}%
\pgfpathlineto{\pgfqpoint{2.914864in}{0.238000in}}%
\pgfpathmoveto{\pgfqpoint{2.789018in}{0.238000in}}%
\pgfpathlineto{\pgfqpoint{2.805780in}{0.312934in}}%
\pgfpathlineto{\pgfqpoint{2.824392in}{0.386306in}}%
\pgfpathlineto{\pgfqpoint{2.844669in}{0.456908in}}%
\pgfpathlineto{\pgfqpoint{2.864966in}{0.519716in}}%
\pgfpathlineto{\pgfqpoint{2.886629in}{0.579649in}}%
\pgfpathlineto{\pgfqpoint{2.909590in}{0.636529in}}%
\pgfpathlineto{\pgfqpoint{2.933763in}{0.690230in}}%
\pgfpathlineto{\pgfqpoint{2.959047in}{0.740684in}}%
\pgfpathlineto{\pgfqpoint{2.985333in}{0.787883in}}%
\pgfpathlineto{\pgfqpoint{3.012509in}{0.831871in}}%
\pgfpathlineto{\pgfqpoint{3.040466in}{0.872740in}}%
\pgfpathlineto{\pgfqpoint{3.069099in}{0.910616in}}%
\pgfpathlineto{\pgfqpoint{3.098315in}{0.945652in}}%
\pgfpathlineto{\pgfqpoint{3.128027in}{0.978015in}}%
\pgfpathlineto{\pgfqpoint{3.160495in}{1.010079in}}%
\pgfpathlineto{\pgfqpoint{3.193373in}{1.039464in}}%
\pgfpathlineto{\pgfqpoint{3.228977in}{1.068216in}}%
\pgfpathlineto{\pgfqpoint{3.264904in}{1.094382in}}%
\pgfpathlineto{\pgfqpoint{3.303517in}{1.119701in}}%
\pgfpathlineto{\pgfqpoint{3.342376in}{1.142591in}}%
\pgfpathlineto{\pgfqpoint{3.383882in}{1.164503in}}%
\pgfpathlineto{\pgfqpoint{3.428028in}{1.185263in}}%
\pgfpathlineto{\pgfqpoint{3.474803in}{1.204721in}}%
\pgfpathlineto{\pgfqpoint{3.524196in}{1.222754in}}%
\pgfpathlineto{\pgfqpoint{3.576193in}{1.239252in}}%
\pgfpathlineto{\pgfqpoint{3.630782in}{1.254121in}}%
\pgfpathlineto{\pgfqpoint{3.685463in}{1.266744in}}%
\pgfpathlineto{\pgfqpoint{3.742705in}{1.277739in}}%
\pgfpathlineto{\pgfqpoint{3.802499in}{1.286990in}}%
\pgfpathlineto{\pgfqpoint{3.864834in}{1.294358in}}%
\pgfpathlineto{\pgfqpoint{3.927205in}{1.299514in}}%
\pgfpathlineto{\pgfqpoint{3.989600in}{1.302509in}}%
\pgfpathlineto{\pgfqpoint{4.052005in}{1.303343in}}%
\pgfpathlineto{\pgfqpoint{4.114408in}{1.301955in}}%
\pgfpathlineto{\pgfqpoint{4.174301in}{1.298414in}}%
\pgfpathlineto{\pgfqpoint{4.231670in}{1.292821in}}%
\pgfpathlineto{\pgfqpoint{4.286502in}{1.285234in}}%
\pgfpathlineto{\pgfqpoint{4.336293in}{1.276188in}}%
\pgfpathlineto{\pgfqpoint{4.383522in}{1.265416in}}%
\pgfpathlineto{\pgfqpoint{4.428171in}{1.252926in}}%
\pgfpathlineto{\pgfqpoint{4.467749in}{1.239630in}}%
\pgfpathlineto{\pgfqpoint{4.504721in}{1.224941in}}%
\pgfpathlineto{\pgfqpoint{4.539066in}{1.208922in}}%
\pgfpathlineto{\pgfqpoint{4.570761in}{1.191661in}}%
\pgfpathlineto{\pgfqpoint{4.599784in}{1.173287in}}%
\pgfpathlineto{\pgfqpoint{4.626116in}{1.153979in}}%
\pgfpathlineto{\pgfqpoint{4.649740in}{1.133992in}}%
\pgfpathlineto{\pgfqpoint{4.672956in}{1.111251in}}%
\pgfpathlineto{\pgfqpoint{4.693374in}{1.088012in}}%
\pgfpathlineto{\pgfqpoint{4.711011in}{1.064791in}}%
\pgfpathlineto{\pgfqpoint{4.728013in}{1.038820in}}%
\pgfpathlineto{\pgfqpoint{4.744187in}{1.009781in}}%
\pgfpathlineto{\pgfqpoint{4.757476in}{0.981649in}}%
\pgfpathlineto{\pgfqpoint{4.769778in}{0.950879in}}%
\pgfpathlineto{\pgfqpoint{4.780918in}{0.917500in}}%
\pgfpathlineto{\pgfqpoint{4.790754in}{0.881687in}}%
\pgfpathlineto{\pgfqpoint{4.800289in}{0.838179in}}%
\pgfpathlineto{\pgfqpoint{4.807995in}{0.792493in}}%
\pgfpathlineto{\pgfqpoint{4.814614in}{0.739250in}}%
\pgfpathlineto{\pgfqpoint{4.819736in}{0.678656in}}%
\pgfpathlineto{\pgfqpoint{4.823152in}{0.611077in}}%
\pgfpathlineto{\pgfqpoint{4.824898in}{0.530696in}}%
\pgfpathlineto{\pgfqpoint{4.824624in}{0.431620in}}%
\pgfpathlineto{\pgfqpoint{4.821848in}{0.307956in}}%
\pgfpathlineto{\pgfqpoint{4.819507in}{0.238000in}}%
\pgfpathlineto{\pgfqpoint{4.819507in}{0.238000in}}%
\pgfusepath{stroke}%
\end{pgfscope}%
\begin{pgfscope}%
\pgfpathrectangle{\pgfqpoint{2.762966in}{0.248000in}}{\pgfqpoint{2.246634in}{1.362506in}} %
\pgfusepath{clip}%
\pgfsetrectcap%
\pgfsetroundjoin%
\pgfsetlinewidth{1.003750pt}%
\definecolor{currentstroke}{rgb}{1.000000,0.388235,0.278431}%
\pgfsetstrokecolor{currentstroke}%
\pgfsetdash{}{0pt}%
\pgfpathmoveto{\pgfqpoint{4.904366in}{0.238000in}}%
\pgfpathlineto{\pgfqpoint{4.936676in}{0.569415in}}%
\pgfpathlineto{\pgfqpoint{4.947584in}{0.696613in}}%
\pgfpathlineto{\pgfqpoint{4.952889in}{0.782298in}}%
\pgfpathlineto{\pgfqpoint{4.955039in}{0.850195in}}%
\pgfpathlineto{\pgfqpoint{4.954781in}{0.905909in}}%
\pgfpathlineto{\pgfqpoint{4.952444in}{0.955087in}}%
\pgfpathlineto{\pgfqpoint{4.948357in}{0.997209in}}%
\pgfpathlineto{\pgfqpoint{4.942086in}{1.037636in}}%
\pgfpathlineto{\pgfqpoint{4.934918in}{1.070236in}}%
\pgfpathlineto{\pgfqpoint{4.926198in}{1.100420in}}%
\pgfpathlineto{\pgfqpoint{4.916135in}{1.127917in}}%
\pgfpathlineto{\pgfqpoint{4.904983in}{1.152701in}}%
\pgfpathlineto{\pgfqpoint{4.890917in}{1.178397in}}%
\pgfpathlineto{\pgfqpoint{4.876007in}{1.200982in}}%
\pgfpathlineto{\pgfqpoint{4.858246in}{1.223610in}}%
\pgfpathlineto{\pgfqpoint{4.839937in}{1.243379in}}%
\pgfpathlineto{\pgfqpoint{4.818881in}{1.262797in}}%
\pgfpathlineto{\pgfqpoint{4.795093in}{1.281554in}}%
\pgfpathlineto{\pgfqpoint{4.768598in}{1.299426in}}%
\pgfpathlineto{\pgfqpoint{4.739422in}{1.316254in}}%
\pgfpathlineto{\pgfqpoint{4.707596in}{1.331960in}}%
\pgfpathlineto{\pgfqpoint{4.673145in}{1.346519in}}%
\pgfpathlineto{\pgfqpoint{4.633618in}{1.360725in}}%
\pgfpathlineto{\pgfqpoint{4.591496in}{1.373502in}}%
\pgfpathlineto{\pgfqpoint{4.544314in}{1.385477in}}%
\pgfpathlineto{\pgfqpoint{4.492078in}{1.396386in}}%
\pgfpathlineto{\pgfqpoint{4.434796in}{1.406017in}}%
\pgfpathlineto{\pgfqpoint{4.372477in}{1.414184in}}%
\pgfpathlineto{\pgfqpoint{4.305130in}{1.420715in}}%
\pgfpathlineto{\pgfqpoint{4.245240in}{1.424474in}}%
\pgfpathlineto{\pgfqpoint{4.170362in}{1.427454in}}%
\pgfpathlineto{\pgfqpoint{4.092979in}{1.428307in}}%
\pgfpathlineto{\pgfqpoint{4.013102in}{1.426915in}}%
\pgfpathlineto{\pgfqpoint{3.933235in}{1.423242in}}%
\pgfpathlineto{\pgfqpoint{3.845906in}{1.416862in}}%
\pgfpathlineto{\pgfqpoint{3.771089in}{1.408806in}}%
\pgfpathlineto{\pgfqpoint{3.698811in}{1.398795in}}%
\pgfpathlineto{\pgfqpoint{3.641516in}{1.389843in}}%
\pgfpathlineto{\pgfqpoint{3.576823in}{1.376932in}}%
\pgfpathlineto{\pgfqpoint{3.514698in}{1.362253in}}%
\pgfpathlineto{\pgfqpoint{3.455157in}{1.345795in}}%
\pgfpathlineto{\pgfqpoint{3.400689in}{1.328406in}}%
\pgfpathlineto{\pgfqpoint{3.388296in}{1.324957in}}%
\pgfpathlineto{\pgfqpoint{3.380849in}{1.323194in}}%
\pgfpathlineto{\pgfqpoint{3.331504in}{1.304381in}}%
\pgfpathlineto{\pgfqpoint{3.265091in}{1.276156in}}%
\pgfpathlineto{\pgfqpoint{3.221079in}{1.253711in}}%
\pgfpathlineto{\pgfqpoint{3.179729in}{1.230060in}}%
\pgfpathlineto{\pgfqpoint{3.141051in}{1.205359in}}%
\pgfpathlineto{\pgfqpoint{3.102661in}{1.178024in}}%
\pgfpathlineto{\pgfqpoint{3.069364in}{1.151710in}}%
\pgfpathlineto{\pgfqpoint{3.034030in}{1.120973in}}%
\pgfpathlineto{\pgfqpoint{3.001502in}{1.089287in}}%
\pgfpathlineto{\pgfqpoint{2.971734in}{1.057235in}}%
\pgfpathlineto{\pgfqpoint{2.942470in}{1.022454in}}%
\pgfpathlineto{\pgfqpoint{2.913795in}{0.984771in}}%
\pgfpathlineto{\pgfqpoint{2.885808in}{0.944031in}}%
\pgfpathlineto{\pgfqpoint{2.875598in}{0.926288in}}%
\pgfpathlineto{\pgfqpoint{2.830759in}{0.847683in}}%
\pgfpathlineto{\pgfqpoint{2.807480in}{0.800926in}}%
\pgfpathlineto{\pgfqpoint{2.785150in}{0.751398in}}%
\pgfpathlineto{\pgfqpoint{2.762112in}{0.694710in}}%
\pgfpathlineto{\pgfqpoint{2.752966in}{0.670413in}}%
\pgfpathlineto{\pgfqpoint{2.752966in}{0.670413in}}%
\pgfusepath{stroke}%
\end{pgfscope}%
\begin{pgfscope}%
\pgfpathrectangle{\pgfqpoint{2.762966in}{0.248000in}}{\pgfqpoint{2.246634in}{1.362506in}} %
\pgfusepath{clip}%
\pgfsetrectcap%
\pgfsetroundjoin%
\pgfsetlinewidth{1.003750pt}%
\definecolor{currentstroke}{rgb}{1.000000,0.388235,0.278431}%
\pgfsetstrokecolor{currentstroke}%
\pgfsetdash{}{0pt}%
\pgfpathmoveto{\pgfqpoint{4.903287in}{0.238000in}}%
\pgfpathlineto{\pgfqpoint{4.937753in}{0.593812in}}%
\pgfpathlineto{\pgfqpoint{4.947122in}{0.709152in}}%
\pgfpathlineto{\pgfqpoint{4.951918in}{0.795018in}}%
\pgfpathlineto{\pgfqpoint{4.953458in}{0.863013in}}%
\pgfpathlineto{\pgfqpoint{4.952515in}{0.918677in}}%
\pgfpathlineto{\pgfqpoint{4.949425in}{0.967595in}}%
\pgfpathlineto{\pgfqpoint{4.944572in}{1.009212in}}%
\pgfpathlineto{\pgfqpoint{4.937501in}{1.048817in}}%
\pgfpathlineto{\pgfqpoint{4.929684in}{1.080487in}}%
\pgfpathlineto{\pgfqpoint{4.920398in}{1.109614in}}%
\pgfpathlineto{\pgfqpoint{4.909873in}{1.136029in}}%
\pgfpathlineto{\pgfqpoint{4.896366in}{1.163502in}}%
\pgfpathlineto{\pgfqpoint{4.881868in}{1.187676in}}%
\pgfpathlineto{\pgfqpoint{4.866658in}{1.208999in}}%
\pgfpathlineto{\pgfqpoint{4.848650in}{1.230396in}}%
\pgfpathlineto{\pgfqpoint{4.827834in}{1.251342in}}%
\pgfpathlineto{\pgfqpoint{4.806604in}{1.269567in}}%
\pgfpathlineto{\pgfqpoint{4.782683in}{1.287258in}}%
\pgfpathlineto{\pgfqpoint{4.756086in}{1.304177in}}%
\pgfpathlineto{\pgfqpoint{4.726835in}{1.320179in}}%
\pgfpathlineto{\pgfqpoint{4.694952in}{1.335169in}}%
\pgfpathlineto{\pgfqpoint{4.657990in}{1.350009in}}%
\pgfpathlineto{\pgfqpoint{4.618422in}{1.363485in}}%
\pgfpathlineto{\pgfqpoint{4.573787in}{1.376284in}}%
\pgfpathlineto{\pgfqpoint{4.524091in}{1.388110in}}%
\pgfpathlineto{\pgfqpoint{4.469341in}{1.398726in}}%
\pgfpathlineto{\pgfqpoint{4.409546in}{1.407931in}}%
\pgfpathlineto{\pgfqpoint{4.347211in}{1.415296in}}%
\pgfpathlineto{\pgfqpoint{4.279852in}{1.421033in}}%
\pgfpathlineto{\pgfqpoint{4.242418in}{1.423154in}}%
\pgfpathlineto{\pgfqpoint{4.167540in}{1.426088in}}%
\pgfpathlineto{\pgfqpoint{4.090157in}{1.426884in}}%
\pgfpathlineto{\pgfqpoint{4.012776in}{1.425503in}}%
\pgfpathlineto{\pgfqpoint{3.935405in}{1.421967in}}%
\pgfpathlineto{\pgfqpoint{3.848075in}{1.415603in}}%
\pgfpathlineto{\pgfqpoint{3.773258in}{1.407584in}}%
\pgfpathlineto{\pgfqpoint{3.700979in}{1.397603in}}%
\pgfpathlineto{\pgfqpoint{3.616305in}{1.383389in}}%
\pgfpathlineto{\pgfqpoint{3.554130in}{1.370090in}}%
\pgfpathlineto{\pgfqpoint{3.494524in}{1.355137in}}%
\pgfpathlineto{\pgfqpoint{3.437503in}{1.338523in}}%
\pgfpathlineto{\pgfqpoint{3.358330in}{1.312618in}}%
\pgfpathlineto{\pgfqpoint{3.289336in}{1.285093in}}%
\pgfpathlineto{\pgfqpoint{3.245227in}{1.263863in}}%
\pgfpathlineto{\pgfqpoint{3.201326in}{1.240116in}}%
\pgfpathlineto{\pgfqpoint{3.160109in}{1.215079in}}%
\pgfpathlineto{\pgfqpoint{3.121588in}{1.188911in}}%
\pgfpathlineto{\pgfqpoint{3.085765in}{1.161884in}}%
\pgfpathlineto{\pgfqpoint{3.052652in}{1.134170in}}%
\pgfpathlineto{\pgfqpoint{3.017551in}{1.101848in}}%
\pgfpathlineto{\pgfqpoint{2.987580in}{1.070989in}}%
\pgfpathlineto{\pgfqpoint{2.958072in}{1.037494in}}%
\pgfpathlineto{\pgfqpoint{2.929113in}{1.001173in}}%
\pgfpathlineto{\pgfqpoint{2.900795in}{0.961865in}}%
\pgfpathlineto{\pgfqpoint{2.879641in}{0.929041in}}%
\pgfpathlineto{\pgfqpoint{2.873697in}{0.917784in}}%
\pgfpathlineto{\pgfqpoint{2.849017in}{0.875677in}}%
\pgfpathlineto{\pgfqpoint{2.825150in}{0.830777in}}%
\pgfpathlineto{\pgfqpoint{2.802182in}{0.783081in}}%
\pgfpathlineto{\pgfqpoint{2.778404in}{0.728306in}}%
\pgfpathlineto{\pgfqpoint{2.755852in}{0.670425in}}%
\pgfpathlineto{\pgfqpoint{2.752966in}{0.662544in}}%
\pgfpathlineto{\pgfqpoint{2.752966in}{0.662544in}}%
\pgfusepath{stroke}%
\end{pgfscope}%
\begin{pgfscope}%
\pgfpathrectangle{\pgfqpoint{2.762966in}{0.248000in}}{\pgfqpoint{2.246634in}{1.362506in}} %
\pgfusepath{clip}%
\pgfsetbuttcap%
\pgfsetroundjoin%
\pgfsetlinewidth{1.003750pt}%
\definecolor{currentstroke}{rgb}{0.000000,0.000000,0.000000}%
\pgfsetstrokecolor{currentstroke}%
\pgfsetdash{{1.000000pt}{1.650000pt}}{0.000000pt}%
\pgfpathmoveto{\pgfqpoint{2.917714in}{0.938981in}}%
\pgfpathlineto{\pgfqpoint{2.945915in}{0.978804in}}%
\pgfpathlineto{\pgfqpoint{2.974771in}{1.015625in}}%
\pgfpathlineto{\pgfqpoint{3.004188in}{1.049607in}}%
\pgfpathlineto{\pgfqpoint{3.034081in}{1.080930in}}%
\pgfpathlineto{\pgfqpoint{3.066721in}{1.111901in}}%
\pgfpathlineto{\pgfqpoint{3.099749in}{1.140232in}}%
\pgfpathlineto{\pgfqpoint{3.135491in}{1.167912in}}%
\pgfpathlineto{\pgfqpoint{3.171534in}{1.193078in}}%
\pgfpathlineto{\pgfqpoint{3.210249in}{1.217417in}}%
\pgfpathlineto{\pgfqpoint{3.251631in}{1.240734in}}%
\pgfpathlineto{\pgfqpoint{3.295668in}{1.262869in}}%
\pgfpathlineto{\pgfqpoint{3.342346in}{1.283701in}}%
\pgfpathlineto{\pgfqpoint{3.391652in}{1.303140in}}%
\pgfpathlineto{\pgfqpoint{3.443570in}{1.321119in}}%
\pgfpathlineto{\pgfqpoint{3.498084in}{1.337594in}}%
\pgfpathlineto{\pgfqpoint{3.555181in}{1.352531in}}%
\pgfpathlineto{\pgfqpoint{3.617335in}{1.366416in}}%
\pgfpathlineto{\pgfqpoint{3.682053in}{1.378544in}}%
\pgfpathlineto{\pgfqpoint{3.749322in}{1.388899in}}%
\pgfpathlineto{\pgfqpoint{3.819132in}{1.397450in}}%
\pgfpathlineto{\pgfqpoint{3.893968in}{1.404338in}}%
\pgfpathlineto{\pgfqpoint{3.968832in}{1.409003in}}%
\pgfpathlineto{\pgfqpoint{4.046208in}{1.411573in}}%
\pgfpathlineto{\pgfqpoint{4.123592in}{1.411865in}}%
\pgfpathlineto{\pgfqpoint{4.198475in}{1.409895in}}%
\pgfpathlineto{\pgfqpoint{4.270847in}{1.405732in}}%
\pgfpathlineto{\pgfqpoint{4.338201in}{1.399656in}}%
\pgfpathlineto{\pgfqpoint{4.400528in}{1.391867in}}%
\pgfpathlineto{\pgfqpoint{4.457818in}{1.382570in}}%
\pgfpathlineto{\pgfqpoint{4.510065in}{1.371968in}}%
\pgfpathlineto{\pgfqpoint{4.559740in}{1.359617in}}%
\pgfpathlineto{\pgfqpoint{4.604346in}{1.346216in}}%
\pgfpathlineto{\pgfqpoint{4.643877in}{1.332076in}}%
\pgfpathlineto{\pgfqpoint{4.680788in}{1.316473in}}%
\pgfpathlineto{\pgfqpoint{4.712608in}{1.300685in}}%
\pgfpathlineto{\pgfqpoint{4.741778in}{1.283802in}}%
\pgfpathlineto{\pgfqpoint{4.768273in}{1.265921in}}%
\pgfpathlineto{\pgfqpoint{4.792066in}{1.247202in}}%
\pgfpathlineto{\pgfqpoint{4.813137in}{1.227887in}}%
\pgfpathlineto{\pgfqpoint{4.831483in}{1.208329in}}%
\pgfpathlineto{\pgfqpoint{4.849317in}{1.186055in}}%
\pgfpathlineto{\pgfqpoint{4.864337in}{1.163918in}}%
\pgfpathlineto{\pgfqpoint{4.878601in}{1.138901in}}%
\pgfpathlineto{\pgfqpoint{4.891840in}{1.110636in}}%
\pgfpathlineto{\pgfqpoint{4.902120in}{1.083635in}}%
\pgfpathlineto{\pgfqpoint{4.911181in}{1.054070in}}%
\pgfpathlineto{\pgfqpoint{4.918827in}{1.022142in}}%
\pgfpathlineto{\pgfqpoint{4.925816in}{0.982443in}}%
\pgfpathlineto{\pgfqpoint{4.930741in}{0.940874in}}%
\pgfpathlineto{\pgfqpoint{4.934089in}{0.892057in}}%
\pgfpathlineto{\pgfqpoint{4.935465in}{0.836446in}}%
\pgfpathlineto{\pgfqpoint{4.934752in}{0.774556in}}%
\pgfpathlineto{\pgfqpoint{4.931394in}{0.694492in}}%
\pgfpathlineto{\pgfqpoint{4.924917in}{0.596723in}}%
\pgfpathlineto{\pgfqpoint{4.913283in}{0.457240in}}%
\pgfpathlineto{\pgfqpoint{4.893196in}{0.238000in}}%
\pgfpathlineto{\pgfqpoint{4.893196in}{0.238000in}}%
\pgfusepath{stroke}%
\end{pgfscope}%
\begin{pgfscope}%
\pgfpathrectangle{\pgfqpoint{2.762966in}{0.248000in}}{\pgfqpoint{2.246634in}{1.362506in}} %
\pgfusepath{clip}%
\pgfsetbuttcap%
\pgfsetroundjoin%
\pgfsetlinewidth{1.003750pt}%
\definecolor{currentstroke}{rgb}{0.000000,0.000000,0.000000}%
\pgfsetstrokecolor{currentstroke}%
\pgfsetdash{{1.000000pt}{1.650000pt}}{0.000000pt}%
\pgfpathmoveto{\pgfqpoint{0.000000in}{0.000000in}}%
\pgfusepath{stroke}%
\end{pgfscope}%
\begin{pgfscope}%
\pgfpathrectangle{\pgfqpoint{2.762966in}{0.248000in}}{\pgfqpoint{2.246634in}{1.362506in}} %
\pgfusepath{clip}%
\pgfsetbuttcap%
\pgfsetroundjoin%
\pgfsetlinewidth{1.003750pt}%
\definecolor{currentstroke}{rgb}{0.000000,0.000000,0.000000}%
\pgfsetstrokecolor{currentstroke}%
\pgfsetdash{{1.000000pt}{1.650000pt}}{0.000000pt}%
\pgfpathmoveto{\pgfqpoint{0.000000in}{0.000000in}}%
\pgfusepath{stroke}%
\end{pgfscope}%
\begin{pgfscope}%
\pgfpathrectangle{\pgfqpoint{2.762966in}{0.248000in}}{\pgfqpoint{2.246634in}{1.362506in}} %
\pgfusepath{clip}%
\pgfsetbuttcap%
\pgfsetroundjoin%
\pgfsetlinewidth{1.003750pt}%
\definecolor{currentstroke}{rgb}{0.000000,0.000000,0.000000}%
\pgfsetstrokecolor{currentstroke}%
\pgfsetdash{{1.000000pt}{1.650000pt}}{0.000000pt}%
\pgfpathmoveto{\pgfqpoint{4.709864in}{0.238000in}}%
\pgfpathlineto{\pgfqpoint{4.705039in}{0.331718in}}%
\pgfpathlineto{\pgfqpoint{4.698554in}{0.416901in}}%
\pgfpathlineto{\pgfqpoint{4.691008in}{0.488807in}}%
\pgfpathlineto{\pgfqpoint{4.682147in}{0.553270in}}%
\pgfpathlineto{\pgfqpoint{4.672242in}{0.610098in}}%
\pgfpathlineto{\pgfqpoint{4.660415in}{0.664615in}}%
\pgfpathlineto{\pgfqpoint{4.648074in}{0.711171in}}%
\pgfpathlineto{\pgfqpoint{4.634154in}{0.754899in}}%
\pgfpathlineto{\pgfqpoint{4.618761in}{0.795477in}}%
\pgfpathlineto{\pgfqpoint{4.602060in}{0.832734in}}%
\pgfpathlineto{\pgfqpoint{4.584240in}{0.866658in}}%
\pgfpathlineto{\pgfqpoint{4.565496in}{0.897367in}}%
\pgfpathlineto{\pgfqpoint{4.546005in}{0.925070in}}%
\pgfpathlineto{\pgfqpoint{4.525918in}{0.950019in}}%
\pgfpathlineto{\pgfqpoint{4.503050in}{0.974836in}}%
\pgfpathlineto{\pgfqpoint{4.479732in}{0.996934in}}%
\pgfpathlineto{\pgfqpoint{4.453686in}{1.018486in}}%
\pgfpathlineto{\pgfqpoint{4.427321in}{1.037513in}}%
\pgfpathlineto{\pgfqpoint{4.398285in}{1.055762in}}%
\pgfpathlineto{\pgfqpoint{4.366585in}{1.072969in}}%
\pgfpathlineto{\pgfqpoint{4.332234in}{1.088917in}}%
\pgfpathlineto{\pgfqpoint{4.297722in}{1.102539in}}%
\pgfpathlineto{\pgfqpoint{4.260611in}{1.114881in}}%
\pgfpathlineto{\pgfqpoint{4.220914in}{1.125783in}}%
\pgfpathlineto{\pgfqpoint{4.178645in}{1.135103in}}%
\pgfpathlineto{\pgfqpoint{4.133817in}{1.142695in}}%
\pgfpathlineto{\pgfqpoint{4.086445in}{1.148420in}}%
\pgfpathlineto{\pgfqpoint{4.036543in}{1.152108in}}%
\pgfpathlineto{\pgfqpoint{3.986622in}{1.153558in}}%
\pgfpathlineto{\pgfqpoint{3.934202in}{1.152791in}}%
\pgfpathlineto{\pgfqpoint{3.881795in}{1.149743in}}%
\pgfpathlineto{\pgfqpoint{3.829418in}{1.144441in}}%
\pgfpathlineto{\pgfqpoint{3.777087in}{1.136846in}}%
\pgfpathlineto{\pgfqpoint{3.727307in}{1.127426in}}%
\pgfpathlineto{\pgfqpoint{3.677604in}{1.115770in}}%
\pgfpathlineto{\pgfqpoint{3.630478in}{1.102505in}}%
\pgfpathlineto{\pgfqpoint{3.583466in}{1.086956in}}%
\pgfpathlineto{\pgfqpoint{3.539059in}{1.069958in}}%
\pgfpathlineto{\pgfqpoint{3.494813in}{1.050561in}}%
\pgfpathlineto{\pgfqpoint{3.453207in}{1.029846in}}%
\pgfpathlineto{\pgfqpoint{3.414248in}{1.008029in}}%
\pgfpathlineto{\pgfqpoint{3.375526in}{0.983757in}}%
\pgfpathlineto{\pgfqpoint{3.339486in}{0.958571in}}%
\pgfpathlineto{\pgfqpoint{3.303753in}{0.930821in}}%
\pgfpathlineto{\pgfqpoint{3.270736in}{0.902411in}}%
\pgfpathlineto{\pgfqpoint{3.238105in}{0.871378in}}%
\pgfpathlineto{\pgfqpoint{3.208215in}{0.840035in}}%
\pgfpathlineto{\pgfqpoint{3.178792in}{0.806089in}}%
\pgfpathlineto{\pgfqpoint{3.149912in}{0.769378in}}%
\pgfpathlineto{\pgfqpoint{3.121667in}{0.729750in}}%
\pgfpathlineto{\pgfqpoint{3.094154in}{0.687068in}}%
\pgfpathlineto{\pgfqpoint{3.069501in}{0.644864in}}%
\pgfpathlineto{\pgfqpoint{3.045652in}{0.599906in}}%
\pgfpathlineto{\pgfqpoint{3.022697in}{0.552170in}}%
\pgfpathlineto{\pgfqpoint{3.000725in}{0.501668in}}%
\pgfpathlineto{\pgfqpoint{2.978128in}{0.443898in}}%
\pgfpathlineto{\pgfqpoint{2.956876in}{0.383067in}}%
\pgfpathlineto{\pgfqpoint{2.937047in}{0.319347in}}%
\pgfpathlineto{\pgfqpoint{2.918699in}{0.252952in}}%
\pgfpathlineto{\pgfqpoint{2.914864in}{0.238000in}}%
\pgfpathmoveto{\pgfqpoint{2.789018in}{0.238000in}}%
\pgfpathlineto{\pgfqpoint{2.805780in}{0.312934in}}%
\pgfpathlineto{\pgfqpoint{2.824392in}{0.386306in}}%
\pgfpathlineto{\pgfqpoint{2.844669in}{0.456908in}}%
\pgfpathlineto{\pgfqpoint{2.864966in}{0.519716in}}%
\pgfpathlineto{\pgfqpoint{2.886629in}{0.579649in}}%
\pgfpathlineto{\pgfqpoint{2.909590in}{0.636529in}}%
\pgfpathlineto{\pgfqpoint{2.933763in}{0.690230in}}%
\pgfpathlineto{\pgfqpoint{2.959047in}{0.740684in}}%
\pgfpathlineto{\pgfqpoint{2.985333in}{0.787883in}}%
\pgfpathlineto{\pgfqpoint{3.012509in}{0.831871in}}%
\pgfpathlineto{\pgfqpoint{3.040466in}{0.872740in}}%
\pgfpathlineto{\pgfqpoint{3.069099in}{0.910616in}}%
\pgfpathlineto{\pgfqpoint{3.098315in}{0.945652in}}%
\pgfpathlineto{\pgfqpoint{3.128027in}{0.978015in}}%
\pgfpathlineto{\pgfqpoint{3.160495in}{1.010079in}}%
\pgfpathlineto{\pgfqpoint{3.193373in}{1.039464in}}%
\pgfpathlineto{\pgfqpoint{3.228977in}{1.068216in}}%
\pgfpathlineto{\pgfqpoint{3.264904in}{1.094382in}}%
\pgfpathlineto{\pgfqpoint{3.303517in}{1.119701in}}%
\pgfpathlineto{\pgfqpoint{3.342376in}{1.142591in}}%
\pgfpathlineto{\pgfqpoint{3.383882in}{1.164503in}}%
\pgfpathlineto{\pgfqpoint{3.428028in}{1.185263in}}%
\pgfpathlineto{\pgfqpoint{3.474803in}{1.204721in}}%
\pgfpathlineto{\pgfqpoint{3.524196in}{1.222754in}}%
\pgfpathlineto{\pgfqpoint{3.576193in}{1.239252in}}%
\pgfpathlineto{\pgfqpoint{3.630782in}{1.254121in}}%
\pgfpathlineto{\pgfqpoint{3.685463in}{1.266744in}}%
\pgfpathlineto{\pgfqpoint{3.742705in}{1.277739in}}%
\pgfpathlineto{\pgfqpoint{3.802499in}{1.286990in}}%
\pgfpathlineto{\pgfqpoint{3.864834in}{1.294358in}}%
\pgfpathlineto{\pgfqpoint{3.927205in}{1.299514in}}%
\pgfpathlineto{\pgfqpoint{3.989600in}{1.302509in}}%
\pgfpathlineto{\pgfqpoint{4.052005in}{1.303343in}}%
\pgfpathlineto{\pgfqpoint{4.114408in}{1.301955in}}%
\pgfpathlineto{\pgfqpoint{4.174301in}{1.298414in}}%
\pgfpathlineto{\pgfqpoint{4.231670in}{1.292821in}}%
\pgfpathlineto{\pgfqpoint{4.286502in}{1.285234in}}%
\pgfpathlineto{\pgfqpoint{4.336293in}{1.276188in}}%
\pgfpathlineto{\pgfqpoint{4.383522in}{1.265416in}}%
\pgfpathlineto{\pgfqpoint{4.428171in}{1.252926in}}%
\pgfpathlineto{\pgfqpoint{4.467749in}{1.239630in}}%
\pgfpathlineto{\pgfqpoint{4.504721in}{1.224941in}}%
\pgfpathlineto{\pgfqpoint{4.539066in}{1.208922in}}%
\pgfpathlineto{\pgfqpoint{4.570761in}{1.191661in}}%
\pgfpathlineto{\pgfqpoint{4.599784in}{1.173287in}}%
\pgfpathlineto{\pgfqpoint{4.626116in}{1.153979in}}%
\pgfpathlineto{\pgfqpoint{4.649740in}{1.133992in}}%
\pgfpathlineto{\pgfqpoint{4.672956in}{1.111251in}}%
\pgfpathlineto{\pgfqpoint{4.693374in}{1.088012in}}%
\pgfpathlineto{\pgfqpoint{4.711011in}{1.064791in}}%
\pgfpathlineto{\pgfqpoint{4.728013in}{1.038820in}}%
\pgfpathlineto{\pgfqpoint{4.744187in}{1.009781in}}%
\pgfpathlineto{\pgfqpoint{4.757476in}{0.981649in}}%
\pgfpathlineto{\pgfqpoint{4.769778in}{0.950879in}}%
\pgfpathlineto{\pgfqpoint{4.780918in}{0.917500in}}%
\pgfpathlineto{\pgfqpoint{4.790754in}{0.881687in}}%
\pgfpathlineto{\pgfqpoint{4.800289in}{0.838179in}}%
\pgfpathlineto{\pgfqpoint{4.807995in}{0.792493in}}%
\pgfpathlineto{\pgfqpoint{4.814614in}{0.739250in}}%
\pgfpathlineto{\pgfqpoint{4.819736in}{0.678656in}}%
\pgfpathlineto{\pgfqpoint{4.823152in}{0.611077in}}%
\pgfpathlineto{\pgfqpoint{4.824898in}{0.530696in}}%
\pgfpathlineto{\pgfqpoint{4.824624in}{0.431620in}}%
\pgfpathlineto{\pgfqpoint{4.821848in}{0.307956in}}%
\pgfpathlineto{\pgfqpoint{4.819507in}{0.238000in}}%
\pgfpathlineto{\pgfqpoint{4.819507in}{0.238000in}}%
\pgfusepath{stroke}%
\end{pgfscope}%
\begin{pgfscope}%
\pgfpathrectangle{\pgfqpoint{2.762966in}{0.248000in}}{\pgfqpoint{2.246634in}{1.362506in}} %
\pgfusepath{clip}%
\pgfsetbuttcap%
\pgfsetroundjoin%
\pgfsetlinewidth{1.003750pt}%
\definecolor{currentstroke}{rgb}{0.000000,0.000000,0.000000}%
\pgfsetstrokecolor{currentstroke}%
\pgfsetdash{{1.000000pt}{1.650000pt}}{0.000000pt}%
\pgfpathmoveto{\pgfqpoint{4.904366in}{0.238000in}}%
\pgfpathlineto{\pgfqpoint{4.936676in}{0.569415in}}%
\pgfpathlineto{\pgfqpoint{4.947584in}{0.696613in}}%
\pgfpathlineto{\pgfqpoint{4.952889in}{0.782298in}}%
\pgfpathlineto{\pgfqpoint{4.955039in}{0.850195in}}%
\pgfpathlineto{\pgfqpoint{4.954781in}{0.905908in}}%
\pgfpathlineto{\pgfqpoint{4.952444in}{0.955087in}}%
\pgfpathlineto{\pgfqpoint{4.948357in}{0.997209in}}%
\pgfpathlineto{\pgfqpoint{4.942086in}{1.037636in}}%
\pgfpathlineto{\pgfqpoint{4.934918in}{1.070236in}}%
\pgfpathlineto{\pgfqpoint{4.926198in}{1.100419in}}%
\pgfpathlineto{\pgfqpoint{4.916135in}{1.127917in}}%
\pgfpathlineto{\pgfqpoint{4.904983in}{1.152701in}}%
\pgfpathlineto{\pgfqpoint{4.890917in}{1.178396in}}%
\pgfpathlineto{\pgfqpoint{4.876007in}{1.200982in}}%
\pgfpathlineto{\pgfqpoint{4.858246in}{1.223609in}}%
\pgfpathlineto{\pgfqpoint{4.839937in}{1.243379in}}%
\pgfpathlineto{\pgfqpoint{4.818881in}{1.262797in}}%
\pgfpathlineto{\pgfqpoint{4.795093in}{1.281554in}}%
\pgfpathlineto{\pgfqpoint{4.768598in}{1.299426in}}%
\pgfpathlineto{\pgfqpoint{4.739422in}{1.316254in}}%
\pgfpathlineto{\pgfqpoint{4.707596in}{1.331960in}}%
\pgfpathlineto{\pgfqpoint{4.673145in}{1.346518in}}%
\pgfpathlineto{\pgfqpoint{4.633618in}{1.360725in}}%
\pgfpathlineto{\pgfqpoint{4.591496in}{1.373502in}}%
\pgfpathlineto{\pgfqpoint{4.544314in}{1.385476in}}%
\pgfpathlineto{\pgfqpoint{4.492078in}{1.396386in}}%
\pgfpathlineto{\pgfqpoint{4.434796in}{1.406017in}}%
\pgfpathlineto{\pgfqpoint{4.372477in}{1.414184in}}%
\pgfpathlineto{\pgfqpoint{4.305130in}{1.420715in}}%
\pgfpathlineto{\pgfqpoint{4.245240in}{1.424474in}}%
\pgfpathlineto{\pgfqpoint{4.170362in}{1.427454in}}%
\pgfpathlineto{\pgfqpoint{4.092979in}{1.428307in}}%
\pgfpathlineto{\pgfqpoint{4.013102in}{1.426915in}}%
\pgfpathlineto{\pgfqpoint{3.933235in}{1.423242in}}%
\pgfpathlineto{\pgfqpoint{3.845906in}{1.416862in}}%
\pgfpathlineto{\pgfqpoint{3.771089in}{1.408806in}}%
\pgfpathlineto{\pgfqpoint{3.698811in}{1.398795in}}%
\pgfpathlineto{\pgfqpoint{3.641516in}{1.389843in}}%
\pgfpathlineto{\pgfqpoint{3.576823in}{1.376932in}}%
\pgfpathlineto{\pgfqpoint{3.514698in}{1.362253in}}%
\pgfpathlineto{\pgfqpoint{3.455157in}{1.345795in}}%
\pgfpathlineto{\pgfqpoint{3.400689in}{1.328406in}}%
\pgfpathlineto{\pgfqpoint{3.388296in}{1.324957in}}%
\pgfpathlineto{\pgfqpoint{3.380849in}{1.323194in}}%
\pgfpathlineto{\pgfqpoint{3.331504in}{1.304381in}}%
\pgfpathlineto{\pgfqpoint{3.265091in}{1.276156in}}%
\pgfpathlineto{\pgfqpoint{3.221079in}{1.253711in}}%
\pgfpathlineto{\pgfqpoint{3.179729in}{1.230060in}}%
\pgfpathlineto{\pgfqpoint{3.141051in}{1.205359in}}%
\pgfpathlineto{\pgfqpoint{3.102661in}{1.178024in}}%
\pgfpathlineto{\pgfqpoint{3.069364in}{1.151710in}}%
\pgfpathlineto{\pgfqpoint{3.034030in}{1.120973in}}%
\pgfpathlineto{\pgfqpoint{3.001502in}{1.089287in}}%
\pgfpathlineto{\pgfqpoint{2.971734in}{1.057235in}}%
\pgfpathlineto{\pgfqpoint{2.942470in}{1.022454in}}%
\pgfpathlineto{\pgfqpoint{2.913795in}{0.984771in}}%
\pgfpathlineto{\pgfqpoint{2.885808in}{0.944031in}}%
\pgfpathlineto{\pgfqpoint{2.875598in}{0.926288in}}%
\pgfpathlineto{\pgfqpoint{2.830759in}{0.847683in}}%
\pgfpathlineto{\pgfqpoint{2.807480in}{0.800926in}}%
\pgfpathlineto{\pgfqpoint{2.785150in}{0.751398in}}%
\pgfpathlineto{\pgfqpoint{2.762112in}{0.694710in}}%
\pgfpathlineto{\pgfqpoint{2.752966in}{0.670413in}}%
\pgfpathlineto{\pgfqpoint{2.752966in}{0.670413in}}%
\pgfusepath{stroke}%
\end{pgfscope}%
\begin{pgfscope}%
\pgfpathrectangle{\pgfqpoint{2.762966in}{0.248000in}}{\pgfqpoint{2.246634in}{1.362506in}} %
\pgfusepath{clip}%
\pgfsetbuttcap%
\pgfsetroundjoin%
\pgfsetlinewidth{1.003750pt}%
\definecolor{currentstroke}{rgb}{0.000000,0.000000,0.000000}%
\pgfsetstrokecolor{currentstroke}%
\pgfsetdash{{1.000000pt}{1.650000pt}}{0.000000pt}%
\pgfpathmoveto{\pgfqpoint{0.000000in}{0.000000in}}%
\pgfusepath{stroke}%
\end{pgfscope}%
\begin{pgfscope}%
\pgfpathrectangle{\pgfqpoint{2.762966in}{0.248000in}}{\pgfqpoint{2.246634in}{1.362506in}} %
\pgfusepath{clip}%
\pgfsetbuttcap%
\pgfsetroundjoin%
\pgfsetlinewidth{1.003750pt}%
\definecolor{currentstroke}{rgb}{0.000000,0.000000,0.000000}%
\pgfsetstrokecolor{currentstroke}%
\pgfsetdash{{1.000000pt}{1.650000pt}}{0.000000pt}%
\pgfpathmoveto{\pgfqpoint{4.903287in}{0.238000in}}%
\pgfpathlineto{\pgfqpoint{4.937753in}{0.593812in}}%
\pgfpathlineto{\pgfqpoint{4.947122in}{0.709151in}}%
\pgfpathlineto{\pgfqpoint{4.951918in}{0.795018in}}%
\pgfpathlineto{\pgfqpoint{4.953458in}{0.863013in}}%
\pgfpathlineto{\pgfqpoint{4.952515in}{0.918677in}}%
\pgfpathlineto{\pgfqpoint{4.949425in}{0.967595in}}%
\pgfpathlineto{\pgfqpoint{4.944572in}{1.009212in}}%
\pgfpathlineto{\pgfqpoint{4.937501in}{1.048817in}}%
\pgfpathlineto{\pgfqpoint{4.929684in}{1.080487in}}%
\pgfpathlineto{\pgfqpoint{4.920398in}{1.109614in}}%
\pgfpathlineto{\pgfqpoint{4.909873in}{1.136029in}}%
\pgfpathlineto{\pgfqpoint{4.896366in}{1.163502in}}%
\pgfpathlineto{\pgfqpoint{4.881868in}{1.187676in}}%
\pgfpathlineto{\pgfqpoint{4.866658in}{1.208999in}}%
\pgfpathlineto{\pgfqpoint{4.848650in}{1.230396in}}%
\pgfpathlineto{\pgfqpoint{4.827834in}{1.251341in}}%
\pgfpathlineto{\pgfqpoint{4.806604in}{1.269567in}}%
\pgfpathlineto{\pgfqpoint{4.782683in}{1.287258in}}%
\pgfpathlineto{\pgfqpoint{4.756086in}{1.304177in}}%
\pgfpathlineto{\pgfqpoint{4.726835in}{1.320179in}}%
\pgfpathlineto{\pgfqpoint{4.694952in}{1.335169in}}%
\pgfpathlineto{\pgfqpoint{4.657990in}{1.350009in}}%
\pgfpathlineto{\pgfqpoint{4.618422in}{1.363485in}}%
\pgfpathlineto{\pgfqpoint{4.573787in}{1.376284in}}%
\pgfpathlineto{\pgfqpoint{4.524091in}{1.388110in}}%
\pgfpathlineto{\pgfqpoint{4.469341in}{1.398726in}}%
\pgfpathlineto{\pgfqpoint{4.409546in}{1.407931in}}%
\pgfpathlineto{\pgfqpoint{4.347211in}{1.415296in}}%
\pgfpathlineto{\pgfqpoint{4.279852in}{1.421033in}}%
\pgfpathlineto{\pgfqpoint{4.242418in}{1.423154in}}%
\pgfpathlineto{\pgfqpoint{4.167540in}{1.426088in}}%
\pgfpathlineto{\pgfqpoint{4.090157in}{1.426884in}}%
\pgfpathlineto{\pgfqpoint{4.012776in}{1.425503in}}%
\pgfpathlineto{\pgfqpoint{3.935405in}{1.421967in}}%
\pgfpathlineto{\pgfqpoint{3.848075in}{1.415603in}}%
\pgfpathlineto{\pgfqpoint{3.773258in}{1.407584in}}%
\pgfpathlineto{\pgfqpoint{3.700979in}{1.397603in}}%
\pgfpathlineto{\pgfqpoint{3.616305in}{1.383389in}}%
\pgfpathlineto{\pgfqpoint{3.554130in}{1.370090in}}%
\pgfpathlineto{\pgfqpoint{3.494524in}{1.355137in}}%
\pgfpathlineto{\pgfqpoint{3.437503in}{1.338523in}}%
\pgfpathlineto{\pgfqpoint{3.358330in}{1.312618in}}%
\pgfpathlineto{\pgfqpoint{3.289336in}{1.285093in}}%
\pgfpathlineto{\pgfqpoint{3.245227in}{1.263863in}}%
\pgfpathlineto{\pgfqpoint{3.201326in}{1.240116in}}%
\pgfpathlineto{\pgfqpoint{3.160109in}{1.215079in}}%
\pgfpathlineto{\pgfqpoint{3.121588in}{1.188911in}}%
\pgfpathlineto{\pgfqpoint{3.085765in}{1.161884in}}%
\pgfpathlineto{\pgfqpoint{3.052652in}{1.134170in}}%
\pgfpathlineto{\pgfqpoint{3.017551in}{1.101848in}}%
\pgfpathlineto{\pgfqpoint{2.987580in}{1.070989in}}%
\pgfpathlineto{\pgfqpoint{2.958072in}{1.037494in}}%
\pgfpathlineto{\pgfqpoint{2.929113in}{1.001173in}}%
\pgfpathlineto{\pgfqpoint{2.900795in}{0.961865in}}%
\pgfpathlineto{\pgfqpoint{2.879641in}{0.929041in}}%
\pgfpathlineto{\pgfqpoint{2.873697in}{0.917784in}}%
\pgfpathlineto{\pgfqpoint{2.849017in}{0.875677in}}%
\pgfpathlineto{\pgfqpoint{2.825150in}{0.830777in}}%
\pgfpathlineto{\pgfqpoint{2.802182in}{0.783081in}}%
\pgfpathlineto{\pgfqpoint{2.778404in}{0.728306in}}%
\pgfpathlineto{\pgfqpoint{2.755852in}{0.670425in}}%
\pgfpathlineto{\pgfqpoint{2.752966in}{0.662544in}}%
\pgfpathlineto{\pgfqpoint{2.752966in}{0.662544in}}%
\pgfusepath{stroke}%
\end{pgfscope}%
\begin{pgfscope}%
\pgfpathrectangle{\pgfqpoint{2.762966in}{0.248000in}}{\pgfqpoint{2.246634in}{1.362506in}} %
\pgfusepath{clip}%
\pgfsetbuttcap%
\pgfsetroundjoin%
\pgfsetlinewidth{1.003750pt}%
\definecolor{currentstroke}{rgb}{0.000000,0.000000,0.000000}%
\pgfsetstrokecolor{currentstroke}%
\pgfsetdash{{1.000000pt}{1.650000pt}}{0.000000pt}%
\pgfpathmoveto{\pgfqpoint{0.000000in}{0.000000in}}%
\pgfusepath{stroke}%
\end{pgfscope}%
\begin{pgfscope}%
\pgfsetrectcap%
\pgfsetmiterjoin%
\pgfsetlinewidth{0.803000pt}%
\definecolor{currentstroke}{rgb}{0.000000,0.000000,0.000000}%
\pgfsetstrokecolor{currentstroke}%
\pgfsetdash{}{0pt}%
\pgfpathmoveto{\pgfqpoint{2.762966in}{0.248000in}}%
\pgfpathlineto{\pgfqpoint{2.762966in}{1.610506in}}%
\pgfusepath{stroke}%
\end{pgfscope}%
\begin{pgfscope}%
\pgfsetrectcap%
\pgfsetmiterjoin%
\pgfsetlinewidth{0.803000pt}%
\definecolor{currentstroke}{rgb}{0.000000,0.000000,0.000000}%
\pgfsetstrokecolor{currentstroke}%
\pgfsetdash{}{0pt}%
\pgfpathmoveto{\pgfqpoint{5.009600in}{0.248000in}}%
\pgfpathlineto{\pgfqpoint{5.009600in}{1.610506in}}%
\pgfusepath{stroke}%
\end{pgfscope}%
\begin{pgfscope}%
\pgfsetrectcap%
\pgfsetmiterjoin%
\pgfsetlinewidth{0.803000pt}%
\definecolor{currentstroke}{rgb}{0.000000,0.000000,0.000000}%
\pgfsetstrokecolor{currentstroke}%
\pgfsetdash{}{0pt}%
\pgfpathmoveto{\pgfqpoint{2.762966in}{0.248000in}}%
\pgfpathlineto{\pgfqpoint{5.009600in}{0.248000in}}%
\pgfusepath{stroke}%
\end{pgfscope}%
\begin{pgfscope}%
\pgfsetrectcap%
\pgfsetmiterjoin%
\pgfsetlinewidth{0.803000pt}%
\definecolor{currentstroke}{rgb}{0.000000,0.000000,0.000000}%
\pgfsetstrokecolor{currentstroke}%
\pgfsetdash{}{0pt}%
\pgfpathmoveto{\pgfqpoint{2.762966in}{1.610506in}}%
\pgfpathlineto{\pgfqpoint{5.009600in}{1.610506in}}%
\pgfusepath{stroke}%
\end{pgfscope}%
\begin{pgfscope}%
\pgfsetbuttcap%
\pgfsetmiterjoin%
\definecolor{currentfill}{rgb}{1.000000,1.000000,1.000000}%
\pgfsetfillcolor{currentfill}%
\pgfsetfillopacity{0.800000}%
\pgfsetlinewidth{1.003750pt}%
\definecolor{currentstroke}{rgb}{0.800000,0.800000,0.800000}%
\pgfsetstrokecolor{currentstroke}%
\pgfsetstrokeopacity{0.800000}%
\pgfsetdash{}{0pt}%
\pgfpathmoveto{\pgfqpoint{3.486334in}{0.317444in}}%
\pgfpathlineto{\pgfqpoint{4.286232in}{0.317444in}}%
\pgfpathquadraticcurveto{\pgfqpoint{4.314010in}{0.317444in}}{\pgfqpoint{4.314010in}{0.345222in}}%
\pgfpathlineto{\pgfqpoint{4.314010in}{0.730222in}}%
\pgfpathquadraticcurveto{\pgfqpoint{4.314010in}{0.758000in}}{\pgfqpoint{4.286232in}{0.758000in}}%
\pgfpathlineto{\pgfqpoint{3.486334in}{0.758000in}}%
\pgfpathquadraticcurveto{\pgfqpoint{3.458556in}{0.758000in}}{\pgfqpoint{3.458556in}{0.730222in}}%
\pgfpathlineto{\pgfqpoint{3.458556in}{0.345222in}}%
\pgfpathquadraticcurveto{\pgfqpoint{3.458556in}{0.317444in}}{\pgfqpoint{3.486334in}{0.317444in}}%
\pgfpathclose%
\pgfusepath{stroke,fill}%
\end{pgfscope}%
\begin{pgfscope}%
\pgfsetrectcap%
\pgfsetroundjoin%
\pgfsetlinewidth{1.003750pt}%
\definecolor{currentstroke}{rgb}{1.000000,0.388235,0.278431}%
\pgfsetstrokecolor{currentstroke}%
\pgfsetdash{}{0pt}%
\pgfpathmoveto{\pgfqpoint{3.514112in}{0.653833in}}%
\pgfpathlineto{\pgfqpoint{3.583556in}{0.653833in}}%
\pgfusepath{stroke}%
\end{pgfscope}%
\begin{pgfscope}%
\pgftext[x=3.694667in,y=0.605222in,left,base]{\rmfamily\fontsize{10.000000}{12.000000}\selectfont \(\displaystyle \textnormal{tol}=10^{-8}\)}%
\end{pgfscope}%
\begin{pgfscope}%
\pgfsetbuttcap%
\pgfsetroundjoin%
\pgfsetlinewidth{1.003750pt}%
\definecolor{currentstroke}{rgb}{0.000000,0.000000,0.000000}%
\pgfsetstrokecolor{currentstroke}%
\pgfsetdash{{1.000000pt}{1.650000pt}}{0.000000pt}%
\pgfpathmoveto{\pgfqpoint{3.514112in}{0.454389in}}%
\pgfpathlineto{\pgfqpoint{3.583556in}{0.454389in}}%
\pgfusepath{stroke}%
\end{pgfscope}%
\begin{pgfscope}%
\pgftext[x=3.694667in,y=0.405778in,left,base]{\rmfamily\fontsize{10.000000}{12.000000}\selectfont Reference}%
\end{pgfscope}%
\begin{pgfscope}%
\pgfsetbuttcap%
\pgfsetmiterjoin%
\definecolor{currentfill}{rgb}{1.000000,1.000000,1.000000}%
\pgfsetfillcolor{currentfill}%
\pgfsetlinewidth{0.000000pt}%
\definecolor{currentstroke}{rgb}{0.000000,0.000000,0.000000}%
\pgfsetstrokecolor{currentstroke}%
\pgfsetstrokeopacity{0.000000}%
\pgfsetdash{}{0pt}%
\pgfpathmoveto{\pgfqpoint{0.404000in}{1.712694in}}%
\pgfpathlineto{\pgfqpoint{2.650634in}{1.712694in}}%
\pgfpathlineto{\pgfqpoint{2.650634in}{3.075200in}}%
\pgfpathlineto{\pgfqpoint{0.404000in}{3.075200in}}%
\pgfpathclose%
\pgfusepath{fill}%
\end{pgfscope}%
\begin{pgfscope}%
\pgfsetbuttcap%
\pgfsetroundjoin%
\definecolor{currentfill}{rgb}{0.000000,0.000000,0.000000}%
\pgfsetfillcolor{currentfill}%
\pgfsetlinewidth{0.803000pt}%
\definecolor{currentstroke}{rgb}{0.000000,0.000000,0.000000}%
\pgfsetstrokecolor{currentstroke}%
\pgfsetdash{}{0pt}%
\pgfsys@defobject{currentmarker}{\pgfqpoint{0.000000in}{-0.048611in}}{\pgfqpoint{0.000000in}{0.000000in}}{%
\pgfpathmoveto{\pgfqpoint{0.000000in}{0.000000in}}%
\pgfpathlineto{\pgfqpoint{0.000000in}{-0.048611in}}%
\pgfusepath{stroke,fill}%
}%
\begin{pgfscope}%
\pgfsys@transformshift{0.528813in}{1.712694in}%
\pgfsys@useobject{currentmarker}{}%
\end{pgfscope}%
\end{pgfscope}%
\begin{pgfscope}%
\pgfsetbuttcap%
\pgfsetroundjoin%
\definecolor{currentfill}{rgb}{0.000000,0.000000,0.000000}%
\pgfsetfillcolor{currentfill}%
\pgfsetlinewidth{0.803000pt}%
\definecolor{currentstroke}{rgb}{0.000000,0.000000,0.000000}%
\pgfsetstrokecolor{currentstroke}%
\pgfsetdash{}{0pt}%
\pgfsys@defobject{currentmarker}{\pgfqpoint{0.000000in}{-0.048611in}}{\pgfqpoint{0.000000in}{0.000000in}}{%
\pgfpathmoveto{\pgfqpoint{0.000000in}{0.000000in}}%
\pgfpathlineto{\pgfqpoint{0.000000in}{-0.048611in}}%
\pgfusepath{stroke,fill}%
}%
\begin{pgfscope}%
\pgfsys@transformshift{0.903252in}{1.712694in}%
\pgfsys@useobject{currentmarker}{}%
\end{pgfscope}%
\end{pgfscope}%
\begin{pgfscope}%
\pgfsetbuttcap%
\pgfsetroundjoin%
\definecolor{currentfill}{rgb}{0.000000,0.000000,0.000000}%
\pgfsetfillcolor{currentfill}%
\pgfsetlinewidth{0.803000pt}%
\definecolor{currentstroke}{rgb}{0.000000,0.000000,0.000000}%
\pgfsetstrokecolor{currentstroke}%
\pgfsetdash{}{0pt}%
\pgfsys@defobject{currentmarker}{\pgfqpoint{0.000000in}{-0.048611in}}{\pgfqpoint{0.000000in}{0.000000in}}{%
\pgfpathmoveto{\pgfqpoint{0.000000in}{0.000000in}}%
\pgfpathlineto{\pgfqpoint{0.000000in}{-0.048611in}}%
\pgfusepath{stroke,fill}%
}%
\begin{pgfscope}%
\pgfsys@transformshift{1.277691in}{1.712694in}%
\pgfsys@useobject{currentmarker}{}%
\end{pgfscope}%
\end{pgfscope}%
\begin{pgfscope}%
\pgfsetbuttcap%
\pgfsetroundjoin%
\definecolor{currentfill}{rgb}{0.000000,0.000000,0.000000}%
\pgfsetfillcolor{currentfill}%
\pgfsetlinewidth{0.803000pt}%
\definecolor{currentstroke}{rgb}{0.000000,0.000000,0.000000}%
\pgfsetstrokecolor{currentstroke}%
\pgfsetdash{}{0pt}%
\pgfsys@defobject{currentmarker}{\pgfqpoint{0.000000in}{-0.048611in}}{\pgfqpoint{0.000000in}{0.000000in}}{%
\pgfpathmoveto{\pgfqpoint{0.000000in}{0.000000in}}%
\pgfpathlineto{\pgfqpoint{0.000000in}{-0.048611in}}%
\pgfusepath{stroke,fill}%
}%
\begin{pgfscope}%
\pgfsys@transformshift{1.652130in}{1.712694in}%
\pgfsys@useobject{currentmarker}{}%
\end{pgfscope}%
\end{pgfscope}%
\begin{pgfscope}%
\pgfsetbuttcap%
\pgfsetroundjoin%
\definecolor{currentfill}{rgb}{0.000000,0.000000,0.000000}%
\pgfsetfillcolor{currentfill}%
\pgfsetlinewidth{0.803000pt}%
\definecolor{currentstroke}{rgb}{0.000000,0.000000,0.000000}%
\pgfsetstrokecolor{currentstroke}%
\pgfsetdash{}{0pt}%
\pgfsys@defobject{currentmarker}{\pgfqpoint{0.000000in}{-0.048611in}}{\pgfqpoint{0.000000in}{0.000000in}}{%
\pgfpathmoveto{\pgfqpoint{0.000000in}{0.000000in}}%
\pgfpathlineto{\pgfqpoint{0.000000in}{-0.048611in}}%
\pgfusepath{stroke,fill}%
}%
\begin{pgfscope}%
\pgfsys@transformshift{2.026569in}{1.712694in}%
\pgfsys@useobject{currentmarker}{}%
\end{pgfscope}%
\end{pgfscope}%
\begin{pgfscope}%
\pgfsetbuttcap%
\pgfsetroundjoin%
\definecolor{currentfill}{rgb}{0.000000,0.000000,0.000000}%
\pgfsetfillcolor{currentfill}%
\pgfsetlinewidth{0.803000pt}%
\definecolor{currentstroke}{rgb}{0.000000,0.000000,0.000000}%
\pgfsetstrokecolor{currentstroke}%
\pgfsetdash{}{0pt}%
\pgfsys@defobject{currentmarker}{\pgfqpoint{0.000000in}{-0.048611in}}{\pgfqpoint{0.000000in}{0.000000in}}{%
\pgfpathmoveto{\pgfqpoint{0.000000in}{0.000000in}}%
\pgfpathlineto{\pgfqpoint{0.000000in}{-0.048611in}}%
\pgfusepath{stroke,fill}%
}%
\begin{pgfscope}%
\pgfsys@transformshift{2.401008in}{1.712694in}%
\pgfsys@useobject{currentmarker}{}%
\end{pgfscope}%
\end{pgfscope}%
\begin{pgfscope}%
\pgfsetbuttcap%
\pgfsetroundjoin%
\definecolor{currentfill}{rgb}{0.000000,0.000000,0.000000}%
\pgfsetfillcolor{currentfill}%
\pgfsetlinewidth{0.803000pt}%
\definecolor{currentstroke}{rgb}{0.000000,0.000000,0.000000}%
\pgfsetstrokecolor{currentstroke}%
\pgfsetdash{}{0pt}%
\pgfsys@defobject{currentmarker}{\pgfqpoint{-0.048611in}{0.000000in}}{\pgfqpoint{0.000000in}{0.000000in}}{%
\pgfpathmoveto{\pgfqpoint{0.000000in}{0.000000in}}%
\pgfpathlineto{\pgfqpoint{-0.048611in}{0.000000in}}%
\pgfusepath{stroke,fill}%
}%
\begin{pgfscope}%
\pgfsys@transformshift{0.404000in}{1.960422in}%
\pgfsys@useobject{currentmarker}{}%
\end{pgfscope}%
\end{pgfscope}%
\begin{pgfscope}%
\pgftext[x=0.073723in,y=1.912471in,left,base]{\rmfamily\fontsize{10.000000}{12.000000}\selectfont \(\displaystyle 0.80\)}%
\end{pgfscope}%
\begin{pgfscope}%
\pgfsetbuttcap%
\pgfsetroundjoin%
\definecolor{currentfill}{rgb}{0.000000,0.000000,0.000000}%
\pgfsetfillcolor{currentfill}%
\pgfsetlinewidth{0.803000pt}%
\definecolor{currentstroke}{rgb}{0.000000,0.000000,0.000000}%
\pgfsetstrokecolor{currentstroke}%
\pgfsetdash{}{0pt}%
\pgfsys@defobject{currentmarker}{\pgfqpoint{-0.048611in}{0.000000in}}{\pgfqpoint{0.000000in}{0.000000in}}{%
\pgfpathmoveto{\pgfqpoint{0.000000in}{0.000000in}}%
\pgfpathlineto{\pgfqpoint{-0.048611in}{0.000000in}}%
\pgfusepath{stroke,fill}%
}%
\begin{pgfscope}%
\pgfsys@transformshift{0.404000in}{2.270083in}%
\pgfsys@useobject{currentmarker}{}%
\end{pgfscope}%
\end{pgfscope}%
\begin{pgfscope}%
\pgftext[x=0.073723in,y=2.222132in,left,base]{\rmfamily\fontsize{10.000000}{12.000000}\selectfont \(\displaystyle 0.85\)}%
\end{pgfscope}%
\begin{pgfscope}%
\pgfsetbuttcap%
\pgfsetroundjoin%
\definecolor{currentfill}{rgb}{0.000000,0.000000,0.000000}%
\pgfsetfillcolor{currentfill}%
\pgfsetlinewidth{0.803000pt}%
\definecolor{currentstroke}{rgb}{0.000000,0.000000,0.000000}%
\pgfsetstrokecolor{currentstroke}%
\pgfsetdash{}{0pt}%
\pgfsys@defobject{currentmarker}{\pgfqpoint{-0.048611in}{0.000000in}}{\pgfqpoint{0.000000in}{0.000000in}}{%
\pgfpathmoveto{\pgfqpoint{0.000000in}{0.000000in}}%
\pgfpathlineto{\pgfqpoint{-0.048611in}{0.000000in}}%
\pgfusepath{stroke,fill}%
}%
\begin{pgfscope}%
\pgfsys@transformshift{0.404000in}{2.579743in}%
\pgfsys@useobject{currentmarker}{}%
\end{pgfscope}%
\end{pgfscope}%
\begin{pgfscope}%
\pgftext[x=0.073723in,y=2.531792in,left,base]{\rmfamily\fontsize{10.000000}{12.000000}\selectfont \(\displaystyle 0.90\)}%
\end{pgfscope}%
\begin{pgfscope}%
\pgfsetbuttcap%
\pgfsetroundjoin%
\definecolor{currentfill}{rgb}{0.000000,0.000000,0.000000}%
\pgfsetfillcolor{currentfill}%
\pgfsetlinewidth{0.803000pt}%
\definecolor{currentstroke}{rgb}{0.000000,0.000000,0.000000}%
\pgfsetstrokecolor{currentstroke}%
\pgfsetdash{}{0pt}%
\pgfsys@defobject{currentmarker}{\pgfqpoint{-0.048611in}{0.000000in}}{\pgfqpoint{0.000000in}{0.000000in}}{%
\pgfpathmoveto{\pgfqpoint{0.000000in}{0.000000in}}%
\pgfpathlineto{\pgfqpoint{-0.048611in}{0.000000in}}%
\pgfusepath{stroke,fill}%
}%
\begin{pgfscope}%
\pgfsys@transformshift{0.404000in}{2.889404in}%
\pgfsys@useobject{currentmarker}{}%
\end{pgfscope}%
\end{pgfscope}%
\begin{pgfscope}%
\pgftext[x=0.073723in,y=2.841452in,left,base]{\rmfamily\fontsize{10.000000}{12.000000}\selectfont \(\displaystyle 0.95\)}%
\end{pgfscope}%
\begin{pgfscope}%
\pgfpathrectangle{\pgfqpoint{0.404000in}{1.712694in}}{\pgfqpoint{2.246634in}{1.362506in}} %
\pgfusepath{clip}%
\pgfsetrectcap%
\pgfsetroundjoin%
\pgfsetlinewidth{1.505625pt}%
\definecolor{currentstroke}{rgb}{0.121569,0.466667,0.705882}%
\pgfsetstrokecolor{currentstroke}%
\pgfsetdash{}{0pt}%
\pgfusepath{stroke}%
\end{pgfscope}%
\begin{pgfscope}%
\pgfpathrectangle{\pgfqpoint{0.404000in}{1.712694in}}{\pgfqpoint{2.246634in}{1.362506in}} %
\pgfusepath{clip}%
\pgfsetrectcap%
\pgfsetroundjoin%
\pgfsetlinewidth{1.505625pt}%
\definecolor{currentstroke}{rgb}{1.000000,0.498039,0.054902}%
\pgfsetstrokecolor{currentstroke}%
\pgfsetdash{}{0pt}%
\pgfusepath{stroke}%
\end{pgfscope}%
\begin{pgfscope}%
\pgfpathrectangle{\pgfqpoint{0.404000in}{1.712694in}}{\pgfqpoint{2.246634in}{1.362506in}} %
\pgfusepath{clip}%
\pgfsetrectcap%
\pgfsetroundjoin%
\pgfsetlinewidth{1.505625pt}%
\definecolor{currentstroke}{rgb}{0.172549,0.627451,0.172549}%
\pgfsetstrokecolor{currentstroke}%
\pgfsetdash{}{0pt}%
\pgfusepath{stroke}%
\end{pgfscope}%
\begin{pgfscope}%
\pgfpathrectangle{\pgfqpoint{0.404000in}{1.712694in}}{\pgfqpoint{2.246634in}{1.362506in}} %
\pgfusepath{clip}%
\pgfsetrectcap%
\pgfsetroundjoin%
\pgfsetlinewidth{1.505625pt}%
\definecolor{currentstroke}{rgb}{0.839216,0.152941,0.156863}%
\pgfsetstrokecolor{currentstroke}%
\pgfsetdash{}{0pt}%
\pgfusepath{stroke}%
\end{pgfscope}%
\begin{pgfscope}%
\pgfpathrectangle{\pgfqpoint{0.404000in}{1.712694in}}{\pgfqpoint{2.246634in}{1.362506in}} %
\pgfusepath{clip}%
\pgfsetrectcap%
\pgfsetroundjoin%
\pgfsetlinewidth{1.505625pt}%
\definecolor{currentstroke}{rgb}{0.580392,0.403922,0.741176}%
\pgfsetstrokecolor{currentstroke}%
\pgfsetdash{}{0pt}%
\pgfusepath{stroke}%
\end{pgfscope}%
\begin{pgfscope}%
\pgfpathrectangle{\pgfqpoint{0.404000in}{1.712694in}}{\pgfqpoint{2.246634in}{1.362506in}} %
\pgfusepath{clip}%
\pgfsetrectcap%
\pgfsetroundjoin%
\pgfsetlinewidth{1.003750pt}%
\definecolor{currentstroke}{rgb}{1.000000,0.388235,0.278431}%
\pgfsetstrokecolor{currentstroke}%
\pgfsetdash{}{0pt}%
\pgfpathmoveto{\pgfqpoint{0.558748in}{2.403675in}}%
\pgfpathlineto{\pgfqpoint{0.586949in}{2.443497in}}%
\pgfpathlineto{\pgfqpoint{0.615805in}{2.480318in}}%
\pgfpathlineto{\pgfqpoint{0.645222in}{2.514301in}}%
\pgfpathlineto{\pgfqpoint{0.675115in}{2.545624in}}%
\pgfpathlineto{\pgfqpoint{0.707755in}{2.576594in}}%
\pgfpathlineto{\pgfqpoint{0.740783in}{2.604926in}}%
\pgfpathlineto{\pgfqpoint{0.776525in}{2.632606in}}%
\pgfpathlineto{\pgfqpoint{0.812568in}{2.657771in}}%
\pgfpathlineto{\pgfqpoint{0.851283in}{2.682111in}}%
\pgfpathlineto{\pgfqpoint{0.892665in}{2.705427in}}%
\pgfpathlineto{\pgfqpoint{0.936702in}{2.727562in}}%
\pgfpathlineto{\pgfqpoint{0.983380in}{2.748395in}}%
\pgfpathlineto{\pgfqpoint{1.032686in}{2.767834in}}%
\pgfpathlineto{\pgfqpoint{1.084604in}{2.785813in}}%
\pgfpathlineto{\pgfqpoint{1.139118in}{2.802288in}}%
\pgfpathlineto{\pgfqpoint{1.196215in}{2.817225in}}%
\pgfpathlineto{\pgfqpoint{1.258369in}{2.831110in}}%
\pgfpathlineto{\pgfqpoint{1.323087in}{2.843238in}}%
\pgfpathlineto{\pgfqpoint{1.390356in}{2.853593in}}%
\pgfpathlineto{\pgfqpoint{1.460166in}{2.862144in}}%
\pgfpathlineto{\pgfqpoint{1.535002in}{2.869032in}}%
\pgfpathlineto{\pgfqpoint{1.609866in}{2.873697in}}%
\pgfpathlineto{\pgfqpoint{1.687242in}{2.876267in}}%
\pgfpathlineto{\pgfqpoint{1.764626in}{2.876559in}}%
\pgfpathlineto{\pgfqpoint{1.839509in}{2.874589in}}%
\pgfpathlineto{\pgfqpoint{1.911881in}{2.870426in}}%
\pgfpathlineto{\pgfqpoint{1.979235in}{2.864350in}}%
\pgfpathlineto{\pgfqpoint{2.041561in}{2.856561in}}%
\pgfpathlineto{\pgfqpoint{2.098852in}{2.847264in}}%
\pgfpathlineto{\pgfqpoint{2.151099in}{2.836662in}}%
\pgfpathlineto{\pgfqpoint{2.200774in}{2.824311in}}%
\pgfpathlineto{\pgfqpoint{2.245380in}{2.810909in}}%
\pgfpathlineto{\pgfqpoint{2.284911in}{2.796770in}}%
\pgfpathlineto{\pgfqpoint{2.321822in}{2.781167in}}%
\pgfpathlineto{\pgfqpoint{2.353642in}{2.765379in}}%
\pgfpathlineto{\pgfqpoint{2.382812in}{2.748496in}}%
\pgfpathlineto{\pgfqpoint{2.409307in}{2.730615in}}%
\pgfpathlineto{\pgfqpoint{2.433100in}{2.711896in}}%
\pgfpathlineto{\pgfqpoint{2.454171in}{2.692581in}}%
\pgfpathlineto{\pgfqpoint{2.472517in}{2.673023in}}%
\pgfpathlineto{\pgfqpoint{2.490351in}{2.650749in}}%
\pgfpathlineto{\pgfqpoint{2.505371in}{2.628612in}}%
\pgfpathlineto{\pgfqpoint{2.519635in}{2.603595in}}%
\pgfpathlineto{\pgfqpoint{2.532874in}{2.575330in}}%
\pgfpathlineto{\pgfqpoint{2.543154in}{2.548329in}}%
\pgfpathlineto{\pgfqpoint{2.552215in}{2.518764in}}%
\pgfpathlineto{\pgfqpoint{2.559861in}{2.486836in}}%
\pgfpathlineto{\pgfqpoint{2.566850in}{2.447137in}}%
\pgfpathlineto{\pgfqpoint{2.571775in}{2.405568in}}%
\pgfpathlineto{\pgfqpoint{2.575123in}{2.356751in}}%
\pgfpathlineto{\pgfqpoint{2.576499in}{2.301140in}}%
\pgfpathlineto{\pgfqpoint{2.575786in}{2.239250in}}%
\pgfpathlineto{\pgfqpoint{2.572428in}{2.159186in}}%
\pgfpathlineto{\pgfqpoint{2.565951in}{2.061417in}}%
\pgfpathlineto{\pgfqpoint{2.554317in}{1.921934in}}%
\pgfpathlineto{\pgfqpoint{2.534230in}{1.702694in}}%
\pgfpathlineto{\pgfqpoint{2.534230in}{1.702694in}}%
\pgfusepath{stroke}%
\end{pgfscope}%
\begin{pgfscope}%
\pgfpathrectangle{\pgfqpoint{0.404000in}{1.712694in}}{\pgfqpoint{2.246634in}{1.362506in}} %
\pgfusepath{clip}%
\pgfsetrectcap%
\pgfsetroundjoin%
\pgfsetlinewidth{1.003750pt}%
\definecolor{currentstroke}{rgb}{1.000000,0.388235,0.278431}%
\pgfsetstrokecolor{currentstroke}%
\pgfsetdash{}{0pt}%
\pgfpathmoveto{\pgfqpoint{2.350898in}{1.702694in}}%
\pgfpathlineto{\pgfqpoint{2.346074in}{1.796416in}}%
\pgfpathlineto{\pgfqpoint{2.339588in}{1.881599in}}%
\pgfpathlineto{\pgfqpoint{2.332042in}{1.953505in}}%
\pgfpathlineto{\pgfqpoint{2.323181in}{2.017968in}}%
\pgfpathlineto{\pgfqpoint{2.313275in}{2.074796in}}%
\pgfpathlineto{\pgfqpoint{2.301449in}{2.129313in}}%
\pgfpathlineto{\pgfqpoint{2.289107in}{2.175868in}}%
\pgfpathlineto{\pgfqpoint{2.275187in}{2.219596in}}%
\pgfpathlineto{\pgfqpoint{2.259795in}{2.260174in}}%
\pgfpathlineto{\pgfqpoint{2.243093in}{2.297431in}}%
\pgfpathlineto{\pgfqpoint{2.225273in}{2.331355in}}%
\pgfpathlineto{\pgfqpoint{2.206529in}{2.362063in}}%
\pgfpathlineto{\pgfqpoint{2.187038in}{2.389766in}}%
\pgfpathlineto{\pgfqpoint{2.166951in}{2.414715in}}%
\pgfpathlineto{\pgfqpoint{2.144083in}{2.439532in}}%
\pgfpathlineto{\pgfqpoint{2.120765in}{2.461629in}}%
\pgfpathlineto{\pgfqpoint{2.094719in}{2.483182in}}%
\pgfpathlineto{\pgfqpoint{2.068354in}{2.502208in}}%
\pgfpathlineto{\pgfqpoint{2.039318in}{2.520457in}}%
\pgfpathlineto{\pgfqpoint{2.007617in}{2.537664in}}%
\pgfpathlineto{\pgfqpoint{1.973267in}{2.553612in}}%
\pgfpathlineto{\pgfqpoint{1.938755in}{2.567234in}}%
\pgfpathlineto{\pgfqpoint{1.901643in}{2.579575in}}%
\pgfpathlineto{\pgfqpoint{1.861947in}{2.590478in}}%
\pgfpathlineto{\pgfqpoint{1.819678in}{2.599797in}}%
\pgfpathlineto{\pgfqpoint{1.774850in}{2.607390in}}%
\pgfpathlineto{\pgfqpoint{1.727478in}{2.613114in}}%
\pgfpathlineto{\pgfqpoint{1.677576in}{2.616802in}}%
\pgfpathlineto{\pgfqpoint{1.627654in}{2.618252in}}%
\pgfpathlineto{\pgfqpoint{1.575235in}{2.617485in}}%
\pgfpathlineto{\pgfqpoint{1.522828in}{2.614437in}}%
\pgfpathlineto{\pgfqpoint{1.470451in}{2.609135in}}%
\pgfpathlineto{\pgfqpoint{1.418120in}{2.601540in}}%
\pgfpathlineto{\pgfqpoint{1.368340in}{2.592120in}}%
\pgfpathlineto{\pgfqpoint{1.318637in}{2.580463in}}%
\pgfpathlineto{\pgfqpoint{1.271511in}{2.567199in}}%
\pgfpathlineto{\pgfqpoint{1.224499in}{2.551650in}}%
\pgfpathlineto{\pgfqpoint{1.180092in}{2.534651in}}%
\pgfpathlineto{\pgfqpoint{1.135846in}{2.515254in}}%
\pgfpathlineto{\pgfqpoint{1.094240in}{2.494540in}}%
\pgfpathlineto{\pgfqpoint{1.055281in}{2.472723in}}%
\pgfpathlineto{\pgfqpoint{1.016559in}{2.448451in}}%
\pgfpathlineto{\pgfqpoint{0.980518in}{2.423264in}}%
\pgfpathlineto{\pgfqpoint{0.944785in}{2.395514in}}%
\pgfpathlineto{\pgfqpoint{0.911769in}{2.367104in}}%
\pgfpathlineto{\pgfqpoint{0.879138in}{2.336071in}}%
\pgfpathlineto{\pgfqpoint{0.849248in}{2.304727in}}%
\pgfpathlineto{\pgfqpoint{0.819824in}{2.270781in}}%
\pgfpathlineto{\pgfqpoint{0.790945in}{2.234071in}}%
\pgfpathlineto{\pgfqpoint{0.762700in}{2.194442in}}%
\pgfpathlineto{\pgfqpoint{0.735187in}{2.151760in}}%
\pgfpathlineto{\pgfqpoint{0.710534in}{2.109556in}}%
\pgfpathlineto{\pgfqpoint{0.686685in}{2.064598in}}%
\pgfpathlineto{\pgfqpoint{0.663730in}{2.016862in}}%
\pgfpathlineto{\pgfqpoint{0.641758in}{1.966359in}}%
\pgfpathlineto{\pgfqpoint{0.619161in}{1.908589in}}%
\pgfpathlineto{\pgfqpoint{0.597909in}{1.847758in}}%
\pgfpathlineto{\pgfqpoint{0.578081in}{1.784038in}}%
\pgfpathlineto{\pgfqpoint{0.559733in}{1.717643in}}%
\pgfpathlineto{\pgfqpoint{0.555898in}{1.702694in}}%
\pgfpathmoveto{\pgfqpoint{0.430052in}{1.702694in}}%
\pgfpathlineto{\pgfqpoint{0.446813in}{1.777627in}}%
\pgfpathlineto{\pgfqpoint{0.465425in}{1.850999in}}%
\pgfpathlineto{\pgfqpoint{0.485703in}{1.921602in}}%
\pgfpathlineto{\pgfqpoint{0.505999in}{1.984410in}}%
\pgfpathlineto{\pgfqpoint{0.527662in}{2.044343in}}%
\pgfpathlineto{\pgfqpoint{0.550623in}{2.101222in}}%
\pgfpathlineto{\pgfqpoint{0.574796in}{2.154924in}}%
\pgfpathlineto{\pgfqpoint{0.600080in}{2.205378in}}%
\pgfpathlineto{\pgfqpoint{0.626366in}{2.252577in}}%
\pgfpathlineto{\pgfqpoint{0.653542in}{2.296565in}}%
\pgfpathlineto{\pgfqpoint{0.681499in}{2.337434in}}%
\pgfpathlineto{\pgfqpoint{0.710132in}{2.375310in}}%
\pgfpathlineto{\pgfqpoint{0.739348in}{2.410346in}}%
\pgfpathlineto{\pgfqpoint{0.769060in}{2.442709in}}%
\pgfpathlineto{\pgfqpoint{0.801528in}{2.474773in}}%
\pgfpathlineto{\pgfqpoint{0.834406in}{2.504158in}}%
\pgfpathlineto{\pgfqpoint{0.870010in}{2.532910in}}%
\pgfpathlineto{\pgfqpoint{0.905937in}{2.559077in}}%
\pgfpathlineto{\pgfqpoint{0.944550in}{2.584395in}}%
\pgfpathlineto{\pgfqpoint{0.983409in}{2.607285in}}%
\pgfpathlineto{\pgfqpoint{1.024915in}{2.629198in}}%
\pgfpathlineto{\pgfqpoint{1.069061in}{2.649957in}}%
\pgfpathlineto{\pgfqpoint{1.115836in}{2.669416in}}%
\pgfpathlineto{\pgfqpoint{1.165229in}{2.687448in}}%
\pgfpathlineto{\pgfqpoint{1.217226in}{2.703947in}}%
\pgfpathlineto{\pgfqpoint{1.271815in}{2.718816in}}%
\pgfpathlineto{\pgfqpoint{1.326496in}{2.731439in}}%
\pgfpathlineto{\pgfqpoint{1.383738in}{2.742434in}}%
\pgfpathlineto{\pgfqpoint{1.443532in}{2.751684in}}%
\pgfpathlineto{\pgfqpoint{1.505867in}{2.759053in}}%
\pgfpathlineto{\pgfqpoint{1.568238in}{2.764209in}}%
\pgfpathlineto{\pgfqpoint{1.630632in}{2.767204in}}%
\pgfpathlineto{\pgfqpoint{1.693038in}{2.768038in}}%
\pgfpathlineto{\pgfqpoint{1.755441in}{2.766650in}}%
\pgfpathlineto{\pgfqpoint{1.815334in}{2.763109in}}%
\pgfpathlineto{\pgfqpoint{1.872703in}{2.757516in}}%
\pgfpathlineto{\pgfqpoint{1.927535in}{2.749929in}}%
\pgfpathlineto{\pgfqpoint{1.977326in}{2.740884in}}%
\pgfpathlineto{\pgfqpoint{2.024555in}{2.730111in}}%
\pgfpathlineto{\pgfqpoint{2.069204in}{2.717622in}}%
\pgfpathlineto{\pgfqpoint{2.108782in}{2.704325in}}%
\pgfpathlineto{\pgfqpoint{2.145754in}{2.689636in}}%
\pgfpathlineto{\pgfqpoint{2.180099in}{2.673618in}}%
\pgfpathlineto{\pgfqpoint{2.211794in}{2.656357in}}%
\pgfpathlineto{\pgfqpoint{2.240818in}{2.637983in}}%
\pgfpathlineto{\pgfqpoint{2.267149in}{2.618675in}}%
\pgfpathlineto{\pgfqpoint{2.290774in}{2.598688in}}%
\pgfpathlineto{\pgfqpoint{2.313989in}{2.575948in}}%
\pgfpathlineto{\pgfqpoint{2.334407in}{2.552709in}}%
\pgfpathlineto{\pgfqpoint{2.352045in}{2.529489in}}%
\pgfpathlineto{\pgfqpoint{2.369047in}{2.503518in}}%
\pgfpathlineto{\pgfqpoint{2.385220in}{2.474479in}}%
\pgfpathlineto{\pgfqpoint{2.398510in}{2.446348in}}%
\pgfpathlineto{\pgfqpoint{2.410811in}{2.415578in}}%
\pgfpathlineto{\pgfqpoint{2.421952in}{2.382199in}}%
\pgfpathlineto{\pgfqpoint{2.431788in}{2.346386in}}%
\pgfpathlineto{\pgfqpoint{2.441323in}{2.302879in}}%
\pgfpathlineto{\pgfqpoint{2.449030in}{2.257192in}}%
\pgfpathlineto{\pgfqpoint{2.455648in}{2.203949in}}%
\pgfpathlineto{\pgfqpoint{2.460771in}{2.143356in}}%
\pgfpathlineto{\pgfqpoint{2.464187in}{2.075776in}}%
\pgfpathlineto{\pgfqpoint{2.465933in}{1.995396in}}%
\pgfpathlineto{\pgfqpoint{2.465659in}{1.896319in}}%
\pgfpathlineto{\pgfqpoint{2.462883in}{1.772656in}}%
\pgfpathlineto{\pgfqpoint{2.460542in}{1.702694in}}%
\pgfpathlineto{\pgfqpoint{2.460542in}{1.702694in}}%
\pgfusepath{stroke}%
\end{pgfscope}%
\begin{pgfscope}%
\pgfpathrectangle{\pgfqpoint{0.404000in}{1.712694in}}{\pgfqpoint{2.246634in}{1.362506in}} %
\pgfusepath{clip}%
\pgfsetrectcap%
\pgfsetroundjoin%
\pgfsetlinewidth{1.003750pt}%
\definecolor{currentstroke}{rgb}{1.000000,0.388235,0.278431}%
\pgfsetstrokecolor{currentstroke}%
\pgfsetdash{}{0pt}%
\pgfpathmoveto{\pgfqpoint{2.545393in}{1.702694in}}%
\pgfpathlineto{\pgfqpoint{2.577696in}{2.034048in}}%
\pgfpathlineto{\pgfqpoint{2.588604in}{2.161246in}}%
\pgfpathlineto{\pgfqpoint{2.593910in}{2.246931in}}%
\pgfpathlineto{\pgfqpoint{2.596060in}{2.314828in}}%
\pgfpathlineto{\pgfqpoint{2.595803in}{2.370541in}}%
\pgfpathlineto{\pgfqpoint{2.593468in}{2.419720in}}%
\pgfpathlineto{\pgfqpoint{2.589383in}{2.461843in}}%
\pgfpathlineto{\pgfqpoint{2.583114in}{2.502272in}}%
\pgfpathlineto{\pgfqpoint{2.575948in}{2.534875in}}%
\pgfpathlineto{\pgfqpoint{2.567229in}{2.565062in}}%
\pgfpathlineto{\pgfqpoint{2.557168in}{2.592563in}}%
\pgfpathlineto{\pgfqpoint{2.546017in}{2.617350in}}%
\pgfpathlineto{\pgfqpoint{2.531953in}{2.643049in}}%
\pgfpathlineto{\pgfqpoint{2.517043in}{2.665638in}}%
\pgfpathlineto{\pgfqpoint{2.499283in}{2.688269in}}%
\pgfpathlineto{\pgfqpoint{2.480975in}{2.708041in}}%
\pgfpathlineto{\pgfqpoint{2.459919in}{2.727461in}}%
\pgfpathlineto{\pgfqpoint{2.436131in}{2.746221in}}%
\pgfpathlineto{\pgfqpoint{2.409636in}{2.764096in}}%
\pgfpathlineto{\pgfqpoint{2.380461in}{2.780926in}}%
\pgfpathlineto{\pgfqpoint{2.348634in}{2.796633in}}%
\pgfpathlineto{\pgfqpoint{2.314184in}{2.811193in}}%
\pgfpathlineto{\pgfqpoint{2.274657in}{2.825401in}}%
\pgfpathlineto{\pgfqpoint{2.232535in}{2.838180in}}%
\pgfpathlineto{\pgfqpoint{2.185353in}{2.850155in}}%
\pgfpathlineto{\pgfqpoint{2.133117in}{2.861065in}}%
\pgfpathlineto{\pgfqpoint{2.075835in}{2.870697in}}%
\pgfpathlineto{\pgfqpoint{2.013516in}{2.878865in}}%
\pgfpathlineto{\pgfqpoint{1.946169in}{2.885397in}}%
\pgfpathlineto{\pgfqpoint{1.886279in}{2.889157in}}%
\pgfpathlineto{\pgfqpoint{1.811401in}{2.892137in}}%
\pgfpathlineto{\pgfqpoint{1.734019in}{2.892990in}}%
\pgfpathlineto{\pgfqpoint{1.654141in}{2.891599in}}%
\pgfpathlineto{\pgfqpoint{1.574275in}{2.887925in}}%
\pgfpathlineto{\pgfqpoint{1.486945in}{2.881546in}}%
\pgfpathlineto{\pgfqpoint{1.412128in}{2.873489in}}%
\pgfpathlineto{\pgfqpoint{1.339850in}{2.863478in}}%
\pgfpathlineto{\pgfqpoint{1.282555in}{2.854526in}}%
\pgfpathlineto{\pgfqpoint{1.217862in}{2.841614in}}%
\pgfpathlineto{\pgfqpoint{1.155737in}{2.826935in}}%
\pgfpathlineto{\pgfqpoint{1.096196in}{2.810477in}}%
\pgfpathlineto{\pgfqpoint{1.041728in}{2.793087in}}%
\pgfpathlineto{\pgfqpoint{1.029335in}{2.789637in}}%
\pgfpathlineto{\pgfqpoint{1.021888in}{2.787876in}}%
\pgfpathlineto{\pgfqpoint{0.972543in}{2.769063in}}%
\pgfpathlineto{\pgfqpoint{0.908582in}{2.741994in}}%
\pgfpathlineto{\pgfqpoint{0.864559in}{2.719685in}}%
\pgfpathlineto{\pgfqpoint{0.823195in}{2.696181in}}%
\pgfpathlineto{\pgfqpoint{0.784501in}{2.671635in}}%
\pgfpathlineto{\pgfqpoint{0.748485in}{2.646233in}}%
\pgfpathlineto{\pgfqpoint{0.715142in}{2.620272in}}%
\pgfpathlineto{\pgfqpoint{0.679753in}{2.589925in}}%
\pgfpathlineto{\pgfqpoint{0.647161in}{2.558642in}}%
\pgfpathlineto{\pgfqpoint{0.617325in}{2.526989in}}%
\pgfpathlineto{\pgfqpoint{0.587978in}{2.492637in}}%
\pgfpathlineto{\pgfqpoint{0.559207in}{2.455409in}}%
\pgfpathlineto{\pgfqpoint{0.531108in}{2.415146in}}%
\pgfpathlineto{\pgfqpoint{0.518597in}{2.394779in}}%
\pgfpathlineto{\pgfqpoint{0.510398in}{2.380672in}}%
\pgfpathlineto{\pgfqpoint{0.485779in}{2.338342in}}%
\pgfpathlineto{\pgfqpoint{0.461982in}{2.293216in}}%
\pgfpathlineto{\pgfqpoint{0.439089in}{2.245297in}}%
\pgfpathlineto{\pgfqpoint{0.415397in}{2.190293in}}%
\pgfpathlineto{\pgfqpoint{0.394000in}{2.135088in}}%
\pgfpathlineto{\pgfqpoint{0.394000in}{2.135088in}}%
\pgfusepath{stroke}%
\end{pgfscope}%
\begin{pgfscope}%
\pgfpathrectangle{\pgfqpoint{0.404000in}{1.712694in}}{\pgfqpoint{2.246634in}{1.362506in}} %
\pgfusepath{clip}%
\pgfsetbuttcap%
\pgfsetroundjoin%
\pgfsetlinewidth{1.003750pt}%
\definecolor{currentstroke}{rgb}{0.000000,0.000000,0.000000}%
\pgfsetstrokecolor{currentstroke}%
\pgfsetdash{{1.000000pt}{1.650000pt}}{0.000000pt}%
\pgfpathmoveto{\pgfqpoint{0.558748in}{2.403675in}}%
\pgfpathlineto{\pgfqpoint{0.586949in}{2.443498in}}%
\pgfpathlineto{\pgfqpoint{0.615805in}{2.480319in}}%
\pgfpathlineto{\pgfqpoint{0.645222in}{2.514301in}}%
\pgfpathlineto{\pgfqpoint{0.675115in}{2.545624in}}%
\pgfpathlineto{\pgfqpoint{0.707755in}{2.576594in}}%
\pgfpathlineto{\pgfqpoint{0.740783in}{2.604926in}}%
\pgfpathlineto{\pgfqpoint{0.776525in}{2.632606in}}%
\pgfpathlineto{\pgfqpoint{0.812568in}{2.657772in}}%
\pgfpathlineto{\pgfqpoint{0.851283in}{2.682111in}}%
\pgfpathlineto{\pgfqpoint{0.892665in}{2.705428in}}%
\pgfpathlineto{\pgfqpoint{0.936702in}{2.727563in}}%
\pgfpathlineto{\pgfqpoint{0.983381in}{2.748395in}}%
\pgfpathlineto{\pgfqpoint{1.032686in}{2.767834in}}%
\pgfpathlineto{\pgfqpoint{1.084604in}{2.785813in}}%
\pgfpathlineto{\pgfqpoint{1.139118in}{2.802288in}}%
\pgfpathlineto{\pgfqpoint{1.196215in}{2.817225in}}%
\pgfpathlineto{\pgfqpoint{1.258369in}{2.831110in}}%
\pgfpathlineto{\pgfqpoint{1.323087in}{2.843238in}}%
\pgfpathlineto{\pgfqpoint{1.390356in}{2.853593in}}%
\pgfpathlineto{\pgfqpoint{1.460166in}{2.862144in}}%
\pgfpathlineto{\pgfqpoint{1.535002in}{2.869032in}}%
\pgfpathlineto{\pgfqpoint{1.609866in}{2.873697in}}%
\pgfpathlineto{\pgfqpoint{1.687242in}{2.876267in}}%
\pgfpathlineto{\pgfqpoint{1.764626in}{2.876559in}}%
\pgfpathlineto{\pgfqpoint{1.839509in}{2.874589in}}%
\pgfpathlineto{\pgfqpoint{1.911881in}{2.870426in}}%
\pgfpathlineto{\pgfqpoint{1.979235in}{2.864350in}}%
\pgfpathlineto{\pgfqpoint{2.041562in}{2.856561in}}%
\pgfpathlineto{\pgfqpoint{2.098853in}{2.847264in}}%
\pgfpathlineto{\pgfqpoint{2.151099in}{2.836662in}}%
\pgfpathlineto{\pgfqpoint{2.200775in}{2.824311in}}%
\pgfpathlineto{\pgfqpoint{2.245381in}{2.810910in}}%
\pgfpathlineto{\pgfqpoint{2.284911in}{2.796770in}}%
\pgfpathlineto{\pgfqpoint{2.321822in}{2.781167in}}%
\pgfpathlineto{\pgfqpoint{2.353642in}{2.765379in}}%
\pgfpathlineto{\pgfqpoint{2.382812in}{2.748496in}}%
\pgfpathlineto{\pgfqpoint{2.409307in}{2.730615in}}%
\pgfpathlineto{\pgfqpoint{2.433100in}{2.711896in}}%
\pgfpathlineto{\pgfqpoint{2.454172in}{2.692581in}}%
\pgfpathlineto{\pgfqpoint{2.472517in}{2.673023in}}%
\pgfpathlineto{\pgfqpoint{2.490351in}{2.650749in}}%
\pgfpathlineto{\pgfqpoint{2.505371in}{2.628612in}}%
\pgfpathlineto{\pgfqpoint{2.519635in}{2.603595in}}%
\pgfpathlineto{\pgfqpoint{2.532874in}{2.575330in}}%
\pgfpathlineto{\pgfqpoint{2.543155in}{2.548329in}}%
\pgfpathlineto{\pgfqpoint{2.552215in}{2.518764in}}%
\pgfpathlineto{\pgfqpoint{2.559861in}{2.486836in}}%
\pgfpathlineto{\pgfqpoint{2.566850in}{2.447137in}}%
\pgfpathlineto{\pgfqpoint{2.571775in}{2.405568in}}%
\pgfpathlineto{\pgfqpoint{2.575123in}{2.356751in}}%
\pgfpathlineto{\pgfqpoint{2.576499in}{2.301140in}}%
\pgfpathlineto{\pgfqpoint{2.575786in}{2.239250in}}%
\pgfpathlineto{\pgfqpoint{2.572428in}{2.159186in}}%
\pgfpathlineto{\pgfqpoint{2.565951in}{2.061417in}}%
\pgfpathlineto{\pgfqpoint{2.554317in}{1.921934in}}%
\pgfpathlineto{\pgfqpoint{2.534230in}{1.702694in}}%
\pgfpathlineto{\pgfqpoint{2.534230in}{1.702694in}}%
\pgfusepath{stroke}%
\end{pgfscope}%
\begin{pgfscope}%
\pgfpathrectangle{\pgfqpoint{0.404000in}{1.712694in}}{\pgfqpoint{2.246634in}{1.362506in}} %
\pgfusepath{clip}%
\pgfsetbuttcap%
\pgfsetroundjoin%
\pgfsetlinewidth{1.003750pt}%
\definecolor{currentstroke}{rgb}{0.000000,0.000000,0.000000}%
\pgfsetstrokecolor{currentstroke}%
\pgfsetdash{{1.000000pt}{1.650000pt}}{0.000000pt}%
\pgfpathmoveto{\pgfqpoint{0.000000in}{0.000000in}}%
\pgfusepath{stroke}%
\end{pgfscope}%
\begin{pgfscope}%
\pgfpathrectangle{\pgfqpoint{0.404000in}{1.712694in}}{\pgfqpoint{2.246634in}{1.362506in}} %
\pgfusepath{clip}%
\pgfsetbuttcap%
\pgfsetroundjoin%
\pgfsetlinewidth{1.003750pt}%
\definecolor{currentstroke}{rgb}{0.000000,0.000000,0.000000}%
\pgfsetstrokecolor{currentstroke}%
\pgfsetdash{{1.000000pt}{1.650000pt}}{0.000000pt}%
\pgfpathmoveto{\pgfqpoint{0.000000in}{0.000000in}}%
\pgfusepath{stroke}%
\end{pgfscope}%
\begin{pgfscope}%
\pgfpathrectangle{\pgfqpoint{0.404000in}{1.712694in}}{\pgfqpoint{2.246634in}{1.362506in}} %
\pgfusepath{clip}%
\pgfsetbuttcap%
\pgfsetroundjoin%
\pgfsetlinewidth{1.003750pt}%
\definecolor{currentstroke}{rgb}{0.000000,0.000000,0.000000}%
\pgfsetstrokecolor{currentstroke}%
\pgfsetdash{{1.000000pt}{1.650000pt}}{0.000000pt}%
\pgfpathmoveto{\pgfqpoint{2.350898in}{1.702694in}}%
\pgfpathlineto{\pgfqpoint{2.346074in}{1.796412in}}%
\pgfpathlineto{\pgfqpoint{2.339588in}{1.881595in}}%
\pgfpathlineto{\pgfqpoint{2.332042in}{1.953501in}}%
\pgfpathlineto{\pgfqpoint{2.323181in}{2.017964in}}%
\pgfpathlineto{\pgfqpoint{2.313276in}{2.074792in}}%
\pgfpathlineto{\pgfqpoint{2.301449in}{2.129309in}}%
\pgfpathlineto{\pgfqpoint{2.289108in}{2.175864in}}%
\pgfpathlineto{\pgfqpoint{2.275188in}{2.219593in}}%
\pgfpathlineto{\pgfqpoint{2.259796in}{2.260171in}}%
\pgfpathlineto{\pgfqpoint{2.243094in}{2.297428in}}%
\pgfpathlineto{\pgfqpoint{2.225274in}{2.331352in}}%
\pgfpathlineto{\pgfqpoint{2.206530in}{2.362061in}}%
\pgfpathlineto{\pgfqpoint{2.187039in}{2.389764in}}%
\pgfpathlineto{\pgfqpoint{2.166952in}{2.414713in}}%
\pgfpathlineto{\pgfqpoint{2.144084in}{2.439530in}}%
\pgfpathlineto{\pgfqpoint{2.120766in}{2.461628in}}%
\pgfpathlineto{\pgfqpoint{2.094720in}{2.483180in}}%
\pgfpathlineto{\pgfqpoint{2.068355in}{2.502207in}}%
\pgfpathlineto{\pgfqpoint{2.039319in}{2.520456in}}%
\pgfpathlineto{\pgfqpoint{2.007619in}{2.537663in}}%
\pgfpathlineto{\pgfqpoint{1.973269in}{2.553611in}}%
\pgfpathlineto{\pgfqpoint{1.938756in}{2.567233in}}%
\pgfpathlineto{\pgfqpoint{1.901645in}{2.579575in}}%
\pgfpathlineto{\pgfqpoint{1.861948in}{2.590477in}}%
\pgfpathlineto{\pgfqpoint{1.819679in}{2.599797in}}%
\pgfpathlineto{\pgfqpoint{1.774851in}{2.607389in}}%
\pgfpathlineto{\pgfqpoint{1.727479in}{2.613114in}}%
\pgfpathlineto{\pgfqpoint{1.677577in}{2.616802in}}%
\pgfpathlineto{\pgfqpoint{1.627656in}{2.618252in}}%
\pgfpathlineto{\pgfqpoint{1.575236in}{2.617485in}}%
\pgfpathlineto{\pgfqpoint{1.522829in}{2.614437in}}%
\pgfpathlineto{\pgfqpoint{1.470452in}{2.609135in}}%
\pgfpathlineto{\pgfqpoint{1.418121in}{2.601540in}}%
\pgfpathlineto{\pgfqpoint{1.368341in}{2.592120in}}%
\pgfpathlineto{\pgfqpoint{1.318638in}{2.580464in}}%
\pgfpathlineto{\pgfqpoint{1.271512in}{2.567199in}}%
\pgfpathlineto{\pgfqpoint{1.224500in}{2.551650in}}%
\pgfpathlineto{\pgfqpoint{1.180093in}{2.534652in}}%
\pgfpathlineto{\pgfqpoint{1.135847in}{2.515255in}}%
\pgfpathlineto{\pgfqpoint{1.094241in}{2.494540in}}%
\pgfpathlineto{\pgfqpoint{1.055282in}{2.472723in}}%
\pgfpathlineto{\pgfqpoint{1.016560in}{2.448451in}}%
\pgfpathlineto{\pgfqpoint{0.980520in}{2.423265in}}%
\pgfpathlineto{\pgfqpoint{0.944787in}{2.395515in}}%
\pgfpathlineto{\pgfqpoint{0.911770in}{2.367105in}}%
\pgfpathlineto{\pgfqpoint{0.879139in}{2.336072in}}%
\pgfpathlineto{\pgfqpoint{0.849250in}{2.304729in}}%
\pgfpathlineto{\pgfqpoint{0.819826in}{2.270783in}}%
\pgfpathlineto{\pgfqpoint{0.790946in}{2.234072in}}%
\pgfpathlineto{\pgfqpoint{0.762701in}{2.194444in}}%
\pgfpathlineto{\pgfqpoint{0.735188in}{2.151762in}}%
\pgfpathlineto{\pgfqpoint{0.710535in}{2.109558in}}%
\pgfpathlineto{\pgfqpoint{0.686686in}{2.064600in}}%
\pgfpathlineto{\pgfqpoint{0.663731in}{2.016864in}}%
\pgfpathlineto{\pgfqpoint{0.641759in}{1.966362in}}%
\pgfpathlineto{\pgfqpoint{0.619162in}{1.908592in}}%
\pgfpathlineto{\pgfqpoint{0.597910in}{1.847761in}}%
\pgfpathlineto{\pgfqpoint{0.578081in}{1.784041in}}%
\pgfpathlineto{\pgfqpoint{0.559733in}{1.717646in}}%
\pgfpathlineto{\pgfqpoint{0.555898in}{1.702694in}}%
\pgfpathmoveto{\pgfqpoint{0.430052in}{1.702694in}}%
\pgfpathlineto{\pgfqpoint{0.446814in}{1.777628in}}%
\pgfpathlineto{\pgfqpoint{0.465426in}{1.851000in}}%
\pgfpathlineto{\pgfqpoint{0.485704in}{1.921602in}}%
\pgfpathlineto{\pgfqpoint{0.506000in}{1.984410in}}%
\pgfpathlineto{\pgfqpoint{0.527663in}{2.044343in}}%
\pgfpathlineto{\pgfqpoint{0.550624in}{2.101223in}}%
\pgfpathlineto{\pgfqpoint{0.574797in}{2.154924in}}%
\pgfpathlineto{\pgfqpoint{0.600081in}{2.205378in}}%
\pgfpathlineto{\pgfqpoint{0.626367in}{2.252577in}}%
\pgfpathlineto{\pgfqpoint{0.653543in}{2.296565in}}%
\pgfpathlineto{\pgfqpoint{0.681500in}{2.337434in}}%
\pgfpathlineto{\pgfqpoint{0.710133in}{2.375310in}}%
\pgfpathlineto{\pgfqpoint{0.739349in}{2.410346in}}%
\pgfpathlineto{\pgfqpoint{0.769061in}{2.442709in}}%
\pgfpathlineto{\pgfqpoint{0.801530in}{2.474773in}}%
\pgfpathlineto{\pgfqpoint{0.834408in}{2.504158in}}%
\pgfpathlineto{\pgfqpoint{0.870011in}{2.532910in}}%
\pgfpathlineto{\pgfqpoint{0.905938in}{2.559076in}}%
\pgfpathlineto{\pgfqpoint{0.944551in}{2.584395in}}%
\pgfpathlineto{\pgfqpoint{0.983410in}{2.607285in}}%
\pgfpathlineto{\pgfqpoint{1.024916in}{2.629197in}}%
\pgfpathlineto{\pgfqpoint{1.069062in}{2.649957in}}%
\pgfpathlineto{\pgfqpoint{1.115837in}{2.669415in}}%
\pgfpathlineto{\pgfqpoint{1.165230in}{2.687448in}}%
\pgfpathlineto{\pgfqpoint{1.217227in}{2.703946in}}%
\pgfpathlineto{\pgfqpoint{1.271816in}{2.718815in}}%
\pgfpathlineto{\pgfqpoint{1.326497in}{2.731438in}}%
\pgfpathlineto{\pgfqpoint{1.383739in}{2.742433in}}%
\pgfpathlineto{\pgfqpoint{1.443533in}{2.751684in}}%
\pgfpathlineto{\pgfqpoint{1.505868in}{2.759052in}}%
\pgfpathlineto{\pgfqpoint{1.568239in}{2.764208in}}%
\pgfpathlineto{\pgfqpoint{1.630634in}{2.767203in}}%
\pgfpathlineto{\pgfqpoint{1.693039in}{2.768037in}}%
\pgfpathlineto{\pgfqpoint{1.755442in}{2.766649in}}%
\pgfpathlineto{\pgfqpoint{1.815335in}{2.763108in}}%
\pgfpathlineto{\pgfqpoint{1.872704in}{2.757515in}}%
\pgfpathlineto{\pgfqpoint{1.927536in}{2.749928in}}%
\pgfpathlineto{\pgfqpoint{1.977327in}{2.740882in}}%
\pgfpathlineto{\pgfqpoint{2.024556in}{2.730110in}}%
\pgfpathlineto{\pgfqpoint{2.069205in}{2.717620in}}%
\pgfpathlineto{\pgfqpoint{2.108783in}{2.704324in}}%
\pgfpathlineto{\pgfqpoint{2.145755in}{2.689635in}}%
\pgfpathlineto{\pgfqpoint{2.180100in}{2.673616in}}%
\pgfpathlineto{\pgfqpoint{2.211795in}{2.656355in}}%
\pgfpathlineto{\pgfqpoint{2.240819in}{2.637981in}}%
\pgfpathlineto{\pgfqpoint{2.267150in}{2.618673in}}%
\pgfpathlineto{\pgfqpoint{2.290775in}{2.598686in}}%
\pgfpathlineto{\pgfqpoint{2.313990in}{2.575945in}}%
\pgfpathlineto{\pgfqpoint{2.334408in}{2.552706in}}%
\pgfpathlineto{\pgfqpoint{2.352045in}{2.529485in}}%
\pgfpathlineto{\pgfqpoint{2.369047in}{2.503514in}}%
\pgfpathlineto{\pgfqpoint{2.385221in}{2.474475in}}%
\pgfpathlineto{\pgfqpoint{2.398511in}{2.446343in}}%
\pgfpathlineto{\pgfqpoint{2.410812in}{2.415573in}}%
\pgfpathlineto{\pgfqpoint{2.421952in}{2.382194in}}%
\pgfpathlineto{\pgfqpoint{2.431788in}{2.346381in}}%
\pgfpathlineto{\pgfqpoint{2.441323in}{2.302873in}}%
\pgfpathlineto{\pgfqpoint{2.449029in}{2.257187in}}%
\pgfpathlineto{\pgfqpoint{2.455648in}{2.203944in}}%
\pgfpathlineto{\pgfqpoint{2.460770in}{2.143350in}}%
\pgfpathlineto{\pgfqpoint{2.464187in}{2.075771in}}%
\pgfpathlineto{\pgfqpoint{2.465932in}{1.995390in}}%
\pgfpathlineto{\pgfqpoint{2.465658in}{1.896314in}}%
\pgfpathlineto{\pgfqpoint{2.462883in}{1.772650in}}%
\pgfpathlineto{\pgfqpoint{2.460541in}{1.702694in}}%
\pgfpathlineto{\pgfqpoint{2.460541in}{1.702694in}}%
\pgfusepath{stroke}%
\end{pgfscope}%
\begin{pgfscope}%
\pgfpathrectangle{\pgfqpoint{0.404000in}{1.712694in}}{\pgfqpoint{2.246634in}{1.362506in}} %
\pgfusepath{clip}%
\pgfsetbuttcap%
\pgfsetroundjoin%
\pgfsetlinewidth{1.003750pt}%
\definecolor{currentstroke}{rgb}{0.000000,0.000000,0.000000}%
\pgfsetstrokecolor{currentstroke}%
\pgfsetdash{{1.000000pt}{1.650000pt}}{0.000000pt}%
\pgfpathmoveto{\pgfqpoint{2.545400in}{1.702694in}}%
\pgfpathlineto{\pgfqpoint{2.577710in}{2.034109in}}%
\pgfpathlineto{\pgfqpoint{2.588618in}{2.161307in}}%
\pgfpathlineto{\pgfqpoint{2.593924in}{2.246992in}}%
\pgfpathlineto{\pgfqpoint{2.596073in}{2.314889in}}%
\pgfpathlineto{\pgfqpoint{2.595815in}{2.370602in}}%
\pgfpathlineto{\pgfqpoint{2.593478in}{2.419781in}}%
\pgfpathlineto{\pgfqpoint{2.589391in}{2.461902in}}%
\pgfpathlineto{\pgfqpoint{2.583120in}{2.502330in}}%
\pgfpathlineto{\pgfqpoint{2.575952in}{2.534930in}}%
\pgfpathlineto{\pgfqpoint{2.567232in}{2.565113in}}%
\pgfpathlineto{\pgfqpoint{2.557169in}{2.592611in}}%
\pgfpathlineto{\pgfqpoint{2.546017in}{2.617395in}}%
\pgfpathlineto{\pgfqpoint{2.531951in}{2.643090in}}%
\pgfpathlineto{\pgfqpoint{2.517041in}{2.665676in}}%
\pgfpathlineto{\pgfqpoint{2.499280in}{2.688303in}}%
\pgfpathlineto{\pgfqpoint{2.480971in}{2.708073in}}%
\pgfpathlineto{\pgfqpoint{2.459915in}{2.727490in}}%
\pgfpathlineto{\pgfqpoint{2.436127in}{2.746248in}}%
\pgfpathlineto{\pgfqpoint{2.409632in}{2.764120in}}%
\pgfpathlineto{\pgfqpoint{2.380456in}{2.780948in}}%
\pgfpathlineto{\pgfqpoint{2.348630in}{2.796654in}}%
\pgfpathlineto{\pgfqpoint{2.314179in}{2.811212in}}%
\pgfpathlineto{\pgfqpoint{2.274653in}{2.825419in}}%
\pgfpathlineto{\pgfqpoint{2.232531in}{2.838196in}}%
\pgfpathlineto{\pgfqpoint{2.185349in}{2.850170in}}%
\pgfpathlineto{\pgfqpoint{2.133112in}{2.861080in}}%
\pgfpathlineto{\pgfqpoint{2.075830in}{2.870711in}}%
\pgfpathlineto{\pgfqpoint{2.013511in}{2.878878in}}%
\pgfpathlineto{\pgfqpoint{1.946164in}{2.885409in}}%
\pgfpathlineto{\pgfqpoint{1.886274in}{2.889168in}}%
\pgfpathlineto{\pgfqpoint{1.811396in}{2.892148in}}%
\pgfpathlineto{\pgfqpoint{1.734014in}{2.893001in}}%
\pgfpathlineto{\pgfqpoint{1.654136in}{2.891609in}}%
\pgfpathlineto{\pgfqpoint{1.574269in}{2.887936in}}%
\pgfpathlineto{\pgfqpoint{1.486940in}{2.881556in}}%
\pgfpathlineto{\pgfqpoint{1.412123in}{2.873500in}}%
\pgfpathlineto{\pgfqpoint{1.339845in}{2.863489in}}%
\pgfpathlineto{\pgfqpoint{1.282550in}{2.854537in}}%
\pgfpathlineto{\pgfqpoint{1.217857in}{2.841626in}}%
\pgfpathlineto{\pgfqpoint{1.155732in}{2.826947in}}%
\pgfpathlineto{\pgfqpoint{1.096191in}{2.810489in}}%
\pgfpathlineto{\pgfqpoint{1.041723in}{2.793100in}}%
\pgfpathlineto{\pgfqpoint{1.029330in}{2.789651in}}%
\pgfpathlineto{\pgfqpoint{1.021883in}{2.787888in}}%
\pgfpathlineto{\pgfqpoint{0.972538in}{2.769075in}}%
\pgfpathlineto{\pgfqpoint{0.906125in}{2.740850in}}%
\pgfpathlineto{\pgfqpoint{0.862114in}{2.718405in}}%
\pgfpathlineto{\pgfqpoint{0.820763in}{2.694754in}}%
\pgfpathlineto{\pgfqpoint{0.782085in}{2.670053in}}%
\pgfpathlineto{\pgfqpoint{0.743696in}{2.642718in}}%
\pgfpathlineto{\pgfqpoint{0.710398in}{2.616404in}}%
\pgfpathlineto{\pgfqpoint{0.675064in}{2.585667in}}%
\pgfpathlineto{\pgfqpoint{0.642536in}{2.553981in}}%
\pgfpathlineto{\pgfqpoint{0.612769in}{2.521929in}}%
\pgfpathlineto{\pgfqpoint{0.583504in}{2.487148in}}%
\pgfpathlineto{\pgfqpoint{0.554829in}{2.449465in}}%
\pgfpathlineto{\pgfqpoint{0.526842in}{2.408725in}}%
\pgfpathlineto{\pgfqpoint{0.516633in}{2.390982in}}%
\pgfpathlineto{\pgfqpoint{0.471794in}{2.312377in}}%
\pgfpathlineto{\pgfqpoint{0.448514in}{2.265620in}}%
\pgfpathlineto{\pgfqpoint{0.426184in}{2.216092in}}%
\pgfpathlineto{\pgfqpoint{0.403146in}{2.159404in}}%
\pgfpathlineto{\pgfqpoint{0.394000in}{2.135107in}}%
\pgfpathlineto{\pgfqpoint{0.394000in}{2.135107in}}%
\pgfusepath{stroke}%
\end{pgfscope}%
\begin{pgfscope}%
\pgfpathrectangle{\pgfqpoint{0.404000in}{1.712694in}}{\pgfqpoint{2.246634in}{1.362506in}} %
\pgfusepath{clip}%
\pgfsetbuttcap%
\pgfsetroundjoin%
\pgfsetlinewidth{1.003750pt}%
\definecolor{currentstroke}{rgb}{0.000000,0.000000,0.000000}%
\pgfsetstrokecolor{currentstroke}%
\pgfsetdash{{1.000000pt}{1.650000pt}}{0.000000pt}%
\pgfpathmoveto{\pgfqpoint{0.000000in}{0.000000in}}%
\pgfusepath{stroke}%
\end{pgfscope}%
\begin{pgfscope}%
\pgfpathrectangle{\pgfqpoint{0.404000in}{1.712694in}}{\pgfqpoint{2.246634in}{1.362506in}} %
\pgfusepath{clip}%
\pgfsetbuttcap%
\pgfsetroundjoin%
\pgfsetlinewidth{1.003750pt}%
\definecolor{currentstroke}{rgb}{0.000000,0.000000,0.000000}%
\pgfsetstrokecolor{currentstroke}%
\pgfsetdash{{1.000000pt}{1.650000pt}}{0.000000pt}%
\pgfpathmoveto{\pgfqpoint{2.544322in}{1.702694in}}%
\pgfpathlineto{\pgfqpoint{2.578787in}{2.058506in}}%
\pgfpathlineto{\pgfqpoint{2.588156in}{2.173845in}}%
\pgfpathlineto{\pgfqpoint{2.592952in}{2.259712in}}%
\pgfpathlineto{\pgfqpoint{2.594492in}{2.327707in}}%
\pgfpathlineto{\pgfqpoint{2.593549in}{2.383371in}}%
\pgfpathlineto{\pgfqpoint{2.590459in}{2.432289in}}%
\pgfpathlineto{\pgfqpoint{2.585606in}{2.473906in}}%
\pgfpathlineto{\pgfqpoint{2.578535in}{2.513511in}}%
\pgfpathlineto{\pgfqpoint{2.570719in}{2.545181in}}%
\pgfpathlineto{\pgfqpoint{2.561432in}{2.574308in}}%
\pgfpathlineto{\pgfqpoint{2.550907in}{2.600723in}}%
\pgfpathlineto{\pgfqpoint{2.537400in}{2.628196in}}%
\pgfpathlineto{\pgfqpoint{2.522902in}{2.652370in}}%
\pgfpathlineto{\pgfqpoint{2.507692in}{2.673693in}}%
\pgfpathlineto{\pgfqpoint{2.489684in}{2.695089in}}%
\pgfpathlineto{\pgfqpoint{2.468868in}{2.716035in}}%
\pgfpathlineto{\pgfqpoint{2.447638in}{2.734261in}}%
\pgfpathlineto{\pgfqpoint{2.423718in}{2.751952in}}%
\pgfpathlineto{\pgfqpoint{2.397121in}{2.768871in}}%
\pgfpathlineto{\pgfqpoint{2.367870in}{2.784873in}}%
\pgfpathlineto{\pgfqpoint{2.335987in}{2.799863in}}%
\pgfpathlineto{\pgfqpoint{2.299024in}{2.814703in}}%
\pgfpathlineto{\pgfqpoint{2.259456in}{2.828179in}}%
\pgfpathlineto{\pgfqpoint{2.214822in}{2.840978in}}%
\pgfpathlineto{\pgfqpoint{2.165125in}{2.852804in}}%
\pgfpathlineto{\pgfqpoint{2.110375in}{2.863420in}}%
\pgfpathlineto{\pgfqpoint{2.050580in}{2.872625in}}%
\pgfpathlineto{\pgfqpoint{1.988245in}{2.879990in}}%
\pgfpathlineto{\pgfqpoint{1.920886in}{2.885727in}}%
\pgfpathlineto{\pgfqpoint{1.883452in}{2.887848in}}%
\pgfpathlineto{\pgfqpoint{1.808574in}{2.890782in}}%
\pgfpathlineto{\pgfqpoint{1.731191in}{2.891578in}}%
\pgfpathlineto{\pgfqpoint{1.653810in}{2.890197in}}%
\pgfpathlineto{\pgfqpoint{1.576439in}{2.886661in}}%
\pgfpathlineto{\pgfqpoint{1.489109in}{2.880297in}}%
\pgfpathlineto{\pgfqpoint{1.414292in}{2.872278in}}%
\pgfpathlineto{\pgfqpoint{1.342013in}{2.862297in}}%
\pgfpathlineto{\pgfqpoint{1.257339in}{2.848083in}}%
\pgfpathlineto{\pgfqpoint{1.195164in}{2.834784in}}%
\pgfpathlineto{\pgfqpoint{1.135558in}{2.819831in}}%
\pgfpathlineto{\pgfqpoint{1.078537in}{2.803217in}}%
\pgfpathlineto{\pgfqpoint{0.999364in}{2.777312in}}%
\pgfpathlineto{\pgfqpoint{0.930370in}{2.749787in}}%
\pgfpathlineto{\pgfqpoint{0.886261in}{2.728557in}}%
\pgfpathlineto{\pgfqpoint{0.842361in}{2.704810in}}%
\pgfpathlineto{\pgfqpoint{0.801143in}{2.679773in}}%
\pgfpathlineto{\pgfqpoint{0.762622in}{2.653605in}}%
\pgfpathlineto{\pgfqpoint{0.726799in}{2.626578in}}%
\pgfpathlineto{\pgfqpoint{0.693686in}{2.598863in}}%
\pgfpathlineto{\pgfqpoint{0.658586in}{2.566541in}}%
\pgfpathlineto{\pgfqpoint{0.628614in}{2.535683in}}%
\pgfpathlineto{\pgfqpoint{0.599106in}{2.502188in}}%
\pgfpathlineto{\pgfqpoint{0.570147in}{2.465867in}}%
\pgfpathlineto{\pgfqpoint{0.541829in}{2.426559in}}%
\pgfpathlineto{\pgfqpoint{0.520676in}{2.393735in}}%
\pgfpathlineto{\pgfqpoint{0.514731in}{2.382478in}}%
\pgfpathlineto{\pgfqpoint{0.490051in}{2.340371in}}%
\pgfpathlineto{\pgfqpoint{0.466184in}{2.295471in}}%
\pgfpathlineto{\pgfqpoint{0.443216in}{2.247775in}}%
\pgfpathlineto{\pgfqpoint{0.419438in}{2.193000in}}%
\pgfpathlineto{\pgfqpoint{0.396886in}{2.135119in}}%
\pgfpathlineto{\pgfqpoint{0.394000in}{2.127238in}}%
\pgfpathlineto{\pgfqpoint{0.394000in}{2.127238in}}%
\pgfusepath{stroke}%
\end{pgfscope}%
\begin{pgfscope}%
\pgfpathrectangle{\pgfqpoint{0.404000in}{1.712694in}}{\pgfqpoint{2.246634in}{1.362506in}} %
\pgfusepath{clip}%
\pgfsetbuttcap%
\pgfsetroundjoin%
\pgfsetlinewidth{1.003750pt}%
\definecolor{currentstroke}{rgb}{0.000000,0.000000,0.000000}%
\pgfsetstrokecolor{currentstroke}%
\pgfsetdash{{1.000000pt}{1.650000pt}}{0.000000pt}%
\pgfpathmoveto{\pgfqpoint{0.000000in}{0.000000in}}%
\pgfusepath{stroke}%
\end{pgfscope}%
\begin{pgfscope}%
\pgfsetrectcap%
\pgfsetmiterjoin%
\pgfsetlinewidth{0.803000pt}%
\definecolor{currentstroke}{rgb}{0.000000,0.000000,0.000000}%
\pgfsetstrokecolor{currentstroke}%
\pgfsetdash{}{0pt}%
\pgfpathmoveto{\pgfqpoint{0.404000in}{1.712694in}}%
\pgfpathlineto{\pgfqpoint{0.404000in}{3.075200in}}%
\pgfusepath{stroke}%
\end{pgfscope}%
\begin{pgfscope}%
\pgfsetrectcap%
\pgfsetmiterjoin%
\pgfsetlinewidth{0.803000pt}%
\definecolor{currentstroke}{rgb}{0.000000,0.000000,0.000000}%
\pgfsetstrokecolor{currentstroke}%
\pgfsetdash{}{0pt}%
\pgfpathmoveto{\pgfqpoint{2.650634in}{1.712694in}}%
\pgfpathlineto{\pgfqpoint{2.650634in}{3.075200in}}%
\pgfusepath{stroke}%
\end{pgfscope}%
\begin{pgfscope}%
\pgfsetrectcap%
\pgfsetmiterjoin%
\pgfsetlinewidth{0.803000pt}%
\definecolor{currentstroke}{rgb}{0.000000,0.000000,0.000000}%
\pgfsetstrokecolor{currentstroke}%
\pgfsetdash{}{0pt}%
\pgfpathmoveto{\pgfqpoint{0.404000in}{1.712694in}}%
\pgfpathlineto{\pgfqpoint{2.650634in}{1.712694in}}%
\pgfusepath{stroke}%
\end{pgfscope}%
\begin{pgfscope}%
\pgfsetrectcap%
\pgfsetmiterjoin%
\pgfsetlinewidth{0.803000pt}%
\definecolor{currentstroke}{rgb}{0.000000,0.000000,0.000000}%
\pgfsetstrokecolor{currentstroke}%
\pgfsetdash{}{0pt}%
\pgfpathmoveto{\pgfqpoint{0.404000in}{3.075200in}}%
\pgfpathlineto{\pgfqpoint{2.650634in}{3.075200in}}%
\pgfusepath{stroke}%
\end{pgfscope}%
\begin{pgfscope}%
\pgfsetbuttcap%
\pgfsetmiterjoin%
\definecolor{currentfill}{rgb}{1.000000,1.000000,1.000000}%
\pgfsetfillcolor{currentfill}%
\pgfsetfillopacity{0.800000}%
\pgfsetlinewidth{1.003750pt}%
\definecolor{currentstroke}{rgb}{0.800000,0.800000,0.800000}%
\pgfsetstrokecolor{currentstroke}%
\pgfsetstrokeopacity{0.800000}%
\pgfsetdash{}{0pt}%
\pgfpathmoveto{\pgfqpoint{1.127368in}{1.782138in}}%
\pgfpathlineto{\pgfqpoint{1.927266in}{1.782138in}}%
\pgfpathquadraticcurveto{\pgfqpoint{1.955044in}{1.782138in}}{\pgfqpoint{1.955044in}{1.809916in}}%
\pgfpathlineto{\pgfqpoint{1.955044in}{2.194916in}}%
\pgfpathquadraticcurveto{\pgfqpoint{1.955044in}{2.222694in}}{\pgfqpoint{1.927266in}{2.222694in}}%
\pgfpathlineto{\pgfqpoint{1.127368in}{2.222694in}}%
\pgfpathquadraticcurveto{\pgfqpoint{1.099590in}{2.222694in}}{\pgfqpoint{1.099590in}{2.194916in}}%
\pgfpathlineto{\pgfqpoint{1.099590in}{1.809916in}}%
\pgfpathquadraticcurveto{\pgfqpoint{1.099590in}{1.782138in}}{\pgfqpoint{1.127368in}{1.782138in}}%
\pgfpathclose%
\pgfusepath{stroke,fill}%
\end{pgfscope}%
\begin{pgfscope}%
\pgfsetrectcap%
\pgfsetroundjoin%
\pgfsetlinewidth{1.003750pt}%
\definecolor{currentstroke}{rgb}{1.000000,0.388235,0.278431}%
\pgfsetstrokecolor{currentstroke}%
\pgfsetdash{}{0pt}%
\pgfpathmoveto{\pgfqpoint{1.155146in}{2.118527in}}%
\pgfpathlineto{\pgfqpoint{1.224590in}{2.118527in}}%
\pgfusepath{stroke}%
\end{pgfscope}%
\begin{pgfscope}%
\pgftext[x=1.335701in,y=2.069916in,left,base]{\rmfamily\fontsize{10.000000}{12.000000}\selectfont \(\displaystyle \textnormal{tol}=10^{-5}\)}%
\end{pgfscope}%
\begin{pgfscope}%
\pgfsetbuttcap%
\pgfsetroundjoin%
\pgfsetlinewidth{1.003750pt}%
\definecolor{currentstroke}{rgb}{0.000000,0.000000,0.000000}%
\pgfsetstrokecolor{currentstroke}%
\pgfsetdash{{1.000000pt}{1.650000pt}}{0.000000pt}%
\pgfpathmoveto{\pgfqpoint{1.155146in}{1.919083in}}%
\pgfpathlineto{\pgfqpoint{1.224590in}{1.919083in}}%
\pgfusepath{stroke}%
\end{pgfscope}%
\begin{pgfscope}%
\pgftext[x=1.335701in,y=1.870472in,left,base]{\rmfamily\fontsize{10.000000}{12.000000}\selectfont Reference}%
\end{pgfscope}%
\begin{pgfscope}%
\pgfsetbuttcap%
\pgfsetmiterjoin%
\definecolor{currentfill}{rgb}{1.000000,1.000000,1.000000}%
\pgfsetfillcolor{currentfill}%
\pgfsetlinewidth{0.000000pt}%
\definecolor{currentstroke}{rgb}{0.000000,0.000000,0.000000}%
\pgfsetstrokecolor{currentstroke}%
\pgfsetstrokeopacity{0.000000}%
\pgfsetdash{}{0pt}%
\pgfpathmoveto{\pgfqpoint{2.762966in}{1.712694in}}%
\pgfpathlineto{\pgfqpoint{5.009600in}{1.712694in}}%
\pgfpathlineto{\pgfqpoint{5.009600in}{3.075200in}}%
\pgfpathlineto{\pgfqpoint{2.762966in}{3.075200in}}%
\pgfpathclose%
\pgfusepath{fill}%
\end{pgfscope}%
\begin{pgfscope}%
\pgfsetbuttcap%
\pgfsetroundjoin%
\definecolor{currentfill}{rgb}{0.000000,0.000000,0.000000}%
\pgfsetfillcolor{currentfill}%
\pgfsetlinewidth{0.803000pt}%
\definecolor{currentstroke}{rgb}{0.000000,0.000000,0.000000}%
\pgfsetstrokecolor{currentstroke}%
\pgfsetdash{}{0pt}%
\pgfsys@defobject{currentmarker}{\pgfqpoint{0.000000in}{-0.048611in}}{\pgfqpoint{0.000000in}{0.000000in}}{%
\pgfpathmoveto{\pgfqpoint{0.000000in}{0.000000in}}%
\pgfpathlineto{\pgfqpoint{0.000000in}{-0.048611in}}%
\pgfusepath{stroke,fill}%
}%
\begin{pgfscope}%
\pgfsys@transformshift{2.887779in}{1.712694in}%
\pgfsys@useobject{currentmarker}{}%
\end{pgfscope}%
\end{pgfscope}%
\begin{pgfscope}%
\pgfsetbuttcap%
\pgfsetroundjoin%
\definecolor{currentfill}{rgb}{0.000000,0.000000,0.000000}%
\pgfsetfillcolor{currentfill}%
\pgfsetlinewidth{0.803000pt}%
\definecolor{currentstroke}{rgb}{0.000000,0.000000,0.000000}%
\pgfsetstrokecolor{currentstroke}%
\pgfsetdash{}{0pt}%
\pgfsys@defobject{currentmarker}{\pgfqpoint{0.000000in}{-0.048611in}}{\pgfqpoint{0.000000in}{0.000000in}}{%
\pgfpathmoveto{\pgfqpoint{0.000000in}{0.000000in}}%
\pgfpathlineto{\pgfqpoint{0.000000in}{-0.048611in}}%
\pgfusepath{stroke,fill}%
}%
\begin{pgfscope}%
\pgfsys@transformshift{3.262218in}{1.712694in}%
\pgfsys@useobject{currentmarker}{}%
\end{pgfscope}%
\end{pgfscope}%
\begin{pgfscope}%
\pgfsetbuttcap%
\pgfsetroundjoin%
\definecolor{currentfill}{rgb}{0.000000,0.000000,0.000000}%
\pgfsetfillcolor{currentfill}%
\pgfsetlinewidth{0.803000pt}%
\definecolor{currentstroke}{rgb}{0.000000,0.000000,0.000000}%
\pgfsetstrokecolor{currentstroke}%
\pgfsetdash{}{0pt}%
\pgfsys@defobject{currentmarker}{\pgfqpoint{0.000000in}{-0.048611in}}{\pgfqpoint{0.000000in}{0.000000in}}{%
\pgfpathmoveto{\pgfqpoint{0.000000in}{0.000000in}}%
\pgfpathlineto{\pgfqpoint{0.000000in}{-0.048611in}}%
\pgfusepath{stroke,fill}%
}%
\begin{pgfscope}%
\pgfsys@transformshift{3.636657in}{1.712694in}%
\pgfsys@useobject{currentmarker}{}%
\end{pgfscope}%
\end{pgfscope}%
\begin{pgfscope}%
\pgfsetbuttcap%
\pgfsetroundjoin%
\definecolor{currentfill}{rgb}{0.000000,0.000000,0.000000}%
\pgfsetfillcolor{currentfill}%
\pgfsetlinewidth{0.803000pt}%
\definecolor{currentstroke}{rgb}{0.000000,0.000000,0.000000}%
\pgfsetstrokecolor{currentstroke}%
\pgfsetdash{}{0pt}%
\pgfsys@defobject{currentmarker}{\pgfqpoint{0.000000in}{-0.048611in}}{\pgfqpoint{0.000000in}{0.000000in}}{%
\pgfpathmoveto{\pgfqpoint{0.000000in}{0.000000in}}%
\pgfpathlineto{\pgfqpoint{0.000000in}{-0.048611in}}%
\pgfusepath{stroke,fill}%
}%
\begin{pgfscope}%
\pgfsys@transformshift{4.011096in}{1.712694in}%
\pgfsys@useobject{currentmarker}{}%
\end{pgfscope}%
\end{pgfscope}%
\begin{pgfscope}%
\pgfsetbuttcap%
\pgfsetroundjoin%
\definecolor{currentfill}{rgb}{0.000000,0.000000,0.000000}%
\pgfsetfillcolor{currentfill}%
\pgfsetlinewidth{0.803000pt}%
\definecolor{currentstroke}{rgb}{0.000000,0.000000,0.000000}%
\pgfsetstrokecolor{currentstroke}%
\pgfsetdash{}{0pt}%
\pgfsys@defobject{currentmarker}{\pgfqpoint{0.000000in}{-0.048611in}}{\pgfqpoint{0.000000in}{0.000000in}}{%
\pgfpathmoveto{\pgfqpoint{0.000000in}{0.000000in}}%
\pgfpathlineto{\pgfqpoint{0.000000in}{-0.048611in}}%
\pgfusepath{stroke,fill}%
}%
\begin{pgfscope}%
\pgfsys@transformshift{4.385535in}{1.712694in}%
\pgfsys@useobject{currentmarker}{}%
\end{pgfscope}%
\end{pgfscope}%
\begin{pgfscope}%
\pgfsetbuttcap%
\pgfsetroundjoin%
\definecolor{currentfill}{rgb}{0.000000,0.000000,0.000000}%
\pgfsetfillcolor{currentfill}%
\pgfsetlinewidth{0.803000pt}%
\definecolor{currentstroke}{rgb}{0.000000,0.000000,0.000000}%
\pgfsetstrokecolor{currentstroke}%
\pgfsetdash{}{0pt}%
\pgfsys@defobject{currentmarker}{\pgfqpoint{0.000000in}{-0.048611in}}{\pgfqpoint{0.000000in}{0.000000in}}{%
\pgfpathmoveto{\pgfqpoint{0.000000in}{0.000000in}}%
\pgfpathlineto{\pgfqpoint{0.000000in}{-0.048611in}}%
\pgfusepath{stroke,fill}%
}%
\begin{pgfscope}%
\pgfsys@transformshift{4.759974in}{1.712694in}%
\pgfsys@useobject{currentmarker}{}%
\end{pgfscope}%
\end{pgfscope}%
\begin{pgfscope}%
\pgfsetbuttcap%
\pgfsetroundjoin%
\definecolor{currentfill}{rgb}{0.000000,0.000000,0.000000}%
\pgfsetfillcolor{currentfill}%
\pgfsetlinewidth{0.803000pt}%
\definecolor{currentstroke}{rgb}{0.000000,0.000000,0.000000}%
\pgfsetstrokecolor{currentstroke}%
\pgfsetdash{}{0pt}%
\pgfsys@defobject{currentmarker}{\pgfqpoint{-0.048611in}{0.000000in}}{\pgfqpoint{0.000000in}{0.000000in}}{%
\pgfpathmoveto{\pgfqpoint{0.000000in}{0.000000in}}%
\pgfpathlineto{\pgfqpoint{-0.048611in}{0.000000in}}%
\pgfusepath{stroke,fill}%
}%
\begin{pgfscope}%
\pgfsys@transformshift{2.762966in}{1.960422in}%
\pgfsys@useobject{currentmarker}{}%
\end{pgfscope}%
\end{pgfscope}%
\begin{pgfscope}%
\pgfsetbuttcap%
\pgfsetroundjoin%
\definecolor{currentfill}{rgb}{0.000000,0.000000,0.000000}%
\pgfsetfillcolor{currentfill}%
\pgfsetlinewidth{0.803000pt}%
\definecolor{currentstroke}{rgb}{0.000000,0.000000,0.000000}%
\pgfsetstrokecolor{currentstroke}%
\pgfsetdash{}{0pt}%
\pgfsys@defobject{currentmarker}{\pgfqpoint{-0.048611in}{0.000000in}}{\pgfqpoint{0.000000in}{0.000000in}}{%
\pgfpathmoveto{\pgfqpoint{0.000000in}{0.000000in}}%
\pgfpathlineto{\pgfqpoint{-0.048611in}{0.000000in}}%
\pgfusepath{stroke,fill}%
}%
\begin{pgfscope}%
\pgfsys@transformshift{2.762966in}{2.270083in}%
\pgfsys@useobject{currentmarker}{}%
\end{pgfscope}%
\end{pgfscope}%
\begin{pgfscope}%
\pgfsetbuttcap%
\pgfsetroundjoin%
\definecolor{currentfill}{rgb}{0.000000,0.000000,0.000000}%
\pgfsetfillcolor{currentfill}%
\pgfsetlinewidth{0.803000pt}%
\definecolor{currentstroke}{rgb}{0.000000,0.000000,0.000000}%
\pgfsetstrokecolor{currentstroke}%
\pgfsetdash{}{0pt}%
\pgfsys@defobject{currentmarker}{\pgfqpoint{-0.048611in}{0.000000in}}{\pgfqpoint{0.000000in}{0.000000in}}{%
\pgfpathmoveto{\pgfqpoint{0.000000in}{0.000000in}}%
\pgfpathlineto{\pgfqpoint{-0.048611in}{0.000000in}}%
\pgfusepath{stroke,fill}%
}%
\begin{pgfscope}%
\pgfsys@transformshift{2.762966in}{2.579743in}%
\pgfsys@useobject{currentmarker}{}%
\end{pgfscope}%
\end{pgfscope}%
\begin{pgfscope}%
\pgfsetbuttcap%
\pgfsetroundjoin%
\definecolor{currentfill}{rgb}{0.000000,0.000000,0.000000}%
\pgfsetfillcolor{currentfill}%
\pgfsetlinewidth{0.803000pt}%
\definecolor{currentstroke}{rgb}{0.000000,0.000000,0.000000}%
\pgfsetstrokecolor{currentstroke}%
\pgfsetdash{}{0pt}%
\pgfsys@defobject{currentmarker}{\pgfqpoint{-0.048611in}{0.000000in}}{\pgfqpoint{0.000000in}{0.000000in}}{%
\pgfpathmoveto{\pgfqpoint{0.000000in}{0.000000in}}%
\pgfpathlineto{\pgfqpoint{-0.048611in}{0.000000in}}%
\pgfusepath{stroke,fill}%
}%
\begin{pgfscope}%
\pgfsys@transformshift{2.762966in}{2.889404in}%
\pgfsys@useobject{currentmarker}{}%
\end{pgfscope}%
\end{pgfscope}%
\begin{pgfscope}%
\pgfpathrectangle{\pgfqpoint{2.762966in}{1.712694in}}{\pgfqpoint{2.246634in}{1.362506in}} %
\pgfusepath{clip}%
\pgfsetrectcap%
\pgfsetroundjoin%
\pgfsetlinewidth{1.505625pt}%
\definecolor{currentstroke}{rgb}{0.121569,0.466667,0.705882}%
\pgfsetstrokecolor{currentstroke}%
\pgfsetdash{}{0pt}%
\pgfusepath{stroke}%
\end{pgfscope}%
\begin{pgfscope}%
\pgfpathrectangle{\pgfqpoint{2.762966in}{1.712694in}}{\pgfqpoint{2.246634in}{1.362506in}} %
\pgfusepath{clip}%
\pgfsetrectcap%
\pgfsetroundjoin%
\pgfsetlinewidth{1.505625pt}%
\definecolor{currentstroke}{rgb}{1.000000,0.498039,0.054902}%
\pgfsetstrokecolor{currentstroke}%
\pgfsetdash{}{0pt}%
\pgfusepath{stroke}%
\end{pgfscope}%
\begin{pgfscope}%
\pgfpathrectangle{\pgfqpoint{2.762966in}{1.712694in}}{\pgfqpoint{2.246634in}{1.362506in}} %
\pgfusepath{clip}%
\pgfsetrectcap%
\pgfsetroundjoin%
\pgfsetlinewidth{1.505625pt}%
\definecolor{currentstroke}{rgb}{0.172549,0.627451,0.172549}%
\pgfsetstrokecolor{currentstroke}%
\pgfsetdash{}{0pt}%
\pgfusepath{stroke}%
\end{pgfscope}%
\begin{pgfscope}%
\pgfpathrectangle{\pgfqpoint{2.762966in}{1.712694in}}{\pgfqpoint{2.246634in}{1.362506in}} %
\pgfusepath{clip}%
\pgfsetrectcap%
\pgfsetroundjoin%
\pgfsetlinewidth{1.505625pt}%
\definecolor{currentstroke}{rgb}{0.839216,0.152941,0.156863}%
\pgfsetstrokecolor{currentstroke}%
\pgfsetdash{}{0pt}%
\pgfusepath{stroke}%
\end{pgfscope}%
\begin{pgfscope}%
\pgfpathrectangle{\pgfqpoint{2.762966in}{1.712694in}}{\pgfqpoint{2.246634in}{1.362506in}} %
\pgfusepath{clip}%
\pgfsetrectcap%
\pgfsetroundjoin%
\pgfsetlinewidth{1.003750pt}%
\definecolor{currentstroke}{rgb}{1.000000,0.388235,0.278431}%
\pgfsetstrokecolor{currentstroke}%
\pgfsetdash{}{0pt}%
\pgfpathmoveto{\pgfqpoint{4.709864in}{1.702694in}}%
\pgfpathlineto{\pgfqpoint{4.705040in}{1.796401in}}%
\pgfpathlineto{\pgfqpoint{4.698554in}{1.881584in}}%
\pgfpathlineto{\pgfqpoint{4.691009in}{1.953491in}}%
\pgfpathlineto{\pgfqpoint{4.682148in}{2.017954in}}%
\pgfpathlineto{\pgfqpoint{4.672243in}{2.074783in}}%
\pgfpathlineto{\pgfqpoint{4.660417in}{2.129300in}}%
\pgfpathlineto{\pgfqpoint{4.648076in}{2.175856in}}%
\pgfpathlineto{\pgfqpoint{4.634156in}{2.219585in}}%
\pgfpathlineto{\pgfqpoint{4.618764in}{2.260164in}}%
\pgfpathlineto{\pgfqpoint{4.602063in}{2.297421in}}%
\pgfpathlineto{\pgfqpoint{4.584243in}{2.331346in}}%
\pgfpathlineto{\pgfqpoint{4.565499in}{2.362055in}}%
\pgfpathlineto{\pgfqpoint{4.546008in}{2.389758in}}%
\pgfpathlineto{\pgfqpoint{4.525921in}{2.414708in}}%
\pgfpathlineto{\pgfqpoint{4.503053in}{2.439526in}}%
\pgfpathlineto{\pgfqpoint{4.479736in}{2.461624in}}%
\pgfpathlineto{\pgfqpoint{4.453690in}{2.483177in}}%
\pgfpathlineto{\pgfqpoint{4.427325in}{2.502204in}}%
\pgfpathlineto{\pgfqpoint{4.398289in}{2.520453in}}%
\pgfpathlineto{\pgfqpoint{4.366589in}{2.537661in}}%
\pgfpathlineto{\pgfqpoint{4.332238in}{2.553609in}}%
\pgfpathlineto{\pgfqpoint{4.297726in}{2.567232in}}%
\pgfpathlineto{\pgfqpoint{4.260615in}{2.579573in}}%
\pgfpathlineto{\pgfqpoint{4.220918in}{2.590476in}}%
\pgfpathlineto{\pgfqpoint{4.178649in}{2.599796in}}%
\pgfpathlineto{\pgfqpoint{4.133821in}{2.607389in}}%
\pgfpathlineto{\pgfqpoint{4.086449in}{2.613113in}}%
\pgfpathlineto{\pgfqpoint{4.036547in}{2.616801in}}%
\pgfpathlineto{\pgfqpoint{3.986626in}{2.618252in}}%
\pgfpathlineto{\pgfqpoint{3.934206in}{2.617485in}}%
\pgfpathlineto{\pgfqpoint{3.881799in}{2.614437in}}%
\pgfpathlineto{\pgfqpoint{3.829422in}{2.609135in}}%
\pgfpathlineto{\pgfqpoint{3.777091in}{2.601541in}}%
\pgfpathlineto{\pgfqpoint{3.727311in}{2.592120in}}%
\pgfpathlineto{\pgfqpoint{3.677608in}{2.580464in}}%
\pgfpathlineto{\pgfqpoint{3.630482in}{2.567200in}}%
\pgfpathlineto{\pgfqpoint{3.583470in}{2.551651in}}%
\pgfpathlineto{\pgfqpoint{3.539063in}{2.534653in}}%
\pgfpathlineto{\pgfqpoint{3.494817in}{2.515256in}}%
\pgfpathlineto{\pgfqpoint{3.453211in}{2.494542in}}%
\pgfpathlineto{\pgfqpoint{3.414252in}{2.472725in}}%
\pgfpathlineto{\pgfqpoint{3.375530in}{2.448454in}}%
\pgfpathlineto{\pgfqpoint{3.339489in}{2.423267in}}%
\pgfpathlineto{\pgfqpoint{3.303756in}{2.395518in}}%
\pgfpathlineto{\pgfqpoint{3.270740in}{2.367108in}}%
\pgfpathlineto{\pgfqpoint{3.238109in}{2.336075in}}%
\pgfpathlineto{\pgfqpoint{3.208219in}{2.304732in}}%
\pgfpathlineto{\pgfqpoint{3.178795in}{2.270787in}}%
\pgfpathlineto{\pgfqpoint{3.149916in}{2.234077in}}%
\pgfpathlineto{\pgfqpoint{3.121670in}{2.194449in}}%
\pgfpathlineto{\pgfqpoint{3.094158in}{2.151767in}}%
\pgfpathlineto{\pgfqpoint{3.069504in}{2.109564in}}%
\pgfpathlineto{\pgfqpoint{3.045655in}{2.064606in}}%
\pgfpathlineto{\pgfqpoint{3.022700in}{2.016870in}}%
\pgfpathlineto{\pgfqpoint{3.000728in}{1.966368in}}%
\pgfpathlineto{\pgfqpoint{2.978130in}{1.908599in}}%
\pgfpathlineto{\pgfqpoint{2.956879in}{1.847768in}}%
\pgfpathlineto{\pgfqpoint{2.937050in}{1.784048in}}%
\pgfpathlineto{\pgfqpoint{2.918701in}{1.717654in}}%
\pgfpathlineto{\pgfqpoint{2.914864in}{1.702694in}}%
\pgfpathmoveto{\pgfqpoint{2.789018in}{1.702694in}}%
\pgfpathlineto{\pgfqpoint{2.805780in}{1.777627in}}%
\pgfpathlineto{\pgfqpoint{2.824392in}{1.850999in}}%
\pgfpathlineto{\pgfqpoint{2.844669in}{1.921602in}}%
\pgfpathlineto{\pgfqpoint{2.864966in}{1.984409in}}%
\pgfpathlineto{\pgfqpoint{2.886628in}{2.044342in}}%
\pgfpathlineto{\pgfqpoint{2.909590in}{2.101222in}}%
\pgfpathlineto{\pgfqpoint{2.933763in}{2.154923in}}%
\pgfpathlineto{\pgfqpoint{2.959047in}{2.205378in}}%
\pgfpathlineto{\pgfqpoint{2.985333in}{2.252577in}}%
\pgfpathlineto{\pgfqpoint{3.012509in}{2.296565in}}%
\pgfpathlineto{\pgfqpoint{3.040465in}{2.337433in}}%
\pgfpathlineto{\pgfqpoint{3.069099in}{2.375310in}}%
\pgfpathlineto{\pgfqpoint{3.098314in}{2.410345in}}%
\pgfpathlineto{\pgfqpoint{3.128027in}{2.442709in}}%
\pgfpathlineto{\pgfqpoint{3.160495in}{2.474772in}}%
\pgfpathlineto{\pgfqpoint{3.193373in}{2.504157in}}%
\pgfpathlineto{\pgfqpoint{3.228977in}{2.532909in}}%
\pgfpathlineto{\pgfqpoint{3.264903in}{2.559076in}}%
\pgfpathlineto{\pgfqpoint{3.303517in}{2.584394in}}%
\pgfpathlineto{\pgfqpoint{3.342376in}{2.607284in}}%
\pgfpathlineto{\pgfqpoint{3.383882in}{2.629196in}}%
\pgfpathlineto{\pgfqpoint{3.428028in}{2.649956in}}%
\pgfpathlineto{\pgfqpoint{3.474803in}{2.669414in}}%
\pgfpathlineto{\pgfqpoint{3.524195in}{2.687447in}}%
\pgfpathlineto{\pgfqpoint{3.576193in}{2.703946in}}%
\pgfpathlineto{\pgfqpoint{3.630782in}{2.718815in}}%
\pgfpathlineto{\pgfqpoint{3.685463in}{2.731437in}}%
\pgfpathlineto{\pgfqpoint{3.742705in}{2.742433in}}%
\pgfpathlineto{\pgfqpoint{3.802498in}{2.751683in}}%
\pgfpathlineto{\pgfqpoint{3.864833in}{2.759052in}}%
\pgfpathlineto{\pgfqpoint{3.927205in}{2.764208in}}%
\pgfpathlineto{\pgfqpoint{3.989599in}{2.767203in}}%
\pgfpathlineto{\pgfqpoint{4.052004in}{2.768037in}}%
\pgfpathlineto{\pgfqpoint{4.114408in}{2.766649in}}%
\pgfpathlineto{\pgfqpoint{4.174300in}{2.763108in}}%
\pgfpathlineto{\pgfqpoint{4.231669in}{2.757514in}}%
\pgfpathlineto{\pgfqpoint{4.286501in}{2.749927in}}%
\pgfpathlineto{\pgfqpoint{4.336293in}{2.740882in}}%
\pgfpathlineto{\pgfqpoint{4.383522in}{2.730109in}}%
\pgfpathlineto{\pgfqpoint{4.428170in}{2.717620in}}%
\pgfpathlineto{\pgfqpoint{4.467749in}{2.704323in}}%
\pgfpathlineto{\pgfqpoint{4.504720in}{2.689634in}}%
\pgfpathlineto{\pgfqpoint{4.539065in}{2.673616in}}%
\pgfpathlineto{\pgfqpoint{4.570761in}{2.656355in}}%
\pgfpathlineto{\pgfqpoint{4.599784in}{2.637980in}}%
\pgfpathlineto{\pgfqpoint{4.626115in}{2.618673in}}%
\pgfpathlineto{\pgfqpoint{4.649740in}{2.598685in}}%
\pgfpathlineto{\pgfqpoint{4.672956in}{2.575945in}}%
\pgfpathlineto{\pgfqpoint{4.693373in}{2.552705in}}%
\pgfpathlineto{\pgfqpoint{4.711011in}{2.529485in}}%
\pgfpathlineto{\pgfqpoint{4.728013in}{2.503514in}}%
\pgfpathlineto{\pgfqpoint{4.744187in}{2.474475in}}%
\pgfpathlineto{\pgfqpoint{4.757476in}{2.446343in}}%
\pgfpathlineto{\pgfqpoint{4.769777in}{2.415573in}}%
\pgfpathlineto{\pgfqpoint{4.780918in}{2.382194in}}%
\pgfpathlineto{\pgfqpoint{4.790754in}{2.346381in}}%
\pgfpathlineto{\pgfqpoint{4.800288in}{2.302873in}}%
\pgfpathlineto{\pgfqpoint{4.807995in}{2.257187in}}%
\pgfpathlineto{\pgfqpoint{4.814613in}{2.203944in}}%
\pgfpathlineto{\pgfqpoint{4.819736in}{2.143350in}}%
\pgfpathlineto{\pgfqpoint{4.823152in}{2.075770in}}%
\pgfpathlineto{\pgfqpoint{4.824898in}{1.995390in}}%
\pgfpathlineto{\pgfqpoint{4.824623in}{1.896313in}}%
\pgfpathlineto{\pgfqpoint{4.821848in}{1.772650in}}%
\pgfpathlineto{\pgfqpoint{4.819507in}{1.702694in}}%
\pgfpathlineto{\pgfqpoint{4.819507in}{1.702694in}}%
\pgfusepath{stroke}%
\end{pgfscope}%
\begin{pgfscope}%
\pgfpathrectangle{\pgfqpoint{2.762966in}{1.712694in}}{\pgfqpoint{2.246634in}{1.362506in}} %
\pgfusepath{clip}%
\pgfsetrectcap%
\pgfsetroundjoin%
\pgfsetlinewidth{1.003750pt}%
\definecolor{currentstroke}{rgb}{1.000000,0.388235,0.278431}%
\pgfsetstrokecolor{currentstroke}%
\pgfsetdash{}{0pt}%
\pgfpathmoveto{\pgfqpoint{4.904368in}{1.702694in}}%
\pgfpathlineto{\pgfqpoint{4.936680in}{2.034120in}}%
\pgfpathlineto{\pgfqpoint{4.947587in}{2.161318in}}%
\pgfpathlineto{\pgfqpoint{4.952893in}{2.247002in}}%
\pgfpathlineto{\pgfqpoint{4.955042in}{2.314900in}}%
\pgfpathlineto{\pgfqpoint{4.954785in}{2.370613in}}%
\pgfpathlineto{\pgfqpoint{4.952448in}{2.419791in}}%
\pgfpathlineto{\pgfqpoint{4.948360in}{2.461913in}}%
\pgfpathlineto{\pgfqpoint{4.942089in}{2.502340in}}%
\pgfpathlineto{\pgfqpoint{4.934920in}{2.534939in}}%
\pgfpathlineto{\pgfqpoint{4.926200in}{2.565123in}}%
\pgfpathlineto{\pgfqpoint{4.916137in}{2.592620in}}%
\pgfpathlineto{\pgfqpoint{4.904985in}{2.617403in}}%
\pgfpathlineto{\pgfqpoint{4.890919in}{2.643098in}}%
\pgfpathlineto{\pgfqpoint{4.876008in}{2.665683in}}%
\pgfpathlineto{\pgfqpoint{4.858247in}{2.688310in}}%
\pgfpathlineto{\pgfqpoint{4.839938in}{2.708079in}}%
\pgfpathlineto{\pgfqpoint{4.818882in}{2.727496in}}%
\pgfpathlineto{\pgfqpoint{4.795094in}{2.746253in}}%
\pgfpathlineto{\pgfqpoint{4.768599in}{2.764125in}}%
\pgfpathlineto{\pgfqpoint{4.739423in}{2.780953in}}%
\pgfpathlineto{\pgfqpoint{4.707597in}{2.796658in}}%
\pgfpathlineto{\pgfqpoint{4.673146in}{2.811217in}}%
\pgfpathlineto{\pgfqpoint{4.633620in}{2.825423in}}%
\pgfpathlineto{\pgfqpoint{4.591498in}{2.838200in}}%
\pgfpathlineto{\pgfqpoint{4.544315in}{2.850174in}}%
\pgfpathlineto{\pgfqpoint{4.492079in}{2.861083in}}%
\pgfpathlineto{\pgfqpoint{4.434797in}{2.870714in}}%
\pgfpathlineto{\pgfqpoint{4.372478in}{2.878881in}}%
\pgfpathlineto{\pgfqpoint{4.305131in}{2.885412in}}%
\pgfpathlineto{\pgfqpoint{4.245241in}{2.889171in}}%
\pgfpathlineto{\pgfqpoint{4.170363in}{2.892151in}}%
\pgfpathlineto{\pgfqpoint{4.092981in}{2.893004in}}%
\pgfpathlineto{\pgfqpoint{4.013103in}{2.891612in}}%
\pgfpathlineto{\pgfqpoint{3.933236in}{2.887939in}}%
\pgfpathlineto{\pgfqpoint{3.845907in}{2.881559in}}%
\pgfpathlineto{\pgfqpoint{3.771090in}{2.873503in}}%
\pgfpathlineto{\pgfqpoint{3.698812in}{2.863493in}}%
\pgfpathlineto{\pgfqpoint{3.641517in}{2.854541in}}%
\pgfpathlineto{\pgfqpoint{3.576824in}{2.841630in}}%
\pgfpathlineto{\pgfqpoint{3.514699in}{2.826951in}}%
\pgfpathlineto{\pgfqpoint{3.455158in}{2.810494in}}%
\pgfpathlineto{\pgfqpoint{3.400690in}{2.793104in}}%
\pgfpathlineto{\pgfqpoint{3.388297in}{2.789655in}}%
\pgfpathlineto{\pgfqpoint{3.380850in}{2.787892in}}%
\pgfpathlineto{\pgfqpoint{3.331505in}{2.769080in}}%
\pgfpathlineto{\pgfqpoint{3.265092in}{2.740850in}}%
\pgfpathlineto{\pgfqpoint{3.221081in}{2.718405in}}%
\pgfpathlineto{\pgfqpoint{3.179730in}{2.694753in}}%
\pgfpathlineto{\pgfqpoint{3.141052in}{2.670053in}}%
\pgfpathlineto{\pgfqpoint{3.102663in}{2.642718in}}%
\pgfpathlineto{\pgfqpoint{3.069365in}{2.616403in}}%
\pgfpathlineto{\pgfqpoint{3.034031in}{2.585667in}}%
\pgfpathlineto{\pgfqpoint{3.001503in}{2.553981in}}%
\pgfpathlineto{\pgfqpoint{2.971736in}{2.521929in}}%
\pgfpathlineto{\pgfqpoint{2.942471in}{2.487148in}}%
\pgfpathlineto{\pgfqpoint{2.913796in}{2.449465in}}%
\pgfpathlineto{\pgfqpoint{2.885809in}{2.408725in}}%
\pgfpathlineto{\pgfqpoint{2.875599in}{2.390984in}}%
\pgfpathlineto{\pgfqpoint{2.830760in}{2.312379in}}%
\pgfpathlineto{\pgfqpoint{2.807480in}{2.265621in}}%
\pgfpathlineto{\pgfqpoint{2.785151in}{2.216093in}}%
\pgfpathlineto{\pgfqpoint{2.762112in}{2.159405in}}%
\pgfpathlineto{\pgfqpoint{2.752966in}{2.135107in}}%
\pgfpathlineto{\pgfqpoint{2.752966in}{2.135107in}}%
\pgfusepath{stroke}%
\end{pgfscope}%
\begin{pgfscope}%
\pgfpathrectangle{\pgfqpoint{2.762966in}{1.712694in}}{\pgfqpoint{2.246634in}{1.362506in}} %
\pgfusepath{clip}%
\pgfsetrectcap%
\pgfsetroundjoin%
\pgfsetlinewidth{1.003750pt}%
\definecolor{currentstroke}{rgb}{1.000000,0.388235,0.278431}%
\pgfsetstrokecolor{currentstroke}%
\pgfsetdash{}{0pt}%
\pgfpathmoveto{\pgfqpoint{4.903290in}{1.702694in}}%
\pgfpathlineto{\pgfqpoint{4.937756in}{2.058519in}}%
\pgfpathlineto{\pgfqpoint{4.947125in}{2.173859in}}%
\pgfpathlineto{\pgfqpoint{4.951922in}{2.259725in}}%
\pgfpathlineto{\pgfqpoint{4.953461in}{2.327720in}}%
\pgfpathlineto{\pgfqpoint{4.952518in}{2.383384in}}%
\pgfpathlineto{\pgfqpoint{4.949428in}{2.432302in}}%
\pgfpathlineto{\pgfqpoint{4.944575in}{2.473918in}}%
\pgfpathlineto{\pgfqpoint{4.937503in}{2.513523in}}%
\pgfpathlineto{\pgfqpoint{4.929686in}{2.545192in}}%
\pgfpathlineto{\pgfqpoint{4.920399in}{2.574319in}}%
\pgfpathlineto{\pgfqpoint{4.909874in}{2.600734in}}%
\pgfpathlineto{\pgfqpoint{4.896367in}{2.628205in}}%
\pgfpathlineto{\pgfqpoint{4.881868in}{2.652379in}}%
\pgfpathlineto{\pgfqpoint{4.866658in}{2.673701in}}%
\pgfpathlineto{\pgfqpoint{4.848650in}{2.695097in}}%
\pgfpathlineto{\pgfqpoint{4.827834in}{2.716042in}}%
\pgfpathlineto{\pgfqpoint{4.806604in}{2.734267in}}%
\pgfpathlineto{\pgfqpoint{4.782684in}{2.751957in}}%
\pgfpathlineto{\pgfqpoint{4.756086in}{2.768876in}}%
\pgfpathlineto{\pgfqpoint{4.726835in}{2.784879in}}%
\pgfpathlineto{\pgfqpoint{4.694952in}{2.799867in}}%
\pgfpathlineto{\pgfqpoint{4.657990in}{2.814707in}}%
\pgfpathlineto{\pgfqpoint{4.618422in}{2.828183in}}%
\pgfpathlineto{\pgfqpoint{4.573787in}{2.840982in}}%
\pgfpathlineto{\pgfqpoint{4.524091in}{2.852807in}}%
\pgfpathlineto{\pgfqpoint{4.469341in}{2.863423in}}%
\pgfpathlineto{\pgfqpoint{4.409546in}{2.872628in}}%
\pgfpathlineto{\pgfqpoint{4.347211in}{2.879993in}}%
\pgfpathlineto{\pgfqpoint{4.279852in}{2.885730in}}%
\pgfpathlineto{\pgfqpoint{4.242418in}{2.887851in}}%
\pgfpathlineto{\pgfqpoint{4.167540in}{2.890784in}}%
\pgfpathlineto{\pgfqpoint{4.090157in}{2.891581in}}%
\pgfpathlineto{\pgfqpoint{4.012776in}{2.890200in}}%
\pgfpathlineto{\pgfqpoint{3.935405in}{2.886664in}}%
\pgfpathlineto{\pgfqpoint{3.848075in}{2.880300in}}%
\pgfpathlineto{\pgfqpoint{3.773258in}{2.872280in}}%
\pgfpathlineto{\pgfqpoint{3.700978in}{2.862300in}}%
\pgfpathlineto{\pgfqpoint{3.616305in}{2.848086in}}%
\pgfpathlineto{\pgfqpoint{3.554130in}{2.834788in}}%
\pgfpathlineto{\pgfqpoint{3.494524in}{2.819834in}}%
\pgfpathlineto{\pgfqpoint{3.437503in}{2.803220in}}%
\pgfpathlineto{\pgfqpoint{3.358329in}{2.777317in}}%
\pgfpathlineto{\pgfqpoint{3.289336in}{2.749787in}}%
\pgfpathlineto{\pgfqpoint{3.245227in}{2.728557in}}%
\pgfpathlineto{\pgfqpoint{3.201326in}{2.704810in}}%
\pgfpathlineto{\pgfqpoint{3.160109in}{2.679773in}}%
\pgfpathlineto{\pgfqpoint{3.121588in}{2.653605in}}%
\pgfpathlineto{\pgfqpoint{3.085765in}{2.626578in}}%
\pgfpathlineto{\pgfqpoint{3.052652in}{2.598864in}}%
\pgfpathlineto{\pgfqpoint{3.017551in}{2.566542in}}%
\pgfpathlineto{\pgfqpoint{2.987580in}{2.535683in}}%
\pgfpathlineto{\pgfqpoint{2.958072in}{2.502188in}}%
\pgfpathlineto{\pgfqpoint{2.929113in}{2.465867in}}%
\pgfpathlineto{\pgfqpoint{2.900795in}{2.426559in}}%
\pgfpathlineto{\pgfqpoint{2.879641in}{2.393736in}}%
\pgfpathlineto{\pgfqpoint{2.873697in}{2.382479in}}%
\pgfpathlineto{\pgfqpoint{2.849016in}{2.340373in}}%
\pgfpathlineto{\pgfqpoint{2.825149in}{2.295473in}}%
\pgfpathlineto{\pgfqpoint{2.802181in}{2.247776in}}%
\pgfpathlineto{\pgfqpoint{2.778403in}{2.193001in}}%
\pgfpathlineto{\pgfqpoint{2.755852in}{2.135121in}}%
\pgfpathlineto{\pgfqpoint{2.752966in}{2.127241in}}%
\pgfpathlineto{\pgfqpoint{2.752966in}{2.127241in}}%
\pgfusepath{stroke}%
\end{pgfscope}%
\begin{pgfscope}%
\pgfpathrectangle{\pgfqpoint{2.762966in}{1.712694in}}{\pgfqpoint{2.246634in}{1.362506in}} %
\pgfusepath{clip}%
\pgfsetbuttcap%
\pgfsetroundjoin%
\pgfsetlinewidth{1.003750pt}%
\definecolor{currentstroke}{rgb}{0.000000,0.000000,0.000000}%
\pgfsetstrokecolor{currentstroke}%
\pgfsetdash{{1.000000pt}{1.650000pt}}{0.000000pt}%
\pgfpathmoveto{\pgfqpoint{2.917714in}{2.403675in}}%
\pgfpathlineto{\pgfqpoint{2.945915in}{2.443498in}}%
\pgfpathlineto{\pgfqpoint{2.974771in}{2.480319in}}%
\pgfpathlineto{\pgfqpoint{3.004188in}{2.514301in}}%
\pgfpathlineto{\pgfqpoint{3.034081in}{2.545624in}}%
\pgfpathlineto{\pgfqpoint{3.066721in}{2.576594in}}%
\pgfpathlineto{\pgfqpoint{3.099749in}{2.604926in}}%
\pgfpathlineto{\pgfqpoint{3.135491in}{2.632606in}}%
\pgfpathlineto{\pgfqpoint{3.171534in}{2.657772in}}%
\pgfpathlineto{\pgfqpoint{3.210249in}{2.682111in}}%
\pgfpathlineto{\pgfqpoint{3.251631in}{2.705428in}}%
\pgfpathlineto{\pgfqpoint{3.295668in}{2.727563in}}%
\pgfpathlineto{\pgfqpoint{3.342346in}{2.748395in}}%
\pgfpathlineto{\pgfqpoint{3.391652in}{2.767834in}}%
\pgfpathlineto{\pgfqpoint{3.443570in}{2.785813in}}%
\pgfpathlineto{\pgfqpoint{3.498084in}{2.802288in}}%
\pgfpathlineto{\pgfqpoint{3.555181in}{2.817225in}}%
\pgfpathlineto{\pgfqpoint{3.617335in}{2.831110in}}%
\pgfpathlineto{\pgfqpoint{3.682053in}{2.843238in}}%
\pgfpathlineto{\pgfqpoint{3.749322in}{2.853593in}}%
\pgfpathlineto{\pgfqpoint{3.819132in}{2.862144in}}%
\pgfpathlineto{\pgfqpoint{3.893968in}{2.869032in}}%
\pgfpathlineto{\pgfqpoint{3.968832in}{2.873697in}}%
\pgfpathlineto{\pgfqpoint{4.046208in}{2.876267in}}%
\pgfpathlineto{\pgfqpoint{4.123592in}{2.876559in}}%
\pgfpathlineto{\pgfqpoint{4.198475in}{2.874589in}}%
\pgfpathlineto{\pgfqpoint{4.270847in}{2.870426in}}%
\pgfpathlineto{\pgfqpoint{4.338201in}{2.864350in}}%
\pgfpathlineto{\pgfqpoint{4.400528in}{2.856561in}}%
\pgfpathlineto{\pgfqpoint{4.457818in}{2.847264in}}%
\pgfpathlineto{\pgfqpoint{4.510065in}{2.836662in}}%
\pgfpathlineto{\pgfqpoint{4.559740in}{2.824311in}}%
\pgfpathlineto{\pgfqpoint{4.604346in}{2.810910in}}%
\pgfpathlineto{\pgfqpoint{4.643877in}{2.796770in}}%
\pgfpathlineto{\pgfqpoint{4.680788in}{2.781167in}}%
\pgfpathlineto{\pgfqpoint{4.712608in}{2.765379in}}%
\pgfpathlineto{\pgfqpoint{4.741778in}{2.748496in}}%
\pgfpathlineto{\pgfqpoint{4.768273in}{2.730615in}}%
\pgfpathlineto{\pgfqpoint{4.792066in}{2.711896in}}%
\pgfpathlineto{\pgfqpoint{4.813137in}{2.692581in}}%
\pgfpathlineto{\pgfqpoint{4.831483in}{2.673023in}}%
\pgfpathlineto{\pgfqpoint{4.849317in}{2.650749in}}%
\pgfpathlineto{\pgfqpoint{4.864337in}{2.628612in}}%
\pgfpathlineto{\pgfqpoint{4.878601in}{2.603595in}}%
\pgfpathlineto{\pgfqpoint{4.891840in}{2.575330in}}%
\pgfpathlineto{\pgfqpoint{4.902120in}{2.548329in}}%
\pgfpathlineto{\pgfqpoint{4.911181in}{2.518764in}}%
\pgfpathlineto{\pgfqpoint{4.918827in}{2.486836in}}%
\pgfpathlineto{\pgfqpoint{4.925816in}{2.447137in}}%
\pgfpathlineto{\pgfqpoint{4.930741in}{2.405568in}}%
\pgfpathlineto{\pgfqpoint{4.934089in}{2.356751in}}%
\pgfpathlineto{\pgfqpoint{4.935465in}{2.301140in}}%
\pgfpathlineto{\pgfqpoint{4.934752in}{2.239250in}}%
\pgfpathlineto{\pgfqpoint{4.931394in}{2.159186in}}%
\pgfpathlineto{\pgfqpoint{4.924917in}{2.061417in}}%
\pgfpathlineto{\pgfqpoint{4.913283in}{1.921934in}}%
\pgfpathlineto{\pgfqpoint{4.893196in}{1.702694in}}%
\pgfpathlineto{\pgfqpoint{4.893196in}{1.702694in}}%
\pgfusepath{stroke}%
\end{pgfscope}%
\begin{pgfscope}%
\pgfpathrectangle{\pgfqpoint{2.762966in}{1.712694in}}{\pgfqpoint{2.246634in}{1.362506in}} %
\pgfusepath{clip}%
\pgfsetbuttcap%
\pgfsetroundjoin%
\pgfsetlinewidth{1.003750pt}%
\definecolor{currentstroke}{rgb}{0.000000,0.000000,0.000000}%
\pgfsetstrokecolor{currentstroke}%
\pgfsetdash{{1.000000pt}{1.650000pt}}{0.000000pt}%
\pgfpathmoveto{\pgfqpoint{0.000000in}{0.000000in}}%
\pgfusepath{stroke}%
\end{pgfscope}%
\begin{pgfscope}%
\pgfpathrectangle{\pgfqpoint{2.762966in}{1.712694in}}{\pgfqpoint{2.246634in}{1.362506in}} %
\pgfusepath{clip}%
\pgfsetbuttcap%
\pgfsetroundjoin%
\pgfsetlinewidth{1.003750pt}%
\definecolor{currentstroke}{rgb}{0.000000,0.000000,0.000000}%
\pgfsetstrokecolor{currentstroke}%
\pgfsetdash{{1.000000pt}{1.650000pt}}{0.000000pt}%
\pgfpathmoveto{\pgfqpoint{0.000000in}{0.000000in}}%
\pgfusepath{stroke}%
\end{pgfscope}%
\begin{pgfscope}%
\pgfpathrectangle{\pgfqpoint{2.762966in}{1.712694in}}{\pgfqpoint{2.246634in}{1.362506in}} %
\pgfusepath{clip}%
\pgfsetbuttcap%
\pgfsetroundjoin%
\pgfsetlinewidth{1.003750pt}%
\definecolor{currentstroke}{rgb}{0.000000,0.000000,0.000000}%
\pgfsetstrokecolor{currentstroke}%
\pgfsetdash{{1.000000pt}{1.650000pt}}{0.000000pt}%
\pgfpathmoveto{\pgfqpoint{4.709864in}{1.702694in}}%
\pgfpathlineto{\pgfqpoint{4.705039in}{1.796412in}}%
\pgfpathlineto{\pgfqpoint{4.698554in}{1.881595in}}%
\pgfpathlineto{\pgfqpoint{4.691008in}{1.953501in}}%
\pgfpathlineto{\pgfqpoint{4.682147in}{2.017964in}}%
\pgfpathlineto{\pgfqpoint{4.672242in}{2.074792in}}%
\pgfpathlineto{\pgfqpoint{4.660415in}{2.129309in}}%
\pgfpathlineto{\pgfqpoint{4.648074in}{2.175864in}}%
\pgfpathlineto{\pgfqpoint{4.634154in}{2.219593in}}%
\pgfpathlineto{\pgfqpoint{4.618761in}{2.260171in}}%
\pgfpathlineto{\pgfqpoint{4.602060in}{2.297428in}}%
\pgfpathlineto{\pgfqpoint{4.584240in}{2.331352in}}%
\pgfpathlineto{\pgfqpoint{4.565496in}{2.362061in}}%
\pgfpathlineto{\pgfqpoint{4.546005in}{2.389764in}}%
\pgfpathlineto{\pgfqpoint{4.525918in}{2.414713in}}%
\pgfpathlineto{\pgfqpoint{4.503050in}{2.439530in}}%
\pgfpathlineto{\pgfqpoint{4.479732in}{2.461628in}}%
\pgfpathlineto{\pgfqpoint{4.453686in}{2.483180in}}%
\pgfpathlineto{\pgfqpoint{4.427321in}{2.502207in}}%
\pgfpathlineto{\pgfqpoint{4.398285in}{2.520456in}}%
\pgfpathlineto{\pgfqpoint{4.366585in}{2.537663in}}%
\pgfpathlineto{\pgfqpoint{4.332234in}{2.553611in}}%
\pgfpathlineto{\pgfqpoint{4.297722in}{2.567233in}}%
\pgfpathlineto{\pgfqpoint{4.260611in}{2.579575in}}%
\pgfpathlineto{\pgfqpoint{4.220914in}{2.590477in}}%
\pgfpathlineto{\pgfqpoint{4.178645in}{2.599797in}}%
\pgfpathlineto{\pgfqpoint{4.133817in}{2.607389in}}%
\pgfpathlineto{\pgfqpoint{4.086445in}{2.613114in}}%
\pgfpathlineto{\pgfqpoint{4.036543in}{2.616802in}}%
\pgfpathlineto{\pgfqpoint{3.986622in}{2.618252in}}%
\pgfpathlineto{\pgfqpoint{3.934202in}{2.617485in}}%
\pgfpathlineto{\pgfqpoint{3.881795in}{2.614437in}}%
\pgfpathlineto{\pgfqpoint{3.829418in}{2.609135in}}%
\pgfpathlineto{\pgfqpoint{3.777087in}{2.601540in}}%
\pgfpathlineto{\pgfqpoint{3.727307in}{2.592120in}}%
\pgfpathlineto{\pgfqpoint{3.677604in}{2.580464in}}%
\pgfpathlineto{\pgfqpoint{3.630478in}{2.567199in}}%
\pgfpathlineto{\pgfqpoint{3.583466in}{2.551650in}}%
\pgfpathlineto{\pgfqpoint{3.539059in}{2.534652in}}%
\pgfpathlineto{\pgfqpoint{3.494813in}{2.515255in}}%
\pgfpathlineto{\pgfqpoint{3.453207in}{2.494540in}}%
\pgfpathlineto{\pgfqpoint{3.414248in}{2.472723in}}%
\pgfpathlineto{\pgfqpoint{3.375526in}{2.448451in}}%
\pgfpathlineto{\pgfqpoint{3.339486in}{2.423265in}}%
\pgfpathlineto{\pgfqpoint{3.303753in}{2.395515in}}%
\pgfpathlineto{\pgfqpoint{3.270736in}{2.367105in}}%
\pgfpathlineto{\pgfqpoint{3.238105in}{2.336072in}}%
\pgfpathlineto{\pgfqpoint{3.208215in}{2.304729in}}%
\pgfpathlineto{\pgfqpoint{3.178792in}{2.270783in}}%
\pgfpathlineto{\pgfqpoint{3.149912in}{2.234072in}}%
\pgfpathlineto{\pgfqpoint{3.121667in}{2.194444in}}%
\pgfpathlineto{\pgfqpoint{3.094154in}{2.151762in}}%
\pgfpathlineto{\pgfqpoint{3.069501in}{2.109558in}}%
\pgfpathlineto{\pgfqpoint{3.045652in}{2.064600in}}%
\pgfpathlineto{\pgfqpoint{3.022697in}{2.016864in}}%
\pgfpathlineto{\pgfqpoint{3.000725in}{1.966362in}}%
\pgfpathlineto{\pgfqpoint{2.978128in}{1.908592in}}%
\pgfpathlineto{\pgfqpoint{2.956876in}{1.847761in}}%
\pgfpathlineto{\pgfqpoint{2.937047in}{1.784041in}}%
\pgfpathlineto{\pgfqpoint{2.918699in}{1.717646in}}%
\pgfpathlineto{\pgfqpoint{2.914864in}{1.702694in}}%
\pgfpathmoveto{\pgfqpoint{2.789018in}{1.702694in}}%
\pgfpathlineto{\pgfqpoint{2.805780in}{1.777628in}}%
\pgfpathlineto{\pgfqpoint{2.824392in}{1.851000in}}%
\pgfpathlineto{\pgfqpoint{2.844669in}{1.921602in}}%
\pgfpathlineto{\pgfqpoint{2.864966in}{1.984410in}}%
\pgfpathlineto{\pgfqpoint{2.886629in}{2.044343in}}%
\pgfpathlineto{\pgfqpoint{2.909590in}{2.101223in}}%
\pgfpathlineto{\pgfqpoint{2.933763in}{2.154924in}}%
\pgfpathlineto{\pgfqpoint{2.959047in}{2.205378in}}%
\pgfpathlineto{\pgfqpoint{2.985333in}{2.252577in}}%
\pgfpathlineto{\pgfqpoint{3.012509in}{2.296565in}}%
\pgfpathlineto{\pgfqpoint{3.040466in}{2.337434in}}%
\pgfpathlineto{\pgfqpoint{3.069099in}{2.375310in}}%
\pgfpathlineto{\pgfqpoint{3.098315in}{2.410346in}}%
\pgfpathlineto{\pgfqpoint{3.128027in}{2.442709in}}%
\pgfpathlineto{\pgfqpoint{3.160495in}{2.474773in}}%
\pgfpathlineto{\pgfqpoint{3.193373in}{2.504158in}}%
\pgfpathlineto{\pgfqpoint{3.228977in}{2.532910in}}%
\pgfpathlineto{\pgfqpoint{3.264904in}{2.559076in}}%
\pgfpathlineto{\pgfqpoint{3.303517in}{2.584395in}}%
\pgfpathlineto{\pgfqpoint{3.342376in}{2.607285in}}%
\pgfpathlineto{\pgfqpoint{3.383882in}{2.629197in}}%
\pgfpathlineto{\pgfqpoint{3.428028in}{2.649957in}}%
\pgfpathlineto{\pgfqpoint{3.474803in}{2.669415in}}%
\pgfpathlineto{\pgfqpoint{3.524196in}{2.687448in}}%
\pgfpathlineto{\pgfqpoint{3.576193in}{2.703946in}}%
\pgfpathlineto{\pgfqpoint{3.630782in}{2.718815in}}%
\pgfpathlineto{\pgfqpoint{3.685463in}{2.731438in}}%
\pgfpathlineto{\pgfqpoint{3.742705in}{2.742433in}}%
\pgfpathlineto{\pgfqpoint{3.802499in}{2.751684in}}%
\pgfpathlineto{\pgfqpoint{3.864834in}{2.759052in}}%
\pgfpathlineto{\pgfqpoint{3.927205in}{2.764208in}}%
\pgfpathlineto{\pgfqpoint{3.989600in}{2.767203in}}%
\pgfpathlineto{\pgfqpoint{4.052005in}{2.768037in}}%
\pgfpathlineto{\pgfqpoint{4.114408in}{2.766649in}}%
\pgfpathlineto{\pgfqpoint{4.174301in}{2.763108in}}%
\pgfpathlineto{\pgfqpoint{4.231670in}{2.757515in}}%
\pgfpathlineto{\pgfqpoint{4.286502in}{2.749928in}}%
\pgfpathlineto{\pgfqpoint{4.336293in}{2.740882in}}%
\pgfpathlineto{\pgfqpoint{4.383522in}{2.730110in}}%
\pgfpathlineto{\pgfqpoint{4.428171in}{2.717620in}}%
\pgfpathlineto{\pgfqpoint{4.467749in}{2.704324in}}%
\pgfpathlineto{\pgfqpoint{4.504721in}{2.689635in}}%
\pgfpathlineto{\pgfqpoint{4.539066in}{2.673616in}}%
\pgfpathlineto{\pgfqpoint{4.570761in}{2.656355in}}%
\pgfpathlineto{\pgfqpoint{4.599784in}{2.637981in}}%
\pgfpathlineto{\pgfqpoint{4.626116in}{2.618673in}}%
\pgfpathlineto{\pgfqpoint{4.649740in}{2.598686in}}%
\pgfpathlineto{\pgfqpoint{4.672956in}{2.575945in}}%
\pgfpathlineto{\pgfqpoint{4.693374in}{2.552706in}}%
\pgfpathlineto{\pgfqpoint{4.711011in}{2.529485in}}%
\pgfpathlineto{\pgfqpoint{4.728013in}{2.503514in}}%
\pgfpathlineto{\pgfqpoint{4.744187in}{2.474475in}}%
\pgfpathlineto{\pgfqpoint{4.757476in}{2.446343in}}%
\pgfpathlineto{\pgfqpoint{4.769778in}{2.415573in}}%
\pgfpathlineto{\pgfqpoint{4.780918in}{2.382194in}}%
\pgfpathlineto{\pgfqpoint{4.790754in}{2.346381in}}%
\pgfpathlineto{\pgfqpoint{4.800289in}{2.302873in}}%
\pgfpathlineto{\pgfqpoint{4.807995in}{2.257187in}}%
\pgfpathlineto{\pgfqpoint{4.814614in}{2.203944in}}%
\pgfpathlineto{\pgfqpoint{4.819736in}{2.143350in}}%
\pgfpathlineto{\pgfqpoint{4.823152in}{2.075771in}}%
\pgfpathlineto{\pgfqpoint{4.824898in}{1.995390in}}%
\pgfpathlineto{\pgfqpoint{4.824624in}{1.896314in}}%
\pgfpathlineto{\pgfqpoint{4.821848in}{1.772650in}}%
\pgfpathlineto{\pgfqpoint{4.819507in}{1.702694in}}%
\pgfpathlineto{\pgfqpoint{4.819507in}{1.702694in}}%
\pgfusepath{stroke}%
\end{pgfscope}%
\begin{pgfscope}%
\pgfpathrectangle{\pgfqpoint{2.762966in}{1.712694in}}{\pgfqpoint{2.246634in}{1.362506in}} %
\pgfusepath{clip}%
\pgfsetbuttcap%
\pgfsetroundjoin%
\pgfsetlinewidth{1.003750pt}%
\definecolor{currentstroke}{rgb}{0.000000,0.000000,0.000000}%
\pgfsetstrokecolor{currentstroke}%
\pgfsetdash{{1.000000pt}{1.650000pt}}{0.000000pt}%
\pgfpathmoveto{\pgfqpoint{4.904366in}{1.702694in}}%
\pgfpathlineto{\pgfqpoint{4.936676in}{2.034109in}}%
\pgfpathlineto{\pgfqpoint{4.947584in}{2.161307in}}%
\pgfpathlineto{\pgfqpoint{4.952889in}{2.246992in}}%
\pgfpathlineto{\pgfqpoint{4.955039in}{2.314889in}}%
\pgfpathlineto{\pgfqpoint{4.954781in}{2.370602in}}%
\pgfpathlineto{\pgfqpoint{4.952444in}{2.419781in}}%
\pgfpathlineto{\pgfqpoint{4.948357in}{2.461902in}}%
\pgfpathlineto{\pgfqpoint{4.942086in}{2.502330in}}%
\pgfpathlineto{\pgfqpoint{4.934918in}{2.534930in}}%
\pgfpathlineto{\pgfqpoint{4.926198in}{2.565113in}}%
\pgfpathlineto{\pgfqpoint{4.916135in}{2.592611in}}%
\pgfpathlineto{\pgfqpoint{4.904983in}{2.617395in}}%
\pgfpathlineto{\pgfqpoint{4.890917in}{2.643090in}}%
\pgfpathlineto{\pgfqpoint{4.876007in}{2.665676in}}%
\pgfpathlineto{\pgfqpoint{4.858246in}{2.688303in}}%
\pgfpathlineto{\pgfqpoint{4.839937in}{2.708073in}}%
\pgfpathlineto{\pgfqpoint{4.818881in}{2.727490in}}%
\pgfpathlineto{\pgfqpoint{4.795093in}{2.746248in}}%
\pgfpathlineto{\pgfqpoint{4.768598in}{2.764120in}}%
\pgfpathlineto{\pgfqpoint{4.739422in}{2.780948in}}%
\pgfpathlineto{\pgfqpoint{4.707596in}{2.796654in}}%
\pgfpathlineto{\pgfqpoint{4.673145in}{2.811212in}}%
\pgfpathlineto{\pgfqpoint{4.633618in}{2.825419in}}%
\pgfpathlineto{\pgfqpoint{4.591496in}{2.838196in}}%
\pgfpathlineto{\pgfqpoint{4.544314in}{2.850170in}}%
\pgfpathlineto{\pgfqpoint{4.492078in}{2.861080in}}%
\pgfpathlineto{\pgfqpoint{4.434796in}{2.870711in}}%
\pgfpathlineto{\pgfqpoint{4.372477in}{2.878878in}}%
\pgfpathlineto{\pgfqpoint{4.305130in}{2.885409in}}%
\pgfpathlineto{\pgfqpoint{4.245240in}{2.889168in}}%
\pgfpathlineto{\pgfqpoint{4.170362in}{2.892148in}}%
\pgfpathlineto{\pgfqpoint{4.092979in}{2.893001in}}%
\pgfpathlineto{\pgfqpoint{4.013102in}{2.891609in}}%
\pgfpathlineto{\pgfqpoint{3.933235in}{2.887936in}}%
\pgfpathlineto{\pgfqpoint{3.845906in}{2.881556in}}%
\pgfpathlineto{\pgfqpoint{3.771089in}{2.873500in}}%
\pgfpathlineto{\pgfqpoint{3.698811in}{2.863489in}}%
\pgfpathlineto{\pgfqpoint{3.641516in}{2.854537in}}%
\pgfpathlineto{\pgfqpoint{3.576823in}{2.841626in}}%
\pgfpathlineto{\pgfqpoint{3.514698in}{2.826947in}}%
\pgfpathlineto{\pgfqpoint{3.455157in}{2.810489in}}%
\pgfpathlineto{\pgfqpoint{3.400689in}{2.793100in}}%
\pgfpathlineto{\pgfqpoint{3.388296in}{2.789651in}}%
\pgfpathlineto{\pgfqpoint{3.380849in}{2.787888in}}%
\pgfpathlineto{\pgfqpoint{3.331504in}{2.769075in}}%
\pgfpathlineto{\pgfqpoint{3.265091in}{2.740850in}}%
\pgfpathlineto{\pgfqpoint{3.221079in}{2.718405in}}%
\pgfpathlineto{\pgfqpoint{3.179729in}{2.694754in}}%
\pgfpathlineto{\pgfqpoint{3.141051in}{2.670053in}}%
\pgfpathlineto{\pgfqpoint{3.102661in}{2.642718in}}%
\pgfpathlineto{\pgfqpoint{3.069364in}{2.616404in}}%
\pgfpathlineto{\pgfqpoint{3.034030in}{2.585667in}}%
\pgfpathlineto{\pgfqpoint{3.001502in}{2.553981in}}%
\pgfpathlineto{\pgfqpoint{2.971734in}{2.521929in}}%
\pgfpathlineto{\pgfqpoint{2.942470in}{2.487148in}}%
\pgfpathlineto{\pgfqpoint{2.913795in}{2.449465in}}%
\pgfpathlineto{\pgfqpoint{2.885808in}{2.408725in}}%
\pgfpathlineto{\pgfqpoint{2.875598in}{2.390982in}}%
\pgfpathlineto{\pgfqpoint{2.830759in}{2.312377in}}%
\pgfpathlineto{\pgfqpoint{2.807480in}{2.265620in}}%
\pgfpathlineto{\pgfqpoint{2.785150in}{2.216092in}}%
\pgfpathlineto{\pgfqpoint{2.762112in}{2.159404in}}%
\pgfpathlineto{\pgfqpoint{2.752966in}{2.135107in}}%
\pgfpathlineto{\pgfqpoint{2.752966in}{2.135107in}}%
\pgfusepath{stroke}%
\end{pgfscope}%
\begin{pgfscope}%
\pgfpathrectangle{\pgfqpoint{2.762966in}{1.712694in}}{\pgfqpoint{2.246634in}{1.362506in}} %
\pgfusepath{clip}%
\pgfsetbuttcap%
\pgfsetroundjoin%
\pgfsetlinewidth{1.003750pt}%
\definecolor{currentstroke}{rgb}{0.000000,0.000000,0.000000}%
\pgfsetstrokecolor{currentstroke}%
\pgfsetdash{{1.000000pt}{1.650000pt}}{0.000000pt}%
\pgfpathmoveto{\pgfqpoint{0.000000in}{0.000000in}}%
\pgfusepath{stroke}%
\end{pgfscope}%
\begin{pgfscope}%
\pgfpathrectangle{\pgfqpoint{2.762966in}{1.712694in}}{\pgfqpoint{2.246634in}{1.362506in}} %
\pgfusepath{clip}%
\pgfsetbuttcap%
\pgfsetroundjoin%
\pgfsetlinewidth{1.003750pt}%
\definecolor{currentstroke}{rgb}{0.000000,0.000000,0.000000}%
\pgfsetstrokecolor{currentstroke}%
\pgfsetdash{{1.000000pt}{1.650000pt}}{0.000000pt}%
\pgfpathmoveto{\pgfqpoint{4.903287in}{1.702694in}}%
\pgfpathlineto{\pgfqpoint{4.937753in}{2.058506in}}%
\pgfpathlineto{\pgfqpoint{4.947122in}{2.173845in}}%
\pgfpathlineto{\pgfqpoint{4.951918in}{2.259712in}}%
\pgfpathlineto{\pgfqpoint{4.953458in}{2.327707in}}%
\pgfpathlineto{\pgfqpoint{4.952515in}{2.383371in}}%
\pgfpathlineto{\pgfqpoint{4.949425in}{2.432289in}}%
\pgfpathlineto{\pgfqpoint{4.944572in}{2.473906in}}%
\pgfpathlineto{\pgfqpoint{4.937501in}{2.513511in}}%
\pgfpathlineto{\pgfqpoint{4.929684in}{2.545181in}}%
\pgfpathlineto{\pgfqpoint{4.920398in}{2.574308in}}%
\pgfpathlineto{\pgfqpoint{4.909873in}{2.600723in}}%
\pgfpathlineto{\pgfqpoint{4.896366in}{2.628196in}}%
\pgfpathlineto{\pgfqpoint{4.881868in}{2.652370in}}%
\pgfpathlineto{\pgfqpoint{4.866658in}{2.673693in}}%
\pgfpathlineto{\pgfqpoint{4.848650in}{2.695089in}}%
\pgfpathlineto{\pgfqpoint{4.827834in}{2.716035in}}%
\pgfpathlineto{\pgfqpoint{4.806604in}{2.734261in}}%
\pgfpathlineto{\pgfqpoint{4.782683in}{2.751952in}}%
\pgfpathlineto{\pgfqpoint{4.756086in}{2.768871in}}%
\pgfpathlineto{\pgfqpoint{4.726835in}{2.784873in}}%
\pgfpathlineto{\pgfqpoint{4.694952in}{2.799863in}}%
\pgfpathlineto{\pgfqpoint{4.657990in}{2.814703in}}%
\pgfpathlineto{\pgfqpoint{4.618422in}{2.828179in}}%
\pgfpathlineto{\pgfqpoint{4.573787in}{2.840978in}}%
\pgfpathlineto{\pgfqpoint{4.524091in}{2.852804in}}%
\pgfpathlineto{\pgfqpoint{4.469341in}{2.863420in}}%
\pgfpathlineto{\pgfqpoint{4.409546in}{2.872625in}}%
\pgfpathlineto{\pgfqpoint{4.347211in}{2.879990in}}%
\pgfpathlineto{\pgfqpoint{4.279852in}{2.885727in}}%
\pgfpathlineto{\pgfqpoint{4.242418in}{2.887848in}}%
\pgfpathlineto{\pgfqpoint{4.167540in}{2.890782in}}%
\pgfpathlineto{\pgfqpoint{4.090157in}{2.891578in}}%
\pgfpathlineto{\pgfqpoint{4.012776in}{2.890197in}}%
\pgfpathlineto{\pgfqpoint{3.935405in}{2.886661in}}%
\pgfpathlineto{\pgfqpoint{3.848075in}{2.880297in}}%
\pgfpathlineto{\pgfqpoint{3.773258in}{2.872278in}}%
\pgfpathlineto{\pgfqpoint{3.700979in}{2.862297in}}%
\pgfpathlineto{\pgfqpoint{3.616305in}{2.848083in}}%
\pgfpathlineto{\pgfqpoint{3.554130in}{2.834784in}}%
\pgfpathlineto{\pgfqpoint{3.494524in}{2.819831in}}%
\pgfpathlineto{\pgfqpoint{3.437503in}{2.803217in}}%
\pgfpathlineto{\pgfqpoint{3.358330in}{2.777312in}}%
\pgfpathlineto{\pgfqpoint{3.289336in}{2.749787in}}%
\pgfpathlineto{\pgfqpoint{3.245227in}{2.728557in}}%
\pgfpathlineto{\pgfqpoint{3.201326in}{2.704810in}}%
\pgfpathlineto{\pgfqpoint{3.160109in}{2.679773in}}%
\pgfpathlineto{\pgfqpoint{3.121588in}{2.653605in}}%
\pgfpathlineto{\pgfqpoint{3.085765in}{2.626578in}}%
\pgfpathlineto{\pgfqpoint{3.052652in}{2.598863in}}%
\pgfpathlineto{\pgfqpoint{3.017551in}{2.566541in}}%
\pgfpathlineto{\pgfqpoint{2.987580in}{2.535683in}}%
\pgfpathlineto{\pgfqpoint{2.958072in}{2.502188in}}%
\pgfpathlineto{\pgfqpoint{2.929113in}{2.465867in}}%
\pgfpathlineto{\pgfqpoint{2.900795in}{2.426559in}}%
\pgfpathlineto{\pgfqpoint{2.879641in}{2.393735in}}%
\pgfpathlineto{\pgfqpoint{2.873697in}{2.382478in}}%
\pgfpathlineto{\pgfqpoint{2.849017in}{2.340371in}}%
\pgfpathlineto{\pgfqpoint{2.825150in}{2.295471in}}%
\pgfpathlineto{\pgfqpoint{2.802182in}{2.247775in}}%
\pgfpathlineto{\pgfqpoint{2.778404in}{2.193000in}}%
\pgfpathlineto{\pgfqpoint{2.755852in}{2.135119in}}%
\pgfpathlineto{\pgfqpoint{2.752966in}{2.127238in}}%
\pgfpathlineto{\pgfqpoint{2.752966in}{2.127238in}}%
\pgfusepath{stroke}%
\end{pgfscope}%
\begin{pgfscope}%
\pgfpathrectangle{\pgfqpoint{2.762966in}{1.712694in}}{\pgfqpoint{2.246634in}{1.362506in}} %
\pgfusepath{clip}%
\pgfsetbuttcap%
\pgfsetroundjoin%
\pgfsetlinewidth{1.003750pt}%
\definecolor{currentstroke}{rgb}{0.000000,0.000000,0.000000}%
\pgfsetstrokecolor{currentstroke}%
\pgfsetdash{{1.000000pt}{1.650000pt}}{0.000000pt}%
\pgfpathmoveto{\pgfqpoint{0.000000in}{0.000000in}}%
\pgfusepath{stroke}%
\end{pgfscope}%
\begin{pgfscope}%
\pgfsetrectcap%
\pgfsetmiterjoin%
\pgfsetlinewidth{0.803000pt}%
\definecolor{currentstroke}{rgb}{0.000000,0.000000,0.000000}%
\pgfsetstrokecolor{currentstroke}%
\pgfsetdash{}{0pt}%
\pgfpathmoveto{\pgfqpoint{2.762966in}{1.712694in}}%
\pgfpathlineto{\pgfqpoint{2.762966in}{3.075200in}}%
\pgfusepath{stroke}%
\end{pgfscope}%
\begin{pgfscope}%
\pgfsetrectcap%
\pgfsetmiterjoin%
\pgfsetlinewidth{0.803000pt}%
\definecolor{currentstroke}{rgb}{0.000000,0.000000,0.000000}%
\pgfsetstrokecolor{currentstroke}%
\pgfsetdash{}{0pt}%
\pgfpathmoveto{\pgfqpoint{5.009600in}{1.712694in}}%
\pgfpathlineto{\pgfqpoint{5.009600in}{3.075200in}}%
\pgfusepath{stroke}%
\end{pgfscope}%
\begin{pgfscope}%
\pgfsetrectcap%
\pgfsetmiterjoin%
\pgfsetlinewidth{0.803000pt}%
\definecolor{currentstroke}{rgb}{0.000000,0.000000,0.000000}%
\pgfsetstrokecolor{currentstroke}%
\pgfsetdash{}{0pt}%
\pgfpathmoveto{\pgfqpoint{2.762966in}{1.712694in}}%
\pgfpathlineto{\pgfqpoint{5.009600in}{1.712694in}}%
\pgfusepath{stroke}%
\end{pgfscope}%
\begin{pgfscope}%
\pgfsetrectcap%
\pgfsetmiterjoin%
\pgfsetlinewidth{0.803000pt}%
\definecolor{currentstroke}{rgb}{0.000000,0.000000,0.000000}%
\pgfsetstrokecolor{currentstroke}%
\pgfsetdash{}{0pt}%
\pgfpathmoveto{\pgfqpoint{2.762966in}{3.075200in}}%
\pgfpathlineto{\pgfqpoint{5.009600in}{3.075200in}}%
\pgfusepath{stroke}%
\end{pgfscope}%
\begin{pgfscope}%
\pgfsetbuttcap%
\pgfsetmiterjoin%
\definecolor{currentfill}{rgb}{1.000000,1.000000,1.000000}%
\pgfsetfillcolor{currentfill}%
\pgfsetfillopacity{0.800000}%
\pgfsetlinewidth{1.003750pt}%
\definecolor{currentstroke}{rgb}{0.800000,0.800000,0.800000}%
\pgfsetstrokecolor{currentstroke}%
\pgfsetstrokeopacity{0.800000}%
\pgfsetdash{}{0pt}%
\pgfpathmoveto{\pgfqpoint{3.486334in}{1.782138in}}%
\pgfpathlineto{\pgfqpoint{4.286232in}{1.782138in}}%
\pgfpathquadraticcurveto{\pgfqpoint{4.314010in}{1.782138in}}{\pgfqpoint{4.314010in}{1.809916in}}%
\pgfpathlineto{\pgfqpoint{4.314010in}{2.194916in}}%
\pgfpathquadraticcurveto{\pgfqpoint{4.314010in}{2.222694in}}{\pgfqpoint{4.286232in}{2.222694in}}%
\pgfpathlineto{\pgfqpoint{3.486334in}{2.222694in}}%
\pgfpathquadraticcurveto{\pgfqpoint{3.458556in}{2.222694in}}{\pgfqpoint{3.458556in}{2.194916in}}%
\pgfpathlineto{\pgfqpoint{3.458556in}{1.809916in}}%
\pgfpathquadraticcurveto{\pgfqpoint{3.458556in}{1.782138in}}{\pgfqpoint{3.486334in}{1.782138in}}%
\pgfpathclose%
\pgfusepath{stroke,fill}%
\end{pgfscope}%
\begin{pgfscope}%
\pgfsetrectcap%
\pgfsetroundjoin%
\pgfsetlinewidth{1.003750pt}%
\definecolor{currentstroke}{rgb}{1.000000,0.388235,0.278431}%
\pgfsetstrokecolor{currentstroke}%
\pgfsetdash{}{0pt}%
\pgfpathmoveto{\pgfqpoint{3.514112in}{2.118527in}}%
\pgfpathlineto{\pgfqpoint{3.583556in}{2.118527in}}%
\pgfusepath{stroke}%
\end{pgfscope}%
\begin{pgfscope}%
\pgftext[x=3.694667in,y=2.069916in,left,base]{\rmfamily\fontsize{10.000000}{12.000000}\selectfont \(\displaystyle \textnormal{tol}=10^{-6}\)}%
\end{pgfscope}%
\begin{pgfscope}%
\pgfsetbuttcap%
\pgfsetroundjoin%
\pgfsetlinewidth{1.003750pt}%
\definecolor{currentstroke}{rgb}{0.000000,0.000000,0.000000}%
\pgfsetstrokecolor{currentstroke}%
\pgfsetdash{{1.000000pt}{1.650000pt}}{0.000000pt}%
\pgfpathmoveto{\pgfqpoint{3.514112in}{1.919083in}}%
\pgfpathlineto{\pgfqpoint{3.583556in}{1.919083in}}%
\pgfusepath{stroke}%
\end{pgfscope}%
\begin{pgfscope}%
\pgftext[x=3.694667in,y=1.870472in,left,base]{\rmfamily\fontsize{10.000000}{12.000000}\selectfont Reference}%
\end{pgfscope}%
\end{pgfpicture}%
\makeatother%
\endgroup%

    \caption[FLYTT MEG!]{FLYTT MEG! Testfig}
    \label{fig:testfigerroroscillations}
\end{figure}
