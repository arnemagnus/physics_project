Lagrange-koherente strukturer (heretter forkortet til LKSer) kan beskrives som
"landskap" i flerdimensjonale rom, som utøver stor innflytelse på
strømningsmønstre i dynamiske system. Denne typen strukturer kan benyttes til
å på forenklet vis kunne forutsi fremtidige tilstander i komplekse system, som
for eksempel ikkelineære mangepartikkelsystem, over et gitt tidsintervall,
sammenlignet med å utføre simuleringer med et økende antall partikler ---
hvilket er den tradisjonelle tilnærmingen. Fra deres underliggende
variasjonsteori, kjennetegnes hyperbolske LKSer som de lokalt sterkest
frastøtende eller tiltrekkende materialoverflatene i systemet, som utgjør
skjelettet av observable følgepartiklers baner.

Dette prosjektet omhandler hvordan den numeriske identifikasjonen av LKSer
avhenger av den numeriske integrasjonsmetoden som blir benyttet til å beregne
dem. Spesifikt ble åtte Runge-Kutta-teknikker undersøkt, hvorav fire
tradisjonelle enkeltstegsmetoder, og fire sammensatte,
dynamisk steglengde-metoder. En numerisk løsning funnet via en høyere
ordens sammensatt metode med svært liten numerisk toleranse -- to
størrelsesordener mindre enn det minste toleransenivået som ble benyttet for
øvrig -- ble brukt som sammenligningsgrunnlag for de ulike metodenes ytelse.
LKSer ble beregnet og sammenlignet med utgangspunkt i et analytisk kjent
hastighetsfelt, hvis LKSer har blitt dokumentert i litteraturen.

Det viser seg at LKSene i eksempelsystemet er nokså robuste, numerisk sett,
i alle fall for de benyttede parameterverdiene. Av den grunn var det ingen
av de numeriske integrasjonsmetodene som sto frem som den absolutt mest
velegnede. Likevel, basert på de de beregnede feilene i flytdiagrammene,
tilrådes bruk av høyere ordens numeriske integratorer for generiske
strømningssystem, fordi de uten unntak resulterer i mer effektive beregninger,
som er mindre utsatt for numeriske avrundingsfeil. Videre foreligger ingen
åpenbar grunn til å forvente at LKSene for et generelt strømningssystem er like
robuste som de viste seg å være for modellsystemet undersøkt her.

Merk at den numeriske implementasjonen av en av eksistensbetingelsene for
LKSer, som stammer fra deres variasjonsteori, til dags dato ikke har blitt
fullstendig beskrevet i litteraturen. Dette er en direkte konsekvens
av at LKSer fortsatt er å regne som et forholdsvis nytt fysisk fenomen.
Tilnærmingsmetoden som ble benyttet som del av dette prosjektet, er basert
på nøye vurderinger av det underliggende systemets egenskaper, hvilket
ikke nødvendigvis er en praktisk hensiktsmessig fremgangsmåte for generiske
system, avhengig av de involverte skalaene. Med andre ord er det for øyeblikket
stort rom for forskning innenfor denne grenen av fysikk.

