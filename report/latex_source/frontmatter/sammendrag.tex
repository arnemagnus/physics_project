Et generisk strømningssystem kan beskrives som et system hvis tilstand
avhenger av flyt av energi-, material- eller informasjonsstrømmer. Tradisjonelle
eksempler på slike systemer er transport av trykk, temperatur og masse i
stasjonære fluidstrømninger, og ladningstransport som forårsakes av elektriske
strømmer. Interessante sider ved andre fenomen, som populasjonsvekst og
trafikkmønstre, kan avdekkes ved å betrakte dem som strømningssystemer.

Lagrange-koherente strukturer (heretter forkortet til LKSer) kan beskrives som
<<landskap>> i flerdimensjonale rom, som utøver stor innflytelse på
strømningsmønstre i dynamiske system. Denne typen strukturer kan benyttes til
å på forenklet vis kunne forutsi fremtidige tilstander i komplekse system, som
for eksempel ikkelineære mangepartikkelsystem, over et gitt tidsintervall,
sammenlignet med å utføre simuleringer på modeller med økende grad av
oppløsning både i rom og tid -- hvilket er den tradisjonelle tilnærmingen. Fra
deres underliggende variasjonsteori, kjennetegnes hyperbolske LKSer som de
lokalt sterkest frastøtende eller tiltrekkende materialoverflatene i systemet.
Noe forenklet kan denne typen overflater betraktes som en generalisering av
banene det underliggende transportfenomenet skaper. Hyperbolske LKSer utgjør
skjelettet av observable strømningsmønstre.

Dette prosjektet omhandler hvordan den numeriske identifikasjonen av LKSer
avhenger av den numeriske integrasjonsmetoden som blir benyttet til å beregne
dem. LKSer ble beregnet og sammenlignet ut fra analytisk
kjent hastighetsfelt, hvis LKSer er gjort kjent i litteraturen. Spesifikt
ble åtte Runge-Kutta-metoder undersøkt, hvorav fire tradisjonelle
enkeltstegsmetoder, og fire sammensatte, dynamisk steglengde-metoder. En
løsning funnet via en høyere ordens sammensatt metode med svært liten
numerisk toleranse -- to størrelsesordener mindre enn det minste toleransenivået
som ble benyttet for
øvrig -- og dermed veldig liten forventet numerisk feil, ble brukt som
sammenligningsgrunnlag for de ulike metodenes ytelse.

Det viser seg at LKSene i eksempelsystemet er nokså robuste, numerisk sett,
i alle fall for de benyttede parameterverdiene. Av den grunn sto ingen av de
numeriske integrasjonsmetodene frem som den absolutt mest
velegnede. Likevel, basert på de de beregnede feilene i flytdiagrammene,
tilrådes bruk av høyere ordens numeriske integratorer generelt, fordi de uten
unntak resulterer i mer effektive beregninger,
som er mindre utsatt for numeriske avrundingsfeil. Videre foreligger ingen
åpenbar grunn til å forvente at LKSene for et generelt strømningssystem er like
robuste som de viste seg å være for modellsystemet undersøkt her.

Merk at den numeriske implementasjonen av en av eksistensbetingelsene for
LKSer, som stammer fra deres variasjonsteori, til dags dato (så vidt
undertegnede vet) ikke har blitt entydig beskrevet i litteraturen. Dette er
et eksempel på hvordan studiet av LKSer fortsatt er å regne som en
forholdsvis ny gren av fysikken. Tilnærmingsmetoden som ble benyttet i
dette prosjektet, er basert på nøye vurderinger av det underliggende systemets
egenskaper, hvilket ikke nødvendigvis er en praktisk hensiktsmessig
fremgangsmåte for generiske system, avhengig av de involverte skalaene. Med
andre ord er det for øyeblikket stort rom for forskning innen feltet.
