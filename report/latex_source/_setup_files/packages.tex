% ---------------- %
% Text-related packages
% ---------------- %

% Margins
\usepackage[margin=2.5cm]{geometry}

% Typeset UTF-8 symbols:
\usepackage[T1]{fontenc}
\usepackage[utf8]{inputenc}

% Manage culturally-determined typographical, and other, rules:
\usepackage[main=english,norsk]{babel}

% In order to load fonts:
\usepackage{fontspec}

% Include author's affiliation
\usepackage[]{authblk}

% Typeset relevant quotations or sayings as epigraphs:
\usepackage[]{epigraph}

% Simple syntax for superscript 1st, 2nd, 3rd etc: \nth{#}
\usepackage[super]{nth}

% Paragraphs separated by blank lines rather than indentation
\usepackage[parfill]{parskip}

% Customize title/heading styles
\usepackage[]{titlesec}

% Control ToC, LoF, LoT etc.
\usepackage[]{tocloft}

% Verbatim
\usepackage[]{verbatim}

% Frame any object in the document
\usepackage[]{framed}

% Break lines in citations which do not fit on a single line
\usepackage[]{breakcites}

% Insert placeholder text
\usepackage[]{lipsum}

% Include the possibility of colored text
\usepackage[]{color}

%
% Float-related packages
%

% Interface for floating objects
\usepackage[]{float}

% Enhanced graphics support
\usepackage[]{graphicx}

% Wrap text around figure
\usepackage[]{wrapfig}

% Format float captions any which way you want.
% My preference:    No more than 90 % of the overall textwidth
%                   Slightly smaller fontsize than the text otherwise
%                   Label typeset in bold
%                   Separate label from caption with colon
\usepackage[%font={\fontsize{11}{13}\selectfont},%
width=0.9\textwidth,%
labelfont=bf,%
labelsep=colon%
]{caption}

% Include entire (external) PDF pages in document
\usepackage[]{pdfpages}



%
% Reference-related packages
%

% Use biblatex for handling refernces
\usepackage[backend=biber,%
style=authoryear,%
language=british,%
dashed=false,%
url=false,%
doi=false%
]{biblatex}

% Clever referrals to objects (figures, tables, etc.)
\usepackage[]{cleveref}

% Make all referrals hyperlinks (within the document)
\usepackage[]{hyperref}

% Bookmark organization for the hyperref package
\usepackage[]{bookmark}

%
% Mathematical functions
%

% Necessary for most kinds of mathematical typesetting:
\usepackage[]{amsmath}

% Package containing useful macros for typsetting vector calculus, linear
% algebra etc.
\usepackage[]{physics}

% Typeset mathematical symbols in bold, useful when working with vectors,
% tensors, etc.
\usepackage[]{bm}


