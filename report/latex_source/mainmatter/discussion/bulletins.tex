%\begin{framed}
%    \begin{itemize}
%        \item \sout{Approach does produce a consistent LCS picture for all numerical integrators considered, provided sensible
%            integration steps or tolerance levels are chosen. This indicates that the calculation of this type
%        of transport barrier is not particularly sensitive to the integration method.}
%    \item \sout{Most crucial part: Compute advection correctly. The error of the computed strain eigenvalues scales
%                like the advection error, while the error in the strain eigendirections is negligible for all time
%                steps. The computed LCSs seem to exhibit sensitive dependence on the calculation of the eigenvalues,
%            which is not particularly surprising, considering the LCS conditions of eq. 3.13 }
%                \begin{itemize}
%                    \item \sout{The precision of the strain eigendirections is likely a direct consequence of the
%                        auxiliary tracers in their computation.}
%                \end{itemize}
%            \item \sout{The number of points in the $\mathcal{U}$ domain is different to the one obtained by
%                Haller et  al, by approximately $3\%$. This, however, is likely related to how the
%            grid of tracers was set up. In their paper, Farazmand and Haller do not provide details.}
%        \item \sout{Notably, the reference LCS as shown in figure (.) consists of seven different strainline
%                segments, unlike the structure found by Haller et al., which is claimed to be a single
%            coherent strainline.}
%
%        \item \sout{The time lenghts or tolerance levels should be chosen based on the system under consideration.
%            In particular, an individual time step should not be so large that too much detail in the local,
%        instantaneous velocity field is glossed over.}
%    \item \sout{Possible approach in order to refine the computations: Use fewer `base' tracers, and an \emph{adaptive multigrid method}. Main con: May lead to inconsistent centered differencing when approaching
%        the Jacobian etc. of the flow map}
%        \item \sout{Regarding the completely different domains obtained via the embedded methods for the highest tolerance level:
%            Check if this corresponds th the error in the flow map larger than some threshold. If (likely) so, this is
%        further evidence that one should exhert great effort in computing the flow map correctly.}
%    \item \sout{In terms of application to discrete velocity data sets, where both spatial and temporal interpolation
%                    may be necessary, the interaction between interpolation and integration scheme is likely to
%                    have a great overall impact. In particular, the underlying interpolation scheme sets a lower accuracy
%                    bound for the entire advection process, practically enforcing restrictions on the numerical step
%                    lengths or tolerance levels which can be considered sensible. This is also based on the considered
%                velocity field.}
%            \item \sout{Regarding the incompressibility of the velocity field}
%                    \begin{itemize}
%                        \item \sout{Incompressibility property is conserved until approximately $t=5$ units. Generally can't expect
%                            this property to hold numerically over time.}
%                        \item \sout{There is no assurance that neighboring tracers will remain nearby after the advection,
%                                leaving the finite difference approximation of the local strain and stretch invalid. This
%                                issue is expected to be most prominent in regions of high repulsion, which are precisely
%                            where accuracy is imperative.}
%                    \end{itemize}
%                \item \sout{Parameter choices}
%            \begin{itemize}
%                \item \sout{Numerical step lengths and tolerance levels}
%                \item \sout{The use of RK4 for all strainline iterations}
%                    \begin{itemize}
%                        \item \sout{Alternative: Use a high order adaptive step method, paying close attention to the strainline curvature}
%                    \end{itemize}
%                \item \sout{Represent tracer positions by means of higher precision floating-point numbers}
%                        \begin{itemize}
%                            \item \sout{Yields more accurate finite difference approximations}
%                            \item \sout{Requires more memory, i.e., less tracers can be advected, leading to
%                                        a more granular flow map representation. Perhaps most relevant for
%                                        well-behaved velocity fields, for which the LCS approach is not really
%                                        practically relevant --- why not simply advect a few particles and see where they
%                                    end up?}
%                        \end{itemize}
%            \end{itemize}
%        \item \sout{The identification process of local strain maximizing strainlines}
%            \begin{itemize}
%                \item \sout{The cutting of strainline tails}
%                \item \sout{Lines in $\mathcal{L}$ $\rightarrow$ not necessarily a robust approach for general flows}
%                \item \sout{Alternative: Clustering algorithm}
%                \item \sout{Approach where both the strainline lengths and avg $\lambda_{2}$ are taken into consideration,
%                    i.e., selecting a longer line with slighly smaller avg $\lambda_{2}$ over a shorter line with larger.}
%            \end{itemize}
%        \item \sout{The special linear interpolation used for the eigenvectors}
%            \begin{itemize}
%                \item \sout{Possible alternative: Higher order interpolation}
%            \end{itemize}
%        \item \sout{Chose mean error, rather than max, because:}
%                \begin{itemize}
%                    \item \sout{In terms of flow map: Error should be limited from above, by the domain extent, as the normal component
%                        of the velocity at the boundaries is zero}
%                    \item \sout{Generally no way of telling where the maximum error occurs}
%                \end{itemize}
%            \item \sout{Regarding the use of $\rmsd$:}
%                    \begin{itemize}
%                        \item \sout{A practical application of empirical standard deviation}
%                        \item \sout{The distribution of errors across the domain is unknown. From the Central Limit thm., however,
%                            the mean error is nearly distributed as a Gaussian.}
%                    \end{itemize}
%    \end{itemize}
%\end{framed}
