\section{General remarks}
\label{sec:general_remarks}

First off, the computed reference LCS, shown in figure
\ref{fig:referencelcs}, is made up of \emph{seven} different strainline segments.
The LCS presented in the article by \textcite{farazmand2012computing} is claimed
to consist of a \emph{single} strainline segment. Comparing the two curves
visually, however, indicates that the resulting LCSs are similar. Likewise, the
domain $\mathcal{U}_{0}$, shown in figure~\ref{fig:u0_domain}, strongly
resembles the one found by \citeauthor{farazmand2012computing}. Nevertheless,
the total number of points in the domain computed here is approximately two
percent larger than what \citeauthor{farazmand2012computing} found. These
discrepancies could originate from different conventions in terms of generating
the grid of tracers. Notably, \citeauthor{farazmand2012computing} fail to
provide a description of their approach.

Overall, the use of a variational framework for computing LCSs appears to
produce robust and consistent LCS curves for all of the numerical integration
schemes considered here, subject to usage of a sufficiently small numerical
step length or tolerance level. This indicates that calculations of this kind
are not particularly sensitive to integration method. Note that the
$\mathcal{U}_{0}$ domains obtained by means of the adaptive stepsize methods
for $\textnormal{tol}=10^{-1}$, shown in figure~\ref{fig:u0_dom_errs}, differ
greatly from the \emph{correct} domain, which correlates well with the computed
$\rmsd$ of the flow maps, shown in figure~\ref{fig:flowmap_err_both}. In
particular, the error in the flow map for the aforementioned tolerance level is
of order $1$ for all of the embedded methods. This error is comparable to the
extent of the numerical domain, that is,
$[0,\hspace{0.5ex}2]\times[0,\hspace{0.5ex}1]$. Naturally, this leads to a
drastically different system.

The above observation implies that the most crucial part of the computation is
advecting the tracers accurately. As mentioned in
\cref{sub:computed_deviations_in_the_strain_eigenvalues_and_vectors}, the error
in the computed strain eigenvalues scales like the error in the computed flow
map. This is to be expected, as the eigenvalues are essentially found by
applying finite differences to the flow map. The eigenvalues are crucial in
terms of identifying LCSs, as can be seen in the numerical reformulation of the
LCS existence theorem, which is given in~\cref{eq:numericalexistence}.
The error in the computed strain eigendirections, however, is consistently
negligible. This is most likely due to them being computed based on the %denser
\emph{auxiliary} set of tracers, which per construction results in more accurate
finite differences.

For the considered double gyre system, there appears to be a lower threshold
in terms of the required advection accuracy, beneath which the computed LCS
curves do not become more precise. This effect is apparent
from inspection of
\cref{fig:lcs_rmsd_fp_nn_fixed,fig:lcs_rmsd_fn_nn_fixed,%
fig:lcs_rmsd_fp_nn_both,fig:lcs_rmsd_fn_nn_both}, where the error of strainline
components identified as LCS constituents flattens abruptly. Notably, this
occurs for larger numerical step lengths or tolerance levels, respectively,
than the corresponding turning points for the error in the flow maps.
For the double gyre system considered here, it appears that this advection
accuracy threshold is of the order $10^{-6}$--$10^{-7}$, which follows from
comparing~\cref{fig:lcs_rmsd_fp_nn_fixed,fig:lcs_rmsd_fn_nn_fixed,%
fig:lcs_rmsd_fp_nn_both,fig:lcs_rmsd_fn_nn_both} to
\cref{fig:flowmap_err_fixed,fig:flowmap_err_both}. In particular, for flow maps
with $\rmsd$ of $10^{-7}$ or lower, the $\rmsd$ of the LCS curves appears to
not decrease further as the flow map precision increases.

However, because a similar flattening of the $\rmsd$ for the strain eigenvalues
and eigenvectors is not apparent in
\cref{fig:lmbd2_err_fixed,fig:lmbd2_err_both,fig:xi2_err_fixed,fig:xi2_err_both},
one may infer that this threshold is likely only valid for the LCS curves
of this particular velocity field --- that is, the system given by
\cref{eq:doublegyre,eq:doublegyrefuns,eq:doublegyreparams}%
%, for which LCSs are
%found by means of the procedure described in
%\cref{sec:advecting_a_set_of_initial_conditions,%
%    sec:calculating_the_cauchy_green_strain_tensor,%
%sec:identifying_lcs_candidates_numerically}
--- which
appear quite robust. The same flow map accuracy threshold probably does not
suffice for other, more volatile flow systems. Investigating
this further for a wider range of systems could result in valuable insight.
Should such a threshold be valid in general, it would naturally be of great
significance when investigating generic transport systems by means of a
similar variational LCS approach. Admittedly, there is no apparent reason why
this should be the case.

The double gyre model considered in this project is obviously not representative
of generic systems, in terms of the exact numerical step lengths or tolerance
levels necessary in order to obtain correct LCSs with a certain
degree of confidence. It does, however, indicate that these quantities should
be chosen based on the considered system. For a fixed stepsize integration
scheme, any single integration time step should not be so large that \emph{too}
much detail in the local and instantaneous velocity field is glossed over.
Similar logic applies when adaptive stepsize methods are used, although it
may be more difficult to enforce, depending on how the step length
update is implemented. One possibility in terms of choosing the time step, is to
find a characteristic velocity for the system, and choose the time step small
enough so that a tracer moving with the characteristic velocity never traverses
a distance greater than the grid spacing, when moving from one time level to
the next.

When computing transport based on discrete data sets, such as snapshots of the
instantaneous velocity fields in oceanic currents, spatial and temporal
interpolation becomes necessary. Together with the inherent precision of the
model data, the choice of interpolation scheme(s) sets a lower bound
in terms of the accucacy with which tracers can be advected. For such cases,
the interaction between the integration and interpolation schemes could
be critical --- both in terms of computation time and memory requirements,
aside from the numerical precision. Independently of the scales at which
well-resolved LCS information is sought in this kind of system, the
aforementioned effects warrant further investigation.
