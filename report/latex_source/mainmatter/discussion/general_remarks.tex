\section{Remarks on the computed LCSs}
\label{sec:general_remarks}

Overall, the use of a variational framework for computing LCSs appears to
produce robust and consistent LCS curves for all of the numerical integration
schemes considered here, subject to usage of a sufficiently small numerical
step length or tolerance level. This is apparent from the LCS curves shown
in~\cref{fig:lcs_euler,fig:lcs_rk2,fig:lcs_rk3,fig:lcs_rk4,fig:lcs_rkbs32,%
fig:lcs_rkbs54,fig:lcs_rkdp54,fig:lcs_rkdp87} conforming visually with the
reference LCS, and the computed errors in the various LCS curves --- shown
in~\cref{fig:lcs_rmsd_fp_nn_both,fig:lcs_rmsd_fn_nn_both} --- being small.
Furthermore, this indicates that calculations of this kind
are not particularly sensitive to integration method. Note that the
$\mathcal{U}_{0}$ domains obtained by means of the adaptive stepsize methods
for $\textnormal{tol}=10^{-1}$, shown in figure~\ref{fig:u0_dom_errs}, differ
greatly from the reference domain (see~\cref{fig:u0_domain}), which correlates
well with the computed $\rmsd$ of the flow maps, shown in
figure~\ref{fig:flowmap_err_both}. In particular, the error in the flow map for
the aforementioned tolerance level is of order $1$ for all of the embedded
methods. This error is comparable to the extent of the numerical domain, that
is, $[0,\hspace{0.5ex}2]\times[0,\hspace{0.5ex}1]$. Naturally, this leads to a
drastically different system.

The above observation implies that the most crucial part of the computation is
advecting the tracers accurately. As mentioned in
\cref{sub:computed_deviations_in_the_strain_eigenvalues_and_vectors}, the errors
in the computed strain eigenvalues and -vectors scale like the error in the
computed flow map. This is to be expected, as the eigenvalues and -vectors are
essentially found by applying finite differences to the flow map. The
eigenvalues are crucial in terms of identifying LCSs, as can be seen in the
numerical reformulation of the LCS existence theorem, which is given in
\cref{eq:numericalexistence}. The error in the computed strain eigenvectors, is
consistently two to three orders of magnitude smaller than the error in the
eigenvalues. This is most likely due to them being computed based on the %denser
\emph{auxiliary} set of tracers, which per construction results in more accurate
finite differences.

From inspection of~\cref{fig:lcs_rmsd_fp_nn_fixed,fig:lcs_rmsd_fn_nn_fixed,%
fig:lcs_rmsd_fp_nn_both,fig:lcs_rmsd_fn_nn_both}, it becomes clear that the
error of the strainline components identified as LCS constituents, for some
configurations with small step lengths or tolerance levels, is dominated by
a seemingly constant contribution of the order $10^{-4}$. For instance,
for the Dormand-Prince 8(7) method,
\cref{fig:lcs_rmsd_fp_nn_both,fig:lcs_rmsd_fn_nn_both} show that for the three
lowest tolerance levels --- $\textnormal{tol}=10^{-8}$, $10^{-9}$ and
$10^{-10}$, respectively --- the $\rmsd$ is of order $10^{-4}$, whereas it is
considerably smaller --- of order $10^{-7}$ --- for $\textnormal{tol}=10^{-7}$.
This is unexpected, as we expect the error to decrease when the tolerance level
is lowered.  Notably, this occurs for larger step lengths or tolerance levels,
respectively, than the corresponding turning points for the error in the flow
maps (i.e., the points at which the error in the flow maps increases with
decreasing step length or tolerance level). However, close inspection of
\cref{fig:lcs_rk2,fig:lcs_rk3,fig:lcs_rk4,fig:lcs_rkbs32,fig:lcs_rkbs54,%
fig:lcs_rkdp54,fig:lcs_rkdp87} reveals that, for the numerical step lengths
and tolerance levels which correspond to the same level of error in the LCS
curves (see~\cref{fig:lcs_rmsd_fp_nn_fixed,fig:lcs_rmsd_fn_nn_fixed,%
fig:lcs_rmsd_fp_nn_both,fig:lcs_rmsd_fn_nn_both}), the computed LCS
approximations are made up of seven different strainline segments each. This
is unlike the reference LCS, which, as previously mentioned, consists
of \emph{eight} strainline segments.
\clearpage
\Cref{fig:lcserroroscillations} shows the computed LCS approximations together
with the reference LCS, for the Dormand-Prince 8(7) method with tolerance
levels $\textnormal{tol}=10^{-5}$ through to $\textnormal{tol}=10^{-8}$ ---
the tolerance levels for which oscillations are visible in the computed errors
of the LCS curves (see~\cref{fig:lcs_rmsd_fp_nn_both,fig:lcs_rmsd_fn_nn_both})
--- for a subset of the computational domain $\mathcal{U}$. Notice in
particular that the second reference LCS curve segment (counting from the top
and downwards) is present in the approximations for $\textnormal{tol}=10^{-5}$
and $10^{-7}$, but \emph{not} for $\textnormal{tol}=10^{-6}$ and $10^{-8}$.
However, the two topmost reference LCS curve segments shown in the figure
are sufficiently close for this error not to be identified as a false negative.
Thus, when computing the LCS $\rmsd$, the offset between the LCS curve segment
corresponding to the topmost one shown in the figure, and the second topmost
reference LCS curve likely dominates the other contributions. Furthermore, the
fact that this offset is nearly constant explains the apparently identical LCS
$\rmsd$ in the cases where the computed LCS approximation consists of seven
strainline segments (as shown in
\cref{fig:lcs_rmsd_fp_nn_fixed,fig:lcs_rmsd_fn_nn_fixed,%
fig:lcs_rmsd_fp_nn_both,fig:lcs_rmsd_fn_nn_both}). For all of the numerical step
lengths and tolerance levels which result in the same level of LCS $\rmsd$, the
situation is the same as for the  tolerance levels $\textnormal{tol}=10^{-6}$
and $\textnormal{tol}=10^{-8}$, shown in~\cref{fig:lcserroroscillations}. Figures
showing the same level of  detail for the remaining integration methods have
thus been omitted for brevity. Note that the two reference LCS curve
segments in question nearly overlap, and their $\overline{\lambda}_{2}$ differ
by less than 5\%; thus, the absence of the second topmost reference LCS curve
segment would likely not have severe consequences for predictions regarding
the overall flow in the system.

\begin{figure}[htpb]
    \centering
    %% Creator: Matplotlib, PGF backend
%%
%% To include the figure in your LaTeX document, write
%%   \input{<filename>.pgf}
%%
%% Make sure the required packages are loaded in your preamble
%%   \usepackage{pgf}
%%
%% Figures using additional raster images can only be included by \input if
%% they are in the same directory as the main LaTeX file. For loading figures
%% from other directories you can use the `import` package
%%   \usepackage{import}
%% and then include the figures with
%%   \import{<path to file>}{<filename>.pgf}
%%
%% Matplotlib used the following preamble
%%   \usepackage[utf8x]{inputenc}
%%   \usepackage[T1]{fontenc}
%%   \usepackage[]{libertine}\usepackage[libertine]{newtxmath}
%%
\begingroup%
\makeatletter%
\begin{pgfpicture}%
\pgfpathrectangle{\pgfpointorigin}{\pgfqpoint{5.050000in}{3.100000in}}%
\pgfusepath{use as bounding box, clip}%
\begin{pgfscope}%
\pgfsetbuttcap%
\pgfsetmiterjoin%
\definecolor{currentfill}{rgb}{1.000000,1.000000,1.000000}%
\pgfsetfillcolor{currentfill}%
\pgfsetlinewidth{0.000000pt}%
\definecolor{currentstroke}{rgb}{1.000000,1.000000,1.000000}%
\pgfsetstrokecolor{currentstroke}%
\pgfsetdash{}{0pt}%
\pgfpathmoveto{\pgfqpoint{0.000000in}{0.000000in}}%
\pgfpathlineto{\pgfqpoint{5.050000in}{0.000000in}}%
\pgfpathlineto{\pgfqpoint{5.050000in}{3.100000in}}%
\pgfpathlineto{\pgfqpoint{0.000000in}{3.100000in}}%
\pgfpathclose%
\pgfusepath{fill}%
\end{pgfscope}%
\begin{pgfscope}%
\pgfsetbuttcap%
\pgfsetmiterjoin%
\definecolor{currentfill}{rgb}{1.000000,1.000000,1.000000}%
\pgfsetfillcolor{currentfill}%
\pgfsetlinewidth{0.000000pt}%
\definecolor{currentstroke}{rgb}{0.000000,0.000000,0.000000}%
\pgfsetstrokecolor{currentstroke}%
\pgfsetstrokeopacity{0.000000}%
\pgfsetdash{}{0pt}%
\pgfpathmoveto{\pgfqpoint{0.303000in}{0.248000in}}%
\pgfpathlineto{\pgfqpoint{2.598902in}{0.248000in}}%
\pgfpathlineto{\pgfqpoint{2.598902in}{1.610506in}}%
\pgfpathlineto{\pgfqpoint{0.303000in}{1.610506in}}%
\pgfpathclose%
\pgfusepath{fill}%
\end{pgfscope}%
\begin{pgfscope}%
\pgfsetbuttcap%
\pgfsetroundjoin%
\definecolor{currentfill}{rgb}{0.000000,0.000000,0.000000}%
\pgfsetfillcolor{currentfill}%
\pgfsetlinewidth{0.803000pt}%
\definecolor{currentstroke}{rgb}{0.000000,0.000000,0.000000}%
\pgfsetstrokecolor{currentstroke}%
\pgfsetdash{}{0pt}%
\pgfsys@defobject{currentmarker}{\pgfqpoint{0.000000in}{-0.048611in}}{\pgfqpoint{0.000000in}{0.000000in}}{%
\pgfpathmoveto{\pgfqpoint{0.000000in}{0.000000in}}%
\pgfpathlineto{\pgfqpoint{0.000000in}{-0.048611in}}%
\pgfusepath{stroke,fill}%
}%
\begin{pgfscope}%
\pgfsys@transformshift{0.573106in}{0.248000in}%
\pgfsys@useobject{currentmarker}{}%
\end{pgfscope}%
\end{pgfscope}%
\begin{pgfscope}%
\pgftext[x=0.573106in,y=0.150778in,,top]{\rmfamily\fontsize{10.000000}{12.000000}\selectfont \(\displaystyle 0.4\)}%
\end{pgfscope}%
\begin{pgfscope}%
\pgfsetbuttcap%
\pgfsetroundjoin%
\definecolor{currentfill}{rgb}{0.000000,0.000000,0.000000}%
\pgfsetfillcolor{currentfill}%
\pgfsetlinewidth{0.803000pt}%
\definecolor{currentstroke}{rgb}{0.000000,0.000000,0.000000}%
\pgfsetstrokecolor{currentstroke}%
\pgfsetdash{}{0pt}%
\pgfsys@defobject{currentmarker}{\pgfqpoint{0.000000in}{-0.048611in}}{\pgfqpoint{0.000000in}{0.000000in}}{%
\pgfpathmoveto{\pgfqpoint{0.000000in}{0.000000in}}%
\pgfpathlineto{\pgfqpoint{0.000000in}{-0.048611in}}%
\pgfusepath{stroke,fill}%
}%
\begin{pgfscope}%
\pgfsys@transformshift{1.113319in}{0.248000in}%
\pgfsys@useobject{currentmarker}{}%
\end{pgfscope}%
\end{pgfscope}%
\begin{pgfscope}%
\pgftext[x=1.113319in,y=0.150778in,,top]{\rmfamily\fontsize{10.000000}{12.000000}\selectfont \(\displaystyle 0.6\)}%
\end{pgfscope}%
\begin{pgfscope}%
\pgfsetbuttcap%
\pgfsetroundjoin%
\definecolor{currentfill}{rgb}{0.000000,0.000000,0.000000}%
\pgfsetfillcolor{currentfill}%
\pgfsetlinewidth{0.803000pt}%
\definecolor{currentstroke}{rgb}{0.000000,0.000000,0.000000}%
\pgfsetstrokecolor{currentstroke}%
\pgfsetdash{}{0pt}%
\pgfsys@defobject{currentmarker}{\pgfqpoint{0.000000in}{-0.048611in}}{\pgfqpoint{0.000000in}{0.000000in}}{%
\pgfpathmoveto{\pgfqpoint{0.000000in}{0.000000in}}%
\pgfpathlineto{\pgfqpoint{0.000000in}{-0.048611in}}%
\pgfusepath{stroke,fill}%
}%
\begin{pgfscope}%
\pgfsys@transformshift{1.653531in}{0.248000in}%
\pgfsys@useobject{currentmarker}{}%
\end{pgfscope}%
\end{pgfscope}%
\begin{pgfscope}%
\pgftext[x=1.653531in,y=0.150778in,,top]{\rmfamily\fontsize{10.000000}{12.000000}\selectfont \(\displaystyle 0.8\)}%
\end{pgfscope}%
\begin{pgfscope}%
\pgfsetbuttcap%
\pgfsetroundjoin%
\definecolor{currentfill}{rgb}{0.000000,0.000000,0.000000}%
\pgfsetfillcolor{currentfill}%
\pgfsetlinewidth{0.803000pt}%
\definecolor{currentstroke}{rgb}{0.000000,0.000000,0.000000}%
\pgfsetstrokecolor{currentstroke}%
\pgfsetdash{}{0pt}%
\pgfsys@defobject{currentmarker}{\pgfqpoint{0.000000in}{-0.048611in}}{\pgfqpoint{0.000000in}{0.000000in}}{%
\pgfpathmoveto{\pgfqpoint{0.000000in}{0.000000in}}%
\pgfpathlineto{\pgfqpoint{0.000000in}{-0.048611in}}%
\pgfusepath{stroke,fill}%
}%
\begin{pgfscope}%
\pgfsys@transformshift{2.193743in}{0.248000in}%
\pgfsys@useobject{currentmarker}{}%
\end{pgfscope}%
\end{pgfscope}%
\begin{pgfscope}%
\pgftext[x=2.193743in,y=0.150778in,,top]{\rmfamily\fontsize{10.000000}{12.000000}\selectfont \(\displaystyle 1.0\)}%
\end{pgfscope}%
\begin{pgfscope}%
\pgfsetbuttcap%
\pgfsetroundjoin%
\definecolor{currentfill}{rgb}{0.000000,0.000000,0.000000}%
\pgfsetfillcolor{currentfill}%
\pgfsetlinewidth{0.803000pt}%
\definecolor{currentstroke}{rgb}{0.000000,0.000000,0.000000}%
\pgfsetstrokecolor{currentstroke}%
\pgfsetdash{}{0pt}%
\pgfsys@defobject{currentmarker}{\pgfqpoint{-0.048611in}{0.000000in}}{\pgfqpoint{0.000000in}{0.000000in}}{%
\pgfpathmoveto{\pgfqpoint{0.000000in}{0.000000in}}%
\pgfpathlineto{\pgfqpoint{-0.048611in}{0.000000in}}%
\pgfusepath{stroke,fill}%
}%
\begin{pgfscope}%
\pgfsys@transformshift{0.303000in}{0.454440in}%
\pgfsys@useobject{currentmarker}{}%
\end{pgfscope}%
\end{pgfscope}%
\begin{pgfscope}%
\pgftext[x=0.037306in,y=0.406489in,left,base]{\rmfamily\fontsize{10.000000}{12.000000}\selectfont \(\displaystyle 0.7\)}%
\end{pgfscope}%
\begin{pgfscope}%
\pgfsetbuttcap%
\pgfsetroundjoin%
\definecolor{currentfill}{rgb}{0.000000,0.000000,0.000000}%
\pgfsetfillcolor{currentfill}%
\pgfsetlinewidth{0.803000pt}%
\definecolor{currentstroke}{rgb}{0.000000,0.000000,0.000000}%
\pgfsetstrokecolor{currentstroke}%
\pgfsetdash{}{0pt}%
\pgfsys@defobject{currentmarker}{\pgfqpoint{-0.048611in}{0.000000in}}{\pgfqpoint{0.000000in}{0.000000in}}{%
\pgfpathmoveto{\pgfqpoint{0.000000in}{0.000000in}}%
\pgfpathlineto{\pgfqpoint{-0.048611in}{0.000000in}}%
\pgfusepath{stroke,fill}%
}%
\begin{pgfscope}%
\pgfsys@transformshift{0.303000in}{0.867321in}%
\pgfsys@useobject{currentmarker}{}%
\end{pgfscope}%
\end{pgfscope}%
\begin{pgfscope}%
\pgftext[x=0.037306in,y=0.819370in,left,base]{\rmfamily\fontsize{10.000000}{12.000000}\selectfont \(\displaystyle 0.8\)}%
\end{pgfscope}%
\begin{pgfscope}%
\pgfsetbuttcap%
\pgfsetroundjoin%
\definecolor{currentfill}{rgb}{0.000000,0.000000,0.000000}%
\pgfsetfillcolor{currentfill}%
\pgfsetlinewidth{0.803000pt}%
\definecolor{currentstroke}{rgb}{0.000000,0.000000,0.000000}%
\pgfsetstrokecolor{currentstroke}%
\pgfsetdash{}{0pt}%
\pgfsys@defobject{currentmarker}{\pgfqpoint{-0.048611in}{0.000000in}}{\pgfqpoint{0.000000in}{0.000000in}}{%
\pgfpathmoveto{\pgfqpoint{0.000000in}{0.000000in}}%
\pgfpathlineto{\pgfqpoint{-0.048611in}{0.000000in}}%
\pgfusepath{stroke,fill}%
}%
\begin{pgfscope}%
\pgfsys@transformshift{0.303000in}{1.280202in}%
\pgfsys@useobject{currentmarker}{}%
\end{pgfscope}%
\end{pgfscope}%
\begin{pgfscope}%
\pgftext[x=0.037306in,y=1.232250in,left,base]{\rmfamily\fontsize{10.000000}{12.000000}\selectfont \(\displaystyle 0.9\)}%
\end{pgfscope}%
\begin{pgfscope}%
\pgfpathrectangle{\pgfqpoint{0.303000in}{0.248000in}}{\pgfqpoint{2.295902in}{1.362506in}} %
\pgfusepath{clip}%
\pgfsetrectcap%
\pgfsetroundjoin%
\pgfsetlinewidth{1.505625pt}%
\definecolor{currentstroke}{rgb}{0.121569,0.466667,0.705882}%
\pgfsetstrokecolor{currentstroke}%
\pgfsetdash{}{0pt}%
\pgfusepath{stroke}%
\end{pgfscope}%
\begin{pgfscope}%
\pgfpathrectangle{\pgfqpoint{0.303000in}{0.248000in}}{\pgfqpoint{2.295902in}{1.362506in}} %
\pgfusepath{clip}%
\pgfsetrectcap%
\pgfsetroundjoin%
\pgfsetlinewidth{1.505625pt}%
\definecolor{currentstroke}{rgb}{1.000000,0.498039,0.054902}%
\pgfsetstrokecolor{currentstroke}%
\pgfsetdash{}{0pt}%
\pgfusepath{stroke}%
\end{pgfscope}%
\begin{pgfscope}%
\pgfpathrectangle{\pgfqpoint{0.303000in}{0.248000in}}{\pgfqpoint{2.295902in}{1.362506in}} %
\pgfusepath{clip}%
\pgfsetrectcap%
\pgfsetroundjoin%
\pgfsetlinewidth{1.505625pt}%
\definecolor{currentstroke}{rgb}{0.172549,0.627451,0.172549}%
\pgfsetstrokecolor{currentstroke}%
\pgfsetdash{}{0pt}%
\pgfusepath{stroke}%
\end{pgfscope}%
\begin{pgfscope}%
\pgfpathrectangle{\pgfqpoint{0.303000in}{0.248000in}}{\pgfqpoint{2.295902in}{1.362506in}} %
\pgfusepath{clip}%
\pgfsetrectcap%
\pgfsetroundjoin%
\pgfsetlinewidth{1.003750pt}%
\definecolor{currentstroke}{rgb}{0.839216,0.152941,0.156863}%
\pgfsetstrokecolor{currentstroke}%
\pgfsetdash{}{0pt}%
\pgfpathmoveto{\pgfqpoint{0.335391in}{1.162823in}}%
\pgfpathlineto{\pgfqpoint{0.368284in}{1.191329in}}%
\pgfpathlineto{\pgfqpoint{0.401989in}{1.217524in}}%
\pgfpathlineto{\pgfqpoint{0.436381in}{1.241545in}}%
\pgfpathlineto{\pgfqpoint{0.473866in}{1.265041in}}%
\pgfpathlineto{\pgfqpoint{0.514445in}{1.287757in}}%
\pgfpathlineto{\pgfqpoint{0.558104in}{1.309487in}}%
\pgfpathlineto{\pgfqpoint{0.602212in}{1.328991in}}%
\pgfpathlineto{\pgfqpoint{0.649301in}{1.347480in}}%
\pgfpathlineto{\pgfqpoint{0.701987in}{1.365714in}}%
\pgfpathlineto{\pgfqpoint{0.757626in}{1.382563in}}%
\pgfpathlineto{\pgfqpoint{0.816182in}{1.398013in}}%
\pgfpathlineto{\pgfqpoint{0.880297in}{1.412635in}}%
\pgfpathlineto{\pgfqpoint{0.949960in}{1.426200in}}%
\pgfpathlineto{\pgfqpoint{1.025158in}{1.438523in}}%
\pgfpathlineto{\pgfqpoint{1.105872in}{1.449451in}}%
\pgfpathlineto{\pgfqpoint{1.192086in}{1.458852in}}%
\pgfpathlineto{\pgfqpoint{1.283782in}{1.466596in}}%
\pgfpathlineto{\pgfqpoint{1.380941in}{1.472536in}}%
\pgfpathlineto{\pgfqpoint{1.480848in}{1.476403in}}%
\pgfpathlineto{\pgfqpoint{1.583481in}{1.478115in}}%
\pgfpathlineto{\pgfqpoint{1.683419in}{1.477568in}}%
\pgfpathlineto{\pgfqpoint{1.780639in}{1.474802in}}%
\pgfpathlineto{\pgfqpoint{1.872419in}{1.469939in}}%
\pgfpathlineto{\pgfqpoint{1.956037in}{1.463271in}}%
\pgfpathlineto{\pgfqpoint{2.031473in}{1.455030in}}%
\pgfpathlineto{\pgfqpoint{2.098707in}{1.445459in}}%
\pgfpathlineto{\pgfqpoint{2.157723in}{1.434852in}}%
\pgfpathlineto{\pgfqpoint{2.208510in}{1.423593in}}%
\pgfpathlineto{\pgfqpoint{2.253729in}{1.411410in}}%
\pgfpathlineto{\pgfqpoint{2.293357in}{1.398536in}}%
\pgfpathlineto{\pgfqpoint{2.327387in}{1.385317in}}%
\pgfpathlineto{\pgfqpoint{2.358402in}{1.370941in}}%
\pgfpathlineto{\pgfqpoint{2.386321in}{1.355427in}}%
\pgfpathlineto{\pgfqpoint{2.408618in}{1.340638in}}%
\pgfpathlineto{\pgfqpoint{2.427832in}{1.325538in}}%
\pgfpathlineto{\pgfqpoint{2.446226in}{1.308223in}}%
\pgfpathlineto{\pgfqpoint{2.461376in}{1.290949in}}%
\pgfpathlineto{\pgfqpoint{2.475321in}{1.271452in}}%
\pgfpathlineto{\pgfqpoint{2.486031in}{1.252875in}}%
\pgfpathlineto{\pgfqpoint{2.495345in}{1.232618in}}%
\pgfpathlineto{\pgfqpoint{2.503076in}{1.210861in}}%
\pgfpathlineto{\pgfqpoint{2.509989in}{1.183982in}}%
\pgfpathlineto{\pgfqpoint{2.514711in}{1.156013in}}%
\pgfpathlineto{\pgfqpoint{2.517741in}{1.123325in}}%
\pgfpathlineto{\pgfqpoint{2.518710in}{1.086209in}}%
\pgfpathlineto{\pgfqpoint{2.517520in}{1.044970in}}%
\pgfpathlineto{\pgfqpoint{2.513909in}{0.995740in}}%
\pgfpathlineto{\pgfqpoint{2.506771in}{0.930593in}}%
\pgfpathlineto{\pgfqpoint{2.494100in}{0.837631in}}%
\pgfpathlineto{\pgfqpoint{2.434891in}{0.422130in}}%
\pgfpathlineto{\pgfqpoint{2.416986in}{0.271839in}}%
\pgfpathlineto{\pgfqpoint{2.413264in}{0.238000in}}%
\pgfpathlineto{\pgfqpoint{2.413264in}{0.238000in}}%
\pgfusepath{stroke}%
\end{pgfscope}%
\begin{pgfscope}%
\pgfpathrectangle{\pgfqpoint{0.303000in}{0.248000in}}{\pgfqpoint{2.295902in}{1.362506in}} %
\pgfusepath{clip}%
\pgfsetrectcap%
\pgfsetroundjoin%
\pgfsetlinewidth{1.003750pt}%
\definecolor{currentstroke}{rgb}{0.839216,0.152941,0.156863}%
\pgfsetstrokecolor{currentstroke}%
\pgfsetdash{}{0pt}%
\pgfpathmoveto{\pgfqpoint{2.276477in}{0.238000in}}%
\pgfpathlineto{\pgfqpoint{2.279923in}{0.399578in}}%
\pgfpathlineto{\pgfqpoint{2.280370in}{0.519306in}}%
\pgfpathlineto{\pgfqpoint{2.278538in}{0.614222in}}%
\pgfpathlineto{\pgfqpoint{2.275039in}{0.688341in}}%
\pgfpathlineto{\pgfqpoint{2.269768in}{0.753900in}}%
\pgfpathlineto{\pgfqpoint{2.262921in}{0.810738in}}%
\pgfpathlineto{\pgfqpoint{2.254939in}{0.858748in}}%
\pgfpathlineto{\pgfqpoint{2.245548in}{0.901824in}}%
\pgfpathlineto{\pgfqpoint{2.235029in}{0.939841in}}%
\pgfpathlineto{\pgfqpoint{2.223800in}{0.972788in}}%
\pgfpathlineto{\pgfqpoint{2.210833in}{1.004206in}}%
\pgfpathlineto{\pgfqpoint{2.196143in}{1.033798in}}%
\pgfpathlineto{\pgfqpoint{2.179829in}{1.061331in}}%
\pgfpathlineto{\pgfqpoint{2.162054in}{1.086665in}}%
\pgfpathlineto{\pgfqpoint{2.143023in}{1.109771in}}%
\pgfpathlineto{\pgfqpoint{2.122946in}{1.130709in}}%
\pgfpathlineto{\pgfqpoint{2.099650in}{1.151590in}}%
\pgfpathlineto{\pgfqpoint{2.075537in}{1.170181in}}%
\pgfpathlineto{\pgfqpoint{2.048290in}{1.188277in}}%
\pgfpathlineto{\pgfqpoint{2.017916in}{1.205557in}}%
\pgfpathlineto{\pgfqpoint{1.987041in}{1.220622in}}%
\pgfpathlineto{\pgfqpoint{1.953176in}{1.234797in}}%
\pgfpathlineto{\pgfqpoint{1.913711in}{1.248793in}}%
\pgfpathlineto{\pgfqpoint{1.871285in}{1.261361in}}%
\pgfpathlineto{\pgfqpoint{1.825946in}{1.272456in}}%
\pgfpathlineto{\pgfqpoint{1.775052in}{1.282519in}}%
\pgfpathlineto{\pgfqpoint{1.721307in}{1.290829in}}%
\pgfpathlineto{\pgfqpoint{1.664748in}{1.297374in}}%
\pgfpathlineto{\pgfqpoint{1.602708in}{1.302285in}}%
\pgfpathlineto{\pgfqpoint{1.537911in}{1.305145in}}%
\pgfpathlineto{\pgfqpoint{1.470387in}{1.305861in}}%
\pgfpathlineto{\pgfqpoint{1.402869in}{1.304386in}}%
\pgfpathlineto{\pgfqpoint{1.332686in}{1.300586in}}%
\pgfpathlineto{\pgfqpoint{1.265270in}{1.294733in}}%
\pgfpathlineto{\pgfqpoint{1.197952in}{1.286652in}}%
\pgfpathlineto{\pgfqpoint{1.133457in}{1.276676in}}%
\pgfpathlineto{\pgfqpoint{1.071812in}{1.264918in}}%
\pgfpathlineto{\pgfqpoint{1.013045in}{1.251476in}}%
\pgfpathlineto{\pgfqpoint{0.957183in}{1.236441in}}%
\pgfpathlineto{\pgfqpoint{0.904258in}{1.219908in}}%
\pgfpathlineto{\pgfqpoint{0.854297in}{1.201985in}}%
\pgfpathlineto{\pgfqpoint{0.807329in}{1.182796in}}%
\pgfpathlineto{\pgfqpoint{0.763376in}{1.162491in}}%
\pgfpathlineto{\pgfqpoint{0.722456in}{1.141253in}}%
\pgfpathlineto{\pgfqpoint{0.682068in}{1.117754in}}%
\pgfpathlineto{\pgfqpoint{0.644793in}{1.093493in}}%
\pgfpathlineto{\pgfqpoint{0.610621in}{1.068748in}}%
\pgfpathlineto{\pgfqpoint{0.577158in}{1.041837in}}%
\pgfpathlineto{\pgfqpoint{0.544527in}{1.012637in}}%
\pgfpathlineto{\pgfqpoint{0.515092in}{0.983379in}}%
\pgfpathlineto{\pgfqpoint{0.486613in}{0.951990in}}%
\pgfpathlineto{\pgfqpoint{0.459212in}{0.918434in}}%
\pgfpathlineto{\pgfqpoint{0.433014in}{0.882706in}}%
\pgfpathlineto{\pgfqpoint{0.408134in}{0.844839in}}%
\pgfpathlineto{\pgfqpoint{0.384681in}{0.804903in}}%
\pgfpathlineto{\pgfqpoint{0.362744in}{0.763001in}}%
\pgfpathlineto{\pgfqpoint{0.342392in}{0.719270in}}%
\pgfpathlineto{\pgfqpoint{0.323673in}{0.673867in}}%
\pgfpathlineto{\pgfqpoint{0.306609in}{0.626964in}}%
\pgfpathlineto{\pgfqpoint{0.293000in}{0.584702in}}%
\pgfpathmoveto{\pgfqpoint{0.293000in}{0.908857in}}%
\pgfpathlineto{\pgfqpoint{0.316884in}{0.946850in}}%
\pgfpathlineto{\pgfqpoint{0.342548in}{0.983474in}}%
\pgfpathlineto{\pgfqpoint{0.369458in}{1.017945in}}%
\pgfpathlineto{\pgfqpoint{0.397498in}{1.050244in}}%
\pgfpathlineto{\pgfqpoint{0.426547in}{1.080387in}}%
\pgfpathlineto{\pgfqpoint{0.456487in}{1.108424in}}%
\pgfpathlineto{\pgfqpoint{0.489592in}{1.136350in}}%
\pgfpathlineto{\pgfqpoint{0.523463in}{1.162042in}}%
\pgfpathlineto{\pgfqpoint{0.560469in}{1.187246in}}%
\pgfpathlineto{\pgfqpoint{0.598091in}{1.210221in}}%
\pgfpathlineto{\pgfqpoint{0.638780in}{1.232472in}}%
\pgfpathlineto{\pgfqpoint{0.682526in}{1.253793in}}%
\pgfpathlineto{\pgfqpoint{0.729309in}{1.274018in}}%
\pgfpathlineto{\pgfqpoint{0.779098in}{1.293020in}}%
\pgfpathlineto{\pgfqpoint{0.831864in}{1.310706in}}%
\pgfpathlineto{\pgfqpoint{0.887573in}{1.327010in}}%
\pgfpathlineto{\pgfqpoint{0.946192in}{1.341889in}}%
\pgfpathlineto{\pgfqpoint{1.010370in}{1.355847in}}%
\pgfpathlineto{\pgfqpoint{1.077414in}{1.368156in}}%
\pgfpathlineto{\pgfqpoint{1.149984in}{1.379177in}}%
\pgfpathlineto{\pgfqpoint{1.225373in}{1.388372in}}%
\pgfpathlineto{\pgfqpoint{1.303555in}{1.395716in}}%
\pgfpathlineto{\pgfqpoint{1.384508in}{1.401146in}}%
\pgfpathlineto{\pgfqpoint{1.468210in}{1.404548in}}%
\pgfpathlineto{\pgfqpoint{1.551938in}{1.405740in}}%
\pgfpathlineto{\pgfqpoint{1.632967in}{1.404737in}}%
\pgfpathlineto{\pgfqpoint{1.711270in}{1.401616in}}%
\pgfpathlineto{\pgfqpoint{1.786821in}{1.396380in}}%
\pgfpathlineto{\pgfqpoint{1.856893in}{1.389274in}}%
\pgfpathlineto{\pgfqpoint{1.921459in}{1.380446in}}%
\pgfpathlineto{\pgfqpoint{1.977810in}{1.370570in}}%
\pgfpathlineto{\pgfqpoint{2.028618in}{1.359525in}}%
\pgfpathlineto{\pgfqpoint{2.073862in}{1.347560in}}%
\pgfpathlineto{\pgfqpoint{2.116167in}{1.334083in}}%
\pgfpathlineto{\pgfqpoint{2.152859in}{1.320114in}}%
\pgfpathlineto{\pgfqpoint{2.186528in}{1.304896in}}%
\pgfpathlineto{\pgfqpoint{2.214583in}{1.289955in}}%
\pgfpathlineto{\pgfqpoint{2.239597in}{1.274388in}}%
\pgfpathlineto{\pgfqpoint{2.263958in}{1.256572in}}%
\pgfpathlineto{\pgfqpoint{2.285113in}{1.238282in}}%
\pgfpathlineto{\pgfqpoint{2.303091in}{1.219970in}}%
\pgfpathlineto{\pgfqpoint{2.320051in}{1.199519in}}%
\pgfpathlineto{\pgfqpoint{2.335722in}{1.176796in}}%
\pgfpathlineto{\pgfqpoint{2.348153in}{1.155032in}}%
\pgfpathlineto{\pgfqpoint{2.359212in}{1.131603in}}%
\pgfpathlineto{\pgfqpoint{2.370027in}{1.103025in}}%
\pgfpathlineto{\pgfqpoint{2.378854in}{1.072892in}}%
\pgfpathlineto{\pgfqpoint{2.385778in}{1.041614in}}%
\pgfpathlineto{\pgfqpoint{2.391522in}{1.005519in}}%
\pgfpathlineto{\pgfqpoint{2.395753in}{0.964751in}}%
\pgfpathlineto{\pgfqpoint{2.398460in}{0.915389in}}%
\pgfpathlineto{\pgfqpoint{2.399196in}{0.857604in}}%
\pgfpathlineto{\pgfqpoint{2.397682in}{0.787459in}}%
\pgfpathlineto{\pgfqpoint{2.393295in}{0.696878in}}%
\pgfpathlineto{\pgfqpoint{2.384407in}{0.565459in}}%
\pgfpathlineto{\pgfqpoint{2.359372in}{0.238000in}}%
\pgfpathlineto{\pgfqpoint{2.359372in}{0.238000in}}%
\pgfusepath{stroke}%
\end{pgfscope}%
\begin{pgfscope}%
\pgfpathrectangle{\pgfqpoint{0.303000in}{0.248000in}}{\pgfqpoint{2.295902in}{1.362506in}} %
\pgfusepath{clip}%
\pgfsetrectcap%
\pgfsetroundjoin%
\pgfsetlinewidth{1.003750pt}%
\definecolor{currentstroke}{rgb}{0.839216,0.152941,0.156863}%
\pgfsetstrokecolor{currentstroke}%
\pgfsetdash{}{0pt}%
\pgfpathmoveto{\pgfqpoint{2.421105in}{0.238000in}}%
\pgfpathlineto{\pgfqpoint{2.438092in}{0.377739in}}%
\pgfpathlineto{\pgfqpoint{2.456998in}{0.515108in}}%
\pgfpathlineto{\pgfqpoint{2.478587in}{0.655794in}}%
\pgfpathlineto{\pgfqpoint{2.526511in}{0.960757in}}%
\pgfpathlineto{\pgfqpoint{2.534967in}{1.029737in}}%
\pgfpathlineto{\pgfqpoint{2.538896in}{1.078907in}}%
\pgfpathlineto{\pgfqpoint{2.540030in}{1.120145in}}%
\pgfpathlineto{\pgfqpoint{2.538922in}{1.153119in}}%
\pgfpathlineto{\pgfqpoint{2.536022in}{1.181665in}}%
\pgfpathlineto{\pgfqpoint{2.530946in}{1.209485in}}%
\pgfpathlineto{\pgfqpoint{2.524681in}{1.232315in}}%
\pgfpathlineto{\pgfqpoint{2.516632in}{1.253799in}}%
\pgfpathlineto{\pgfqpoint{2.506936in}{1.273631in}}%
\pgfpathlineto{\pgfqpoint{2.495832in}{1.291659in}}%
\pgfpathlineto{\pgfqpoint{2.481464in}{1.310424in}}%
\pgfpathlineto{\pgfqpoint{2.465951in}{1.326928in}}%
\pgfpathlineto{\pgfqpoint{2.447226in}{1.343395in}}%
\pgfpathlineto{\pgfqpoint{2.425287in}{1.359380in}}%
\pgfpathlineto{\pgfqpoint{2.400180in}{1.374590in}}%
\pgfpathlineto{\pgfqpoint{2.371973in}{1.388843in}}%
\pgfpathlineto{\pgfqpoint{2.340746in}{1.402106in}}%
\pgfpathlineto{\pgfqpoint{2.303917in}{1.415202in}}%
\pgfpathlineto{\pgfqpoint{2.261488in}{1.427737in}}%
\pgfpathlineto{\pgfqpoint{2.213477in}{1.439442in}}%
\pgfpathlineto{\pgfqpoint{2.159913in}{1.450142in}}%
\pgfpathlineto{\pgfqpoint{2.098133in}{1.460101in}}%
\pgfpathlineto{\pgfqpoint{2.030841in}{1.468685in}}%
\pgfpathlineto{\pgfqpoint{1.955367in}{1.476086in}}%
\pgfpathlineto{\pgfqpoint{1.871727in}{1.482064in}}%
\pgfpathlineto{\pgfqpoint{1.788034in}{1.485907in}}%
\pgfpathlineto{\pgfqpoint{1.690811in}{1.488472in}}%
\pgfpathlineto{\pgfqpoint{1.588172in}{1.488967in}}%
\pgfpathlineto{\pgfqpoint{1.482838in}{1.487227in}}%
\pgfpathlineto{\pgfqpoint{1.369428in}{1.483011in}}%
\pgfpathlineto{\pgfqpoint{1.269572in}{1.476843in}}%
\pgfpathlineto{\pgfqpoint{1.156299in}{1.467432in}}%
\pgfpathlineto{\pgfqpoint{1.070106in}{1.457600in}}%
\pgfpathlineto{\pgfqpoint{0.989415in}{1.446263in}}%
\pgfpathlineto{\pgfqpoint{0.914246in}{1.433540in}}%
\pgfpathlineto{\pgfqpoint{0.815148in}{1.414132in}}%
\pgfpathlineto{\pgfqpoint{0.735157in}{1.394486in}}%
\pgfpathlineto{\pgfqpoint{0.679495in}{1.377818in}}%
\pgfpathlineto{\pgfqpoint{0.626779in}{1.359782in}}%
\pgfpathlineto{\pgfqpoint{0.577051in}{1.340408in}}%
\pgfpathlineto{\pgfqpoint{0.530344in}{1.319783in}}%
\pgfpathlineto{\pgfqpoint{0.486699in}{1.297982in}}%
\pgfpathlineto{\pgfqpoint{0.443591in}{1.273824in}}%
\pgfpathlineto{\pgfqpoint{0.406173in}{1.250084in}}%
\pgfpathlineto{\pgfqpoint{0.371858in}{1.225807in}}%
\pgfpathlineto{\pgfqpoint{0.338249in}{1.199324in}}%
\pgfpathlineto{\pgfqpoint{0.307784in}{1.172642in}}%
\pgfpathlineto{\pgfqpoint{0.293000in}{1.158197in}}%
\pgfpathlineto{\pgfqpoint{0.293000in}{1.158197in}}%
\pgfusepath{stroke}%
\end{pgfscope}%
\begin{pgfscope}%
\pgfpathrectangle{\pgfqpoint{0.303000in}{0.248000in}}{\pgfqpoint{2.295902in}{1.362506in}} %
\pgfusepath{clip}%
\pgfsetrectcap%
\pgfsetroundjoin%
\pgfsetlinewidth{1.003750pt}%
\definecolor{currentstroke}{rgb}{0.839216,0.152941,0.156863}%
\pgfsetstrokecolor{currentstroke}%
\pgfsetdash{}{0pt}%
\pgfpathmoveto{\pgfqpoint{2.420345in}{0.238000in}}%
\pgfpathlineto{\pgfqpoint{2.437242in}{0.377704in}}%
\pgfpathlineto{\pgfqpoint{2.456042in}{0.515108in}}%
\pgfpathlineto{\pgfqpoint{2.478123in}{0.659858in}}%
\pgfpathlineto{\pgfqpoint{2.525732in}{0.964993in}}%
\pgfpathlineto{\pgfqpoint{2.533445in}{1.029985in}}%
\pgfpathlineto{\pgfqpoint{2.537199in}{1.079186in}}%
\pgfpathlineto{\pgfqpoint{2.538143in}{1.120435in}}%
\pgfpathlineto{\pgfqpoint{2.536853in}{1.153394in}}%
\pgfpathlineto{\pgfqpoint{2.533791in}{1.181900in}}%
\pgfpathlineto{\pgfqpoint{2.528540in}{1.209644in}}%
\pgfpathlineto{\pgfqpoint{2.522131in}{1.232381in}}%
\pgfpathlineto{\pgfqpoint{2.513956in}{1.253753in}}%
\pgfpathlineto{\pgfqpoint{2.504157in}{1.273467in}}%
\pgfpathlineto{\pgfqpoint{2.492973in}{1.291379in}}%
\pgfpathlineto{\pgfqpoint{2.478540in}{1.310028in}}%
\pgfpathlineto{\pgfqpoint{2.462985in}{1.326441in}}%
\pgfpathlineto{\pgfqpoint{2.444233in}{1.342835in}}%
\pgfpathlineto{\pgfqpoint{2.422273in}{1.358754in}}%
\pgfpathlineto{\pgfqpoint{2.397151in}{1.373906in}}%
\pgfpathlineto{\pgfqpoint{2.368937in}{1.388126in}}%
\pgfpathlineto{\pgfqpoint{2.337705in}{1.401359in}}%
\pgfpathlineto{\pgfqpoint{2.300872in}{1.414430in}}%
\pgfpathlineto{\pgfqpoint{2.258440in}{1.426948in}}%
\pgfpathlineto{\pgfqpoint{2.210428in}{1.438638in}}%
\pgfpathlineto{\pgfqpoint{2.156863in}{1.449326in}}%
\pgfpathlineto{\pgfqpoint{2.095082in}{1.459274in}}%
\pgfpathlineto{\pgfqpoint{2.027789in}{1.467845in}}%
\pgfpathlineto{\pgfqpoint{1.952315in}{1.475229in}}%
\pgfpathlineto{\pgfqpoint{1.868674in}{1.481188in}}%
\pgfpathlineto{\pgfqpoint{1.784981in}{1.485033in}}%
\pgfpathlineto{\pgfqpoint{1.687757in}{1.487561in}}%
\pgfpathlineto{\pgfqpoint{1.585118in}{1.488009in}}%
\pgfpathlineto{\pgfqpoint{1.479784in}{1.486208in}}%
\pgfpathlineto{\pgfqpoint{1.369076in}{1.482025in}}%
\pgfpathlineto{\pgfqpoint{1.269220in}{1.475828in}}%
\pgfpathlineto{\pgfqpoint{1.158646in}{1.466578in}}%
\pgfpathlineto{\pgfqpoint{1.072451in}{1.456762in}}%
\pgfpathlineto{\pgfqpoint{0.991760in}{1.445437in}}%
\pgfpathlineto{\pgfqpoint{0.916590in}{1.432723in}}%
\pgfpathlineto{\pgfqpoint{0.820167in}{1.413778in}}%
\pgfpathlineto{\pgfqpoint{0.737502in}{1.393564in}}%
\pgfpathlineto{\pgfqpoint{0.681840in}{1.376892in}}%
\pgfpathlineto{\pgfqpoint{0.629125in}{1.358854in}}%
\pgfpathlineto{\pgfqpoint{0.579397in}{1.339477in}}%
\pgfpathlineto{\pgfqpoint{0.532691in}{1.318850in}}%
\pgfpathlineto{\pgfqpoint{0.491599in}{1.298399in}}%
\pgfpathlineto{\pgfqpoint{0.448448in}{1.274419in}}%
\pgfpathlineto{\pgfqpoint{0.410989in}{1.250828in}}%
\pgfpathlineto{\pgfqpoint{0.376628in}{1.226704in}}%
\pgfpathlineto{\pgfqpoint{0.342964in}{1.200385in}}%
\pgfpathlineto{\pgfqpoint{0.310123in}{1.171738in}}%
\pgfpathlineto{\pgfqpoint{0.293000in}{1.154609in}}%
\pgfpathlineto{\pgfqpoint{0.293000in}{1.154609in}}%
\pgfusepath{stroke}%
\end{pgfscope}%
\begin{pgfscope}%
\pgfpathrectangle{\pgfqpoint{0.303000in}{0.248000in}}{\pgfqpoint{2.295902in}{1.362506in}} %
\pgfusepath{clip}%
\pgfsetbuttcap%
\pgfsetroundjoin%
\pgfsetlinewidth{1.003750pt}%
\definecolor{currentstroke}{rgb}{0.000000,0.000000,0.000000}%
\pgfsetstrokecolor{currentstroke}%
\pgfsetdash{{3.700000pt}{1.600000pt}}{0.000000pt}%
\pgfpathmoveto{\pgfqpoint{0.335391in}{1.162823in}}%
\pgfpathlineto{\pgfqpoint{0.368284in}{1.191329in}}%
\pgfpathlineto{\pgfqpoint{0.401989in}{1.217524in}}%
\pgfpathlineto{\pgfqpoint{0.436381in}{1.241545in}}%
\pgfpathlineto{\pgfqpoint{0.473866in}{1.265041in}}%
\pgfpathlineto{\pgfqpoint{0.514445in}{1.287757in}}%
\pgfpathlineto{\pgfqpoint{0.558104in}{1.309487in}}%
\pgfpathlineto{\pgfqpoint{0.602212in}{1.328991in}}%
\pgfpathlineto{\pgfqpoint{0.649301in}{1.347480in}}%
\pgfpathlineto{\pgfqpoint{0.701987in}{1.365714in}}%
\pgfpathlineto{\pgfqpoint{0.757626in}{1.382563in}}%
\pgfpathlineto{\pgfqpoint{0.816182in}{1.398013in}}%
\pgfpathlineto{\pgfqpoint{0.880297in}{1.412635in}}%
\pgfpathlineto{\pgfqpoint{0.949960in}{1.426200in}}%
\pgfpathlineto{\pgfqpoint{1.025158in}{1.438523in}}%
\pgfpathlineto{\pgfqpoint{1.105872in}{1.449451in}}%
\pgfpathlineto{\pgfqpoint{1.192086in}{1.458852in}}%
\pgfpathlineto{\pgfqpoint{1.283782in}{1.466596in}}%
\pgfpathlineto{\pgfqpoint{1.380941in}{1.472536in}}%
\pgfpathlineto{\pgfqpoint{1.480848in}{1.476403in}}%
\pgfpathlineto{\pgfqpoint{1.583481in}{1.478115in}}%
\pgfpathlineto{\pgfqpoint{1.683419in}{1.477568in}}%
\pgfpathlineto{\pgfqpoint{1.780639in}{1.474802in}}%
\pgfpathlineto{\pgfqpoint{1.872419in}{1.469939in}}%
\pgfpathlineto{\pgfqpoint{1.956037in}{1.463271in}}%
\pgfpathlineto{\pgfqpoint{2.031473in}{1.455030in}}%
\pgfpathlineto{\pgfqpoint{2.098707in}{1.445459in}}%
\pgfpathlineto{\pgfqpoint{2.157723in}{1.434852in}}%
\pgfpathlineto{\pgfqpoint{2.208510in}{1.423593in}}%
\pgfpathlineto{\pgfqpoint{2.253729in}{1.411410in}}%
\pgfpathlineto{\pgfqpoint{2.293357in}{1.398536in}}%
\pgfpathlineto{\pgfqpoint{2.327387in}{1.385317in}}%
\pgfpathlineto{\pgfqpoint{2.358402in}{1.370941in}}%
\pgfpathlineto{\pgfqpoint{2.386321in}{1.355427in}}%
\pgfpathlineto{\pgfqpoint{2.408618in}{1.340638in}}%
\pgfpathlineto{\pgfqpoint{2.427832in}{1.325538in}}%
\pgfpathlineto{\pgfqpoint{2.446226in}{1.308223in}}%
\pgfpathlineto{\pgfqpoint{2.461376in}{1.290949in}}%
\pgfpathlineto{\pgfqpoint{2.475321in}{1.271452in}}%
\pgfpathlineto{\pgfqpoint{2.486031in}{1.252875in}}%
\pgfpathlineto{\pgfqpoint{2.495345in}{1.232618in}}%
\pgfpathlineto{\pgfqpoint{2.503076in}{1.210861in}}%
\pgfpathlineto{\pgfqpoint{2.509989in}{1.183982in}}%
\pgfpathlineto{\pgfqpoint{2.514711in}{1.156013in}}%
\pgfpathlineto{\pgfqpoint{2.517741in}{1.123325in}}%
\pgfpathlineto{\pgfqpoint{2.518710in}{1.086209in}}%
\pgfpathlineto{\pgfqpoint{2.517520in}{1.044970in}}%
\pgfpathlineto{\pgfqpoint{2.513909in}{0.995740in}}%
\pgfpathlineto{\pgfqpoint{2.506771in}{0.930593in}}%
\pgfpathlineto{\pgfqpoint{2.494100in}{0.837631in}}%
\pgfpathlineto{\pgfqpoint{2.434891in}{0.422130in}}%
\pgfpathlineto{\pgfqpoint{2.416986in}{0.271839in}}%
\pgfpathlineto{\pgfqpoint{2.413264in}{0.238000in}}%
\pgfpathlineto{\pgfqpoint{2.413264in}{0.238000in}}%
\pgfusepath{stroke}%
\end{pgfscope}%
\begin{pgfscope}%
\pgfpathrectangle{\pgfqpoint{0.303000in}{0.248000in}}{\pgfqpoint{2.295902in}{1.362506in}} %
\pgfusepath{clip}%
\pgfsetbuttcap%
\pgfsetroundjoin%
\pgfsetlinewidth{1.003750pt}%
\definecolor{currentstroke}{rgb}{0.000000,0.000000,0.000000}%
\pgfsetstrokecolor{currentstroke}%
\pgfsetdash{{3.700000pt}{1.600000pt}}{0.000000pt}%
\pgfpathmoveto{\pgfqpoint{0.000000in}{0.000000in}}%
\pgfusepath{stroke}%
\end{pgfscope}%
\begin{pgfscope}%
\pgfpathrectangle{\pgfqpoint{0.303000in}{0.248000in}}{\pgfqpoint{2.295902in}{1.362506in}} %
\pgfusepath{clip}%
\pgfsetbuttcap%
\pgfsetroundjoin%
\pgfsetlinewidth{1.003750pt}%
\definecolor{currentstroke}{rgb}{0.000000,0.000000,0.000000}%
\pgfsetstrokecolor{currentstroke}%
\pgfsetdash{{3.700000pt}{1.600000pt}}{0.000000pt}%
\pgfpathmoveto{\pgfqpoint{0.000000in}{0.000000in}}%
\pgfusepath{stroke}%
\end{pgfscope}%
\begin{pgfscope}%
\pgfpathrectangle{\pgfqpoint{0.303000in}{0.248000in}}{\pgfqpoint{2.295902in}{1.362506in}} %
\pgfusepath{clip}%
\pgfsetbuttcap%
\pgfsetroundjoin%
\pgfsetlinewidth{1.003750pt}%
\definecolor{currentstroke}{rgb}{0.000000,0.000000,0.000000}%
\pgfsetstrokecolor{currentstroke}%
\pgfsetdash{{3.700000pt}{1.600000pt}}{0.000000pt}%
\pgfpathmoveto{\pgfqpoint{2.276477in}{0.238000in}}%
\pgfpathlineto{\pgfqpoint{2.279923in}{0.399578in}}%
\pgfpathlineto{\pgfqpoint{2.280370in}{0.519306in}}%
\pgfpathlineto{\pgfqpoint{2.278538in}{0.614222in}}%
\pgfpathlineto{\pgfqpoint{2.275039in}{0.688341in}}%
\pgfpathlineto{\pgfqpoint{2.269768in}{0.753900in}}%
\pgfpathlineto{\pgfqpoint{2.262921in}{0.810738in}}%
\pgfpathlineto{\pgfqpoint{2.254939in}{0.858748in}}%
\pgfpathlineto{\pgfqpoint{2.245548in}{0.901824in}}%
\pgfpathlineto{\pgfqpoint{2.235029in}{0.939841in}}%
\pgfpathlineto{\pgfqpoint{2.223800in}{0.972787in}}%
\pgfpathlineto{\pgfqpoint{2.210833in}{1.004206in}}%
\pgfpathlineto{\pgfqpoint{2.196143in}{1.033798in}}%
\pgfpathlineto{\pgfqpoint{2.179829in}{1.061331in}}%
\pgfpathlineto{\pgfqpoint{2.162054in}{1.086665in}}%
\pgfpathlineto{\pgfqpoint{2.143023in}{1.109771in}}%
\pgfpathlineto{\pgfqpoint{2.122946in}{1.130709in}}%
\pgfpathlineto{\pgfqpoint{2.099650in}{1.151590in}}%
\pgfpathlineto{\pgfqpoint{2.075537in}{1.170181in}}%
\pgfpathlineto{\pgfqpoint{2.048290in}{1.188277in}}%
\pgfpathlineto{\pgfqpoint{2.017916in}{1.205557in}}%
\pgfpathlineto{\pgfqpoint{1.987041in}{1.220622in}}%
\pgfpathlineto{\pgfqpoint{1.953176in}{1.234797in}}%
\pgfpathlineto{\pgfqpoint{1.913711in}{1.248793in}}%
\pgfpathlineto{\pgfqpoint{1.871285in}{1.261361in}}%
\pgfpathlineto{\pgfqpoint{1.825946in}{1.272456in}}%
\pgfpathlineto{\pgfqpoint{1.775052in}{1.282519in}}%
\pgfpathlineto{\pgfqpoint{1.721307in}{1.290829in}}%
\pgfpathlineto{\pgfqpoint{1.664748in}{1.297374in}}%
\pgfpathlineto{\pgfqpoint{1.602708in}{1.302285in}}%
\pgfpathlineto{\pgfqpoint{1.537911in}{1.305145in}}%
\pgfpathlineto{\pgfqpoint{1.470387in}{1.305861in}}%
\pgfpathlineto{\pgfqpoint{1.402869in}{1.304386in}}%
\pgfpathlineto{\pgfqpoint{1.332686in}{1.300586in}}%
\pgfpathlineto{\pgfqpoint{1.265270in}{1.294733in}}%
\pgfpathlineto{\pgfqpoint{1.197952in}{1.286652in}}%
\pgfpathlineto{\pgfqpoint{1.133457in}{1.276676in}}%
\pgfpathlineto{\pgfqpoint{1.071812in}{1.264918in}}%
\pgfpathlineto{\pgfqpoint{1.013045in}{1.251476in}}%
\pgfpathlineto{\pgfqpoint{0.957183in}{1.236441in}}%
\pgfpathlineto{\pgfqpoint{0.904258in}{1.219908in}}%
\pgfpathlineto{\pgfqpoint{0.854297in}{1.201985in}}%
\pgfpathlineto{\pgfqpoint{0.807329in}{1.182796in}}%
\pgfpathlineto{\pgfqpoint{0.763376in}{1.162491in}}%
\pgfpathlineto{\pgfqpoint{0.722456in}{1.141253in}}%
\pgfpathlineto{\pgfqpoint{0.682068in}{1.117754in}}%
\pgfpathlineto{\pgfqpoint{0.644793in}{1.093493in}}%
\pgfpathlineto{\pgfqpoint{0.610621in}{1.068748in}}%
\pgfpathlineto{\pgfqpoint{0.577158in}{1.041837in}}%
\pgfpathlineto{\pgfqpoint{0.544527in}{1.012637in}}%
\pgfpathlineto{\pgfqpoint{0.515092in}{0.983379in}}%
\pgfpathlineto{\pgfqpoint{0.486613in}{0.951990in}}%
\pgfpathlineto{\pgfqpoint{0.459212in}{0.918434in}}%
\pgfpathlineto{\pgfqpoint{0.433014in}{0.882706in}}%
\pgfpathlineto{\pgfqpoint{0.408134in}{0.844839in}}%
\pgfpathlineto{\pgfqpoint{0.384681in}{0.804903in}}%
\pgfpathlineto{\pgfqpoint{0.362744in}{0.763001in}}%
\pgfpathlineto{\pgfqpoint{0.342392in}{0.719270in}}%
\pgfpathlineto{\pgfqpoint{0.323673in}{0.673867in}}%
\pgfpathlineto{\pgfqpoint{0.306609in}{0.626964in}}%
\pgfpathlineto{\pgfqpoint{0.293000in}{0.584702in}}%
\pgfpathmoveto{\pgfqpoint{0.293000in}{0.908857in}}%
\pgfpathlineto{\pgfqpoint{0.316884in}{0.946850in}}%
\pgfpathlineto{\pgfqpoint{0.342548in}{0.983474in}}%
\pgfpathlineto{\pgfqpoint{0.369458in}{1.017945in}}%
\pgfpathlineto{\pgfqpoint{0.397498in}{1.050244in}}%
\pgfpathlineto{\pgfqpoint{0.426547in}{1.080387in}}%
\pgfpathlineto{\pgfqpoint{0.456487in}{1.108424in}}%
\pgfpathlineto{\pgfqpoint{0.489592in}{1.136350in}}%
\pgfpathlineto{\pgfqpoint{0.523463in}{1.162042in}}%
\pgfpathlineto{\pgfqpoint{0.560469in}{1.187246in}}%
\pgfpathlineto{\pgfqpoint{0.598091in}{1.210221in}}%
\pgfpathlineto{\pgfqpoint{0.638780in}{1.232472in}}%
\pgfpathlineto{\pgfqpoint{0.682527in}{1.253793in}}%
\pgfpathlineto{\pgfqpoint{0.729309in}{1.274018in}}%
\pgfpathlineto{\pgfqpoint{0.779098in}{1.293020in}}%
\pgfpathlineto{\pgfqpoint{0.831864in}{1.310706in}}%
\pgfpathlineto{\pgfqpoint{0.887573in}{1.327010in}}%
\pgfpathlineto{\pgfqpoint{0.946192in}{1.341889in}}%
\pgfpathlineto{\pgfqpoint{1.010370in}{1.355847in}}%
\pgfpathlineto{\pgfqpoint{1.077414in}{1.368156in}}%
\pgfpathlineto{\pgfqpoint{1.149984in}{1.379177in}}%
\pgfpathlineto{\pgfqpoint{1.225373in}{1.388372in}}%
\pgfpathlineto{\pgfqpoint{1.303555in}{1.395716in}}%
\pgfpathlineto{\pgfqpoint{1.384508in}{1.401146in}}%
\pgfpathlineto{\pgfqpoint{1.468210in}{1.404548in}}%
\pgfpathlineto{\pgfqpoint{1.551938in}{1.405740in}}%
\pgfpathlineto{\pgfqpoint{1.632967in}{1.404737in}}%
\pgfpathlineto{\pgfqpoint{1.711270in}{1.401616in}}%
\pgfpathlineto{\pgfqpoint{1.786821in}{1.396380in}}%
\pgfpathlineto{\pgfqpoint{1.856893in}{1.389274in}}%
\pgfpathlineto{\pgfqpoint{1.921459in}{1.380446in}}%
\pgfpathlineto{\pgfqpoint{1.977810in}{1.370570in}}%
\pgfpathlineto{\pgfqpoint{2.028618in}{1.359525in}}%
\pgfpathlineto{\pgfqpoint{2.073862in}{1.347560in}}%
\pgfpathlineto{\pgfqpoint{2.116167in}{1.334083in}}%
\pgfpathlineto{\pgfqpoint{2.152859in}{1.320114in}}%
\pgfpathlineto{\pgfqpoint{2.186528in}{1.304896in}}%
\pgfpathlineto{\pgfqpoint{2.214583in}{1.289955in}}%
\pgfpathlineto{\pgfqpoint{2.239597in}{1.274388in}}%
\pgfpathlineto{\pgfqpoint{2.263958in}{1.256572in}}%
\pgfpathlineto{\pgfqpoint{2.285113in}{1.238282in}}%
\pgfpathlineto{\pgfqpoint{2.303091in}{1.219970in}}%
\pgfpathlineto{\pgfqpoint{2.320051in}{1.199519in}}%
\pgfpathlineto{\pgfqpoint{2.335722in}{1.176796in}}%
\pgfpathlineto{\pgfqpoint{2.348153in}{1.155032in}}%
\pgfpathlineto{\pgfqpoint{2.359212in}{1.131603in}}%
\pgfpathlineto{\pgfqpoint{2.370027in}{1.103025in}}%
\pgfpathlineto{\pgfqpoint{2.378854in}{1.072892in}}%
\pgfpathlineto{\pgfqpoint{2.385778in}{1.041614in}}%
\pgfpathlineto{\pgfqpoint{2.391522in}{1.005519in}}%
\pgfpathlineto{\pgfqpoint{2.395753in}{0.964751in}}%
\pgfpathlineto{\pgfqpoint{2.398460in}{0.915389in}}%
\pgfpathlineto{\pgfqpoint{2.399196in}{0.857604in}}%
\pgfpathlineto{\pgfqpoint{2.397682in}{0.787459in}}%
\pgfpathlineto{\pgfqpoint{2.393295in}{0.696878in}}%
\pgfpathlineto{\pgfqpoint{2.384407in}{0.565459in}}%
\pgfpathlineto{\pgfqpoint{2.359372in}{0.238000in}}%
\pgfpathlineto{\pgfqpoint{2.359372in}{0.238000in}}%
\pgfusepath{stroke}%
\end{pgfscope}%
\begin{pgfscope}%
\pgfpathrectangle{\pgfqpoint{0.303000in}{0.248000in}}{\pgfqpoint{2.295902in}{1.362506in}} %
\pgfusepath{clip}%
\pgfsetbuttcap%
\pgfsetroundjoin%
\pgfsetlinewidth{1.003750pt}%
\definecolor{currentstroke}{rgb}{0.000000,0.000000,0.000000}%
\pgfsetstrokecolor{currentstroke}%
\pgfsetdash{{3.700000pt}{1.600000pt}}{0.000000pt}%
\pgfpathmoveto{\pgfqpoint{2.421105in}{0.238000in}}%
\pgfpathlineto{\pgfqpoint{2.438091in}{0.377738in}}%
\pgfpathlineto{\pgfqpoint{2.456997in}{0.515107in}}%
\pgfpathlineto{\pgfqpoint{2.478587in}{0.655794in}}%
\pgfpathlineto{\pgfqpoint{2.526510in}{0.960756in}}%
\pgfpathlineto{\pgfqpoint{2.534966in}{1.029736in}}%
\pgfpathlineto{\pgfqpoint{2.538895in}{1.078906in}}%
\pgfpathlineto{\pgfqpoint{2.540030in}{1.120144in}}%
\pgfpathlineto{\pgfqpoint{2.538922in}{1.153118in}}%
\pgfpathlineto{\pgfqpoint{2.536022in}{1.181664in}}%
\pgfpathlineto{\pgfqpoint{2.530946in}{1.209484in}}%
\pgfpathlineto{\pgfqpoint{2.524680in}{1.232314in}}%
\pgfpathlineto{\pgfqpoint{2.516632in}{1.253798in}}%
\pgfpathlineto{\pgfqpoint{2.506936in}{1.273630in}}%
\pgfpathlineto{\pgfqpoint{2.495832in}{1.291658in}}%
\pgfpathlineto{\pgfqpoint{2.481464in}{1.310423in}}%
\pgfpathlineto{\pgfqpoint{2.465951in}{1.326927in}}%
\pgfpathlineto{\pgfqpoint{2.447226in}{1.343394in}}%
\pgfpathlineto{\pgfqpoint{2.425286in}{1.359380in}}%
\pgfpathlineto{\pgfqpoint{2.400180in}{1.374589in}}%
\pgfpathlineto{\pgfqpoint{2.371973in}{1.388843in}}%
\pgfpathlineto{\pgfqpoint{2.340746in}{1.402106in}}%
\pgfpathlineto{\pgfqpoint{2.303917in}{1.415201in}}%
\pgfpathlineto{\pgfqpoint{2.261488in}{1.427737in}}%
\pgfpathlineto{\pgfqpoint{2.213477in}{1.439442in}}%
\pgfpathlineto{\pgfqpoint{2.159913in}{1.450142in}}%
\pgfpathlineto{\pgfqpoint{2.098132in}{1.460101in}}%
\pgfpathlineto{\pgfqpoint{2.030841in}{1.468685in}}%
\pgfpathlineto{\pgfqpoint{1.955367in}{1.476085in}}%
\pgfpathlineto{\pgfqpoint{1.871727in}{1.482064in}}%
\pgfpathlineto{\pgfqpoint{1.788034in}{1.485907in}}%
\pgfpathlineto{\pgfqpoint{1.690811in}{1.488472in}}%
\pgfpathlineto{\pgfqpoint{1.588172in}{1.488967in}}%
\pgfpathlineto{\pgfqpoint{1.482837in}{1.487227in}}%
\pgfpathlineto{\pgfqpoint{1.369428in}{1.483011in}}%
\pgfpathlineto{\pgfqpoint{1.269571in}{1.476842in}}%
\pgfpathlineto{\pgfqpoint{1.156299in}{1.467431in}}%
\pgfpathlineto{\pgfqpoint{1.070105in}{1.457600in}}%
\pgfpathlineto{\pgfqpoint{0.989415in}{1.446262in}}%
\pgfpathlineto{\pgfqpoint{0.914246in}{1.433539in}}%
\pgfpathlineto{\pgfqpoint{0.815147in}{1.414132in}}%
\pgfpathlineto{\pgfqpoint{0.735157in}{1.394486in}}%
\pgfpathlineto{\pgfqpoint{0.679494in}{1.377818in}}%
\pgfpathlineto{\pgfqpoint{0.626779in}{1.359782in}}%
\pgfpathlineto{\pgfqpoint{0.577051in}{1.340408in}}%
\pgfpathlineto{\pgfqpoint{0.530344in}{1.319783in}}%
\pgfpathlineto{\pgfqpoint{0.486699in}{1.297982in}}%
\pgfpathlineto{\pgfqpoint{0.443591in}{1.273824in}}%
\pgfpathlineto{\pgfqpoint{0.406173in}{1.250084in}}%
\pgfpathlineto{\pgfqpoint{0.371858in}{1.225807in}}%
\pgfpathlineto{\pgfqpoint{0.338249in}{1.199324in}}%
\pgfpathlineto{\pgfqpoint{0.307784in}{1.172642in}}%
\pgfpathlineto{\pgfqpoint{0.293000in}{1.158197in}}%
\pgfpathlineto{\pgfqpoint{0.293000in}{1.158197in}}%
\pgfusepath{stroke}%
\end{pgfscope}%
\begin{pgfscope}%
\pgfpathrectangle{\pgfqpoint{0.303000in}{0.248000in}}{\pgfqpoint{2.295902in}{1.362506in}} %
\pgfusepath{clip}%
\pgfsetbuttcap%
\pgfsetroundjoin%
\pgfsetlinewidth{1.003750pt}%
\definecolor{currentstroke}{rgb}{0.000000,0.000000,0.000000}%
\pgfsetstrokecolor{currentstroke}%
\pgfsetdash{{3.700000pt}{1.600000pt}}{0.000000pt}%
\pgfpathmoveto{\pgfqpoint{0.000000in}{0.000000in}}%
\pgfusepath{stroke}%
\end{pgfscope}%
\begin{pgfscope}%
\pgfpathrectangle{\pgfqpoint{0.303000in}{0.248000in}}{\pgfqpoint{2.295902in}{1.362506in}} %
\pgfusepath{clip}%
\pgfsetbuttcap%
\pgfsetroundjoin%
\pgfsetlinewidth{1.003750pt}%
\definecolor{currentstroke}{rgb}{0.000000,0.000000,0.000000}%
\pgfsetstrokecolor{currentstroke}%
\pgfsetdash{{3.700000pt}{1.600000pt}}{0.000000pt}%
\pgfpathmoveto{\pgfqpoint{2.420345in}{0.238000in}}%
\pgfpathlineto{\pgfqpoint{2.437242in}{0.377703in}}%
\pgfpathlineto{\pgfqpoint{2.456042in}{0.515107in}}%
\pgfpathlineto{\pgfqpoint{2.478123in}{0.659857in}}%
\pgfpathlineto{\pgfqpoint{2.525731in}{0.964993in}}%
\pgfpathlineto{\pgfqpoint{2.533444in}{1.029984in}}%
\pgfpathlineto{\pgfqpoint{2.537199in}{1.079185in}}%
\pgfpathlineto{\pgfqpoint{2.538142in}{1.120435in}}%
\pgfpathlineto{\pgfqpoint{2.536853in}{1.153393in}}%
\pgfpathlineto{\pgfqpoint{2.533791in}{1.181899in}}%
\pgfpathlineto{\pgfqpoint{2.528540in}{1.209643in}}%
\pgfpathlineto{\pgfqpoint{2.522131in}{1.232380in}}%
\pgfpathlineto{\pgfqpoint{2.513955in}{1.253752in}}%
\pgfpathlineto{\pgfqpoint{2.504157in}{1.273466in}}%
\pgfpathlineto{\pgfqpoint{2.492973in}{1.291378in}}%
\pgfpathlineto{\pgfqpoint{2.478540in}{1.310027in}}%
\pgfpathlineto{\pgfqpoint{2.462985in}{1.326441in}}%
\pgfpathlineto{\pgfqpoint{2.444233in}{1.342835in}}%
\pgfpathlineto{\pgfqpoint{2.422273in}{1.358753in}}%
\pgfpathlineto{\pgfqpoint{2.397151in}{1.373905in}}%
\pgfpathlineto{\pgfqpoint{2.368937in}{1.388125in}}%
\pgfpathlineto{\pgfqpoint{2.337705in}{1.401359in}}%
\pgfpathlineto{\pgfqpoint{2.300872in}{1.414430in}}%
\pgfpathlineto{\pgfqpoint{2.258440in}{1.426948in}}%
\pgfpathlineto{\pgfqpoint{2.210428in}{1.438638in}}%
\pgfpathlineto{\pgfqpoint{2.156863in}{1.449325in}}%
\pgfpathlineto{\pgfqpoint{2.095082in}{1.459273in}}%
\pgfpathlineto{\pgfqpoint{2.027789in}{1.467844in}}%
\pgfpathlineto{\pgfqpoint{1.952315in}{1.475229in}}%
\pgfpathlineto{\pgfqpoint{1.868674in}{1.481188in}}%
\pgfpathlineto{\pgfqpoint{1.784981in}{1.485032in}}%
\pgfpathlineto{\pgfqpoint{1.687757in}{1.487560in}}%
\pgfpathlineto{\pgfqpoint{1.585118in}{1.488009in}}%
\pgfpathlineto{\pgfqpoint{1.479784in}{1.486208in}}%
\pgfpathlineto{\pgfqpoint{1.369076in}{1.482025in}}%
\pgfpathlineto{\pgfqpoint{1.269220in}{1.475827in}}%
\pgfpathlineto{\pgfqpoint{1.158646in}{1.466578in}}%
\pgfpathlineto{\pgfqpoint{1.072452in}{1.456762in}}%
\pgfpathlineto{\pgfqpoint{0.991760in}{1.445436in}}%
\pgfpathlineto{\pgfqpoint{0.916590in}{1.432723in}}%
\pgfpathlineto{\pgfqpoint{0.820167in}{1.413778in}}%
\pgfpathlineto{\pgfqpoint{0.737502in}{1.393564in}}%
\pgfpathlineto{\pgfqpoint{0.681840in}{1.376892in}}%
\pgfpathlineto{\pgfqpoint{0.629125in}{1.358854in}}%
\pgfpathlineto{\pgfqpoint{0.579397in}{1.339477in}}%
\pgfpathlineto{\pgfqpoint{0.532691in}{1.318850in}}%
\pgfpathlineto{\pgfqpoint{0.491599in}{1.298399in}}%
\pgfpathlineto{\pgfqpoint{0.448448in}{1.274418in}}%
\pgfpathlineto{\pgfqpoint{0.410989in}{1.250828in}}%
\pgfpathlineto{\pgfqpoint{0.376628in}{1.226704in}}%
\pgfpathlineto{\pgfqpoint{0.342964in}{1.200385in}}%
\pgfpathlineto{\pgfqpoint{0.310123in}{1.171738in}}%
\pgfpathlineto{\pgfqpoint{0.293000in}{1.154609in}}%
\pgfpathlineto{\pgfqpoint{0.293000in}{1.154609in}}%
\pgfusepath{stroke}%
\end{pgfscope}%
\begin{pgfscope}%
\pgfpathrectangle{\pgfqpoint{0.303000in}{0.248000in}}{\pgfqpoint{2.295902in}{1.362506in}} %
\pgfusepath{clip}%
\pgfsetbuttcap%
\pgfsetroundjoin%
\pgfsetlinewidth{1.003750pt}%
\definecolor{currentstroke}{rgb}{0.000000,0.000000,0.000000}%
\pgfsetstrokecolor{currentstroke}%
\pgfsetdash{{3.700000pt}{1.600000pt}}{0.000000pt}%
\pgfpathmoveto{\pgfqpoint{0.000000in}{0.000000in}}%
\pgfusepath{stroke}%
\end{pgfscope}%
\begin{pgfscope}%
\pgfsetrectcap%
\pgfsetmiterjoin%
\pgfsetlinewidth{0.803000pt}%
\definecolor{currentstroke}{rgb}{0.000000,0.000000,0.000000}%
\pgfsetstrokecolor{currentstroke}%
\pgfsetdash{}{0pt}%
\pgfpathmoveto{\pgfqpoint{0.303000in}{0.248000in}}%
\pgfpathlineto{\pgfqpoint{0.303000in}{1.610506in}}%
\pgfusepath{stroke}%
\end{pgfscope}%
\begin{pgfscope}%
\pgfsetrectcap%
\pgfsetmiterjoin%
\pgfsetlinewidth{0.803000pt}%
\definecolor{currentstroke}{rgb}{0.000000,0.000000,0.000000}%
\pgfsetstrokecolor{currentstroke}%
\pgfsetdash{}{0pt}%
\pgfpathmoveto{\pgfqpoint{2.598902in}{0.248000in}}%
\pgfpathlineto{\pgfqpoint{2.598902in}{1.610506in}}%
\pgfusepath{stroke}%
\end{pgfscope}%
\begin{pgfscope}%
\pgfsetrectcap%
\pgfsetmiterjoin%
\pgfsetlinewidth{0.803000pt}%
\definecolor{currentstroke}{rgb}{0.000000,0.000000,0.000000}%
\pgfsetstrokecolor{currentstroke}%
\pgfsetdash{}{0pt}%
\pgfpathmoveto{\pgfqpoint{0.303000in}{0.248000in}}%
\pgfpathlineto{\pgfqpoint{2.598902in}{0.248000in}}%
\pgfusepath{stroke}%
\end{pgfscope}%
\begin{pgfscope}%
\pgfsetrectcap%
\pgfsetmiterjoin%
\pgfsetlinewidth{0.803000pt}%
\definecolor{currentstroke}{rgb}{0.000000,0.000000,0.000000}%
\pgfsetstrokecolor{currentstroke}%
\pgfsetdash{}{0pt}%
\pgfpathmoveto{\pgfqpoint{0.303000in}{1.610506in}}%
\pgfpathlineto{\pgfqpoint{2.598902in}{1.610506in}}%
\pgfusepath{stroke}%
\end{pgfscope}%
\begin{pgfscope}%
\pgfsetbuttcap%
\pgfsetmiterjoin%
\definecolor{currentfill}{rgb}{1.000000,1.000000,1.000000}%
\pgfsetfillcolor{currentfill}%
\pgfsetfillopacity{0.800000}%
\pgfsetlinewidth{1.003750pt}%
\definecolor{currentstroke}{rgb}{0.800000,0.800000,0.800000}%
\pgfsetstrokecolor{currentstroke}%
\pgfsetstrokeopacity{0.800000}%
\pgfsetdash{}{0pt}%
\pgfpathmoveto{\pgfqpoint{1.051002in}{0.317444in}}%
\pgfpathlineto{\pgfqpoint{1.850900in}{0.317444in}}%
\pgfpathquadraticcurveto{\pgfqpoint{1.878678in}{0.317444in}}{\pgfqpoint{1.878678in}{0.345222in}}%
\pgfpathlineto{\pgfqpoint{1.878678in}{0.730222in}}%
\pgfpathquadraticcurveto{\pgfqpoint{1.878678in}{0.758000in}}{\pgfqpoint{1.850900in}{0.758000in}}%
\pgfpathlineto{\pgfqpoint{1.051002in}{0.758000in}}%
\pgfpathquadraticcurveto{\pgfqpoint{1.023224in}{0.758000in}}{\pgfqpoint{1.023224in}{0.730222in}}%
\pgfpathlineto{\pgfqpoint{1.023224in}{0.345222in}}%
\pgfpathquadraticcurveto{\pgfqpoint{1.023224in}{0.317444in}}{\pgfqpoint{1.051002in}{0.317444in}}%
\pgfpathclose%
\pgfusepath{stroke,fill}%
\end{pgfscope}%
\begin{pgfscope}%
\pgfsetrectcap%
\pgfsetroundjoin%
\pgfsetlinewidth{1.003750pt}%
\definecolor{currentstroke}{rgb}{0.839216,0.152941,0.156863}%
\pgfsetstrokecolor{currentstroke}%
\pgfsetdash{}{0pt}%
\pgfpathmoveto{\pgfqpoint{1.078780in}{0.653833in}}%
\pgfpathlineto{\pgfqpoint{1.148224in}{0.653833in}}%
\pgfusepath{stroke}%
\end{pgfscope}%
\begin{pgfscope}%
\pgftext[x=1.259335in,y=0.605222in,left,base]{\rmfamily\fontsize{10.000000}{12.000000}\selectfont \(\displaystyle \textnormal{tol}=10^{-7}\)}%
\end{pgfscope}%
\begin{pgfscope}%
\pgfsetbuttcap%
\pgfsetroundjoin%
\pgfsetlinewidth{1.003750pt}%
\definecolor{currentstroke}{rgb}{0.000000,0.000000,0.000000}%
\pgfsetstrokecolor{currentstroke}%
\pgfsetdash{{3.700000pt}{1.600000pt}}{0.000000pt}%
\pgfpathmoveto{\pgfqpoint{1.078780in}{0.454389in}}%
\pgfpathlineto{\pgfqpoint{1.148224in}{0.454389in}}%
\pgfusepath{stroke}%
\end{pgfscope}%
\begin{pgfscope}%
\pgftext[x=1.259335in,y=0.405778in,left,base]{\rmfamily\fontsize{10.000000}{12.000000}\selectfont Reference}%
\end{pgfscope}%
\begin{pgfscope}%
\pgfsetbuttcap%
\pgfsetmiterjoin%
\definecolor{currentfill}{rgb}{1.000000,1.000000,1.000000}%
\pgfsetfillcolor{currentfill}%
\pgfsetlinewidth{0.000000pt}%
\definecolor{currentstroke}{rgb}{0.000000,0.000000,0.000000}%
\pgfsetstrokecolor{currentstroke}%
\pgfsetstrokeopacity{0.000000}%
\pgfsetdash{}{0pt}%
\pgfpathmoveto{\pgfqpoint{2.713698in}{0.248000in}}%
\pgfpathlineto{\pgfqpoint{5.009600in}{0.248000in}}%
\pgfpathlineto{\pgfqpoint{5.009600in}{1.610506in}}%
\pgfpathlineto{\pgfqpoint{2.713698in}{1.610506in}}%
\pgfpathclose%
\pgfusepath{fill}%
\end{pgfscope}%
\begin{pgfscope}%
\pgfsetbuttcap%
\pgfsetroundjoin%
\definecolor{currentfill}{rgb}{0.000000,0.000000,0.000000}%
\pgfsetfillcolor{currentfill}%
\pgfsetlinewidth{0.803000pt}%
\definecolor{currentstroke}{rgb}{0.000000,0.000000,0.000000}%
\pgfsetstrokecolor{currentstroke}%
\pgfsetdash{}{0pt}%
\pgfsys@defobject{currentmarker}{\pgfqpoint{0.000000in}{-0.048611in}}{\pgfqpoint{0.000000in}{0.000000in}}{%
\pgfpathmoveto{\pgfqpoint{0.000000in}{0.000000in}}%
\pgfpathlineto{\pgfqpoint{0.000000in}{-0.048611in}}%
\pgfusepath{stroke,fill}%
}%
\begin{pgfscope}%
\pgfsys@transformshift{2.983804in}{0.248000in}%
\pgfsys@useobject{currentmarker}{}%
\end{pgfscope}%
\end{pgfscope}%
\begin{pgfscope}%
\pgftext[x=2.983804in,y=0.150778in,,top]{\rmfamily\fontsize{10.000000}{12.000000}\selectfont \(\displaystyle 0.4\)}%
\end{pgfscope}%
\begin{pgfscope}%
\pgfsetbuttcap%
\pgfsetroundjoin%
\definecolor{currentfill}{rgb}{0.000000,0.000000,0.000000}%
\pgfsetfillcolor{currentfill}%
\pgfsetlinewidth{0.803000pt}%
\definecolor{currentstroke}{rgb}{0.000000,0.000000,0.000000}%
\pgfsetstrokecolor{currentstroke}%
\pgfsetdash{}{0pt}%
\pgfsys@defobject{currentmarker}{\pgfqpoint{0.000000in}{-0.048611in}}{\pgfqpoint{0.000000in}{0.000000in}}{%
\pgfpathmoveto{\pgfqpoint{0.000000in}{0.000000in}}%
\pgfpathlineto{\pgfqpoint{0.000000in}{-0.048611in}}%
\pgfusepath{stroke,fill}%
}%
\begin{pgfscope}%
\pgfsys@transformshift{3.524016in}{0.248000in}%
\pgfsys@useobject{currentmarker}{}%
\end{pgfscope}%
\end{pgfscope}%
\begin{pgfscope}%
\pgftext[x=3.524016in,y=0.150778in,,top]{\rmfamily\fontsize{10.000000}{12.000000}\selectfont \(\displaystyle 0.6\)}%
\end{pgfscope}%
\begin{pgfscope}%
\pgfsetbuttcap%
\pgfsetroundjoin%
\definecolor{currentfill}{rgb}{0.000000,0.000000,0.000000}%
\pgfsetfillcolor{currentfill}%
\pgfsetlinewidth{0.803000pt}%
\definecolor{currentstroke}{rgb}{0.000000,0.000000,0.000000}%
\pgfsetstrokecolor{currentstroke}%
\pgfsetdash{}{0pt}%
\pgfsys@defobject{currentmarker}{\pgfqpoint{0.000000in}{-0.048611in}}{\pgfqpoint{0.000000in}{0.000000in}}{%
\pgfpathmoveto{\pgfqpoint{0.000000in}{0.000000in}}%
\pgfpathlineto{\pgfqpoint{0.000000in}{-0.048611in}}%
\pgfusepath{stroke,fill}%
}%
\begin{pgfscope}%
\pgfsys@transformshift{4.064228in}{0.248000in}%
\pgfsys@useobject{currentmarker}{}%
\end{pgfscope}%
\end{pgfscope}%
\begin{pgfscope}%
\pgftext[x=4.064228in,y=0.150778in,,top]{\rmfamily\fontsize{10.000000}{12.000000}\selectfont \(\displaystyle 0.8\)}%
\end{pgfscope}%
\begin{pgfscope}%
\pgfsetbuttcap%
\pgfsetroundjoin%
\definecolor{currentfill}{rgb}{0.000000,0.000000,0.000000}%
\pgfsetfillcolor{currentfill}%
\pgfsetlinewidth{0.803000pt}%
\definecolor{currentstroke}{rgb}{0.000000,0.000000,0.000000}%
\pgfsetstrokecolor{currentstroke}%
\pgfsetdash{}{0pt}%
\pgfsys@defobject{currentmarker}{\pgfqpoint{0.000000in}{-0.048611in}}{\pgfqpoint{0.000000in}{0.000000in}}{%
\pgfpathmoveto{\pgfqpoint{0.000000in}{0.000000in}}%
\pgfpathlineto{\pgfqpoint{0.000000in}{-0.048611in}}%
\pgfusepath{stroke,fill}%
}%
\begin{pgfscope}%
\pgfsys@transformshift{4.604441in}{0.248000in}%
\pgfsys@useobject{currentmarker}{}%
\end{pgfscope}%
\end{pgfscope}%
\begin{pgfscope}%
\pgftext[x=4.604441in,y=0.150778in,,top]{\rmfamily\fontsize{10.000000}{12.000000}\selectfont \(\displaystyle 1.0\)}%
\end{pgfscope}%
\begin{pgfscope}%
\pgfsetbuttcap%
\pgfsetroundjoin%
\definecolor{currentfill}{rgb}{0.000000,0.000000,0.000000}%
\pgfsetfillcolor{currentfill}%
\pgfsetlinewidth{0.803000pt}%
\definecolor{currentstroke}{rgb}{0.000000,0.000000,0.000000}%
\pgfsetstrokecolor{currentstroke}%
\pgfsetdash{}{0pt}%
\pgfsys@defobject{currentmarker}{\pgfqpoint{-0.048611in}{0.000000in}}{\pgfqpoint{0.000000in}{0.000000in}}{%
\pgfpathmoveto{\pgfqpoint{0.000000in}{0.000000in}}%
\pgfpathlineto{\pgfqpoint{-0.048611in}{0.000000in}}%
\pgfusepath{stroke,fill}%
}%
\begin{pgfscope}%
\pgfsys@transformshift{2.713698in}{0.454440in}%
\pgfsys@useobject{currentmarker}{}%
\end{pgfscope}%
\end{pgfscope}%
\begin{pgfscope}%
\pgfsetbuttcap%
\pgfsetroundjoin%
\definecolor{currentfill}{rgb}{0.000000,0.000000,0.000000}%
\pgfsetfillcolor{currentfill}%
\pgfsetlinewidth{0.803000pt}%
\definecolor{currentstroke}{rgb}{0.000000,0.000000,0.000000}%
\pgfsetstrokecolor{currentstroke}%
\pgfsetdash{}{0pt}%
\pgfsys@defobject{currentmarker}{\pgfqpoint{-0.048611in}{0.000000in}}{\pgfqpoint{0.000000in}{0.000000in}}{%
\pgfpathmoveto{\pgfqpoint{0.000000in}{0.000000in}}%
\pgfpathlineto{\pgfqpoint{-0.048611in}{0.000000in}}%
\pgfusepath{stroke,fill}%
}%
\begin{pgfscope}%
\pgfsys@transformshift{2.713698in}{0.867321in}%
\pgfsys@useobject{currentmarker}{}%
\end{pgfscope}%
\end{pgfscope}%
\begin{pgfscope}%
\pgfsetbuttcap%
\pgfsetroundjoin%
\definecolor{currentfill}{rgb}{0.000000,0.000000,0.000000}%
\pgfsetfillcolor{currentfill}%
\pgfsetlinewidth{0.803000pt}%
\definecolor{currentstroke}{rgb}{0.000000,0.000000,0.000000}%
\pgfsetstrokecolor{currentstroke}%
\pgfsetdash{}{0pt}%
\pgfsys@defobject{currentmarker}{\pgfqpoint{-0.048611in}{0.000000in}}{\pgfqpoint{0.000000in}{0.000000in}}{%
\pgfpathmoveto{\pgfqpoint{0.000000in}{0.000000in}}%
\pgfpathlineto{\pgfqpoint{-0.048611in}{0.000000in}}%
\pgfusepath{stroke,fill}%
}%
\begin{pgfscope}%
\pgfsys@transformshift{2.713698in}{1.280202in}%
\pgfsys@useobject{currentmarker}{}%
\end{pgfscope}%
\end{pgfscope}%
\begin{pgfscope}%
\pgfpathrectangle{\pgfqpoint{2.713698in}{0.248000in}}{\pgfqpoint{2.295902in}{1.362506in}} %
\pgfusepath{clip}%
\pgfsetrectcap%
\pgfsetroundjoin%
\pgfsetlinewidth{1.505625pt}%
\definecolor{currentstroke}{rgb}{0.121569,0.466667,0.705882}%
\pgfsetstrokecolor{currentstroke}%
\pgfsetdash{}{0pt}%
\pgfusepath{stroke}%
\end{pgfscope}%
\begin{pgfscope}%
\pgfpathrectangle{\pgfqpoint{2.713698in}{0.248000in}}{\pgfqpoint{2.295902in}{1.362506in}} %
\pgfusepath{clip}%
\pgfsetrectcap%
\pgfsetroundjoin%
\pgfsetlinewidth{1.505625pt}%
\definecolor{currentstroke}{rgb}{1.000000,0.498039,0.054902}%
\pgfsetstrokecolor{currentstroke}%
\pgfsetdash{}{0pt}%
\pgfusepath{stroke}%
\end{pgfscope}%
\begin{pgfscope}%
\pgfpathrectangle{\pgfqpoint{2.713698in}{0.248000in}}{\pgfqpoint{2.295902in}{1.362506in}} %
\pgfusepath{clip}%
\pgfsetrectcap%
\pgfsetroundjoin%
\pgfsetlinewidth{1.003750pt}%
\definecolor{currentstroke}{rgb}{0.172549,0.627451,0.172549}%
\pgfsetstrokecolor{currentstroke}%
\pgfsetdash{}{0pt}%
\pgfpathmoveto{\pgfqpoint{4.687175in}{0.238000in}}%
\pgfpathlineto{\pgfqpoint{4.690620in}{0.399578in}}%
\pgfpathlineto{\pgfqpoint{4.691068in}{0.519306in}}%
\pgfpathlineto{\pgfqpoint{4.689236in}{0.614221in}}%
\pgfpathlineto{\pgfqpoint{4.685737in}{0.688341in}}%
\pgfpathlineto{\pgfqpoint{4.680465in}{0.753900in}}%
\pgfpathlineto{\pgfqpoint{4.673619in}{0.810738in}}%
\pgfpathlineto{\pgfqpoint{4.665637in}{0.858748in}}%
\pgfpathlineto{\pgfqpoint{4.656245in}{0.901824in}}%
\pgfpathlineto{\pgfqpoint{4.645727in}{0.939841in}}%
\pgfpathlineto{\pgfqpoint{4.634498in}{0.972787in}}%
\pgfpathlineto{\pgfqpoint{4.621531in}{1.004205in}}%
\pgfpathlineto{\pgfqpoint{4.606841in}{1.033798in}}%
\pgfpathlineto{\pgfqpoint{4.590526in}{1.061331in}}%
\pgfpathlineto{\pgfqpoint{4.572752in}{1.086665in}}%
\pgfpathlineto{\pgfqpoint{4.553721in}{1.109771in}}%
\pgfpathlineto{\pgfqpoint{4.533644in}{1.130709in}}%
\pgfpathlineto{\pgfqpoint{4.510348in}{1.151590in}}%
\pgfpathlineto{\pgfqpoint{4.486235in}{1.170181in}}%
\pgfpathlineto{\pgfqpoint{4.458987in}{1.188277in}}%
\pgfpathlineto{\pgfqpoint{4.428614in}{1.205557in}}%
\pgfpathlineto{\pgfqpoint{4.397738in}{1.220622in}}%
\pgfpathlineto{\pgfqpoint{4.363874in}{1.234797in}}%
\pgfpathlineto{\pgfqpoint{4.324408in}{1.248793in}}%
\pgfpathlineto{\pgfqpoint{4.281983in}{1.261361in}}%
\pgfpathlineto{\pgfqpoint{4.236644in}{1.272456in}}%
\pgfpathlineto{\pgfqpoint{4.185750in}{1.282519in}}%
\pgfpathlineto{\pgfqpoint{4.132004in}{1.290829in}}%
\pgfpathlineto{\pgfqpoint{4.075445in}{1.297374in}}%
\pgfpathlineto{\pgfqpoint{4.013405in}{1.302285in}}%
\pgfpathlineto{\pgfqpoint{3.948608in}{1.305145in}}%
\pgfpathlineto{\pgfqpoint{3.881085in}{1.305861in}}%
\pgfpathlineto{\pgfqpoint{3.813566in}{1.304386in}}%
\pgfpathlineto{\pgfqpoint{3.743384in}{1.300586in}}%
\pgfpathlineto{\pgfqpoint{3.675967in}{1.294733in}}%
\pgfpathlineto{\pgfqpoint{3.608650in}{1.286652in}}%
\pgfpathlineto{\pgfqpoint{3.544155in}{1.276676in}}%
\pgfpathlineto{\pgfqpoint{3.482510in}{1.264918in}}%
\pgfpathlineto{\pgfqpoint{3.423742in}{1.251476in}}%
\pgfpathlineto{\pgfqpoint{3.367881in}{1.236441in}}%
\pgfpathlineto{\pgfqpoint{3.314955in}{1.219908in}}%
\pgfpathlineto{\pgfqpoint{3.264995in}{1.201985in}}%
\pgfpathlineto{\pgfqpoint{3.218026in}{1.182796in}}%
\pgfpathlineto{\pgfqpoint{3.174074in}{1.162491in}}%
\pgfpathlineto{\pgfqpoint{3.133153in}{1.141253in}}%
\pgfpathlineto{\pgfqpoint{3.092766in}{1.117754in}}%
\pgfpathlineto{\pgfqpoint{3.055491in}{1.093493in}}%
\pgfpathlineto{\pgfqpoint{3.021319in}{1.068748in}}%
\pgfpathlineto{\pgfqpoint{2.987856in}{1.041837in}}%
\pgfpathlineto{\pgfqpoint{2.955225in}{1.012637in}}%
\pgfpathlineto{\pgfqpoint{2.925790in}{0.983379in}}%
\pgfpathlineto{\pgfqpoint{2.897311in}{0.951990in}}%
\pgfpathlineto{\pgfqpoint{2.869910in}{0.918434in}}%
\pgfpathlineto{\pgfqpoint{2.843711in}{0.882707in}}%
\pgfpathlineto{\pgfqpoint{2.818832in}{0.844840in}}%
\pgfpathlineto{\pgfqpoint{2.795379in}{0.804903in}}%
\pgfpathlineto{\pgfqpoint{2.773442in}{0.763001in}}%
\pgfpathlineto{\pgfqpoint{2.753090in}{0.719270in}}%
\pgfpathlineto{\pgfqpoint{2.734370in}{0.673867in}}%
\pgfpathlineto{\pgfqpoint{2.717306in}{0.626964in}}%
\pgfpathlineto{\pgfqpoint{2.703698in}{0.584702in}}%
\pgfpathmoveto{\pgfqpoint{2.703698in}{0.908857in}}%
\pgfpathlineto{\pgfqpoint{2.727582in}{0.946850in}}%
\pgfpathlineto{\pgfqpoint{2.753246in}{0.983474in}}%
\pgfpathlineto{\pgfqpoint{2.780155in}{1.017945in}}%
\pgfpathlineto{\pgfqpoint{2.808195in}{1.050244in}}%
\pgfpathlineto{\pgfqpoint{2.837245in}{1.080387in}}%
\pgfpathlineto{\pgfqpoint{2.867184in}{1.108424in}}%
\pgfpathlineto{\pgfqpoint{2.900289in}{1.136350in}}%
\pgfpathlineto{\pgfqpoint{2.934160in}{1.162042in}}%
\pgfpathlineto{\pgfqpoint{2.971166in}{1.187246in}}%
\pgfpathlineto{\pgfqpoint{3.008789in}{1.210221in}}%
\pgfpathlineto{\pgfqpoint{3.049478in}{1.232472in}}%
\pgfpathlineto{\pgfqpoint{3.093224in}{1.253793in}}%
\pgfpathlineto{\pgfqpoint{3.140006in}{1.274018in}}%
\pgfpathlineto{\pgfqpoint{3.189796in}{1.293020in}}%
\pgfpathlineto{\pgfqpoint{3.242562in}{1.310706in}}%
\pgfpathlineto{\pgfqpoint{3.298271in}{1.327010in}}%
\pgfpathlineto{\pgfqpoint{3.356890in}{1.341889in}}%
\pgfpathlineto{\pgfqpoint{3.421068in}{1.355847in}}%
\pgfpathlineto{\pgfqpoint{3.488111in}{1.368156in}}%
\pgfpathlineto{\pgfqpoint{3.560681in}{1.379177in}}%
\pgfpathlineto{\pgfqpoint{3.636070in}{1.388372in}}%
\pgfpathlineto{\pgfqpoint{3.714253in}{1.395716in}}%
\pgfpathlineto{\pgfqpoint{3.795206in}{1.401146in}}%
\pgfpathlineto{\pgfqpoint{3.878908in}{1.404548in}}%
\pgfpathlineto{\pgfqpoint{3.962636in}{1.405740in}}%
\pgfpathlineto{\pgfqpoint{4.043664in}{1.404737in}}%
\pgfpathlineto{\pgfqpoint{4.121967in}{1.401616in}}%
\pgfpathlineto{\pgfqpoint{4.197518in}{1.396380in}}%
\pgfpathlineto{\pgfqpoint{4.267590in}{1.389274in}}%
\pgfpathlineto{\pgfqpoint{4.332156in}{1.380446in}}%
\pgfpathlineto{\pgfqpoint{4.388508in}{1.370570in}}%
\pgfpathlineto{\pgfqpoint{4.439315in}{1.359525in}}%
\pgfpathlineto{\pgfqpoint{4.484559in}{1.347560in}}%
\pgfpathlineto{\pgfqpoint{4.526865in}{1.334083in}}%
\pgfpathlineto{\pgfqpoint{4.563556in}{1.320114in}}%
\pgfpathlineto{\pgfqpoint{4.597226in}{1.304896in}}%
\pgfpathlineto{\pgfqpoint{4.625280in}{1.289955in}}%
\pgfpathlineto{\pgfqpoint{4.650294in}{1.274388in}}%
\pgfpathlineto{\pgfqpoint{4.674655in}{1.256572in}}%
\pgfpathlineto{\pgfqpoint{4.695810in}{1.238282in}}%
\pgfpathlineto{\pgfqpoint{4.713788in}{1.219970in}}%
\pgfpathlineto{\pgfqpoint{4.730748in}{1.199519in}}%
\pgfpathlineto{\pgfqpoint{4.746419in}{1.176796in}}%
\pgfpathlineto{\pgfqpoint{4.758850in}{1.155032in}}%
\pgfpathlineto{\pgfqpoint{4.769909in}{1.131603in}}%
\pgfpathlineto{\pgfqpoint{4.780724in}{1.103025in}}%
\pgfpathlineto{\pgfqpoint{4.789551in}{1.072892in}}%
\pgfpathlineto{\pgfqpoint{4.796475in}{1.041614in}}%
\pgfpathlineto{\pgfqpoint{4.802219in}{1.005519in}}%
\pgfpathlineto{\pgfqpoint{4.806451in}{0.964751in}}%
\pgfpathlineto{\pgfqpoint{4.809158in}{0.915389in}}%
\pgfpathlineto{\pgfqpoint{4.809893in}{0.857604in}}%
\pgfpathlineto{\pgfqpoint{4.808380in}{0.787459in}}%
\pgfpathlineto{\pgfqpoint{4.803993in}{0.696878in}}%
\pgfpathlineto{\pgfqpoint{4.795104in}{0.565459in}}%
\pgfpathlineto{\pgfqpoint{4.770069in}{0.238000in}}%
\pgfpathlineto{\pgfqpoint{4.770069in}{0.238000in}}%
\pgfusepath{stroke}%
\end{pgfscope}%
\begin{pgfscope}%
\pgfpathrectangle{\pgfqpoint{2.713698in}{0.248000in}}{\pgfqpoint{2.295902in}{1.362506in}} %
\pgfusepath{clip}%
\pgfsetrectcap%
\pgfsetroundjoin%
\pgfsetlinewidth{1.003750pt}%
\definecolor{currentstroke}{rgb}{0.172549,0.627451,0.172549}%
\pgfsetstrokecolor{currentstroke}%
\pgfsetdash{}{0pt}%
\pgfpathmoveto{\pgfqpoint{4.831803in}{0.238000in}}%
\pgfpathlineto{\pgfqpoint{4.848789in}{0.377738in}}%
\pgfpathlineto{\pgfqpoint{4.867695in}{0.515107in}}%
\pgfpathlineto{\pgfqpoint{4.889284in}{0.655794in}}%
\pgfpathlineto{\pgfqpoint{4.937208in}{0.960757in}}%
\pgfpathlineto{\pgfqpoint{4.945664in}{1.029736in}}%
\pgfpathlineto{\pgfqpoint{4.949593in}{1.078906in}}%
\pgfpathlineto{\pgfqpoint{4.950727in}{1.120144in}}%
\pgfpathlineto{\pgfqpoint{4.949619in}{1.153118in}}%
\pgfpathlineto{\pgfqpoint{4.946720in}{1.181664in}}%
\pgfpathlineto{\pgfqpoint{4.941643in}{1.209484in}}%
\pgfpathlineto{\pgfqpoint{4.935378in}{1.232314in}}%
\pgfpathlineto{\pgfqpoint{4.927330in}{1.253798in}}%
\pgfpathlineto{\pgfqpoint{4.917634in}{1.273630in}}%
\pgfpathlineto{\pgfqpoint{4.906530in}{1.291658in}}%
\pgfpathlineto{\pgfqpoint{4.892162in}{1.310423in}}%
\pgfpathlineto{\pgfqpoint{4.876648in}{1.326927in}}%
\pgfpathlineto{\pgfqpoint{4.857923in}{1.343394in}}%
\pgfpathlineto{\pgfqpoint{4.835984in}{1.359380in}}%
\pgfpathlineto{\pgfqpoint{4.810878in}{1.374589in}}%
\pgfpathlineto{\pgfqpoint{4.782671in}{1.388843in}}%
\pgfpathlineto{\pgfqpoint{4.751444in}{1.402106in}}%
\pgfpathlineto{\pgfqpoint{4.714615in}{1.415201in}}%
\pgfpathlineto{\pgfqpoint{4.672185in}{1.427737in}}%
\pgfpathlineto{\pgfqpoint{4.624175in}{1.439442in}}%
\pgfpathlineto{\pgfqpoint{4.570610in}{1.450142in}}%
\pgfpathlineto{\pgfqpoint{4.508830in}{1.460101in}}%
\pgfpathlineto{\pgfqpoint{4.441539in}{1.468685in}}%
\pgfpathlineto{\pgfqpoint{4.366065in}{1.476085in}}%
\pgfpathlineto{\pgfqpoint{4.282424in}{1.482064in}}%
\pgfpathlineto{\pgfqpoint{4.198731in}{1.485907in}}%
\pgfpathlineto{\pgfqpoint{4.101508in}{1.488472in}}%
\pgfpathlineto{\pgfqpoint{3.998869in}{1.488967in}}%
\pgfpathlineto{\pgfqpoint{3.893535in}{1.487227in}}%
\pgfpathlineto{\pgfqpoint{3.780126in}{1.483011in}}%
\pgfpathlineto{\pgfqpoint{3.680269in}{1.476842in}}%
\pgfpathlineto{\pgfqpoint{3.566996in}{1.467431in}}%
\pgfpathlineto{\pgfqpoint{3.480803in}{1.457600in}}%
\pgfpathlineto{\pgfqpoint{3.400112in}{1.446262in}}%
\pgfpathlineto{\pgfqpoint{3.324943in}{1.433539in}}%
\pgfpathlineto{\pgfqpoint{3.225845in}{1.414132in}}%
\pgfpathlineto{\pgfqpoint{3.145855in}{1.394486in}}%
\pgfpathlineto{\pgfqpoint{3.090192in}{1.377818in}}%
\pgfpathlineto{\pgfqpoint{3.037477in}{1.359782in}}%
\pgfpathlineto{\pgfqpoint{2.987749in}{1.340408in}}%
\pgfpathlineto{\pgfqpoint{2.941041in}{1.319783in}}%
\pgfpathlineto{\pgfqpoint{2.897397in}{1.297982in}}%
\pgfpathlineto{\pgfqpoint{2.854289in}{1.273824in}}%
\pgfpathlineto{\pgfqpoint{2.816870in}{1.250084in}}%
\pgfpathlineto{\pgfqpoint{2.782555in}{1.225807in}}%
\pgfpathlineto{\pgfqpoint{2.748946in}{1.199324in}}%
\pgfpathlineto{\pgfqpoint{2.718482in}{1.172642in}}%
\pgfpathlineto{\pgfqpoint{2.703698in}{1.158197in}}%
\pgfpathlineto{\pgfqpoint{2.703698in}{1.158197in}}%
\pgfusepath{stroke}%
\end{pgfscope}%
\begin{pgfscope}%
\pgfpathrectangle{\pgfqpoint{2.713698in}{0.248000in}}{\pgfqpoint{2.295902in}{1.362506in}} %
\pgfusepath{clip}%
\pgfsetrectcap%
\pgfsetroundjoin%
\pgfsetlinewidth{1.003750pt}%
\definecolor{currentstroke}{rgb}{0.172549,0.627451,0.172549}%
\pgfsetstrokecolor{currentstroke}%
\pgfsetdash{}{0pt}%
\pgfpathmoveto{\pgfqpoint{4.831042in}{0.238000in}}%
\pgfpathlineto{\pgfqpoint{4.847940in}{0.377703in}}%
\pgfpathlineto{\pgfqpoint{4.866739in}{0.515107in}}%
\pgfpathlineto{\pgfqpoint{4.888820in}{0.659858in}}%
\pgfpathlineto{\pgfqpoint{4.936429in}{0.964993in}}%
\pgfpathlineto{\pgfqpoint{4.944142in}{1.029984in}}%
\pgfpathlineto{\pgfqpoint{4.947897in}{1.079186in}}%
\pgfpathlineto{\pgfqpoint{4.948840in}{1.120435in}}%
\pgfpathlineto{\pgfqpoint{4.947551in}{1.153393in}}%
\pgfpathlineto{\pgfqpoint{4.944488in}{1.181899in}}%
\pgfpathlineto{\pgfqpoint{4.939237in}{1.209643in}}%
\pgfpathlineto{\pgfqpoint{4.932828in}{1.232380in}}%
\pgfpathlineto{\pgfqpoint{4.924653in}{1.253752in}}%
\pgfpathlineto{\pgfqpoint{4.914854in}{1.273466in}}%
\pgfpathlineto{\pgfqpoint{4.903670in}{1.291378in}}%
\pgfpathlineto{\pgfqpoint{4.889237in}{1.310027in}}%
\pgfpathlineto{\pgfqpoint{4.873683in}{1.326441in}}%
\pgfpathlineto{\pgfqpoint{4.854930in}{1.342835in}}%
\pgfpathlineto{\pgfqpoint{4.832970in}{1.358753in}}%
\pgfpathlineto{\pgfqpoint{4.807849in}{1.373905in}}%
\pgfpathlineto{\pgfqpoint{4.779635in}{1.388125in}}%
\pgfpathlineto{\pgfqpoint{4.748402in}{1.401359in}}%
\pgfpathlineto{\pgfqpoint{4.711569in}{1.414430in}}%
\pgfpathlineto{\pgfqpoint{4.669138in}{1.426948in}}%
\pgfpathlineto{\pgfqpoint{4.621126in}{1.438638in}}%
\pgfpathlineto{\pgfqpoint{4.567560in}{1.449325in}}%
\pgfpathlineto{\pgfqpoint{4.505779in}{1.459273in}}%
\pgfpathlineto{\pgfqpoint{4.438487in}{1.467844in}}%
\pgfpathlineto{\pgfqpoint{4.363013in}{1.475229in}}%
\pgfpathlineto{\pgfqpoint{4.279372in}{1.481188in}}%
\pgfpathlineto{\pgfqpoint{4.195678in}{1.485032in}}%
\pgfpathlineto{\pgfqpoint{4.098455in}{1.487560in}}%
\pgfpathlineto{\pgfqpoint{3.995816in}{1.488009in}}%
\pgfpathlineto{\pgfqpoint{3.890482in}{1.486208in}}%
\pgfpathlineto{\pgfqpoint{3.779774in}{1.482025in}}%
\pgfpathlineto{\pgfqpoint{3.679918in}{1.475827in}}%
\pgfpathlineto{\pgfqpoint{3.569343in}{1.466578in}}%
\pgfpathlineto{\pgfqpoint{3.483149in}{1.456762in}}%
\pgfpathlineto{\pgfqpoint{3.402458in}{1.445437in}}%
\pgfpathlineto{\pgfqpoint{3.327288in}{1.432723in}}%
\pgfpathlineto{\pgfqpoint{3.230865in}{1.413778in}}%
\pgfpathlineto{\pgfqpoint{3.148200in}{1.393564in}}%
\pgfpathlineto{\pgfqpoint{3.092538in}{1.376892in}}%
\pgfpathlineto{\pgfqpoint{3.039823in}{1.358854in}}%
\pgfpathlineto{\pgfqpoint{2.990095in}{1.339477in}}%
\pgfpathlineto{\pgfqpoint{2.943388in}{1.318850in}}%
\pgfpathlineto{\pgfqpoint{2.902296in}{1.298399in}}%
\pgfpathlineto{\pgfqpoint{2.859145in}{1.274418in}}%
\pgfpathlineto{\pgfqpoint{2.821687in}{1.250828in}}%
\pgfpathlineto{\pgfqpoint{2.787325in}{1.226704in}}%
\pgfpathlineto{\pgfqpoint{2.753661in}{1.200385in}}%
\pgfpathlineto{\pgfqpoint{2.720821in}{1.171738in}}%
\pgfpathlineto{\pgfqpoint{2.703698in}{1.154609in}}%
\pgfpathlineto{\pgfqpoint{2.703698in}{1.154609in}}%
\pgfusepath{stroke}%
\end{pgfscope}%
\begin{pgfscope}%
\pgfpathrectangle{\pgfqpoint{2.713698in}{0.248000in}}{\pgfqpoint{2.295902in}{1.362506in}} %
\pgfusepath{clip}%
\pgfsetbuttcap%
\pgfsetroundjoin%
\pgfsetlinewidth{1.003750pt}%
\definecolor{currentstroke}{rgb}{0.000000,0.000000,0.000000}%
\pgfsetstrokecolor{currentstroke}%
\pgfsetdash{{3.700000pt}{1.600000pt}}{0.000000pt}%
\pgfpathmoveto{\pgfqpoint{2.746089in}{1.162823in}}%
\pgfpathlineto{\pgfqpoint{2.778981in}{1.191329in}}%
\pgfpathlineto{\pgfqpoint{2.812687in}{1.217524in}}%
\pgfpathlineto{\pgfqpoint{2.847079in}{1.241545in}}%
\pgfpathlineto{\pgfqpoint{2.884563in}{1.265041in}}%
\pgfpathlineto{\pgfqpoint{2.925142in}{1.287757in}}%
\pgfpathlineto{\pgfqpoint{2.968802in}{1.309487in}}%
\pgfpathlineto{\pgfqpoint{3.012909in}{1.328991in}}%
\pgfpathlineto{\pgfqpoint{3.059998in}{1.347480in}}%
\pgfpathlineto{\pgfqpoint{3.112684in}{1.365714in}}%
\pgfpathlineto{\pgfqpoint{3.168324in}{1.382563in}}%
\pgfpathlineto{\pgfqpoint{3.226880in}{1.398013in}}%
\pgfpathlineto{\pgfqpoint{3.290994in}{1.412635in}}%
\pgfpathlineto{\pgfqpoint{3.360658in}{1.426200in}}%
\pgfpathlineto{\pgfqpoint{3.435855in}{1.438523in}}%
\pgfpathlineto{\pgfqpoint{3.516570in}{1.449451in}}%
\pgfpathlineto{\pgfqpoint{3.602784in}{1.458852in}}%
\pgfpathlineto{\pgfqpoint{3.694479in}{1.466596in}}%
\pgfpathlineto{\pgfqpoint{3.791639in}{1.472536in}}%
\pgfpathlineto{\pgfqpoint{3.891545in}{1.476403in}}%
\pgfpathlineto{\pgfqpoint{3.994179in}{1.478115in}}%
\pgfpathlineto{\pgfqpoint{4.094116in}{1.477568in}}%
\pgfpathlineto{\pgfqpoint{4.191337in}{1.474802in}}%
\pgfpathlineto{\pgfqpoint{4.283117in}{1.469939in}}%
\pgfpathlineto{\pgfqpoint{4.366735in}{1.463271in}}%
\pgfpathlineto{\pgfqpoint{4.442171in}{1.455030in}}%
\pgfpathlineto{\pgfqpoint{4.509405in}{1.445459in}}%
\pgfpathlineto{\pgfqpoint{4.568420in}{1.434852in}}%
\pgfpathlineto{\pgfqpoint{4.619207in}{1.423593in}}%
\pgfpathlineto{\pgfqpoint{4.664426in}{1.411410in}}%
\pgfpathlineto{\pgfqpoint{4.704055in}{1.398536in}}%
\pgfpathlineto{\pgfqpoint{4.738084in}{1.385317in}}%
\pgfpathlineto{\pgfqpoint{4.769099in}{1.370941in}}%
\pgfpathlineto{\pgfqpoint{4.797019in}{1.355427in}}%
\pgfpathlineto{\pgfqpoint{4.819316in}{1.340638in}}%
\pgfpathlineto{\pgfqpoint{4.838530in}{1.325538in}}%
\pgfpathlineto{\pgfqpoint{4.856923in}{1.308223in}}%
\pgfpathlineto{\pgfqpoint{4.872074in}{1.290949in}}%
\pgfpathlineto{\pgfqpoint{4.886018in}{1.271452in}}%
\pgfpathlineto{\pgfqpoint{4.896729in}{1.252875in}}%
\pgfpathlineto{\pgfqpoint{4.906042in}{1.232618in}}%
\pgfpathlineto{\pgfqpoint{4.913773in}{1.210861in}}%
\pgfpathlineto{\pgfqpoint{4.920687in}{1.183982in}}%
\pgfpathlineto{\pgfqpoint{4.925408in}{1.156013in}}%
\pgfpathlineto{\pgfqpoint{4.928439in}{1.123325in}}%
\pgfpathlineto{\pgfqpoint{4.929407in}{1.086209in}}%
\pgfpathlineto{\pgfqpoint{4.928218in}{1.044970in}}%
\pgfpathlineto{\pgfqpoint{4.924607in}{0.995740in}}%
\pgfpathlineto{\pgfqpoint{4.917469in}{0.930593in}}%
\pgfpathlineto{\pgfqpoint{4.904797in}{0.837631in}}%
\pgfpathlineto{\pgfqpoint{4.845589in}{0.422130in}}%
\pgfpathlineto{\pgfqpoint{4.827683in}{0.271839in}}%
\pgfpathlineto{\pgfqpoint{4.823961in}{0.238000in}}%
\pgfpathlineto{\pgfqpoint{4.823961in}{0.238000in}}%
\pgfusepath{stroke}%
\end{pgfscope}%
\begin{pgfscope}%
\pgfpathrectangle{\pgfqpoint{2.713698in}{0.248000in}}{\pgfqpoint{2.295902in}{1.362506in}} %
\pgfusepath{clip}%
\pgfsetbuttcap%
\pgfsetroundjoin%
\pgfsetlinewidth{1.003750pt}%
\definecolor{currentstroke}{rgb}{0.000000,0.000000,0.000000}%
\pgfsetstrokecolor{currentstroke}%
\pgfsetdash{{3.700000pt}{1.600000pt}}{0.000000pt}%
\pgfpathmoveto{\pgfqpoint{0.000000in}{0.000000in}}%
\pgfusepath{stroke}%
\end{pgfscope}%
\begin{pgfscope}%
\pgfpathrectangle{\pgfqpoint{2.713698in}{0.248000in}}{\pgfqpoint{2.295902in}{1.362506in}} %
\pgfusepath{clip}%
\pgfsetbuttcap%
\pgfsetroundjoin%
\pgfsetlinewidth{1.003750pt}%
\definecolor{currentstroke}{rgb}{0.000000,0.000000,0.000000}%
\pgfsetstrokecolor{currentstroke}%
\pgfsetdash{{3.700000pt}{1.600000pt}}{0.000000pt}%
\pgfpathmoveto{\pgfqpoint{0.000000in}{0.000000in}}%
\pgfusepath{stroke}%
\end{pgfscope}%
\begin{pgfscope}%
\pgfpathrectangle{\pgfqpoint{2.713698in}{0.248000in}}{\pgfqpoint{2.295902in}{1.362506in}} %
\pgfusepath{clip}%
\pgfsetbuttcap%
\pgfsetroundjoin%
\pgfsetlinewidth{1.003750pt}%
\definecolor{currentstroke}{rgb}{0.000000,0.000000,0.000000}%
\pgfsetstrokecolor{currentstroke}%
\pgfsetdash{{3.700000pt}{1.600000pt}}{0.000000pt}%
\pgfpathmoveto{\pgfqpoint{4.687175in}{0.238000in}}%
\pgfpathlineto{\pgfqpoint{4.690620in}{0.399578in}}%
\pgfpathlineto{\pgfqpoint{4.691068in}{0.519306in}}%
\pgfpathlineto{\pgfqpoint{4.689236in}{0.614222in}}%
\pgfpathlineto{\pgfqpoint{4.685737in}{0.688341in}}%
\pgfpathlineto{\pgfqpoint{4.680465in}{0.753900in}}%
\pgfpathlineto{\pgfqpoint{4.673619in}{0.810738in}}%
\pgfpathlineto{\pgfqpoint{4.665637in}{0.858748in}}%
\pgfpathlineto{\pgfqpoint{4.656245in}{0.901824in}}%
\pgfpathlineto{\pgfqpoint{4.645727in}{0.939841in}}%
\pgfpathlineto{\pgfqpoint{4.634498in}{0.972787in}}%
\pgfpathlineto{\pgfqpoint{4.621531in}{1.004206in}}%
\pgfpathlineto{\pgfqpoint{4.606841in}{1.033798in}}%
\pgfpathlineto{\pgfqpoint{4.590526in}{1.061331in}}%
\pgfpathlineto{\pgfqpoint{4.572752in}{1.086665in}}%
\pgfpathlineto{\pgfqpoint{4.553721in}{1.109771in}}%
\pgfpathlineto{\pgfqpoint{4.533644in}{1.130709in}}%
\pgfpathlineto{\pgfqpoint{4.510348in}{1.151590in}}%
\pgfpathlineto{\pgfqpoint{4.486235in}{1.170181in}}%
\pgfpathlineto{\pgfqpoint{4.458987in}{1.188277in}}%
\pgfpathlineto{\pgfqpoint{4.428614in}{1.205557in}}%
\pgfpathlineto{\pgfqpoint{4.397738in}{1.220622in}}%
\pgfpathlineto{\pgfqpoint{4.363874in}{1.234797in}}%
\pgfpathlineto{\pgfqpoint{4.324408in}{1.248793in}}%
\pgfpathlineto{\pgfqpoint{4.281983in}{1.261361in}}%
\pgfpathlineto{\pgfqpoint{4.236644in}{1.272456in}}%
\pgfpathlineto{\pgfqpoint{4.185750in}{1.282519in}}%
\pgfpathlineto{\pgfqpoint{4.132004in}{1.290829in}}%
\pgfpathlineto{\pgfqpoint{4.075445in}{1.297374in}}%
\pgfpathlineto{\pgfqpoint{4.013405in}{1.302285in}}%
\pgfpathlineto{\pgfqpoint{3.948608in}{1.305145in}}%
\pgfpathlineto{\pgfqpoint{3.881085in}{1.305861in}}%
\pgfpathlineto{\pgfqpoint{3.813566in}{1.304386in}}%
\pgfpathlineto{\pgfqpoint{3.743384in}{1.300586in}}%
\pgfpathlineto{\pgfqpoint{3.675967in}{1.294733in}}%
\pgfpathlineto{\pgfqpoint{3.608649in}{1.286652in}}%
\pgfpathlineto{\pgfqpoint{3.544155in}{1.276676in}}%
\pgfpathlineto{\pgfqpoint{3.482510in}{1.264918in}}%
\pgfpathlineto{\pgfqpoint{3.423742in}{1.251476in}}%
\pgfpathlineto{\pgfqpoint{3.367881in}{1.236441in}}%
\pgfpathlineto{\pgfqpoint{3.314955in}{1.219908in}}%
\pgfpathlineto{\pgfqpoint{3.264994in}{1.201985in}}%
\pgfpathlineto{\pgfqpoint{3.218026in}{1.182796in}}%
\pgfpathlineto{\pgfqpoint{3.174074in}{1.162491in}}%
\pgfpathlineto{\pgfqpoint{3.133153in}{1.141253in}}%
\pgfpathlineto{\pgfqpoint{3.092766in}{1.117754in}}%
\pgfpathlineto{\pgfqpoint{3.055491in}{1.093493in}}%
\pgfpathlineto{\pgfqpoint{3.021319in}{1.068748in}}%
\pgfpathlineto{\pgfqpoint{2.987855in}{1.041837in}}%
\pgfpathlineto{\pgfqpoint{2.955225in}{1.012637in}}%
\pgfpathlineto{\pgfqpoint{2.925790in}{0.983379in}}%
\pgfpathlineto{\pgfqpoint{2.897311in}{0.951990in}}%
\pgfpathlineto{\pgfqpoint{2.869910in}{0.918434in}}%
\pgfpathlineto{\pgfqpoint{2.843711in}{0.882706in}}%
\pgfpathlineto{\pgfqpoint{2.818832in}{0.844839in}}%
\pgfpathlineto{\pgfqpoint{2.795379in}{0.804903in}}%
\pgfpathlineto{\pgfqpoint{2.773441in}{0.763001in}}%
\pgfpathlineto{\pgfqpoint{2.753090in}{0.719270in}}%
\pgfpathlineto{\pgfqpoint{2.734370in}{0.673867in}}%
\pgfpathlineto{\pgfqpoint{2.717306in}{0.626964in}}%
\pgfpathlineto{\pgfqpoint{2.703698in}{0.584702in}}%
\pgfpathmoveto{\pgfqpoint{2.703698in}{0.908857in}}%
\pgfpathlineto{\pgfqpoint{2.727582in}{0.946850in}}%
\pgfpathlineto{\pgfqpoint{2.753246in}{0.983474in}}%
\pgfpathlineto{\pgfqpoint{2.780155in}{1.017945in}}%
\pgfpathlineto{\pgfqpoint{2.808195in}{1.050244in}}%
\pgfpathlineto{\pgfqpoint{2.837245in}{1.080387in}}%
\pgfpathlineto{\pgfqpoint{2.867184in}{1.108424in}}%
\pgfpathlineto{\pgfqpoint{2.900289in}{1.136350in}}%
\pgfpathlineto{\pgfqpoint{2.934160in}{1.162042in}}%
\pgfpathlineto{\pgfqpoint{2.971166in}{1.187246in}}%
\pgfpathlineto{\pgfqpoint{3.008789in}{1.210221in}}%
\pgfpathlineto{\pgfqpoint{3.049478in}{1.232472in}}%
\pgfpathlineto{\pgfqpoint{3.093224in}{1.253793in}}%
\pgfpathlineto{\pgfqpoint{3.140006in}{1.274018in}}%
\pgfpathlineto{\pgfqpoint{3.189796in}{1.293020in}}%
\pgfpathlineto{\pgfqpoint{3.242562in}{1.310706in}}%
\pgfpathlineto{\pgfqpoint{3.298271in}{1.327010in}}%
\pgfpathlineto{\pgfqpoint{3.356890in}{1.341889in}}%
\pgfpathlineto{\pgfqpoint{3.421068in}{1.355847in}}%
\pgfpathlineto{\pgfqpoint{3.488111in}{1.368156in}}%
\pgfpathlineto{\pgfqpoint{3.560681in}{1.379177in}}%
\pgfpathlineto{\pgfqpoint{3.636070in}{1.388372in}}%
\pgfpathlineto{\pgfqpoint{3.714253in}{1.395716in}}%
\pgfpathlineto{\pgfqpoint{3.795206in}{1.401146in}}%
\pgfpathlineto{\pgfqpoint{3.878908in}{1.404548in}}%
\pgfpathlineto{\pgfqpoint{3.962636in}{1.405740in}}%
\pgfpathlineto{\pgfqpoint{4.043664in}{1.404737in}}%
\pgfpathlineto{\pgfqpoint{4.121967in}{1.401616in}}%
\pgfpathlineto{\pgfqpoint{4.197518in}{1.396380in}}%
\pgfpathlineto{\pgfqpoint{4.267590in}{1.389274in}}%
\pgfpathlineto{\pgfqpoint{4.332156in}{1.380446in}}%
\pgfpathlineto{\pgfqpoint{4.388508in}{1.370570in}}%
\pgfpathlineto{\pgfqpoint{4.439315in}{1.359525in}}%
\pgfpathlineto{\pgfqpoint{4.484559in}{1.347560in}}%
\pgfpathlineto{\pgfqpoint{4.526865in}{1.334083in}}%
\pgfpathlineto{\pgfqpoint{4.563556in}{1.320114in}}%
\pgfpathlineto{\pgfqpoint{4.597226in}{1.304896in}}%
\pgfpathlineto{\pgfqpoint{4.625280in}{1.289955in}}%
\pgfpathlineto{\pgfqpoint{4.650294in}{1.274388in}}%
\pgfpathlineto{\pgfqpoint{4.674655in}{1.256572in}}%
\pgfpathlineto{\pgfqpoint{4.695810in}{1.238282in}}%
\pgfpathlineto{\pgfqpoint{4.713788in}{1.219970in}}%
\pgfpathlineto{\pgfqpoint{4.730748in}{1.199519in}}%
\pgfpathlineto{\pgfqpoint{4.746419in}{1.176796in}}%
\pgfpathlineto{\pgfqpoint{4.758850in}{1.155032in}}%
\pgfpathlineto{\pgfqpoint{4.769909in}{1.131603in}}%
\pgfpathlineto{\pgfqpoint{4.780724in}{1.103025in}}%
\pgfpathlineto{\pgfqpoint{4.789551in}{1.072892in}}%
\pgfpathlineto{\pgfqpoint{4.796475in}{1.041614in}}%
\pgfpathlineto{\pgfqpoint{4.802219in}{1.005519in}}%
\pgfpathlineto{\pgfqpoint{4.806451in}{0.964751in}}%
\pgfpathlineto{\pgfqpoint{4.809158in}{0.915389in}}%
\pgfpathlineto{\pgfqpoint{4.809893in}{0.857604in}}%
\pgfpathlineto{\pgfqpoint{4.808380in}{0.787459in}}%
\pgfpathlineto{\pgfqpoint{4.803993in}{0.696878in}}%
\pgfpathlineto{\pgfqpoint{4.795104in}{0.565459in}}%
\pgfpathlineto{\pgfqpoint{4.770069in}{0.238000in}}%
\pgfpathlineto{\pgfqpoint{4.770069in}{0.238000in}}%
\pgfusepath{stroke}%
\end{pgfscope}%
\begin{pgfscope}%
\pgfpathrectangle{\pgfqpoint{2.713698in}{0.248000in}}{\pgfqpoint{2.295902in}{1.362506in}} %
\pgfusepath{clip}%
\pgfsetbuttcap%
\pgfsetroundjoin%
\pgfsetlinewidth{1.003750pt}%
\definecolor{currentstroke}{rgb}{0.000000,0.000000,0.000000}%
\pgfsetstrokecolor{currentstroke}%
\pgfsetdash{{3.700000pt}{1.600000pt}}{0.000000pt}%
\pgfpathmoveto{\pgfqpoint{4.831803in}{0.238000in}}%
\pgfpathlineto{\pgfqpoint{4.848789in}{0.377738in}}%
\pgfpathlineto{\pgfqpoint{4.867695in}{0.515107in}}%
\pgfpathlineto{\pgfqpoint{4.889284in}{0.655794in}}%
\pgfpathlineto{\pgfqpoint{4.937208in}{0.960756in}}%
\pgfpathlineto{\pgfqpoint{4.945664in}{1.029736in}}%
\pgfpathlineto{\pgfqpoint{4.949593in}{1.078906in}}%
\pgfpathlineto{\pgfqpoint{4.950727in}{1.120144in}}%
\pgfpathlineto{\pgfqpoint{4.949619in}{1.153118in}}%
\pgfpathlineto{\pgfqpoint{4.946720in}{1.181664in}}%
\pgfpathlineto{\pgfqpoint{4.941643in}{1.209484in}}%
\pgfpathlineto{\pgfqpoint{4.935378in}{1.232314in}}%
\pgfpathlineto{\pgfqpoint{4.927330in}{1.253798in}}%
\pgfpathlineto{\pgfqpoint{4.917634in}{1.273630in}}%
\pgfpathlineto{\pgfqpoint{4.906530in}{1.291658in}}%
\pgfpathlineto{\pgfqpoint{4.892162in}{1.310423in}}%
\pgfpathlineto{\pgfqpoint{4.876648in}{1.326927in}}%
\pgfpathlineto{\pgfqpoint{4.857923in}{1.343394in}}%
\pgfpathlineto{\pgfqpoint{4.835984in}{1.359380in}}%
\pgfpathlineto{\pgfqpoint{4.810878in}{1.374589in}}%
\pgfpathlineto{\pgfqpoint{4.782671in}{1.388843in}}%
\pgfpathlineto{\pgfqpoint{4.751444in}{1.402106in}}%
\pgfpathlineto{\pgfqpoint{4.714615in}{1.415201in}}%
\pgfpathlineto{\pgfqpoint{4.672185in}{1.427737in}}%
\pgfpathlineto{\pgfqpoint{4.624175in}{1.439442in}}%
\pgfpathlineto{\pgfqpoint{4.570610in}{1.450142in}}%
\pgfpathlineto{\pgfqpoint{4.508830in}{1.460101in}}%
\pgfpathlineto{\pgfqpoint{4.441539in}{1.468685in}}%
\pgfpathlineto{\pgfqpoint{4.366065in}{1.476085in}}%
\pgfpathlineto{\pgfqpoint{4.282424in}{1.482064in}}%
\pgfpathlineto{\pgfqpoint{4.198731in}{1.485907in}}%
\pgfpathlineto{\pgfqpoint{4.101508in}{1.488472in}}%
\pgfpathlineto{\pgfqpoint{3.998869in}{1.488967in}}%
\pgfpathlineto{\pgfqpoint{3.893535in}{1.487227in}}%
\pgfpathlineto{\pgfqpoint{3.780126in}{1.483011in}}%
\pgfpathlineto{\pgfqpoint{3.680269in}{1.476842in}}%
\pgfpathlineto{\pgfqpoint{3.566996in}{1.467431in}}%
\pgfpathlineto{\pgfqpoint{3.480803in}{1.457600in}}%
\pgfpathlineto{\pgfqpoint{3.400112in}{1.446262in}}%
\pgfpathlineto{\pgfqpoint{3.324943in}{1.433539in}}%
\pgfpathlineto{\pgfqpoint{3.225845in}{1.414132in}}%
\pgfpathlineto{\pgfqpoint{3.145855in}{1.394486in}}%
\pgfpathlineto{\pgfqpoint{3.090192in}{1.377818in}}%
\pgfpathlineto{\pgfqpoint{3.037477in}{1.359782in}}%
\pgfpathlineto{\pgfqpoint{2.987749in}{1.340408in}}%
\pgfpathlineto{\pgfqpoint{2.941041in}{1.319783in}}%
\pgfpathlineto{\pgfqpoint{2.897397in}{1.297982in}}%
\pgfpathlineto{\pgfqpoint{2.854289in}{1.273824in}}%
\pgfpathlineto{\pgfqpoint{2.816870in}{1.250084in}}%
\pgfpathlineto{\pgfqpoint{2.782555in}{1.225807in}}%
\pgfpathlineto{\pgfqpoint{2.748946in}{1.199324in}}%
\pgfpathlineto{\pgfqpoint{2.718482in}{1.172642in}}%
\pgfpathlineto{\pgfqpoint{2.703698in}{1.158197in}}%
\pgfpathlineto{\pgfqpoint{2.703698in}{1.158197in}}%
\pgfusepath{stroke}%
\end{pgfscope}%
\begin{pgfscope}%
\pgfpathrectangle{\pgfqpoint{2.713698in}{0.248000in}}{\pgfqpoint{2.295902in}{1.362506in}} %
\pgfusepath{clip}%
\pgfsetbuttcap%
\pgfsetroundjoin%
\pgfsetlinewidth{1.003750pt}%
\definecolor{currentstroke}{rgb}{0.000000,0.000000,0.000000}%
\pgfsetstrokecolor{currentstroke}%
\pgfsetdash{{3.700000pt}{1.600000pt}}{0.000000pt}%
\pgfpathmoveto{\pgfqpoint{0.000000in}{0.000000in}}%
\pgfusepath{stroke}%
\end{pgfscope}%
\begin{pgfscope}%
\pgfpathrectangle{\pgfqpoint{2.713698in}{0.248000in}}{\pgfqpoint{2.295902in}{1.362506in}} %
\pgfusepath{clip}%
\pgfsetbuttcap%
\pgfsetroundjoin%
\pgfsetlinewidth{1.003750pt}%
\definecolor{currentstroke}{rgb}{0.000000,0.000000,0.000000}%
\pgfsetstrokecolor{currentstroke}%
\pgfsetdash{{3.700000pt}{1.600000pt}}{0.000000pt}%
\pgfpathmoveto{\pgfqpoint{4.831042in}{0.238000in}}%
\pgfpathlineto{\pgfqpoint{4.847940in}{0.377703in}}%
\pgfpathlineto{\pgfqpoint{4.866739in}{0.515107in}}%
\pgfpathlineto{\pgfqpoint{4.888820in}{0.659857in}}%
\pgfpathlineto{\pgfqpoint{4.936429in}{0.964993in}}%
\pgfpathlineto{\pgfqpoint{4.944142in}{1.029984in}}%
\pgfpathlineto{\pgfqpoint{4.947897in}{1.079185in}}%
\pgfpathlineto{\pgfqpoint{4.948840in}{1.120435in}}%
\pgfpathlineto{\pgfqpoint{4.947551in}{1.153393in}}%
\pgfpathlineto{\pgfqpoint{4.944488in}{1.181899in}}%
\pgfpathlineto{\pgfqpoint{4.939237in}{1.209643in}}%
\pgfpathlineto{\pgfqpoint{4.932828in}{1.232380in}}%
\pgfpathlineto{\pgfqpoint{4.924653in}{1.253752in}}%
\pgfpathlineto{\pgfqpoint{4.914854in}{1.273466in}}%
\pgfpathlineto{\pgfqpoint{4.903670in}{1.291378in}}%
\pgfpathlineto{\pgfqpoint{4.889237in}{1.310027in}}%
\pgfpathlineto{\pgfqpoint{4.873683in}{1.326441in}}%
\pgfpathlineto{\pgfqpoint{4.854930in}{1.342835in}}%
\pgfpathlineto{\pgfqpoint{4.832970in}{1.358753in}}%
\pgfpathlineto{\pgfqpoint{4.807849in}{1.373905in}}%
\pgfpathlineto{\pgfqpoint{4.779635in}{1.388125in}}%
\pgfpathlineto{\pgfqpoint{4.748402in}{1.401359in}}%
\pgfpathlineto{\pgfqpoint{4.711569in}{1.414430in}}%
\pgfpathlineto{\pgfqpoint{4.669138in}{1.426948in}}%
\pgfpathlineto{\pgfqpoint{4.621126in}{1.438638in}}%
\pgfpathlineto{\pgfqpoint{4.567560in}{1.449325in}}%
\pgfpathlineto{\pgfqpoint{4.505779in}{1.459273in}}%
\pgfpathlineto{\pgfqpoint{4.438487in}{1.467844in}}%
\pgfpathlineto{\pgfqpoint{4.363013in}{1.475229in}}%
\pgfpathlineto{\pgfqpoint{4.279372in}{1.481188in}}%
\pgfpathlineto{\pgfqpoint{4.195678in}{1.485032in}}%
\pgfpathlineto{\pgfqpoint{4.098455in}{1.487560in}}%
\pgfpathlineto{\pgfqpoint{3.995816in}{1.488009in}}%
\pgfpathlineto{\pgfqpoint{3.890482in}{1.486208in}}%
\pgfpathlineto{\pgfqpoint{3.779774in}{1.482025in}}%
\pgfpathlineto{\pgfqpoint{3.679918in}{1.475827in}}%
\pgfpathlineto{\pgfqpoint{3.569343in}{1.466578in}}%
\pgfpathlineto{\pgfqpoint{3.483149in}{1.456762in}}%
\pgfpathlineto{\pgfqpoint{3.402458in}{1.445436in}}%
\pgfpathlineto{\pgfqpoint{3.327288in}{1.432723in}}%
\pgfpathlineto{\pgfqpoint{3.230865in}{1.413778in}}%
\pgfpathlineto{\pgfqpoint{3.148200in}{1.393564in}}%
\pgfpathlineto{\pgfqpoint{3.092538in}{1.376892in}}%
\pgfpathlineto{\pgfqpoint{3.039823in}{1.358854in}}%
\pgfpathlineto{\pgfqpoint{2.990095in}{1.339477in}}%
\pgfpathlineto{\pgfqpoint{2.943388in}{1.318850in}}%
\pgfpathlineto{\pgfqpoint{2.902296in}{1.298399in}}%
\pgfpathlineto{\pgfqpoint{2.859145in}{1.274418in}}%
\pgfpathlineto{\pgfqpoint{2.821687in}{1.250828in}}%
\pgfpathlineto{\pgfqpoint{2.787325in}{1.226704in}}%
\pgfpathlineto{\pgfqpoint{2.753661in}{1.200385in}}%
\pgfpathlineto{\pgfqpoint{2.720821in}{1.171738in}}%
\pgfpathlineto{\pgfqpoint{2.703698in}{1.154609in}}%
\pgfpathlineto{\pgfqpoint{2.703698in}{1.154609in}}%
\pgfusepath{stroke}%
\end{pgfscope}%
\begin{pgfscope}%
\pgfpathrectangle{\pgfqpoint{2.713698in}{0.248000in}}{\pgfqpoint{2.295902in}{1.362506in}} %
\pgfusepath{clip}%
\pgfsetbuttcap%
\pgfsetroundjoin%
\pgfsetlinewidth{1.003750pt}%
\definecolor{currentstroke}{rgb}{0.000000,0.000000,0.000000}%
\pgfsetstrokecolor{currentstroke}%
\pgfsetdash{{3.700000pt}{1.600000pt}}{0.000000pt}%
\pgfpathmoveto{\pgfqpoint{0.000000in}{0.000000in}}%
\pgfusepath{stroke}%
\end{pgfscope}%
\begin{pgfscope}%
\pgfsetrectcap%
\pgfsetmiterjoin%
\pgfsetlinewidth{0.803000pt}%
\definecolor{currentstroke}{rgb}{0.000000,0.000000,0.000000}%
\pgfsetstrokecolor{currentstroke}%
\pgfsetdash{}{0pt}%
\pgfpathmoveto{\pgfqpoint{2.713698in}{0.248000in}}%
\pgfpathlineto{\pgfqpoint{2.713698in}{1.610506in}}%
\pgfusepath{stroke}%
\end{pgfscope}%
\begin{pgfscope}%
\pgfsetrectcap%
\pgfsetmiterjoin%
\pgfsetlinewidth{0.803000pt}%
\definecolor{currentstroke}{rgb}{0.000000,0.000000,0.000000}%
\pgfsetstrokecolor{currentstroke}%
\pgfsetdash{}{0pt}%
\pgfpathmoveto{\pgfqpoint{5.009600in}{0.248000in}}%
\pgfpathlineto{\pgfqpoint{5.009600in}{1.610506in}}%
\pgfusepath{stroke}%
\end{pgfscope}%
\begin{pgfscope}%
\pgfsetrectcap%
\pgfsetmiterjoin%
\pgfsetlinewidth{0.803000pt}%
\definecolor{currentstroke}{rgb}{0.000000,0.000000,0.000000}%
\pgfsetstrokecolor{currentstroke}%
\pgfsetdash{}{0pt}%
\pgfpathmoveto{\pgfqpoint{2.713698in}{0.248000in}}%
\pgfpathlineto{\pgfqpoint{5.009600in}{0.248000in}}%
\pgfusepath{stroke}%
\end{pgfscope}%
\begin{pgfscope}%
\pgfsetrectcap%
\pgfsetmiterjoin%
\pgfsetlinewidth{0.803000pt}%
\definecolor{currentstroke}{rgb}{0.000000,0.000000,0.000000}%
\pgfsetstrokecolor{currentstroke}%
\pgfsetdash{}{0pt}%
\pgfpathmoveto{\pgfqpoint{2.713698in}{1.610506in}}%
\pgfpathlineto{\pgfqpoint{5.009600in}{1.610506in}}%
\pgfusepath{stroke}%
\end{pgfscope}%
\begin{pgfscope}%
\pgfsetbuttcap%
\pgfsetmiterjoin%
\definecolor{currentfill}{rgb}{1.000000,1.000000,1.000000}%
\pgfsetfillcolor{currentfill}%
\pgfsetfillopacity{0.800000}%
\pgfsetlinewidth{1.003750pt}%
\definecolor{currentstroke}{rgb}{0.800000,0.800000,0.800000}%
\pgfsetstrokecolor{currentstroke}%
\pgfsetstrokeopacity{0.800000}%
\pgfsetdash{}{0pt}%
\pgfpathmoveto{\pgfqpoint{3.461700in}{0.317444in}}%
\pgfpathlineto{\pgfqpoint{4.261598in}{0.317444in}}%
\pgfpathquadraticcurveto{\pgfqpoint{4.289376in}{0.317444in}}{\pgfqpoint{4.289376in}{0.345222in}}%
\pgfpathlineto{\pgfqpoint{4.289376in}{0.730222in}}%
\pgfpathquadraticcurveto{\pgfqpoint{4.289376in}{0.758000in}}{\pgfqpoint{4.261598in}{0.758000in}}%
\pgfpathlineto{\pgfqpoint{3.461700in}{0.758000in}}%
\pgfpathquadraticcurveto{\pgfqpoint{3.433922in}{0.758000in}}{\pgfqpoint{3.433922in}{0.730222in}}%
\pgfpathlineto{\pgfqpoint{3.433922in}{0.345222in}}%
\pgfpathquadraticcurveto{\pgfqpoint{3.433922in}{0.317444in}}{\pgfqpoint{3.461700in}{0.317444in}}%
\pgfpathclose%
\pgfusepath{stroke,fill}%
\end{pgfscope}%
\begin{pgfscope}%
\pgfsetrectcap%
\pgfsetroundjoin%
\pgfsetlinewidth{1.003750pt}%
\definecolor{currentstroke}{rgb}{0.172549,0.627451,0.172549}%
\pgfsetstrokecolor{currentstroke}%
\pgfsetdash{}{0pt}%
\pgfpathmoveto{\pgfqpoint{3.489477in}{0.653833in}}%
\pgfpathlineto{\pgfqpoint{3.558922in}{0.653833in}}%
\pgfusepath{stroke}%
\end{pgfscope}%
\begin{pgfscope}%
\pgftext[x=3.670033in,y=0.605222in,left,base]{\rmfamily\fontsize{10.000000}{12.000000}\selectfont \(\displaystyle \textnormal{tol}=10^{-8}\)}%
\end{pgfscope}%
\begin{pgfscope}%
\pgfsetbuttcap%
\pgfsetroundjoin%
\pgfsetlinewidth{1.003750pt}%
\definecolor{currentstroke}{rgb}{0.000000,0.000000,0.000000}%
\pgfsetstrokecolor{currentstroke}%
\pgfsetdash{{3.700000pt}{1.600000pt}}{0.000000pt}%
\pgfpathmoveto{\pgfqpoint{3.489477in}{0.454389in}}%
\pgfpathlineto{\pgfqpoint{3.558922in}{0.454389in}}%
\pgfusepath{stroke}%
\end{pgfscope}%
\begin{pgfscope}%
\pgftext[x=3.670033in,y=0.405778in,left,base]{\rmfamily\fontsize{10.000000}{12.000000}\selectfont Reference}%
\end{pgfscope}%
\begin{pgfscope}%
\pgfsetbuttcap%
\pgfsetmiterjoin%
\definecolor{currentfill}{rgb}{1.000000,1.000000,1.000000}%
\pgfsetfillcolor{currentfill}%
\pgfsetlinewidth{0.000000pt}%
\definecolor{currentstroke}{rgb}{0.000000,0.000000,0.000000}%
\pgfsetstrokecolor{currentstroke}%
\pgfsetstrokeopacity{0.000000}%
\pgfsetdash{}{0pt}%
\pgfpathmoveto{\pgfqpoint{0.303000in}{1.712694in}}%
\pgfpathlineto{\pgfqpoint{2.598902in}{1.712694in}}%
\pgfpathlineto{\pgfqpoint{2.598902in}{3.075200in}}%
\pgfpathlineto{\pgfqpoint{0.303000in}{3.075200in}}%
\pgfpathclose%
\pgfusepath{fill}%
\end{pgfscope}%
\begin{pgfscope}%
\pgfsetbuttcap%
\pgfsetroundjoin%
\definecolor{currentfill}{rgb}{0.000000,0.000000,0.000000}%
\pgfsetfillcolor{currentfill}%
\pgfsetlinewidth{0.803000pt}%
\definecolor{currentstroke}{rgb}{0.000000,0.000000,0.000000}%
\pgfsetstrokecolor{currentstroke}%
\pgfsetdash{}{0pt}%
\pgfsys@defobject{currentmarker}{\pgfqpoint{0.000000in}{-0.048611in}}{\pgfqpoint{0.000000in}{0.000000in}}{%
\pgfpathmoveto{\pgfqpoint{0.000000in}{0.000000in}}%
\pgfpathlineto{\pgfqpoint{0.000000in}{-0.048611in}}%
\pgfusepath{stroke,fill}%
}%
\begin{pgfscope}%
\pgfsys@transformshift{0.573106in}{1.712694in}%
\pgfsys@useobject{currentmarker}{}%
\end{pgfscope}%
\end{pgfscope}%
\begin{pgfscope}%
\pgfsetbuttcap%
\pgfsetroundjoin%
\definecolor{currentfill}{rgb}{0.000000,0.000000,0.000000}%
\pgfsetfillcolor{currentfill}%
\pgfsetlinewidth{0.803000pt}%
\definecolor{currentstroke}{rgb}{0.000000,0.000000,0.000000}%
\pgfsetstrokecolor{currentstroke}%
\pgfsetdash{}{0pt}%
\pgfsys@defobject{currentmarker}{\pgfqpoint{0.000000in}{-0.048611in}}{\pgfqpoint{0.000000in}{0.000000in}}{%
\pgfpathmoveto{\pgfqpoint{0.000000in}{0.000000in}}%
\pgfpathlineto{\pgfqpoint{0.000000in}{-0.048611in}}%
\pgfusepath{stroke,fill}%
}%
\begin{pgfscope}%
\pgfsys@transformshift{1.113319in}{1.712694in}%
\pgfsys@useobject{currentmarker}{}%
\end{pgfscope}%
\end{pgfscope}%
\begin{pgfscope}%
\pgfsetbuttcap%
\pgfsetroundjoin%
\definecolor{currentfill}{rgb}{0.000000,0.000000,0.000000}%
\pgfsetfillcolor{currentfill}%
\pgfsetlinewidth{0.803000pt}%
\definecolor{currentstroke}{rgb}{0.000000,0.000000,0.000000}%
\pgfsetstrokecolor{currentstroke}%
\pgfsetdash{}{0pt}%
\pgfsys@defobject{currentmarker}{\pgfqpoint{0.000000in}{-0.048611in}}{\pgfqpoint{0.000000in}{0.000000in}}{%
\pgfpathmoveto{\pgfqpoint{0.000000in}{0.000000in}}%
\pgfpathlineto{\pgfqpoint{0.000000in}{-0.048611in}}%
\pgfusepath{stroke,fill}%
}%
\begin{pgfscope}%
\pgfsys@transformshift{1.653531in}{1.712694in}%
\pgfsys@useobject{currentmarker}{}%
\end{pgfscope}%
\end{pgfscope}%
\begin{pgfscope}%
\pgfsetbuttcap%
\pgfsetroundjoin%
\definecolor{currentfill}{rgb}{0.000000,0.000000,0.000000}%
\pgfsetfillcolor{currentfill}%
\pgfsetlinewidth{0.803000pt}%
\definecolor{currentstroke}{rgb}{0.000000,0.000000,0.000000}%
\pgfsetstrokecolor{currentstroke}%
\pgfsetdash{}{0pt}%
\pgfsys@defobject{currentmarker}{\pgfqpoint{0.000000in}{-0.048611in}}{\pgfqpoint{0.000000in}{0.000000in}}{%
\pgfpathmoveto{\pgfqpoint{0.000000in}{0.000000in}}%
\pgfpathlineto{\pgfqpoint{0.000000in}{-0.048611in}}%
\pgfusepath{stroke,fill}%
}%
\begin{pgfscope}%
\pgfsys@transformshift{2.193743in}{1.712694in}%
\pgfsys@useobject{currentmarker}{}%
\end{pgfscope}%
\end{pgfscope}%
\begin{pgfscope}%
\pgfsetbuttcap%
\pgfsetroundjoin%
\definecolor{currentfill}{rgb}{0.000000,0.000000,0.000000}%
\pgfsetfillcolor{currentfill}%
\pgfsetlinewidth{0.803000pt}%
\definecolor{currentstroke}{rgb}{0.000000,0.000000,0.000000}%
\pgfsetstrokecolor{currentstroke}%
\pgfsetdash{}{0pt}%
\pgfsys@defobject{currentmarker}{\pgfqpoint{-0.048611in}{0.000000in}}{\pgfqpoint{0.000000in}{0.000000in}}{%
\pgfpathmoveto{\pgfqpoint{0.000000in}{0.000000in}}%
\pgfpathlineto{\pgfqpoint{-0.048611in}{0.000000in}}%
\pgfusepath{stroke,fill}%
}%
\begin{pgfscope}%
\pgfsys@transformshift{0.303000in}{1.919134in}%
\pgfsys@useobject{currentmarker}{}%
\end{pgfscope}%
\end{pgfscope}%
\begin{pgfscope}%
\pgftext[x=0.037306in,y=1.871183in,left,base]{\rmfamily\fontsize{10.000000}{12.000000}\selectfont \(\displaystyle 0.7\)}%
\end{pgfscope}%
\begin{pgfscope}%
\pgfsetbuttcap%
\pgfsetroundjoin%
\definecolor{currentfill}{rgb}{0.000000,0.000000,0.000000}%
\pgfsetfillcolor{currentfill}%
\pgfsetlinewidth{0.803000pt}%
\definecolor{currentstroke}{rgb}{0.000000,0.000000,0.000000}%
\pgfsetstrokecolor{currentstroke}%
\pgfsetdash{}{0pt}%
\pgfsys@defobject{currentmarker}{\pgfqpoint{-0.048611in}{0.000000in}}{\pgfqpoint{0.000000in}{0.000000in}}{%
\pgfpathmoveto{\pgfqpoint{0.000000in}{0.000000in}}%
\pgfpathlineto{\pgfqpoint{-0.048611in}{0.000000in}}%
\pgfusepath{stroke,fill}%
}%
\begin{pgfscope}%
\pgfsys@transformshift{0.303000in}{2.332015in}%
\pgfsys@useobject{currentmarker}{}%
\end{pgfscope}%
\end{pgfscope}%
\begin{pgfscope}%
\pgftext[x=0.037306in,y=2.284064in,left,base]{\rmfamily\fontsize{10.000000}{12.000000}\selectfont \(\displaystyle 0.8\)}%
\end{pgfscope}%
\begin{pgfscope}%
\pgfsetbuttcap%
\pgfsetroundjoin%
\definecolor{currentfill}{rgb}{0.000000,0.000000,0.000000}%
\pgfsetfillcolor{currentfill}%
\pgfsetlinewidth{0.803000pt}%
\definecolor{currentstroke}{rgb}{0.000000,0.000000,0.000000}%
\pgfsetstrokecolor{currentstroke}%
\pgfsetdash{}{0pt}%
\pgfsys@defobject{currentmarker}{\pgfqpoint{-0.048611in}{0.000000in}}{\pgfqpoint{0.000000in}{0.000000in}}{%
\pgfpathmoveto{\pgfqpoint{0.000000in}{0.000000in}}%
\pgfpathlineto{\pgfqpoint{-0.048611in}{0.000000in}}%
\pgfusepath{stroke,fill}%
}%
\begin{pgfscope}%
\pgfsys@transformshift{0.303000in}{2.744896in}%
\pgfsys@useobject{currentmarker}{}%
\end{pgfscope}%
\end{pgfscope}%
\begin{pgfscope}%
\pgftext[x=0.037306in,y=2.696944in,left,base]{\rmfamily\fontsize{10.000000}{12.000000}\selectfont \(\displaystyle 0.9\)}%
\end{pgfscope}%
\begin{pgfscope}%
\pgfpathrectangle{\pgfqpoint{0.303000in}{1.712694in}}{\pgfqpoint{2.295902in}{1.362506in}} %
\pgfusepath{clip}%
\pgfsetrectcap%
\pgfsetroundjoin%
\pgfsetlinewidth{1.505625pt}%
\definecolor{currentstroke}{rgb}{0.121569,0.466667,0.705882}%
\pgfsetstrokecolor{currentstroke}%
\pgfsetdash{}{0pt}%
\pgfusepath{stroke}%
\end{pgfscope}%
\begin{pgfscope}%
\pgfpathrectangle{\pgfqpoint{0.303000in}{1.712694in}}{\pgfqpoint{2.295902in}{1.362506in}} %
\pgfusepath{clip}%
\pgfsetrectcap%
\pgfsetroundjoin%
\pgfsetlinewidth{1.505625pt}%
\definecolor{currentstroke}{rgb}{1.000000,0.498039,0.054902}%
\pgfsetstrokecolor{currentstroke}%
\pgfsetdash{}{0pt}%
\pgfusepath{stroke}%
\end{pgfscope}%
\begin{pgfscope}%
\pgfpathrectangle{\pgfqpoint{0.303000in}{1.712694in}}{\pgfqpoint{2.295902in}{1.362506in}} %
\pgfusepath{clip}%
\pgfsetrectcap%
\pgfsetroundjoin%
\pgfsetlinewidth{1.505625pt}%
\definecolor{currentstroke}{rgb}{0.172549,0.627451,0.172549}%
\pgfsetstrokecolor{currentstroke}%
\pgfsetdash{}{0pt}%
\pgfusepath{stroke}%
\end{pgfscope}%
\begin{pgfscope}%
\pgfpathrectangle{\pgfqpoint{0.303000in}{1.712694in}}{\pgfqpoint{2.295902in}{1.362506in}} %
\pgfusepath{clip}%
\pgfsetrectcap%
\pgfsetroundjoin%
\pgfsetlinewidth{1.505625pt}%
\definecolor{currentstroke}{rgb}{0.839216,0.152941,0.156863}%
\pgfsetstrokecolor{currentstroke}%
\pgfsetdash{}{0pt}%
\pgfusepath{stroke}%
\end{pgfscope}%
\begin{pgfscope}%
\pgfpathrectangle{\pgfqpoint{0.303000in}{1.712694in}}{\pgfqpoint{2.295902in}{1.362506in}} %
\pgfusepath{clip}%
\pgfsetrectcap%
\pgfsetroundjoin%
\pgfsetlinewidth{1.505625pt}%
\definecolor{currentstroke}{rgb}{0.580392,0.403922,0.741176}%
\pgfsetstrokecolor{currentstroke}%
\pgfsetdash{}{0pt}%
\pgfusepath{stroke}%
\end{pgfscope}%
\begin{pgfscope}%
\pgfpathrectangle{\pgfqpoint{0.303000in}{1.712694in}}{\pgfqpoint{2.295902in}{1.362506in}} %
\pgfusepath{clip}%
\pgfsetrectcap%
\pgfsetroundjoin%
\pgfsetlinewidth{1.003750pt}%
\definecolor{currentstroke}{rgb}{0.549020,0.337255,0.294118}%
\pgfsetstrokecolor{currentstroke}%
\pgfsetdash{}{0pt}%
\pgfpathmoveto{\pgfqpoint{0.335391in}{2.627517in}}%
\pgfpathlineto{\pgfqpoint{0.368283in}{2.656023in}}%
\pgfpathlineto{\pgfqpoint{0.401989in}{2.682218in}}%
\pgfpathlineto{\pgfqpoint{0.436381in}{2.706239in}}%
\pgfpathlineto{\pgfqpoint{0.473866in}{2.729735in}}%
\pgfpathlineto{\pgfqpoint{0.514445in}{2.752450in}}%
\pgfpathlineto{\pgfqpoint{0.558104in}{2.774180in}}%
\pgfpathlineto{\pgfqpoint{0.602212in}{2.793685in}}%
\pgfpathlineto{\pgfqpoint{0.649301in}{2.812174in}}%
\pgfpathlineto{\pgfqpoint{0.701986in}{2.830407in}}%
\pgfpathlineto{\pgfqpoint{0.757626in}{2.847256in}}%
\pgfpathlineto{\pgfqpoint{0.816182in}{2.862707in}}%
\pgfpathlineto{\pgfqpoint{0.880296in}{2.877329in}}%
\pgfpathlineto{\pgfqpoint{0.949960in}{2.890894in}}%
\pgfpathlineto{\pgfqpoint{1.025157in}{2.903217in}}%
\pgfpathlineto{\pgfqpoint{1.105872in}{2.914145in}}%
\pgfpathlineto{\pgfqpoint{1.192086in}{2.923546in}}%
\pgfpathlineto{\pgfqpoint{1.283781in}{2.931290in}}%
\pgfpathlineto{\pgfqpoint{1.380941in}{2.937230in}}%
\pgfpathlineto{\pgfqpoint{1.480848in}{2.941096in}}%
\pgfpathlineto{\pgfqpoint{1.583481in}{2.942809in}}%
\pgfpathlineto{\pgfqpoint{1.683419in}{2.942262in}}%
\pgfpathlineto{\pgfqpoint{1.780639in}{2.939496in}}%
\pgfpathlineto{\pgfqpoint{1.872419in}{2.934633in}}%
\pgfpathlineto{\pgfqpoint{1.956037in}{2.927965in}}%
\pgfpathlineto{\pgfqpoint{2.031473in}{2.919724in}}%
\pgfpathlineto{\pgfqpoint{2.098707in}{2.910152in}}%
\pgfpathlineto{\pgfqpoint{2.157722in}{2.899546in}}%
\pgfpathlineto{\pgfqpoint{2.208510in}{2.888287in}}%
\pgfpathlineto{\pgfqpoint{2.253728in}{2.876104in}}%
\pgfpathlineto{\pgfqpoint{2.293357in}{2.863230in}}%
\pgfpathlineto{\pgfqpoint{2.327387in}{2.850011in}}%
\pgfpathlineto{\pgfqpoint{2.358401in}{2.835635in}}%
\pgfpathlineto{\pgfqpoint{2.386321in}{2.820121in}}%
\pgfpathlineto{\pgfqpoint{2.408618in}{2.805332in}}%
\pgfpathlineto{\pgfqpoint{2.427832in}{2.790232in}}%
\pgfpathlineto{\pgfqpoint{2.446226in}{2.772917in}}%
\pgfpathlineto{\pgfqpoint{2.461376in}{2.755643in}}%
\pgfpathlineto{\pgfqpoint{2.475320in}{2.736146in}}%
\pgfpathlineto{\pgfqpoint{2.486031in}{2.717569in}}%
\pgfpathlineto{\pgfqpoint{2.495344in}{2.697312in}}%
\pgfpathlineto{\pgfqpoint{2.503075in}{2.675555in}}%
\pgfpathlineto{\pgfqpoint{2.509989in}{2.648676in}}%
\pgfpathlineto{\pgfqpoint{2.514710in}{2.620707in}}%
\pgfpathlineto{\pgfqpoint{2.517741in}{2.588019in}}%
\pgfpathlineto{\pgfqpoint{2.518710in}{2.550903in}}%
\pgfpathlineto{\pgfqpoint{2.517520in}{2.509664in}}%
\pgfpathlineto{\pgfqpoint{2.513909in}{2.460434in}}%
\pgfpathlineto{\pgfqpoint{2.506771in}{2.395287in}}%
\pgfpathlineto{\pgfqpoint{2.494100in}{2.302325in}}%
\pgfpathlineto{\pgfqpoint{2.434891in}{1.886824in}}%
\pgfpathlineto{\pgfqpoint{2.416986in}{1.736533in}}%
\pgfpathlineto{\pgfqpoint{2.413264in}{1.702694in}}%
\pgfpathlineto{\pgfqpoint{2.413264in}{1.702694in}}%
\pgfusepath{stroke}%
\end{pgfscope}%
\begin{pgfscope}%
\pgfpathrectangle{\pgfqpoint{0.303000in}{1.712694in}}{\pgfqpoint{2.295902in}{1.362506in}} %
\pgfusepath{clip}%
\pgfsetrectcap%
\pgfsetroundjoin%
\pgfsetlinewidth{1.003750pt}%
\definecolor{currentstroke}{rgb}{0.549020,0.337255,0.294118}%
\pgfsetstrokecolor{currentstroke}%
\pgfsetdash{}{0pt}%
\pgfpathmoveto{\pgfqpoint{2.276478in}{1.702694in}}%
\pgfpathlineto{\pgfqpoint{2.279923in}{1.864275in}}%
\pgfpathlineto{\pgfqpoint{2.280370in}{1.984003in}}%
\pgfpathlineto{\pgfqpoint{2.278538in}{2.078918in}}%
\pgfpathlineto{\pgfqpoint{2.275039in}{2.153038in}}%
\pgfpathlineto{\pgfqpoint{2.269768in}{2.218597in}}%
\pgfpathlineto{\pgfqpoint{2.262921in}{2.275435in}}%
\pgfpathlineto{\pgfqpoint{2.254939in}{2.323444in}}%
\pgfpathlineto{\pgfqpoint{2.245547in}{2.366521in}}%
\pgfpathlineto{\pgfqpoint{2.235029in}{2.404538in}}%
\pgfpathlineto{\pgfqpoint{2.223799in}{2.437484in}}%
\pgfpathlineto{\pgfqpoint{2.210832in}{2.468902in}}%
\pgfpathlineto{\pgfqpoint{2.196142in}{2.498494in}}%
\pgfpathlineto{\pgfqpoint{2.179828in}{2.526027in}}%
\pgfpathlineto{\pgfqpoint{2.162053in}{2.551361in}}%
\pgfpathlineto{\pgfqpoint{2.143022in}{2.574467in}}%
\pgfpathlineto{\pgfqpoint{2.122945in}{2.595405in}}%
\pgfpathlineto{\pgfqpoint{2.099649in}{2.616286in}}%
\pgfpathlineto{\pgfqpoint{2.075536in}{2.634877in}}%
\pgfpathlineto{\pgfqpoint{2.048288in}{2.652972in}}%
\pgfpathlineto{\pgfqpoint{2.017915in}{2.670252in}}%
\pgfpathlineto{\pgfqpoint{1.987039in}{2.685317in}}%
\pgfpathlineto{\pgfqpoint{1.953175in}{2.699492in}}%
\pgfpathlineto{\pgfqpoint{1.913709in}{2.713488in}}%
\pgfpathlineto{\pgfqpoint{1.871284in}{2.726056in}}%
\pgfpathlineto{\pgfqpoint{1.825945in}{2.737151in}}%
\pgfpathlineto{\pgfqpoint{1.775051in}{2.747213in}}%
\pgfpathlineto{\pgfqpoint{1.721305in}{2.755523in}}%
\pgfpathlineto{\pgfqpoint{1.664746in}{2.762068in}}%
\pgfpathlineto{\pgfqpoint{1.602706in}{2.766979in}}%
\pgfpathlineto{\pgfqpoint{1.537909in}{2.769839in}}%
\pgfpathlineto{\pgfqpoint{1.470386in}{2.770555in}}%
\pgfpathlineto{\pgfqpoint{1.402867in}{2.769080in}}%
\pgfpathlineto{\pgfqpoint{1.332685in}{2.765280in}}%
\pgfpathlineto{\pgfqpoint{1.265268in}{2.759427in}}%
\pgfpathlineto{\pgfqpoint{1.197950in}{2.751346in}}%
\pgfpathlineto{\pgfqpoint{1.133456in}{2.741369in}}%
\pgfpathlineto{\pgfqpoint{1.071811in}{2.729612in}}%
\pgfpathlineto{\pgfqpoint{1.013043in}{2.716170in}}%
\pgfpathlineto{\pgfqpoint{0.957182in}{2.701135in}}%
\pgfpathlineto{\pgfqpoint{0.904256in}{2.684602in}}%
\pgfpathlineto{\pgfqpoint{0.854295in}{2.666679in}}%
\pgfpathlineto{\pgfqpoint{0.807327in}{2.647490in}}%
\pgfpathlineto{\pgfqpoint{0.763375in}{2.627185in}}%
\pgfpathlineto{\pgfqpoint{0.722454in}{2.605946in}}%
\pgfpathlineto{\pgfqpoint{0.682067in}{2.582447in}}%
\pgfpathlineto{\pgfqpoint{0.644792in}{2.558186in}}%
\pgfpathlineto{\pgfqpoint{0.610620in}{2.533441in}}%
\pgfpathlineto{\pgfqpoint{0.577157in}{2.506530in}}%
\pgfpathlineto{\pgfqpoint{0.544526in}{2.477329in}}%
\pgfpathlineto{\pgfqpoint{0.515091in}{2.448071in}}%
\pgfpathlineto{\pgfqpoint{0.486612in}{2.416682in}}%
\pgfpathlineto{\pgfqpoint{0.459211in}{2.383126in}}%
\pgfpathlineto{\pgfqpoint{0.433013in}{2.347399in}}%
\pgfpathlineto{\pgfqpoint{0.408133in}{2.309532in}}%
\pgfpathlineto{\pgfqpoint{0.384680in}{2.269595in}}%
\pgfpathlineto{\pgfqpoint{0.362743in}{2.227693in}}%
\pgfpathlineto{\pgfqpoint{0.342392in}{2.183962in}}%
\pgfpathlineto{\pgfqpoint{0.323672in}{2.138559in}}%
\pgfpathlineto{\pgfqpoint{0.306608in}{2.091656in}}%
\pgfpathlineto{\pgfqpoint{0.293000in}{2.049395in}}%
\pgfpathmoveto{\pgfqpoint{0.293000in}{2.373553in}}%
\pgfpathlineto{\pgfqpoint{0.316883in}{2.411544in}}%
\pgfpathlineto{\pgfqpoint{0.342547in}{2.448167in}}%
\pgfpathlineto{\pgfqpoint{0.369457in}{2.482639in}}%
\pgfpathlineto{\pgfqpoint{0.397497in}{2.514938in}}%
\pgfpathlineto{\pgfqpoint{0.426546in}{2.545081in}}%
\pgfpathlineto{\pgfqpoint{0.456485in}{2.573118in}}%
\pgfpathlineto{\pgfqpoint{0.489591in}{2.601044in}}%
\pgfpathlineto{\pgfqpoint{0.523462in}{2.626736in}}%
\pgfpathlineto{\pgfqpoint{0.560468in}{2.651940in}}%
\pgfpathlineto{\pgfqpoint{0.598090in}{2.674915in}}%
\pgfpathlineto{\pgfqpoint{0.638779in}{2.697167in}}%
\pgfpathlineto{\pgfqpoint{0.682525in}{2.718487in}}%
\pgfpathlineto{\pgfqpoint{0.729307in}{2.738712in}}%
\pgfpathlineto{\pgfqpoint{0.779097in}{2.757714in}}%
\pgfpathlineto{\pgfqpoint{0.831863in}{2.775400in}}%
\pgfpathlineto{\pgfqpoint{0.887572in}{2.791705in}}%
\pgfpathlineto{\pgfqpoint{0.946191in}{2.806583in}}%
\pgfpathlineto{\pgfqpoint{1.010369in}{2.820542in}}%
\pgfpathlineto{\pgfqpoint{1.077412in}{2.832850in}}%
\pgfpathlineto{\pgfqpoint{1.149982in}{2.843872in}}%
\pgfpathlineto{\pgfqpoint{1.225371in}{2.853067in}}%
\pgfpathlineto{\pgfqpoint{1.303554in}{2.860410in}}%
\pgfpathlineto{\pgfqpoint{1.384507in}{2.865840in}}%
\pgfpathlineto{\pgfqpoint{1.468209in}{2.869243in}}%
\pgfpathlineto{\pgfqpoint{1.551937in}{2.870435in}}%
\pgfpathlineto{\pgfqpoint{1.632965in}{2.869431in}}%
\pgfpathlineto{\pgfqpoint{1.711268in}{2.866311in}}%
\pgfpathlineto{\pgfqpoint{1.786819in}{2.861075in}}%
\pgfpathlineto{\pgfqpoint{1.856891in}{2.853968in}}%
\pgfpathlineto{\pgfqpoint{1.921458in}{2.845141in}}%
\pgfpathlineto{\pgfqpoint{1.977809in}{2.835265in}}%
\pgfpathlineto{\pgfqpoint{2.028616in}{2.824219in}}%
\pgfpathlineto{\pgfqpoint{2.073860in}{2.812255in}}%
\pgfpathlineto{\pgfqpoint{2.116166in}{2.798778in}}%
\pgfpathlineto{\pgfqpoint{2.152857in}{2.784810in}}%
\pgfpathlineto{\pgfqpoint{2.186527in}{2.769592in}}%
\pgfpathlineto{\pgfqpoint{2.214582in}{2.754650in}}%
\pgfpathlineto{\pgfqpoint{2.239596in}{2.739084in}}%
\pgfpathlineto{\pgfqpoint{2.263957in}{2.721268in}}%
\pgfpathlineto{\pgfqpoint{2.285112in}{2.702978in}}%
\pgfpathlineto{\pgfqpoint{2.303090in}{2.684667in}}%
\pgfpathlineto{\pgfqpoint{2.320050in}{2.664216in}}%
\pgfpathlineto{\pgfqpoint{2.335721in}{2.641493in}}%
\pgfpathlineto{\pgfqpoint{2.348152in}{2.619730in}}%
\pgfpathlineto{\pgfqpoint{2.359211in}{2.596301in}}%
\pgfpathlineto{\pgfqpoint{2.370027in}{2.567723in}}%
\pgfpathlineto{\pgfqpoint{2.378854in}{2.537589in}}%
\pgfpathlineto{\pgfqpoint{2.385778in}{2.506312in}}%
\pgfpathlineto{\pgfqpoint{2.391522in}{2.470217in}}%
\pgfpathlineto{\pgfqpoint{2.395754in}{2.429449in}}%
\pgfpathlineto{\pgfqpoint{2.398461in}{2.380086in}}%
\pgfpathlineto{\pgfqpoint{2.399196in}{2.322302in}}%
\pgfpathlineto{\pgfqpoint{2.397683in}{2.252157in}}%
\pgfpathlineto{\pgfqpoint{2.393296in}{2.161576in}}%
\pgfpathlineto{\pgfqpoint{2.384407in}{2.030157in}}%
\pgfpathlineto{\pgfqpoint{2.359372in}{1.702694in}}%
\pgfpathlineto{\pgfqpoint{2.359372in}{1.702694in}}%
\pgfusepath{stroke}%
\end{pgfscope}%
\begin{pgfscope}%
\pgfpathrectangle{\pgfqpoint{0.303000in}{1.712694in}}{\pgfqpoint{2.295902in}{1.362506in}} %
\pgfusepath{clip}%
\pgfsetrectcap%
\pgfsetroundjoin%
\pgfsetlinewidth{1.003750pt}%
\definecolor{currentstroke}{rgb}{0.549020,0.337255,0.294118}%
\pgfsetstrokecolor{currentstroke}%
\pgfsetdash{}{0pt}%
\pgfpathmoveto{\pgfqpoint{2.421100in}{1.702694in}}%
\pgfpathlineto{\pgfqpoint{2.438080in}{1.842389in}}%
\pgfpathlineto{\pgfqpoint{2.456985in}{1.979759in}}%
\pgfpathlineto{\pgfqpoint{2.478573in}{2.120446in}}%
\pgfpathlineto{\pgfqpoint{2.526495in}{2.425410in}}%
\pgfpathlineto{\pgfqpoint{2.534951in}{2.494389in}}%
\pgfpathlineto{\pgfqpoint{2.538881in}{2.543559in}}%
\pgfpathlineto{\pgfqpoint{2.540016in}{2.584797in}}%
\pgfpathlineto{\pgfqpoint{2.538910in}{2.617771in}}%
\pgfpathlineto{\pgfqpoint{2.536011in}{2.646317in}}%
\pgfpathlineto{\pgfqpoint{2.530937in}{2.674138in}}%
\pgfpathlineto{\pgfqpoint{2.524674in}{2.696970in}}%
\pgfpathlineto{\pgfqpoint{2.516628in}{2.718456in}}%
\pgfpathlineto{\pgfqpoint{2.506934in}{2.738290in}}%
\pgfpathlineto{\pgfqpoint{2.495831in}{2.756320in}}%
\pgfpathlineto{\pgfqpoint{2.481465in}{2.775088in}}%
\pgfpathlineto{\pgfqpoint{2.465952in}{2.791594in}}%
\pgfpathlineto{\pgfqpoint{2.447228in}{2.808064in}}%
\pgfpathlineto{\pgfqpoint{2.425290in}{2.824052in}}%
\pgfpathlineto{\pgfqpoint{2.400184in}{2.839263in}}%
\pgfpathlineto{\pgfqpoint{2.371977in}{2.853519in}}%
\pgfpathlineto{\pgfqpoint{2.340751in}{2.866784in}}%
\pgfpathlineto{\pgfqpoint{2.303922in}{2.879880in}}%
\pgfpathlineto{\pgfqpoint{2.261493in}{2.892418in}}%
\pgfpathlineto{\pgfqpoint{2.213482in}{2.904124in}}%
\pgfpathlineto{\pgfqpoint{2.159918in}{2.914824in}}%
\pgfpathlineto{\pgfqpoint{2.098138in}{2.924784in}}%
\pgfpathlineto{\pgfqpoint{2.030846in}{2.933369in}}%
\pgfpathlineto{\pgfqpoint{1.955373in}{2.940771in}}%
\pgfpathlineto{\pgfqpoint{1.871732in}{2.946750in}}%
\pgfpathlineto{\pgfqpoint{1.788039in}{2.950593in}}%
\pgfpathlineto{\pgfqpoint{1.690816in}{2.953158in}}%
\pgfpathlineto{\pgfqpoint{1.588177in}{2.953654in}}%
\pgfpathlineto{\pgfqpoint{1.482843in}{2.951914in}}%
\pgfpathlineto{\pgfqpoint{1.369434in}{2.947698in}}%
\pgfpathlineto{\pgfqpoint{1.269577in}{2.941529in}}%
\pgfpathlineto{\pgfqpoint{1.156304in}{2.932118in}}%
\pgfpathlineto{\pgfqpoint{1.070111in}{2.922286in}}%
\pgfpathlineto{\pgfqpoint{0.989420in}{2.910948in}}%
\pgfpathlineto{\pgfqpoint{0.914251in}{2.898225in}}%
\pgfpathlineto{\pgfqpoint{0.815153in}{2.878818in}}%
\pgfpathlineto{\pgfqpoint{0.735164in}{2.859159in}}%
\pgfpathlineto{\pgfqpoint{0.679501in}{2.842489in}}%
\pgfpathlineto{\pgfqpoint{0.626786in}{2.824452in}}%
\pgfpathlineto{\pgfqpoint{0.577058in}{2.805077in}}%
\pgfpathlineto{\pgfqpoint{0.530351in}{2.784451in}}%
\pgfpathlineto{\pgfqpoint{0.486707in}{2.762649in}}%
\pgfpathlineto{\pgfqpoint{0.443599in}{2.738491in}}%
\pgfpathlineto{\pgfqpoint{0.406181in}{2.714750in}}%
\pgfpathlineto{\pgfqpoint{0.371866in}{2.690471in}}%
\pgfpathlineto{\pgfqpoint{0.338258in}{2.663986in}}%
\pgfpathlineto{\pgfqpoint{0.307793in}{2.637303in}}%
\pgfpathlineto{\pgfqpoint{0.293000in}{2.622857in}}%
\pgfpathlineto{\pgfqpoint{0.293000in}{2.622857in}}%
\pgfusepath{stroke}%
\end{pgfscope}%
\begin{pgfscope}%
\pgfpathrectangle{\pgfqpoint{0.303000in}{1.712694in}}{\pgfqpoint{2.295902in}{1.362506in}} %
\pgfusepath{clip}%
\pgfsetbuttcap%
\pgfsetroundjoin%
\pgfsetlinewidth{1.003750pt}%
\definecolor{currentstroke}{rgb}{0.000000,0.000000,0.000000}%
\pgfsetstrokecolor{currentstroke}%
\pgfsetdash{{3.700000pt}{1.600000pt}}{0.000000pt}%
\pgfpathmoveto{\pgfqpoint{0.335391in}{2.627517in}}%
\pgfpathlineto{\pgfqpoint{0.368284in}{2.656023in}}%
\pgfpathlineto{\pgfqpoint{0.401989in}{2.682218in}}%
\pgfpathlineto{\pgfqpoint{0.436381in}{2.706239in}}%
\pgfpathlineto{\pgfqpoint{0.473866in}{2.729735in}}%
\pgfpathlineto{\pgfqpoint{0.514445in}{2.752451in}}%
\pgfpathlineto{\pgfqpoint{0.558104in}{2.774181in}}%
\pgfpathlineto{\pgfqpoint{0.602212in}{2.793685in}}%
\pgfpathlineto{\pgfqpoint{0.649301in}{2.812174in}}%
\pgfpathlineto{\pgfqpoint{0.701987in}{2.830408in}}%
\pgfpathlineto{\pgfqpoint{0.757626in}{2.847257in}}%
\pgfpathlineto{\pgfqpoint{0.816182in}{2.862707in}}%
\pgfpathlineto{\pgfqpoint{0.880297in}{2.877329in}}%
\pgfpathlineto{\pgfqpoint{0.949960in}{2.890894in}}%
\pgfpathlineto{\pgfqpoint{1.025158in}{2.903217in}}%
\pgfpathlineto{\pgfqpoint{1.105872in}{2.914145in}}%
\pgfpathlineto{\pgfqpoint{1.192086in}{2.923546in}}%
\pgfpathlineto{\pgfqpoint{1.283782in}{2.931290in}}%
\pgfpathlineto{\pgfqpoint{1.380941in}{2.937230in}}%
\pgfpathlineto{\pgfqpoint{1.480848in}{2.941097in}}%
\pgfpathlineto{\pgfqpoint{1.583481in}{2.942809in}}%
\pgfpathlineto{\pgfqpoint{1.683419in}{2.942262in}}%
\pgfpathlineto{\pgfqpoint{1.780639in}{2.939496in}}%
\pgfpathlineto{\pgfqpoint{1.872419in}{2.934633in}}%
\pgfpathlineto{\pgfqpoint{1.956037in}{2.927965in}}%
\pgfpathlineto{\pgfqpoint{2.031473in}{2.919724in}}%
\pgfpathlineto{\pgfqpoint{2.098707in}{2.910152in}}%
\pgfpathlineto{\pgfqpoint{2.157723in}{2.899546in}}%
\pgfpathlineto{\pgfqpoint{2.208510in}{2.888287in}}%
\pgfpathlineto{\pgfqpoint{2.253729in}{2.876104in}}%
\pgfpathlineto{\pgfqpoint{2.293357in}{2.863230in}}%
\pgfpathlineto{\pgfqpoint{2.327387in}{2.850011in}}%
\pgfpathlineto{\pgfqpoint{2.358402in}{2.835635in}}%
\pgfpathlineto{\pgfqpoint{2.386321in}{2.820121in}}%
\pgfpathlineto{\pgfqpoint{2.408618in}{2.805332in}}%
\pgfpathlineto{\pgfqpoint{2.427832in}{2.790232in}}%
\pgfpathlineto{\pgfqpoint{2.446226in}{2.772917in}}%
\pgfpathlineto{\pgfqpoint{2.461376in}{2.755643in}}%
\pgfpathlineto{\pgfqpoint{2.475321in}{2.736146in}}%
\pgfpathlineto{\pgfqpoint{2.486031in}{2.717569in}}%
\pgfpathlineto{\pgfqpoint{2.495345in}{2.697312in}}%
\pgfpathlineto{\pgfqpoint{2.503076in}{2.675555in}}%
\pgfpathlineto{\pgfqpoint{2.509989in}{2.648676in}}%
\pgfpathlineto{\pgfqpoint{2.514711in}{2.620707in}}%
\pgfpathlineto{\pgfqpoint{2.517741in}{2.588019in}}%
\pgfpathlineto{\pgfqpoint{2.518710in}{2.550903in}}%
\pgfpathlineto{\pgfqpoint{2.517520in}{2.509664in}}%
\pgfpathlineto{\pgfqpoint{2.513909in}{2.460434in}}%
\pgfpathlineto{\pgfqpoint{2.506771in}{2.395287in}}%
\pgfpathlineto{\pgfqpoint{2.494100in}{2.302325in}}%
\pgfpathlineto{\pgfqpoint{2.434891in}{1.886824in}}%
\pgfpathlineto{\pgfqpoint{2.416986in}{1.736533in}}%
\pgfpathlineto{\pgfqpoint{2.413264in}{1.702694in}}%
\pgfpathlineto{\pgfqpoint{2.413264in}{1.702694in}}%
\pgfusepath{stroke}%
\end{pgfscope}%
\begin{pgfscope}%
\pgfpathrectangle{\pgfqpoint{0.303000in}{1.712694in}}{\pgfqpoint{2.295902in}{1.362506in}} %
\pgfusepath{clip}%
\pgfsetbuttcap%
\pgfsetroundjoin%
\pgfsetlinewidth{1.003750pt}%
\definecolor{currentstroke}{rgb}{0.000000,0.000000,0.000000}%
\pgfsetstrokecolor{currentstroke}%
\pgfsetdash{{3.700000pt}{1.600000pt}}{0.000000pt}%
\pgfpathmoveto{\pgfqpoint{0.000000in}{0.000000in}}%
\pgfusepath{stroke}%
\end{pgfscope}%
\begin{pgfscope}%
\pgfpathrectangle{\pgfqpoint{0.303000in}{1.712694in}}{\pgfqpoint{2.295902in}{1.362506in}} %
\pgfusepath{clip}%
\pgfsetbuttcap%
\pgfsetroundjoin%
\pgfsetlinewidth{1.003750pt}%
\definecolor{currentstroke}{rgb}{0.000000,0.000000,0.000000}%
\pgfsetstrokecolor{currentstroke}%
\pgfsetdash{{3.700000pt}{1.600000pt}}{0.000000pt}%
\pgfpathmoveto{\pgfqpoint{0.000000in}{0.000000in}}%
\pgfusepath{stroke}%
\end{pgfscope}%
\begin{pgfscope}%
\pgfpathrectangle{\pgfqpoint{0.303000in}{1.712694in}}{\pgfqpoint{2.295902in}{1.362506in}} %
\pgfusepath{clip}%
\pgfsetbuttcap%
\pgfsetroundjoin%
\pgfsetlinewidth{1.003750pt}%
\definecolor{currentstroke}{rgb}{0.000000,0.000000,0.000000}%
\pgfsetstrokecolor{currentstroke}%
\pgfsetdash{{3.700000pt}{1.600000pt}}{0.000000pt}%
\pgfpathmoveto{\pgfqpoint{2.276477in}{1.702694in}}%
\pgfpathlineto{\pgfqpoint{2.279923in}{1.864272in}}%
\pgfpathlineto{\pgfqpoint{2.280370in}{1.984000in}}%
\pgfpathlineto{\pgfqpoint{2.278538in}{2.078915in}}%
\pgfpathlineto{\pgfqpoint{2.275039in}{2.153035in}}%
\pgfpathlineto{\pgfqpoint{2.269768in}{2.218594in}}%
\pgfpathlineto{\pgfqpoint{2.262921in}{2.275432in}}%
\pgfpathlineto{\pgfqpoint{2.254939in}{2.323442in}}%
\pgfpathlineto{\pgfqpoint{2.245548in}{2.366518in}}%
\pgfpathlineto{\pgfqpoint{2.235029in}{2.404535in}}%
\pgfpathlineto{\pgfqpoint{2.223800in}{2.437481in}}%
\pgfpathlineto{\pgfqpoint{2.210833in}{2.468899in}}%
\pgfpathlineto{\pgfqpoint{2.196143in}{2.498492in}}%
\pgfpathlineto{\pgfqpoint{2.179829in}{2.526025in}}%
\pgfpathlineto{\pgfqpoint{2.162054in}{2.551359in}}%
\pgfpathlineto{\pgfqpoint{2.143023in}{2.574465in}}%
\pgfpathlineto{\pgfqpoint{2.122946in}{2.595403in}}%
\pgfpathlineto{\pgfqpoint{2.099650in}{2.616284in}}%
\pgfpathlineto{\pgfqpoint{2.075537in}{2.634875in}}%
\pgfpathlineto{\pgfqpoint{2.048290in}{2.652971in}}%
\pgfpathlineto{\pgfqpoint{2.017916in}{2.670251in}}%
\pgfpathlineto{\pgfqpoint{1.987041in}{2.685316in}}%
\pgfpathlineto{\pgfqpoint{1.953176in}{2.699491in}}%
\pgfpathlineto{\pgfqpoint{1.913711in}{2.713487in}}%
\pgfpathlineto{\pgfqpoint{1.871285in}{2.726055in}}%
\pgfpathlineto{\pgfqpoint{1.825946in}{2.737150in}}%
\pgfpathlineto{\pgfqpoint{1.775052in}{2.747212in}}%
\pgfpathlineto{\pgfqpoint{1.721307in}{2.755523in}}%
\pgfpathlineto{\pgfqpoint{1.664748in}{2.762068in}}%
\pgfpathlineto{\pgfqpoint{1.602708in}{2.766979in}}%
\pgfpathlineto{\pgfqpoint{1.537911in}{2.769839in}}%
\pgfpathlineto{\pgfqpoint{1.470387in}{2.770555in}}%
\pgfpathlineto{\pgfqpoint{1.402869in}{2.769080in}}%
\pgfpathlineto{\pgfqpoint{1.332686in}{2.765280in}}%
\pgfpathlineto{\pgfqpoint{1.265270in}{2.759427in}}%
\pgfpathlineto{\pgfqpoint{1.197952in}{2.751346in}}%
\pgfpathlineto{\pgfqpoint{1.133457in}{2.741370in}}%
\pgfpathlineto{\pgfqpoint{1.071812in}{2.729612in}}%
\pgfpathlineto{\pgfqpoint{1.013045in}{2.716170in}}%
\pgfpathlineto{\pgfqpoint{0.957183in}{2.701135in}}%
\pgfpathlineto{\pgfqpoint{0.904258in}{2.684602in}}%
\pgfpathlineto{\pgfqpoint{0.854297in}{2.666679in}}%
\pgfpathlineto{\pgfqpoint{0.807329in}{2.647490in}}%
\pgfpathlineto{\pgfqpoint{0.763376in}{2.627185in}}%
\pgfpathlineto{\pgfqpoint{0.722456in}{2.605947in}}%
\pgfpathlineto{\pgfqpoint{0.682068in}{2.582448in}}%
\pgfpathlineto{\pgfqpoint{0.644793in}{2.558187in}}%
\pgfpathlineto{\pgfqpoint{0.610621in}{2.533442in}}%
\pgfpathlineto{\pgfqpoint{0.577158in}{2.506531in}}%
\pgfpathlineto{\pgfqpoint{0.544527in}{2.477331in}}%
\pgfpathlineto{\pgfqpoint{0.515092in}{2.448073in}}%
\pgfpathlineto{\pgfqpoint{0.486613in}{2.416684in}}%
\pgfpathlineto{\pgfqpoint{0.459212in}{2.383128in}}%
\pgfpathlineto{\pgfqpoint{0.433014in}{2.347400in}}%
\pgfpathlineto{\pgfqpoint{0.408134in}{2.309533in}}%
\pgfpathlineto{\pgfqpoint{0.384681in}{2.269597in}}%
\pgfpathlineto{\pgfqpoint{0.362744in}{2.227695in}}%
\pgfpathlineto{\pgfqpoint{0.342392in}{2.183964in}}%
\pgfpathlineto{\pgfqpoint{0.323673in}{2.138561in}}%
\pgfpathlineto{\pgfqpoint{0.306609in}{2.091658in}}%
\pgfpathlineto{\pgfqpoint{0.293000in}{2.049396in}}%
\pgfpathmoveto{\pgfqpoint{0.293000in}{2.373551in}}%
\pgfpathlineto{\pgfqpoint{0.316884in}{2.411544in}}%
\pgfpathlineto{\pgfqpoint{0.342548in}{2.448168in}}%
\pgfpathlineto{\pgfqpoint{0.369458in}{2.482639in}}%
\pgfpathlineto{\pgfqpoint{0.397498in}{2.514938in}}%
\pgfpathlineto{\pgfqpoint{0.426547in}{2.545081in}}%
\pgfpathlineto{\pgfqpoint{0.456487in}{2.573118in}}%
\pgfpathlineto{\pgfqpoint{0.489592in}{2.601044in}}%
\pgfpathlineto{\pgfqpoint{0.523463in}{2.626736in}}%
\pgfpathlineto{\pgfqpoint{0.560469in}{2.651940in}}%
\pgfpathlineto{\pgfqpoint{0.598091in}{2.674915in}}%
\pgfpathlineto{\pgfqpoint{0.638780in}{2.697166in}}%
\pgfpathlineto{\pgfqpoint{0.682527in}{2.718487in}}%
\pgfpathlineto{\pgfqpoint{0.729309in}{2.738712in}}%
\pgfpathlineto{\pgfqpoint{0.779098in}{2.757714in}}%
\pgfpathlineto{\pgfqpoint{0.831864in}{2.775400in}}%
\pgfpathlineto{\pgfqpoint{0.887573in}{2.791704in}}%
\pgfpathlineto{\pgfqpoint{0.946192in}{2.806583in}}%
\pgfpathlineto{\pgfqpoint{1.010370in}{2.820541in}}%
\pgfpathlineto{\pgfqpoint{1.077414in}{2.832850in}}%
\pgfpathlineto{\pgfqpoint{1.149984in}{2.843871in}}%
\pgfpathlineto{\pgfqpoint{1.225373in}{2.853066in}}%
\pgfpathlineto{\pgfqpoint{1.303555in}{2.860410in}}%
\pgfpathlineto{\pgfqpoint{1.384508in}{2.865840in}}%
\pgfpathlineto{\pgfqpoint{1.468210in}{2.869242in}}%
\pgfpathlineto{\pgfqpoint{1.551938in}{2.870434in}}%
\pgfpathlineto{\pgfqpoint{1.632967in}{2.869431in}}%
\pgfpathlineto{\pgfqpoint{1.711270in}{2.866310in}}%
\pgfpathlineto{\pgfqpoint{1.786821in}{2.861074in}}%
\pgfpathlineto{\pgfqpoint{1.856893in}{2.853968in}}%
\pgfpathlineto{\pgfqpoint{1.921459in}{2.845140in}}%
\pgfpathlineto{\pgfqpoint{1.977810in}{2.835264in}}%
\pgfpathlineto{\pgfqpoint{2.028618in}{2.824218in}}%
\pgfpathlineto{\pgfqpoint{2.073862in}{2.812254in}}%
\pgfpathlineto{\pgfqpoint{2.116167in}{2.798777in}}%
\pgfpathlineto{\pgfqpoint{2.152859in}{2.784808in}}%
\pgfpathlineto{\pgfqpoint{2.186528in}{2.769590in}}%
\pgfpathlineto{\pgfqpoint{2.214583in}{2.754648in}}%
\pgfpathlineto{\pgfqpoint{2.239597in}{2.739082in}}%
\pgfpathlineto{\pgfqpoint{2.263958in}{2.721266in}}%
\pgfpathlineto{\pgfqpoint{2.285113in}{2.702976in}}%
\pgfpathlineto{\pgfqpoint{2.303091in}{2.684664in}}%
\pgfpathlineto{\pgfqpoint{2.320051in}{2.664213in}}%
\pgfpathlineto{\pgfqpoint{2.335722in}{2.641490in}}%
\pgfpathlineto{\pgfqpoint{2.348153in}{2.619726in}}%
\pgfpathlineto{\pgfqpoint{2.359212in}{2.596297in}}%
\pgfpathlineto{\pgfqpoint{2.370027in}{2.567719in}}%
\pgfpathlineto{\pgfqpoint{2.378854in}{2.537586in}}%
\pgfpathlineto{\pgfqpoint{2.385778in}{2.506308in}}%
\pgfpathlineto{\pgfqpoint{2.391522in}{2.470213in}}%
\pgfpathlineto{\pgfqpoint{2.395753in}{2.429445in}}%
\pgfpathlineto{\pgfqpoint{2.398460in}{2.380082in}}%
\pgfpathlineto{\pgfqpoint{2.399196in}{2.322298in}}%
\pgfpathlineto{\pgfqpoint{2.397682in}{2.252153in}}%
\pgfpathlineto{\pgfqpoint{2.393295in}{2.161572in}}%
\pgfpathlineto{\pgfqpoint{2.384407in}{2.030153in}}%
\pgfpathlineto{\pgfqpoint{2.359372in}{1.702694in}}%
\pgfpathlineto{\pgfqpoint{2.359372in}{1.702694in}}%
\pgfusepath{stroke}%
\end{pgfscope}%
\begin{pgfscope}%
\pgfpathrectangle{\pgfqpoint{0.303000in}{1.712694in}}{\pgfqpoint{2.295902in}{1.362506in}} %
\pgfusepath{clip}%
\pgfsetbuttcap%
\pgfsetroundjoin%
\pgfsetlinewidth{1.003750pt}%
\definecolor{currentstroke}{rgb}{0.000000,0.000000,0.000000}%
\pgfsetstrokecolor{currentstroke}%
\pgfsetdash{{3.700000pt}{1.600000pt}}{0.000000pt}%
\pgfpathmoveto{\pgfqpoint{2.421105in}{1.702694in}}%
\pgfpathlineto{\pgfqpoint{2.438091in}{1.842432in}}%
\pgfpathlineto{\pgfqpoint{2.456997in}{1.979801in}}%
\pgfpathlineto{\pgfqpoint{2.478587in}{2.120487in}}%
\pgfpathlineto{\pgfqpoint{2.526510in}{2.425450in}}%
\pgfpathlineto{\pgfqpoint{2.534966in}{2.494430in}}%
\pgfpathlineto{\pgfqpoint{2.538895in}{2.543600in}}%
\pgfpathlineto{\pgfqpoint{2.540030in}{2.584838in}}%
\pgfpathlineto{\pgfqpoint{2.538922in}{2.617812in}}%
\pgfpathlineto{\pgfqpoint{2.536022in}{2.646358in}}%
\pgfpathlineto{\pgfqpoint{2.530946in}{2.674178in}}%
\pgfpathlineto{\pgfqpoint{2.524680in}{2.697008in}}%
\pgfpathlineto{\pgfqpoint{2.516632in}{2.718492in}}%
\pgfpathlineto{\pgfqpoint{2.506936in}{2.738324in}}%
\pgfpathlineto{\pgfqpoint{2.495832in}{2.756352in}}%
\pgfpathlineto{\pgfqpoint{2.481464in}{2.775117in}}%
\pgfpathlineto{\pgfqpoint{2.465951in}{2.791621in}}%
\pgfpathlineto{\pgfqpoint{2.447226in}{2.808088in}}%
\pgfpathlineto{\pgfqpoint{2.425286in}{2.824074in}}%
\pgfpathlineto{\pgfqpoint{2.400180in}{2.839283in}}%
\pgfpathlineto{\pgfqpoint{2.371973in}{2.853537in}}%
\pgfpathlineto{\pgfqpoint{2.340746in}{2.866800in}}%
\pgfpathlineto{\pgfqpoint{2.303917in}{2.879895in}}%
\pgfpathlineto{\pgfqpoint{2.261488in}{2.892431in}}%
\pgfpathlineto{\pgfqpoint{2.213477in}{2.904136in}}%
\pgfpathlineto{\pgfqpoint{2.159913in}{2.914836in}}%
\pgfpathlineto{\pgfqpoint{2.098132in}{2.924795in}}%
\pgfpathlineto{\pgfqpoint{2.030841in}{2.933379in}}%
\pgfpathlineto{\pgfqpoint{1.955367in}{2.940779in}}%
\pgfpathlineto{\pgfqpoint{1.871727in}{2.946758in}}%
\pgfpathlineto{\pgfqpoint{1.788034in}{2.950601in}}%
\pgfpathlineto{\pgfqpoint{1.690811in}{2.953166in}}%
\pgfpathlineto{\pgfqpoint{1.588172in}{2.953661in}}%
\pgfpathlineto{\pgfqpoint{1.482837in}{2.951921in}}%
\pgfpathlineto{\pgfqpoint{1.369428in}{2.947705in}}%
\pgfpathlineto{\pgfqpoint{1.269571in}{2.941536in}}%
\pgfpathlineto{\pgfqpoint{1.156299in}{2.932125in}}%
\pgfpathlineto{\pgfqpoint{1.070105in}{2.922294in}}%
\pgfpathlineto{\pgfqpoint{0.989415in}{2.910956in}}%
\pgfpathlineto{\pgfqpoint{0.914246in}{2.898233in}}%
\pgfpathlineto{\pgfqpoint{0.815147in}{2.878826in}}%
\pgfpathlineto{\pgfqpoint{0.735157in}{2.859180in}}%
\pgfpathlineto{\pgfqpoint{0.679494in}{2.842512in}}%
\pgfpathlineto{\pgfqpoint{0.626779in}{2.824476in}}%
\pgfpathlineto{\pgfqpoint{0.577051in}{2.805102in}}%
\pgfpathlineto{\pgfqpoint{0.530344in}{2.784477in}}%
\pgfpathlineto{\pgfqpoint{0.486699in}{2.762676in}}%
\pgfpathlineto{\pgfqpoint{0.443591in}{2.738518in}}%
\pgfpathlineto{\pgfqpoint{0.406173in}{2.714778in}}%
\pgfpathlineto{\pgfqpoint{0.371858in}{2.690501in}}%
\pgfpathlineto{\pgfqpoint{0.338249in}{2.664018in}}%
\pgfpathlineto{\pgfqpoint{0.307784in}{2.637336in}}%
\pgfpathlineto{\pgfqpoint{0.293000in}{2.622891in}}%
\pgfpathlineto{\pgfqpoint{0.293000in}{2.622891in}}%
\pgfusepath{stroke}%
\end{pgfscope}%
\begin{pgfscope}%
\pgfpathrectangle{\pgfqpoint{0.303000in}{1.712694in}}{\pgfqpoint{2.295902in}{1.362506in}} %
\pgfusepath{clip}%
\pgfsetbuttcap%
\pgfsetroundjoin%
\pgfsetlinewidth{1.003750pt}%
\definecolor{currentstroke}{rgb}{0.000000,0.000000,0.000000}%
\pgfsetstrokecolor{currentstroke}%
\pgfsetdash{{3.700000pt}{1.600000pt}}{0.000000pt}%
\pgfpathmoveto{\pgfqpoint{0.000000in}{0.000000in}}%
\pgfusepath{stroke}%
\end{pgfscope}%
\begin{pgfscope}%
\pgfpathrectangle{\pgfqpoint{0.303000in}{1.712694in}}{\pgfqpoint{2.295902in}{1.362506in}} %
\pgfusepath{clip}%
\pgfsetbuttcap%
\pgfsetroundjoin%
\pgfsetlinewidth{1.003750pt}%
\definecolor{currentstroke}{rgb}{0.000000,0.000000,0.000000}%
\pgfsetstrokecolor{currentstroke}%
\pgfsetdash{{3.700000pt}{1.600000pt}}{0.000000pt}%
\pgfpathmoveto{\pgfqpoint{2.420345in}{1.702694in}}%
\pgfpathlineto{\pgfqpoint{2.437242in}{1.842397in}}%
\pgfpathlineto{\pgfqpoint{2.456042in}{1.979801in}}%
\pgfpathlineto{\pgfqpoint{2.478123in}{2.124551in}}%
\pgfpathlineto{\pgfqpoint{2.525731in}{2.429687in}}%
\pgfpathlineto{\pgfqpoint{2.533444in}{2.494678in}}%
\pgfpathlineto{\pgfqpoint{2.537199in}{2.543879in}}%
\pgfpathlineto{\pgfqpoint{2.538142in}{2.585129in}}%
\pgfpathlineto{\pgfqpoint{2.536853in}{2.618087in}}%
\pgfpathlineto{\pgfqpoint{2.533791in}{2.646593in}}%
\pgfpathlineto{\pgfqpoint{2.528540in}{2.674337in}}%
\pgfpathlineto{\pgfqpoint{2.522131in}{2.697074in}}%
\pgfpathlineto{\pgfqpoint{2.513955in}{2.718446in}}%
\pgfpathlineto{\pgfqpoint{2.504157in}{2.738160in}}%
\pgfpathlineto{\pgfqpoint{2.492973in}{2.756072in}}%
\pgfpathlineto{\pgfqpoint{2.478540in}{2.774721in}}%
\pgfpathlineto{\pgfqpoint{2.462985in}{2.791135in}}%
\pgfpathlineto{\pgfqpoint{2.444233in}{2.807529in}}%
\pgfpathlineto{\pgfqpoint{2.422273in}{2.823447in}}%
\pgfpathlineto{\pgfqpoint{2.397151in}{2.838599in}}%
\pgfpathlineto{\pgfqpoint{2.368937in}{2.852819in}}%
\pgfpathlineto{\pgfqpoint{2.337705in}{2.866053in}}%
\pgfpathlineto{\pgfqpoint{2.300872in}{2.879124in}}%
\pgfpathlineto{\pgfqpoint{2.258440in}{2.891642in}}%
\pgfpathlineto{\pgfqpoint{2.210428in}{2.903332in}}%
\pgfpathlineto{\pgfqpoint{2.156863in}{2.914019in}}%
\pgfpathlineto{\pgfqpoint{2.095082in}{2.923967in}}%
\pgfpathlineto{\pgfqpoint{2.027789in}{2.932538in}}%
\pgfpathlineto{\pgfqpoint{1.952315in}{2.939923in}}%
\pgfpathlineto{\pgfqpoint{1.868674in}{2.945882in}}%
\pgfpathlineto{\pgfqpoint{1.784981in}{2.949726in}}%
\pgfpathlineto{\pgfqpoint{1.687757in}{2.952254in}}%
\pgfpathlineto{\pgfqpoint{1.585118in}{2.952703in}}%
\pgfpathlineto{\pgfqpoint{1.479784in}{2.950902in}}%
\pgfpathlineto{\pgfqpoint{1.369076in}{2.946719in}}%
\pgfpathlineto{\pgfqpoint{1.269220in}{2.940521in}}%
\pgfpathlineto{\pgfqpoint{1.158646in}{2.931272in}}%
\pgfpathlineto{\pgfqpoint{1.072452in}{2.921455in}}%
\pgfpathlineto{\pgfqpoint{0.991760in}{2.910130in}}%
\pgfpathlineto{\pgfqpoint{0.916590in}{2.897417in}}%
\pgfpathlineto{\pgfqpoint{0.820167in}{2.878472in}}%
\pgfpathlineto{\pgfqpoint{0.737502in}{2.858258in}}%
\pgfpathlineto{\pgfqpoint{0.681840in}{2.841586in}}%
\pgfpathlineto{\pgfqpoint{0.629125in}{2.823548in}}%
\pgfpathlineto{\pgfqpoint{0.579397in}{2.804171in}}%
\pgfpathlineto{\pgfqpoint{0.532691in}{2.783544in}}%
\pgfpathlineto{\pgfqpoint{0.491599in}{2.763093in}}%
\pgfpathlineto{\pgfqpoint{0.448448in}{2.739112in}}%
\pgfpathlineto{\pgfqpoint{0.410989in}{2.715522in}}%
\pgfpathlineto{\pgfqpoint{0.376628in}{2.691398in}}%
\pgfpathlineto{\pgfqpoint{0.342964in}{2.665079in}}%
\pgfpathlineto{\pgfqpoint{0.310123in}{2.636432in}}%
\pgfpathlineto{\pgfqpoint{0.293000in}{2.619303in}}%
\pgfpathlineto{\pgfqpoint{0.293000in}{2.619303in}}%
\pgfusepath{stroke}%
\end{pgfscope}%
\begin{pgfscope}%
\pgfpathrectangle{\pgfqpoint{0.303000in}{1.712694in}}{\pgfqpoint{2.295902in}{1.362506in}} %
\pgfusepath{clip}%
\pgfsetbuttcap%
\pgfsetroundjoin%
\pgfsetlinewidth{1.003750pt}%
\definecolor{currentstroke}{rgb}{0.000000,0.000000,0.000000}%
\pgfsetstrokecolor{currentstroke}%
\pgfsetdash{{3.700000pt}{1.600000pt}}{0.000000pt}%
\pgfpathmoveto{\pgfqpoint{0.000000in}{0.000000in}}%
\pgfusepath{stroke}%
\end{pgfscope}%
\begin{pgfscope}%
\pgfsetrectcap%
\pgfsetmiterjoin%
\pgfsetlinewidth{0.803000pt}%
\definecolor{currentstroke}{rgb}{0.000000,0.000000,0.000000}%
\pgfsetstrokecolor{currentstroke}%
\pgfsetdash{}{0pt}%
\pgfpathmoveto{\pgfqpoint{0.303000in}{1.712694in}}%
\pgfpathlineto{\pgfqpoint{0.303000in}{3.075200in}}%
\pgfusepath{stroke}%
\end{pgfscope}%
\begin{pgfscope}%
\pgfsetrectcap%
\pgfsetmiterjoin%
\pgfsetlinewidth{0.803000pt}%
\definecolor{currentstroke}{rgb}{0.000000,0.000000,0.000000}%
\pgfsetstrokecolor{currentstroke}%
\pgfsetdash{}{0pt}%
\pgfpathmoveto{\pgfqpoint{2.598902in}{1.712694in}}%
\pgfpathlineto{\pgfqpoint{2.598902in}{3.075200in}}%
\pgfusepath{stroke}%
\end{pgfscope}%
\begin{pgfscope}%
\pgfsetrectcap%
\pgfsetmiterjoin%
\pgfsetlinewidth{0.803000pt}%
\definecolor{currentstroke}{rgb}{0.000000,0.000000,0.000000}%
\pgfsetstrokecolor{currentstroke}%
\pgfsetdash{}{0pt}%
\pgfpathmoveto{\pgfqpoint{0.303000in}{1.712694in}}%
\pgfpathlineto{\pgfqpoint{2.598902in}{1.712694in}}%
\pgfusepath{stroke}%
\end{pgfscope}%
\begin{pgfscope}%
\pgfsetrectcap%
\pgfsetmiterjoin%
\pgfsetlinewidth{0.803000pt}%
\definecolor{currentstroke}{rgb}{0.000000,0.000000,0.000000}%
\pgfsetstrokecolor{currentstroke}%
\pgfsetdash{}{0pt}%
\pgfpathmoveto{\pgfqpoint{0.303000in}{3.075200in}}%
\pgfpathlineto{\pgfqpoint{2.598902in}{3.075200in}}%
\pgfusepath{stroke}%
\end{pgfscope}%
\begin{pgfscope}%
\pgfsetbuttcap%
\pgfsetmiterjoin%
\definecolor{currentfill}{rgb}{1.000000,1.000000,1.000000}%
\pgfsetfillcolor{currentfill}%
\pgfsetfillopacity{0.800000}%
\pgfsetlinewidth{1.003750pt}%
\definecolor{currentstroke}{rgb}{0.800000,0.800000,0.800000}%
\pgfsetstrokecolor{currentstroke}%
\pgfsetstrokeopacity{0.800000}%
\pgfsetdash{}{0pt}%
\pgfpathmoveto{\pgfqpoint{1.051002in}{1.782138in}}%
\pgfpathlineto{\pgfqpoint{1.850900in}{1.782138in}}%
\pgfpathquadraticcurveto{\pgfqpoint{1.878678in}{1.782138in}}{\pgfqpoint{1.878678in}{1.809916in}}%
\pgfpathlineto{\pgfqpoint{1.878678in}{2.194916in}}%
\pgfpathquadraticcurveto{\pgfqpoint{1.878678in}{2.222694in}}{\pgfqpoint{1.850900in}{2.222694in}}%
\pgfpathlineto{\pgfqpoint{1.051002in}{2.222694in}}%
\pgfpathquadraticcurveto{\pgfqpoint{1.023224in}{2.222694in}}{\pgfqpoint{1.023224in}{2.194916in}}%
\pgfpathlineto{\pgfqpoint{1.023224in}{1.809916in}}%
\pgfpathquadraticcurveto{\pgfqpoint{1.023224in}{1.782138in}}{\pgfqpoint{1.051002in}{1.782138in}}%
\pgfpathclose%
\pgfusepath{stroke,fill}%
\end{pgfscope}%
\begin{pgfscope}%
\pgfsetrectcap%
\pgfsetroundjoin%
\pgfsetlinewidth{1.003750pt}%
\definecolor{currentstroke}{rgb}{0.549020,0.337255,0.294118}%
\pgfsetstrokecolor{currentstroke}%
\pgfsetdash{}{0pt}%
\pgfpathmoveto{\pgfqpoint{1.078780in}{2.118527in}}%
\pgfpathlineto{\pgfqpoint{1.148224in}{2.118527in}}%
\pgfusepath{stroke}%
\end{pgfscope}%
\begin{pgfscope}%
\pgftext[x=1.259335in,y=2.069916in,left,base]{\rmfamily\fontsize{10.000000}{12.000000}\selectfont \(\displaystyle \textnormal{tol}=10^{-5}\)}%
\end{pgfscope}%
\begin{pgfscope}%
\pgfsetbuttcap%
\pgfsetroundjoin%
\pgfsetlinewidth{1.003750pt}%
\definecolor{currentstroke}{rgb}{0.000000,0.000000,0.000000}%
\pgfsetstrokecolor{currentstroke}%
\pgfsetdash{{3.700000pt}{1.600000pt}}{0.000000pt}%
\pgfpathmoveto{\pgfqpoint{1.078780in}{1.919083in}}%
\pgfpathlineto{\pgfqpoint{1.148224in}{1.919083in}}%
\pgfusepath{stroke}%
\end{pgfscope}%
\begin{pgfscope}%
\pgftext[x=1.259335in,y=1.870472in,left,base]{\rmfamily\fontsize{10.000000}{12.000000}\selectfont Reference}%
\end{pgfscope}%
\begin{pgfscope}%
\pgfsetbuttcap%
\pgfsetmiterjoin%
\definecolor{currentfill}{rgb}{1.000000,1.000000,1.000000}%
\pgfsetfillcolor{currentfill}%
\pgfsetlinewidth{0.000000pt}%
\definecolor{currentstroke}{rgb}{0.000000,0.000000,0.000000}%
\pgfsetstrokecolor{currentstroke}%
\pgfsetstrokeopacity{0.000000}%
\pgfsetdash{}{0pt}%
\pgfpathmoveto{\pgfqpoint{2.713698in}{1.712694in}}%
\pgfpathlineto{\pgfqpoint{5.009600in}{1.712694in}}%
\pgfpathlineto{\pgfqpoint{5.009600in}{3.075200in}}%
\pgfpathlineto{\pgfqpoint{2.713698in}{3.075200in}}%
\pgfpathclose%
\pgfusepath{fill}%
\end{pgfscope}%
\begin{pgfscope}%
\pgfsetbuttcap%
\pgfsetroundjoin%
\definecolor{currentfill}{rgb}{0.000000,0.000000,0.000000}%
\pgfsetfillcolor{currentfill}%
\pgfsetlinewidth{0.803000pt}%
\definecolor{currentstroke}{rgb}{0.000000,0.000000,0.000000}%
\pgfsetstrokecolor{currentstroke}%
\pgfsetdash{}{0pt}%
\pgfsys@defobject{currentmarker}{\pgfqpoint{0.000000in}{-0.048611in}}{\pgfqpoint{0.000000in}{0.000000in}}{%
\pgfpathmoveto{\pgfqpoint{0.000000in}{0.000000in}}%
\pgfpathlineto{\pgfqpoint{0.000000in}{-0.048611in}}%
\pgfusepath{stroke,fill}%
}%
\begin{pgfscope}%
\pgfsys@transformshift{2.983804in}{1.712694in}%
\pgfsys@useobject{currentmarker}{}%
\end{pgfscope}%
\end{pgfscope}%
\begin{pgfscope}%
\pgfsetbuttcap%
\pgfsetroundjoin%
\definecolor{currentfill}{rgb}{0.000000,0.000000,0.000000}%
\pgfsetfillcolor{currentfill}%
\pgfsetlinewidth{0.803000pt}%
\definecolor{currentstroke}{rgb}{0.000000,0.000000,0.000000}%
\pgfsetstrokecolor{currentstroke}%
\pgfsetdash{}{0pt}%
\pgfsys@defobject{currentmarker}{\pgfqpoint{0.000000in}{-0.048611in}}{\pgfqpoint{0.000000in}{0.000000in}}{%
\pgfpathmoveto{\pgfqpoint{0.000000in}{0.000000in}}%
\pgfpathlineto{\pgfqpoint{0.000000in}{-0.048611in}}%
\pgfusepath{stroke,fill}%
}%
\begin{pgfscope}%
\pgfsys@transformshift{3.524016in}{1.712694in}%
\pgfsys@useobject{currentmarker}{}%
\end{pgfscope}%
\end{pgfscope}%
\begin{pgfscope}%
\pgfsetbuttcap%
\pgfsetroundjoin%
\definecolor{currentfill}{rgb}{0.000000,0.000000,0.000000}%
\pgfsetfillcolor{currentfill}%
\pgfsetlinewidth{0.803000pt}%
\definecolor{currentstroke}{rgb}{0.000000,0.000000,0.000000}%
\pgfsetstrokecolor{currentstroke}%
\pgfsetdash{}{0pt}%
\pgfsys@defobject{currentmarker}{\pgfqpoint{0.000000in}{-0.048611in}}{\pgfqpoint{0.000000in}{0.000000in}}{%
\pgfpathmoveto{\pgfqpoint{0.000000in}{0.000000in}}%
\pgfpathlineto{\pgfqpoint{0.000000in}{-0.048611in}}%
\pgfusepath{stroke,fill}%
}%
\begin{pgfscope}%
\pgfsys@transformshift{4.064228in}{1.712694in}%
\pgfsys@useobject{currentmarker}{}%
\end{pgfscope}%
\end{pgfscope}%
\begin{pgfscope}%
\pgfsetbuttcap%
\pgfsetroundjoin%
\definecolor{currentfill}{rgb}{0.000000,0.000000,0.000000}%
\pgfsetfillcolor{currentfill}%
\pgfsetlinewidth{0.803000pt}%
\definecolor{currentstroke}{rgb}{0.000000,0.000000,0.000000}%
\pgfsetstrokecolor{currentstroke}%
\pgfsetdash{}{0pt}%
\pgfsys@defobject{currentmarker}{\pgfqpoint{0.000000in}{-0.048611in}}{\pgfqpoint{0.000000in}{0.000000in}}{%
\pgfpathmoveto{\pgfqpoint{0.000000in}{0.000000in}}%
\pgfpathlineto{\pgfqpoint{0.000000in}{-0.048611in}}%
\pgfusepath{stroke,fill}%
}%
\begin{pgfscope}%
\pgfsys@transformshift{4.604441in}{1.712694in}%
\pgfsys@useobject{currentmarker}{}%
\end{pgfscope}%
\end{pgfscope}%
\begin{pgfscope}%
\pgfsetbuttcap%
\pgfsetroundjoin%
\definecolor{currentfill}{rgb}{0.000000,0.000000,0.000000}%
\pgfsetfillcolor{currentfill}%
\pgfsetlinewidth{0.803000pt}%
\definecolor{currentstroke}{rgb}{0.000000,0.000000,0.000000}%
\pgfsetstrokecolor{currentstroke}%
\pgfsetdash{}{0pt}%
\pgfsys@defobject{currentmarker}{\pgfqpoint{-0.048611in}{0.000000in}}{\pgfqpoint{0.000000in}{0.000000in}}{%
\pgfpathmoveto{\pgfqpoint{0.000000in}{0.000000in}}%
\pgfpathlineto{\pgfqpoint{-0.048611in}{0.000000in}}%
\pgfusepath{stroke,fill}%
}%
\begin{pgfscope}%
\pgfsys@transformshift{2.713698in}{1.919134in}%
\pgfsys@useobject{currentmarker}{}%
\end{pgfscope}%
\end{pgfscope}%
\begin{pgfscope}%
\pgfsetbuttcap%
\pgfsetroundjoin%
\definecolor{currentfill}{rgb}{0.000000,0.000000,0.000000}%
\pgfsetfillcolor{currentfill}%
\pgfsetlinewidth{0.803000pt}%
\definecolor{currentstroke}{rgb}{0.000000,0.000000,0.000000}%
\pgfsetstrokecolor{currentstroke}%
\pgfsetdash{}{0pt}%
\pgfsys@defobject{currentmarker}{\pgfqpoint{-0.048611in}{0.000000in}}{\pgfqpoint{0.000000in}{0.000000in}}{%
\pgfpathmoveto{\pgfqpoint{0.000000in}{0.000000in}}%
\pgfpathlineto{\pgfqpoint{-0.048611in}{0.000000in}}%
\pgfusepath{stroke,fill}%
}%
\begin{pgfscope}%
\pgfsys@transformshift{2.713698in}{2.332015in}%
\pgfsys@useobject{currentmarker}{}%
\end{pgfscope}%
\end{pgfscope}%
\begin{pgfscope}%
\pgfsetbuttcap%
\pgfsetroundjoin%
\definecolor{currentfill}{rgb}{0.000000,0.000000,0.000000}%
\pgfsetfillcolor{currentfill}%
\pgfsetlinewidth{0.803000pt}%
\definecolor{currentstroke}{rgb}{0.000000,0.000000,0.000000}%
\pgfsetstrokecolor{currentstroke}%
\pgfsetdash{}{0pt}%
\pgfsys@defobject{currentmarker}{\pgfqpoint{-0.048611in}{0.000000in}}{\pgfqpoint{0.000000in}{0.000000in}}{%
\pgfpathmoveto{\pgfqpoint{0.000000in}{0.000000in}}%
\pgfpathlineto{\pgfqpoint{-0.048611in}{0.000000in}}%
\pgfusepath{stroke,fill}%
}%
\begin{pgfscope}%
\pgfsys@transformshift{2.713698in}{2.744896in}%
\pgfsys@useobject{currentmarker}{}%
\end{pgfscope}%
\end{pgfscope}%
\begin{pgfscope}%
\pgfpathrectangle{\pgfqpoint{2.713698in}{1.712694in}}{\pgfqpoint{2.295902in}{1.362506in}} %
\pgfusepath{clip}%
\pgfsetrectcap%
\pgfsetroundjoin%
\pgfsetlinewidth{1.505625pt}%
\definecolor{currentstroke}{rgb}{0.121569,0.466667,0.705882}%
\pgfsetstrokecolor{currentstroke}%
\pgfsetdash{}{0pt}%
\pgfusepath{stroke}%
\end{pgfscope}%
\begin{pgfscope}%
\pgfpathrectangle{\pgfqpoint{2.713698in}{1.712694in}}{\pgfqpoint{2.295902in}{1.362506in}} %
\pgfusepath{clip}%
\pgfsetrectcap%
\pgfsetroundjoin%
\pgfsetlinewidth{1.505625pt}%
\definecolor{currentstroke}{rgb}{1.000000,0.498039,0.054902}%
\pgfsetstrokecolor{currentstroke}%
\pgfsetdash{}{0pt}%
\pgfusepath{stroke}%
\end{pgfscope}%
\begin{pgfscope}%
\pgfpathrectangle{\pgfqpoint{2.713698in}{1.712694in}}{\pgfqpoint{2.295902in}{1.362506in}} %
\pgfusepath{clip}%
\pgfsetrectcap%
\pgfsetroundjoin%
\pgfsetlinewidth{1.505625pt}%
\definecolor{currentstroke}{rgb}{0.172549,0.627451,0.172549}%
\pgfsetstrokecolor{currentstroke}%
\pgfsetdash{}{0pt}%
\pgfusepath{stroke}%
\end{pgfscope}%
\begin{pgfscope}%
\pgfpathrectangle{\pgfqpoint{2.713698in}{1.712694in}}{\pgfqpoint{2.295902in}{1.362506in}} %
\pgfusepath{clip}%
\pgfsetrectcap%
\pgfsetroundjoin%
\pgfsetlinewidth{1.505625pt}%
\definecolor{currentstroke}{rgb}{0.839216,0.152941,0.156863}%
\pgfsetstrokecolor{currentstroke}%
\pgfsetdash{}{0pt}%
\pgfusepath{stroke}%
\end{pgfscope}%
\begin{pgfscope}%
\pgfpathrectangle{\pgfqpoint{2.713698in}{1.712694in}}{\pgfqpoint{2.295902in}{1.362506in}} %
\pgfusepath{clip}%
\pgfsetrectcap%
\pgfsetroundjoin%
\pgfsetlinewidth{1.003750pt}%
\definecolor{currentstroke}{rgb}{0.580392,0.403922,0.741176}%
\pgfsetstrokecolor{currentstroke}%
\pgfsetdash{}{0pt}%
\pgfpathmoveto{\pgfqpoint{4.687175in}{1.702694in}}%
\pgfpathlineto{\pgfqpoint{4.690620in}{1.864265in}}%
\pgfpathlineto{\pgfqpoint{4.691068in}{1.983993in}}%
\pgfpathlineto{\pgfqpoint{4.689236in}{2.078908in}}%
\pgfpathlineto{\pgfqpoint{4.685737in}{2.153028in}}%
\pgfpathlineto{\pgfqpoint{4.680466in}{2.218587in}}%
\pgfpathlineto{\pgfqpoint{4.673620in}{2.275425in}}%
\pgfpathlineto{\pgfqpoint{4.665638in}{2.323435in}}%
\pgfpathlineto{\pgfqpoint{4.656247in}{2.366512in}}%
\pgfpathlineto{\pgfqpoint{4.645729in}{2.404529in}}%
\pgfpathlineto{\pgfqpoint{4.634500in}{2.437475in}}%
\pgfpathlineto{\pgfqpoint{4.621533in}{2.468894in}}%
\pgfpathlineto{\pgfqpoint{4.606843in}{2.498487in}}%
\pgfpathlineto{\pgfqpoint{4.590529in}{2.526020in}}%
\pgfpathlineto{\pgfqpoint{4.572755in}{2.551355in}}%
\pgfpathlineto{\pgfqpoint{4.553724in}{2.574461in}}%
\pgfpathlineto{\pgfqpoint{4.533647in}{2.595399in}}%
\pgfpathlineto{\pgfqpoint{4.510352in}{2.616281in}}%
\pgfpathlineto{\pgfqpoint{4.486239in}{2.634872in}}%
\pgfpathlineto{\pgfqpoint{4.458991in}{2.652968in}}%
\pgfpathlineto{\pgfqpoint{4.428618in}{2.670249in}}%
\pgfpathlineto{\pgfqpoint{4.397742in}{2.685313in}}%
\pgfpathlineto{\pgfqpoint{4.363878in}{2.699489in}}%
\pgfpathlineto{\pgfqpoint{4.324413in}{2.713485in}}%
\pgfpathlineto{\pgfqpoint{4.281987in}{2.726054in}}%
\pgfpathlineto{\pgfqpoint{4.236648in}{2.737149in}}%
\pgfpathlineto{\pgfqpoint{4.185754in}{2.747212in}}%
\pgfpathlineto{\pgfqpoint{4.132009in}{2.755522in}}%
\pgfpathlineto{\pgfqpoint{4.075450in}{2.762067in}}%
\pgfpathlineto{\pgfqpoint{4.013410in}{2.766978in}}%
\pgfpathlineto{\pgfqpoint{3.948613in}{2.769839in}}%
\pgfpathlineto{\pgfqpoint{3.881089in}{2.770554in}}%
\pgfpathlineto{\pgfqpoint{3.813571in}{2.769080in}}%
\pgfpathlineto{\pgfqpoint{3.743388in}{2.765280in}}%
\pgfpathlineto{\pgfqpoint{3.675972in}{2.759427in}}%
\pgfpathlineto{\pgfqpoint{3.608654in}{2.751346in}}%
\pgfpathlineto{\pgfqpoint{3.544159in}{2.741370in}}%
\pgfpathlineto{\pgfqpoint{3.482514in}{2.729613in}}%
\pgfpathlineto{\pgfqpoint{3.423747in}{2.716171in}}%
\pgfpathlineto{\pgfqpoint{3.367885in}{2.701136in}}%
\pgfpathlineto{\pgfqpoint{3.314960in}{2.684604in}}%
\pgfpathlineto{\pgfqpoint{3.264999in}{2.666681in}}%
\pgfpathlineto{\pgfqpoint{3.218030in}{2.647492in}}%
\pgfpathlineto{\pgfqpoint{3.174078in}{2.627187in}}%
\pgfpathlineto{\pgfqpoint{3.133157in}{2.605949in}}%
\pgfpathlineto{\pgfqpoint{3.092770in}{2.582450in}}%
\pgfpathlineto{\pgfqpoint{3.055495in}{2.558189in}}%
\pgfpathlineto{\pgfqpoint{3.021323in}{2.533444in}}%
\pgfpathlineto{\pgfqpoint{2.987859in}{2.506534in}}%
\pgfpathlineto{\pgfqpoint{2.955228in}{2.477334in}}%
\pgfpathlineto{\pgfqpoint{2.925794in}{2.448076in}}%
\pgfpathlineto{\pgfqpoint{2.897314in}{2.416688in}}%
\pgfpathlineto{\pgfqpoint{2.869913in}{2.383132in}}%
\pgfpathlineto{\pgfqpoint{2.843714in}{2.347405in}}%
\pgfpathlineto{\pgfqpoint{2.818835in}{2.309538in}}%
\pgfpathlineto{\pgfqpoint{2.795381in}{2.269601in}}%
\pgfpathlineto{\pgfqpoint{2.773444in}{2.227700in}}%
\pgfpathlineto{\pgfqpoint{2.753092in}{2.183969in}}%
\pgfpathlineto{\pgfqpoint{2.734373in}{2.138567in}}%
\pgfpathlineto{\pgfqpoint{2.717308in}{2.091664in}}%
\pgfpathlineto{\pgfqpoint{2.703698in}{2.049395in}}%
\pgfpathmoveto{\pgfqpoint{2.703698in}{2.373552in}}%
\pgfpathlineto{\pgfqpoint{2.727581in}{2.411544in}}%
\pgfpathlineto{\pgfqpoint{2.753245in}{2.448167in}}%
\pgfpathlineto{\pgfqpoint{2.780155in}{2.482639in}}%
\pgfpathlineto{\pgfqpoint{2.808195in}{2.514938in}}%
\pgfpathlineto{\pgfqpoint{2.837245in}{2.545081in}}%
\pgfpathlineto{\pgfqpoint{2.867184in}{2.573118in}}%
\pgfpathlineto{\pgfqpoint{2.900289in}{2.601044in}}%
\pgfpathlineto{\pgfqpoint{2.934160in}{2.626736in}}%
\pgfpathlineto{\pgfqpoint{2.971166in}{2.651940in}}%
\pgfpathlineto{\pgfqpoint{3.008788in}{2.674915in}}%
\pgfpathlineto{\pgfqpoint{3.049478in}{2.697166in}}%
\pgfpathlineto{\pgfqpoint{3.093224in}{2.718487in}}%
\pgfpathlineto{\pgfqpoint{3.140006in}{2.738711in}}%
\pgfpathlineto{\pgfqpoint{3.189796in}{2.757713in}}%
\pgfpathlineto{\pgfqpoint{3.242561in}{2.775400in}}%
\pgfpathlineto{\pgfqpoint{3.298270in}{2.791704in}}%
\pgfpathlineto{\pgfqpoint{3.356890in}{2.806582in}}%
\pgfpathlineto{\pgfqpoint{3.421067in}{2.820541in}}%
\pgfpathlineto{\pgfqpoint{3.488111in}{2.832850in}}%
\pgfpathlineto{\pgfqpoint{3.560681in}{2.843871in}}%
\pgfpathlineto{\pgfqpoint{3.636070in}{2.853066in}}%
\pgfpathlineto{\pgfqpoint{3.714252in}{2.860409in}}%
\pgfpathlineto{\pgfqpoint{3.795205in}{2.865839in}}%
\pgfpathlineto{\pgfqpoint{3.878907in}{2.869242in}}%
\pgfpathlineto{\pgfqpoint{3.962636in}{2.870434in}}%
\pgfpathlineto{\pgfqpoint{4.043664in}{2.869430in}}%
\pgfpathlineto{\pgfqpoint{4.121967in}{2.866310in}}%
\pgfpathlineto{\pgfqpoint{4.197518in}{2.861074in}}%
\pgfpathlineto{\pgfqpoint{4.267590in}{2.853967in}}%
\pgfpathlineto{\pgfqpoint{4.332156in}{2.845140in}}%
\pgfpathlineto{\pgfqpoint{4.388508in}{2.835264in}}%
\pgfpathlineto{\pgfqpoint{4.439315in}{2.824218in}}%
\pgfpathlineto{\pgfqpoint{4.484559in}{2.812254in}}%
\pgfpathlineto{\pgfqpoint{4.526864in}{2.798777in}}%
\pgfpathlineto{\pgfqpoint{4.563556in}{2.784808in}}%
\pgfpathlineto{\pgfqpoint{4.597225in}{2.769590in}}%
\pgfpathlineto{\pgfqpoint{4.625280in}{2.754648in}}%
\pgfpathlineto{\pgfqpoint{4.650294in}{2.739082in}}%
\pgfpathlineto{\pgfqpoint{4.674655in}{2.721266in}}%
\pgfpathlineto{\pgfqpoint{4.695810in}{2.702976in}}%
\pgfpathlineto{\pgfqpoint{4.713788in}{2.684664in}}%
\pgfpathlineto{\pgfqpoint{4.730748in}{2.664213in}}%
\pgfpathlineto{\pgfqpoint{4.746419in}{2.641490in}}%
\pgfpathlineto{\pgfqpoint{4.758850in}{2.619726in}}%
\pgfpathlineto{\pgfqpoint{4.769909in}{2.596297in}}%
\pgfpathlineto{\pgfqpoint{4.780724in}{2.567719in}}%
\pgfpathlineto{\pgfqpoint{4.789551in}{2.537585in}}%
\pgfpathlineto{\pgfqpoint{4.796475in}{2.506308in}}%
\pgfpathlineto{\pgfqpoint{4.802219in}{2.470213in}}%
\pgfpathlineto{\pgfqpoint{4.806450in}{2.429445in}}%
\pgfpathlineto{\pgfqpoint{4.809157in}{2.380082in}}%
\pgfpathlineto{\pgfqpoint{4.809893in}{2.322298in}}%
\pgfpathlineto{\pgfqpoint{4.808380in}{2.252153in}}%
\pgfpathlineto{\pgfqpoint{4.803992in}{2.161572in}}%
\pgfpathlineto{\pgfqpoint{4.795104in}{2.030153in}}%
\pgfpathlineto{\pgfqpoint{4.770069in}{1.702694in}}%
\pgfpathlineto{\pgfqpoint{4.770069in}{1.702694in}}%
\pgfusepath{stroke}%
\end{pgfscope}%
\begin{pgfscope}%
\pgfpathrectangle{\pgfqpoint{2.713698in}{1.712694in}}{\pgfqpoint{2.295902in}{1.362506in}} %
\pgfusepath{clip}%
\pgfsetrectcap%
\pgfsetroundjoin%
\pgfsetlinewidth{1.003750pt}%
\definecolor{currentstroke}{rgb}{0.580392,0.403922,0.741176}%
\pgfsetstrokecolor{currentstroke}%
\pgfsetdash{}{0pt}%
\pgfpathmoveto{\pgfqpoint{4.831804in}{1.702694in}}%
\pgfpathlineto{\pgfqpoint{4.848791in}{1.842440in}}%
\pgfpathlineto{\pgfqpoint{4.867698in}{1.979809in}}%
\pgfpathlineto{\pgfqpoint{4.889288in}{2.120495in}}%
\pgfpathlineto{\pgfqpoint{4.937211in}{2.425457in}}%
\pgfpathlineto{\pgfqpoint{4.945668in}{2.494437in}}%
\pgfpathlineto{\pgfqpoint{4.949597in}{2.543607in}}%
\pgfpathlineto{\pgfqpoint{4.950731in}{2.584845in}}%
\pgfpathlineto{\pgfqpoint{4.949623in}{2.617819in}}%
\pgfpathlineto{\pgfqpoint{4.946723in}{2.646365in}}%
\pgfpathlineto{\pgfqpoint{4.941646in}{2.674185in}}%
\pgfpathlineto{\pgfqpoint{4.935381in}{2.697015in}}%
\pgfpathlineto{\pgfqpoint{4.927332in}{2.718499in}}%
\pgfpathlineto{\pgfqpoint{4.917636in}{2.738330in}}%
\pgfpathlineto{\pgfqpoint{4.906532in}{2.756358in}}%
\pgfpathlineto{\pgfqpoint{4.892164in}{2.775123in}}%
\pgfpathlineto{\pgfqpoint{4.876650in}{2.791626in}}%
\pgfpathlineto{\pgfqpoint{4.857925in}{2.808093in}}%
\pgfpathlineto{\pgfqpoint{4.835986in}{2.824078in}}%
\pgfpathlineto{\pgfqpoint{4.810879in}{2.839287in}}%
\pgfpathlineto{\pgfqpoint{4.782672in}{2.853540in}}%
\pgfpathlineto{\pgfqpoint{4.751445in}{2.866803in}}%
\pgfpathlineto{\pgfqpoint{4.714616in}{2.879898in}}%
\pgfpathlineto{\pgfqpoint{4.672186in}{2.892434in}}%
\pgfpathlineto{\pgfqpoint{4.624176in}{2.904139in}}%
\pgfpathlineto{\pgfqpoint{4.570612in}{2.914838in}}%
\pgfpathlineto{\pgfqpoint{4.508831in}{2.924797in}}%
\pgfpathlineto{\pgfqpoint{4.441540in}{2.933381in}}%
\pgfpathlineto{\pgfqpoint{4.366066in}{2.940782in}}%
\pgfpathlineto{\pgfqpoint{4.282426in}{2.946760in}}%
\pgfpathlineto{\pgfqpoint{4.198733in}{2.950603in}}%
\pgfpathlineto{\pgfqpoint{4.101510in}{2.953167in}}%
\pgfpathlineto{\pgfqpoint{3.998871in}{2.953663in}}%
\pgfpathlineto{\pgfqpoint{3.893536in}{2.951923in}}%
\pgfpathlineto{\pgfqpoint{3.780127in}{2.947707in}}%
\pgfpathlineto{\pgfqpoint{3.680270in}{2.941538in}}%
\pgfpathlineto{\pgfqpoint{3.566998in}{2.932128in}}%
\pgfpathlineto{\pgfqpoint{3.480804in}{2.922296in}}%
\pgfpathlineto{\pgfqpoint{3.400114in}{2.910959in}}%
\pgfpathlineto{\pgfqpoint{3.324944in}{2.898236in}}%
\pgfpathlineto{\pgfqpoint{3.225846in}{2.878829in}}%
\pgfpathlineto{\pgfqpoint{3.145856in}{2.859180in}}%
\pgfpathlineto{\pgfqpoint{3.090193in}{2.842512in}}%
\pgfpathlineto{\pgfqpoint{3.037478in}{2.824476in}}%
\pgfpathlineto{\pgfqpoint{2.987750in}{2.805102in}}%
\pgfpathlineto{\pgfqpoint{2.941043in}{2.784477in}}%
\pgfpathlineto{\pgfqpoint{2.897398in}{2.762676in}}%
\pgfpathlineto{\pgfqpoint{2.854290in}{2.738518in}}%
\pgfpathlineto{\pgfqpoint{2.816872in}{2.714778in}}%
\pgfpathlineto{\pgfqpoint{2.782557in}{2.690501in}}%
\pgfpathlineto{\pgfqpoint{2.748948in}{2.664018in}}%
\pgfpathlineto{\pgfqpoint{2.718483in}{2.637336in}}%
\pgfpathlineto{\pgfqpoint{2.703698in}{2.622890in}}%
\pgfpathlineto{\pgfqpoint{2.703698in}{2.622890in}}%
\pgfusepath{stroke}%
\end{pgfscope}%
\begin{pgfscope}%
\pgfpathrectangle{\pgfqpoint{2.713698in}{1.712694in}}{\pgfqpoint{2.295902in}{1.362506in}} %
\pgfusepath{clip}%
\pgfsetrectcap%
\pgfsetroundjoin%
\pgfsetlinewidth{1.003750pt}%
\definecolor{currentstroke}{rgb}{0.580392,0.403922,0.741176}%
\pgfsetstrokecolor{currentstroke}%
\pgfsetdash{}{0pt}%
\pgfpathmoveto{\pgfqpoint{4.831044in}{1.702694in}}%
\pgfpathlineto{\pgfqpoint{4.847942in}{1.842407in}}%
\pgfpathlineto{\pgfqpoint{4.866742in}{1.979810in}}%
\pgfpathlineto{\pgfqpoint{4.888824in}{2.124560in}}%
\pgfpathlineto{\pgfqpoint{4.936433in}{2.429695in}}%
\pgfpathlineto{\pgfqpoint{4.944146in}{2.494687in}}%
\pgfpathlineto{\pgfqpoint{4.947901in}{2.543888in}}%
\pgfpathlineto{\pgfqpoint{4.948844in}{2.585137in}}%
\pgfpathlineto{\pgfqpoint{4.947554in}{2.618096in}}%
\pgfpathlineto{\pgfqpoint{4.944491in}{2.646601in}}%
\pgfpathlineto{\pgfqpoint{4.939240in}{2.674346in}}%
\pgfpathlineto{\pgfqpoint{4.932831in}{2.697083in}}%
\pgfpathlineto{\pgfqpoint{4.924655in}{2.718454in}}%
\pgfpathlineto{\pgfqpoint{4.914856in}{2.738167in}}%
\pgfpathlineto{\pgfqpoint{4.903672in}{2.756079in}}%
\pgfpathlineto{\pgfqpoint{4.889238in}{2.774728in}}%
\pgfpathlineto{\pgfqpoint{4.873683in}{2.791141in}}%
\pgfpathlineto{\pgfqpoint{4.854931in}{2.807534in}}%
\pgfpathlineto{\pgfqpoint{4.832970in}{2.823452in}}%
\pgfpathlineto{\pgfqpoint{4.807849in}{2.838604in}}%
\pgfpathlineto{\pgfqpoint{4.779635in}{2.852824in}}%
\pgfpathlineto{\pgfqpoint{4.748402in}{2.866057in}}%
\pgfpathlineto{\pgfqpoint{4.711569in}{2.879127in}}%
\pgfpathlineto{\pgfqpoint{4.669138in}{2.891645in}}%
\pgfpathlineto{\pgfqpoint{4.621126in}{2.903335in}}%
\pgfpathlineto{\pgfqpoint{4.567560in}{2.914022in}}%
\pgfpathlineto{\pgfqpoint{4.505779in}{2.923970in}}%
\pgfpathlineto{\pgfqpoint{4.438487in}{2.932541in}}%
\pgfpathlineto{\pgfqpoint{4.363013in}{2.939925in}}%
\pgfpathlineto{\pgfqpoint{4.279371in}{2.945884in}}%
\pgfpathlineto{\pgfqpoint{4.195678in}{2.949728in}}%
\pgfpathlineto{\pgfqpoint{4.098455in}{2.952256in}}%
\pgfpathlineto{\pgfqpoint{3.995816in}{2.952705in}}%
\pgfpathlineto{\pgfqpoint{3.890482in}{2.950904in}}%
\pgfpathlineto{\pgfqpoint{3.779774in}{2.946721in}}%
\pgfpathlineto{\pgfqpoint{3.679918in}{2.940523in}}%
\pgfpathlineto{\pgfqpoint{3.569343in}{2.931274in}}%
\pgfpathlineto{\pgfqpoint{3.483149in}{2.921458in}}%
\pgfpathlineto{\pgfqpoint{3.402458in}{2.910133in}}%
\pgfpathlineto{\pgfqpoint{3.327288in}{2.897419in}}%
\pgfpathlineto{\pgfqpoint{3.230865in}{2.878475in}}%
\pgfpathlineto{\pgfqpoint{3.148200in}{2.858258in}}%
\pgfpathlineto{\pgfqpoint{3.092538in}{2.841586in}}%
\pgfpathlineto{\pgfqpoint{3.039823in}{2.823548in}}%
\pgfpathlineto{\pgfqpoint{2.990095in}{2.804171in}}%
\pgfpathlineto{\pgfqpoint{2.943388in}{2.783544in}}%
\pgfpathlineto{\pgfqpoint{2.902296in}{2.763093in}}%
\pgfpathlineto{\pgfqpoint{2.859145in}{2.739113in}}%
\pgfpathlineto{\pgfqpoint{2.821687in}{2.715522in}}%
\pgfpathlineto{\pgfqpoint{2.787325in}{2.691398in}}%
\pgfpathlineto{\pgfqpoint{2.753661in}{2.665079in}}%
\pgfpathlineto{\pgfqpoint{2.720821in}{2.636432in}}%
\pgfpathlineto{\pgfqpoint{2.703698in}{2.619304in}}%
\pgfpathlineto{\pgfqpoint{2.703698in}{2.619304in}}%
\pgfusepath{stroke}%
\end{pgfscope}%
\begin{pgfscope}%
\pgfpathrectangle{\pgfqpoint{2.713698in}{1.712694in}}{\pgfqpoint{2.295902in}{1.362506in}} %
\pgfusepath{clip}%
\pgfsetbuttcap%
\pgfsetroundjoin%
\pgfsetlinewidth{1.003750pt}%
\definecolor{currentstroke}{rgb}{0.000000,0.000000,0.000000}%
\pgfsetstrokecolor{currentstroke}%
\pgfsetdash{{3.700000pt}{1.600000pt}}{0.000000pt}%
\pgfpathmoveto{\pgfqpoint{2.746089in}{2.627517in}}%
\pgfpathlineto{\pgfqpoint{2.778981in}{2.656023in}}%
\pgfpathlineto{\pgfqpoint{2.812687in}{2.682218in}}%
\pgfpathlineto{\pgfqpoint{2.847079in}{2.706239in}}%
\pgfpathlineto{\pgfqpoint{2.884563in}{2.729735in}}%
\pgfpathlineto{\pgfqpoint{2.925142in}{2.752451in}}%
\pgfpathlineto{\pgfqpoint{2.968802in}{2.774181in}}%
\pgfpathlineto{\pgfqpoint{3.012909in}{2.793685in}}%
\pgfpathlineto{\pgfqpoint{3.059998in}{2.812174in}}%
\pgfpathlineto{\pgfqpoint{3.112684in}{2.830408in}}%
\pgfpathlineto{\pgfqpoint{3.168324in}{2.847257in}}%
\pgfpathlineto{\pgfqpoint{3.226880in}{2.862707in}}%
\pgfpathlineto{\pgfqpoint{3.290994in}{2.877329in}}%
\pgfpathlineto{\pgfqpoint{3.360658in}{2.890894in}}%
\pgfpathlineto{\pgfqpoint{3.435855in}{2.903217in}}%
\pgfpathlineto{\pgfqpoint{3.516570in}{2.914145in}}%
\pgfpathlineto{\pgfqpoint{3.602784in}{2.923546in}}%
\pgfpathlineto{\pgfqpoint{3.694479in}{2.931290in}}%
\pgfpathlineto{\pgfqpoint{3.791639in}{2.937230in}}%
\pgfpathlineto{\pgfqpoint{3.891545in}{2.941097in}}%
\pgfpathlineto{\pgfqpoint{3.994179in}{2.942809in}}%
\pgfpathlineto{\pgfqpoint{4.094116in}{2.942262in}}%
\pgfpathlineto{\pgfqpoint{4.191337in}{2.939496in}}%
\pgfpathlineto{\pgfqpoint{4.283117in}{2.934633in}}%
\pgfpathlineto{\pgfqpoint{4.366735in}{2.927965in}}%
\pgfpathlineto{\pgfqpoint{4.442171in}{2.919724in}}%
\pgfpathlineto{\pgfqpoint{4.509405in}{2.910152in}}%
\pgfpathlineto{\pgfqpoint{4.568420in}{2.899546in}}%
\pgfpathlineto{\pgfqpoint{4.619207in}{2.888287in}}%
\pgfpathlineto{\pgfqpoint{4.664426in}{2.876104in}}%
\pgfpathlineto{\pgfqpoint{4.704055in}{2.863230in}}%
\pgfpathlineto{\pgfqpoint{4.738084in}{2.850011in}}%
\pgfpathlineto{\pgfqpoint{4.769099in}{2.835635in}}%
\pgfpathlineto{\pgfqpoint{4.797019in}{2.820121in}}%
\pgfpathlineto{\pgfqpoint{4.819316in}{2.805332in}}%
\pgfpathlineto{\pgfqpoint{4.838530in}{2.790232in}}%
\pgfpathlineto{\pgfqpoint{4.856923in}{2.772917in}}%
\pgfpathlineto{\pgfqpoint{4.872074in}{2.755643in}}%
\pgfpathlineto{\pgfqpoint{4.886018in}{2.736146in}}%
\pgfpathlineto{\pgfqpoint{4.896729in}{2.717569in}}%
\pgfpathlineto{\pgfqpoint{4.906042in}{2.697312in}}%
\pgfpathlineto{\pgfqpoint{4.913773in}{2.675555in}}%
\pgfpathlineto{\pgfqpoint{4.920687in}{2.648676in}}%
\pgfpathlineto{\pgfqpoint{4.925408in}{2.620707in}}%
\pgfpathlineto{\pgfqpoint{4.928439in}{2.588019in}}%
\pgfpathlineto{\pgfqpoint{4.929407in}{2.550903in}}%
\pgfpathlineto{\pgfqpoint{4.928218in}{2.509664in}}%
\pgfpathlineto{\pgfqpoint{4.924607in}{2.460434in}}%
\pgfpathlineto{\pgfqpoint{4.917469in}{2.395287in}}%
\pgfpathlineto{\pgfqpoint{4.904797in}{2.302325in}}%
\pgfpathlineto{\pgfqpoint{4.845589in}{1.886824in}}%
\pgfpathlineto{\pgfqpoint{4.827683in}{1.736533in}}%
\pgfpathlineto{\pgfqpoint{4.823961in}{1.702694in}}%
\pgfpathlineto{\pgfqpoint{4.823961in}{1.702694in}}%
\pgfusepath{stroke}%
\end{pgfscope}%
\begin{pgfscope}%
\pgfpathrectangle{\pgfqpoint{2.713698in}{1.712694in}}{\pgfqpoint{2.295902in}{1.362506in}} %
\pgfusepath{clip}%
\pgfsetbuttcap%
\pgfsetroundjoin%
\pgfsetlinewidth{1.003750pt}%
\definecolor{currentstroke}{rgb}{0.000000,0.000000,0.000000}%
\pgfsetstrokecolor{currentstroke}%
\pgfsetdash{{3.700000pt}{1.600000pt}}{0.000000pt}%
\pgfpathmoveto{\pgfqpoint{0.000000in}{0.000000in}}%
\pgfusepath{stroke}%
\end{pgfscope}%
\begin{pgfscope}%
\pgfpathrectangle{\pgfqpoint{2.713698in}{1.712694in}}{\pgfqpoint{2.295902in}{1.362506in}} %
\pgfusepath{clip}%
\pgfsetbuttcap%
\pgfsetroundjoin%
\pgfsetlinewidth{1.003750pt}%
\definecolor{currentstroke}{rgb}{0.000000,0.000000,0.000000}%
\pgfsetstrokecolor{currentstroke}%
\pgfsetdash{{3.700000pt}{1.600000pt}}{0.000000pt}%
\pgfpathmoveto{\pgfqpoint{0.000000in}{0.000000in}}%
\pgfusepath{stroke}%
\end{pgfscope}%
\begin{pgfscope}%
\pgfpathrectangle{\pgfqpoint{2.713698in}{1.712694in}}{\pgfqpoint{2.295902in}{1.362506in}} %
\pgfusepath{clip}%
\pgfsetbuttcap%
\pgfsetroundjoin%
\pgfsetlinewidth{1.003750pt}%
\definecolor{currentstroke}{rgb}{0.000000,0.000000,0.000000}%
\pgfsetstrokecolor{currentstroke}%
\pgfsetdash{{3.700000pt}{1.600000pt}}{0.000000pt}%
\pgfpathmoveto{\pgfqpoint{4.687175in}{1.702694in}}%
\pgfpathlineto{\pgfqpoint{4.690620in}{1.864272in}}%
\pgfpathlineto{\pgfqpoint{4.691068in}{1.984000in}}%
\pgfpathlineto{\pgfqpoint{4.689236in}{2.078915in}}%
\pgfpathlineto{\pgfqpoint{4.685737in}{2.153035in}}%
\pgfpathlineto{\pgfqpoint{4.680465in}{2.218594in}}%
\pgfpathlineto{\pgfqpoint{4.673619in}{2.275432in}}%
\pgfpathlineto{\pgfqpoint{4.665637in}{2.323442in}}%
\pgfpathlineto{\pgfqpoint{4.656245in}{2.366518in}}%
\pgfpathlineto{\pgfqpoint{4.645727in}{2.404535in}}%
\pgfpathlineto{\pgfqpoint{4.634498in}{2.437481in}}%
\pgfpathlineto{\pgfqpoint{4.621531in}{2.468899in}}%
\pgfpathlineto{\pgfqpoint{4.606841in}{2.498492in}}%
\pgfpathlineto{\pgfqpoint{4.590526in}{2.526025in}}%
\pgfpathlineto{\pgfqpoint{4.572752in}{2.551359in}}%
\pgfpathlineto{\pgfqpoint{4.553721in}{2.574465in}}%
\pgfpathlineto{\pgfqpoint{4.533644in}{2.595403in}}%
\pgfpathlineto{\pgfqpoint{4.510348in}{2.616284in}}%
\pgfpathlineto{\pgfqpoint{4.486235in}{2.634875in}}%
\pgfpathlineto{\pgfqpoint{4.458987in}{2.652971in}}%
\pgfpathlineto{\pgfqpoint{4.428614in}{2.670251in}}%
\pgfpathlineto{\pgfqpoint{4.397738in}{2.685316in}}%
\pgfpathlineto{\pgfqpoint{4.363874in}{2.699491in}}%
\pgfpathlineto{\pgfqpoint{4.324408in}{2.713487in}}%
\pgfpathlineto{\pgfqpoint{4.281983in}{2.726055in}}%
\pgfpathlineto{\pgfqpoint{4.236644in}{2.737150in}}%
\pgfpathlineto{\pgfqpoint{4.185750in}{2.747212in}}%
\pgfpathlineto{\pgfqpoint{4.132004in}{2.755523in}}%
\pgfpathlineto{\pgfqpoint{4.075445in}{2.762068in}}%
\pgfpathlineto{\pgfqpoint{4.013405in}{2.766979in}}%
\pgfpathlineto{\pgfqpoint{3.948608in}{2.769839in}}%
\pgfpathlineto{\pgfqpoint{3.881085in}{2.770555in}}%
\pgfpathlineto{\pgfqpoint{3.813566in}{2.769080in}}%
\pgfpathlineto{\pgfqpoint{3.743384in}{2.765280in}}%
\pgfpathlineto{\pgfqpoint{3.675967in}{2.759427in}}%
\pgfpathlineto{\pgfqpoint{3.608649in}{2.751346in}}%
\pgfpathlineto{\pgfqpoint{3.544155in}{2.741370in}}%
\pgfpathlineto{\pgfqpoint{3.482510in}{2.729612in}}%
\pgfpathlineto{\pgfqpoint{3.423742in}{2.716170in}}%
\pgfpathlineto{\pgfqpoint{3.367881in}{2.701135in}}%
\pgfpathlineto{\pgfqpoint{3.314955in}{2.684602in}}%
\pgfpathlineto{\pgfqpoint{3.264994in}{2.666679in}}%
\pgfpathlineto{\pgfqpoint{3.218026in}{2.647490in}}%
\pgfpathlineto{\pgfqpoint{3.174074in}{2.627185in}}%
\pgfpathlineto{\pgfqpoint{3.133153in}{2.605947in}}%
\pgfpathlineto{\pgfqpoint{3.092766in}{2.582448in}}%
\pgfpathlineto{\pgfqpoint{3.055491in}{2.558187in}}%
\pgfpathlineto{\pgfqpoint{3.021319in}{2.533442in}}%
\pgfpathlineto{\pgfqpoint{2.987855in}{2.506531in}}%
\pgfpathlineto{\pgfqpoint{2.955225in}{2.477331in}}%
\pgfpathlineto{\pgfqpoint{2.925790in}{2.448073in}}%
\pgfpathlineto{\pgfqpoint{2.897311in}{2.416684in}}%
\pgfpathlineto{\pgfqpoint{2.869910in}{2.383128in}}%
\pgfpathlineto{\pgfqpoint{2.843711in}{2.347400in}}%
\pgfpathlineto{\pgfqpoint{2.818832in}{2.309533in}}%
\pgfpathlineto{\pgfqpoint{2.795379in}{2.269597in}}%
\pgfpathlineto{\pgfqpoint{2.773441in}{2.227695in}}%
\pgfpathlineto{\pgfqpoint{2.753090in}{2.183964in}}%
\pgfpathlineto{\pgfqpoint{2.734370in}{2.138561in}}%
\pgfpathlineto{\pgfqpoint{2.717306in}{2.091658in}}%
\pgfpathlineto{\pgfqpoint{2.703698in}{2.049396in}}%
\pgfpathmoveto{\pgfqpoint{2.703698in}{2.373551in}}%
\pgfpathlineto{\pgfqpoint{2.727582in}{2.411544in}}%
\pgfpathlineto{\pgfqpoint{2.753246in}{2.448168in}}%
\pgfpathlineto{\pgfqpoint{2.780155in}{2.482639in}}%
\pgfpathlineto{\pgfqpoint{2.808195in}{2.514938in}}%
\pgfpathlineto{\pgfqpoint{2.837245in}{2.545081in}}%
\pgfpathlineto{\pgfqpoint{2.867184in}{2.573118in}}%
\pgfpathlineto{\pgfqpoint{2.900289in}{2.601044in}}%
\pgfpathlineto{\pgfqpoint{2.934160in}{2.626736in}}%
\pgfpathlineto{\pgfqpoint{2.971166in}{2.651940in}}%
\pgfpathlineto{\pgfqpoint{3.008789in}{2.674915in}}%
\pgfpathlineto{\pgfqpoint{3.049478in}{2.697166in}}%
\pgfpathlineto{\pgfqpoint{3.093224in}{2.718487in}}%
\pgfpathlineto{\pgfqpoint{3.140006in}{2.738712in}}%
\pgfpathlineto{\pgfqpoint{3.189796in}{2.757714in}}%
\pgfpathlineto{\pgfqpoint{3.242562in}{2.775400in}}%
\pgfpathlineto{\pgfqpoint{3.298271in}{2.791704in}}%
\pgfpathlineto{\pgfqpoint{3.356890in}{2.806583in}}%
\pgfpathlineto{\pgfqpoint{3.421068in}{2.820541in}}%
\pgfpathlineto{\pgfqpoint{3.488111in}{2.832850in}}%
\pgfpathlineto{\pgfqpoint{3.560681in}{2.843871in}}%
\pgfpathlineto{\pgfqpoint{3.636070in}{2.853066in}}%
\pgfpathlineto{\pgfqpoint{3.714253in}{2.860410in}}%
\pgfpathlineto{\pgfqpoint{3.795206in}{2.865840in}}%
\pgfpathlineto{\pgfqpoint{3.878908in}{2.869242in}}%
\pgfpathlineto{\pgfqpoint{3.962636in}{2.870434in}}%
\pgfpathlineto{\pgfqpoint{4.043664in}{2.869431in}}%
\pgfpathlineto{\pgfqpoint{4.121967in}{2.866310in}}%
\pgfpathlineto{\pgfqpoint{4.197518in}{2.861074in}}%
\pgfpathlineto{\pgfqpoint{4.267590in}{2.853968in}}%
\pgfpathlineto{\pgfqpoint{4.332156in}{2.845140in}}%
\pgfpathlineto{\pgfqpoint{4.388508in}{2.835264in}}%
\pgfpathlineto{\pgfqpoint{4.439315in}{2.824218in}}%
\pgfpathlineto{\pgfqpoint{4.484559in}{2.812254in}}%
\pgfpathlineto{\pgfqpoint{4.526865in}{2.798777in}}%
\pgfpathlineto{\pgfqpoint{4.563556in}{2.784808in}}%
\pgfpathlineto{\pgfqpoint{4.597226in}{2.769590in}}%
\pgfpathlineto{\pgfqpoint{4.625280in}{2.754648in}}%
\pgfpathlineto{\pgfqpoint{4.650294in}{2.739082in}}%
\pgfpathlineto{\pgfqpoint{4.674655in}{2.721266in}}%
\pgfpathlineto{\pgfqpoint{4.695810in}{2.702976in}}%
\pgfpathlineto{\pgfqpoint{4.713788in}{2.684664in}}%
\pgfpathlineto{\pgfqpoint{4.730748in}{2.664213in}}%
\pgfpathlineto{\pgfqpoint{4.746419in}{2.641490in}}%
\pgfpathlineto{\pgfqpoint{4.758850in}{2.619726in}}%
\pgfpathlineto{\pgfqpoint{4.769909in}{2.596297in}}%
\pgfpathlineto{\pgfqpoint{4.780724in}{2.567719in}}%
\pgfpathlineto{\pgfqpoint{4.789551in}{2.537586in}}%
\pgfpathlineto{\pgfqpoint{4.796475in}{2.506308in}}%
\pgfpathlineto{\pgfqpoint{4.802219in}{2.470213in}}%
\pgfpathlineto{\pgfqpoint{4.806451in}{2.429445in}}%
\pgfpathlineto{\pgfqpoint{4.809158in}{2.380082in}}%
\pgfpathlineto{\pgfqpoint{4.809893in}{2.322298in}}%
\pgfpathlineto{\pgfqpoint{4.808380in}{2.252153in}}%
\pgfpathlineto{\pgfqpoint{4.803993in}{2.161572in}}%
\pgfpathlineto{\pgfqpoint{4.795104in}{2.030153in}}%
\pgfpathlineto{\pgfqpoint{4.770069in}{1.702694in}}%
\pgfpathlineto{\pgfqpoint{4.770069in}{1.702694in}}%
\pgfusepath{stroke}%
\end{pgfscope}%
\begin{pgfscope}%
\pgfpathrectangle{\pgfqpoint{2.713698in}{1.712694in}}{\pgfqpoint{2.295902in}{1.362506in}} %
\pgfusepath{clip}%
\pgfsetbuttcap%
\pgfsetroundjoin%
\pgfsetlinewidth{1.003750pt}%
\definecolor{currentstroke}{rgb}{0.000000,0.000000,0.000000}%
\pgfsetstrokecolor{currentstroke}%
\pgfsetdash{{3.700000pt}{1.600000pt}}{0.000000pt}%
\pgfpathmoveto{\pgfqpoint{4.831803in}{1.702694in}}%
\pgfpathlineto{\pgfqpoint{4.848789in}{1.842432in}}%
\pgfpathlineto{\pgfqpoint{4.867695in}{1.979801in}}%
\pgfpathlineto{\pgfqpoint{4.889284in}{2.120487in}}%
\pgfpathlineto{\pgfqpoint{4.937208in}{2.425450in}}%
\pgfpathlineto{\pgfqpoint{4.945664in}{2.494430in}}%
\pgfpathlineto{\pgfqpoint{4.949593in}{2.543600in}}%
\pgfpathlineto{\pgfqpoint{4.950727in}{2.584838in}}%
\pgfpathlineto{\pgfqpoint{4.949619in}{2.617812in}}%
\pgfpathlineto{\pgfqpoint{4.946720in}{2.646358in}}%
\pgfpathlineto{\pgfqpoint{4.941643in}{2.674178in}}%
\pgfpathlineto{\pgfqpoint{4.935378in}{2.697008in}}%
\pgfpathlineto{\pgfqpoint{4.927330in}{2.718492in}}%
\pgfpathlineto{\pgfqpoint{4.917634in}{2.738324in}}%
\pgfpathlineto{\pgfqpoint{4.906530in}{2.756352in}}%
\pgfpathlineto{\pgfqpoint{4.892162in}{2.775117in}}%
\pgfpathlineto{\pgfqpoint{4.876648in}{2.791621in}}%
\pgfpathlineto{\pgfqpoint{4.857923in}{2.808088in}}%
\pgfpathlineto{\pgfqpoint{4.835984in}{2.824074in}}%
\pgfpathlineto{\pgfqpoint{4.810878in}{2.839283in}}%
\pgfpathlineto{\pgfqpoint{4.782671in}{2.853537in}}%
\pgfpathlineto{\pgfqpoint{4.751444in}{2.866800in}}%
\pgfpathlineto{\pgfqpoint{4.714615in}{2.879895in}}%
\pgfpathlineto{\pgfqpoint{4.672185in}{2.892431in}}%
\pgfpathlineto{\pgfqpoint{4.624175in}{2.904136in}}%
\pgfpathlineto{\pgfqpoint{4.570610in}{2.914836in}}%
\pgfpathlineto{\pgfqpoint{4.508830in}{2.924795in}}%
\pgfpathlineto{\pgfqpoint{4.441539in}{2.933379in}}%
\pgfpathlineto{\pgfqpoint{4.366065in}{2.940779in}}%
\pgfpathlineto{\pgfqpoint{4.282424in}{2.946758in}}%
\pgfpathlineto{\pgfqpoint{4.198731in}{2.950601in}}%
\pgfpathlineto{\pgfqpoint{4.101508in}{2.953166in}}%
\pgfpathlineto{\pgfqpoint{3.998869in}{2.953661in}}%
\pgfpathlineto{\pgfqpoint{3.893535in}{2.951921in}}%
\pgfpathlineto{\pgfqpoint{3.780126in}{2.947705in}}%
\pgfpathlineto{\pgfqpoint{3.680269in}{2.941536in}}%
\pgfpathlineto{\pgfqpoint{3.566996in}{2.932125in}}%
\pgfpathlineto{\pgfqpoint{3.480803in}{2.922294in}}%
\pgfpathlineto{\pgfqpoint{3.400112in}{2.910956in}}%
\pgfpathlineto{\pgfqpoint{3.324943in}{2.898233in}}%
\pgfpathlineto{\pgfqpoint{3.225845in}{2.878826in}}%
\pgfpathlineto{\pgfqpoint{3.145855in}{2.859180in}}%
\pgfpathlineto{\pgfqpoint{3.090192in}{2.842512in}}%
\pgfpathlineto{\pgfqpoint{3.037477in}{2.824476in}}%
\pgfpathlineto{\pgfqpoint{2.987749in}{2.805102in}}%
\pgfpathlineto{\pgfqpoint{2.941041in}{2.784477in}}%
\pgfpathlineto{\pgfqpoint{2.897397in}{2.762676in}}%
\pgfpathlineto{\pgfqpoint{2.854289in}{2.738518in}}%
\pgfpathlineto{\pgfqpoint{2.816870in}{2.714778in}}%
\pgfpathlineto{\pgfqpoint{2.782555in}{2.690501in}}%
\pgfpathlineto{\pgfqpoint{2.748946in}{2.664018in}}%
\pgfpathlineto{\pgfqpoint{2.718482in}{2.637336in}}%
\pgfpathlineto{\pgfqpoint{2.703698in}{2.622891in}}%
\pgfpathlineto{\pgfqpoint{2.703698in}{2.622891in}}%
\pgfusepath{stroke}%
\end{pgfscope}%
\begin{pgfscope}%
\pgfpathrectangle{\pgfqpoint{2.713698in}{1.712694in}}{\pgfqpoint{2.295902in}{1.362506in}} %
\pgfusepath{clip}%
\pgfsetbuttcap%
\pgfsetroundjoin%
\pgfsetlinewidth{1.003750pt}%
\definecolor{currentstroke}{rgb}{0.000000,0.000000,0.000000}%
\pgfsetstrokecolor{currentstroke}%
\pgfsetdash{{3.700000pt}{1.600000pt}}{0.000000pt}%
\pgfpathmoveto{\pgfqpoint{0.000000in}{0.000000in}}%
\pgfusepath{stroke}%
\end{pgfscope}%
\begin{pgfscope}%
\pgfpathrectangle{\pgfqpoint{2.713698in}{1.712694in}}{\pgfqpoint{2.295902in}{1.362506in}} %
\pgfusepath{clip}%
\pgfsetbuttcap%
\pgfsetroundjoin%
\pgfsetlinewidth{1.003750pt}%
\definecolor{currentstroke}{rgb}{0.000000,0.000000,0.000000}%
\pgfsetstrokecolor{currentstroke}%
\pgfsetdash{{3.700000pt}{1.600000pt}}{0.000000pt}%
\pgfpathmoveto{\pgfqpoint{4.831042in}{1.702694in}}%
\pgfpathlineto{\pgfqpoint{4.847940in}{1.842397in}}%
\pgfpathlineto{\pgfqpoint{4.866739in}{1.979801in}}%
\pgfpathlineto{\pgfqpoint{4.888820in}{2.124551in}}%
\pgfpathlineto{\pgfqpoint{4.936429in}{2.429687in}}%
\pgfpathlineto{\pgfqpoint{4.944142in}{2.494678in}}%
\pgfpathlineto{\pgfqpoint{4.947897in}{2.543879in}}%
\pgfpathlineto{\pgfqpoint{4.948840in}{2.585129in}}%
\pgfpathlineto{\pgfqpoint{4.947551in}{2.618087in}}%
\pgfpathlineto{\pgfqpoint{4.944488in}{2.646593in}}%
\pgfpathlineto{\pgfqpoint{4.939237in}{2.674337in}}%
\pgfpathlineto{\pgfqpoint{4.932828in}{2.697074in}}%
\pgfpathlineto{\pgfqpoint{4.924653in}{2.718446in}}%
\pgfpathlineto{\pgfqpoint{4.914854in}{2.738160in}}%
\pgfpathlineto{\pgfqpoint{4.903670in}{2.756072in}}%
\pgfpathlineto{\pgfqpoint{4.889237in}{2.774721in}}%
\pgfpathlineto{\pgfqpoint{4.873683in}{2.791135in}}%
\pgfpathlineto{\pgfqpoint{4.854930in}{2.807529in}}%
\pgfpathlineto{\pgfqpoint{4.832970in}{2.823447in}}%
\pgfpathlineto{\pgfqpoint{4.807849in}{2.838599in}}%
\pgfpathlineto{\pgfqpoint{4.779635in}{2.852819in}}%
\pgfpathlineto{\pgfqpoint{4.748402in}{2.866053in}}%
\pgfpathlineto{\pgfqpoint{4.711569in}{2.879124in}}%
\pgfpathlineto{\pgfqpoint{4.669138in}{2.891642in}}%
\pgfpathlineto{\pgfqpoint{4.621126in}{2.903332in}}%
\pgfpathlineto{\pgfqpoint{4.567560in}{2.914019in}}%
\pgfpathlineto{\pgfqpoint{4.505779in}{2.923967in}}%
\pgfpathlineto{\pgfqpoint{4.438487in}{2.932538in}}%
\pgfpathlineto{\pgfqpoint{4.363013in}{2.939923in}}%
\pgfpathlineto{\pgfqpoint{4.279372in}{2.945882in}}%
\pgfpathlineto{\pgfqpoint{4.195678in}{2.949726in}}%
\pgfpathlineto{\pgfqpoint{4.098455in}{2.952254in}}%
\pgfpathlineto{\pgfqpoint{3.995816in}{2.952703in}}%
\pgfpathlineto{\pgfqpoint{3.890482in}{2.950902in}}%
\pgfpathlineto{\pgfqpoint{3.779774in}{2.946719in}}%
\pgfpathlineto{\pgfqpoint{3.679918in}{2.940521in}}%
\pgfpathlineto{\pgfqpoint{3.569343in}{2.931272in}}%
\pgfpathlineto{\pgfqpoint{3.483149in}{2.921455in}}%
\pgfpathlineto{\pgfqpoint{3.402458in}{2.910130in}}%
\pgfpathlineto{\pgfqpoint{3.327288in}{2.897417in}}%
\pgfpathlineto{\pgfqpoint{3.230865in}{2.878472in}}%
\pgfpathlineto{\pgfqpoint{3.148200in}{2.858258in}}%
\pgfpathlineto{\pgfqpoint{3.092538in}{2.841586in}}%
\pgfpathlineto{\pgfqpoint{3.039823in}{2.823548in}}%
\pgfpathlineto{\pgfqpoint{2.990095in}{2.804171in}}%
\pgfpathlineto{\pgfqpoint{2.943388in}{2.783544in}}%
\pgfpathlineto{\pgfqpoint{2.902296in}{2.763093in}}%
\pgfpathlineto{\pgfqpoint{2.859145in}{2.739112in}}%
\pgfpathlineto{\pgfqpoint{2.821687in}{2.715522in}}%
\pgfpathlineto{\pgfqpoint{2.787325in}{2.691398in}}%
\pgfpathlineto{\pgfqpoint{2.753661in}{2.665079in}}%
\pgfpathlineto{\pgfqpoint{2.720821in}{2.636432in}}%
\pgfpathlineto{\pgfqpoint{2.703698in}{2.619303in}}%
\pgfpathlineto{\pgfqpoint{2.703698in}{2.619303in}}%
\pgfusepath{stroke}%
\end{pgfscope}%
\begin{pgfscope}%
\pgfpathrectangle{\pgfqpoint{2.713698in}{1.712694in}}{\pgfqpoint{2.295902in}{1.362506in}} %
\pgfusepath{clip}%
\pgfsetbuttcap%
\pgfsetroundjoin%
\pgfsetlinewidth{1.003750pt}%
\definecolor{currentstroke}{rgb}{0.000000,0.000000,0.000000}%
\pgfsetstrokecolor{currentstroke}%
\pgfsetdash{{3.700000pt}{1.600000pt}}{0.000000pt}%
\pgfpathmoveto{\pgfqpoint{0.000000in}{0.000000in}}%
\pgfusepath{stroke}%
\end{pgfscope}%
\begin{pgfscope}%
\pgfsetrectcap%
\pgfsetmiterjoin%
\pgfsetlinewidth{0.803000pt}%
\definecolor{currentstroke}{rgb}{0.000000,0.000000,0.000000}%
\pgfsetstrokecolor{currentstroke}%
\pgfsetdash{}{0pt}%
\pgfpathmoveto{\pgfqpoint{2.713698in}{1.712694in}}%
\pgfpathlineto{\pgfqpoint{2.713698in}{3.075200in}}%
\pgfusepath{stroke}%
\end{pgfscope}%
\begin{pgfscope}%
\pgfsetrectcap%
\pgfsetmiterjoin%
\pgfsetlinewidth{0.803000pt}%
\definecolor{currentstroke}{rgb}{0.000000,0.000000,0.000000}%
\pgfsetstrokecolor{currentstroke}%
\pgfsetdash{}{0pt}%
\pgfpathmoveto{\pgfqpoint{5.009600in}{1.712694in}}%
\pgfpathlineto{\pgfqpoint{5.009600in}{3.075200in}}%
\pgfusepath{stroke}%
\end{pgfscope}%
\begin{pgfscope}%
\pgfsetrectcap%
\pgfsetmiterjoin%
\pgfsetlinewidth{0.803000pt}%
\definecolor{currentstroke}{rgb}{0.000000,0.000000,0.000000}%
\pgfsetstrokecolor{currentstroke}%
\pgfsetdash{}{0pt}%
\pgfpathmoveto{\pgfqpoint{2.713698in}{1.712694in}}%
\pgfpathlineto{\pgfqpoint{5.009600in}{1.712694in}}%
\pgfusepath{stroke}%
\end{pgfscope}%
\begin{pgfscope}%
\pgfsetrectcap%
\pgfsetmiterjoin%
\pgfsetlinewidth{0.803000pt}%
\definecolor{currentstroke}{rgb}{0.000000,0.000000,0.000000}%
\pgfsetstrokecolor{currentstroke}%
\pgfsetdash{}{0pt}%
\pgfpathmoveto{\pgfqpoint{2.713698in}{3.075200in}}%
\pgfpathlineto{\pgfqpoint{5.009600in}{3.075200in}}%
\pgfusepath{stroke}%
\end{pgfscope}%
\begin{pgfscope}%
\pgfsetbuttcap%
\pgfsetmiterjoin%
\definecolor{currentfill}{rgb}{1.000000,1.000000,1.000000}%
\pgfsetfillcolor{currentfill}%
\pgfsetfillopacity{0.800000}%
\pgfsetlinewidth{1.003750pt}%
\definecolor{currentstroke}{rgb}{0.800000,0.800000,0.800000}%
\pgfsetstrokecolor{currentstroke}%
\pgfsetstrokeopacity{0.800000}%
\pgfsetdash{}{0pt}%
\pgfpathmoveto{\pgfqpoint{3.461700in}{1.782138in}}%
\pgfpathlineto{\pgfqpoint{4.261598in}{1.782138in}}%
\pgfpathquadraticcurveto{\pgfqpoint{4.289376in}{1.782138in}}{\pgfqpoint{4.289376in}{1.809916in}}%
\pgfpathlineto{\pgfqpoint{4.289376in}{2.194916in}}%
\pgfpathquadraticcurveto{\pgfqpoint{4.289376in}{2.222694in}}{\pgfqpoint{4.261598in}{2.222694in}}%
\pgfpathlineto{\pgfqpoint{3.461700in}{2.222694in}}%
\pgfpathquadraticcurveto{\pgfqpoint{3.433922in}{2.222694in}}{\pgfqpoint{3.433922in}{2.194916in}}%
\pgfpathlineto{\pgfqpoint{3.433922in}{1.809916in}}%
\pgfpathquadraticcurveto{\pgfqpoint{3.433922in}{1.782138in}}{\pgfqpoint{3.461700in}{1.782138in}}%
\pgfpathclose%
\pgfusepath{stroke,fill}%
\end{pgfscope}%
\begin{pgfscope}%
\pgfsetrectcap%
\pgfsetroundjoin%
\pgfsetlinewidth{1.003750pt}%
\definecolor{currentstroke}{rgb}{0.580392,0.403922,0.741176}%
\pgfsetstrokecolor{currentstroke}%
\pgfsetdash{}{0pt}%
\pgfpathmoveto{\pgfqpoint{3.489477in}{2.118527in}}%
\pgfpathlineto{\pgfqpoint{3.558922in}{2.118527in}}%
\pgfusepath{stroke}%
\end{pgfscope}%
\begin{pgfscope}%
\pgftext[x=3.670033in,y=2.069916in,left,base]{\rmfamily\fontsize{10.000000}{12.000000}\selectfont \(\displaystyle \textnormal{tol}=10^{-6}\)}%
\end{pgfscope}%
\begin{pgfscope}%
\pgfsetbuttcap%
\pgfsetroundjoin%
\pgfsetlinewidth{1.003750pt}%
\definecolor{currentstroke}{rgb}{0.000000,0.000000,0.000000}%
\pgfsetstrokecolor{currentstroke}%
\pgfsetdash{{3.700000pt}{1.600000pt}}{0.000000pt}%
\pgfpathmoveto{\pgfqpoint{3.489477in}{1.919083in}}%
\pgfpathlineto{\pgfqpoint{3.558922in}{1.919083in}}%
\pgfusepath{stroke}%
\end{pgfscope}%
\begin{pgfscope}%
\pgftext[x=3.670033in,y=1.870472in,left,base]{\rmfamily\fontsize{10.000000}{12.000000}\selectfont Reference}%
\end{pgfscope}%
\end{pgfpicture}%
\makeatother%
\endgroup%

    \caption[A possible explanation for the behaviour of the $\rmsd$ of the
    computed LCS curves]
    {A possible explanation for the behaviour of the
        $\rmsd$ of the computed LCS curves. The computed LCS curves obtained by
        means of the Dormand-Prince 8(7) method, for tolerance levels
        $\textnormal{tol}=10^{-5}$ through to $\textnormal{tol}=10^{-8}$, are shown
        (solid) together with the reference LCS (dashed) for a subset of the
        computational domain $\mathcal{U}$. Notice how the second reference LCS
        curve segment (counting from the top and downwards) is present in the LCS
        approximations obtained for $\textnormal{tol}=10^{-5}$ and
        $\textnormal{tol}=10^{-7}$, but \emph{not} for $\textnormal{tol}=10^{-6}$
    and $\textnormal{tol}=10^{-8}$. The offset between the between the two
topmost reference LCS curve segments shown in the figure is too small for the
absence of the second topmost reference LCS curve segment to be flagged as a
false negative. For all numerical step lengths and tolerance levels which
correspond to the same level of $\rmsd$ in the computed LCS curves
(see~\cref{fig:lcs_rmsd_fp_nn_fixed,fig:lcs_rmsd_fp_nn_both,%
fig:lcs_rmsd_fn_nn_fixed,fig:lcs_rmsd_fn_nn_both}), the second topmost reference
LCS curve segment is not present in the LCS approximation. For those cases,
the offset between the analogue to the topmost one shown in this figure, and
the second topmost reference LCS curve segment is approximately constant, and
likely dominates any other errors --- which leads to the flattening of the
$\rmsd$, visible in~\cref{fig:lcs_rmsd_fp_nn_fixed,fig:lcs_rmsd_fp_nn_both,%
fig:lcs_rmsd_fn_nn_fixed,fig:lcs_rmsd_fn_nn_both}.}
    \label{fig:lcserroroscillations}
\end{figure}


Following the above discussion, the results obtained here do not indicate
the existence of a lower sufficiency threshold in terms of the required
advection accuracy, beneath which the computed LCS curves do not become more
precise. Any such threshold would, however, likely only be valid for the LCS
curves of this particular velocity field --- that is, the system given by
\cref{eq:doublegyre,eq:doublegyrefuns,eq:doublegyreparams} --- which
appear quite robust overall. Other, more volatile systems, would probably
require more accurate advection. However, investigating this further for a
wider range of systems could result in valuable insight. Should such an
advection threshold exist, and be linked to the scales of the given system,
it would naturally be of great significance when investigating generic transport
systems by means of a similar variational LCS approach. Admittedly, there is no
apparent reason why this should be the case.

%For the considered double gyre system, there appears to be a lower threshold
%in terms of the required advection accuracy, beneath which the computed LCS
%curves do not become more precise. This effect is apparent
%from inspection of
%\cref{fig:lcs_rmsd_fp_nn_fixed,fig:lcs_rmsd_fn_nn_fixed,%
%fig:lcs_rmsd_fp_nn_both,fig:lcs_rmsd_fn_nn_both}, where the error of strainline
%components identified as LCS constituents flattens abruptly. Notably, this
%occurs for larger numerical step lengths or tolerance levels, respectively,
%than the corresponding turning points for the error in the flow maps.
%For the double gyre system considered here, it appears that this advection
%accuracy threshold is of the order $10^{-6}$--$10^{-7}$, which follows from
%comparing~\cref{fig:lcs_rmsd_fp_nn_fixed,fig:lcs_rmsd_fn_nn_fixed,%
%fig:lcs_rmsd_fp_nn_both,fig:lcs_rmsd_fn_nn_both} to
%\cref{fig:flowmap_err_fixed,fig:flowmap_err_both}. In particular, for flow maps
%with $\rmsd$ of $10^{-7}$ or lower, the $\rmsd$ of the LCS curves appears to
%not decrease further as the flow map precision increases.
%
%However, because a similar flattening of the $\rmsd$ for the strain eigenvalues
%and eigenvectors is not apparent in
%\cref{fig:lmbd2_err_fixed,fig:lmbd2_err_both,fig:xi2_err_fixed,fig:xi2_err_both},
%one may infer that this threshold is likely only valid for the LCS curves
%of this particular velocity field --- that is, the system given by
%\cref{eq:doublegyre,eq:doublegyrefuns,eq:doublegyreparams} %
%%, for which LCSs are
%%found by means of the procedure described in
%%\cref{sec:advecting_a_set_of_initial_conditions,%
%%%    sec:calculating_the_cauchy_green_strain_tensor,%
%%%sec:identifying_lcs_candidates_numerically}
%--- which
%appear quite robust. The same flow map accuracy threshold probably does not
%suffice for other, more volatile flow systems. Investigating
%this further for a wider range of systems could result in valuable insight.
%Should such a threshold be valid in general, it would naturally be of great
%significance when investigating generic transport systems by means of a
%variational LCS approach. Admittedly, there is no apparent reason why
%this should be the case.
%
The double gyre model considered in this project is obviously not representative
of generic systems, in terms of the exact numerical step lengths or tolerance
levels necessary in order to obtain correct LCSs with a certain
degree of confidence. It does, however, indicate that these quantities should
be chosen based on the considered system. For a fixed stepsize integration
scheme, any single integration time step should not be so large that \emph{too}
much detail in the local and instantaneous velocity field is glossed over.
Similar logic applies when adaptive stepsize methods are used, although it
may be more difficult to enforce, depending on how the step length
update is implemented. One possibility in terms of choosing the time step, is to
find a characteristic velocity for the system, and choose the time step small
enough so that a tracer moving with the characteristic velocity never traverses
a distance greater than the grid spacing, when moving from one time level to
the next.
\clearpage
The computed reference LCS for the double gyre system considered here, shown in
\cref{fig:referencelcs}, is made up of \emph{eight} different strainline
segments. The LCS presented in the article by \textcite{farazmand2012computing}
is claimed to consist of a \emph{single} strainline segment. Comparing the two
curves visually, however, indicates that the resulting LCSs are similar.
Likewise, the domain $\mathcal{U}_{0}$, shown in figure~\ref{fig:u0_domain},
strongly resembles the one found by \citeauthor{farazmand2012computing}.
Nevertheless, the total number of points in the domain computed here is
approximately two percent larger than what \citeauthor{farazmand2012computing}
found. These discrepancies could originate from different conventions in terms
of generating the grid of tracers. Notably, \citeauthor{farazmand2012computing}
fail to provide a description of their approach.

When computing transport based on discrete data sets, such as snapshots of the
instantaneous velocity fields in oceanic currents, spatial and temporal
interpolation becomes necessary. Together with the inherent precision of the
model data, the choice of interpolation scheme(s) sets an upper bound
in terms of the accuracy with which tracers can be advected. For such cases,
the interaction between the integration and interpolation schemes could
be critical --- both in terms of computation time and memory requirements,
aside from the numerical precision. Independently of the scales at which
well-resolved LCS information is sought in this kind of system, the
aforementioned effects warrant further investigation.
