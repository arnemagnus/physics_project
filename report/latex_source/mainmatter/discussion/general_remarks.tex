\section{General remarks}
\label{sec:general_remarks}

First off, the computed LCS used as the reference, shown in figure
\ref{fig:referencelcs}, is made up of \emph{seven} different strainline segments.
The LCS presented in the article by \textcite{farazmand2012computing} is claimed
to consist of a \emph{single} strainline segment. Comparing the two curves
visually, however, indicates that the resulting LCSs are similar. Likewise, the
domain $\mathcal{U}_{0}$, shown in figure~\ref{fig:u0_domain}, strongly
resembles the one found by \citeauthor{farazmand2012computing}. Nevertheless,
the total number of points in the domain computed here is approximately two
percent larger than what \citeauthor{farazmand2012computing} found. These
discrepancies could originate from different conventions in terms of generating
the grid of tracers. Notably, \citeauthor{farazmand2012computing} fail to
provide a description of their approach.

Overall, the use of a variational framework for computing LCSs appears to
produce robust and consistent LCS curves for all of the numerical integration
schemes considered here, subject to usage of a sufficiently small numerical
step length or tolerance level. This indicates that calculations of this kind
are not particularly sensitive to integration method. Note that the
$\mathcal{U}_{0}$ domains obtained by means of the adaptive stepsize methods
for $\textnormal{tol}=10^{-1}$, shown in figure~\ref{fig:u0_dom_errs}, differ
greatly to the \emph{correct} domain, which correlates well with the computed
$\rmsd$ of the flow maps, shown in figure~\ref{fig:flowmap_err_both}. In
particular, the error for the aforementioned tolerance level is of order $1$,
which is comparable to the extent of the numerical domain, that is,
$[0\hspace{1ex}2]\times[0\hspace{1ex}1]$. Naturally, this leads to a drastically
different system.

The above observation implies that the most crucial part of the computation is
advecting the tracers accurately. As mentioned in
\cref{sub:computed_deviations_in_the_strain_eigenvalues_and_vectors}, the error
in the computed strain eigenvalues scales like the error in the computed flow
map. This is to be expected, as the eigenvalues are found by applying finite
differences to the \emph{main} tracers. The eigenvalues are essential in terms
of identifying LCSs, as can be seen in the numerical reformulation of the
LCS existence theorem, which is given in equation~\eqref{eq:numericalexistence}.
The strain eigendirections are important too, of course --- however, the
error in the computed strain eigendirections is consistently negligible.
This is most likely due to them being computed based on the denser
\emph{auxiliary} set of tracers, which per construction results in more accurate
finite differences.

The double gyre model considered in this project is clearly not representative
of a generic system, in terms of exact numerical step lengths or tolerance
levels which are sufficient in order to obtain correct LCSs with a certain
degree of confidence. It does, however, indicate that these quantities should
be chosen based on the considered system. For a fixed timestep integration
scheme, any single integration time step should not be so large that \emph{too}
much detail in the local and instantaneous velocity profile is glossed over.
Similar logic applies when adaptive stepsize methods are used, although it
may be more difficult to enforce, depending on how the step length
update is implemented. One possibility in terms of choosing the time step, is to
find a characteristic velocity for the system, and choose the time step small
enough so that a tracer moving with the characteristic velocity never skips over
a grid cell entirely, when moving from one time level to the next.

When computing transport based on discrete data sets, such as snapshots of the
instantaneous velocity profiles in oceanic currents, spatial and temporal
interpolation becomes necessary. Together with the inherent precision of the
measurement data, the choice of interpolation scheme(s) set a lower bound
in terms of the accucacy with which tracers can be advected. For such cases,
the interaction between the integration and interpolation schemes could
be critical --- both in terms of computation time and memory requirements,
aside from the numerical precision. Independently of the scales at which
well-resolved LCS information is sought in this kind of system, the
aforementioned effects warrant further investigation.
