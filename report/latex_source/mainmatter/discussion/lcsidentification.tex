\section{The identification of strain maximizing strainlines}
\label{sec:the_identification_of_strain_maximizing_strainlines}

As mentioned in~\cref{sub:extracting_hyperbolic_lcss_from_strainlines},
the lines in the set $\mathcal{L}$, used in order to find strainline segments
satisfying the numerical LCS existence condition given in
\cref{eq:numericalexistence4} --- regarding the identification of strainlines
which serve as local maxima for $\overline{\lambda}_{2}$ --- had to be selected
with great care in order to accurately reproduce the LCS curve found in the
literature. Although the LCS identification with the particular set of
horizontal and vertical lines proved robust across all integration methods, the
fact that only certain sets $\mathcal{L}$ do the job, is a damning indicament
that the procedure as such is not particularly robust. Although not investigated
here, an alternative approach could be to identify all strainlines which follow
very similar trajectories, either manually or by means of some numerical
clustering algorithm, then extracting the most strongly repellent strainline
segments of each bunch as LCSs for the system.

The strainline tail end cutting procedure which was employed, briefly brought
up in
\cref{sub:extracting_hyperbolic_lcss_from_strainlines} and illustrated in
\cref{fig:tailcutting}, is most certainly debatable. The reason it was
considered in the first place, is that the specific wording used by
\textcite{farazmand2012computing} on the subject of comparing strainlines is
somewhat ambiguous, in that they also describe that part of the process as
comparison of \emph{curve segments}. Moreover, they also state that a
\emph{part} of a strainline may qualify as an LCS. In addition, cutting the
tail ends of strainlines which are stopped due to continious failures of one
or more of the LCS conditions given in
\cref{eq:numericalexistence} could also, quite logically, be extended so that
parts of the remaining strainline curve which, for example due to numerical
noise, do not satisfy all of the aforementioned conditions, are excluded from
the ensuing LCS identification. That is, those parts would not be considered
when computing the $\overline{\lambda}_{2}$, the averaged $\lambda_{2}$ of
the strainline segment, on the strainline as a whole, nor
the strainline length. Effectively, this could result in a strainline being
chopped into several shorter segments, further resulting in
disjointed LCSs. As mentioned previously, the tail end cutting proved necessary
in order to reproduce the LCS curve found by \textcite{farazmand2012computing},
meaning that this concept can, and most likely should, be investigated further.

Lastly, the rationale of filtering out LCS candidate curves which are shorter
than the preselect length $l_{\textnormal{f}}=1$, described in
\cref{sub:extracting_hyperbolic_lcss_from_strainlines} and inspired by
\textcite{farazmand2012computing}, was that excessively short LCSs are
expected to have a negligible impact on the overall flow in the system.
However, \citeauthor{farazmand2012computing} do not provide a justification
for why they chose that particular filtering length. Another way to perform this
sifting, would be to consider $\overline{\lambda}_{2}$ together with the length
of the strainline segment. This could be done in such a way that, when selecting
an LCS candidate from two strainlines, if one is somewhat longer but has a
slightly smaller $\overline{\lambda}_{2}$, that is, is slightly less repelling
than, the other, then the longer strainline is selected. This sort of routine
should naturally be based upon sound mathematical logic. In short, there is a
lot of room for research in terms of how to enforce the LCS condition given
by~\cref{eq:numericalexistence4}. The conditions given by
\cref{eq:numericalexistence1,eq:numericalexistence2,eq:numericalexistence3}
are quite unambigious, in comparison.

%\vspace{\fill}
%\clearpage
