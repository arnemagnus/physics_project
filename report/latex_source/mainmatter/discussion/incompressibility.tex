\section{On the incompressibility of the velocity field}
\label{sec:on_the_incompressibility_of_the_velocity_field}

As explained in~\cref{sec:the_double_gyre_model}, the double gyre velocity field
is incompressible per construction. Thus, per
\cref{eq:cauchygreenincomprlambda}, the product of the strain eigenvalues
should equal one. Regardless of the integration method used, however, this
property was lost after approximately five units of time. This is not
particularly surprising, because this property only really holds for
infinitesimal fluid elements. Seeing as there is no assurance that neighboring
tracers will remain nearby under the flow map, the finite difference
approximation to the local stretch and strain could practically be rendered
invalid. This issue is expected to be most prominent in regions of high local
repulsion, which are precisely where accuracy is most imperative.
\clearpage
Thus, one should not really expect the property $\lambda_{1}\lambda_{2}=1$ to
hold numerically, especially for an indefinite time. However, sample tests
indicate that the incompressibility property was preserved for longer when
using a denser grid of tracers. In fact, the strain eigenvalue product was
found to be unity beyond 20 units of time, that is, the integration time used in
this project, for an initial grid spacing of $10^{-12}$. However, due to the
machine epsilon of double-precision floating point numbers being of order
$10^{-16}$, this would leave very few significant digits with which one could
perform finite differencing, as mentioned in
\cref{sub:on_the_choice_of_numerical_step_lengths_and_tolerance_levels}.
In the end, the grid spacings $\Delta{x}\simeq\Delta{y}\simeq0.002$
and $\delta{x}=\delta{y}=10^{-5}$ were chosen, because these were the
resolutions at which repelling LCS curves for the double
gyre system have previously been described in the literature, cf.
\textcite{farazmand2012computing}. For further analysis, auxiliary grid spacings
$\delta{x}$ and $\delta{y}$ of approximate order $10^{-7}$--$10^{-8}$ could be
considered, when working with double-precision floating-point numbers. This
would probably result in the incompressibility property being preserved for
longer, while also leaving up to 7 or 8 significant digits for the finite
difference approximation of the flow map, which is outlined in
\cref{sec:calculating_the_cauchy_green_strain_tensor}.

An entirely different approach, suggested by \textcite{onu2015lcstool}, is to
simply \emph{define} the smaller strain eigenvalue as the reciprocal of the
larger one, that is, $\lambda_{1}\equiv\lambda_{2}^{-1}$. Computing the smaller
eigenvalue in this way could influence the computation of LCSs through
the condition~\eqref{eq:numericalexistence1}. This was not done here,
primarily because \citeauthor{onu2015lcstool} did not provide an illustration
of their computed $\mathcal{U}_{0}$ domain, that is, a figure similar
to~\cref{fig:u0_domain}; unlike \textcite{farazmand2012computing}, who
did not follow the same approach as \citeauthor{onu2015lcstool}. Thus, to our
knowledge, following the convention $\lambda_{1}=\lambda_{2}^{-1}$ would result
in LCSs without an established reference in the literature. Moreover, as is
evident from inspection of the numerical LCS existence conditions, given in
\cref{eq:numericalexistence}, the computations are considerably more sensitive
to $\lambda_{2}$ than $\lambda_{1}$, which suggests that accurate computations
of LCSs for the system under consideration is not particularly susceptible to
the numerical conservation of its inherent incompressibility.
%for the velocity field than the ones given in~\cref{eq:doublegyreparams},
%because the flow incompressibility should matter little with regards to LCSs
%--- in fact, $\lambda_{1}$ only enters in \emph{one} of the numerical
%LCS conditions, namely the one given by~\cref{eq:numericalexistence1}.
%Accordingly, erroneous $\lambda_{1}$ values were only really relevant when
%In addition, subject to the accuracy of the numerical integration scheme,
%no tracer which starts out within the computational domain
%$[0,\hspace{1ex}2]\times[0,\hspace{1ex}1]$ ever leave it, due to the normal
%component of the velocity field being zero along the domain edges. This
%follows from the definition of the associated velocity field, given in
%\cref{eq:doublegyre}.
