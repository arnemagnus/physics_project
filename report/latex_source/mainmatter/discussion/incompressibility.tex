\section{On the incompressibility of the velocity field}
\label{sec:on_the_incompressibility_of_the_velocity_field}

As explained in~\cref{sec:the_double_gyre_model}, the double gyre velocity field
is incompressible per construction. Thus, per equation
\eqref{eq:cauchygreenincomprlambda}, the product of the strain eigenvalues
should equal one. Regardless of the integration method used, however, this
property was lost after approximately five units of time. This is not
particularly surprising, because this property only really holds for
infinitesimal fluid elements. Seeing as there is no assurance that neighboring
tracers will remain nearby under the flow map, the finite difference
approximation to the local stretch and strain could practically be rendered
invalid. This issue is expected to be most prominent in regions of high local
repulsion, which are precisely where accuracy is most imperative.

Thus, one should not really expect this property to hold numerically, especially
for an indefinite time. However, sample tests indicate that the
incompressibility property was preserved for longer when using a denser
grid of tracers. In fact, the strain eigenvalue product was found to be unity
beyond 20 units of time, that is, the integration time used in this project,
for an initial grid spacing of $10^{-12}$. However, due to the inherent
inaccuracy of double-precision floating-point numbers, this would leave very
few significant digits with which one could perform finite differencing, as
mentioned in
\cref{sub:on_the_choice_of_numerical_step_lengths_and_tolerance_levels}.
In the end, the grid spacing $\Delta{x}\simeq\Delta{y}\simeq0.02$ was chosen,
because it was the resolution at which repelling LCS curves for the double
gyre system have previously been described in the literature, cf.
\textcite{farazmand2012computing}.

For further analysis, grid spacings of approximate order $10^{-7}$--$10^{-8}$
could be considered, when working with double-precision floating-point numbers.
This would probably result in the incompressibility property being preserved
for longer, while also leaving up to 7 or 8 significant digits for finite
differencing. An entirely different approach, as suggested by
\textcite{onu2015lcstool}, is to simply \emph{define} the smaller strain
eigenvalue as  the reciprocal of the larger one, that is,
$\lambda_{1}\equiv\lambda_{2}^{-1}$. This was not done here, because the
flow incompressibility has no obvious practical consequences for LCSs.
In addition, subject to the accuracy of the numerical integration scheme,
no tracer which starts out within the computational domain
$[0\hspace{1ex}2]\times[0\hspace{1ex}1]$ ever leave it, due to the normal
component of the velocity field being zero along the domain edges. This
follows from the definition of the associated velocity field, given in
equation~\eqref{eq:doublegyre}.
