\section{Concerning the numerical representation of tracers}
\label{sec:concerning_the_numerical_representation_of_tracers}

One way of increasing the numerical accuracy in the flow map, would be
to use higher precision floating-point numbers to represent the tracer
coordinates. The main drawback of making such a change, is that this
neccessitates an increase in memory usage, which has the practical consequence
that less tracers can be advected at once. The immediately obvious workarounds
are to either use a high performance scientific computer, or to perform several
advections of smaller sets of tracers, calculating a piecewise representation
of the overall flow map. However, should this prove impossible, using higher
precision floating-point numbers leads to a more granular representation of the
flow map. Such an approach is perhaps most sensible for systems with velocity
fields that are well-behaved, or even largely spatially invariant. However,
in such cases, calculating LCSs have little practical relevance, seeing as
the overall behaviour of the flow system could be estimated purely by
advecting a smaller set of tracers, or even by simple inspection.

Rather than generating and advecting a fixed, large amount of tracer particles,
another possible approach would be to use a set of fewer `base' tracers, and
making use of an \emph{adaptive multigrid method}. A practical implementation
could involve the dynamic introduction of increasingly finer grids in regions
where the local velocities would have the largest Euclidean norm, for instance.
Such grids, however, would have to move \emph{with} the flow, as the
benefit over simply increasing the initial tracer density would diminish
otherwise. The main reason why this was not implemented for this project,
aside from it not being used in the literature, is that it in all likelyhood
leads to inconsistencies when applying finite differences. That is, unless some
sort of interpolation scheme was applied to the flow map. Seeing as this
project is centered around integration methods, this idea was scrapped.
Regardless, this technique seems promising --- at least on paper, which
warrants further investigation.

\vspace{\fill}
