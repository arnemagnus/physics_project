\section{Regarding the computation of strainlines}
\label{sec:regarding_the_computation_of_strainlines}
As outlined in~\cref{sub:a_framework_for_computing_smooth_strainlines}, a
special sort of rectifying linear interpolation routine was implemented in
order to eliminate local orientational discontinuities in the $\vct{\xi}_{1}$
strain eigenvector field. The logical next step would be to consider a larger
local subset of grid points, for instance, the $3\times3$ or $4\times4$ square
of the 9 or 16 nearest neighbors, respectively, systematically reorientating the
eigenvectors if necessary, and then using a higher order interpolation scheme
in order to approximate the local strain eigenvector. This sort of
generalization could make the strainline computation process more robust ---
although the linear approximation approach proved sufficient for the
velocity field considered here, that may not be the case for more complex or
volatile flow systems.

The classical Runge-Kutta method, with a numerical step length
$\Delta=10^{-3}$, was used for all computations of strainlines, regardless
of advection integration scheme. This was a conscious choice, based on the idea
of using the available information in the flow maps with the same degree of
precision for all integrator configurations. If, however, the entire process of
numerical integration, including both the advection of tracers and the
calculation of strainlines, were to be performed with the same integrator
configuration, comparisons between the resulting LCSs could have been
misleading. That in, the strain information present in the flow maps would
necessarily not have been utilized to the full extent, if the strainline
integration was performed by means of an imprecise integration scheme, i.e.,
for large time steps or tolerance levels. For instance, the LCSs curves obtained
by means of the Euler method, shown in figure~\ref{fig:lcs_euler}, could have
exhibited even more false positives and negatives if the same method was used
in order to obtain strainlines as well, due to it only being \nth{1}-order
accurate.

An alternative approach to the numerical integration of strainlines, would be
to make use of a high order adaptive step length method. This could, in
principle, reduce the required computational time, in addition to reducing
the impact of floating-point arithmetic error. However, in order to encapsulate
the local strain dynamics accurately, the embedded automatic step size control
should probably be more elaborate than the implementation outlined in
\cref{sub:on_the_implementation_of_embedded_runge_kutta_methods}. In particular,
it would be advisable to incorporate the local strainline curvature somehow,
in order to minimize deviations from the \emph{true} trajectories. Furthermore,
the use of a higher order strainline integration scheme might prove a fruitless
exercise, unless a higher order strain eigenvector interpolation routine, as
mentioned above, was implemented in tandem. This is because the effective
accuracy of the strainline integration method depends strongly on the
interpolation method, as evidenced by the reformulation of the basic
strainline ODE given in~\cref{eq:strainlinebasicode} to the ODE which
was used here, given in~\cref{eq:strainlineode}. As for the transport of tracers
based on discrete velocity data, mentioned in~\cref{sec:general_remarks}, the
interaction between strain eigenvector interpolation method, and strainline
integration method, seems like an interesting research topic for potential
future endeavors.

%
%\vspace{\fill}
%\clearpage
