\section{About the measures of error}
\label{sec:about_the_measures_of_error}

For all the measures of error introduced in~\cref{sec:estimation_of_errors},
except the one concerning false positives and negatives with regards to
LCS identification, RMS averaging was utilized. Regarding the flow maps,
this choice was made based on the fact that it encapsulates the effect of
outliers, that is, any tracers for which the advection error becomes
relatively large. These are expected to have the most severe impact on the
ensuing LCS identification procedure. As shown in
\cref{fig:flowmap_err_fixed,fig:flowmap_err_both}, the errors in the flow map
resulting of the Euler method and all of the adaptive stepsize methods, for the
largest numerical step length and tolerance level, respectively, are of order 1,
similar to the dimensions of the computational domain, that is,
$[0\hspace{1ex}2]\times[0\hspace{1ex}1]$. However, because there is no general
a priori way of knowing the regions of the domain in which the flow maps will be
resolved the most poorly, the maximum error is not of particular interest
in this kind of analysis.

Because the normal component of the velocity field is zero at the domain
boundaries, one expects the flow map error to, in some sense, be limited from
above by the domain extent. Thus, one could argue that using the median error
as the measure in such cases would be more appropriate. This is, however,
clearly an artifact of the particular velocity field studied here. For a
generic, compressible flow system, this need not be the case. In order to
conform to the general case, and include the effect of outlier errors, the
$\rmsd$ was chosen as the measure of error for all flow maps. Moreover, given
that the error in the flow maps follows some statistical distribution ---
the details of which was not investigated in this project --- the $\rmsd$
can be viewed as a first approximation of the standard deviation.

%For reasons of consistency, in addition to
%the inclusion of the error of outliers, the $\rmsd$ was used as the measure of
%error for all flow maps. Similar arguments apply for the other measures
%of errors involving $\rmsd$. Moreover, each particular kind of error
%is expected to follow some sort of statistical distribution. Although the
%exact natures of the distribution have not been explored in detail in this
%project, because the tracers are advected independently, each error in
%the flow maps, for instance, can be considered as an independent
%random variable. Because of the large numbers of individual data samples,
%the $\rmsd$ can be considered a measure of the standard deviation of the
%distribution of errors as a whole.
%
Regarding the errors in the strain eigenvalues, one could argue that using
\emph{relative} errors would be sensible. However, the property
given in~\cref{eq:cauchygreenincomprlambda}, that is, $\lambda_{1}\lambda_{2}=1$,
does not hold for general, compressible flows --- nor was it preserved for
more than a few units of time for the flow system considered here, as mentioned
in~\cref{sec:on_the_incompressibility_of_the_velocity_field} ---, for which very
small eigenvalues would result in deceptively large relative errors.
This would, again, be undesirable, because one would expect the eigenvalue
errors to scale similarly to that of the flow map, due to the computation of
the Cauchy-Green strain tensor field being based on finite differencing applied
to the flow map, as described in
\cref{sec:calculating_the_cauchy_green_strain_tensor}. So, in order to adhere
to the case of a general flow, the $\rmsd$ became the method of choice in
computing the error of the strain eigenvalues. Moreover, seeing as the
azimuthal angles of the strain eigenvectors are calculated relative to what is
really an arbitrary axis, as far as the overall flow patterns are concerned, the
\emph{absolute} errors in the strain eigendirections is the only sensible
measure. Thus, the $\rmsd$ was chosen as the measure of error for the
directions of all strain eigenvectors, too.
%
%While one could argue that using \emph{relative} errors with regards to
%the strain eigenvalues would be more applicable, the fact that several
%$\lambda_{1}$ values were close to zero would result in the errors being
%deceptively large. As a consequence the relative errors in the
%%$\lambda_{1}$ and $\lambda_{2}$ fields would scale differently, which would,
%again, be deceptive. In addition, because the azimuthal angles of the strain
%eigenvectors are calculated relative to what is really an arbitrary axis,
%as far as the overall flow patterns are concerned, the \emph{absolute} errors
%regarding the strain eigendirections is the only sensible measure. So, besides
%the issues arising for small $\lambda_{1}$ values, the use of $\rmsd$ for
%the eigenvalues was made for reasons of consistency.

Lastly, regarding the estimation of the offset of false positive and false
negative LCS curve segments, there certainly exists other relevant measures.
An important aspect of any such principle, however, is the elimination of
subjective assessment, for instance, one should not have to identify
what LCS curve segment a false positive most closely resembles by means of
inspection. Furthermore, the offset of false positives and negatives is not
an independent error measure --- it depends intrinsically on the errors
in the strain eigenvalue and eigenvector fields. As such, an interesting
approach could be to consider the entire set of computed strainlines, employing
some means of identifying the curve segments which resemble the reference
LCS curves the most, and examining the reason for which any false negatives
are eliminated, or why any false positives are included, in the final picture.
This sort of approach could provide useful insight into the nature of LCSs, but
was ultimately not pursued in this project, as it simply fell outside the
overall scope of this project, that is, the dependence of LCSs on the
underlying numerical integration scheme.

%\begin{framed}
%    Handwaving wrt selecting the curve segments which `most closely resemble'
%    the reference LCS. Need a method which avoids these kinds of subjective
%    evaluations.
%\end{framed}

%Regarding the estimation of the offset of falsely positively or negatively
%identified LCS curves, there exists other measures of error.
