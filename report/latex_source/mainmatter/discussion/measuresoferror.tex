\section{About the measures of error}
\label{sec:about_the_measures_of_error}

For all the measures of error introduced in~\cref{sec:estimation_of_errors},
except the one concerning false positives and negatives with regards to
LCS identification, a sort of averaging was utilized. Regarding the flow maps,
this choice was made based on the fact that it encapsulates the effect of
outliers, that is, any tracers for which the advection error becomes
relatively large. These are expected to have the most severe impact on the
ensuing LCS identification procedure. As shown in
\cref{fig:flowmap_err_fixed,fig:flowmap_err_both}, the errors of the Euler
method and all of the adaptive stepsize methods, for the largest numerical step
length and tolerance level, respectively, are of order 1, similar to the
dimensions of the computational domain, that is,
$[0\hspace{1ex}2]\times[0\hspace{1ex}1]$. However, because there is no general
a priori way of knowing the regions of the domain in which the flow maps will be
resolved the most poorly, the maximum error is not of particular interest
in this kind of analysis.

Because the normal component of the velocity field is zero at the domain
boundaries, one expects the flow map error to, in some sense, be limited from
above by the domain extent. Thus, one could argue that using the median error
as the measure in such cases would be more appropriate. This is clearly an
artifact of the particular velocity field, however. For a generic compressible
system, this need not be the case. For reasons of consistency, in addition to
the inclusion of the error of outliers, the $\rmsd$ was used as the measure of
error for all flow maps. Similar arguments apply for the other measures
of errors involving $\rmsd$. Moreover, each particular kind of error
is expected to follow some sort of statistical distribution. Although the
exact natures of the distribution have not been explored in detail in this
project, because the tracers are advected independently, each error in
the flow maps, for instance, can be considered as an independent
random variable. Because of the large numbers of individual data samples,
the $\rmsd$ can be considered a measure of the standard deviation of the
distribution of errors as a whole.

The reason for which the offset of falsely positively or negatively identified
LCS curves was calculated as the

\begin{framed}
    Handwaving wrt selecting the curve segments which `most closely resemble'
    the reference LCS. Need a method which avoids these kinds of subjective
    evaluations.
\end{framed}

Regarding the estimation of the offset of falsely positively or negatively
identified LCS curves, there exists other measures of error.
