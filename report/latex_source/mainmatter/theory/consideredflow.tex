\section{The type of flow systems considered}
\label{sec:typeofflow}

We consider flow in two-dimensional dynamical systems of the form

\begin{equation}
    \label{eq:odeflowsystem}
\dot{\vct{x}} = \vct{v}(t,\vct{x}),\quad\vct{x}\in\mathcal{U},\quad{}t\in[t_{0},t_{1}],
\end{equation}

i.e., systems defined for the finite time interval $[t_{0},t_{1}]$, on an open,
bounded subset $\mathcal{U}$ of $\mathbb{R}^{2}$. In addition, the velocity
field $\vct{v}$ is assumed to be smooth in its arguments. Depending on the
exact nature of the velocity field $\vct{v}$, analytical particle trajectories,
i.e., analytical solutions of system~\eqref{eq:odeflowsystem}, may or may not be
computed. The flow particles are assumed to be infinitesimall and massless,
i.e., non-interacting \emph{tracers} of the overall circulation.

Letting $\vct{x}(t;t_{0},\vct{x}_{0})$ denote the trajectory of a tracer in the
system defined by~\eqref{eq:odeflowsystem}, the flow map is defined as

\begin{equation}
    \label{eq:flowmap}
    \vct{F}_{t_{0}}^{t}(\vct{x}_{0}) = \vct{x}(t;t_{0},\vct{x}_{0}),
\end{equation}

i.e., the flow map describes the mathematical movement of the tracers from
one point in time (e.g.\ the initial condition) to another. Generally, the
flow map is as smooth as the velocity field $\vct{v}$ in system
\eqref{eq:odeflowsystem} \parencite{farazmand2012computing}. For sufficiently
smooth velocity fields, the right Cauchy-Green strain tensor field is defined as

\begin{equation}
    \label{eq:cauchygreen}
    \mtrx{C}_{t_{0}}^{t}(\vct{x}_{0}) = %
    {\big(\vct{\nabla}\vct{F}_{t_{0}}^{t}(\vct{x}_{0})\big)}^{\ast} %
    \,\vct{\nabla}\vct{F}_{t_{0}}^{t}(\vct{x}_{0}),
\end{equation}

where $\vct{\nabla}\vct{F}_{t_{0}}^{t}$ denotes the Jacobian matrix of the flow
map $\vct{F}_{t_{0}}^{t}$, and the asterisk refers to the adjoint operation,
which, because the Jacobian $\vct{\nabla}\vct{F}_{t_{0}}^{t}$ is real-valued,
equates to matrix transposition. Component-wise, the Jacobian matrix of a
general vector-valued function $\vct{f}$ is defined as

\begin{equation}
    \label{eq:jacobiancomponent}
    {(\vct{\nabla}\vct{f})}_{i,j} = \pdv{f_{i}}{x_{j}}
\end{equation}

which, for our two-dimensional flow, reduces to

\begin{equation}
    \label{eq:jacobiantotal}
    \renewcommand{\arraystretch}{2.5}
    \vct{\nabla}\vct{f} = \begin{pmatrix} %
        \dfrac{\partial{}f_{1}}{\partial{}x} %
                                    & \dfrac{\partial{}f_{1}}{\partial{}y}\\
        \dfrac{\partial{}f_{2}}{\partial{}x} %
                                    & \dfrac{\partial{}f_{2}}{\partial{}y}%
                        \end{pmatrix}.
\end{equation}

By construction, the Cauchy-Green strain tensor $\mtrx{C}_{t_{0}}^{t}$ is
symmetric and positive definite. Thus, it has two real, positive eigenvalues
and orthogonal, real eigenvectors. Its eigenvalues $\lambda_{i}$ and
corresponding unit eigenvectors $\vct{\xi}_{i}$ are defined by

\begin{equation}
    \label{eq:cauchygreencharacteristics}
    \begin{gathered}
        \mtrx{C}_{t_{0}}^{t}(\vct{x}_{0})\vct{\xi}_{i}(\vct{x}_{0}) %
            = \lambda_{i}\vct{\xi}_{i}(\vct{x}_{0}),%
            \quad\norm{\vct{\xi}_{i}(\vct{x}_{0})}=1,\quad{}i=1,2,\\
        0<\lambda_{1}(\vct{x}_{0})\leq\lambda_{2}(\vct{x}_{0}),
    \end{gathered}
\end{equation}

where, for the sake of notational transparency, the dependence of $\lambda_{i}$
and $\vct{\xi}_{i}$ on $t_{0}$ and $t$ has been suppressed. The geometrical
interpretation of equation~\eqref{eq:cauchygreencharacteristics} is that
a fluid element undergoes the most stretching along the $\vct{\xi}_{2}$ axis,
and the least along the $\vct{\xi}_{1}$ axis. This is illustrated in figure
\ref{fig:stretch_and_strain}.

\begin{figure}[htpb]
    \centering
    \def\svgwidth{0.8\linewidth}
    \begin{figure}[htpb]
    \centering
    \def\svgwidth{0.8\linewidth}
    \begin{figure}[htpb]
    \centering
    \def\svgwidth{0.8\linewidth}
    \input{figures/stretch_and_strain.pdf_tex}
    \caption[Geometric interpretation of the eigenvectors of the Cauchy-Green
        strain tensor]{Geometric interpretation of the eigenvectors of the
        Cauchy-Green strain tensor. The central unit cell is stretched and
        deformed under the flow map $\vct{F}_{t_{0}}^{t}(\vct{x}_{0})$. The
        local stretching is the largest in the direction of $\vct{\xi}_{2}$, the
        eigenvector which corresponds to the largest eigenvalue $\lambda_{2}$
        of the Cauchy-Green strain tensor, defined in equation
        \eqref{eq:cauchygreencharacteristics}. Along the $\vct{\xi}_{2}$ and
        $\vct{\xi}_{1}$ axes, the stretch factors are $\sqrt{\lambda_{2}}$ and
        $\sqrt{\lambda_{1}}$, respectively.}
    \label{fig:stretch_and_strain}
\end{figure}

    \caption[Geometric interpretation of the eigenvectors of the Cauchy-Green
        strain tensor]{Geometric interpretation of the eigenvectors of the
        Cauchy-Green strain tensor. The central unit cell is stretched and
        deformed under the flow map $\vct{F}_{t_{0}}^{t}(\vct{x}_{0})$. The
        local stretching is the largest in the direction of $\vct{\xi}_{2}$, the
        eigenvector which corresponds to the largest eigenvalue $\lambda_{2}$
        of the Cauchy-Green strain tensor, defined in equation
        \eqref{eq:cauchygreencharacteristics}. Along the $\vct{\xi}_{2}$ and
        $\vct{\xi}_{1}$ axes, the stretch factors are $\sqrt{\lambda_{2}}$ and
        $\sqrt{\lambda_{1}}$, respectively.}
    \label{fig:stretch_and_strain}
\end{figure}

    \caption[Geometric interpretation of the eigenvectors of the Cauchy-Green
        strain tensor]{Geometric interpretation of the eigenvectors of the
        Cauchy-Green strain tensor. The central unit cell is stretched and
        deformed under the flow map $\vct{F}_{t_{0}}^{t}(\vct{x}_{0})$. The
        local stretching is the largest in the direction of $\vct{\xi}_{2}$, the
        eigenvector which corresponds to the largest eigenvalue $\lambda_{2}$
        of the Cauchy-Green strain tensor, defined in equation
        \eqref{eq:cauchygreencharacteristics}. Along the $\vct{\xi}_{2}$ and
        $\vct{\xi}_{1}$ axes, the stretch factors are $\sqrt{\lambda_{2}}$ and
        $\sqrt{\lambda_{1}}$, respectively.}
    \label{fig:stretch_and_strain}
\end{figure}




Because the stretch factors along the $\vct{\xi}_{2}$ and $\vct{\xi}_{1}$ axes
are given by the square root of the corresponding eigenvalues, cf.\
equation~\eqref{eq:cauchygreencharacteristics}, for incompressible flow, the
eigenvalues satisfy

\begin{equation}
    \label{eq:cauchygreenincomprlambda}
    \lambda_{1}(\vct{x}_{0})\lambda_{2}(\vct{x}_{0})=1%
        \quad\forall\hskip0.5em\vct{x}_{0}\in\mathcal{U}
\end{equation}


where, for our case, incompressibility is equivalent to the velocity field
$\vct{v}$ being divergence-free (i.e., $\vct{\nabla}\vdot\vct{v}\equiv0$).
This is due to the tracer particles begin massless, per definiton.

\section{Definition of LCSs for two-dimensional flows}
\label{sec:definition_of_lcss_for_two-dimensional_flow}
Lagrangian coherent structures (henceforth abbreviated to LCSs) can be described
as time-evolving surfaces which shape coherent trajectory patterns in dynamical
systems, defined over a finite time interval \parencite{haller2010variational}.
There are three main types of LCSs, namely \emph{elliptic},
\emph{hyperbolic} and \emph{parabolic}. Roughly speaking, parabolic structures
outline cores of jet-like trajectories, elliptic structures describe vortex
boundaries, whereas hyperbolic structures illustrate overall attractive or
repelling manifolds. As such, hyperbolic LCSs practically act as organizing
centers of observable tracer patterns \parencite{onu2015lcstool}. Because
hyperbolic LCSs provide the most readily applicable insight in terms of
forecasting flow in e.g.\ oceanic currents, such structures have been the focus
of this project.

\subsection{Hyperbolic LCSs}
\label{sub:hyperbolic_lcss}

The use of LCSs for reliable forecasting requires sufficiency and necessity
conditions, supported by mathematical theorems. \textcite{haller2010variational}
derived a variational LCS theory based on the Cauchy-Green strain tensor,
defined by equation~\eqref{eq:cauchygreen}, from which the aforementioned
conditions follow. The immediately relevant parts of Haller's theory
are summarized in
\cref{def:normalrepellence,def:repellinglcs,def:attractinglcs,def:hyperboliclcs}.\\

\begin{defn}
    \label{def:normalrepellence}
    A \emph{normally repelling material line} over the time interval
    $[t_{0},t_{0}+T]$ is a compact material line segment $\mathcal{M}(t)$
    which is overall repelling, and on which the normal repulsion rate
    is greater than the tangential repulsion rate.
\end{defn}

A \emph{material line} is a smooth curve $\mathcal{M}(t_{0})$ at time $t_{0}$,
which is advected by the flow map, cf.\ equation~\eqref{eq:flowmap}, into
the dynamic material line
$\mathcal{M}(t)=\vct{F}_{t_{0}}^{t}\mathcal{M}(t_{0})$. The required
\emph{compactness} of the material line segment signifies that, in some sense,
it must be topologically well-behaved. That the material line is
\emph{overall repelling} means that nearby trajectories are repelled from,
rather than attracted to, the material line. Lastly, requiring that the
\emph{normal repulsion rate} is greater than the
\emph{tangential repulsion rate} means that nearby trajectories are in fact
driven away from the material line, rather than being stretched \emph{on}
it, due to shear stress.
\\
\begin{defn}
    \label{def:repellinglcs}
    A \emph{repelling LCS} over the time interval $[t_{0},t_{0}+T]$ is a
    normally repelling material line $\mathcal{M}(t_{0})$ whose normal repulsion
    admits a pointwise non-degenerate maximum relative to any nearby material
    line $\widehat{\mathcal{M}}(t_{0})$.\\
\end{defn}
\begin{defn}
    \label{def:attractinglcs}
    An \emph{attracting LCS}  over the time interval $[t_{0},t_{0}+T]$ is
    defined as a repelling LCS over the \emph{backward} time interval
    $[t_{0}+T,t_{0}]$.\\
\end{defn}
\begin{defn}
    \label{def:hyperboliclcs}
    A \emph{hyperbolic} LCS over the time interval $[t_{0},t_{0}+T]$ is a
    \emph{repelling} or \emph{attracting} LCS over the same time interval.\\
\end{defn}

Note that the above definitions associate LCSs with the time interval $I$ over
which the dynamical system under consideration is known, or, at the very least,
over which information regarding the behaviour of tracers, is sought. Generally,
LCSs obtained over a time interval $I$ do not exist over different time
intervals \parencite{farazmand2012computing}.

For two-dimensional flow, the above definitions can be summarized as a set of
mathematical existence criteria, based on the Cauchy-Green strain tensor,
cf.\ equation~\eqref{eq:cauchygreen}
\parencite{haller2010variational,farazmand2011erratum}. These are given in
\cref{thm:lcsexistence}:\\


\begin{thm}[Sufficient and necessary conditions for LCSs in two-dimensional
    flows]
    \label{thm:lcsexistence}
    Consider a compact material line $\mathcal{M}(t)\subset\mathcal{U}$ evolving
    over the time interval $[t_{0},t_{0}+T]$. $\mathcal{M}(t)$ is a repelling
    LCS over $[t_{0},t_{0}+T]$ if and only if all the following hold for all
    initial conditions $\vct{x}_{0}\in\mathcal{M}(t_{0})$:
    \begin{subequations}
        \label{eq:lcsexistence}
        \begin{align}
            \label{eq:lcsexistence1}
            &\lambda_{1}(\vct{x}_{0})\neq\lambda_{2}(\vct{x}_{0})>1\\
            \label{eq:lcsexistence2}
            &\big\langle\vct{\xi}_{2}(\vct{x}_{0}),%
            \mtrx{H}_{\lambda_{2}}(\vct{x}_{0})\vct{\xi}_{2}(\vct{x}_{0})\big\rangle<0\\
            \label{eq:lcsexistence3}
            &\vct{\xi}_{2}(\vct{x}_{0})\perp\mathcal{M}(t_{0})\\
            \label{eq:lcsexistence4}
            &\big\langle\vct{\nabla}\lambda_{2}(\vct{x}_{0}),%
                \vct{\xi}_{2}(\vct{x}_{0})\big\rangle=0
        \end{align}
    \end{subequations}
\end{thm}

In \cref{thm:lcsexistence}, $\langle\cdot,\cdot\rangle$ denotes the Euclidean
inner product, and $\mtrx{H}_{\lambda_{2}}$ denotes the Hessian matrix of the
set of largest eigenvalues of the Cauchy-Green strain tensor. Component-wise,
the Hessian matrix of a general, smooth, scalar-valued function $f$ is defined
as

\begin{equation}
    \label{eq:hessiancomponent}
    {(\mtrx{H}_{f})}_{i,j} = \pdv[2]{f}{x_{i}}{x_{j}},
\end{equation}

which, for our two-dimensional flow, reduces to

\begin{equation}
    \label{eq:hessiantotal}
    \renewcommand{\arraystretch}{2.5}
    \mtrx{H}_{f} = \begin{pmatrix}%
    \dfrac{\partial{}^{2}f}{\partial{x}^{2}} & %
                \dfrac{\partial{}^{2}f}{\partial{x}\partial{y}} \\
                \dfrac{\partial{}^{2}f}{\partial{y}\partial{x}} & %
                \dfrac{\partial{}^{2}f}{\partial{y}^{2}} \\
            \end{pmatrix}
\end{equation}

Condition~\eqref{eq:lcsexistence1} ensures that the normal repulsion rate is
larger than the tangential stretch due to shear strain along the LCS, cf.
\cref{def:normalrepellence}. Conditions~\eqref{eq:lcsexistence3} and
\eqref{eq:lcsexistence4} suffice to ensure that the normal repulsion rate
attains a local extremum along the LCS, relative to all nearby material lines.
Lastly, condition~\eqref{eq:lcsexistence2} forces this to be a strict
local apex.

\section{FTLE fields as predictors for LCSs}
\label{sec:ftle_fields_as_predictors_for_lcss}

Finite-time Lyapunov exponent (hereafter abbreviated to FTLE) fields provide
a measure of the extent to which particles which start out close to eachother,
are separated in a given time interval. Mathematically, Lyapunov exponents
quantify the asymptotic divergence or convergence of trajectories of
tracers which start out infinitesimally close to each other
\parencite[pp.328--330]{strogatz2014nonlinear}. The FTLE field
is intrinsically linked to the Cauchy-Green strain tensor, cf.\ equation
\eqref{eq:cauchygreen}. When a two-dimensional system is evolved from time
$t_{0}$ to $t_{0}+T$, the FTLE field is defined as

\begin{equation}
    \label{eq:ftledef}
    \sigma(\vct{x}_{0}) = \frac{\ln\big(\lambda_{2}(\vct{x}_{0})\big)}{2\abs{T}},
\end{equation}

where $\lambda_{2}(\vct{x}_{0})$ is the largest eigenvalue of the Cauchy-Green
strain tensor, cf.\ equation~\eqref{eq:cauchygreencharacteristics}. The
absolute value of the integration time $T$ is taken because one generally can
consider the evolution of a system in either temporal direction. As per
\cref{def:attractinglcs}, attracting LCSs are identified as repelling LCSs
in backwards time.

\textcite{shadden2005definition} in fact \emph{define} hyperbolic LCSs as ridges
of the FTLE field. By \emph{ridges}, they mean gradient lines which are
orthogonal to the direction of minimum curvature.
\textcite{haller2010variational} showed, by means of explicit examples, that the
sole use of the FTLE field for LCS detection is prone to both false positives
and false negatives, even for conceptually simple flows. Furthermore, the FTLE
field is generally less well-resolved than the LCSs obtained from Haller's
variational formalism. For these reasons, the FTLE field can generally be used
as a first-order approximation in terms of where one can reasonably expect LCSs
to be found, but it does not represent the blueprint for the actual LCSs of a
system in general.
