\section{The type of flow systems considered}
\label{sec:typeofflow}

We consider flow in two-dimensional dynamical systems of the form

\begin{equation}
    \label{eq:odeflowsystem}
\dot{\vct{x}} = \vct{v}(t,\vct{x}),\quad\vct{x}\in\mathcal{U},\quad{}t\in[t_{0},t_{1}],
\end{equation}

i.e., systems defined for the finite time interval $[t_{0},t_{1}]$, on an open,
bounded subset $\mathcal{U}$ of $\mathbb{R}^{2}$. In addition, the velocity
field $\vct{v}$ is assumed to be smooth in its arguments. Depending on the
exact nature of the velocity field $\vct{v}$, analytical particle trajectories,
i.e., analytical solutions of system~\eqref{eq:odeflowsystem}, may or may not be
computed. The flow particles are assumed to be infinitesimall and massless,
i.e., non-interacting \emph{tracers} of the overall circulation.

Letting $\vct{x}(t;t_{0},\vct{x}_{0})$ denote the trajectory of a tracer in the
system defined by~\eqref{eq:odeflowsystem}, the flow map is defined as

\begin{equation}
    \label{eq:flowmap}
    \vct{F}_{t_{0}}^{t}(\vct{x}_{0}) = \vct{x}(t;t_{0},\vct{x}_{0}),
\end{equation}

i.e., the flow map describes the mathematical movement of the tracers from
one point in time (e.g.\ the initial condition) to another. Generally, the
flow map is as smooth as the velocity field $\vct{v}$ in system
\eqref{eq:odeflowsystem} \parencite{farazmand2012computing}. For sufficiently
smooth velocity fields, the right Cauchy-Green strain tensor field is defined as

\begin{equation}
    \label{eq:cauchygreen}
    \mtrx{C}_{t_{0}}^{t}(\vct{x}_{0}) = %
    {\big(\vct{\nabla}\vct{F}_{t_{0}}^{t}(\vct{x}_{0})\big)}^{\mathsf{T}} %
    \vct{\nabla}\vct{F}_{t_{0}}^{t}(\vct{x}_{0}),
\end{equation}

where $\vct{\nabla}\vct{F}_{t_{0}}^{t}$ denotes the Jacobian matrix of the flow
map $\vct{F}_{t_{0}}^{t}$, and the $\mathsf{T}$ refers to matrix
transposition. Component-wise, the Jacobian matrix of a general vector-valued
function $\vct{f}$ is defined as

\begin{equation}
    \label{eq:jacobiancomponent}
    {(\vct{\nabla}\vct{f})}_{i,j} = \pdv{f_{i}}{x_{j}}
\end{equation}

which, for our two-dimensional flow, reduces to

\begin{equation}
    \label{eq:jacobiantotal}
    \renewcommand{\arraystretch}{2.5}
    \vct{\nabla}\vct{f} = \begin{pmatrix} %
        \dfrac{\partial{}f_{1}}{\partial{}x_{1}} %
                                    & \dfrac{\partial{}f_{1}}{\partial{}x_{2}}\\
        \dfrac{\partial{}f_{2}}{\partial{}x_{1}} %
                                    & \dfrac{\partial{}f_{2}}{\partial{}x_{2}}%
                        \end{pmatrix}.
\end{equation}

By construction, the Cauchy-Green strain tensor $\mtrx{C}_{t_{0}}^{t}$ is
symmetric and positive definite. Thus, it has two real, positive eigenvalues
and orthogonal, real eigenvectors. Its eigenvalues $\lambda_{i}$ and
corresponding unit eigenvectors $\vct{\xi}_{i}$ are defined by

\begin{equation}
    \label{eq:cauchygreencharacteristics}
    \begin{gathered}
        \mtrx{C}_{t_{0}}^{t}(\vct{x}_{0})\vct{\xi}_{i}(\vct{x}_{0}) %
            = \lambda_{i}\vct{\xi}_{i}(\vct{x}_{0}),%
            \quad\norm{\vct{\xi}_{i}(\vct{x}_{0})}=1,\quad{}i=1,2,\\
        0<\lambda_{1}(\vct{x}_{0})\leq\lambda_{2}(\vct{x}_{0}),
    \end{gathered}
\end{equation}

where, for the sake of notational transparency, the dependence of $\lambda_{i}$
and $\vct{\xi}_{i}$ on $t_{0}$ and $t$ has been suppressed. If the flow is
incompressible, the eigenvalues satisfy

\begin{equation}
    \label{eq:cauchygreenincomprlambda}
    \lambda_{1}(\vct{x}_{0})\lambda_{2}(\vct{x}_{0})=1%
        \quad\forall\hskip0.5em\vct{x}_{0}\in\mathcal{U}
\end{equation}


where, for our case, incompressibility is equivalent to the velocity field
$\vct{v}$ being divergence-free (i.e., $\vct{\nabla}\vdot\vct{v}\equiv0$),
because the tracer particles are massless~\parencite{farazmand2012computing}.
