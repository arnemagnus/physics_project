\subsection{The Runge-Kutta family of numerical ODE solvers}
\label{sub:the_runge_kutta_family_of_numerical_methods}

In numerical analysis, the Runge-Kutta family of methods is a popular
collection of implicit and explicit iterative methods, used in temporal
discretization in order to obtain numerical approximations of the \emph{true}
solutions of systems like \eqref{eq:ivpsystem}. The German mathematicians C.
Runge and M. W. Kutta developed the first of the family's methods at the turn
of the twentieth century \parencite[p.134]{hairer1993solving}. The general
scheme of what is now known as a Runge-Kutta method is as follows: \\

\begin{defn}
    \label{def:generalrungekutta}
    Let $s$ be an integer and $a_{1,1},a_{1,2},\ldots,a_{1,s},a_{2,1},
    a_{2,2},\ldots,a_{2,s},\ldots,a_{s,1},a_{s,2},\ldots,a_{s,s}$,
    $b_{1},b_{2},\ldots,b_{s}$ and $c_{1},c_{2},\ldots,c_{s}$ be real
    coefficients. Let $h$ be the numerical step length used in the
    temporal discretization. Then, the method
\begin{equation}
    \label{eq:generalrungekutta}
    \begin{aligned}
        k_{i} &= f\bigg(t_{n}+c_{i}h,x_{n}+
                h\sum\limits_{j=1}^{s}a_{i,j}k_{j}\bigg),\quad{}i=1,\ldots,s\\
        x_{n+1} &= x_{n} + h\sum\limits_{i=1}^{s}b_{i}k_{i}
    \end{aligned}
\end{equation}
is called an \emph{s-stage Runge-Kutta method} for the system
\eqref{eq:ivpsystem}.
\end{defn}

The main reason to include multiple stages in a Runge-Kutta method,
is to improve the numerical accuracy of the computed solutions.
The \emph{order} of a Runge-Kutta method can be defined as follows:\\

\begin{defn}
    \label{def:rungekuttaorder}
    A Runge-Kutta method, given by~\cref{eq:generalrungekutta}, is
    said to be of \emph{order} $p$ if, for sufficiently smooth systems
    \eqref{eq:ivpsystem}, the local error $e_{n}$ scales as $h^{p+1}$, that is:
    \begin{equation}
        \label{eq:rungekuttaorder}
        e_{n}=\norm{x_{n}-u_{n-1}(t_{n})} \leq Kh^{p+1},
    \end{equation}
    where $u_{n-1}(t)$ is the exact solution of the ODE in system
    \eqref{eq:ivpsystem} at time $t$, subject to the initial condition
    $u_{n-1}(t_{n-1})=x_{n-1}$, and $K$ is a numerical constant. This is true,
    if the Taylor series for the exact solution $u_{n-1}(t_{n})$ and the
    numerical solution $x_{n}$ coincide up to (and including) the term $h^p$.
\end{defn}

The \emph{global} error
\begin{equation}
    \label{eq:rungekuttaglobalorderdef}
    E_{n} = x_{n}-x(t_{n}),
\end{equation}
where $x(t)$ is the exact solution of system~\eqref{eq:ivpsystem} at time $t$,
accumulated by $n$ repeated applications of the numerical method, can be
estimated by
\begin{equation}
    \label{eq:rungekuttaglobalorderapprox}
    \abs{E_{n}} \leq C\sum\limits_{l=1}^{n}\abs{e_{l}},
\end{equation}
where $C$ is a numerical constant, depending on both the right hand side of
the ODE in system~\eqref{eq:ivpsystem} and the difference $t_{n}-t_{0}$.
Making use of~\cref{def:rungekuttaorder}, the global error can be estimated
by
\begin{gather}
    \label{eq:rungekuttaglobalorderestimate}
    \begin{aligned}
        \abs{E_{n}}&\leq C\sum\limits_{l=1}^{n}\abs{e_{l}} %
        \leq C\sum\limits_{l=1}^{n}\abs{K_{l}}\hspace{0.5ex}h^{p+1} \\
        &\leq C\hspace{0.5ex}\max\limits_{l}\big\{\abs{K_{l}}\big\}\hspace{0.5ex}n\hspace{0.5ex}h^{p+1}%
        \leq C\hspace{0.5ex}\max\limits_{l}\big\{\abs{K_{l}}\big\}\hspace{0.5ex}\frac{t_{n}-t_{0}}{h}\hspace{0.5ex}h^{p+1}\\
        &\leq \widetilde{K}h^{p},
    \end{aligned}
\end{gather}
where $\widetilde{K}$ is a numerical constant.
\Cref{eq:rungekuttaglobalorderestimate} demonstrates that, for a $p$-th
order Runge-Kutta method, the global error can be expected to scale
as $h^{p}$.

%It is easy to show that if the local error of a Runge-Kutta method is of order
%$p+1$, the global error, i.e., the total accumulated error resulting of
%applying the algorithm a number of times, is expected to scale as $h^{p}$.
%Showing this is left as an exercise for the interested reader.
%
In definition~\ref{def:generalrungekutta}, the matrix $(a_{i,j})$ is commonly
called the \emph{Runge-Kutta matrix}, while $b_{i}$ and $c_{i}$ are known as
the \emph{weights} and \emph{nodes}, respectively.  Since the 1960s, it has
been customary to represent Runge-Kutta methods, given by
\cref{eq:generalrungekutta}, symbolically, by means of mnemonic devices known
as Butcher tableaus \parencite[p.134]{hairer1993solving}. The Butcher tableau
for a general \emph{s}-stage Runge-Kutta method, introduced in definition
\ref{def:generalrungekutta}, is presented in table~\ref{tab:generalbutcher}.

\clearpage

\begin{table}[htpb]
    \centering
    \caption[Butcher tableau representation of a general $s$-stage
                Runge-Kutta method]{Butcher tableau representation of a general
                    $s$-stage Runge-Kutta method.}
    \label{tab:generalbutcher}
    \[\renewcommand{\arraystretch}{1.25}
        \begin{array}{c|cccc}
            \toprule
            c_{1} & a_{1,1} & a_{1,2} & \ldots & a_{1,s}\\
            c_{2} & a_{2,1} & a_{2,2} & \ldots & a_{2,s}\\
            \vdots & \vdots & \vdots & \ddots & \vdots \\
            c_{s} & a_{s,1} & a_{s,2} & \ldots & a_{s,s}\\
            \hline
            & b_{1} & b_{2} & \ldots & b_{s}\\
            \bottomrule
    \end{array}
\]
\end{table}

For explicit Runge-Kutta methods, the Runge-Kutta matrix $(a_{i,j})$ is lower
triangular. Similarly, for fully implicit Runge-Kutta methods, the Runge-Kutta
matrix is upper triangular. The difference between explicit and implicit
methods is outlined in~\cref{eq:exim}.
%Unlike explicit methods, implicit methods require
%the solution of a linear system at every time level, making them more
%computationally demanding than their explicit siblings. The main selling point
%of implicit methods is that they are more numerically stable than explicit
%methods. This property means that implicit methods are particularly well-suited
%for \emph{stiff} systems, i.e., physical systems with highly disparate time
%scales~\parencite[p.2]{hairer1996solving}. For such systems,
%most explicit methods are highly numerically unstable, unless the numerical step
%size is made exceptionally small, rendering most explicit methods practically
%useless. For \emph{nonstiff} systems, however, implicit methods behave similarly
%to their explicit analogues in terms of numerical accuracy and
%convergence properties.
%
%\clearpage

During the first half of the twentieth century, a substantial amount of research
was conducted in order to develop numerically robust, high-order, explicit
Runge-Kutta methods. The idea was that using such methods would mean one could
resort to larger time increments $h$ without sacrificing precision in the
computational solution. However, the number of stages $s$ grows quicker than
linearly as a function of the required order $p$. It has been proven
that, for $p\geq5$, no explicit Runge-Kutta method of order $p$ with $s=p$
stages exists \parencite[p.173]{hairer1993solving}. This is
one of the reasons for the attention shift from the latter half of the 1950s
and onwards, towards so-called \emph{embedded} Runge-Kutta methods.

The basic idea of embedded Runge-Kutta methods is that they, aside from the
numerical approximation $x_{n+1}$, yield a second approximation
$\widehat{x}_{n+1}$. The difference between the two approximations then yields
an estimate of the local error of the less precise result, which can be used for
automatic step size control~\parencite[pp.167--168]{hairer1993solving}. The
trick is to construct two independent, explicit Runge-Kutta methods which both
use the \emph{same} function evaluations. This results in practically obtaining
the two solutions for the price of one, in terms of computational complexity.
The Butcher tableau of an embedded, general, explicit Runge-Kutta method is
illustrated in~\cref{tab:generalembeddedbutcher}.

\begin{table}[htpb]
    \centering
    \caption[Butcher tableau representation of general, embedded, explicit
    Runge-Kutta methods]{Butcher tableau representation of general, embedded,
        explicit Runge-Kutta methods.}
    \label{tab:generalembeddedbutcher}
    \[\renewcommand{\arraystretch}{1.25}
    \begin{array}{c|ccccc}
    \toprule
    0 \\
    c_{2} & a_{2,1} \\
    c_{3} & a_{3,1} & a_{3,2} \\
    \vdots & \vdots & \vdots & \ddots\\
    c_{s} & a_{s,1} & a_{s,2} & \ldots & a_{s,s-1}\\
    \hline
    & b_{1} & b_{2} & \ldots & b_{s-1} & b_{s} \\
    \hline
    & \widehat{b}_{1} & \widehat{b}_{2} & \ldots & \widehat{b}_{s-1}& \widehat{b}_{s}\\
    \bottomrule
    \end{array}
\]
\end{table}

For embedded methods, the coefficients are tuned such that
\begin{subequations}
    \begin{equation}
        \label{eq:embeddedsol}
        x_{n+1} = x_{n} + h\sum\limits_{i=1}^{s}b_{i}k_{i}
    \end{equation}
is of order $p$, and
    \begin{equation}
        \label{eq:embeddedinterp}
        \widehat{x}_{n+1} = x_{n} + h\sum\limits_{i=1}^{s}\widehat{b}_{i}k_{i}
    \end{equation}
\end{subequations}
is of order $\widehat{p}$, typically with $\widehat{p} = p \pm 1$. Which
of the solutions are used to continue the numerical integration, depends on
the integration scheme in question. In the following, the solution which is
\emph{not} used to continue the integration, will be referred to as the
\emph{interpolant} solution.

%\clearpage
\subsection{The Runge-Kutta methods under consideration}
\label{sub:the_runge_kutta_methods_under_consideration}

There exists an abundance of Runge-Kutta methods. Many of them are
fine-tuned for specific constraints, such as problems of varying degrees of
stiffness. It is neither possible nor meaningful to investigate them all
in the context of general flow dynamics. For this reason, we consider two classes
of explicit Runge-Kutta methods, namely singlestep and adaptive stepsize
methods. From both classes, we include four different general-purpose ODE solvers
of varying order.

\subsubsection{Singlestep methods}
\label{ssub:singlestep_methods}

The singlestep methods under consideration are the classical, explicit
Runge-Kutta methods of orders one through to four, i.e., the \emph{Euler},
\emph{Heun}, \emph{Kutta} and \emph{classical Runge-Kutta} methods. The
Euler method is \nth{1}-order accurate, and requires a single function
evaluation of the right hand side of the ODE of system
\eqref{eq:ivpsystem} or~\eqref{eq:ivpsystemhigherdimensions} at each time step.
Its Butcher tableau representation can be found in~\cref{tab:butchereuler}.
It is the simplest explicit method for numerical integration of ordinary
differential equations. The Euler method is often used as a basis to construct
more complex methods, such as the Heun method, which is also known as the
\emph{improved Euler method} or the \emph{explicit trapezoidal rule}. The Heun
method is \nth{2}-order accurate, and requires two function evaluations at each
time step. Its Butcher tableau representation can be found in
\cref{tab:butcherrk2}.

\begin{figure}[htpb]
    \centering
    \includegraphics[width=0.8\linewidth]{figures/lcs_figures/euler.pdf}
    \caption[LCS curves found by means of the Euler integration scheme]{
        LCS curves found by means of the Euler integration scheme. The
        reference LCS, as shown by itself in figure~\ref{fig:referencelcs},
        is plotted on the bottom layer. There is a clearly visible offset
        compared to the reference, for all but the two smallest numerical step
        lengths considered.}
    \label{fig:lcs_euler}
\end{figure}

\clearpage
\begin{figure}[htpb]
    \centering
    \includegraphics[width=0.8\linewidth]{figures/lcs_figures/rk2.pdf}
    \caption[LCS curves found by means of the Heun integration scheme]{
        LCS curves found by means of the Heun integration scheme. The
        reference LCS, as shown by itself in figure~\ref{fig:referencelcs},
        is plotted on the bottom layer. There are clear discrepancies with
        regards to the reference for the two largest numerical time step
        lengths considered.}
    \label{fig:lcs_rk2}
\end{figure}


The Kutta method is \nth{3}-order accurate, and requires three function
evaluations of the right hand side of the ordinary differential
\cref{eq:ivpsystem} or ~\eqref{eq:ivpsystemhigherdimensions} at each time
step. Its Butcher tableau representation can be found in \cref{tab:butcherrk3}.
The classical Runge-Kutta method is \nth{4}-order accurate, and perhaps the most
well-known and frequently used of the four singlestep schemes discussed in this
project. One reason for its popularity is that it is exceptionally stable
numerically (of the aforementioned singlestep methods, the classical
Runge-Kutta method has the largest numerical stability domain). Another is that,
as mentioned previously, for $p\geq5$, no explicit Runge-Kutta method of order
$p$ with $s=p$ stages exist
\parencite[p.173]{hairer1993solving} -- in other words,
the required number of function evaluations grows at a disproportional rate with
the required accuracy order. For systems with right hand sides which are
computationally costly to evaluate, this means that one frequently is able to
obtain the desired numerical accuracy more effectively by using, for instance,
the classical Runge-Kutta method with a finer step length. The Butcher tableau
representation of the classical Runge-Kutta method can be found in
\cref{tab:butcherrk4}.

\begin{figure}[htpb]
    \centering
    %% Creator: Matplotlib, PGF backend
%%
%% To include the figure in your LaTeX document, write
%%   \input{<filename>.pgf}
%%
%% Make sure the required packages are loaded in your preamble
%%   \usepackage{pgf}
%%
%% Figures using additional raster images can only be included by \input if
%% they are in the same directory as the main LaTeX file. For loading figures
%% from other directories you can use the `import` package
%%   \usepackage{import}
%% and then include the figures with
%%   \import{<path to file>}{<filename>.pgf}
%%
%% Matplotlib used the following preamble
%%   \usepackage[utf8x]{inputenc}
%%   \usepackage[T1]{fontenc}
%%   \usepackage[]{libertine}\usepackage[libertine]{newtxmath}
%%
\begingroup%
\makeatletter%
\begin{pgfpicture}%
\pgfpathrectangle{\pgfpointorigin}{\pgfqpoint{5.050000in}{3.100000in}}%
\pgfusepath{use as bounding box, clip}%
\begin{pgfscope}%
\pgfsetbuttcap%
\pgfsetmiterjoin%
\definecolor{currentfill}{rgb}{1.000000,1.000000,1.000000}%
\pgfsetfillcolor{currentfill}%
\pgfsetlinewidth{0.000000pt}%
\definecolor{currentstroke}{rgb}{1.000000,1.000000,1.000000}%
\pgfsetstrokecolor{currentstroke}%
\pgfsetdash{}{0pt}%
\pgfpathmoveto{\pgfqpoint{0.000000in}{0.000000in}}%
\pgfpathlineto{\pgfqpoint{5.050000in}{0.000000in}}%
\pgfpathlineto{\pgfqpoint{5.050000in}{3.100000in}}%
\pgfpathlineto{\pgfqpoint{0.000000in}{3.100000in}}%
\pgfpathclose%
\pgfusepath{fill}%
\end{pgfscope}%
\begin{pgfscope}%
\pgfsetbuttcap%
\pgfsetmiterjoin%
\definecolor{currentfill}{rgb}{1.000000,1.000000,1.000000}%
\pgfsetfillcolor{currentfill}%
\pgfsetlinewidth{0.000000pt}%
\definecolor{currentstroke}{rgb}{0.000000,0.000000,0.000000}%
\pgfsetstrokecolor{currentstroke}%
\pgfsetstrokeopacity{0.000000}%
\pgfsetdash{}{0pt}%
\pgfpathmoveto{\pgfqpoint{0.448634in}{0.402556in}}%
\pgfpathlineto{\pgfqpoint{4.799294in}{0.402556in}}%
\pgfpathlineto{\pgfqpoint{4.799294in}{2.891760in}}%
\pgfpathlineto{\pgfqpoint{0.448634in}{2.891760in}}%
\pgfpathclose%
\pgfusepath{fill}%
\end{pgfscope}%
\begin{pgfscope}%
\pgfsetbuttcap%
\pgfsetroundjoin%
\definecolor{currentfill}{rgb}{0.000000,0.000000,0.000000}%
\pgfsetfillcolor{currentfill}%
\pgfsetlinewidth{0.803000pt}%
\definecolor{currentstroke}{rgb}{0.000000,0.000000,0.000000}%
\pgfsetstrokecolor{currentstroke}%
\pgfsetdash{}{0pt}%
\pgfsys@defobject{currentmarker}{\pgfqpoint{0.000000in}{-0.048611in}}{\pgfqpoint{0.000000in}{0.000000in}}{%
\pgfpathmoveto{\pgfqpoint{0.000000in}{0.000000in}}%
\pgfpathlineto{\pgfqpoint{0.000000in}{-0.048611in}}%
\pgfusepath{stroke,fill}%
}%
\begin{pgfscope}%
\pgfsys@transformshift{0.448634in}{0.402556in}%
\pgfsys@useobject{currentmarker}{}%
\end{pgfscope}%
\end{pgfscope}%
\begin{pgfscope}%
\pgftext[x=0.448634in,y=0.305334in,,top]{\rmfamily\fontsize{12.000000}{14.400000}\selectfont \(\displaystyle 0.00\)}%
\end{pgfscope}%
\begin{pgfscope}%
\pgfsetbuttcap%
\pgfsetroundjoin%
\definecolor{currentfill}{rgb}{0.000000,0.000000,0.000000}%
\pgfsetfillcolor{currentfill}%
\pgfsetlinewidth{0.803000pt}%
\definecolor{currentstroke}{rgb}{0.000000,0.000000,0.000000}%
\pgfsetstrokecolor{currentstroke}%
\pgfsetdash{}{0pt}%
\pgfsys@defobject{currentmarker}{\pgfqpoint{0.000000in}{-0.048611in}}{\pgfqpoint{0.000000in}{0.000000in}}{%
\pgfpathmoveto{\pgfqpoint{0.000000in}{0.000000in}}%
\pgfpathlineto{\pgfqpoint{0.000000in}{-0.048611in}}%
\pgfusepath{stroke,fill}%
}%
\begin{pgfscope}%
\pgfsys@transformshift{0.992466in}{0.402556in}%
\pgfsys@useobject{currentmarker}{}%
\end{pgfscope}%
\end{pgfscope}%
\begin{pgfscope}%
\pgftext[x=0.992466in,y=0.305334in,,top]{\rmfamily\fontsize{12.000000}{14.400000}\selectfont \(\displaystyle 0.25\)}%
\end{pgfscope}%
\begin{pgfscope}%
\pgfsetbuttcap%
\pgfsetroundjoin%
\definecolor{currentfill}{rgb}{0.000000,0.000000,0.000000}%
\pgfsetfillcolor{currentfill}%
\pgfsetlinewidth{0.803000pt}%
\definecolor{currentstroke}{rgb}{0.000000,0.000000,0.000000}%
\pgfsetstrokecolor{currentstroke}%
\pgfsetdash{}{0pt}%
\pgfsys@defobject{currentmarker}{\pgfqpoint{0.000000in}{-0.048611in}}{\pgfqpoint{0.000000in}{0.000000in}}{%
\pgfpathmoveto{\pgfqpoint{0.000000in}{0.000000in}}%
\pgfpathlineto{\pgfqpoint{0.000000in}{-0.048611in}}%
\pgfusepath{stroke,fill}%
}%
\begin{pgfscope}%
\pgfsys@transformshift{1.536299in}{0.402556in}%
\pgfsys@useobject{currentmarker}{}%
\end{pgfscope}%
\end{pgfscope}%
\begin{pgfscope}%
\pgftext[x=1.536299in,y=0.305334in,,top]{\rmfamily\fontsize{12.000000}{14.400000}\selectfont \(\displaystyle 0.50\)}%
\end{pgfscope}%
\begin{pgfscope}%
\pgfsetbuttcap%
\pgfsetroundjoin%
\definecolor{currentfill}{rgb}{0.000000,0.000000,0.000000}%
\pgfsetfillcolor{currentfill}%
\pgfsetlinewidth{0.803000pt}%
\definecolor{currentstroke}{rgb}{0.000000,0.000000,0.000000}%
\pgfsetstrokecolor{currentstroke}%
\pgfsetdash{}{0pt}%
\pgfsys@defobject{currentmarker}{\pgfqpoint{0.000000in}{-0.048611in}}{\pgfqpoint{0.000000in}{0.000000in}}{%
\pgfpathmoveto{\pgfqpoint{0.000000in}{0.000000in}}%
\pgfpathlineto{\pgfqpoint{0.000000in}{-0.048611in}}%
\pgfusepath{stroke,fill}%
}%
\begin{pgfscope}%
\pgfsys@transformshift{2.080131in}{0.402556in}%
\pgfsys@useobject{currentmarker}{}%
\end{pgfscope}%
\end{pgfscope}%
\begin{pgfscope}%
\pgftext[x=2.080131in,y=0.305334in,,top]{\rmfamily\fontsize{12.000000}{14.400000}\selectfont \(\displaystyle 0.75\)}%
\end{pgfscope}%
\begin{pgfscope}%
\pgfsetbuttcap%
\pgfsetroundjoin%
\definecolor{currentfill}{rgb}{0.000000,0.000000,0.000000}%
\pgfsetfillcolor{currentfill}%
\pgfsetlinewidth{0.803000pt}%
\definecolor{currentstroke}{rgb}{0.000000,0.000000,0.000000}%
\pgfsetstrokecolor{currentstroke}%
\pgfsetdash{}{0pt}%
\pgfsys@defobject{currentmarker}{\pgfqpoint{0.000000in}{-0.048611in}}{\pgfqpoint{0.000000in}{0.000000in}}{%
\pgfpathmoveto{\pgfqpoint{0.000000in}{0.000000in}}%
\pgfpathlineto{\pgfqpoint{0.000000in}{-0.048611in}}%
\pgfusepath{stroke,fill}%
}%
\begin{pgfscope}%
\pgfsys@transformshift{2.623964in}{0.402556in}%
\pgfsys@useobject{currentmarker}{}%
\end{pgfscope}%
\end{pgfscope}%
\begin{pgfscope}%
\pgftext[x=2.623964in,y=0.305334in,,top]{\rmfamily\fontsize{12.000000}{14.400000}\selectfont \(\displaystyle 1.00\)}%
\end{pgfscope}%
\begin{pgfscope}%
\pgfsetbuttcap%
\pgfsetroundjoin%
\definecolor{currentfill}{rgb}{0.000000,0.000000,0.000000}%
\pgfsetfillcolor{currentfill}%
\pgfsetlinewidth{0.803000pt}%
\definecolor{currentstroke}{rgb}{0.000000,0.000000,0.000000}%
\pgfsetstrokecolor{currentstroke}%
\pgfsetdash{}{0pt}%
\pgfsys@defobject{currentmarker}{\pgfqpoint{0.000000in}{-0.048611in}}{\pgfqpoint{0.000000in}{0.000000in}}{%
\pgfpathmoveto{\pgfqpoint{0.000000in}{0.000000in}}%
\pgfpathlineto{\pgfqpoint{0.000000in}{-0.048611in}}%
\pgfusepath{stroke,fill}%
}%
\begin{pgfscope}%
\pgfsys@transformshift{3.167797in}{0.402556in}%
\pgfsys@useobject{currentmarker}{}%
\end{pgfscope}%
\end{pgfscope}%
\begin{pgfscope}%
\pgftext[x=3.167797in,y=0.305334in,,top]{\rmfamily\fontsize{12.000000}{14.400000}\selectfont \(\displaystyle 1.25\)}%
\end{pgfscope}%
\begin{pgfscope}%
\pgfsetbuttcap%
\pgfsetroundjoin%
\definecolor{currentfill}{rgb}{0.000000,0.000000,0.000000}%
\pgfsetfillcolor{currentfill}%
\pgfsetlinewidth{0.803000pt}%
\definecolor{currentstroke}{rgb}{0.000000,0.000000,0.000000}%
\pgfsetstrokecolor{currentstroke}%
\pgfsetdash{}{0pt}%
\pgfsys@defobject{currentmarker}{\pgfqpoint{0.000000in}{-0.048611in}}{\pgfqpoint{0.000000in}{0.000000in}}{%
\pgfpathmoveto{\pgfqpoint{0.000000in}{0.000000in}}%
\pgfpathlineto{\pgfqpoint{0.000000in}{-0.048611in}}%
\pgfusepath{stroke,fill}%
}%
\begin{pgfscope}%
\pgfsys@transformshift{3.711629in}{0.402556in}%
\pgfsys@useobject{currentmarker}{}%
\end{pgfscope}%
\end{pgfscope}%
\begin{pgfscope}%
\pgftext[x=3.711629in,y=0.305334in,,top]{\rmfamily\fontsize{12.000000}{14.400000}\selectfont \(\displaystyle 1.50\)}%
\end{pgfscope}%
\begin{pgfscope}%
\pgfsetbuttcap%
\pgfsetroundjoin%
\definecolor{currentfill}{rgb}{0.000000,0.000000,0.000000}%
\pgfsetfillcolor{currentfill}%
\pgfsetlinewidth{0.803000pt}%
\definecolor{currentstroke}{rgb}{0.000000,0.000000,0.000000}%
\pgfsetstrokecolor{currentstroke}%
\pgfsetdash{}{0pt}%
\pgfsys@defobject{currentmarker}{\pgfqpoint{0.000000in}{-0.048611in}}{\pgfqpoint{0.000000in}{0.000000in}}{%
\pgfpathmoveto{\pgfqpoint{0.000000in}{0.000000in}}%
\pgfpathlineto{\pgfqpoint{0.000000in}{-0.048611in}}%
\pgfusepath{stroke,fill}%
}%
\begin{pgfscope}%
\pgfsys@transformshift{4.255462in}{0.402556in}%
\pgfsys@useobject{currentmarker}{}%
\end{pgfscope}%
\end{pgfscope}%
\begin{pgfscope}%
\pgftext[x=4.255462in,y=0.305334in,,top]{\rmfamily\fontsize{12.000000}{14.400000}\selectfont \(\displaystyle 1.75\)}%
\end{pgfscope}%
\begin{pgfscope}%
\pgfsetbuttcap%
\pgfsetroundjoin%
\definecolor{currentfill}{rgb}{0.000000,0.000000,0.000000}%
\pgfsetfillcolor{currentfill}%
\pgfsetlinewidth{0.803000pt}%
\definecolor{currentstroke}{rgb}{0.000000,0.000000,0.000000}%
\pgfsetstrokecolor{currentstroke}%
\pgfsetdash{}{0pt}%
\pgfsys@defobject{currentmarker}{\pgfqpoint{0.000000in}{-0.048611in}}{\pgfqpoint{0.000000in}{0.000000in}}{%
\pgfpathmoveto{\pgfqpoint{0.000000in}{0.000000in}}%
\pgfpathlineto{\pgfqpoint{0.000000in}{-0.048611in}}%
\pgfusepath{stroke,fill}%
}%
\begin{pgfscope}%
\pgfsys@transformshift{4.799294in}{0.402556in}%
\pgfsys@useobject{currentmarker}{}%
\end{pgfscope}%
\end{pgfscope}%
\begin{pgfscope}%
\pgftext[x=4.799294in,y=0.305334in,,top]{\rmfamily\fontsize{12.000000}{14.400000}\selectfont \(\displaystyle 2.00\)}%
\end{pgfscope}%
\begin{pgfscope}%
\pgfsetbuttcap%
\pgfsetroundjoin%
\definecolor{currentfill}{rgb}{0.000000,0.000000,0.000000}%
\pgfsetfillcolor{currentfill}%
\pgfsetlinewidth{0.803000pt}%
\definecolor{currentstroke}{rgb}{0.000000,0.000000,0.000000}%
\pgfsetstrokecolor{currentstroke}%
\pgfsetdash{}{0pt}%
\pgfsys@defobject{currentmarker}{\pgfqpoint{-0.048611in}{0.000000in}}{\pgfqpoint{0.000000in}{0.000000in}}{%
\pgfpathmoveto{\pgfqpoint{0.000000in}{0.000000in}}%
\pgfpathlineto{\pgfqpoint{-0.048611in}{0.000000in}}%
\pgfusepath{stroke,fill}%
}%
\begin{pgfscope}%
\pgfsys@transformshift{0.448634in}{0.402556in}%
\pgfsys@useobject{currentmarker}{}%
\end{pgfscope}%
\end{pgfscope}%
\begin{pgfscope}%
\pgftext[x=0.149245in,y=0.345015in,left,base]{\rmfamily\fontsize{12.000000}{14.400000}\selectfont \(\displaystyle 0.0\)}%
\end{pgfscope}%
\begin{pgfscope}%
\pgfsetbuttcap%
\pgfsetroundjoin%
\definecolor{currentfill}{rgb}{0.000000,0.000000,0.000000}%
\pgfsetfillcolor{currentfill}%
\pgfsetlinewidth{0.803000pt}%
\definecolor{currentstroke}{rgb}{0.000000,0.000000,0.000000}%
\pgfsetstrokecolor{currentstroke}%
\pgfsetdash{}{0pt}%
\pgfsys@defobject{currentmarker}{\pgfqpoint{-0.048611in}{0.000000in}}{\pgfqpoint{0.000000in}{0.000000in}}{%
\pgfpathmoveto{\pgfqpoint{0.000000in}{0.000000in}}%
\pgfpathlineto{\pgfqpoint{-0.048611in}{0.000000in}}%
\pgfusepath{stroke,fill}%
}%
\begin{pgfscope}%
\pgfsys@transformshift{0.448634in}{0.900397in}%
\pgfsys@useobject{currentmarker}{}%
\end{pgfscope}%
\end{pgfscope}%
\begin{pgfscope}%
\pgftext[x=0.149245in,y=0.842855in,left,base]{\rmfamily\fontsize{12.000000}{14.400000}\selectfont \(\displaystyle 0.2\)}%
\end{pgfscope}%
\begin{pgfscope}%
\pgfsetbuttcap%
\pgfsetroundjoin%
\definecolor{currentfill}{rgb}{0.000000,0.000000,0.000000}%
\pgfsetfillcolor{currentfill}%
\pgfsetlinewidth{0.803000pt}%
\definecolor{currentstroke}{rgb}{0.000000,0.000000,0.000000}%
\pgfsetstrokecolor{currentstroke}%
\pgfsetdash{}{0pt}%
\pgfsys@defobject{currentmarker}{\pgfqpoint{-0.048611in}{0.000000in}}{\pgfqpoint{0.000000in}{0.000000in}}{%
\pgfpathmoveto{\pgfqpoint{0.000000in}{0.000000in}}%
\pgfpathlineto{\pgfqpoint{-0.048611in}{0.000000in}}%
\pgfusepath{stroke,fill}%
}%
\begin{pgfscope}%
\pgfsys@transformshift{0.448634in}{1.398238in}%
\pgfsys@useobject{currentmarker}{}%
\end{pgfscope}%
\end{pgfscope}%
\begin{pgfscope}%
\pgftext[x=0.149245in,y=1.340696in,left,base]{\rmfamily\fontsize{12.000000}{14.400000}\selectfont \(\displaystyle 0.4\)}%
\end{pgfscope}%
\begin{pgfscope}%
\pgfsetbuttcap%
\pgfsetroundjoin%
\definecolor{currentfill}{rgb}{0.000000,0.000000,0.000000}%
\pgfsetfillcolor{currentfill}%
\pgfsetlinewidth{0.803000pt}%
\definecolor{currentstroke}{rgb}{0.000000,0.000000,0.000000}%
\pgfsetstrokecolor{currentstroke}%
\pgfsetdash{}{0pt}%
\pgfsys@defobject{currentmarker}{\pgfqpoint{-0.048611in}{0.000000in}}{\pgfqpoint{0.000000in}{0.000000in}}{%
\pgfpathmoveto{\pgfqpoint{0.000000in}{0.000000in}}%
\pgfpathlineto{\pgfqpoint{-0.048611in}{0.000000in}}%
\pgfusepath{stroke,fill}%
}%
\begin{pgfscope}%
\pgfsys@transformshift{0.448634in}{1.896079in}%
\pgfsys@useobject{currentmarker}{}%
\end{pgfscope}%
\end{pgfscope}%
\begin{pgfscope}%
\pgftext[x=0.149245in,y=1.838537in,left,base]{\rmfamily\fontsize{12.000000}{14.400000}\selectfont \(\displaystyle 0.6\)}%
\end{pgfscope}%
\begin{pgfscope}%
\pgfsetbuttcap%
\pgfsetroundjoin%
\definecolor{currentfill}{rgb}{0.000000,0.000000,0.000000}%
\pgfsetfillcolor{currentfill}%
\pgfsetlinewidth{0.803000pt}%
\definecolor{currentstroke}{rgb}{0.000000,0.000000,0.000000}%
\pgfsetstrokecolor{currentstroke}%
\pgfsetdash{}{0pt}%
\pgfsys@defobject{currentmarker}{\pgfqpoint{-0.048611in}{0.000000in}}{\pgfqpoint{0.000000in}{0.000000in}}{%
\pgfpathmoveto{\pgfqpoint{0.000000in}{0.000000in}}%
\pgfpathlineto{\pgfqpoint{-0.048611in}{0.000000in}}%
\pgfusepath{stroke,fill}%
}%
\begin{pgfscope}%
\pgfsys@transformshift{0.448634in}{2.393919in}%
\pgfsys@useobject{currentmarker}{}%
\end{pgfscope}%
\end{pgfscope}%
\begin{pgfscope}%
\pgftext[x=0.149245in,y=2.336378in,left,base]{\rmfamily\fontsize{12.000000}{14.400000}\selectfont \(\displaystyle 0.8\)}%
\end{pgfscope}%
\begin{pgfscope}%
\pgfsetbuttcap%
\pgfsetroundjoin%
\definecolor{currentfill}{rgb}{0.000000,0.000000,0.000000}%
\pgfsetfillcolor{currentfill}%
\pgfsetlinewidth{0.803000pt}%
\definecolor{currentstroke}{rgb}{0.000000,0.000000,0.000000}%
\pgfsetstrokecolor{currentstroke}%
\pgfsetdash{}{0pt}%
\pgfsys@defobject{currentmarker}{\pgfqpoint{-0.048611in}{0.000000in}}{\pgfqpoint{0.000000in}{0.000000in}}{%
\pgfpathmoveto{\pgfqpoint{0.000000in}{0.000000in}}%
\pgfpathlineto{\pgfqpoint{-0.048611in}{0.000000in}}%
\pgfusepath{stroke,fill}%
}%
\begin{pgfscope}%
\pgfsys@transformshift{0.448634in}{2.891760in}%
\pgfsys@useobject{currentmarker}{}%
\end{pgfscope}%
\end{pgfscope}%
\begin{pgfscope}%
\pgftext[x=0.149245in,y=2.834219in,left,base]{\rmfamily\fontsize{12.000000}{14.400000}\selectfont \(\displaystyle 1.0\)}%
\end{pgfscope}%
\begin{pgfscope}%
\pgfpathrectangle{\pgfqpoint{0.448634in}{0.402556in}}{\pgfqpoint{4.350661in}{2.489204in}} %
\pgfusepath{clip}%
\pgfsetrectcap%
\pgfsetroundjoin%
\pgfsetlinewidth{1.003750pt}%
\definecolor{currentstroke}{rgb}{1.000000,0.388235,0.278431}%
\pgfsetstrokecolor{currentstroke}%
\pgfsetdash{}{0pt}%
\pgfpathmoveto{\pgfqpoint{1.127319in}{2.572074in}}%
\pgfpathlineto{\pgfqpoint{1.159575in}{2.592758in}}%
\pgfpathlineto{\pgfqpoint{1.192763in}{2.611414in}}%
\pgfpathlineto{\pgfqpoint{1.228726in}{2.629126in}}%
\pgfpathlineto{\pgfqpoint{1.267413in}{2.645758in}}%
\pgfpathlineto{\pgfqpoint{1.310846in}{2.661945in}}%
\pgfpathlineto{\pgfqpoint{1.356920in}{2.676740in}}%
\pgfpathlineto{\pgfqpoint{1.407680in}{2.690702in}}%
\pgfpathlineto{\pgfqpoint{1.463094in}{2.703640in}}%
\pgfpathlineto{\pgfqpoint{1.525273in}{2.715813in}}%
\pgfpathlineto{\pgfqpoint{1.594199in}{2.726937in}}%
\pgfpathlineto{\pgfqpoint{1.669843in}{2.736808in}}%
\pgfpathlineto{\pgfqpoint{1.752172in}{2.745271in}}%
\pgfpathlineto{\pgfqpoint{1.843325in}{2.752344in}}%
\pgfpathlineto{\pgfqpoint{1.941103in}{2.757656in}}%
\pgfpathlineto{\pgfqpoint{2.043301in}{2.760987in}}%
\pgfpathlineto{\pgfqpoint{2.147710in}{2.762199in}}%
\pgfpathlineto{\pgfqpoint{2.249945in}{2.761215in}}%
\pgfpathlineto{\pgfqpoint{2.345620in}{2.758145in}}%
\pgfpathlineto{\pgfqpoint{2.432525in}{2.753210in}}%
\pgfpathlineto{\pgfqpoint{2.508451in}{2.746766in}}%
\pgfpathlineto{\pgfqpoint{2.573368in}{2.739156in}}%
\pgfpathlineto{\pgfqpoint{2.629410in}{2.730451in}}%
\pgfpathlineto{\pgfqpoint{2.676543in}{2.720985in}}%
\pgfpathlineto{\pgfqpoint{2.716874in}{2.710666in}}%
\pgfpathlineto{\pgfqpoint{2.750366in}{2.699848in}}%
\pgfpathlineto{\pgfqpoint{2.779059in}{2.688192in}}%
\pgfpathlineto{\pgfqpoint{2.802882in}{2.676004in}}%
\pgfpathlineto{\pgfqpoint{2.821842in}{2.663819in}}%
\pgfpathlineto{\pgfqpoint{2.837815in}{2.650886in}}%
\pgfpathlineto{\pgfqpoint{2.850736in}{2.637564in}}%
\pgfpathlineto{\pgfqpoint{2.860694in}{2.624398in}}%
\pgfpathlineto{\pgfqpoint{2.869084in}{2.609873in}}%
\pgfpathlineto{\pgfqpoint{2.875698in}{2.594192in}}%
\pgfpathlineto{\pgfqpoint{2.881035in}{2.575255in}}%
\pgfpathlineto{\pgfqpoint{2.884200in}{2.555685in}}%
\pgfpathlineto{\pgfqpoint{2.885619in}{2.533351in}}%
\pgfpathlineto{\pgfqpoint{2.885038in}{2.505987in}}%
\pgfpathlineto{\pgfqpoint{2.882112in}{2.473807in}}%
\pgfpathlineto{\pgfqpoint{2.875657in}{2.429620in}}%
\pgfpathlineto{\pgfqpoint{2.863489in}{2.363873in}}%
\pgfpathlineto{\pgfqpoint{2.821102in}{2.142619in}}%
\pgfpathlineto{\pgfqpoint{2.804859in}{2.042271in}}%
\pgfpathlineto{\pgfqpoint{2.790421in}{1.939040in}}%
\pgfpathlineto{\pgfqpoint{2.777207in}{1.828054in}}%
\pgfpathlineto{\pgfqpoint{2.765338in}{1.709349in}}%
\pgfpathlineto{\pgfqpoint{2.754471in}{1.578010in}}%
\pgfpathlineto{\pgfqpoint{2.744640in}{1.431580in}}%
\pgfpathlineto{\pgfqpoint{2.735914in}{1.267598in}}%
\pgfpathlineto{\pgfqpoint{2.728277in}{1.081114in}}%
\pgfpathlineto{\pgfqpoint{2.721437in}{0.857223in}}%
\pgfpathlineto{\pgfqpoint{2.711961in}{0.541290in}}%
\pgfpathlineto{\pgfqpoint{2.708250in}{0.491694in}}%
\pgfpathlineto{\pgfqpoint{2.703951in}{0.462246in}}%
\pgfpathlineto{\pgfqpoint{2.699504in}{0.445599in}}%
\pgfpathlineto{\pgfqpoint{2.694517in}{0.434563in}}%
\pgfpathlineto{\pgfqpoint{2.688942in}{0.426947in}}%
\pgfpathlineto{\pgfqpoint{2.681980in}{0.421009in}}%
\pgfpathlineto{\pgfqpoint{2.672064in}{0.415948in}}%
\pgfpathlineto{\pgfqpoint{2.659429in}{0.412247in}}%
\pgfpathlineto{\pgfqpoint{2.640044in}{0.409163in}}%
\pgfpathlineto{\pgfqpoint{2.607490in}{0.406692in}}%
\pgfpathlineto{\pgfqpoint{2.548779in}{0.404894in}}%
\pgfpathlineto{\pgfqpoint{2.422615in}{0.403701in}}%
\pgfpathlineto{\pgfqpoint{2.026705in}{0.403016in}}%
\pgfpathlineto{\pgfqpoint{0.623617in}{0.403253in}}%
\pgfpathlineto{\pgfqpoint{0.477880in}{0.404742in}}%
\pgfpathlineto{\pgfqpoint{0.458368in}{0.406382in}}%
\pgfpathlineto{\pgfqpoint{0.452304in}{0.408937in}}%
\pgfpathlineto{\pgfqpoint{0.450213in}{0.413215in}}%
\pgfpathlineto{\pgfqpoint{0.449165in}{0.423080in}}%
\pgfpathlineto{\pgfqpoint{0.448735in}{0.465392in}}%
\pgfpathlineto{\pgfqpoint{0.448637in}{0.983146in}}%
\pgfpathlineto{\pgfqpoint{0.448652in}{2.889876in}}%
\pgfpathlineto{\pgfqpoint{0.448652in}{2.889876in}}%
\pgfusepath{stroke}%
\end{pgfscope}%
\begin{pgfscope}%
\pgfpathrectangle{\pgfqpoint{0.448634in}{0.402556in}}{\pgfqpoint{4.350661in}{2.489204in}} %
\pgfusepath{clip}%
\pgfsetrectcap%
\pgfsetroundjoin%
\pgfsetlinewidth{1.003750pt}%
\definecolor{currentstroke}{rgb}{1.000000,0.388235,0.278431}%
\pgfsetstrokecolor{currentstroke}%
\pgfsetdash{}{0pt}%
\pgfpathmoveto{\pgfqpoint{0.448634in}{2.896245in}}%
\pgfpathlineto{\pgfqpoint{0.448593in}{0.407043in}}%
\pgfpathlineto{\pgfqpoint{0.448593in}{0.407043in}}%
\pgfusepath{stroke}%
\end{pgfscope}%
\begin{pgfscope}%
\pgfpathrectangle{\pgfqpoint{0.448634in}{0.402556in}}{\pgfqpoint{4.350661in}{2.489204in}} %
\pgfusepath{clip}%
\pgfsetrectcap%
\pgfsetroundjoin%
\pgfsetlinewidth{1.003750pt}%
\definecolor{currentstroke}{rgb}{1.000000,0.388235,0.278431}%
\pgfsetstrokecolor{currentstroke}%
\pgfsetdash{}{0pt}%
\pgfpathmoveto{\pgfqpoint{0.576853in}{1.760817in}}%
\pgfpathlineto{\pgfqpoint{0.569394in}{1.840010in}}%
\pgfpathlineto{\pgfqpoint{0.563209in}{1.929338in}}%
\pgfpathlineto{\pgfqpoint{0.558592in}{2.028764in}}%
\pgfpathlineto{\pgfqpoint{0.555985in}{2.133265in}}%
\pgfpathlineto{\pgfqpoint{0.555566in}{2.237808in}}%
\pgfpathlineto{\pgfqpoint{0.557371in}{2.337352in}}%
\pgfpathlineto{\pgfqpoint{0.561096in}{2.424366in}}%
\pgfpathlineto{\pgfqpoint{0.566403in}{2.498791in}}%
\pgfpathlineto{\pgfqpoint{0.572909in}{2.560570in}}%
\pgfpathlineto{\pgfqpoint{0.580458in}{2.612119in}}%
\pgfpathlineto{\pgfqpoint{0.589086in}{2.655816in}}%
\pgfpathlineto{\pgfqpoint{0.598406in}{2.691589in}}%
\pgfpathlineto{\pgfqpoint{0.608613in}{2.721757in}}%
\pgfpathlineto{\pgfqpoint{0.619241in}{2.746278in}}%
\pgfpathlineto{\pgfqpoint{0.630817in}{2.767339in}}%
\pgfpathlineto{\pgfqpoint{0.642975in}{2.784884in}}%
\pgfpathlineto{\pgfqpoint{0.656813in}{2.800712in}}%
\pgfpathlineto{\pgfqpoint{0.672197in}{2.814549in}}%
\pgfpathlineto{\pgfqpoint{0.688853in}{2.826301in}}%
\pgfpathlineto{\pgfqpoint{0.706461in}{2.836076in}}%
\pgfpathlineto{\pgfqpoint{0.726804in}{2.844875in}}%
\pgfpathlineto{\pgfqpoint{0.751866in}{2.853203in}}%
\pgfpathlineto{\pgfqpoint{0.781631in}{2.860547in}}%
\pgfpathlineto{\pgfqpoint{0.818168in}{2.867054in}}%
\pgfpathlineto{\pgfqpoint{0.863581in}{2.872685in}}%
\pgfpathlineto{\pgfqpoint{0.922161in}{2.877518in}}%
\pgfpathlineto{\pgfqpoint{1.000391in}{2.881567in}}%
\pgfpathlineto{\pgfqpoint{1.111294in}{2.884881in}}%
\pgfpathlineto{\pgfqpoint{1.274428in}{2.887367in}}%
\pgfpathlineto{\pgfqpoint{1.552865in}{2.889263in}}%
\pgfpathlineto{\pgfqpoint{2.107573in}{2.890457in}}%
\pgfpathlineto{\pgfqpoint{3.343161in}{2.890573in}}%
\pgfpathlineto{\pgfqpoint{4.043615in}{2.888941in}}%
\pgfpathlineto{\pgfqpoint{4.289417in}{2.886404in}}%
\pgfpathlineto{\pgfqpoint{4.413375in}{2.883093in}}%
\pgfpathlineto{\pgfqpoint{4.489424in}{2.878997in}}%
\pgfpathlineto{\pgfqpoint{4.541451in}{2.874081in}}%
\pgfpathlineto{\pgfqpoint{4.578100in}{2.868470in}}%
\pgfpathlineto{\pgfqpoint{4.605818in}{2.862092in}}%
\pgfpathlineto{\pgfqpoint{4.626725in}{2.855245in}}%
\pgfpathlineto{\pgfqpoint{4.644925in}{2.847018in}}%
\pgfpathlineto{\pgfqpoint{4.660241in}{2.837590in}}%
\pgfpathlineto{\pgfqpoint{4.672623in}{2.827468in}}%
\pgfpathlineto{\pgfqpoint{4.683751in}{2.815592in}}%
\pgfpathlineto{\pgfqpoint{4.693406in}{2.802135in}}%
\pgfpathlineto{\pgfqpoint{4.702740in}{2.785343in}}%
\pgfpathlineto{\pgfqpoint{4.711277in}{2.765194in}}%
\pgfpathlineto{\pgfqpoint{4.719482in}{2.739484in}}%
\pgfpathlineto{\pgfqpoint{4.726293in}{2.710657in}}%
\pgfpathlineto{\pgfqpoint{4.733259in}{2.671643in}}%
\pgfpathlineto{\pgfqpoint{4.739604in}{2.622396in}}%
\pgfpathlineto{\pgfqpoint{4.745236in}{2.560504in}}%
\pgfpathlineto{\pgfqpoint{4.750164in}{2.481052in}}%
\pgfpathlineto{\pgfqpoint{4.754367in}{2.376618in}}%
\pgfpathlineto{\pgfqpoint{4.757443in}{2.242249in}}%
\pgfpathlineto{\pgfqpoint{4.758977in}{2.075483in}}%
\pgfpathlineto{\pgfqpoint{4.758447in}{1.888795in}}%
\pgfpathlineto{\pgfqpoint{4.755756in}{1.707111in}}%
\pgfpathlineto{\pgfqpoint{4.750925in}{1.532957in}}%
\pgfpathlineto{\pgfqpoint{4.744785in}{1.398726in}}%
\pgfpathlineto{\pgfqpoint{4.737575in}{1.289516in}}%
\pgfpathlineto{\pgfqpoint{4.728714in}{1.190470in}}%
\pgfpathlineto{\pgfqpoint{4.719652in}{1.116521in}}%
\pgfpathlineto{\pgfqpoint{4.710036in}{1.055276in}}%
\pgfpathlineto{\pgfqpoint{4.699503in}{1.001861in}}%
\pgfpathlineto{\pgfqpoint{4.689040in}{0.958690in}}%
\pgfpathlineto{\pgfqpoint{4.677219in}{0.918600in}}%
\pgfpathlineto{\pgfqpoint{4.664034in}{0.881749in}}%
\pgfpathlineto{\pgfqpoint{4.650584in}{0.850492in}}%
\pgfpathlineto{\pgfqpoint{4.636303in}{0.822570in}}%
\pgfpathlineto{\pgfqpoint{4.620207in}{0.795974in}}%
\pgfpathlineto{\pgfqpoint{4.603640in}{0.772901in}}%
\pgfpathlineto{\pgfqpoint{4.585488in}{0.751446in}}%
\pgfpathlineto{\pgfqpoint{4.565874in}{0.731749in}}%
\pgfpathlineto{\pgfqpoint{4.544964in}{0.713879in}}%
\pgfpathlineto{\pgfqpoint{4.522958in}{0.697824in}}%
\pgfpathlineto{\pgfqpoint{4.496157in}{0.681290in}}%
\pgfpathlineto{\pgfqpoint{4.470397in}{0.667953in}}%
\pgfpathlineto{\pgfqpoint{4.439961in}{0.654509in}}%
\pgfpathlineto{\pgfqpoint{4.406841in}{0.642281in}}%
\pgfpathlineto{\pgfqpoint{4.369009in}{0.630748in}}%
\pgfpathlineto{\pgfqpoint{4.326489in}{0.620226in}}%
\pgfpathlineto{\pgfqpoint{4.279327in}{0.610949in}}%
\pgfpathlineto{\pgfqpoint{4.227576in}{0.603085in}}%
\pgfpathlineto{\pgfqpoint{4.173450in}{0.597063in}}%
\pgfpathlineto{\pgfqpoint{4.110511in}{0.592203in}}%
\pgfpathlineto{\pgfqpoint{4.047471in}{0.589537in}}%
\pgfpathlineto{\pgfqpoint{3.977867in}{0.588624in}}%
\pgfpathlineto{\pgfqpoint{3.906093in}{0.589934in}}%
\pgfpathlineto{\pgfqpoint{3.834377in}{0.593496in}}%
\pgfpathlineto{\pgfqpoint{3.767120in}{0.599067in}}%
\pgfpathlineto{\pgfqpoint{3.704364in}{0.606392in}}%
\pgfpathlineto{\pgfqpoint{3.678516in}{0.610510in}}%
\pgfpathlineto{\pgfqpoint{3.620438in}{0.620500in}}%
\pgfpathlineto{\pgfqpoint{3.586319in}{0.628207in}}%
\pgfpathlineto{\pgfqpoint{3.495240in}{0.652428in}}%
\pgfpathlineto{\pgfqpoint{3.451528in}{0.667583in}}%
\pgfpathlineto{\pgfqpoint{3.408538in}{0.685220in}}%
\pgfpathlineto{\pgfqpoint{3.374594in}{0.702001in}}%
\pgfpathlineto{\pgfqpoint{3.345407in}{0.718682in}}%
\pgfpathlineto{\pgfqpoint{3.315236in}{0.738520in}}%
\pgfpathlineto{\pgfqpoint{3.288127in}{0.759290in}}%
\pgfpathlineto{\pgfqpoint{3.264004in}{0.780551in}}%
\pgfpathlineto{\pgfqpoint{3.241208in}{0.803648in}}%
\pgfpathlineto{\pgfqpoint{3.219894in}{0.828530in}}%
\pgfpathlineto{\pgfqpoint{3.200189in}{0.855091in}}%
\pgfpathlineto{\pgfqpoint{3.182177in}{0.883182in}}%
\pgfpathlineto{\pgfqpoint{3.165906in}{0.912633in}}%
\pgfpathlineto{\pgfqpoint{3.150351in}{0.945448in}}%
\pgfpathlineto{\pgfqpoint{3.136682in}{0.979345in}}%
\pgfpathlineto{\pgfqpoint{3.124073in}{1.016460in}}%
\pgfpathlineto{\pgfqpoint{3.112834in}{1.056769in}}%
\pgfpathlineto{\pgfqpoint{3.103046in}{1.100146in}}%
\pgfpathlineto{\pgfqpoint{3.095343in}{1.144071in}}%
\pgfpathlineto{\pgfqpoint{3.089208in}{1.190837in}}%
\pgfpathlineto{\pgfqpoint{3.084595in}{1.242838in}}%
\pgfpathlineto{\pgfqpoint{3.082137in}{1.295031in}}%
\pgfpathlineto{\pgfqpoint{3.081687in}{1.349787in}}%
\pgfpathlineto{\pgfqpoint{3.083451in}{1.406998in}}%
\pgfpathlineto{\pgfqpoint{3.087181in}{1.461589in}}%
\pgfpathlineto{\pgfqpoint{3.093485in}{1.520888in}}%
\pgfpathlineto{\pgfqpoint{3.101823in}{1.577334in}}%
\pgfpathlineto{\pgfqpoint{3.111930in}{1.630856in}}%
\pgfpathlineto{\pgfqpoint{3.124690in}{1.686208in}}%
\pgfpathlineto{\pgfqpoint{3.139178in}{1.738395in}}%
\pgfpathlineto{\pgfqpoint{3.155145in}{1.787366in}}%
\pgfpathlineto{\pgfqpoint{3.172353in}{1.833085in}}%
\pgfpathlineto{\pgfqpoint{3.191618in}{1.877716in}}%
\pgfpathlineto{\pgfqpoint{3.214026in}{1.923261in}}%
\pgfpathlineto{\pgfqpoint{3.236214in}{1.963157in}}%
\pgfpathlineto{\pgfqpoint{3.260178in}{2.001684in}}%
\pgfpathlineto{\pgfqpoint{3.285814in}{2.038776in}}%
\pgfpathlineto{\pgfqpoint{3.314415in}{2.076285in}}%
\pgfpathlineto{\pgfqpoint{3.348944in}{2.117711in}}%
\pgfpathlineto{\pgfqpoint{3.417133in}{2.198022in}}%
\pgfpathlineto{\pgfqpoint{3.426053in}{2.212128in}}%
\pgfpathlineto{\pgfqpoint{3.430798in}{2.223297in}}%
\pgfpathlineto{\pgfqpoint{3.432034in}{2.230603in}}%
\pgfpathlineto{\pgfqpoint{3.430773in}{2.237856in}}%
\pgfpathlineto{\pgfqpoint{3.426621in}{2.243526in}}%
\pgfpathlineto{\pgfqpoint{3.420908in}{2.247084in}}%
\pgfpathlineto{\pgfqpoint{3.412501in}{2.249583in}}%
\pgfpathlineto{\pgfqpoint{3.399499in}{2.250689in}}%
\pgfpathlineto{\pgfqpoint{3.384305in}{2.249671in}}%
\pgfpathlineto{\pgfqpoint{3.364985in}{2.246098in}}%
\pgfpathlineto{\pgfqpoint{3.341804in}{2.239342in}}%
\pgfpathlineto{\pgfqpoint{3.317109in}{2.229682in}}%
\pgfpathlineto{\pgfqpoint{3.291104in}{2.216986in}}%
\pgfpathlineto{\pgfqpoint{3.265928in}{2.202261in}}%
\pgfpathlineto{\pgfqpoint{3.239805in}{2.184361in}}%
\pgfpathlineto{\pgfqpoint{3.214775in}{2.164519in}}%
\pgfpathlineto{\pgfqpoint{3.190900in}{2.142893in}}%
\pgfpathlineto{\pgfqpoint{3.166657in}{2.117912in}}%
\pgfpathlineto{\pgfqpoint{3.143835in}{2.091233in}}%
\pgfpathlineto{\pgfqpoint{3.121079in}{2.061107in}}%
\pgfpathlineto{\pgfqpoint{3.099952in}{2.029463in}}%
\pgfpathlineto{\pgfqpoint{3.079251in}{1.994406in}}%
\pgfpathlineto{\pgfqpoint{3.059218in}{1.955915in}}%
\pgfpathlineto{\pgfqpoint{3.040058in}{1.914015in}}%
\pgfpathlineto{\pgfqpoint{3.022809in}{1.871041in}}%
\pgfpathlineto{\pgfqpoint{3.005790in}{1.822536in}}%
\pgfpathlineto{\pgfqpoint{2.990067in}{1.770819in}}%
\pgfpathlineto{\pgfqpoint{2.975708in}{1.715979in}}%
\pgfpathlineto{\pgfqpoint{2.962284in}{1.655680in}}%
\pgfpathlineto{\pgfqpoint{2.950496in}{1.592386in}}%
\pgfpathlineto{\pgfqpoint{2.940383in}{1.526185in}}%
\pgfpathlineto{\pgfqpoint{2.931745in}{1.454681in}}%
\pgfpathlineto{\pgfqpoint{2.925082in}{1.380399in}}%
\pgfpathlineto{\pgfqpoint{2.920647in}{1.305899in}}%
\pgfpathlineto{\pgfqpoint{2.918444in}{1.231270in}}%
\pgfpathlineto{\pgfqpoint{2.918545in}{1.159087in}}%
\pgfpathlineto{\pgfqpoint{2.920787in}{1.091931in}}%
\pgfpathlineto{\pgfqpoint{2.925177in}{1.027412in}}%
\pgfpathlineto{\pgfqpoint{2.931192in}{0.970580in}}%
\pgfpathlineto{\pgfqpoint{2.938760in}{0.919034in}}%
\pgfpathlineto{\pgfqpoint{2.947651in}{0.872852in}}%
\pgfpathlineto{\pgfqpoint{2.958213in}{0.829714in}}%
\pgfpathlineto{\pgfqpoint{2.969670in}{0.792114in}}%
\pgfpathlineto{\pgfqpoint{2.982463in}{0.757773in}}%
\pgfpathlineto{\pgfqpoint{2.996425in}{0.726812in}}%
\pgfpathlineto{\pgfqpoint{3.011299in}{0.699300in}}%
\pgfpathlineto{\pgfqpoint{3.026739in}{0.675225in}}%
\pgfpathlineto{\pgfqpoint{3.043828in}{0.652656in}}%
\pgfpathlineto{\pgfqpoint{3.062495in}{0.631788in}}%
\pgfpathlineto{\pgfqpoint{3.082602in}{0.612753in}}%
\pgfpathlineto{\pgfqpoint{3.103961in}{0.595592in}}%
\pgfpathlineto{\pgfqpoint{3.128268in}{0.579069in}}%
\pgfpathlineto{\pgfqpoint{3.153537in}{0.564554in}}%
\pgfpathlineto{\pgfqpoint{3.181571in}{0.550952in}}%
\pgfpathlineto{\pgfqpoint{3.214371in}{0.537647in}}%
\pgfpathlineto{\pgfqpoint{3.249846in}{0.525712in}}%
\pgfpathlineto{\pgfqpoint{3.290011in}{0.514571in}}%
\pgfpathlineto{\pgfqpoint{3.334820in}{0.504423in}}%
\pgfpathlineto{\pgfqpoint{3.386372in}{0.494999in}}%
\pgfpathlineto{\pgfqpoint{3.446798in}{0.486257in}}%
\pgfpathlineto{\pgfqpoint{3.518243in}{0.478282in}}%
\pgfpathlineto{\pgfqpoint{3.600685in}{0.471409in}}%
\pgfpathlineto{\pgfqpoint{3.696268in}{0.465713in}}%
\pgfpathlineto{\pgfqpoint{3.807144in}{0.461369in}}%
\pgfpathlineto{\pgfqpoint{3.933291in}{0.458719in}}%
\pgfpathlineto{\pgfqpoint{4.063808in}{0.458211in}}%
\pgfpathlineto{\pgfqpoint{4.187792in}{0.459914in}}%
\pgfpathlineto{\pgfqpoint{4.294335in}{0.463521in}}%
\pgfpathlineto{\pgfqpoint{4.381234in}{0.468574in}}%
\pgfpathlineto{\pgfqpoint{4.450636in}{0.474701in}}%
\pgfpathlineto{\pgfqpoint{4.506850in}{0.481799in}}%
\pgfpathlineto{\pgfqpoint{4.552009in}{0.489658in}}%
\pgfpathlineto{\pgfqpoint{4.588239in}{0.498115in}}%
\pgfpathlineto{\pgfqpoint{4.617656in}{0.507110in}}%
\pgfpathlineto{\pgfqpoint{4.642328in}{0.516843in}}%
\pgfpathlineto{\pgfqpoint{4.664194in}{0.527940in}}%
\pgfpathlineto{\pgfqpoint{4.681238in}{0.538945in}}%
\pgfpathlineto{\pgfqpoint{4.697164in}{0.551953in}}%
\pgfpathlineto{\pgfqpoint{4.710076in}{0.565289in}}%
\pgfpathlineto{\pgfqpoint{4.721578in}{0.580218in}}%
\pgfpathlineto{\pgfqpoint{4.731557in}{0.596521in}}%
\pgfpathlineto{\pgfqpoint{4.741000in}{0.616134in}}%
\pgfpathlineto{\pgfqpoint{4.749521in}{0.639027in}}%
\pgfpathlineto{\pgfqpoint{4.757522in}{0.667450in}}%
\pgfpathlineto{\pgfqpoint{4.764572in}{0.701345in}}%
\pgfpathlineto{\pgfqpoint{4.770840in}{0.743043in}}%
\pgfpathlineto{\pgfqpoint{4.776327in}{0.794934in}}%
\pgfpathlineto{\pgfqpoint{4.781278in}{0.864398in}}%
\pgfpathlineto{\pgfqpoint{4.785468in}{0.956371in}}%
\pgfpathlineto{\pgfqpoint{4.789000in}{1.085745in}}%
\pgfpathlineto{\pgfqpoint{4.791852in}{1.277385in}}%
\pgfpathlineto{\pgfqpoint{4.793959in}{1.581057in}}%
\pgfpathlineto{\pgfqpoint{4.794962in}{2.071429in}}%
\pgfpathlineto{\pgfqpoint{4.793967in}{2.559311in}}%
\pgfpathlineto{\pgfqpoint{4.791733in}{2.745981in}}%
\pgfpathlineto{\pgfqpoint{4.788955in}{2.818091in}}%
\pgfpathlineto{\pgfqpoint{4.785731in}{2.850227in}}%
\pgfpathlineto{\pgfqpoint{4.781879in}{2.867057in}}%
\pgfpathlineto{\pgfqpoint{4.777744in}{2.875780in}}%
\pgfpathlineto{\pgfqpoint{4.773097in}{2.880982in}}%
\pgfpathlineto{\pgfqpoint{4.767363in}{2.884504in}}%
\pgfpathlineto{\pgfqpoint{4.756853in}{2.887622in}}%
\pgfpathlineto{\pgfqpoint{4.739548in}{2.889639in}}%
\pgfpathlineto{\pgfqpoint{4.704762in}{2.890882in}}%
\pgfpathlineto{\pgfqpoint{4.602524in}{2.891538in}}%
\pgfpathlineto{\pgfqpoint{3.952100in}{2.891742in}}%
\pgfpathlineto{\pgfqpoint{0.617321in}{2.890753in}}%
\pgfpathlineto{\pgfqpoint{0.549910in}{2.888858in}}%
\pgfpathlineto{\pgfqpoint{0.521735in}{2.886179in}}%
\pgfpathlineto{\pgfqpoint{0.504666in}{2.882389in}}%
\pgfpathlineto{\pgfqpoint{0.494501in}{2.878011in}}%
\pgfpathlineto{\pgfqpoint{0.487180in}{2.872667in}}%
\pgfpathlineto{\pgfqpoint{0.481152in}{2.865519in}}%
\pgfpathlineto{\pgfqpoint{0.475664in}{2.854804in}}%
\pgfpathlineto{\pgfqpoint{0.471318in}{2.840737in}}%
\pgfpathlineto{\pgfqpoint{0.467301in}{2.818823in}}%
\pgfpathlineto{\pgfqpoint{0.463927in}{2.786700in}}%
\pgfpathlineto{\pgfqpoint{0.460918in}{2.734544in}}%
\pgfpathlineto{\pgfqpoint{0.458363in}{2.647473in}}%
\pgfpathlineto{\pgfqpoint{0.456575in}{2.523031in}}%
\pgfpathlineto{\pgfqpoint{0.456575in}{2.523031in}}%
\pgfusepath{stroke}%
\end{pgfscope}%
\begin{pgfscope}%
\pgfpathrectangle{\pgfqpoint{0.448634in}{0.402556in}}{\pgfqpoint{4.350661in}{2.489204in}} %
\pgfusepath{clip}%
\pgfsetrectcap%
\pgfsetroundjoin%
\pgfsetlinewidth{1.003750pt}%
\definecolor{currentstroke}{rgb}{1.000000,0.388235,0.278431}%
\pgfsetstrokecolor{currentstroke}%
\pgfsetdash{}{0pt}%
\pgfpathmoveto{\pgfqpoint{0.456424in}{1.370137in}}%
\pgfpathlineto{\pgfqpoint{0.459610in}{1.118755in}}%
\pgfpathlineto{\pgfqpoint{0.463695in}{0.962007in}}%
\pgfpathlineto{\pgfqpoint{0.468519in}{0.857610in}}%
\pgfpathlineto{\pgfqpoint{0.474082in}{0.783210in}}%
\pgfpathlineto{\pgfqpoint{0.480226in}{0.728906in}}%
\pgfpathlineto{\pgfqpoint{0.486970in}{0.687306in}}%
\pgfpathlineto{\pgfqpoint{0.494537in}{0.653558in}}%
\pgfpathlineto{\pgfqpoint{0.503107in}{0.625355in}}%
\pgfpathlineto{\pgfqpoint{0.512193in}{0.602749in}}%
\pgfpathlineto{\pgfqpoint{0.522200in}{0.583508in}}%
\pgfpathlineto{\pgfqpoint{0.534108in}{0.565743in}}%
\pgfpathlineto{\pgfqpoint{0.546263in}{0.551507in}}%
\pgfpathlineto{\pgfqpoint{0.559728in}{0.538907in}}%
\pgfpathlineto{\pgfqpoint{0.576129in}{0.526693in}}%
\pgfpathlineto{\pgfqpoint{0.595483in}{0.515351in}}%
\pgfpathlineto{\pgfqpoint{0.617681in}{0.505147in}}%
\pgfpathlineto{\pgfqpoint{0.642568in}{0.496153in}}%
\pgfpathlineto{\pgfqpoint{0.672126in}{0.487778in}}%
\pgfpathlineto{\pgfqpoint{0.708443in}{0.479824in}}%
\pgfpathlineto{\pgfqpoint{0.753649in}{0.472325in}}%
\pgfpathlineto{\pgfqpoint{0.807717in}{0.465660in}}%
\pgfpathlineto{\pgfqpoint{0.877116in}{0.459475in}}%
\pgfpathlineto{\pgfqpoint{0.961828in}{0.454230in}}%
\pgfpathlineto{\pgfqpoint{1.068351in}{0.449916in}}%
\pgfpathlineto{\pgfqpoint{1.201018in}{0.446839in}}%
\pgfpathlineto{\pgfqpoint{1.357637in}{0.445481in}}%
\pgfpathlineto{\pgfqpoint{1.525135in}{0.446232in}}%
\pgfpathlineto{\pgfqpoint{1.686088in}{0.449142in}}%
\pgfpathlineto{\pgfqpoint{1.823074in}{0.453747in}}%
\pgfpathlineto{\pgfqpoint{1.938245in}{0.459764in}}%
\pgfpathlineto{\pgfqpoint{2.031582in}{0.466759in}}%
\pgfpathlineto{\pgfqpoint{2.109580in}{0.474745in}}%
\pgfpathlineto{\pgfqpoint{2.174384in}{0.483535in}}%
\pgfpathlineto{\pgfqpoint{2.228139in}{0.492940in}}%
\pgfpathlineto{\pgfqpoint{2.275119in}{0.503356in}}%
\pgfpathlineto{\pgfqpoint{2.315282in}{0.514501in}}%
\pgfpathlineto{\pgfqpoint{2.350698in}{0.526659in}}%
\pgfpathlineto{\pgfqpoint{2.381320in}{0.539536in}}%
\pgfpathlineto{\pgfqpoint{2.407164in}{0.552659in}}%
\pgfpathlineto{\pgfqpoint{2.430226in}{0.566639in}}%
\pgfpathlineto{\pgfqpoint{2.452282in}{0.582602in}}%
\pgfpathlineto{\pgfqpoint{2.471391in}{0.599069in}}%
\pgfpathlineto{\pgfqpoint{2.489240in}{0.617293in}}%
\pgfpathlineto{\pgfqpoint{2.505678in}{0.637180in}}%
\pgfpathlineto{\pgfqpoint{2.520620in}{0.658557in}}%
\pgfpathlineto{\pgfqpoint{2.535213in}{0.683314in}}%
\pgfpathlineto{\pgfqpoint{2.549115in}{0.711484in}}%
\pgfpathlineto{\pgfqpoint{2.562091in}{0.743004in}}%
\pgfpathlineto{\pgfqpoint{2.574020in}{0.777751in}}%
\pgfpathlineto{\pgfqpoint{2.585502in}{0.817970in}}%
\pgfpathlineto{\pgfqpoint{2.596809in}{0.866038in}}%
\pgfpathlineto{\pgfqpoint{2.607562in}{0.921948in}}%
\pgfpathlineto{\pgfqpoint{2.617925in}{0.988098in}}%
\pgfpathlineto{\pgfqpoint{2.627958in}{1.066918in}}%
\pgfpathlineto{\pgfqpoint{2.637941in}{1.163320in}}%
\pgfpathlineto{\pgfqpoint{2.648424in}{1.287199in}}%
\pgfpathlineto{\pgfqpoint{2.660103in}{1.453438in}}%
\pgfpathlineto{\pgfqpoint{2.674773in}{1.696801in}}%
\pgfpathlineto{\pgfqpoint{2.687716in}{1.945279in}}%
\pgfpathlineto{\pgfqpoint{2.692670in}{2.079573in}}%
\pgfpathlineto{\pgfqpoint{2.693829in}{2.166682in}}%
\pgfpathlineto{\pgfqpoint{2.692565in}{2.233870in}}%
\pgfpathlineto{\pgfqpoint{2.689436in}{2.286015in}}%
\pgfpathlineto{\pgfqpoint{2.684859in}{2.327999in}}%
\pgfpathlineto{\pgfqpoint{2.678725in}{2.364664in}}%
\pgfpathlineto{\pgfqpoint{2.671356in}{2.395897in}}%
\pgfpathlineto{\pgfqpoint{2.662489in}{2.423981in}}%
\pgfpathlineto{\pgfqpoint{2.652361in}{2.448778in}}%
\pgfpathlineto{\pgfqpoint{2.641365in}{2.470245in}}%
\pgfpathlineto{\pgfqpoint{2.628643in}{2.490425in}}%
\pgfpathlineto{\pgfqpoint{2.614279in}{2.509106in}}%
\pgfpathlineto{\pgfqpoint{2.598443in}{2.526159in}}%
\pgfpathlineto{\pgfqpoint{2.579590in}{2.543005in}}%
\pgfpathlineto{\pgfqpoint{2.559532in}{2.557923in}}%
\pgfpathlineto{\pgfqpoint{2.536602in}{2.572183in}}%
\pgfpathlineto{\pgfqpoint{2.510850in}{2.585538in}}%
\pgfpathlineto{\pgfqpoint{2.482360in}{2.597837in}}%
\pgfpathlineto{\pgfqpoint{2.449134in}{2.609683in}}%
\pgfpathlineto{\pgfqpoint{2.411184in}{2.620696in}}%
\pgfpathlineto{\pgfqpoint{2.368552in}{2.630606in}}%
\pgfpathlineto{\pgfqpoint{2.321294in}{2.639221in}}%
\pgfpathlineto{\pgfqpoint{2.269467in}{2.646399in}}%
\pgfpathlineto{\pgfqpoint{2.210954in}{2.652193in}}%
\pgfpathlineto{\pgfqpoint{2.147967in}{2.656153in}}%
\pgfpathlineto{\pgfqpoint{2.080556in}{2.658135in}}%
\pgfpathlineto{\pgfqpoint{2.010948in}{2.657971in}}%
\pgfpathlineto{\pgfqpoint{1.939195in}{2.655572in}}%
\pgfpathlineto{\pgfqpoint{1.867527in}{2.650913in}}%
\pgfpathlineto{\pgfqpoint{1.798171in}{2.644140in}}%
\pgfpathlineto{\pgfqpoint{1.733341in}{2.635606in}}%
\pgfpathlineto{\pgfqpoint{1.673075in}{2.625521in}}%
\pgfpathlineto{\pgfqpoint{1.615274in}{2.613610in}}%
\pgfpathlineto{\pgfqpoint{1.562133in}{2.600402in}}%
\pgfpathlineto{\pgfqpoint{1.513681in}{2.586139in}}%
\pgfpathlineto{\pgfqpoint{1.467862in}{2.570344in}}%
\pgfpathlineto{\pgfqpoint{1.426794in}{2.553923in}}%
\pgfpathlineto{\pgfqpoint{1.388447in}{2.536289in}}%
\pgfpathlineto{\pgfqpoint{1.352878in}{2.517566in}}%
\pgfpathlineto{\pgfqpoint{1.320128in}{2.497922in}}%
\pgfpathlineto{\pgfqpoint{1.288379in}{2.476236in}}%
\pgfpathlineto{\pgfqpoint{1.259592in}{2.453861in}}%
\pgfpathlineto{\pgfqpoint{1.232050in}{2.429520in}}%
\pgfpathlineto{\pgfqpoint{1.207527in}{2.404898in}}%
\pgfpathlineto{\pgfqpoint{1.184409in}{2.378557in}}%
\pgfpathlineto{\pgfqpoint{1.162828in}{2.350561in}}%
\pgfpathlineto{\pgfqpoint{1.142891in}{2.321011in}}%
\pgfpathlineto{\pgfqpoint{1.124675in}{2.290041in}}%
\pgfpathlineto{\pgfqpoint{1.108225in}{2.257802in}}%
\pgfpathlineto{\pgfqpoint{1.092639in}{2.222199in}}%
\pgfpathlineto{\pgfqpoint{1.079059in}{2.185535in}}%
\pgfpathlineto{\pgfqpoint{1.067443in}{2.147998in}}%
\pgfpathlineto{\pgfqpoint{1.057187in}{2.107348in}}%
\pgfpathlineto{\pgfqpoint{1.049004in}{2.066086in}}%
\pgfpathlineto{\pgfqpoint{1.042513in}{2.021906in}}%
\pgfpathlineto{\pgfqpoint{1.038177in}{1.977382in}}%
\pgfpathlineto{\pgfqpoint{1.035866in}{1.930167in}}%
\pgfpathlineto{\pgfqpoint{1.035826in}{1.882878in}}%
\pgfpathlineto{\pgfqpoint{1.038031in}{1.835656in}}%
\pgfpathlineto{\pgfqpoint{1.042474in}{1.788641in}}%
\pgfpathlineto{\pgfqpoint{1.049176in}{1.741979in}}%
\pgfpathlineto{\pgfqpoint{1.057644in}{1.698239in}}%
\pgfpathlineto{\pgfqpoint{1.068221in}{1.655105in}}%
\pgfpathlineto{\pgfqpoint{1.080962in}{1.612745in}}%
\pgfpathlineto{\pgfqpoint{1.095031in}{1.573617in}}%
\pgfpathlineto{\pgfqpoint{1.111115in}{1.535520in}}%
\pgfpathlineto{\pgfqpoint{1.128118in}{1.500775in}}%
\pgfpathlineto{\pgfqpoint{1.146930in}{1.467274in}}%
\pgfpathlineto{\pgfqpoint{1.167531in}{1.435181in}}%
\pgfpathlineto{\pgfqpoint{1.189874in}{1.404652in}}%
\pgfpathlineto{\pgfqpoint{1.213884in}{1.375828in}}%
\pgfpathlineto{\pgfqpoint{1.237817in}{1.350457in}}%
\pgfpathlineto{\pgfqpoint{1.264748in}{1.325237in}}%
\pgfpathlineto{\pgfqpoint{1.292991in}{1.301972in}}%
\pgfpathlineto{\pgfqpoint{1.322398in}{1.280678in}}%
\pgfpathlineto{\pgfqpoint{1.352820in}{1.261340in}}%
\pgfpathlineto{\pgfqpoint{1.386095in}{1.242889in}}%
\pgfpathlineto{\pgfqpoint{1.420190in}{1.226516in}}%
\pgfpathlineto{\pgfqpoint{1.457024in}{1.211329in}}%
\pgfpathlineto{\pgfqpoint{1.496554in}{1.197536in}}%
\pgfpathlineto{\pgfqpoint{1.538719in}{1.185287in}}%
\pgfpathlineto{\pgfqpoint{1.583441in}{1.174641in}}%
\pgfpathlineto{\pgfqpoint{1.634929in}{1.164775in}}%
\pgfpathlineto{\pgfqpoint{1.706063in}{1.153745in}}%
\pgfpathlineto{\pgfqpoint{1.768492in}{1.143417in}}%
\pgfpathlineto{\pgfqpoint{1.796122in}{1.136567in}}%
\pgfpathlineto{\pgfqpoint{1.812683in}{1.130481in}}%
\pgfpathlineto{\pgfqpoint{1.824471in}{1.124102in}}%
\pgfpathlineto{\pgfqpoint{1.833209in}{1.116741in}}%
\pgfpathlineto{\pgfqpoint{1.838498in}{1.108890in}}%
\pgfpathlineto{\pgfqpoint{1.840588in}{1.101849in}}%
\pgfpathlineto{\pgfqpoint{1.840619in}{1.094412in}}%
\pgfpathlineto{\pgfqpoint{1.837931in}{1.084986in}}%
\pgfpathlineto{\pgfqpoint{1.833246in}{1.076615in}}%
\pgfpathlineto{\pgfqpoint{1.825819in}{1.067542in}}%
\pgfpathlineto{\pgfqpoint{1.813813in}{1.056850in}}%
\pgfpathlineto{\pgfqpoint{1.798819in}{1.046763in}}%
\pgfpathlineto{\pgfqpoint{1.781016in}{1.037462in}}%
\pgfpathlineto{\pgfqpoint{1.758447in}{1.028391in}}%
\pgfpathlineto{\pgfqpoint{1.733203in}{1.020815in}}%
\pgfpathlineto{\pgfqpoint{1.705410in}{1.014872in}}%
\pgfpathlineto{\pgfqpoint{1.675178in}{1.010714in}}%
\pgfpathlineto{\pgfqpoint{1.642610in}{1.008507in}}%
\pgfpathlineto{\pgfqpoint{1.607809in}{1.008432in}}%
\pgfpathlineto{\pgfqpoint{1.570886in}{1.010691in}}%
\pgfpathlineto{\pgfqpoint{1.534118in}{1.015181in}}%
\pgfpathlineto{\pgfqpoint{1.495454in}{1.022233in}}%
\pgfpathlineto{\pgfqpoint{1.457161in}{1.031563in}}%
\pgfpathlineto{\pgfqpoint{1.419337in}{1.043132in}}%
\pgfpathlineto{\pgfqpoint{1.382089in}{1.056929in}}%
\pgfpathlineto{\pgfqpoint{1.347544in}{1.072019in}}%
\pgfpathlineto{\pgfqpoint{1.313727in}{1.089133in}}%
\pgfpathlineto{\pgfqpoint{1.280762in}{1.108299in}}%
\pgfpathlineto{\pgfqpoint{1.248782in}{1.129536in}}%
\pgfpathlineto{\pgfqpoint{1.219708in}{1.151422in}}%
\pgfpathlineto{\pgfqpoint{1.191752in}{1.175138in}}%
\pgfpathlineto{\pgfqpoint{1.165031in}{1.200649in}}%
\pgfpathlineto{\pgfqpoint{1.139653in}{1.227898in}}%
\pgfpathlineto{\pgfqpoint{1.115714in}{1.256800in}}%
\pgfpathlineto{\pgfqpoint{1.093288in}{1.287251in}}%
\pgfpathlineto{\pgfqpoint{1.071178in}{1.321163in}}%
\pgfpathlineto{\pgfqpoint{1.050868in}{1.356520in}}%
\pgfpathlineto{\pgfqpoint{1.032365in}{1.393152in}}%
\pgfpathlineto{\pgfqpoint{1.014718in}{1.433142in}}%
\pgfpathlineto{\pgfqpoint{0.999024in}{1.474185in}}%
\pgfpathlineto{\pgfqpoint{0.984506in}{1.518461in}}%
\pgfpathlineto{\pgfqpoint{0.972010in}{1.563537in}}%
\pgfpathlineto{\pgfqpoint{0.960944in}{1.611678in}}%
\pgfpathlineto{\pgfqpoint{0.951530in}{1.662824in}}%
\pgfpathlineto{\pgfqpoint{0.944286in}{1.714431in}}%
\pgfpathlineto{\pgfqpoint{0.938950in}{1.768847in}}%
\pgfpathlineto{\pgfqpoint{0.935870in}{1.823491in}}%
\pgfpathlineto{\pgfqpoint{0.935034in}{1.878240in}}%
\pgfpathlineto{\pgfqpoint{0.936466in}{1.932973in}}%
\pgfpathlineto{\pgfqpoint{0.940005in}{1.985084in}}%
\pgfpathlineto{\pgfqpoint{0.945759in}{2.036935in}}%
\pgfpathlineto{\pgfqpoint{0.953410in}{2.085938in}}%
\pgfpathlineto{\pgfqpoint{0.962764in}{2.132000in}}%
\pgfpathlineto{\pgfqpoint{0.974287in}{2.177414in}}%
\pgfpathlineto{\pgfqpoint{0.987332in}{2.219653in}}%
\pgfpathlineto{\pgfqpoint{1.001667in}{2.258654in}}%
\pgfpathlineto{\pgfqpoint{1.018051in}{2.296583in}}%
\pgfpathlineto{\pgfqpoint{1.035401in}{2.331101in}}%
\pgfpathlineto{\pgfqpoint{1.054650in}{2.364275in}}%
\pgfpathlineto{\pgfqpoint{1.074406in}{2.393984in}}%
\pgfpathlineto{\pgfqpoint{1.095771in}{2.422197in}}%
\pgfpathlineto{\pgfqpoint{1.118662in}{2.448797in}}%
\pgfpathlineto{\pgfqpoint{1.142967in}{2.473701in}}%
\pgfpathlineto{\pgfqpoint{1.168550in}{2.496867in}}%
\pgfpathlineto{\pgfqpoint{1.197085in}{2.519662in}}%
\pgfpathlineto{\pgfqpoint{1.226727in}{2.540526in}}%
\pgfpathlineto{\pgfqpoint{1.259242in}{2.560673in}}%
\pgfpathlineto{\pgfqpoint{1.294612in}{2.579881in}}%
\pgfpathlineto{\pgfqpoint{1.332792in}{2.597982in}}%
\pgfpathlineto{\pgfqpoint{1.373719in}{2.614859in}}%
\pgfpathlineto{\pgfqpoint{1.417319in}{2.630445in}}%
\pgfpathlineto{\pgfqpoint{1.465632in}{2.645312in}}%
\pgfpathlineto{\pgfqpoint{1.518640in}{2.659204in}}%
\pgfpathlineto{\pgfqpoint{1.576309in}{2.671929in}}%
\pgfpathlineto{\pgfqpoint{1.638597in}{2.683344in}}%
\pgfpathlineto{\pgfqpoint{1.705462in}{2.693343in}}%
\pgfpathlineto{\pgfqpoint{1.779027in}{2.702064in}}%
\pgfpathlineto{\pgfqpoint{1.857097in}{2.709077in}}%
\pgfpathlineto{\pgfqpoint{1.939633in}{2.714280in}}%
\pgfpathlineto{\pgfqpoint{2.026598in}{2.717513in}}%
\pgfpathlineto{\pgfqpoint{2.113605in}{2.718523in}}%
\pgfpathlineto{\pgfqpoint{2.198435in}{2.717303in}}%
\pgfpathlineto{\pgfqpoint{2.278866in}{2.713929in}}%
\pgfpathlineto{\pgfqpoint{2.352678in}{2.708598in}}%
\pgfpathlineto{\pgfqpoint{2.417657in}{2.701709in}}%
\pgfpathlineto{\pgfqpoint{2.473770in}{2.693630in}}%
\pgfpathlineto{\pgfqpoint{2.523140in}{2.684368in}}%
\pgfpathlineto{\pgfqpoint{2.565726in}{2.674202in}}%
\pgfpathlineto{\pgfqpoint{2.601510in}{2.663544in}}%
\pgfpathlineto{\pgfqpoint{2.632577in}{2.652142in}}%
\pgfpathlineto{\pgfqpoint{2.658899in}{2.640331in}}%
\pgfpathlineto{\pgfqpoint{2.682438in}{2.627436in}}%
\pgfpathlineto{\pgfqpoint{2.703062in}{2.613571in}}%
\pgfpathlineto{\pgfqpoint{2.720674in}{2.598978in}}%
\pgfpathlineto{\pgfqpoint{2.735263in}{2.584053in}}%
\pgfpathlineto{\pgfqpoint{2.748320in}{2.567377in}}%
\pgfpathlineto{\pgfqpoint{2.759553in}{2.549046in}}%
\pgfpathlineto{\pgfqpoint{2.768788in}{2.529306in}}%
\pgfpathlineto{\pgfqpoint{2.776017in}{2.508498in}}%
\pgfpathlineto{\pgfqpoint{2.781884in}{2.484540in}}%
\pgfpathlineto{\pgfqpoint{2.786102in}{2.457597in}}%
\pgfpathlineto{\pgfqpoint{2.788720in}{2.425384in}}%
\pgfpathlineto{\pgfqpoint{2.789427in}{2.388061in}}%
\pgfpathlineto{\pgfqpoint{2.787962in}{2.340801in}}%
\pgfpathlineto{\pgfqpoint{2.783672in}{2.278768in}}%
\pgfpathlineto{\pgfqpoint{2.774289in}{2.179783in}}%
\pgfpathlineto{\pgfqpoint{2.743611in}{1.868119in}}%
\pgfpathlineto{\pgfqpoint{2.730112in}{1.702060in}}%
\pgfpathlineto{\pgfqpoint{2.717287in}{1.515949in}}%
\pgfpathlineto{\pgfqpoint{2.702602in}{1.267597in}}%
\pgfpathlineto{\pgfqpoint{2.684434in}{0.964630in}}%
\pgfpathlineto{\pgfqpoint{2.675374in}{0.850600in}}%
\pgfpathlineto{\pgfqpoint{2.667030in}{0.771523in}}%
\pgfpathlineto{\pgfqpoint{2.658752in}{0.712543in}}%
\pgfpathlineto{\pgfqpoint{2.650176in}{0.666284in}}%
\pgfpathlineto{\pgfqpoint{2.640820in}{0.627931in}}%
\pgfpathlineto{\pgfqpoint{2.631145in}{0.597534in}}%
\pgfpathlineto{\pgfqpoint{2.621004in}{0.572745in}}%
\pgfpathlineto{\pgfqpoint{2.609856in}{0.551383in}}%
\pgfpathlineto{\pgfqpoint{2.598042in}{0.533534in}}%
\pgfpathlineto{\pgfqpoint{2.584496in}{0.517378in}}%
\pgfpathlineto{\pgfqpoint{2.571109in}{0.504669in}}%
\pgfpathlineto{\pgfqpoint{2.554789in}{0.492313in}}%
\pgfpathlineto{\pgfqpoint{2.537457in}{0.481914in}}%
\pgfpathlineto{\pgfqpoint{2.517374in}{0.472367in}}%
\pgfpathlineto{\pgfqpoint{2.492542in}{0.463178in}}%
\pgfpathlineto{\pgfqpoint{2.462979in}{0.454833in}}%
\pgfpathlineto{\pgfqpoint{2.428766in}{0.447542in}}%
\pgfpathlineto{\pgfqpoint{2.385671in}{0.440735in}}%
\pgfpathlineto{\pgfqpoint{2.331557in}{0.434581in}}%
\pgfpathlineto{\pgfqpoint{2.262115in}{0.429077in}}%
\pgfpathlineto{\pgfqpoint{2.170851in}{0.424236in}}%
\pgfpathlineto{\pgfqpoint{2.049086in}{0.420134in}}%
\pgfpathlineto{\pgfqpoint{1.879436in}{0.416783in}}%
\pgfpathlineto{\pgfqpoint{1.640159in}{0.414418in}}%
\pgfpathlineto{\pgfqpoint{1.322562in}{0.413569in}}%
\pgfpathlineto{\pgfqpoint{1.020194in}{0.414850in}}%
\pgfpathlineto{\pgfqpoint{0.822256in}{0.417715in}}%
\pgfpathlineto{\pgfqpoint{0.704835in}{0.421430in}}%
\pgfpathlineto{\pgfqpoint{0.630976in}{0.425829in}}%
\pgfpathlineto{\pgfqpoint{0.583316in}{0.430734in}}%
\pgfpathlineto{\pgfqpoint{0.551033in}{0.436123in}}%
\pgfpathlineto{\pgfqpoint{0.527708in}{0.442189in}}%
\pgfpathlineto{\pgfqpoint{0.511250in}{0.448625in}}%
\pgfpathlineto{\pgfqpoint{0.499549in}{0.455216in}}%
\pgfpathlineto{\pgfqpoint{0.488916in}{0.463841in}}%
\pgfpathlineto{\pgfqpoint{0.481322in}{0.472730in}}%
\pgfpathlineto{\pgfqpoint{0.474078in}{0.485127in}}%
\pgfpathlineto{\pgfqpoint{0.468753in}{0.498748in}}%
\pgfpathlineto{\pgfqpoint{0.463870in}{0.517848in}}%
\pgfpathlineto{\pgfqpoint{0.459679in}{0.544796in}}%
\pgfpathlineto{\pgfqpoint{0.456386in}{0.581938in}}%
\pgfpathlineto{\pgfqpoint{0.453731in}{0.639106in}}%
\pgfpathlineto{\pgfqpoint{0.451681in}{0.736155in}}%
\pgfpathlineto{\pgfqpoint{0.450220in}{0.927815in}}%
\pgfpathlineto{\pgfqpoint{0.449345in}{1.403252in}}%
\pgfpathlineto{\pgfqpoint{0.449543in}{2.682703in}}%
\pgfpathlineto{\pgfqpoint{0.451011in}{2.856932in}}%
\pgfpathlineto{\pgfqpoint{0.452802in}{2.879219in}}%
\pgfpathlineto{\pgfqpoint{0.455188in}{2.886108in}}%
\pgfpathlineto{\pgfqpoint{0.458626in}{2.889028in}}%
\pgfpathlineto{\pgfqpoint{0.464996in}{2.890553in}}%
\pgfpathlineto{\pgfqpoint{0.482377in}{2.891423in}}%
\pgfpathlineto{\pgfqpoint{0.565038in}{2.891729in}}%
\pgfpathlineto{\pgfqpoint{2.733842in}{2.891760in}}%
\pgfpathlineto{\pgfqpoint{4.789510in}{2.890885in}}%
\pgfpathlineto{\pgfqpoint{4.793727in}{2.889730in}}%
\pgfpathlineto{\pgfqpoint{4.795481in}{2.888307in}}%
\pgfpathlineto{\pgfqpoint{4.797106in}{2.881145in}}%
\pgfpathlineto{\pgfqpoint{4.797997in}{2.858771in}}%
\pgfpathlineto{\pgfqpoint{4.798039in}{2.856283in}}%
\pgfpathlineto{\pgfqpoint{4.798039in}{2.856283in}}%
\pgfusepath{stroke}%
\end{pgfscope}%
\begin{pgfscope}%
\pgfpathrectangle{\pgfqpoint{0.448634in}{0.402556in}}{\pgfqpoint{4.350661in}{2.489204in}} %
\pgfusepath{clip}%
\pgfsetrectcap%
\pgfsetroundjoin%
\pgfsetlinewidth{1.003750pt}%
\definecolor{currentstroke}{rgb}{1.000000,0.388235,0.278431}%
\pgfsetstrokecolor{currentstroke}%
\pgfsetdash{}{0pt}%
\pgfpathmoveto{\pgfqpoint{3.428772in}{0.402610in}}%
\pgfpathlineto{\pgfqpoint{2.806632in}{0.403760in}}%
\pgfpathlineto{\pgfqpoint{2.769692in}{0.405578in}}%
\pgfpathlineto{\pgfqpoint{2.754632in}{0.408064in}}%
\pgfpathlineto{\pgfqpoint{2.746391in}{0.411198in}}%
\pgfpathlineto{\pgfqpoint{2.740943in}{0.415265in}}%
\pgfpathlineto{\pgfqpoint{2.736784in}{0.420984in}}%
\pgfpathlineto{\pgfqpoint{2.733281in}{0.430071in}}%
\pgfpathlineto{\pgfqpoint{2.730449in}{0.444636in}}%
\pgfpathlineto{\pgfqpoint{2.728238in}{0.469392in}}%
\pgfpathlineto{\pgfqpoint{2.726470in}{0.519131in}}%
\pgfpathlineto{\pgfqpoint{2.725711in}{0.613715in}}%
\pgfpathlineto{\pgfqpoint{2.726842in}{0.768038in}}%
\pgfpathlineto{\pgfqpoint{2.730556in}{0.962148in}}%
\pgfpathlineto{\pgfqpoint{2.736611in}{1.158670in}}%
\pgfpathlineto{\pgfqpoint{2.744092in}{1.327718in}}%
\pgfpathlineto{\pgfqpoint{2.753201in}{1.484189in}}%
\pgfpathlineto{\pgfqpoint{2.763257in}{1.620609in}}%
\pgfpathlineto{\pgfqpoint{2.776118in}{1.764216in}}%
\pgfpathlineto{\pgfqpoint{2.788914in}{1.877776in}}%
\pgfpathlineto{\pgfqpoint{2.805748in}{2.005740in}}%
\pgfpathlineto{\pgfqpoint{2.821176in}{2.101198in}}%
\pgfpathlineto{\pgfqpoint{2.838359in}{2.193718in}}%
\pgfpathlineto{\pgfqpoint{2.859135in}{2.292966in}}%
\pgfpathlineto{\pgfqpoint{2.887209in}{2.425960in}}%
\pgfpathlineto{\pgfqpoint{2.896991in}{2.479559in}}%
\pgfpathlineto{\pgfqpoint{2.901543in}{2.516523in}}%
\pgfpathlineto{\pgfqpoint{2.902849in}{2.543854in}}%
\pgfpathlineto{\pgfqpoint{2.901957in}{2.566223in}}%
\pgfpathlineto{\pgfqpoint{2.899151in}{2.585863in}}%
\pgfpathlineto{\pgfqpoint{2.894794in}{2.602546in}}%
\pgfpathlineto{\pgfqpoint{2.888484in}{2.618388in}}%
\pgfpathlineto{\pgfqpoint{2.880257in}{2.633033in}}%
\pgfpathlineto{\pgfqpoint{2.870348in}{2.646246in}}%
\pgfpathlineto{\pgfqpoint{2.857400in}{2.659530in}}%
\pgfpathlineto{\pgfqpoint{2.843189in}{2.671010in}}%
\pgfpathlineto{\pgfqpoint{2.824237in}{2.683209in}}%
\pgfpathlineto{\pgfqpoint{2.802413in}{2.694418in}}%
\pgfpathlineto{\pgfqpoint{2.775809in}{2.705369in}}%
\pgfpathlineto{\pgfqpoint{2.744461in}{2.715715in}}%
\pgfpathlineto{\pgfqpoint{2.708436in}{2.725252in}}%
\pgfpathlineto{\pgfqpoint{2.665655in}{2.734289in}}%
\pgfpathlineto{\pgfqpoint{2.613991in}{2.742869in}}%
\pgfpathlineto{\pgfqpoint{2.553459in}{2.750589in}}%
\pgfpathlineto{\pgfqpoint{2.481920in}{2.757365in}}%
\pgfpathlineto{\pgfqpoint{2.399398in}{2.762839in}}%
\pgfpathlineto{\pgfqpoint{2.310269in}{2.766482in}}%
\pgfpathlineto{\pgfqpoint{2.175416in}{2.768725in}}%
\pgfpathlineto{\pgfqpoint{2.066653in}{2.767942in}}%
\pgfpathlineto{\pgfqpoint{1.953570in}{2.764859in}}%
\pgfpathlineto{\pgfqpoint{1.851429in}{2.759759in}}%
\pgfpathlineto{\pgfqpoint{1.745051in}{2.752169in}}%
\pgfpathlineto{\pgfqpoint{1.658373in}{2.743453in}}%
\pgfpathlineto{\pgfqpoint{1.580552in}{2.733461in}}%
\pgfpathlineto{\pgfqpoint{1.490057in}{2.719338in}}%
\pgfpathlineto{\pgfqpoint{1.417231in}{2.704698in}}%
\pgfpathlineto{\pgfqpoint{1.361992in}{2.690818in}}%
\pgfpathlineto{\pgfqpoint{1.311460in}{2.675819in}}%
\pgfpathlineto{\pgfqpoint{1.265667in}{2.659924in}}%
\pgfpathlineto{\pgfqpoint{1.222575in}{2.642586in}}%
\pgfpathlineto{\pgfqpoint{1.184324in}{2.624682in}}%
\pgfpathlineto{\pgfqpoint{1.148892in}{2.605623in}}%
\pgfpathlineto{\pgfqpoint{1.116331in}{2.585573in}}%
\pgfpathlineto{\pgfqpoint{1.092327in}{2.568512in}}%
\pgfpathlineto{\pgfqpoint{1.079760in}{2.558686in}}%
\pgfpathlineto{\pgfqpoint{1.051544in}{2.535379in}}%
\pgfpathlineto{\pgfqpoint{1.026312in}{2.511712in}}%
\pgfpathlineto{\pgfqpoint{1.002399in}{2.486318in}}%
\pgfpathlineto{\pgfqpoint{0.979913in}{2.459269in}}%
\pgfpathlineto{\pgfqpoint{0.958934in}{2.430678in}}%
\pgfpathlineto{\pgfqpoint{0.938264in}{2.398643in}}%
\pgfpathlineto{\pgfqpoint{0.923047in}{2.371385in}}%
\pgfpathlineto{\pgfqpoint{0.904513in}{2.334774in}}%
\pgfpathlineto{\pgfqpoint{0.887854in}{2.297001in}}%
\pgfpathlineto{\pgfqpoint{0.872131in}{2.255971in}}%
\pgfpathlineto{\pgfqpoint{0.857508in}{2.211741in}}%
\pgfpathlineto{\pgfqpoint{0.844762in}{2.166757in}}%
\pgfpathlineto{\pgfqpoint{0.838624in}{2.140306in}}%
\pgfpathlineto{\pgfqpoint{0.826982in}{2.087194in}}%
\pgfpathlineto{\pgfqpoint{0.816322in}{2.028715in}}%
\pgfpathlineto{\pgfqpoint{0.810087in}{1.984495in}}%
\pgfpathlineto{\pgfqpoint{0.808026in}{1.967238in}}%
\pgfpathlineto{\pgfqpoint{0.800076in}{1.898140in}}%
\pgfpathlineto{\pgfqpoint{0.793713in}{1.823823in}}%
\pgfpathlineto{\pgfqpoint{0.788799in}{1.741875in}}%
\pgfpathlineto{\pgfqpoint{0.786199in}{1.677225in}}%
\pgfpathlineto{\pgfqpoint{0.776951in}{1.453481in}}%
\pgfpathlineto{\pgfqpoint{0.773280in}{1.418894in}}%
\pgfpathlineto{\pgfqpoint{0.768298in}{1.389582in}}%
\pgfpathlineto{\pgfqpoint{0.762752in}{1.368108in}}%
\pgfpathlineto{\pgfqpoint{0.756722in}{1.352123in}}%
\pgfpathlineto{\pgfqpoint{0.749752in}{1.339519in}}%
\pgfpathlineto{\pgfqpoint{0.742201in}{1.330599in}}%
\pgfpathlineto{\pgfqpoint{0.734854in}{1.325312in}}%
\pgfpathlineto{\pgfqpoint{0.726558in}{1.322419in}}%
\pgfpathlineto{\pgfqpoint{0.717884in}{1.322223in}}%
\pgfpathlineto{\pgfqpoint{0.709412in}{1.324411in}}%
\pgfpathlineto{\pgfqpoint{0.699548in}{1.329604in}}%
\pgfpathlineto{\pgfqpoint{0.688894in}{1.338203in}}%
\pgfpathlineto{\pgfqpoint{0.677907in}{1.350248in}}%
\pgfpathlineto{\pgfqpoint{0.666886in}{1.365647in}}%
\pgfpathlineto{\pgfqpoint{0.654913in}{1.386417in}}%
\pgfpathlineto{\pgfqpoint{0.642574in}{1.412730in}}%
\pgfpathlineto{\pgfqpoint{0.630328in}{1.444629in}}%
\pgfpathlineto{\pgfqpoint{0.618504in}{1.482081in}}%
\pgfpathlineto{\pgfqpoint{0.608613in}{1.520256in}}%
\pgfpathlineto{\pgfqpoint{0.590203in}{1.612445in}}%
\pgfpathlineto{\pgfqpoint{0.581848in}{1.668884in}}%
\pgfpathlineto{\pgfqpoint{0.573137in}{1.740376in}}%
\pgfpathlineto{\pgfqpoint{0.567062in}{1.807213in}}%
\pgfpathlineto{\pgfqpoint{0.560532in}{1.896510in}}%
\pgfpathlineto{\pgfqpoint{0.555526in}{1.995910in}}%
\pgfpathlineto{\pgfqpoint{0.552564in}{2.097908in}}%
\pgfpathlineto{\pgfqpoint{0.551526in}{2.204935in}}%
\pgfpathlineto{\pgfqpoint{0.552728in}{2.309470in}}%
\pgfpathlineto{\pgfqpoint{0.556011in}{2.403981in}}%
\pgfpathlineto{\pgfqpoint{0.560953in}{2.483430in}}%
\pgfpathlineto{\pgfqpoint{0.567303in}{2.550240in}}%
\pgfpathlineto{\pgfqpoint{0.574928in}{2.606817in}}%
\pgfpathlineto{\pgfqpoint{0.582988in}{2.650657in}}%
\pgfpathlineto{\pgfqpoint{0.592756in}{2.691452in}}%
\pgfpathlineto{\pgfqpoint{0.602650in}{2.721756in}}%
\pgfpathlineto{\pgfqpoint{0.612983in}{2.746441in}}%
\pgfpathlineto{\pgfqpoint{0.624292in}{2.767692in}}%
\pgfpathlineto{\pgfqpoint{0.636231in}{2.785433in}}%
\pgfpathlineto{\pgfqpoint{0.649892in}{2.801461in}}%
\pgfpathlineto{\pgfqpoint{0.663386in}{2.814020in}}%
\pgfpathlineto{\pgfqpoint{0.679842in}{2.826135in}}%
\pgfpathlineto{\pgfqpoint{0.697326in}{2.836197in}}%
\pgfpathlineto{\pgfqpoint{0.715574in}{2.844285in}}%
\pgfpathlineto{\pgfqpoint{0.738439in}{2.852335in}}%
\pgfpathlineto{\pgfqpoint{0.765983in}{2.859639in}}%
\pgfpathlineto{\pgfqpoint{0.800300in}{2.866256in}}%
\pgfpathlineto{\pgfqpoint{0.841340in}{2.871832in}}%
\pgfpathlineto{\pgfqpoint{0.895547in}{2.876803in}}%
\pgfpathlineto{\pgfqpoint{0.969413in}{2.881069in}}%
\pgfpathlineto{\pgfqpoint{1.071608in}{2.884501in}}%
\pgfpathlineto{\pgfqpoint{1.219512in}{2.887074in}}%
\pgfpathlineto{\pgfqpoint{1.471844in}{2.889091in}}%
\pgfpathlineto{\pgfqpoint{1.956941in}{2.890384in}}%
\pgfpathlineto{\pgfqpoint{3.096814in}{2.890781in}}%
\pgfpathlineto{\pgfqpoint{3.995224in}{2.889388in}}%
\pgfpathlineto{\pgfqpoint{4.275833in}{2.887011in}}%
\pgfpathlineto{\pgfqpoint{4.412847in}{2.883743in}}%
\pgfpathlineto{\pgfqpoint{4.491081in}{2.879810in}}%
\pgfpathlineto{\pgfqpoint{4.543127in}{2.875163in}}%
\pgfpathlineto{\pgfqpoint{4.579810in}{2.869841in}}%
\pgfpathlineto{\pgfqpoint{4.607580in}{2.863763in}}%
\pgfpathlineto{\pgfqpoint{4.630623in}{2.856424in}}%
\pgfpathlineto{\pgfqpoint{4.648833in}{2.848228in}}%
\pgfpathlineto{\pgfqpoint{4.664136in}{2.838773in}}%
\pgfpathlineto{\pgfqpoint{4.676470in}{2.828576in}}%
\pgfpathlineto{\pgfqpoint{4.687502in}{2.816585in}}%
\pgfpathlineto{\pgfqpoint{4.697051in}{2.803027in}}%
\pgfpathlineto{\pgfqpoint{4.706194in}{2.786098in}}%
\pgfpathlineto{\pgfqpoint{4.714508in}{2.765827in}}%
\pgfpathlineto{\pgfqpoint{4.722462in}{2.740013in}}%
\pgfpathlineto{\pgfqpoint{4.729577in}{2.708703in}}%
\pgfpathlineto{\pgfqpoint{4.736162in}{2.669601in}}%
\pgfpathlineto{\pgfqpoint{4.742419in}{2.617826in}}%
\pgfpathlineto{\pgfqpoint{4.747859in}{2.553410in}}%
\pgfpathlineto{\pgfqpoint{4.752661in}{2.468958in}}%
\pgfpathlineto{\pgfqpoint{4.756610in}{2.359528in}}%
\pgfpathlineto{\pgfqpoint{4.759416in}{2.217681in}}%
\pgfpathlineto{\pgfqpoint{4.760596in}{2.043444in}}%
\pgfpathlineto{\pgfqpoint{4.759662in}{1.851779in}}%
\pgfpathlineto{\pgfqpoint{4.756587in}{1.667613in}}%
\pgfpathlineto{\pgfqpoint{4.751596in}{1.503428in}}%
\pgfpathlineto{\pgfqpoint{4.745410in}{1.374185in}}%
\pgfpathlineto{\pgfqpoint{4.738113in}{1.267479in}}%
\pgfpathlineto{\pgfqpoint{4.729621in}{1.175896in}}%
\pgfpathlineto{\pgfqpoint{4.720762in}{1.104428in}}%
\pgfpathlineto{\pgfqpoint{4.711045in}{1.043204in}}%
\pgfpathlineto{\pgfqpoint{4.700364in}{0.989829in}}%
\pgfpathlineto{\pgfqpoint{4.689055in}{0.944345in}}%
\pgfpathlineto{\pgfqpoint{4.676881in}{0.904394in}}%
\pgfpathlineto{\pgfqpoint{4.676095in}{0.902073in}}%
\pgfpathlineto{\pgfqpoint{4.676095in}{0.902073in}}%
\pgfusepath{stroke}%
\end{pgfscope}%
\begin{pgfscope}%
\pgfpathrectangle{\pgfqpoint{0.448634in}{0.402556in}}{\pgfqpoint{4.350661in}{2.489204in}} %
\pgfusepath{clip}%
\pgfsetrectcap%
\pgfsetroundjoin%
\pgfsetlinewidth{1.003750pt}%
\definecolor{currentstroke}{rgb}{1.000000,0.388235,0.278431}%
\pgfsetstrokecolor{currentstroke}%
\pgfsetdash{}{0pt}%
\pgfpathmoveto{\pgfqpoint{2.795520in}{1.982745in}}%
\pgfpathlineto{\pgfqpoint{2.781780in}{1.874357in}}%
\pgfpathlineto{\pgfqpoint{2.769351in}{1.758234in}}%
\pgfpathlineto{\pgfqpoint{2.758095in}{1.631942in}}%
\pgfpathlineto{\pgfqpoint{2.747786in}{1.490551in}}%
\pgfpathlineto{\pgfqpoint{2.738644in}{1.334082in}}%
\pgfpathlineto{\pgfqpoint{2.730580in}{1.157591in}}%
\pgfpathlineto{\pgfqpoint{2.723334in}{0.948663in}}%
\pgfpathlineto{\pgfqpoint{2.709783in}{0.530788in}}%
\pgfpathlineto{\pgfqpoint{2.705868in}{0.488716in}}%
\pgfpathlineto{\pgfqpoint{2.701769in}{0.464281in}}%
\pgfpathlineto{\pgfqpoint{2.697021in}{0.447744in}}%
\pgfpathlineto{\pgfqpoint{2.691859in}{0.436812in}}%
\pgfpathlineto{\pgfqpoint{2.686245in}{0.429229in}}%
\pgfpathlineto{\pgfqpoint{2.679348in}{0.423188in}}%
\pgfpathlineto{\pgfqpoint{2.669540in}{0.417856in}}%
\pgfpathlineto{\pgfqpoint{2.656987in}{0.413810in}}%
\pgfpathlineto{\pgfqpoint{2.637654in}{0.410337in}}%
\pgfpathlineto{\pgfqpoint{2.607297in}{0.407617in}}%
\pgfpathlineto{\pgfqpoint{2.555121in}{0.405574in}}%
\pgfpathlineto{\pgfqpoint{2.450714in}{0.404139in}}%
\pgfpathlineto{\pgfqpoint{2.176624in}{0.403275in}}%
\pgfpathlineto{\pgfqpoint{1.130290in}{0.402953in}}%
\pgfpathlineto{\pgfqpoint{0.516849in}{0.404175in}}%
\pgfpathlineto{\pgfqpoint{0.466848in}{0.405970in}}%
\pgfpathlineto{\pgfqpoint{0.456130in}{0.407931in}}%
\pgfpathlineto{\pgfqpoint{0.452340in}{0.410303in}}%
\pgfpathlineto{\pgfqpoint{0.450346in}{0.414662in}}%
\pgfpathlineto{\pgfqpoint{0.449266in}{0.424524in}}%
\pgfpathlineto{\pgfqpoint{0.448771in}{0.464344in}}%
\pgfpathlineto{\pgfqpoint{0.448640in}{0.850171in}}%
\pgfpathlineto{\pgfqpoint{0.448653in}{2.891318in}}%
\pgfpathlineto{\pgfqpoint{0.448653in}{2.891318in}}%
\pgfusepath{stroke}%
\end{pgfscope}%
\begin{pgfscope}%
\pgfpathrectangle{\pgfqpoint{0.448634in}{0.402556in}}{\pgfqpoint{4.350661in}{2.489204in}} %
\pgfusepath{clip}%
\pgfsetrectcap%
\pgfsetroundjoin%
\pgfsetlinewidth{1.003750pt}%
\definecolor{currentstroke}{rgb}{1.000000,0.388235,0.278431}%
\pgfsetstrokecolor{currentstroke}%
\pgfsetdash{}{0pt}%
\pgfpathmoveto{\pgfqpoint{3.428189in}{0.402586in}}%
\pgfpathlineto{\pgfqpoint{2.782121in}{0.403701in}}%
\pgfpathlineto{\pgfqpoint{2.753906in}{0.405674in}}%
\pgfpathlineto{\pgfqpoint{2.743328in}{0.408443in}}%
\pgfpathlineto{\pgfqpoint{2.737717in}{0.412188in}}%
\pgfpathlineto{\pgfqpoint{2.733668in}{0.417995in}}%
\pgfpathlineto{\pgfqpoint{2.730649in}{0.427307in}}%
\pgfpathlineto{\pgfqpoint{2.728388in}{0.442004in}}%
\pgfpathlineto{\pgfqpoint{2.726544in}{0.471794in}}%
\pgfpathlineto{\pgfqpoint{2.725216in}{0.534003in}}%
\pgfpathlineto{\pgfqpoint{2.725169in}{0.655973in}}%
\pgfpathlineto{\pgfqpoint{2.727377in}{0.832687in}}%
\pgfpathlineto{\pgfqpoint{2.732259in}{1.041703in}}%
\pgfpathlineto{\pgfqpoint{2.738851in}{1.223257in}}%
\pgfpathlineto{\pgfqpoint{2.747078in}{1.389766in}}%
\pgfpathlineto{\pgfqpoint{2.756608in}{1.538717in}}%
\pgfpathlineto{\pgfqpoint{2.768955in}{1.694887in}}%
\pgfpathlineto{\pgfqpoint{2.781228in}{1.816044in}}%
\pgfpathlineto{\pgfqpoint{2.794401in}{1.924524in}}%
\pgfpathlineto{\pgfqpoint{2.812737in}{2.054722in}}%
\pgfpathlineto{\pgfqpoint{2.828774in}{2.147512in}}%
\pgfpathlineto{\pgfqpoint{2.847382in}{2.242224in}}%
\pgfpathlineto{\pgfqpoint{2.895818in}{2.479699in}}%
\pgfpathlineto{\pgfqpoint{2.900204in}{2.516689in}}%
\pgfpathlineto{\pgfqpoint{2.901346in}{2.544029in}}%
\pgfpathlineto{\pgfqpoint{2.900291in}{2.566388in}}%
\pgfpathlineto{\pgfqpoint{2.897334in}{2.585999in}}%
\pgfpathlineto{\pgfqpoint{2.892836in}{2.602633in}}%
\pgfpathlineto{\pgfqpoint{2.886394in}{2.618405in}}%
\pgfpathlineto{\pgfqpoint{2.878058in}{2.632969in}}%
\pgfpathlineto{\pgfqpoint{2.868065in}{2.646100in}}%
\pgfpathlineto{\pgfqpoint{2.855050in}{2.659300in}}%
\pgfpathlineto{\pgfqpoint{2.840801in}{2.670717in}}%
\pgfpathlineto{\pgfqpoint{2.821822in}{2.682861in}}%
\pgfpathlineto{\pgfqpoint{2.799980in}{2.694026in}}%
\pgfpathlineto{\pgfqpoint{2.773366in}{2.704944in}}%
\pgfpathlineto{\pgfqpoint{2.742012in}{2.715266in}}%
\pgfpathlineto{\pgfqpoint{2.705983in}{2.724785in}}%
\pgfpathlineto{\pgfqpoint{2.663200in}{2.733810in}}%
\pgfpathlineto{\pgfqpoint{2.611535in}{2.742379in}}%
\pgfpathlineto{\pgfqpoint{2.551002in}{2.750090in}}%
\pgfpathlineto{\pgfqpoint{2.481632in}{2.756682in}}%
\pgfpathlineto{\pgfqpoint{2.399112in}{2.762200in}}%
\pgfpathlineto{\pgfqpoint{2.309985in}{2.765886in}}%
\pgfpathlineto{\pgfqpoint{2.188184in}{2.768096in}}%
\pgfpathlineto{\pgfqpoint{2.081595in}{2.767619in}}%
\pgfpathlineto{\pgfqpoint{1.968506in}{2.764840in}}%
\pgfpathlineto{\pgfqpoint{1.864180in}{2.759918in}}%
\pgfpathlineto{\pgfqpoint{1.757786in}{2.752593in}}%
\pgfpathlineto{\pgfqpoint{1.671087in}{2.744171in}}%
\pgfpathlineto{\pgfqpoint{1.591076in}{2.734193in}}%
\pgfpathlineto{\pgfqpoint{1.502689in}{2.720717in}}%
\pgfpathlineto{\pgfqpoint{1.427655in}{2.706083in}}%
\pgfpathlineto{\pgfqpoint{1.372350in}{2.692544in}}%
\pgfpathlineto{\pgfqpoint{1.321734in}{2.677921in}}%
\pgfpathlineto{\pgfqpoint{1.273765in}{2.661664in}}%
\pgfpathlineto{\pgfqpoint{1.230567in}{2.644672in}}%
\pgfpathlineto{\pgfqpoint{1.192197in}{2.627106in}}%
\pgfpathlineto{\pgfqpoint{1.156620in}{2.608403in}}%
\pgfpathlineto{\pgfqpoint{1.123890in}{2.588716in}}%
\pgfpathlineto{\pgfqpoint{1.095883in}{2.569568in}}%
\pgfpathlineto{\pgfqpoint{1.063936in}{2.543701in}}%
\pgfpathlineto{\pgfqpoint{1.038217in}{2.520732in}}%
\pgfpathlineto{\pgfqpoint{1.013766in}{2.496016in}}%
\pgfpathlineto{\pgfqpoint{0.990704in}{2.469610in}}%
\pgfpathlineto{\pgfqpoint{0.969124in}{2.441612in}}%
\pgfpathlineto{\pgfqpoint{0.949083in}{2.412154in}}%
\pgfpathlineto{\pgfqpoint{0.930604in}{2.381387in}}%
\pgfpathlineto{\pgfqpoint{0.906555in}{2.334052in}}%
\pgfpathlineto{\pgfqpoint{0.889925in}{2.296262in}}%
\pgfpathlineto{\pgfqpoint{0.874241in}{2.255213in}}%
\pgfpathlineto{\pgfqpoint{0.859667in}{2.210961in}}%
\pgfpathlineto{\pgfqpoint{0.846986in}{2.165954in}}%
\pgfpathlineto{\pgfqpoint{0.839633in}{2.134715in}}%
\pgfpathlineto{\pgfqpoint{0.828238in}{2.081532in}}%
\pgfpathlineto{\pgfqpoint{0.817866in}{2.022986in}}%
\pgfpathlineto{\pgfqpoint{0.810784in}{1.971352in}}%
\pgfpathlineto{\pgfqpoint{0.802846in}{1.902252in}}%
\pgfpathlineto{\pgfqpoint{0.796554in}{1.827927in}}%
\pgfpathlineto{\pgfqpoint{0.791696in}{1.743480in}}%
\pgfpathlineto{\pgfqpoint{0.787773in}{1.621595in}}%
\pgfpathlineto{\pgfqpoint{0.785408in}{1.522064in}}%
\pgfpathlineto{\pgfqpoint{0.785408in}{1.522064in}}%
\pgfusepath{stroke}%
\end{pgfscope}%
\begin{pgfscope}%
\pgfpathrectangle{\pgfqpoint{0.448634in}{0.402556in}}{\pgfqpoint{4.350661in}{2.489204in}} %
\pgfusepath{clip}%
\pgfsetrectcap%
\pgfsetroundjoin%
\pgfsetlinewidth{1.003750pt}%
\definecolor{currentstroke}{rgb}{1.000000,0.388235,0.278431}%
\pgfsetstrokecolor{currentstroke}%
\pgfsetdash{}{0pt}%
\pgfpathmoveto{\pgfqpoint{2.028735in}{0.425754in}}%
\pgfpathlineto{\pgfqpoint{1.878677in}{0.421879in}}%
\pgfpathlineto{\pgfqpoint{1.676387in}{0.418997in}}%
\pgfpathlineto{\pgfqpoint{1.413176in}{0.417558in}}%
\pgfpathlineto{\pgfqpoint{1.134735in}{0.418204in}}%
\pgfpathlineto{\pgfqpoint{0.921565in}{0.420769in}}%
\pgfpathlineto{\pgfqpoint{0.782384in}{0.424523in}}%
\pgfpathlineto{\pgfqpoint{0.693283in}{0.428974in}}%
\pgfpathlineto{\pgfqpoint{0.632541in}{0.434091in}}%
\pgfpathlineto{\pgfqpoint{0.591492in}{0.439564in}}%
\pgfpathlineto{\pgfqpoint{0.561503in}{0.445595in}}%
\pgfpathlineto{\pgfqpoint{0.538349in}{0.452466in}}%
\pgfpathlineto{\pgfqpoint{0.522042in}{0.459394in}}%
\pgfpathlineto{\pgfqpoint{0.508540in}{0.467420in}}%
\pgfpathlineto{\pgfqpoint{0.497973in}{0.476161in}}%
\pgfpathlineto{\pgfqpoint{0.488790in}{0.486750in}}%
\pgfpathlineto{\pgfqpoint{0.481284in}{0.498948in}}%
\pgfpathlineto{\pgfqpoint{0.474590in}{0.514580in}}%
\pgfpathlineto{\pgfqpoint{0.469106in}{0.533467in}}%
\pgfpathlineto{\pgfqpoint{0.464439in}{0.557771in}}%
\pgfpathlineto{\pgfqpoint{0.460297in}{0.592289in}}%
\pgfpathlineto{\pgfqpoint{0.456856in}{0.641912in}}%
\pgfpathlineto{\pgfqpoint{0.454122in}{0.716520in}}%
\pgfpathlineto{\pgfqpoint{0.451978in}{0.843444in}}%
\pgfpathlineto{\pgfqpoint{0.450459in}{1.087380in}}%
\pgfpathlineto{\pgfqpoint{0.449596in}{1.657406in}}%
\pgfpathlineto{\pgfqpoint{0.450150in}{2.687936in}}%
\pgfpathlineto{\pgfqpoint{0.451781in}{2.839761in}}%
\pgfpathlineto{\pgfqpoint{0.453975in}{2.872003in}}%
\pgfpathlineto{\pgfqpoint{0.456339in}{2.881553in}}%
\pgfpathlineto{\pgfqpoint{0.458888in}{2.885549in}}%
\pgfpathlineto{\pgfqpoint{0.462554in}{2.888171in}}%
\pgfpathlineto{\pgfqpoint{0.471046in}{2.890205in}}%
\pgfpathlineto{\pgfqpoint{0.490597in}{2.891263in}}%
\pgfpathlineto{\pgfqpoint{0.564556in}{2.891692in}}%
\pgfpathlineto{\pgfqpoint{1.569559in}{2.891759in}}%
\pgfpathlineto{\pgfqpoint{4.784679in}{2.890785in}}%
\pgfpathlineto{\pgfqpoint{4.791005in}{2.889098in}}%
\pgfpathlineto{\pgfqpoint{4.793910in}{2.885555in}}%
\pgfpathlineto{\pgfqpoint{4.795579in}{2.878366in}}%
\pgfpathlineto{\pgfqpoint{4.796850in}{2.858513in}}%
\pgfpathlineto{\pgfqpoint{4.796850in}{2.858513in}}%
\pgfusepath{stroke}%
\end{pgfscope}%
\begin{pgfscope}%
\pgfpathrectangle{\pgfqpoint{0.448634in}{0.402556in}}{\pgfqpoint{4.350661in}{2.489204in}} %
\pgfusepath{clip}%
\pgfsetrectcap%
\pgfsetroundjoin%
\pgfsetlinewidth{1.003750pt}%
\definecolor{currentstroke}{rgb}{0.121569,0.466667,0.705882}%
\pgfsetstrokecolor{currentstroke}%
\pgfsetdash{}{0pt}%
\pgfpathmoveto{\pgfqpoint{1.127319in}{2.572074in}}%
\pgfpathlineto{\pgfqpoint{1.159575in}{2.592758in}}%
\pgfpathlineto{\pgfqpoint{1.192763in}{2.611414in}}%
\pgfpathlineto{\pgfqpoint{1.228726in}{2.629126in}}%
\pgfpathlineto{\pgfqpoint{1.267413in}{2.645758in}}%
\pgfpathlineto{\pgfqpoint{1.310846in}{2.661945in}}%
\pgfpathlineto{\pgfqpoint{1.356920in}{2.676740in}}%
\pgfpathlineto{\pgfqpoint{1.407680in}{2.690702in}}%
\pgfpathlineto{\pgfqpoint{1.463094in}{2.703640in}}%
\pgfpathlineto{\pgfqpoint{1.525273in}{2.715813in}}%
\pgfpathlineto{\pgfqpoint{1.594199in}{2.726937in}}%
\pgfpathlineto{\pgfqpoint{1.669843in}{2.736808in}}%
\pgfpathlineto{\pgfqpoint{1.752172in}{2.745271in}}%
\pgfpathlineto{\pgfqpoint{1.843325in}{2.752344in}}%
\pgfpathlineto{\pgfqpoint{1.941103in}{2.757656in}}%
\pgfpathlineto{\pgfqpoint{2.043301in}{2.760987in}}%
\pgfpathlineto{\pgfqpoint{2.147710in}{2.762199in}}%
\pgfpathlineto{\pgfqpoint{2.249945in}{2.761215in}}%
\pgfpathlineto{\pgfqpoint{2.345620in}{2.758145in}}%
\pgfpathlineto{\pgfqpoint{2.432525in}{2.753210in}}%
\pgfpathlineto{\pgfqpoint{2.508451in}{2.746766in}}%
\pgfpathlineto{\pgfqpoint{2.573368in}{2.739156in}}%
\pgfpathlineto{\pgfqpoint{2.629410in}{2.730451in}}%
\pgfpathlineto{\pgfqpoint{2.676543in}{2.720985in}}%
\pgfpathlineto{\pgfqpoint{2.716874in}{2.710666in}}%
\pgfpathlineto{\pgfqpoint{2.750366in}{2.699848in}}%
\pgfpathlineto{\pgfqpoint{2.779059in}{2.688192in}}%
\pgfpathlineto{\pgfqpoint{2.802882in}{2.676004in}}%
\pgfpathlineto{\pgfqpoint{2.821842in}{2.663819in}}%
\pgfpathlineto{\pgfqpoint{2.837815in}{2.650886in}}%
\pgfpathlineto{\pgfqpoint{2.850736in}{2.637564in}}%
\pgfpathlineto{\pgfqpoint{2.860694in}{2.624398in}}%
\pgfpathlineto{\pgfqpoint{2.869084in}{2.609873in}}%
\pgfpathlineto{\pgfqpoint{2.875698in}{2.594192in}}%
\pgfpathlineto{\pgfqpoint{2.881035in}{2.575255in}}%
\pgfpathlineto{\pgfqpoint{2.884200in}{2.555685in}}%
\pgfpathlineto{\pgfqpoint{2.885619in}{2.533351in}}%
\pgfpathlineto{\pgfqpoint{2.885038in}{2.505987in}}%
\pgfpathlineto{\pgfqpoint{2.882112in}{2.473807in}}%
\pgfpathlineto{\pgfqpoint{2.875657in}{2.429620in}}%
\pgfpathlineto{\pgfqpoint{2.863489in}{2.363873in}}%
\pgfpathlineto{\pgfqpoint{2.821102in}{2.142619in}}%
\pgfpathlineto{\pgfqpoint{2.804859in}{2.042271in}}%
\pgfpathlineto{\pgfqpoint{2.790421in}{1.939040in}}%
\pgfpathlineto{\pgfqpoint{2.777207in}{1.828054in}}%
\pgfpathlineto{\pgfqpoint{2.765338in}{1.709349in}}%
\pgfpathlineto{\pgfqpoint{2.754471in}{1.578010in}}%
\pgfpathlineto{\pgfqpoint{2.744640in}{1.431580in}}%
\pgfpathlineto{\pgfqpoint{2.735914in}{1.267598in}}%
\pgfpathlineto{\pgfqpoint{2.728277in}{1.081114in}}%
\pgfpathlineto{\pgfqpoint{2.721437in}{0.857223in}}%
\pgfpathlineto{\pgfqpoint{2.711961in}{0.541290in}}%
\pgfpathlineto{\pgfqpoint{2.708250in}{0.491694in}}%
\pgfpathlineto{\pgfqpoint{2.703951in}{0.462246in}}%
\pgfpathlineto{\pgfqpoint{2.699504in}{0.445599in}}%
\pgfpathlineto{\pgfqpoint{2.694517in}{0.434563in}}%
\pgfpathlineto{\pgfqpoint{2.688942in}{0.426947in}}%
\pgfpathlineto{\pgfqpoint{2.681980in}{0.421009in}}%
\pgfpathlineto{\pgfqpoint{2.672064in}{0.415948in}}%
\pgfpathlineto{\pgfqpoint{2.659429in}{0.412247in}}%
\pgfpathlineto{\pgfqpoint{2.640044in}{0.409163in}}%
\pgfpathlineto{\pgfqpoint{2.607490in}{0.406692in}}%
\pgfpathlineto{\pgfqpoint{2.548779in}{0.404894in}}%
\pgfpathlineto{\pgfqpoint{2.422615in}{0.403701in}}%
\pgfpathlineto{\pgfqpoint{2.026705in}{0.403016in}}%
\pgfpathlineto{\pgfqpoint{0.623617in}{0.403253in}}%
\pgfpathlineto{\pgfqpoint{0.477880in}{0.404742in}}%
\pgfpathlineto{\pgfqpoint{0.458368in}{0.406382in}}%
\pgfpathlineto{\pgfqpoint{0.452304in}{0.408937in}}%
\pgfpathlineto{\pgfqpoint{0.450213in}{0.413215in}}%
\pgfpathlineto{\pgfqpoint{0.449165in}{0.423080in}}%
\pgfpathlineto{\pgfqpoint{0.448735in}{0.465392in}}%
\pgfpathlineto{\pgfqpoint{0.448637in}{0.983146in}}%
\pgfpathlineto{\pgfqpoint{0.448652in}{2.889876in}}%
\pgfpathlineto{\pgfqpoint{0.448652in}{2.889876in}}%
\pgfusepath{stroke}%
\end{pgfscope}%
\begin{pgfscope}%
\pgfpathrectangle{\pgfqpoint{0.448634in}{0.402556in}}{\pgfqpoint{4.350661in}{2.489204in}} %
\pgfusepath{clip}%
\pgfsetrectcap%
\pgfsetroundjoin%
\pgfsetlinewidth{1.003750pt}%
\definecolor{currentstroke}{rgb}{0.121569,0.466667,0.705882}%
\pgfsetstrokecolor{currentstroke}%
\pgfsetdash{}{0pt}%
\pgfpathmoveto{\pgfqpoint{0.448634in}{2.896245in}}%
\pgfpathlineto{\pgfqpoint{0.448593in}{0.407043in}}%
\pgfpathlineto{\pgfqpoint{0.448593in}{0.407043in}}%
\pgfusepath{stroke}%
\end{pgfscope}%
\begin{pgfscope}%
\pgfpathrectangle{\pgfqpoint{0.448634in}{0.402556in}}{\pgfqpoint{4.350661in}{2.489204in}} %
\pgfusepath{clip}%
\pgfsetrectcap%
\pgfsetroundjoin%
\pgfsetlinewidth{1.003750pt}%
\definecolor{currentstroke}{rgb}{0.121569,0.466667,0.705882}%
\pgfsetstrokecolor{currentstroke}%
\pgfsetdash{}{0pt}%
\pgfpathmoveto{\pgfqpoint{0.576853in}{1.760817in}}%
\pgfpathlineto{\pgfqpoint{0.569394in}{1.840010in}}%
\pgfpathlineto{\pgfqpoint{0.563209in}{1.929338in}}%
\pgfpathlineto{\pgfqpoint{0.558592in}{2.028764in}}%
\pgfpathlineto{\pgfqpoint{0.555985in}{2.133265in}}%
\pgfpathlineto{\pgfqpoint{0.555566in}{2.237808in}}%
\pgfpathlineto{\pgfqpoint{0.557371in}{2.337352in}}%
\pgfpathlineto{\pgfqpoint{0.561096in}{2.424366in}}%
\pgfpathlineto{\pgfqpoint{0.566403in}{2.498791in}}%
\pgfpathlineto{\pgfqpoint{0.572909in}{2.560570in}}%
\pgfpathlineto{\pgfqpoint{0.580458in}{2.612119in}}%
\pgfpathlineto{\pgfqpoint{0.589086in}{2.655816in}}%
\pgfpathlineto{\pgfqpoint{0.598406in}{2.691589in}}%
\pgfpathlineto{\pgfqpoint{0.608613in}{2.721757in}}%
\pgfpathlineto{\pgfqpoint{0.619241in}{2.746278in}}%
\pgfpathlineto{\pgfqpoint{0.630817in}{2.767339in}}%
\pgfpathlineto{\pgfqpoint{0.642975in}{2.784884in}}%
\pgfpathlineto{\pgfqpoint{0.656813in}{2.800712in}}%
\pgfpathlineto{\pgfqpoint{0.672197in}{2.814549in}}%
\pgfpathlineto{\pgfqpoint{0.688853in}{2.826301in}}%
\pgfpathlineto{\pgfqpoint{0.706461in}{2.836076in}}%
\pgfpathlineto{\pgfqpoint{0.726804in}{2.844875in}}%
\pgfpathlineto{\pgfqpoint{0.751866in}{2.853203in}}%
\pgfpathlineto{\pgfqpoint{0.781631in}{2.860547in}}%
\pgfpathlineto{\pgfqpoint{0.818168in}{2.867054in}}%
\pgfpathlineto{\pgfqpoint{0.863581in}{2.872685in}}%
\pgfpathlineto{\pgfqpoint{0.922161in}{2.877518in}}%
\pgfpathlineto{\pgfqpoint{1.000391in}{2.881567in}}%
\pgfpathlineto{\pgfqpoint{1.111294in}{2.884881in}}%
\pgfpathlineto{\pgfqpoint{1.274428in}{2.887367in}}%
\pgfpathlineto{\pgfqpoint{1.552865in}{2.889263in}}%
\pgfpathlineto{\pgfqpoint{2.107573in}{2.890457in}}%
\pgfpathlineto{\pgfqpoint{3.343161in}{2.890573in}}%
\pgfpathlineto{\pgfqpoint{4.043615in}{2.888941in}}%
\pgfpathlineto{\pgfqpoint{4.289417in}{2.886404in}}%
\pgfpathlineto{\pgfqpoint{4.413375in}{2.883093in}}%
\pgfpathlineto{\pgfqpoint{4.489424in}{2.878997in}}%
\pgfpathlineto{\pgfqpoint{4.541451in}{2.874081in}}%
\pgfpathlineto{\pgfqpoint{4.578100in}{2.868470in}}%
\pgfpathlineto{\pgfqpoint{4.605818in}{2.862092in}}%
\pgfpathlineto{\pgfqpoint{4.626725in}{2.855245in}}%
\pgfpathlineto{\pgfqpoint{4.644925in}{2.847018in}}%
\pgfpathlineto{\pgfqpoint{4.660241in}{2.837590in}}%
\pgfpathlineto{\pgfqpoint{4.672623in}{2.827468in}}%
\pgfpathlineto{\pgfqpoint{4.683751in}{2.815592in}}%
\pgfpathlineto{\pgfqpoint{4.693406in}{2.802135in}}%
\pgfpathlineto{\pgfqpoint{4.702740in}{2.785343in}}%
\pgfpathlineto{\pgfqpoint{4.711277in}{2.765194in}}%
\pgfpathlineto{\pgfqpoint{4.719482in}{2.739484in}}%
\pgfpathlineto{\pgfqpoint{4.726293in}{2.710657in}}%
\pgfpathlineto{\pgfqpoint{4.733259in}{2.671643in}}%
\pgfpathlineto{\pgfqpoint{4.739604in}{2.622396in}}%
\pgfpathlineto{\pgfqpoint{4.745236in}{2.560504in}}%
\pgfpathlineto{\pgfqpoint{4.750164in}{2.481052in}}%
\pgfpathlineto{\pgfqpoint{4.754367in}{2.376618in}}%
\pgfpathlineto{\pgfqpoint{4.757443in}{2.242249in}}%
\pgfpathlineto{\pgfqpoint{4.758977in}{2.075483in}}%
\pgfpathlineto{\pgfqpoint{4.758447in}{1.888795in}}%
\pgfpathlineto{\pgfqpoint{4.755756in}{1.707111in}}%
\pgfpathlineto{\pgfqpoint{4.750925in}{1.532957in}}%
\pgfpathlineto{\pgfqpoint{4.744785in}{1.398727in}}%
\pgfpathlineto{\pgfqpoint{4.737575in}{1.289516in}}%
\pgfpathlineto{\pgfqpoint{4.728714in}{1.190470in}}%
\pgfpathlineto{\pgfqpoint{4.719652in}{1.116521in}}%
\pgfpathlineto{\pgfqpoint{4.710036in}{1.055276in}}%
\pgfpathlineto{\pgfqpoint{4.699503in}{1.001861in}}%
\pgfpathlineto{\pgfqpoint{4.689040in}{0.958690in}}%
\pgfpathlineto{\pgfqpoint{4.677219in}{0.918600in}}%
\pgfpathlineto{\pgfqpoint{4.664034in}{0.881749in}}%
\pgfpathlineto{\pgfqpoint{4.650584in}{0.850492in}}%
\pgfpathlineto{\pgfqpoint{4.636303in}{0.822570in}}%
\pgfpathlineto{\pgfqpoint{4.620207in}{0.795974in}}%
\pgfpathlineto{\pgfqpoint{4.603640in}{0.772901in}}%
\pgfpathlineto{\pgfqpoint{4.585488in}{0.751446in}}%
\pgfpathlineto{\pgfqpoint{4.565874in}{0.731749in}}%
\pgfpathlineto{\pgfqpoint{4.544964in}{0.713879in}}%
\pgfpathlineto{\pgfqpoint{4.522958in}{0.697824in}}%
\pgfpathlineto{\pgfqpoint{4.496157in}{0.681290in}}%
\pgfpathlineto{\pgfqpoint{4.470397in}{0.667953in}}%
\pgfpathlineto{\pgfqpoint{4.439961in}{0.654509in}}%
\pgfpathlineto{\pgfqpoint{4.406841in}{0.642281in}}%
\pgfpathlineto{\pgfqpoint{4.369009in}{0.630748in}}%
\pgfpathlineto{\pgfqpoint{4.326489in}{0.620226in}}%
\pgfpathlineto{\pgfqpoint{4.279327in}{0.610949in}}%
\pgfpathlineto{\pgfqpoint{4.227576in}{0.603085in}}%
\pgfpathlineto{\pgfqpoint{4.173450in}{0.597063in}}%
\pgfpathlineto{\pgfqpoint{4.110511in}{0.592203in}}%
\pgfpathlineto{\pgfqpoint{4.047471in}{0.589537in}}%
\pgfpathlineto{\pgfqpoint{3.977867in}{0.588624in}}%
\pgfpathlineto{\pgfqpoint{3.906093in}{0.589934in}}%
\pgfpathlineto{\pgfqpoint{3.834377in}{0.593496in}}%
\pgfpathlineto{\pgfqpoint{3.767120in}{0.599067in}}%
\pgfpathlineto{\pgfqpoint{3.704364in}{0.606392in}}%
\pgfpathlineto{\pgfqpoint{3.678516in}{0.610510in}}%
\pgfpathlineto{\pgfqpoint{3.620438in}{0.620500in}}%
\pgfpathlineto{\pgfqpoint{3.586319in}{0.628207in}}%
\pgfpathlineto{\pgfqpoint{3.495240in}{0.652428in}}%
\pgfpathlineto{\pgfqpoint{3.451528in}{0.667583in}}%
\pgfpathlineto{\pgfqpoint{3.408538in}{0.685220in}}%
\pgfpathlineto{\pgfqpoint{3.374594in}{0.702001in}}%
\pgfpathlineto{\pgfqpoint{3.345407in}{0.718682in}}%
\pgfpathlineto{\pgfqpoint{3.315236in}{0.738520in}}%
\pgfpathlineto{\pgfqpoint{3.288127in}{0.759290in}}%
\pgfpathlineto{\pgfqpoint{3.264004in}{0.780551in}}%
\pgfpathlineto{\pgfqpoint{3.241208in}{0.803648in}}%
\pgfpathlineto{\pgfqpoint{3.219894in}{0.828530in}}%
\pgfpathlineto{\pgfqpoint{3.200189in}{0.855091in}}%
\pgfpathlineto{\pgfqpoint{3.182177in}{0.883182in}}%
\pgfpathlineto{\pgfqpoint{3.165906in}{0.912633in}}%
\pgfpathlineto{\pgfqpoint{3.150351in}{0.945448in}}%
\pgfpathlineto{\pgfqpoint{3.136682in}{0.979345in}}%
\pgfpathlineto{\pgfqpoint{3.124073in}{1.016460in}}%
\pgfpathlineto{\pgfqpoint{3.112834in}{1.056769in}}%
\pgfpathlineto{\pgfqpoint{3.103046in}{1.100146in}}%
\pgfpathlineto{\pgfqpoint{3.095343in}{1.144071in}}%
\pgfpathlineto{\pgfqpoint{3.089208in}{1.190837in}}%
\pgfpathlineto{\pgfqpoint{3.084595in}{1.242838in}}%
\pgfpathlineto{\pgfqpoint{3.082137in}{1.295031in}}%
\pgfpathlineto{\pgfqpoint{3.081687in}{1.349787in}}%
\pgfpathlineto{\pgfqpoint{3.083451in}{1.406998in}}%
\pgfpathlineto{\pgfqpoint{3.087181in}{1.461589in}}%
\pgfpathlineto{\pgfqpoint{3.093485in}{1.520888in}}%
\pgfpathlineto{\pgfqpoint{3.101823in}{1.577334in}}%
\pgfpathlineto{\pgfqpoint{3.111930in}{1.630856in}}%
\pgfpathlineto{\pgfqpoint{3.124690in}{1.686208in}}%
\pgfpathlineto{\pgfqpoint{3.139178in}{1.738395in}}%
\pgfpathlineto{\pgfqpoint{3.155145in}{1.787366in}}%
\pgfpathlineto{\pgfqpoint{3.172353in}{1.833085in}}%
\pgfpathlineto{\pgfqpoint{3.191618in}{1.877716in}}%
\pgfpathlineto{\pgfqpoint{3.214026in}{1.923261in}}%
\pgfpathlineto{\pgfqpoint{3.236214in}{1.963157in}}%
\pgfpathlineto{\pgfqpoint{3.260178in}{2.001684in}}%
\pgfpathlineto{\pgfqpoint{3.285814in}{2.038776in}}%
\pgfpathlineto{\pgfqpoint{3.314415in}{2.076285in}}%
\pgfpathlineto{\pgfqpoint{3.348944in}{2.117711in}}%
\pgfpathlineto{\pgfqpoint{3.417133in}{2.198022in}}%
\pgfpathlineto{\pgfqpoint{3.426053in}{2.212128in}}%
\pgfpathlineto{\pgfqpoint{3.430798in}{2.223297in}}%
\pgfpathlineto{\pgfqpoint{3.432034in}{2.230603in}}%
\pgfpathlineto{\pgfqpoint{3.430773in}{2.237856in}}%
\pgfpathlineto{\pgfqpoint{3.426621in}{2.243526in}}%
\pgfpathlineto{\pgfqpoint{3.420908in}{2.247084in}}%
\pgfpathlineto{\pgfqpoint{3.412501in}{2.249583in}}%
\pgfpathlineto{\pgfqpoint{3.399499in}{2.250689in}}%
\pgfpathlineto{\pgfqpoint{3.384305in}{2.249671in}}%
\pgfpathlineto{\pgfqpoint{3.364985in}{2.246098in}}%
\pgfpathlineto{\pgfqpoint{3.341804in}{2.239342in}}%
\pgfpathlineto{\pgfqpoint{3.317109in}{2.229682in}}%
\pgfpathlineto{\pgfqpoint{3.291104in}{2.216986in}}%
\pgfpathlineto{\pgfqpoint{3.265928in}{2.202261in}}%
\pgfpathlineto{\pgfqpoint{3.239805in}{2.184361in}}%
\pgfpathlineto{\pgfqpoint{3.214775in}{2.164519in}}%
\pgfpathlineto{\pgfqpoint{3.190900in}{2.142893in}}%
\pgfpathlineto{\pgfqpoint{3.166657in}{2.117912in}}%
\pgfpathlineto{\pgfqpoint{3.143835in}{2.091233in}}%
\pgfpathlineto{\pgfqpoint{3.121079in}{2.061107in}}%
\pgfpathlineto{\pgfqpoint{3.099952in}{2.029463in}}%
\pgfpathlineto{\pgfqpoint{3.079251in}{1.994406in}}%
\pgfpathlineto{\pgfqpoint{3.059218in}{1.955915in}}%
\pgfpathlineto{\pgfqpoint{3.040058in}{1.914015in}}%
\pgfpathlineto{\pgfqpoint{3.022809in}{1.871041in}}%
\pgfpathlineto{\pgfqpoint{3.005790in}{1.822536in}}%
\pgfpathlineto{\pgfqpoint{2.990067in}{1.770819in}}%
\pgfpathlineto{\pgfqpoint{2.975708in}{1.715979in}}%
\pgfpathlineto{\pgfqpoint{2.962284in}{1.655680in}}%
\pgfpathlineto{\pgfqpoint{2.950496in}{1.592386in}}%
\pgfpathlineto{\pgfqpoint{2.940383in}{1.526185in}}%
\pgfpathlineto{\pgfqpoint{2.931745in}{1.454681in}}%
\pgfpathlineto{\pgfqpoint{2.925082in}{1.380399in}}%
\pgfpathlineto{\pgfqpoint{2.920647in}{1.305899in}}%
\pgfpathlineto{\pgfqpoint{2.918444in}{1.231270in}}%
\pgfpathlineto{\pgfqpoint{2.918545in}{1.159087in}}%
\pgfpathlineto{\pgfqpoint{2.920787in}{1.091931in}}%
\pgfpathlineto{\pgfqpoint{2.925177in}{1.027412in}}%
\pgfpathlineto{\pgfqpoint{2.931192in}{0.970580in}}%
\pgfpathlineto{\pgfqpoint{2.938760in}{0.919034in}}%
\pgfpathlineto{\pgfqpoint{2.947651in}{0.872852in}}%
\pgfpathlineto{\pgfqpoint{2.958213in}{0.829714in}}%
\pgfpathlineto{\pgfqpoint{2.969670in}{0.792114in}}%
\pgfpathlineto{\pgfqpoint{2.982463in}{0.757773in}}%
\pgfpathlineto{\pgfqpoint{2.996425in}{0.726812in}}%
\pgfpathlineto{\pgfqpoint{3.011299in}{0.699300in}}%
\pgfpathlineto{\pgfqpoint{3.026739in}{0.675225in}}%
\pgfpathlineto{\pgfqpoint{3.043828in}{0.652656in}}%
\pgfpathlineto{\pgfqpoint{3.062495in}{0.631788in}}%
\pgfpathlineto{\pgfqpoint{3.082602in}{0.612753in}}%
\pgfpathlineto{\pgfqpoint{3.103961in}{0.595592in}}%
\pgfpathlineto{\pgfqpoint{3.128268in}{0.579069in}}%
\pgfpathlineto{\pgfqpoint{3.153537in}{0.564554in}}%
\pgfpathlineto{\pgfqpoint{3.181571in}{0.550952in}}%
\pgfpathlineto{\pgfqpoint{3.214371in}{0.537647in}}%
\pgfpathlineto{\pgfqpoint{3.249846in}{0.525712in}}%
\pgfpathlineto{\pgfqpoint{3.290011in}{0.514571in}}%
\pgfpathlineto{\pgfqpoint{3.334820in}{0.504423in}}%
\pgfpathlineto{\pgfqpoint{3.386372in}{0.494999in}}%
\pgfpathlineto{\pgfqpoint{3.446798in}{0.486257in}}%
\pgfpathlineto{\pgfqpoint{3.518243in}{0.478282in}}%
\pgfpathlineto{\pgfqpoint{3.600685in}{0.471409in}}%
\pgfpathlineto{\pgfqpoint{3.696268in}{0.465713in}}%
\pgfpathlineto{\pgfqpoint{3.807144in}{0.461369in}}%
\pgfpathlineto{\pgfqpoint{3.933291in}{0.458719in}}%
\pgfpathlineto{\pgfqpoint{4.063808in}{0.458211in}}%
\pgfpathlineto{\pgfqpoint{4.187792in}{0.459914in}}%
\pgfpathlineto{\pgfqpoint{4.294335in}{0.463521in}}%
\pgfpathlineto{\pgfqpoint{4.381234in}{0.468574in}}%
\pgfpathlineto{\pgfqpoint{4.450636in}{0.474701in}}%
\pgfpathlineto{\pgfqpoint{4.506850in}{0.481799in}}%
\pgfpathlineto{\pgfqpoint{4.552009in}{0.489658in}}%
\pgfpathlineto{\pgfqpoint{4.588239in}{0.498115in}}%
\pgfpathlineto{\pgfqpoint{4.617656in}{0.507110in}}%
\pgfpathlineto{\pgfqpoint{4.642328in}{0.516843in}}%
\pgfpathlineto{\pgfqpoint{4.664194in}{0.527940in}}%
\pgfpathlineto{\pgfqpoint{4.681238in}{0.538945in}}%
\pgfpathlineto{\pgfqpoint{4.697164in}{0.551953in}}%
\pgfpathlineto{\pgfqpoint{4.710076in}{0.565289in}}%
\pgfpathlineto{\pgfqpoint{4.721578in}{0.580218in}}%
\pgfpathlineto{\pgfqpoint{4.731557in}{0.596521in}}%
\pgfpathlineto{\pgfqpoint{4.741000in}{0.616134in}}%
\pgfpathlineto{\pgfqpoint{4.749521in}{0.639027in}}%
\pgfpathlineto{\pgfqpoint{4.757522in}{0.667450in}}%
\pgfpathlineto{\pgfqpoint{4.764572in}{0.701345in}}%
\pgfpathlineto{\pgfqpoint{4.770840in}{0.743043in}}%
\pgfpathlineto{\pgfqpoint{4.776327in}{0.794934in}}%
\pgfpathlineto{\pgfqpoint{4.781278in}{0.864398in}}%
\pgfpathlineto{\pgfqpoint{4.785468in}{0.956371in}}%
\pgfpathlineto{\pgfqpoint{4.789000in}{1.085745in}}%
\pgfpathlineto{\pgfqpoint{4.791852in}{1.277385in}}%
\pgfpathlineto{\pgfqpoint{4.793959in}{1.581057in}}%
\pgfpathlineto{\pgfqpoint{4.794962in}{2.071429in}}%
\pgfpathlineto{\pgfqpoint{4.793967in}{2.559311in}}%
\pgfpathlineto{\pgfqpoint{4.791733in}{2.745981in}}%
\pgfpathlineto{\pgfqpoint{4.788955in}{2.818091in}}%
\pgfpathlineto{\pgfqpoint{4.785731in}{2.850227in}}%
\pgfpathlineto{\pgfqpoint{4.781879in}{2.867057in}}%
\pgfpathlineto{\pgfqpoint{4.777744in}{2.875780in}}%
\pgfpathlineto{\pgfqpoint{4.773097in}{2.880982in}}%
\pgfpathlineto{\pgfqpoint{4.767363in}{2.884504in}}%
\pgfpathlineto{\pgfqpoint{4.756853in}{2.887622in}}%
\pgfpathlineto{\pgfqpoint{4.739548in}{2.889639in}}%
\pgfpathlineto{\pgfqpoint{4.704762in}{2.890882in}}%
\pgfpathlineto{\pgfqpoint{4.602524in}{2.891538in}}%
\pgfpathlineto{\pgfqpoint{3.952100in}{2.891742in}}%
\pgfpathlineto{\pgfqpoint{0.617321in}{2.890753in}}%
\pgfpathlineto{\pgfqpoint{0.549910in}{2.888858in}}%
\pgfpathlineto{\pgfqpoint{0.521735in}{2.886179in}}%
\pgfpathlineto{\pgfqpoint{0.504666in}{2.882389in}}%
\pgfpathlineto{\pgfqpoint{0.494501in}{2.878011in}}%
\pgfpathlineto{\pgfqpoint{0.487180in}{2.872667in}}%
\pgfpathlineto{\pgfqpoint{0.481152in}{2.865519in}}%
\pgfpathlineto{\pgfqpoint{0.475664in}{2.854804in}}%
\pgfpathlineto{\pgfqpoint{0.471318in}{2.840737in}}%
\pgfpathlineto{\pgfqpoint{0.467301in}{2.818823in}}%
\pgfpathlineto{\pgfqpoint{0.463927in}{2.786700in}}%
\pgfpathlineto{\pgfqpoint{0.460918in}{2.734544in}}%
\pgfpathlineto{\pgfqpoint{0.458363in}{2.647473in}}%
\pgfpathlineto{\pgfqpoint{0.456575in}{2.523031in}}%
\pgfpathlineto{\pgfqpoint{0.456575in}{2.523031in}}%
\pgfusepath{stroke}%
\end{pgfscope}%
\begin{pgfscope}%
\pgfpathrectangle{\pgfqpoint{0.448634in}{0.402556in}}{\pgfqpoint{4.350661in}{2.489204in}} %
\pgfusepath{clip}%
\pgfsetrectcap%
\pgfsetroundjoin%
\pgfsetlinewidth{1.003750pt}%
\definecolor{currentstroke}{rgb}{0.121569,0.466667,0.705882}%
\pgfsetstrokecolor{currentstroke}%
\pgfsetdash{}{0pt}%
\pgfpathmoveto{\pgfqpoint{0.456424in}{1.370137in}}%
\pgfpathlineto{\pgfqpoint{0.459610in}{1.118755in}}%
\pgfpathlineto{\pgfqpoint{0.463695in}{0.962007in}}%
\pgfpathlineto{\pgfqpoint{0.468519in}{0.857610in}}%
\pgfpathlineto{\pgfqpoint{0.474082in}{0.783210in}}%
\pgfpathlineto{\pgfqpoint{0.480226in}{0.728906in}}%
\pgfpathlineto{\pgfqpoint{0.486970in}{0.687306in}}%
\pgfpathlineto{\pgfqpoint{0.494537in}{0.653558in}}%
\pgfpathlineto{\pgfqpoint{0.503107in}{0.625355in}}%
\pgfpathlineto{\pgfqpoint{0.512193in}{0.602749in}}%
\pgfpathlineto{\pgfqpoint{0.522200in}{0.583508in}}%
\pgfpathlineto{\pgfqpoint{0.534108in}{0.565743in}}%
\pgfpathlineto{\pgfqpoint{0.546263in}{0.551507in}}%
\pgfpathlineto{\pgfqpoint{0.559728in}{0.538907in}}%
\pgfpathlineto{\pgfqpoint{0.576129in}{0.526693in}}%
\pgfpathlineto{\pgfqpoint{0.595483in}{0.515351in}}%
\pgfpathlineto{\pgfqpoint{0.617681in}{0.505147in}}%
\pgfpathlineto{\pgfqpoint{0.642568in}{0.496153in}}%
\pgfpathlineto{\pgfqpoint{0.672126in}{0.487778in}}%
\pgfpathlineto{\pgfqpoint{0.708443in}{0.479824in}}%
\pgfpathlineto{\pgfqpoint{0.753649in}{0.472325in}}%
\pgfpathlineto{\pgfqpoint{0.807717in}{0.465660in}}%
\pgfpathlineto{\pgfqpoint{0.877116in}{0.459475in}}%
\pgfpathlineto{\pgfqpoint{0.961828in}{0.454230in}}%
\pgfpathlineto{\pgfqpoint{1.068351in}{0.449916in}}%
\pgfpathlineto{\pgfqpoint{1.201018in}{0.446839in}}%
\pgfpathlineto{\pgfqpoint{1.357637in}{0.445481in}}%
\pgfpathlineto{\pgfqpoint{1.525135in}{0.446232in}}%
\pgfpathlineto{\pgfqpoint{1.686088in}{0.449142in}}%
\pgfpathlineto{\pgfqpoint{1.823074in}{0.453747in}}%
\pgfpathlineto{\pgfqpoint{1.938245in}{0.459764in}}%
\pgfpathlineto{\pgfqpoint{2.031582in}{0.466759in}}%
\pgfpathlineto{\pgfqpoint{2.109580in}{0.474745in}}%
\pgfpathlineto{\pgfqpoint{2.174384in}{0.483535in}}%
\pgfpathlineto{\pgfqpoint{2.228139in}{0.492940in}}%
\pgfpathlineto{\pgfqpoint{2.275119in}{0.503356in}}%
\pgfpathlineto{\pgfqpoint{2.315282in}{0.514501in}}%
\pgfpathlineto{\pgfqpoint{2.350698in}{0.526659in}}%
\pgfpathlineto{\pgfqpoint{2.381320in}{0.539536in}}%
\pgfpathlineto{\pgfqpoint{2.407164in}{0.552659in}}%
\pgfpathlineto{\pgfqpoint{2.430226in}{0.566639in}}%
\pgfpathlineto{\pgfqpoint{2.452282in}{0.582602in}}%
\pgfpathlineto{\pgfqpoint{2.471391in}{0.599069in}}%
\pgfpathlineto{\pgfqpoint{2.489240in}{0.617293in}}%
\pgfpathlineto{\pgfqpoint{2.505678in}{0.637180in}}%
\pgfpathlineto{\pgfqpoint{2.520620in}{0.658557in}}%
\pgfpathlineto{\pgfqpoint{2.535213in}{0.683314in}}%
\pgfpathlineto{\pgfqpoint{2.549115in}{0.711484in}}%
\pgfpathlineto{\pgfqpoint{2.562091in}{0.743004in}}%
\pgfpathlineto{\pgfqpoint{2.574020in}{0.777751in}}%
\pgfpathlineto{\pgfqpoint{2.585502in}{0.817970in}}%
\pgfpathlineto{\pgfqpoint{2.596809in}{0.866038in}}%
\pgfpathlineto{\pgfqpoint{2.607562in}{0.921948in}}%
\pgfpathlineto{\pgfqpoint{2.617925in}{0.988098in}}%
\pgfpathlineto{\pgfqpoint{2.627958in}{1.066918in}}%
\pgfpathlineto{\pgfqpoint{2.637941in}{1.163320in}}%
\pgfpathlineto{\pgfqpoint{2.648424in}{1.287199in}}%
\pgfpathlineto{\pgfqpoint{2.660103in}{1.453438in}}%
\pgfpathlineto{\pgfqpoint{2.674773in}{1.696801in}}%
\pgfpathlineto{\pgfqpoint{2.687716in}{1.945279in}}%
\pgfpathlineto{\pgfqpoint{2.692670in}{2.079573in}}%
\pgfpathlineto{\pgfqpoint{2.693829in}{2.166682in}}%
\pgfpathlineto{\pgfqpoint{2.692565in}{2.233870in}}%
\pgfpathlineto{\pgfqpoint{2.689436in}{2.286015in}}%
\pgfpathlineto{\pgfqpoint{2.684859in}{2.327999in}}%
\pgfpathlineto{\pgfqpoint{2.678725in}{2.364664in}}%
\pgfpathlineto{\pgfqpoint{2.671356in}{2.395897in}}%
\pgfpathlineto{\pgfqpoint{2.662489in}{2.423981in}}%
\pgfpathlineto{\pgfqpoint{2.652361in}{2.448778in}}%
\pgfpathlineto{\pgfqpoint{2.641365in}{2.470245in}}%
\pgfpathlineto{\pgfqpoint{2.628643in}{2.490425in}}%
\pgfpathlineto{\pgfqpoint{2.614279in}{2.509106in}}%
\pgfpathlineto{\pgfqpoint{2.598443in}{2.526159in}}%
\pgfpathlineto{\pgfqpoint{2.579590in}{2.543005in}}%
\pgfpathlineto{\pgfqpoint{2.559532in}{2.557923in}}%
\pgfpathlineto{\pgfqpoint{2.536602in}{2.572183in}}%
\pgfpathlineto{\pgfqpoint{2.510850in}{2.585538in}}%
\pgfpathlineto{\pgfqpoint{2.482360in}{2.597837in}}%
\pgfpathlineto{\pgfqpoint{2.449134in}{2.609683in}}%
\pgfpathlineto{\pgfqpoint{2.411184in}{2.620696in}}%
\pgfpathlineto{\pgfqpoint{2.368552in}{2.630606in}}%
\pgfpathlineto{\pgfqpoint{2.321294in}{2.639221in}}%
\pgfpathlineto{\pgfqpoint{2.269467in}{2.646399in}}%
\pgfpathlineto{\pgfqpoint{2.210954in}{2.652193in}}%
\pgfpathlineto{\pgfqpoint{2.147967in}{2.656153in}}%
\pgfpathlineto{\pgfqpoint{2.080556in}{2.658135in}}%
\pgfpathlineto{\pgfqpoint{2.010948in}{2.657971in}}%
\pgfpathlineto{\pgfqpoint{1.939195in}{2.655572in}}%
\pgfpathlineto{\pgfqpoint{1.867527in}{2.650913in}}%
\pgfpathlineto{\pgfqpoint{1.798171in}{2.644140in}}%
\pgfpathlineto{\pgfqpoint{1.733341in}{2.635606in}}%
\pgfpathlineto{\pgfqpoint{1.673075in}{2.625521in}}%
\pgfpathlineto{\pgfqpoint{1.615274in}{2.613610in}}%
\pgfpathlineto{\pgfqpoint{1.562133in}{2.600402in}}%
\pgfpathlineto{\pgfqpoint{1.513681in}{2.586139in}}%
\pgfpathlineto{\pgfqpoint{1.467862in}{2.570344in}}%
\pgfpathlineto{\pgfqpoint{1.426794in}{2.553923in}}%
\pgfpathlineto{\pgfqpoint{1.388447in}{2.536289in}}%
\pgfpathlineto{\pgfqpoint{1.352878in}{2.517566in}}%
\pgfpathlineto{\pgfqpoint{1.320128in}{2.497922in}}%
\pgfpathlineto{\pgfqpoint{1.288379in}{2.476236in}}%
\pgfpathlineto{\pgfqpoint{1.259592in}{2.453861in}}%
\pgfpathlineto{\pgfqpoint{1.232050in}{2.429520in}}%
\pgfpathlineto{\pgfqpoint{1.207527in}{2.404898in}}%
\pgfpathlineto{\pgfqpoint{1.184409in}{2.378557in}}%
\pgfpathlineto{\pgfqpoint{1.162828in}{2.350561in}}%
\pgfpathlineto{\pgfqpoint{1.142891in}{2.321011in}}%
\pgfpathlineto{\pgfqpoint{1.124675in}{2.290041in}}%
\pgfpathlineto{\pgfqpoint{1.108225in}{2.257802in}}%
\pgfpathlineto{\pgfqpoint{1.092639in}{2.222199in}}%
\pgfpathlineto{\pgfqpoint{1.079059in}{2.185535in}}%
\pgfpathlineto{\pgfqpoint{1.067443in}{2.147998in}}%
\pgfpathlineto{\pgfqpoint{1.057187in}{2.107348in}}%
\pgfpathlineto{\pgfqpoint{1.049004in}{2.066086in}}%
\pgfpathlineto{\pgfqpoint{1.042513in}{2.021906in}}%
\pgfpathlineto{\pgfqpoint{1.038177in}{1.977382in}}%
\pgfpathlineto{\pgfqpoint{1.035866in}{1.930167in}}%
\pgfpathlineto{\pgfqpoint{1.035826in}{1.882878in}}%
\pgfpathlineto{\pgfqpoint{1.038031in}{1.835656in}}%
\pgfpathlineto{\pgfqpoint{1.042474in}{1.788641in}}%
\pgfpathlineto{\pgfqpoint{1.049176in}{1.741979in}}%
\pgfpathlineto{\pgfqpoint{1.057644in}{1.698239in}}%
\pgfpathlineto{\pgfqpoint{1.068221in}{1.655105in}}%
\pgfpathlineto{\pgfqpoint{1.080962in}{1.612745in}}%
\pgfpathlineto{\pgfqpoint{1.095031in}{1.573617in}}%
\pgfpathlineto{\pgfqpoint{1.111115in}{1.535520in}}%
\pgfpathlineto{\pgfqpoint{1.128118in}{1.500775in}}%
\pgfpathlineto{\pgfqpoint{1.146930in}{1.467274in}}%
\pgfpathlineto{\pgfqpoint{1.167531in}{1.435181in}}%
\pgfpathlineto{\pgfqpoint{1.189874in}{1.404652in}}%
\pgfpathlineto{\pgfqpoint{1.213884in}{1.375828in}}%
\pgfpathlineto{\pgfqpoint{1.237817in}{1.350457in}}%
\pgfpathlineto{\pgfqpoint{1.264748in}{1.325237in}}%
\pgfpathlineto{\pgfqpoint{1.292991in}{1.301972in}}%
\pgfpathlineto{\pgfqpoint{1.322398in}{1.280678in}}%
\pgfpathlineto{\pgfqpoint{1.352820in}{1.261340in}}%
\pgfpathlineto{\pgfqpoint{1.386095in}{1.242889in}}%
\pgfpathlineto{\pgfqpoint{1.420190in}{1.226516in}}%
\pgfpathlineto{\pgfqpoint{1.457024in}{1.211329in}}%
\pgfpathlineto{\pgfqpoint{1.496554in}{1.197536in}}%
\pgfpathlineto{\pgfqpoint{1.538719in}{1.185287in}}%
\pgfpathlineto{\pgfqpoint{1.583441in}{1.174641in}}%
\pgfpathlineto{\pgfqpoint{1.634929in}{1.164775in}}%
\pgfpathlineto{\pgfqpoint{1.706063in}{1.153745in}}%
\pgfpathlineto{\pgfqpoint{1.768492in}{1.143417in}}%
\pgfpathlineto{\pgfqpoint{1.796122in}{1.136567in}}%
\pgfpathlineto{\pgfqpoint{1.812683in}{1.130481in}}%
\pgfpathlineto{\pgfqpoint{1.824471in}{1.124102in}}%
\pgfpathlineto{\pgfqpoint{1.833209in}{1.116741in}}%
\pgfpathlineto{\pgfqpoint{1.838498in}{1.108890in}}%
\pgfpathlineto{\pgfqpoint{1.840588in}{1.101849in}}%
\pgfpathlineto{\pgfqpoint{1.840619in}{1.094412in}}%
\pgfpathlineto{\pgfqpoint{1.837931in}{1.084986in}}%
\pgfpathlineto{\pgfqpoint{1.833246in}{1.076615in}}%
\pgfpathlineto{\pgfqpoint{1.825819in}{1.067542in}}%
\pgfpathlineto{\pgfqpoint{1.813813in}{1.056850in}}%
\pgfpathlineto{\pgfqpoint{1.798819in}{1.046763in}}%
\pgfpathlineto{\pgfqpoint{1.781016in}{1.037462in}}%
\pgfpathlineto{\pgfqpoint{1.758447in}{1.028391in}}%
\pgfpathlineto{\pgfqpoint{1.733203in}{1.020815in}}%
\pgfpathlineto{\pgfqpoint{1.705410in}{1.014872in}}%
\pgfpathlineto{\pgfqpoint{1.675178in}{1.010714in}}%
\pgfpathlineto{\pgfqpoint{1.642610in}{1.008507in}}%
\pgfpathlineto{\pgfqpoint{1.607809in}{1.008432in}}%
\pgfpathlineto{\pgfqpoint{1.570886in}{1.010691in}}%
\pgfpathlineto{\pgfqpoint{1.534118in}{1.015181in}}%
\pgfpathlineto{\pgfqpoint{1.495454in}{1.022233in}}%
\pgfpathlineto{\pgfqpoint{1.457161in}{1.031563in}}%
\pgfpathlineto{\pgfqpoint{1.419337in}{1.043132in}}%
\pgfpathlineto{\pgfqpoint{1.382089in}{1.056929in}}%
\pgfpathlineto{\pgfqpoint{1.347544in}{1.072019in}}%
\pgfpathlineto{\pgfqpoint{1.313727in}{1.089133in}}%
\pgfpathlineto{\pgfqpoint{1.280762in}{1.108299in}}%
\pgfpathlineto{\pgfqpoint{1.248782in}{1.129536in}}%
\pgfpathlineto{\pgfqpoint{1.219708in}{1.151422in}}%
\pgfpathlineto{\pgfqpoint{1.191752in}{1.175138in}}%
\pgfpathlineto{\pgfqpoint{1.165031in}{1.200649in}}%
\pgfpathlineto{\pgfqpoint{1.139653in}{1.227898in}}%
\pgfpathlineto{\pgfqpoint{1.115714in}{1.256800in}}%
\pgfpathlineto{\pgfqpoint{1.093288in}{1.287251in}}%
\pgfpathlineto{\pgfqpoint{1.071178in}{1.321163in}}%
\pgfpathlineto{\pgfqpoint{1.050868in}{1.356520in}}%
\pgfpathlineto{\pgfqpoint{1.032365in}{1.393152in}}%
\pgfpathlineto{\pgfqpoint{1.014718in}{1.433142in}}%
\pgfpathlineto{\pgfqpoint{0.999024in}{1.474185in}}%
\pgfpathlineto{\pgfqpoint{0.984506in}{1.518461in}}%
\pgfpathlineto{\pgfqpoint{0.972010in}{1.563537in}}%
\pgfpathlineto{\pgfqpoint{0.960944in}{1.611678in}}%
\pgfpathlineto{\pgfqpoint{0.951530in}{1.662824in}}%
\pgfpathlineto{\pgfqpoint{0.944286in}{1.714431in}}%
\pgfpathlineto{\pgfqpoint{0.938950in}{1.768847in}}%
\pgfpathlineto{\pgfqpoint{0.935870in}{1.823491in}}%
\pgfpathlineto{\pgfqpoint{0.935034in}{1.878240in}}%
\pgfpathlineto{\pgfqpoint{0.936466in}{1.932973in}}%
\pgfpathlineto{\pgfqpoint{0.940005in}{1.985084in}}%
\pgfpathlineto{\pgfqpoint{0.945759in}{2.036935in}}%
\pgfpathlineto{\pgfqpoint{0.953410in}{2.085938in}}%
\pgfpathlineto{\pgfqpoint{0.962764in}{2.132000in}}%
\pgfpathlineto{\pgfqpoint{0.974287in}{2.177414in}}%
\pgfpathlineto{\pgfqpoint{0.987332in}{2.219653in}}%
\pgfpathlineto{\pgfqpoint{1.001667in}{2.258654in}}%
\pgfpathlineto{\pgfqpoint{1.018051in}{2.296583in}}%
\pgfpathlineto{\pgfqpoint{1.035401in}{2.331101in}}%
\pgfpathlineto{\pgfqpoint{1.054650in}{2.364275in}}%
\pgfpathlineto{\pgfqpoint{1.074406in}{2.393984in}}%
\pgfpathlineto{\pgfqpoint{1.095771in}{2.422197in}}%
\pgfpathlineto{\pgfqpoint{1.118662in}{2.448797in}}%
\pgfpathlineto{\pgfqpoint{1.142967in}{2.473701in}}%
\pgfpathlineto{\pgfqpoint{1.168550in}{2.496867in}}%
\pgfpathlineto{\pgfqpoint{1.197085in}{2.519662in}}%
\pgfpathlineto{\pgfqpoint{1.226727in}{2.540526in}}%
\pgfpathlineto{\pgfqpoint{1.259242in}{2.560673in}}%
\pgfpathlineto{\pgfqpoint{1.294612in}{2.579881in}}%
\pgfpathlineto{\pgfqpoint{1.332792in}{2.597982in}}%
\pgfpathlineto{\pgfqpoint{1.373719in}{2.614859in}}%
\pgfpathlineto{\pgfqpoint{1.417319in}{2.630445in}}%
\pgfpathlineto{\pgfqpoint{1.465632in}{2.645312in}}%
\pgfpathlineto{\pgfqpoint{1.518640in}{2.659204in}}%
\pgfpathlineto{\pgfqpoint{1.576309in}{2.671929in}}%
\pgfpathlineto{\pgfqpoint{1.638597in}{2.683344in}}%
\pgfpathlineto{\pgfqpoint{1.705462in}{2.693343in}}%
\pgfpathlineto{\pgfqpoint{1.779027in}{2.702064in}}%
\pgfpathlineto{\pgfqpoint{1.857097in}{2.709077in}}%
\pgfpathlineto{\pgfqpoint{1.939633in}{2.714280in}}%
\pgfpathlineto{\pgfqpoint{2.026598in}{2.717513in}}%
\pgfpathlineto{\pgfqpoint{2.113605in}{2.718523in}}%
\pgfpathlineto{\pgfqpoint{2.198435in}{2.717303in}}%
\pgfpathlineto{\pgfqpoint{2.278866in}{2.713929in}}%
\pgfpathlineto{\pgfqpoint{2.352678in}{2.708598in}}%
\pgfpathlineto{\pgfqpoint{2.417657in}{2.701709in}}%
\pgfpathlineto{\pgfqpoint{2.473770in}{2.693630in}}%
\pgfpathlineto{\pgfqpoint{2.523140in}{2.684368in}}%
\pgfpathlineto{\pgfqpoint{2.565726in}{2.674202in}}%
\pgfpathlineto{\pgfqpoint{2.601510in}{2.663544in}}%
\pgfpathlineto{\pgfqpoint{2.632577in}{2.652142in}}%
\pgfpathlineto{\pgfqpoint{2.658899in}{2.640331in}}%
\pgfpathlineto{\pgfqpoint{2.682438in}{2.627436in}}%
\pgfpathlineto{\pgfqpoint{2.703062in}{2.613571in}}%
\pgfpathlineto{\pgfqpoint{2.720674in}{2.598978in}}%
\pgfpathlineto{\pgfqpoint{2.735263in}{2.584053in}}%
\pgfpathlineto{\pgfqpoint{2.748320in}{2.567377in}}%
\pgfpathlineto{\pgfqpoint{2.759553in}{2.549046in}}%
\pgfpathlineto{\pgfqpoint{2.768788in}{2.529306in}}%
\pgfpathlineto{\pgfqpoint{2.776017in}{2.508498in}}%
\pgfpathlineto{\pgfqpoint{2.781884in}{2.484540in}}%
\pgfpathlineto{\pgfqpoint{2.786102in}{2.457597in}}%
\pgfpathlineto{\pgfqpoint{2.788720in}{2.425384in}}%
\pgfpathlineto{\pgfqpoint{2.789427in}{2.388061in}}%
\pgfpathlineto{\pgfqpoint{2.787962in}{2.340801in}}%
\pgfpathlineto{\pgfqpoint{2.783672in}{2.278768in}}%
\pgfpathlineto{\pgfqpoint{2.774289in}{2.179783in}}%
\pgfpathlineto{\pgfqpoint{2.743611in}{1.868119in}}%
\pgfpathlineto{\pgfqpoint{2.730112in}{1.702060in}}%
\pgfpathlineto{\pgfqpoint{2.717287in}{1.515949in}}%
\pgfpathlineto{\pgfqpoint{2.702602in}{1.267597in}}%
\pgfpathlineto{\pgfqpoint{2.684434in}{0.964630in}}%
\pgfpathlineto{\pgfqpoint{2.675374in}{0.850600in}}%
\pgfpathlineto{\pgfqpoint{2.667030in}{0.771523in}}%
\pgfpathlineto{\pgfqpoint{2.658752in}{0.712543in}}%
\pgfpathlineto{\pgfqpoint{2.650176in}{0.666284in}}%
\pgfpathlineto{\pgfqpoint{2.640820in}{0.627931in}}%
\pgfpathlineto{\pgfqpoint{2.631145in}{0.597534in}}%
\pgfpathlineto{\pgfqpoint{2.621004in}{0.572745in}}%
\pgfpathlineto{\pgfqpoint{2.609856in}{0.551383in}}%
\pgfpathlineto{\pgfqpoint{2.598042in}{0.533534in}}%
\pgfpathlineto{\pgfqpoint{2.584496in}{0.517378in}}%
\pgfpathlineto{\pgfqpoint{2.571109in}{0.504669in}}%
\pgfpathlineto{\pgfqpoint{2.554789in}{0.492313in}}%
\pgfpathlineto{\pgfqpoint{2.537457in}{0.481914in}}%
\pgfpathlineto{\pgfqpoint{2.517374in}{0.472367in}}%
\pgfpathlineto{\pgfqpoint{2.492542in}{0.463178in}}%
\pgfpathlineto{\pgfqpoint{2.462979in}{0.454833in}}%
\pgfpathlineto{\pgfqpoint{2.428766in}{0.447542in}}%
\pgfpathlineto{\pgfqpoint{2.385671in}{0.440735in}}%
\pgfpathlineto{\pgfqpoint{2.331557in}{0.434581in}}%
\pgfpathlineto{\pgfqpoint{2.262115in}{0.429077in}}%
\pgfpathlineto{\pgfqpoint{2.170851in}{0.424236in}}%
\pgfpathlineto{\pgfqpoint{2.049086in}{0.420134in}}%
\pgfpathlineto{\pgfqpoint{1.879436in}{0.416783in}}%
\pgfpathlineto{\pgfqpoint{1.640159in}{0.414418in}}%
\pgfpathlineto{\pgfqpoint{1.322562in}{0.413569in}}%
\pgfpathlineto{\pgfqpoint{1.020194in}{0.414850in}}%
\pgfpathlineto{\pgfqpoint{0.822256in}{0.417715in}}%
\pgfpathlineto{\pgfqpoint{0.704835in}{0.421430in}}%
\pgfpathlineto{\pgfqpoint{0.630976in}{0.425829in}}%
\pgfpathlineto{\pgfqpoint{0.583316in}{0.430734in}}%
\pgfpathlineto{\pgfqpoint{0.551033in}{0.436123in}}%
\pgfpathlineto{\pgfqpoint{0.527708in}{0.442189in}}%
\pgfpathlineto{\pgfqpoint{0.511250in}{0.448625in}}%
\pgfpathlineto{\pgfqpoint{0.499549in}{0.455216in}}%
\pgfpathlineto{\pgfqpoint{0.488916in}{0.463841in}}%
\pgfpathlineto{\pgfqpoint{0.481322in}{0.472730in}}%
\pgfpathlineto{\pgfqpoint{0.474078in}{0.485127in}}%
\pgfpathlineto{\pgfqpoint{0.468753in}{0.498748in}}%
\pgfpathlineto{\pgfqpoint{0.463870in}{0.517848in}}%
\pgfpathlineto{\pgfqpoint{0.459679in}{0.544796in}}%
\pgfpathlineto{\pgfqpoint{0.456386in}{0.581938in}}%
\pgfpathlineto{\pgfqpoint{0.453731in}{0.639106in}}%
\pgfpathlineto{\pgfqpoint{0.451681in}{0.736155in}}%
\pgfpathlineto{\pgfqpoint{0.450220in}{0.927815in}}%
\pgfpathlineto{\pgfqpoint{0.449345in}{1.403252in}}%
\pgfpathlineto{\pgfqpoint{0.449543in}{2.682703in}}%
\pgfpathlineto{\pgfqpoint{0.451011in}{2.856932in}}%
\pgfpathlineto{\pgfqpoint{0.452802in}{2.879219in}}%
\pgfpathlineto{\pgfqpoint{0.455188in}{2.886108in}}%
\pgfpathlineto{\pgfqpoint{0.458626in}{2.889028in}}%
\pgfpathlineto{\pgfqpoint{0.464996in}{2.890553in}}%
\pgfpathlineto{\pgfqpoint{0.482377in}{2.891423in}}%
\pgfpathlineto{\pgfqpoint{0.565038in}{2.891729in}}%
\pgfpathlineto{\pgfqpoint{2.733842in}{2.891760in}}%
\pgfpathlineto{\pgfqpoint{4.789510in}{2.890885in}}%
\pgfpathlineto{\pgfqpoint{4.793727in}{2.889730in}}%
\pgfpathlineto{\pgfqpoint{4.795481in}{2.888307in}}%
\pgfpathlineto{\pgfqpoint{4.797106in}{2.881145in}}%
\pgfpathlineto{\pgfqpoint{4.797997in}{2.858771in}}%
\pgfpathlineto{\pgfqpoint{4.798039in}{2.856283in}}%
\pgfpathlineto{\pgfqpoint{4.798039in}{2.856283in}}%
\pgfusepath{stroke}%
\end{pgfscope}%
\begin{pgfscope}%
\pgfpathrectangle{\pgfqpoint{0.448634in}{0.402556in}}{\pgfqpoint{4.350661in}{2.489204in}} %
\pgfusepath{clip}%
\pgfsetrectcap%
\pgfsetroundjoin%
\pgfsetlinewidth{1.003750pt}%
\definecolor{currentstroke}{rgb}{0.121569,0.466667,0.705882}%
\pgfsetstrokecolor{currentstroke}%
\pgfsetdash{}{0pt}%
\pgfpathmoveto{\pgfqpoint{3.428772in}{0.402610in}}%
\pgfpathlineto{\pgfqpoint{2.806632in}{0.403760in}}%
\pgfpathlineto{\pgfqpoint{2.769692in}{0.405578in}}%
\pgfpathlineto{\pgfqpoint{2.754632in}{0.408064in}}%
\pgfpathlineto{\pgfqpoint{2.746391in}{0.411198in}}%
\pgfpathlineto{\pgfqpoint{2.740943in}{0.415265in}}%
\pgfpathlineto{\pgfqpoint{2.736784in}{0.420984in}}%
\pgfpathlineto{\pgfqpoint{2.733281in}{0.430071in}}%
\pgfpathlineto{\pgfqpoint{2.730449in}{0.444636in}}%
\pgfpathlineto{\pgfqpoint{2.728238in}{0.469392in}}%
\pgfpathlineto{\pgfqpoint{2.726470in}{0.519131in}}%
\pgfpathlineto{\pgfqpoint{2.725711in}{0.613715in}}%
\pgfpathlineto{\pgfqpoint{2.726842in}{0.768038in}}%
\pgfpathlineto{\pgfqpoint{2.730556in}{0.962148in}}%
\pgfpathlineto{\pgfqpoint{2.736611in}{1.158670in}}%
\pgfpathlineto{\pgfqpoint{2.744092in}{1.327718in}}%
\pgfpathlineto{\pgfqpoint{2.753201in}{1.484189in}}%
\pgfpathlineto{\pgfqpoint{2.763257in}{1.620609in}}%
\pgfpathlineto{\pgfqpoint{2.776118in}{1.764216in}}%
\pgfpathlineto{\pgfqpoint{2.788914in}{1.877776in}}%
\pgfpathlineto{\pgfqpoint{2.805748in}{2.005740in}}%
\pgfpathlineto{\pgfqpoint{2.821176in}{2.101198in}}%
\pgfpathlineto{\pgfqpoint{2.838359in}{2.193718in}}%
\pgfpathlineto{\pgfqpoint{2.859135in}{2.292966in}}%
\pgfpathlineto{\pgfqpoint{2.887209in}{2.425960in}}%
\pgfpathlineto{\pgfqpoint{2.896991in}{2.479560in}}%
\pgfpathlineto{\pgfqpoint{2.901543in}{2.516523in}}%
\pgfpathlineto{\pgfqpoint{2.902849in}{2.543854in}}%
\pgfpathlineto{\pgfqpoint{2.901957in}{2.566223in}}%
\pgfpathlineto{\pgfqpoint{2.899151in}{2.585863in}}%
\pgfpathlineto{\pgfqpoint{2.894794in}{2.602546in}}%
\pgfpathlineto{\pgfqpoint{2.888484in}{2.618388in}}%
\pgfpathlineto{\pgfqpoint{2.880257in}{2.633033in}}%
\pgfpathlineto{\pgfqpoint{2.870348in}{2.646246in}}%
\pgfpathlineto{\pgfqpoint{2.857399in}{2.659530in}}%
\pgfpathlineto{\pgfqpoint{2.843189in}{2.671010in}}%
\pgfpathlineto{\pgfqpoint{2.824237in}{2.683209in}}%
\pgfpathlineto{\pgfqpoint{2.802413in}{2.694418in}}%
\pgfpathlineto{\pgfqpoint{2.775809in}{2.705369in}}%
\pgfpathlineto{\pgfqpoint{2.744461in}{2.715715in}}%
\pgfpathlineto{\pgfqpoint{2.708436in}{2.725252in}}%
\pgfpathlineto{\pgfqpoint{2.665655in}{2.734289in}}%
\pgfpathlineto{\pgfqpoint{2.613991in}{2.742869in}}%
\pgfpathlineto{\pgfqpoint{2.553459in}{2.750589in}}%
\pgfpathlineto{\pgfqpoint{2.481920in}{2.757365in}}%
\pgfpathlineto{\pgfqpoint{2.399398in}{2.762839in}}%
\pgfpathlineto{\pgfqpoint{2.310269in}{2.766482in}}%
\pgfpathlineto{\pgfqpoint{2.175416in}{2.768725in}}%
\pgfpathlineto{\pgfqpoint{2.066653in}{2.767942in}}%
\pgfpathlineto{\pgfqpoint{1.953570in}{2.764859in}}%
\pgfpathlineto{\pgfqpoint{1.851429in}{2.759759in}}%
\pgfpathlineto{\pgfqpoint{1.745051in}{2.752169in}}%
\pgfpathlineto{\pgfqpoint{1.658373in}{2.743453in}}%
\pgfpathlineto{\pgfqpoint{1.580552in}{2.733461in}}%
\pgfpathlineto{\pgfqpoint{1.490057in}{2.719338in}}%
\pgfpathlineto{\pgfqpoint{1.417231in}{2.704698in}}%
\pgfpathlineto{\pgfqpoint{1.361992in}{2.690818in}}%
\pgfpathlineto{\pgfqpoint{1.311460in}{2.675819in}}%
\pgfpathlineto{\pgfqpoint{1.265667in}{2.659924in}}%
\pgfpathlineto{\pgfqpoint{1.222575in}{2.642586in}}%
\pgfpathlineto{\pgfqpoint{1.184324in}{2.624682in}}%
\pgfpathlineto{\pgfqpoint{1.148892in}{2.605623in}}%
\pgfpathlineto{\pgfqpoint{1.116331in}{2.585573in}}%
\pgfpathlineto{\pgfqpoint{1.092327in}{2.568512in}}%
\pgfpathlineto{\pgfqpoint{1.079760in}{2.558686in}}%
\pgfpathlineto{\pgfqpoint{1.051544in}{2.535379in}}%
\pgfpathlineto{\pgfqpoint{1.026312in}{2.511712in}}%
\pgfpathlineto{\pgfqpoint{1.002399in}{2.486318in}}%
\pgfpathlineto{\pgfqpoint{0.979913in}{2.459269in}}%
\pgfpathlineto{\pgfqpoint{0.958934in}{2.430678in}}%
\pgfpathlineto{\pgfqpoint{0.938264in}{2.398643in}}%
\pgfpathlineto{\pgfqpoint{0.923047in}{2.371385in}}%
\pgfpathlineto{\pgfqpoint{0.904513in}{2.334774in}}%
\pgfpathlineto{\pgfqpoint{0.887854in}{2.297001in}}%
\pgfpathlineto{\pgfqpoint{0.872131in}{2.255971in}}%
\pgfpathlineto{\pgfqpoint{0.857508in}{2.211741in}}%
\pgfpathlineto{\pgfqpoint{0.844762in}{2.166757in}}%
\pgfpathlineto{\pgfqpoint{0.838624in}{2.140306in}}%
\pgfpathlineto{\pgfqpoint{0.826982in}{2.087194in}}%
\pgfpathlineto{\pgfqpoint{0.816322in}{2.028715in}}%
\pgfpathlineto{\pgfqpoint{0.810087in}{1.984495in}}%
\pgfpathlineto{\pgfqpoint{0.808026in}{1.967238in}}%
\pgfpathlineto{\pgfqpoint{0.800076in}{1.898140in}}%
\pgfpathlineto{\pgfqpoint{0.793713in}{1.823823in}}%
\pgfpathlineto{\pgfqpoint{0.788799in}{1.741875in}}%
\pgfpathlineto{\pgfqpoint{0.786199in}{1.677225in}}%
\pgfpathlineto{\pgfqpoint{0.776951in}{1.453481in}}%
\pgfpathlineto{\pgfqpoint{0.773280in}{1.418894in}}%
\pgfpathlineto{\pgfqpoint{0.768298in}{1.389582in}}%
\pgfpathlineto{\pgfqpoint{0.762752in}{1.368108in}}%
\pgfpathlineto{\pgfqpoint{0.756722in}{1.352123in}}%
\pgfpathlineto{\pgfqpoint{0.749752in}{1.339519in}}%
\pgfpathlineto{\pgfqpoint{0.742201in}{1.330599in}}%
\pgfpathlineto{\pgfqpoint{0.734854in}{1.325312in}}%
\pgfpathlineto{\pgfqpoint{0.726558in}{1.322419in}}%
\pgfpathlineto{\pgfqpoint{0.717884in}{1.322223in}}%
\pgfpathlineto{\pgfqpoint{0.709412in}{1.324411in}}%
\pgfpathlineto{\pgfqpoint{0.699548in}{1.329604in}}%
\pgfpathlineto{\pgfqpoint{0.688894in}{1.338203in}}%
\pgfpathlineto{\pgfqpoint{0.677907in}{1.350248in}}%
\pgfpathlineto{\pgfqpoint{0.666886in}{1.365647in}}%
\pgfpathlineto{\pgfqpoint{0.654913in}{1.386417in}}%
\pgfpathlineto{\pgfqpoint{0.642574in}{1.412730in}}%
\pgfpathlineto{\pgfqpoint{0.630328in}{1.444629in}}%
\pgfpathlineto{\pgfqpoint{0.618504in}{1.482081in}}%
\pgfpathlineto{\pgfqpoint{0.608613in}{1.520256in}}%
\pgfpathlineto{\pgfqpoint{0.590203in}{1.612445in}}%
\pgfpathlineto{\pgfqpoint{0.581848in}{1.668884in}}%
\pgfpathlineto{\pgfqpoint{0.573137in}{1.740376in}}%
\pgfpathlineto{\pgfqpoint{0.567062in}{1.807213in}}%
\pgfpathlineto{\pgfqpoint{0.560532in}{1.896510in}}%
\pgfpathlineto{\pgfqpoint{0.555526in}{1.995910in}}%
\pgfpathlineto{\pgfqpoint{0.552564in}{2.097908in}}%
\pgfpathlineto{\pgfqpoint{0.551526in}{2.204935in}}%
\pgfpathlineto{\pgfqpoint{0.552728in}{2.309470in}}%
\pgfpathlineto{\pgfqpoint{0.556011in}{2.403981in}}%
\pgfpathlineto{\pgfqpoint{0.560953in}{2.483430in}}%
\pgfpathlineto{\pgfqpoint{0.567303in}{2.550240in}}%
\pgfpathlineto{\pgfqpoint{0.574928in}{2.606817in}}%
\pgfpathlineto{\pgfqpoint{0.582988in}{2.650657in}}%
\pgfpathlineto{\pgfqpoint{0.592756in}{2.691452in}}%
\pgfpathlineto{\pgfqpoint{0.602650in}{2.721756in}}%
\pgfpathlineto{\pgfqpoint{0.612983in}{2.746441in}}%
\pgfpathlineto{\pgfqpoint{0.624292in}{2.767692in}}%
\pgfpathlineto{\pgfqpoint{0.636231in}{2.785433in}}%
\pgfpathlineto{\pgfqpoint{0.649892in}{2.801461in}}%
\pgfpathlineto{\pgfqpoint{0.663386in}{2.814020in}}%
\pgfpathlineto{\pgfqpoint{0.679842in}{2.826135in}}%
\pgfpathlineto{\pgfqpoint{0.697326in}{2.836197in}}%
\pgfpathlineto{\pgfqpoint{0.715574in}{2.844285in}}%
\pgfpathlineto{\pgfqpoint{0.738439in}{2.852335in}}%
\pgfpathlineto{\pgfqpoint{0.765983in}{2.859639in}}%
\pgfpathlineto{\pgfqpoint{0.800300in}{2.866256in}}%
\pgfpathlineto{\pgfqpoint{0.841340in}{2.871832in}}%
\pgfpathlineto{\pgfqpoint{0.895547in}{2.876803in}}%
\pgfpathlineto{\pgfqpoint{0.969413in}{2.881069in}}%
\pgfpathlineto{\pgfqpoint{1.071608in}{2.884501in}}%
\pgfpathlineto{\pgfqpoint{1.219512in}{2.887074in}}%
\pgfpathlineto{\pgfqpoint{1.471844in}{2.889091in}}%
\pgfpathlineto{\pgfqpoint{1.956941in}{2.890384in}}%
\pgfpathlineto{\pgfqpoint{3.096814in}{2.890781in}}%
\pgfpathlineto{\pgfqpoint{3.995224in}{2.889388in}}%
\pgfpathlineto{\pgfqpoint{4.275833in}{2.887011in}}%
\pgfpathlineto{\pgfqpoint{4.412847in}{2.883743in}}%
\pgfpathlineto{\pgfqpoint{4.491081in}{2.879810in}}%
\pgfpathlineto{\pgfqpoint{4.543127in}{2.875163in}}%
\pgfpathlineto{\pgfqpoint{4.579810in}{2.869841in}}%
\pgfpathlineto{\pgfqpoint{4.607580in}{2.863763in}}%
\pgfpathlineto{\pgfqpoint{4.630623in}{2.856424in}}%
\pgfpathlineto{\pgfqpoint{4.648833in}{2.848228in}}%
\pgfpathlineto{\pgfqpoint{4.664136in}{2.838773in}}%
\pgfpathlineto{\pgfqpoint{4.676470in}{2.828576in}}%
\pgfpathlineto{\pgfqpoint{4.687502in}{2.816585in}}%
\pgfpathlineto{\pgfqpoint{4.697051in}{2.803027in}}%
\pgfpathlineto{\pgfqpoint{4.706194in}{2.786098in}}%
\pgfpathlineto{\pgfqpoint{4.714508in}{2.765827in}}%
\pgfpathlineto{\pgfqpoint{4.722462in}{2.740013in}}%
\pgfpathlineto{\pgfqpoint{4.729577in}{2.708703in}}%
\pgfpathlineto{\pgfqpoint{4.736162in}{2.669601in}}%
\pgfpathlineto{\pgfqpoint{4.742419in}{2.617826in}}%
\pgfpathlineto{\pgfqpoint{4.747859in}{2.553410in}}%
\pgfpathlineto{\pgfqpoint{4.752661in}{2.468958in}}%
\pgfpathlineto{\pgfqpoint{4.756610in}{2.359528in}}%
\pgfpathlineto{\pgfqpoint{4.759416in}{2.217681in}}%
\pgfpathlineto{\pgfqpoint{4.760596in}{2.043444in}}%
\pgfpathlineto{\pgfqpoint{4.759662in}{1.851779in}}%
\pgfpathlineto{\pgfqpoint{4.756587in}{1.667613in}}%
\pgfpathlineto{\pgfqpoint{4.751596in}{1.503428in}}%
\pgfpathlineto{\pgfqpoint{4.745410in}{1.374185in}}%
\pgfpathlineto{\pgfqpoint{4.738113in}{1.267479in}}%
\pgfpathlineto{\pgfqpoint{4.729621in}{1.175896in}}%
\pgfpathlineto{\pgfqpoint{4.720762in}{1.104428in}}%
\pgfpathlineto{\pgfqpoint{4.711045in}{1.043204in}}%
\pgfpathlineto{\pgfqpoint{4.700364in}{0.989829in}}%
\pgfpathlineto{\pgfqpoint{4.689055in}{0.944345in}}%
\pgfpathlineto{\pgfqpoint{4.676881in}{0.904394in}}%
\pgfpathlineto{\pgfqpoint{4.676095in}{0.902073in}}%
\pgfpathlineto{\pgfqpoint{4.676095in}{0.902073in}}%
\pgfusepath{stroke}%
\end{pgfscope}%
\begin{pgfscope}%
\pgfpathrectangle{\pgfqpoint{0.448634in}{0.402556in}}{\pgfqpoint{4.350661in}{2.489204in}} %
\pgfusepath{clip}%
\pgfsetrectcap%
\pgfsetroundjoin%
\pgfsetlinewidth{1.003750pt}%
\definecolor{currentstroke}{rgb}{0.121569,0.466667,0.705882}%
\pgfsetstrokecolor{currentstroke}%
\pgfsetdash{}{0pt}%
\pgfpathmoveto{\pgfqpoint{2.795520in}{1.982745in}}%
\pgfpathlineto{\pgfqpoint{2.781780in}{1.874357in}}%
\pgfpathlineto{\pgfqpoint{2.769351in}{1.758234in}}%
\pgfpathlineto{\pgfqpoint{2.758095in}{1.631942in}}%
\pgfpathlineto{\pgfqpoint{2.747786in}{1.490551in}}%
\pgfpathlineto{\pgfqpoint{2.738644in}{1.334082in}}%
\pgfpathlineto{\pgfqpoint{2.730580in}{1.157591in}}%
\pgfpathlineto{\pgfqpoint{2.723334in}{0.948663in}}%
\pgfpathlineto{\pgfqpoint{2.709783in}{0.530788in}}%
\pgfpathlineto{\pgfqpoint{2.705868in}{0.488716in}}%
\pgfpathlineto{\pgfqpoint{2.701769in}{0.464281in}}%
\pgfpathlineto{\pgfqpoint{2.697021in}{0.447744in}}%
\pgfpathlineto{\pgfqpoint{2.691859in}{0.436812in}}%
\pgfpathlineto{\pgfqpoint{2.686245in}{0.429229in}}%
\pgfpathlineto{\pgfqpoint{2.679348in}{0.423188in}}%
\pgfpathlineto{\pgfqpoint{2.669540in}{0.417856in}}%
\pgfpathlineto{\pgfqpoint{2.656987in}{0.413810in}}%
\pgfpathlineto{\pgfqpoint{2.637654in}{0.410337in}}%
\pgfpathlineto{\pgfqpoint{2.607297in}{0.407617in}}%
\pgfpathlineto{\pgfqpoint{2.555121in}{0.405574in}}%
\pgfpathlineto{\pgfqpoint{2.450714in}{0.404139in}}%
\pgfpathlineto{\pgfqpoint{2.176624in}{0.403275in}}%
\pgfpathlineto{\pgfqpoint{1.130290in}{0.402953in}}%
\pgfpathlineto{\pgfqpoint{0.516849in}{0.404175in}}%
\pgfpathlineto{\pgfqpoint{0.466848in}{0.405970in}}%
\pgfpathlineto{\pgfqpoint{0.456130in}{0.407931in}}%
\pgfpathlineto{\pgfqpoint{0.452340in}{0.410303in}}%
\pgfpathlineto{\pgfqpoint{0.450346in}{0.414662in}}%
\pgfpathlineto{\pgfqpoint{0.449266in}{0.424524in}}%
\pgfpathlineto{\pgfqpoint{0.448771in}{0.464344in}}%
\pgfpathlineto{\pgfqpoint{0.448640in}{0.850171in}}%
\pgfpathlineto{\pgfqpoint{0.448653in}{2.891318in}}%
\pgfpathlineto{\pgfqpoint{0.448653in}{2.891318in}}%
\pgfusepath{stroke}%
\end{pgfscope}%
\begin{pgfscope}%
\pgfpathrectangle{\pgfqpoint{0.448634in}{0.402556in}}{\pgfqpoint{4.350661in}{2.489204in}} %
\pgfusepath{clip}%
\pgfsetrectcap%
\pgfsetroundjoin%
\pgfsetlinewidth{1.003750pt}%
\definecolor{currentstroke}{rgb}{0.121569,0.466667,0.705882}%
\pgfsetstrokecolor{currentstroke}%
\pgfsetdash{}{0pt}%
\pgfpathmoveto{\pgfqpoint{3.428189in}{0.402586in}}%
\pgfpathlineto{\pgfqpoint{2.782121in}{0.403701in}}%
\pgfpathlineto{\pgfqpoint{2.753906in}{0.405674in}}%
\pgfpathlineto{\pgfqpoint{2.743328in}{0.408443in}}%
\pgfpathlineto{\pgfqpoint{2.737717in}{0.412188in}}%
\pgfpathlineto{\pgfqpoint{2.733668in}{0.417995in}}%
\pgfpathlineto{\pgfqpoint{2.730649in}{0.427307in}}%
\pgfpathlineto{\pgfqpoint{2.728388in}{0.442004in}}%
\pgfpathlineto{\pgfqpoint{2.726544in}{0.471794in}}%
\pgfpathlineto{\pgfqpoint{2.725216in}{0.534003in}}%
\pgfpathlineto{\pgfqpoint{2.725169in}{0.655973in}}%
\pgfpathlineto{\pgfqpoint{2.727377in}{0.832687in}}%
\pgfpathlineto{\pgfqpoint{2.732259in}{1.041703in}}%
\pgfpathlineto{\pgfqpoint{2.738851in}{1.223257in}}%
\pgfpathlineto{\pgfqpoint{2.747078in}{1.389766in}}%
\pgfpathlineto{\pgfqpoint{2.756608in}{1.538717in}}%
\pgfpathlineto{\pgfqpoint{2.768955in}{1.694887in}}%
\pgfpathlineto{\pgfqpoint{2.781228in}{1.816044in}}%
\pgfpathlineto{\pgfqpoint{2.794401in}{1.924524in}}%
\pgfpathlineto{\pgfqpoint{2.812737in}{2.054722in}}%
\pgfpathlineto{\pgfqpoint{2.828774in}{2.147512in}}%
\pgfpathlineto{\pgfqpoint{2.847382in}{2.242224in}}%
\pgfpathlineto{\pgfqpoint{2.895818in}{2.479699in}}%
\pgfpathlineto{\pgfqpoint{2.900204in}{2.516689in}}%
\pgfpathlineto{\pgfqpoint{2.901346in}{2.544029in}}%
\pgfpathlineto{\pgfqpoint{2.900291in}{2.566388in}}%
\pgfpathlineto{\pgfqpoint{2.897334in}{2.585999in}}%
\pgfpathlineto{\pgfqpoint{2.892836in}{2.602633in}}%
\pgfpathlineto{\pgfqpoint{2.886394in}{2.618405in}}%
\pgfpathlineto{\pgfqpoint{2.878058in}{2.632969in}}%
\pgfpathlineto{\pgfqpoint{2.868065in}{2.646100in}}%
\pgfpathlineto{\pgfqpoint{2.855050in}{2.659300in}}%
\pgfpathlineto{\pgfqpoint{2.840801in}{2.670717in}}%
\pgfpathlineto{\pgfqpoint{2.821822in}{2.682861in}}%
\pgfpathlineto{\pgfqpoint{2.799980in}{2.694026in}}%
\pgfpathlineto{\pgfqpoint{2.773366in}{2.704944in}}%
\pgfpathlineto{\pgfqpoint{2.742012in}{2.715266in}}%
\pgfpathlineto{\pgfqpoint{2.705983in}{2.724785in}}%
\pgfpathlineto{\pgfqpoint{2.663200in}{2.733810in}}%
\pgfpathlineto{\pgfqpoint{2.611535in}{2.742379in}}%
\pgfpathlineto{\pgfqpoint{2.551002in}{2.750090in}}%
\pgfpathlineto{\pgfqpoint{2.481632in}{2.756682in}}%
\pgfpathlineto{\pgfqpoint{2.399112in}{2.762200in}}%
\pgfpathlineto{\pgfqpoint{2.309985in}{2.765886in}}%
\pgfpathlineto{\pgfqpoint{2.188184in}{2.768096in}}%
\pgfpathlineto{\pgfqpoint{2.081595in}{2.767619in}}%
\pgfpathlineto{\pgfqpoint{1.968506in}{2.764840in}}%
\pgfpathlineto{\pgfqpoint{1.864180in}{2.759918in}}%
\pgfpathlineto{\pgfqpoint{1.757786in}{2.752593in}}%
\pgfpathlineto{\pgfqpoint{1.671087in}{2.744171in}}%
\pgfpathlineto{\pgfqpoint{1.591076in}{2.734193in}}%
\pgfpathlineto{\pgfqpoint{1.502689in}{2.720717in}}%
\pgfpathlineto{\pgfqpoint{1.427655in}{2.706083in}}%
\pgfpathlineto{\pgfqpoint{1.372350in}{2.692544in}}%
\pgfpathlineto{\pgfqpoint{1.321734in}{2.677921in}}%
\pgfpathlineto{\pgfqpoint{1.273765in}{2.661664in}}%
\pgfpathlineto{\pgfqpoint{1.230567in}{2.644672in}}%
\pgfpathlineto{\pgfqpoint{1.192197in}{2.627106in}}%
\pgfpathlineto{\pgfqpoint{1.156620in}{2.608403in}}%
\pgfpathlineto{\pgfqpoint{1.123890in}{2.588716in}}%
\pgfpathlineto{\pgfqpoint{1.095883in}{2.569568in}}%
\pgfpathlineto{\pgfqpoint{1.063936in}{2.543701in}}%
\pgfpathlineto{\pgfqpoint{1.038217in}{2.520732in}}%
\pgfpathlineto{\pgfqpoint{1.013766in}{2.496016in}}%
\pgfpathlineto{\pgfqpoint{0.990704in}{2.469610in}}%
\pgfpathlineto{\pgfqpoint{0.969124in}{2.441612in}}%
\pgfpathlineto{\pgfqpoint{0.949083in}{2.412154in}}%
\pgfpathlineto{\pgfqpoint{0.930604in}{2.381387in}}%
\pgfpathlineto{\pgfqpoint{0.906555in}{2.334052in}}%
\pgfpathlineto{\pgfqpoint{0.889925in}{2.296262in}}%
\pgfpathlineto{\pgfqpoint{0.874241in}{2.255213in}}%
\pgfpathlineto{\pgfqpoint{0.859667in}{2.210961in}}%
\pgfpathlineto{\pgfqpoint{0.846986in}{2.165954in}}%
\pgfpathlineto{\pgfqpoint{0.839633in}{2.134715in}}%
\pgfpathlineto{\pgfqpoint{0.828238in}{2.081532in}}%
\pgfpathlineto{\pgfqpoint{0.817866in}{2.022986in}}%
\pgfpathlineto{\pgfqpoint{0.810784in}{1.971352in}}%
\pgfpathlineto{\pgfqpoint{0.802846in}{1.902252in}}%
\pgfpathlineto{\pgfqpoint{0.796554in}{1.827927in}}%
\pgfpathlineto{\pgfqpoint{0.791696in}{1.743480in}}%
\pgfpathlineto{\pgfqpoint{0.787773in}{1.621595in}}%
\pgfpathlineto{\pgfqpoint{0.785408in}{1.522064in}}%
\pgfpathlineto{\pgfqpoint{0.785408in}{1.522064in}}%
\pgfusepath{stroke}%
\end{pgfscope}%
\begin{pgfscope}%
\pgfpathrectangle{\pgfqpoint{0.448634in}{0.402556in}}{\pgfqpoint{4.350661in}{2.489204in}} %
\pgfusepath{clip}%
\pgfsetrectcap%
\pgfsetroundjoin%
\pgfsetlinewidth{1.003750pt}%
\definecolor{currentstroke}{rgb}{0.121569,0.466667,0.705882}%
\pgfsetstrokecolor{currentstroke}%
\pgfsetdash{}{0pt}%
\pgfpathmoveto{\pgfqpoint{2.028735in}{0.425754in}}%
\pgfpathlineto{\pgfqpoint{1.878677in}{0.421879in}}%
\pgfpathlineto{\pgfqpoint{1.676387in}{0.418997in}}%
\pgfpathlineto{\pgfqpoint{1.413176in}{0.417558in}}%
\pgfpathlineto{\pgfqpoint{1.134735in}{0.418204in}}%
\pgfpathlineto{\pgfqpoint{0.921565in}{0.420769in}}%
\pgfpathlineto{\pgfqpoint{0.782384in}{0.424523in}}%
\pgfpathlineto{\pgfqpoint{0.693283in}{0.428974in}}%
\pgfpathlineto{\pgfqpoint{0.632541in}{0.434091in}}%
\pgfpathlineto{\pgfqpoint{0.591492in}{0.439564in}}%
\pgfpathlineto{\pgfqpoint{0.561503in}{0.445595in}}%
\pgfpathlineto{\pgfqpoint{0.538349in}{0.452466in}}%
\pgfpathlineto{\pgfqpoint{0.522042in}{0.459394in}}%
\pgfpathlineto{\pgfqpoint{0.508540in}{0.467420in}}%
\pgfpathlineto{\pgfqpoint{0.497973in}{0.476161in}}%
\pgfpathlineto{\pgfqpoint{0.488790in}{0.486750in}}%
\pgfpathlineto{\pgfqpoint{0.481284in}{0.498948in}}%
\pgfpathlineto{\pgfqpoint{0.474590in}{0.514580in}}%
\pgfpathlineto{\pgfqpoint{0.469106in}{0.533467in}}%
\pgfpathlineto{\pgfqpoint{0.464439in}{0.557771in}}%
\pgfpathlineto{\pgfqpoint{0.460297in}{0.592289in}}%
\pgfpathlineto{\pgfqpoint{0.456856in}{0.641912in}}%
\pgfpathlineto{\pgfqpoint{0.454122in}{0.716520in}}%
\pgfpathlineto{\pgfqpoint{0.451978in}{0.843444in}}%
\pgfpathlineto{\pgfqpoint{0.450459in}{1.087380in}}%
\pgfpathlineto{\pgfqpoint{0.449596in}{1.657406in}}%
\pgfpathlineto{\pgfqpoint{0.450150in}{2.687936in}}%
\pgfpathlineto{\pgfqpoint{0.451781in}{2.839761in}}%
\pgfpathlineto{\pgfqpoint{0.453975in}{2.872003in}}%
\pgfpathlineto{\pgfqpoint{0.456339in}{2.881553in}}%
\pgfpathlineto{\pgfqpoint{0.458888in}{2.885549in}}%
\pgfpathlineto{\pgfqpoint{0.462554in}{2.888171in}}%
\pgfpathlineto{\pgfqpoint{0.471046in}{2.890205in}}%
\pgfpathlineto{\pgfqpoint{0.490597in}{2.891263in}}%
\pgfpathlineto{\pgfqpoint{0.564556in}{2.891692in}}%
\pgfpathlineto{\pgfqpoint{1.569559in}{2.891759in}}%
\pgfpathlineto{\pgfqpoint{4.784679in}{2.890785in}}%
\pgfpathlineto{\pgfqpoint{4.791005in}{2.889098in}}%
\pgfpathlineto{\pgfqpoint{4.793910in}{2.885555in}}%
\pgfpathlineto{\pgfqpoint{4.795579in}{2.878366in}}%
\pgfpathlineto{\pgfqpoint{4.796850in}{2.858513in}}%
\pgfpathlineto{\pgfqpoint{4.796850in}{2.858513in}}%
\pgfusepath{stroke}%
\end{pgfscope}%
\begin{pgfscope}%
\pgfpathrectangle{\pgfqpoint{0.448634in}{0.402556in}}{\pgfqpoint{4.350661in}{2.489204in}} %
\pgfusepath{clip}%
\pgfsetrectcap%
\pgfsetroundjoin%
\pgfsetlinewidth{1.003750pt}%
\definecolor{currentstroke}{rgb}{1.000000,0.498039,0.054902}%
\pgfsetstrokecolor{currentstroke}%
\pgfsetdash{}{0pt}%
\pgfpathmoveto{\pgfqpoint{0.448634in}{2.896245in}}%
\pgfpathlineto{\pgfqpoint{0.448593in}{0.407043in}}%
\pgfpathlineto{\pgfqpoint{0.448593in}{0.407043in}}%
\pgfusepath{stroke}%
\end{pgfscope}%
\begin{pgfscope}%
\pgfpathrectangle{\pgfqpoint{0.448634in}{0.402556in}}{\pgfqpoint{4.350661in}{2.489204in}} %
\pgfusepath{clip}%
\pgfsetrectcap%
\pgfsetroundjoin%
\pgfsetlinewidth{1.003750pt}%
\definecolor{currentstroke}{rgb}{1.000000,0.498039,0.054902}%
\pgfsetstrokecolor{currentstroke}%
\pgfsetdash{}{0pt}%
\pgfpathmoveto{\pgfqpoint{0.576853in}{1.760817in}}%
\pgfpathlineto{\pgfqpoint{0.569394in}{1.840010in}}%
\pgfpathlineto{\pgfqpoint{0.563209in}{1.929338in}}%
\pgfpathlineto{\pgfqpoint{0.558592in}{2.028764in}}%
\pgfpathlineto{\pgfqpoint{0.555985in}{2.133265in}}%
\pgfpathlineto{\pgfqpoint{0.555566in}{2.237808in}}%
\pgfpathlineto{\pgfqpoint{0.557371in}{2.337352in}}%
\pgfpathlineto{\pgfqpoint{0.561096in}{2.424366in}}%
\pgfpathlineto{\pgfqpoint{0.566403in}{2.498791in}}%
\pgfpathlineto{\pgfqpoint{0.572909in}{2.560570in}}%
\pgfpathlineto{\pgfqpoint{0.580458in}{2.612119in}}%
\pgfpathlineto{\pgfqpoint{0.589086in}{2.655816in}}%
\pgfpathlineto{\pgfqpoint{0.598406in}{2.691589in}}%
\pgfpathlineto{\pgfqpoint{0.608613in}{2.721757in}}%
\pgfpathlineto{\pgfqpoint{0.619241in}{2.746278in}}%
\pgfpathlineto{\pgfqpoint{0.630817in}{2.767339in}}%
\pgfpathlineto{\pgfqpoint{0.642975in}{2.784884in}}%
\pgfpathlineto{\pgfqpoint{0.656813in}{2.800712in}}%
\pgfpathlineto{\pgfqpoint{0.672197in}{2.814549in}}%
\pgfpathlineto{\pgfqpoint{0.688853in}{2.826301in}}%
\pgfpathlineto{\pgfqpoint{0.706461in}{2.836076in}}%
\pgfpathlineto{\pgfqpoint{0.726804in}{2.844875in}}%
\pgfpathlineto{\pgfqpoint{0.751866in}{2.853203in}}%
\pgfpathlineto{\pgfqpoint{0.781631in}{2.860547in}}%
\pgfpathlineto{\pgfqpoint{0.818168in}{2.867054in}}%
\pgfpathlineto{\pgfqpoint{0.863581in}{2.872685in}}%
\pgfpathlineto{\pgfqpoint{0.922161in}{2.877518in}}%
\pgfpathlineto{\pgfqpoint{1.000391in}{2.881567in}}%
\pgfpathlineto{\pgfqpoint{1.111294in}{2.884881in}}%
\pgfpathlineto{\pgfqpoint{1.274428in}{2.887367in}}%
\pgfpathlineto{\pgfqpoint{1.552865in}{2.889263in}}%
\pgfpathlineto{\pgfqpoint{2.107573in}{2.890457in}}%
\pgfpathlineto{\pgfqpoint{3.343161in}{2.890573in}}%
\pgfpathlineto{\pgfqpoint{4.043615in}{2.888941in}}%
\pgfpathlineto{\pgfqpoint{4.289417in}{2.886404in}}%
\pgfpathlineto{\pgfqpoint{4.413375in}{2.883093in}}%
\pgfpathlineto{\pgfqpoint{4.489424in}{2.878997in}}%
\pgfpathlineto{\pgfqpoint{4.541451in}{2.874081in}}%
\pgfpathlineto{\pgfqpoint{4.578100in}{2.868470in}}%
\pgfpathlineto{\pgfqpoint{4.605818in}{2.862092in}}%
\pgfpathlineto{\pgfqpoint{4.626725in}{2.855245in}}%
\pgfpathlineto{\pgfqpoint{4.644925in}{2.847018in}}%
\pgfpathlineto{\pgfqpoint{4.660241in}{2.837590in}}%
\pgfpathlineto{\pgfqpoint{4.672623in}{2.827468in}}%
\pgfpathlineto{\pgfqpoint{4.683751in}{2.815592in}}%
\pgfpathlineto{\pgfqpoint{4.693406in}{2.802135in}}%
\pgfpathlineto{\pgfqpoint{4.702740in}{2.785343in}}%
\pgfpathlineto{\pgfqpoint{4.711277in}{2.765194in}}%
\pgfpathlineto{\pgfqpoint{4.719482in}{2.739484in}}%
\pgfpathlineto{\pgfqpoint{4.726293in}{2.710657in}}%
\pgfpathlineto{\pgfqpoint{4.733259in}{2.671643in}}%
\pgfpathlineto{\pgfqpoint{4.739604in}{2.622396in}}%
\pgfpathlineto{\pgfqpoint{4.745236in}{2.560504in}}%
\pgfpathlineto{\pgfqpoint{4.750164in}{2.481052in}}%
\pgfpathlineto{\pgfqpoint{4.754367in}{2.376618in}}%
\pgfpathlineto{\pgfqpoint{4.757443in}{2.242249in}}%
\pgfpathlineto{\pgfqpoint{4.758977in}{2.075483in}}%
\pgfpathlineto{\pgfqpoint{4.758447in}{1.888795in}}%
\pgfpathlineto{\pgfqpoint{4.755756in}{1.707111in}}%
\pgfpathlineto{\pgfqpoint{4.750925in}{1.532957in}}%
\pgfpathlineto{\pgfqpoint{4.744785in}{1.398726in}}%
\pgfpathlineto{\pgfqpoint{4.737575in}{1.289516in}}%
\pgfpathlineto{\pgfqpoint{4.728714in}{1.190470in}}%
\pgfpathlineto{\pgfqpoint{4.719652in}{1.116521in}}%
\pgfpathlineto{\pgfqpoint{4.710036in}{1.055276in}}%
\pgfpathlineto{\pgfqpoint{4.699503in}{1.001861in}}%
\pgfpathlineto{\pgfqpoint{4.689040in}{0.958690in}}%
\pgfpathlineto{\pgfqpoint{4.677219in}{0.918600in}}%
\pgfpathlineto{\pgfqpoint{4.664034in}{0.881749in}}%
\pgfpathlineto{\pgfqpoint{4.650584in}{0.850492in}}%
\pgfpathlineto{\pgfqpoint{4.636303in}{0.822570in}}%
\pgfpathlineto{\pgfqpoint{4.620207in}{0.795974in}}%
\pgfpathlineto{\pgfqpoint{4.603640in}{0.772901in}}%
\pgfpathlineto{\pgfqpoint{4.585488in}{0.751446in}}%
\pgfpathlineto{\pgfqpoint{4.565874in}{0.731749in}}%
\pgfpathlineto{\pgfqpoint{4.544964in}{0.713879in}}%
\pgfpathlineto{\pgfqpoint{4.522958in}{0.697824in}}%
\pgfpathlineto{\pgfqpoint{4.496157in}{0.681290in}}%
\pgfpathlineto{\pgfqpoint{4.470397in}{0.667953in}}%
\pgfpathlineto{\pgfqpoint{4.439961in}{0.654509in}}%
\pgfpathlineto{\pgfqpoint{4.406841in}{0.642281in}}%
\pgfpathlineto{\pgfqpoint{4.369009in}{0.630748in}}%
\pgfpathlineto{\pgfqpoint{4.326489in}{0.620226in}}%
\pgfpathlineto{\pgfqpoint{4.279327in}{0.610949in}}%
\pgfpathlineto{\pgfqpoint{4.227576in}{0.603085in}}%
\pgfpathlineto{\pgfqpoint{4.173450in}{0.597063in}}%
\pgfpathlineto{\pgfqpoint{4.110511in}{0.592203in}}%
\pgfpathlineto{\pgfqpoint{4.047471in}{0.589537in}}%
\pgfpathlineto{\pgfqpoint{3.977867in}{0.588624in}}%
\pgfpathlineto{\pgfqpoint{3.906093in}{0.589934in}}%
\pgfpathlineto{\pgfqpoint{3.834377in}{0.593496in}}%
\pgfpathlineto{\pgfqpoint{3.767120in}{0.599067in}}%
\pgfpathlineto{\pgfqpoint{3.704364in}{0.606392in}}%
\pgfpathlineto{\pgfqpoint{3.678516in}{0.610510in}}%
\pgfpathlineto{\pgfqpoint{3.620438in}{0.620500in}}%
\pgfpathlineto{\pgfqpoint{3.586319in}{0.628207in}}%
\pgfpathlineto{\pgfqpoint{3.495240in}{0.652428in}}%
\pgfpathlineto{\pgfqpoint{3.451528in}{0.667583in}}%
\pgfpathlineto{\pgfqpoint{3.408538in}{0.685220in}}%
\pgfpathlineto{\pgfqpoint{3.374594in}{0.702001in}}%
\pgfpathlineto{\pgfqpoint{3.345407in}{0.718682in}}%
\pgfpathlineto{\pgfqpoint{3.315236in}{0.738520in}}%
\pgfpathlineto{\pgfqpoint{3.288127in}{0.759290in}}%
\pgfpathlineto{\pgfqpoint{3.264004in}{0.780551in}}%
\pgfpathlineto{\pgfqpoint{3.241208in}{0.803648in}}%
\pgfpathlineto{\pgfqpoint{3.219894in}{0.828530in}}%
\pgfpathlineto{\pgfqpoint{3.200189in}{0.855091in}}%
\pgfpathlineto{\pgfqpoint{3.182177in}{0.883182in}}%
\pgfpathlineto{\pgfqpoint{3.165906in}{0.912633in}}%
\pgfpathlineto{\pgfqpoint{3.150351in}{0.945448in}}%
\pgfpathlineto{\pgfqpoint{3.136682in}{0.979345in}}%
\pgfpathlineto{\pgfqpoint{3.124073in}{1.016460in}}%
\pgfpathlineto{\pgfqpoint{3.112834in}{1.056769in}}%
\pgfpathlineto{\pgfqpoint{3.103046in}{1.100146in}}%
\pgfpathlineto{\pgfqpoint{3.095343in}{1.144071in}}%
\pgfpathlineto{\pgfqpoint{3.089208in}{1.190837in}}%
\pgfpathlineto{\pgfqpoint{3.084595in}{1.242838in}}%
\pgfpathlineto{\pgfqpoint{3.082137in}{1.295031in}}%
\pgfpathlineto{\pgfqpoint{3.081687in}{1.349787in}}%
\pgfpathlineto{\pgfqpoint{3.083451in}{1.406998in}}%
\pgfpathlineto{\pgfqpoint{3.087181in}{1.461589in}}%
\pgfpathlineto{\pgfqpoint{3.093485in}{1.520888in}}%
\pgfpathlineto{\pgfqpoint{3.101823in}{1.577334in}}%
\pgfpathlineto{\pgfqpoint{3.111930in}{1.630856in}}%
\pgfpathlineto{\pgfqpoint{3.124690in}{1.686208in}}%
\pgfpathlineto{\pgfqpoint{3.139178in}{1.738395in}}%
\pgfpathlineto{\pgfqpoint{3.155145in}{1.787366in}}%
\pgfpathlineto{\pgfqpoint{3.172353in}{1.833085in}}%
\pgfpathlineto{\pgfqpoint{3.191618in}{1.877716in}}%
\pgfpathlineto{\pgfqpoint{3.214026in}{1.923261in}}%
\pgfpathlineto{\pgfqpoint{3.236214in}{1.963157in}}%
\pgfpathlineto{\pgfqpoint{3.260178in}{2.001684in}}%
\pgfpathlineto{\pgfqpoint{3.285814in}{2.038776in}}%
\pgfpathlineto{\pgfqpoint{3.314415in}{2.076285in}}%
\pgfpathlineto{\pgfqpoint{3.348944in}{2.117711in}}%
\pgfpathlineto{\pgfqpoint{3.417133in}{2.198022in}}%
\pgfpathlineto{\pgfqpoint{3.426053in}{2.212128in}}%
\pgfpathlineto{\pgfqpoint{3.430798in}{2.223297in}}%
\pgfpathlineto{\pgfqpoint{3.432034in}{2.230603in}}%
\pgfpathlineto{\pgfqpoint{3.430773in}{2.237856in}}%
\pgfpathlineto{\pgfqpoint{3.426621in}{2.243526in}}%
\pgfpathlineto{\pgfqpoint{3.420908in}{2.247084in}}%
\pgfpathlineto{\pgfqpoint{3.412501in}{2.249583in}}%
\pgfpathlineto{\pgfqpoint{3.399499in}{2.250689in}}%
\pgfpathlineto{\pgfqpoint{3.384305in}{2.249671in}}%
\pgfpathlineto{\pgfqpoint{3.364985in}{2.246098in}}%
\pgfpathlineto{\pgfqpoint{3.341804in}{2.239342in}}%
\pgfpathlineto{\pgfqpoint{3.317109in}{2.229682in}}%
\pgfpathlineto{\pgfqpoint{3.291104in}{2.216986in}}%
\pgfpathlineto{\pgfqpoint{3.265928in}{2.202261in}}%
\pgfpathlineto{\pgfqpoint{3.239805in}{2.184361in}}%
\pgfpathlineto{\pgfqpoint{3.214775in}{2.164519in}}%
\pgfpathlineto{\pgfqpoint{3.190900in}{2.142893in}}%
\pgfpathlineto{\pgfqpoint{3.166657in}{2.117912in}}%
\pgfpathlineto{\pgfqpoint{3.143835in}{2.091233in}}%
\pgfpathlineto{\pgfqpoint{3.121079in}{2.061107in}}%
\pgfpathlineto{\pgfqpoint{3.099952in}{2.029463in}}%
\pgfpathlineto{\pgfqpoint{3.079251in}{1.994406in}}%
\pgfpathlineto{\pgfqpoint{3.059218in}{1.955915in}}%
\pgfpathlineto{\pgfqpoint{3.040058in}{1.914015in}}%
\pgfpathlineto{\pgfqpoint{3.022809in}{1.871041in}}%
\pgfpathlineto{\pgfqpoint{3.005790in}{1.822536in}}%
\pgfpathlineto{\pgfqpoint{2.990067in}{1.770819in}}%
\pgfpathlineto{\pgfqpoint{2.975708in}{1.715979in}}%
\pgfpathlineto{\pgfqpoint{2.962284in}{1.655680in}}%
\pgfpathlineto{\pgfqpoint{2.950496in}{1.592386in}}%
\pgfpathlineto{\pgfqpoint{2.940383in}{1.526185in}}%
\pgfpathlineto{\pgfqpoint{2.931745in}{1.454681in}}%
\pgfpathlineto{\pgfqpoint{2.925082in}{1.380399in}}%
\pgfpathlineto{\pgfqpoint{2.920647in}{1.305899in}}%
\pgfpathlineto{\pgfqpoint{2.918444in}{1.231270in}}%
\pgfpathlineto{\pgfqpoint{2.918545in}{1.159087in}}%
\pgfpathlineto{\pgfqpoint{2.920787in}{1.091931in}}%
\pgfpathlineto{\pgfqpoint{2.925177in}{1.027412in}}%
\pgfpathlineto{\pgfqpoint{2.931192in}{0.970580in}}%
\pgfpathlineto{\pgfqpoint{2.938760in}{0.919034in}}%
\pgfpathlineto{\pgfqpoint{2.947651in}{0.872852in}}%
\pgfpathlineto{\pgfqpoint{2.958213in}{0.829714in}}%
\pgfpathlineto{\pgfqpoint{2.969670in}{0.792114in}}%
\pgfpathlineto{\pgfqpoint{2.982463in}{0.757773in}}%
\pgfpathlineto{\pgfqpoint{2.996425in}{0.726812in}}%
\pgfpathlineto{\pgfqpoint{3.011299in}{0.699300in}}%
\pgfpathlineto{\pgfqpoint{3.026739in}{0.675225in}}%
\pgfpathlineto{\pgfqpoint{3.043828in}{0.652656in}}%
\pgfpathlineto{\pgfqpoint{3.062495in}{0.631788in}}%
\pgfpathlineto{\pgfqpoint{3.082602in}{0.612753in}}%
\pgfpathlineto{\pgfqpoint{3.103961in}{0.595592in}}%
\pgfpathlineto{\pgfqpoint{3.128268in}{0.579069in}}%
\pgfpathlineto{\pgfqpoint{3.153537in}{0.564554in}}%
\pgfpathlineto{\pgfqpoint{3.181571in}{0.550952in}}%
\pgfpathlineto{\pgfqpoint{3.214371in}{0.537647in}}%
\pgfpathlineto{\pgfqpoint{3.249846in}{0.525712in}}%
\pgfpathlineto{\pgfqpoint{3.290011in}{0.514571in}}%
\pgfpathlineto{\pgfqpoint{3.334820in}{0.504423in}}%
\pgfpathlineto{\pgfqpoint{3.386372in}{0.494999in}}%
\pgfpathlineto{\pgfqpoint{3.446798in}{0.486257in}}%
\pgfpathlineto{\pgfqpoint{3.518243in}{0.478282in}}%
\pgfpathlineto{\pgfqpoint{3.600685in}{0.471409in}}%
\pgfpathlineto{\pgfqpoint{3.696268in}{0.465713in}}%
\pgfpathlineto{\pgfqpoint{3.807144in}{0.461369in}}%
\pgfpathlineto{\pgfqpoint{3.933291in}{0.458719in}}%
\pgfpathlineto{\pgfqpoint{4.063808in}{0.458211in}}%
\pgfpathlineto{\pgfqpoint{4.187792in}{0.459914in}}%
\pgfpathlineto{\pgfqpoint{4.294335in}{0.463521in}}%
\pgfpathlineto{\pgfqpoint{4.381234in}{0.468574in}}%
\pgfpathlineto{\pgfqpoint{4.450636in}{0.474701in}}%
\pgfpathlineto{\pgfqpoint{4.506850in}{0.481799in}}%
\pgfpathlineto{\pgfqpoint{4.552009in}{0.489658in}}%
\pgfpathlineto{\pgfqpoint{4.588239in}{0.498115in}}%
\pgfpathlineto{\pgfqpoint{4.617656in}{0.507110in}}%
\pgfpathlineto{\pgfqpoint{4.642328in}{0.516843in}}%
\pgfpathlineto{\pgfqpoint{4.664194in}{0.527940in}}%
\pgfpathlineto{\pgfqpoint{4.681238in}{0.538945in}}%
\pgfpathlineto{\pgfqpoint{4.697164in}{0.551953in}}%
\pgfpathlineto{\pgfqpoint{4.710076in}{0.565289in}}%
\pgfpathlineto{\pgfqpoint{4.721578in}{0.580218in}}%
\pgfpathlineto{\pgfqpoint{4.731557in}{0.596521in}}%
\pgfpathlineto{\pgfqpoint{4.741000in}{0.616134in}}%
\pgfpathlineto{\pgfqpoint{4.749521in}{0.639027in}}%
\pgfpathlineto{\pgfqpoint{4.757522in}{0.667450in}}%
\pgfpathlineto{\pgfqpoint{4.764572in}{0.701345in}}%
\pgfpathlineto{\pgfqpoint{4.770840in}{0.743043in}}%
\pgfpathlineto{\pgfqpoint{4.776327in}{0.794934in}}%
\pgfpathlineto{\pgfqpoint{4.781278in}{0.864398in}}%
\pgfpathlineto{\pgfqpoint{4.785468in}{0.956371in}}%
\pgfpathlineto{\pgfqpoint{4.789000in}{1.085745in}}%
\pgfpathlineto{\pgfqpoint{4.791852in}{1.277385in}}%
\pgfpathlineto{\pgfqpoint{4.793959in}{1.581057in}}%
\pgfpathlineto{\pgfqpoint{4.794962in}{2.071429in}}%
\pgfpathlineto{\pgfqpoint{4.793967in}{2.559311in}}%
\pgfpathlineto{\pgfqpoint{4.791733in}{2.745981in}}%
\pgfpathlineto{\pgfqpoint{4.788955in}{2.818091in}}%
\pgfpathlineto{\pgfqpoint{4.785731in}{2.850227in}}%
\pgfpathlineto{\pgfqpoint{4.781879in}{2.867057in}}%
\pgfpathlineto{\pgfqpoint{4.777744in}{2.875780in}}%
\pgfpathlineto{\pgfqpoint{4.773097in}{2.880982in}}%
\pgfpathlineto{\pgfqpoint{4.767363in}{2.884504in}}%
\pgfpathlineto{\pgfqpoint{4.756853in}{2.887622in}}%
\pgfpathlineto{\pgfqpoint{4.739548in}{2.889639in}}%
\pgfpathlineto{\pgfqpoint{4.704762in}{2.890882in}}%
\pgfpathlineto{\pgfqpoint{4.602524in}{2.891538in}}%
\pgfpathlineto{\pgfqpoint{3.952100in}{2.891742in}}%
\pgfpathlineto{\pgfqpoint{0.617321in}{2.890753in}}%
\pgfpathlineto{\pgfqpoint{0.549910in}{2.888858in}}%
\pgfpathlineto{\pgfqpoint{0.521735in}{2.886179in}}%
\pgfpathlineto{\pgfqpoint{0.504666in}{2.882389in}}%
\pgfpathlineto{\pgfqpoint{0.494501in}{2.878011in}}%
\pgfpathlineto{\pgfqpoint{0.487180in}{2.872667in}}%
\pgfpathlineto{\pgfqpoint{0.481152in}{2.865519in}}%
\pgfpathlineto{\pgfqpoint{0.475664in}{2.854804in}}%
\pgfpathlineto{\pgfqpoint{0.471318in}{2.840737in}}%
\pgfpathlineto{\pgfqpoint{0.467301in}{2.818823in}}%
\pgfpathlineto{\pgfqpoint{0.463927in}{2.786700in}}%
\pgfpathlineto{\pgfqpoint{0.460918in}{2.734544in}}%
\pgfpathlineto{\pgfqpoint{0.458363in}{2.647473in}}%
\pgfpathlineto{\pgfqpoint{0.456575in}{2.523031in}}%
\pgfpathlineto{\pgfqpoint{0.456575in}{2.523031in}}%
\pgfusepath{stroke}%
\end{pgfscope}%
\begin{pgfscope}%
\pgfpathrectangle{\pgfqpoint{0.448634in}{0.402556in}}{\pgfqpoint{4.350661in}{2.489204in}} %
\pgfusepath{clip}%
\pgfsetrectcap%
\pgfsetroundjoin%
\pgfsetlinewidth{1.003750pt}%
\definecolor{currentstroke}{rgb}{1.000000,0.498039,0.054902}%
\pgfsetstrokecolor{currentstroke}%
\pgfsetdash{}{0pt}%
\pgfpathmoveto{\pgfqpoint{4.798840in}{2.852369in}}%
\pgfpathlineto{\pgfqpoint{4.797564in}{2.889610in}}%
\pgfpathlineto{\pgfqpoint{4.796215in}{2.891483in}}%
\pgfpathlineto{\pgfqpoint{4.787551in}{2.891760in}}%
\pgfpathlineto{\pgfqpoint{0.452128in}{2.891659in}}%
\pgfpathlineto{\pgfqpoint{0.450530in}{2.890082in}}%
\pgfpathlineto{\pgfqpoint{0.449454in}{2.882763in}}%
\pgfpathlineto{\pgfqpoint{0.448970in}{2.845432in}}%
\pgfpathlineto{\pgfqpoint{0.448743in}{2.494454in}}%
\pgfpathlineto{\pgfqpoint{0.449624in}{0.615107in}}%
\pgfpathlineto{\pgfqpoint{0.451433in}{0.510586in}}%
\pgfpathlineto{\pgfqpoint{0.453993in}{0.473374in}}%
\pgfpathlineto{\pgfqpoint{0.457406in}{0.453868in}}%
\pgfpathlineto{\pgfqpoint{0.461540in}{0.442384in}}%
\pgfpathlineto{\pgfqpoint{0.466739in}{0.434437in}}%
\pgfpathlineto{\pgfqpoint{0.473595in}{0.428350in}}%
\pgfpathlineto{\pgfqpoint{0.483492in}{0.423244in}}%
\pgfpathlineto{\pgfqpoint{0.491854in}{0.420501in}}%
\pgfpathlineto{\pgfqpoint{0.491854in}{0.420501in}}%
\pgfusepath{stroke}%
\end{pgfscope}%
\begin{pgfscope}%
\pgfpathrectangle{\pgfqpoint{0.448634in}{0.402556in}}{\pgfqpoint{4.350661in}{2.489204in}} %
\pgfusepath{clip}%
\pgfsetrectcap%
\pgfsetroundjoin%
\pgfsetlinewidth{1.003750pt}%
\definecolor{currentstroke}{rgb}{1.000000,0.498039,0.054902}%
\pgfsetstrokecolor{currentstroke}%
\pgfsetdash{}{0pt}%
\pgfpathmoveto{\pgfqpoint{0.456424in}{1.370137in}}%
\pgfpathlineto{\pgfqpoint{0.459610in}{1.118755in}}%
\pgfpathlineto{\pgfqpoint{0.463695in}{0.962007in}}%
\pgfpathlineto{\pgfqpoint{0.468519in}{0.857610in}}%
\pgfpathlineto{\pgfqpoint{0.474082in}{0.783210in}}%
\pgfpathlineto{\pgfqpoint{0.480226in}{0.728906in}}%
\pgfpathlineto{\pgfqpoint{0.486970in}{0.687306in}}%
\pgfpathlineto{\pgfqpoint{0.494537in}{0.653558in}}%
\pgfpathlineto{\pgfqpoint{0.503107in}{0.625355in}}%
\pgfpathlineto{\pgfqpoint{0.512193in}{0.602749in}}%
\pgfpathlineto{\pgfqpoint{0.522200in}{0.583508in}}%
\pgfpathlineto{\pgfqpoint{0.534108in}{0.565743in}}%
\pgfpathlineto{\pgfqpoint{0.546263in}{0.551507in}}%
\pgfpathlineto{\pgfqpoint{0.559728in}{0.538907in}}%
\pgfpathlineto{\pgfqpoint{0.576129in}{0.526693in}}%
\pgfpathlineto{\pgfqpoint{0.595483in}{0.515351in}}%
\pgfpathlineto{\pgfqpoint{0.617681in}{0.505147in}}%
\pgfpathlineto{\pgfqpoint{0.642568in}{0.496153in}}%
\pgfpathlineto{\pgfqpoint{0.672126in}{0.487778in}}%
\pgfpathlineto{\pgfqpoint{0.708443in}{0.479824in}}%
\pgfpathlineto{\pgfqpoint{0.753649in}{0.472325in}}%
\pgfpathlineto{\pgfqpoint{0.807717in}{0.465660in}}%
\pgfpathlineto{\pgfqpoint{0.877116in}{0.459475in}}%
\pgfpathlineto{\pgfqpoint{0.961828in}{0.454230in}}%
\pgfpathlineto{\pgfqpoint{1.068351in}{0.449916in}}%
\pgfpathlineto{\pgfqpoint{1.201018in}{0.446839in}}%
\pgfpathlineto{\pgfqpoint{1.357637in}{0.445481in}}%
\pgfpathlineto{\pgfqpoint{1.525135in}{0.446232in}}%
\pgfpathlineto{\pgfqpoint{1.686088in}{0.449142in}}%
\pgfpathlineto{\pgfqpoint{1.823074in}{0.453747in}}%
\pgfpathlineto{\pgfqpoint{1.938245in}{0.459764in}}%
\pgfpathlineto{\pgfqpoint{2.031582in}{0.466759in}}%
\pgfpathlineto{\pgfqpoint{2.109580in}{0.474745in}}%
\pgfpathlineto{\pgfqpoint{2.174384in}{0.483535in}}%
\pgfpathlineto{\pgfqpoint{2.228139in}{0.492940in}}%
\pgfpathlineto{\pgfqpoint{2.275119in}{0.503356in}}%
\pgfpathlineto{\pgfqpoint{2.315282in}{0.514501in}}%
\pgfpathlineto{\pgfqpoint{2.350698in}{0.526659in}}%
\pgfpathlineto{\pgfqpoint{2.381320in}{0.539536in}}%
\pgfpathlineto{\pgfqpoint{2.407164in}{0.552659in}}%
\pgfpathlineto{\pgfqpoint{2.430226in}{0.566639in}}%
\pgfpathlineto{\pgfqpoint{2.452282in}{0.582602in}}%
\pgfpathlineto{\pgfqpoint{2.471391in}{0.599069in}}%
\pgfpathlineto{\pgfqpoint{2.489240in}{0.617293in}}%
\pgfpathlineto{\pgfqpoint{2.505678in}{0.637180in}}%
\pgfpathlineto{\pgfqpoint{2.520620in}{0.658557in}}%
\pgfpathlineto{\pgfqpoint{2.535213in}{0.683314in}}%
\pgfpathlineto{\pgfqpoint{2.549115in}{0.711484in}}%
\pgfpathlineto{\pgfqpoint{2.562091in}{0.743004in}}%
\pgfpathlineto{\pgfqpoint{2.574020in}{0.777751in}}%
\pgfpathlineto{\pgfqpoint{2.585502in}{0.817970in}}%
\pgfpathlineto{\pgfqpoint{2.596809in}{0.866038in}}%
\pgfpathlineto{\pgfqpoint{2.607562in}{0.921948in}}%
\pgfpathlineto{\pgfqpoint{2.617925in}{0.988098in}}%
\pgfpathlineto{\pgfqpoint{2.627958in}{1.066918in}}%
\pgfpathlineto{\pgfqpoint{2.637941in}{1.163320in}}%
\pgfpathlineto{\pgfqpoint{2.648424in}{1.287199in}}%
\pgfpathlineto{\pgfqpoint{2.660103in}{1.453438in}}%
\pgfpathlineto{\pgfqpoint{2.674773in}{1.696801in}}%
\pgfpathlineto{\pgfqpoint{2.687716in}{1.945279in}}%
\pgfpathlineto{\pgfqpoint{2.692670in}{2.079573in}}%
\pgfpathlineto{\pgfqpoint{2.693829in}{2.166682in}}%
\pgfpathlineto{\pgfqpoint{2.692565in}{2.233870in}}%
\pgfpathlineto{\pgfqpoint{2.689436in}{2.286015in}}%
\pgfpathlineto{\pgfqpoint{2.684859in}{2.327999in}}%
\pgfpathlineto{\pgfqpoint{2.678725in}{2.364664in}}%
\pgfpathlineto{\pgfqpoint{2.671356in}{2.395897in}}%
\pgfpathlineto{\pgfqpoint{2.662489in}{2.423981in}}%
\pgfpathlineto{\pgfqpoint{2.652361in}{2.448778in}}%
\pgfpathlineto{\pgfqpoint{2.641365in}{2.470245in}}%
\pgfpathlineto{\pgfqpoint{2.628643in}{2.490425in}}%
\pgfpathlineto{\pgfqpoint{2.614279in}{2.509106in}}%
\pgfpathlineto{\pgfqpoint{2.598443in}{2.526159in}}%
\pgfpathlineto{\pgfqpoint{2.579590in}{2.543005in}}%
\pgfpathlineto{\pgfqpoint{2.559532in}{2.557923in}}%
\pgfpathlineto{\pgfqpoint{2.536602in}{2.572183in}}%
\pgfpathlineto{\pgfqpoint{2.510850in}{2.585538in}}%
\pgfpathlineto{\pgfqpoint{2.482360in}{2.597837in}}%
\pgfpathlineto{\pgfqpoint{2.449134in}{2.609683in}}%
\pgfpathlineto{\pgfqpoint{2.411184in}{2.620696in}}%
\pgfpathlineto{\pgfqpoint{2.368552in}{2.630606in}}%
\pgfpathlineto{\pgfqpoint{2.321294in}{2.639221in}}%
\pgfpathlineto{\pgfqpoint{2.269467in}{2.646399in}}%
\pgfpathlineto{\pgfqpoint{2.210954in}{2.652193in}}%
\pgfpathlineto{\pgfqpoint{2.147967in}{2.656153in}}%
\pgfpathlineto{\pgfqpoint{2.080556in}{2.658135in}}%
\pgfpathlineto{\pgfqpoint{2.010948in}{2.657971in}}%
\pgfpathlineto{\pgfqpoint{1.939195in}{2.655572in}}%
\pgfpathlineto{\pgfqpoint{1.867527in}{2.650913in}}%
\pgfpathlineto{\pgfqpoint{1.798171in}{2.644140in}}%
\pgfpathlineto{\pgfqpoint{1.733341in}{2.635606in}}%
\pgfpathlineto{\pgfqpoint{1.673075in}{2.625521in}}%
\pgfpathlineto{\pgfqpoint{1.615274in}{2.613610in}}%
\pgfpathlineto{\pgfqpoint{1.562133in}{2.600402in}}%
\pgfpathlineto{\pgfqpoint{1.513681in}{2.586139in}}%
\pgfpathlineto{\pgfqpoint{1.467862in}{2.570344in}}%
\pgfpathlineto{\pgfqpoint{1.426794in}{2.553923in}}%
\pgfpathlineto{\pgfqpoint{1.388447in}{2.536289in}}%
\pgfpathlineto{\pgfqpoint{1.352878in}{2.517566in}}%
\pgfpathlineto{\pgfqpoint{1.320128in}{2.497922in}}%
\pgfpathlineto{\pgfqpoint{1.288379in}{2.476236in}}%
\pgfpathlineto{\pgfqpoint{1.259592in}{2.453861in}}%
\pgfpathlineto{\pgfqpoint{1.232050in}{2.429520in}}%
\pgfpathlineto{\pgfqpoint{1.207527in}{2.404898in}}%
\pgfpathlineto{\pgfqpoint{1.184409in}{2.378557in}}%
\pgfpathlineto{\pgfqpoint{1.162828in}{2.350561in}}%
\pgfpathlineto{\pgfqpoint{1.142891in}{2.321011in}}%
\pgfpathlineto{\pgfqpoint{1.124675in}{2.290041in}}%
\pgfpathlineto{\pgfqpoint{1.108225in}{2.257802in}}%
\pgfpathlineto{\pgfqpoint{1.092639in}{2.222199in}}%
\pgfpathlineto{\pgfqpoint{1.079059in}{2.185535in}}%
\pgfpathlineto{\pgfqpoint{1.067443in}{2.147998in}}%
\pgfpathlineto{\pgfqpoint{1.057187in}{2.107348in}}%
\pgfpathlineto{\pgfqpoint{1.049004in}{2.066086in}}%
\pgfpathlineto{\pgfqpoint{1.042513in}{2.021906in}}%
\pgfpathlineto{\pgfqpoint{1.038177in}{1.977382in}}%
\pgfpathlineto{\pgfqpoint{1.035866in}{1.930167in}}%
\pgfpathlineto{\pgfqpoint{1.035826in}{1.882878in}}%
\pgfpathlineto{\pgfqpoint{1.038031in}{1.835656in}}%
\pgfpathlineto{\pgfqpoint{1.042474in}{1.788641in}}%
\pgfpathlineto{\pgfqpoint{1.049176in}{1.741979in}}%
\pgfpathlineto{\pgfqpoint{1.057644in}{1.698239in}}%
\pgfpathlineto{\pgfqpoint{1.068221in}{1.655105in}}%
\pgfpathlineto{\pgfqpoint{1.080962in}{1.612745in}}%
\pgfpathlineto{\pgfqpoint{1.095031in}{1.573617in}}%
\pgfpathlineto{\pgfqpoint{1.111115in}{1.535520in}}%
\pgfpathlineto{\pgfqpoint{1.128118in}{1.500775in}}%
\pgfpathlineto{\pgfqpoint{1.146930in}{1.467274in}}%
\pgfpathlineto{\pgfqpoint{1.167531in}{1.435181in}}%
\pgfpathlineto{\pgfqpoint{1.189874in}{1.404652in}}%
\pgfpathlineto{\pgfqpoint{1.213884in}{1.375828in}}%
\pgfpathlineto{\pgfqpoint{1.237817in}{1.350457in}}%
\pgfpathlineto{\pgfqpoint{1.264748in}{1.325237in}}%
\pgfpathlineto{\pgfqpoint{1.292991in}{1.301972in}}%
\pgfpathlineto{\pgfqpoint{1.322398in}{1.280678in}}%
\pgfpathlineto{\pgfqpoint{1.352820in}{1.261340in}}%
\pgfpathlineto{\pgfqpoint{1.386095in}{1.242889in}}%
\pgfpathlineto{\pgfqpoint{1.420190in}{1.226516in}}%
\pgfpathlineto{\pgfqpoint{1.457024in}{1.211329in}}%
\pgfpathlineto{\pgfqpoint{1.496554in}{1.197536in}}%
\pgfpathlineto{\pgfqpoint{1.538719in}{1.185287in}}%
\pgfpathlineto{\pgfqpoint{1.583441in}{1.174641in}}%
\pgfpathlineto{\pgfqpoint{1.634929in}{1.164775in}}%
\pgfpathlineto{\pgfqpoint{1.706063in}{1.153745in}}%
\pgfpathlineto{\pgfqpoint{1.768492in}{1.143417in}}%
\pgfpathlineto{\pgfqpoint{1.796122in}{1.136567in}}%
\pgfpathlineto{\pgfqpoint{1.812683in}{1.130481in}}%
\pgfpathlineto{\pgfqpoint{1.824471in}{1.124102in}}%
\pgfpathlineto{\pgfqpoint{1.833209in}{1.116741in}}%
\pgfpathlineto{\pgfqpoint{1.838498in}{1.108890in}}%
\pgfpathlineto{\pgfqpoint{1.840588in}{1.101849in}}%
\pgfpathlineto{\pgfqpoint{1.840619in}{1.094412in}}%
\pgfpathlineto{\pgfqpoint{1.837931in}{1.084986in}}%
\pgfpathlineto{\pgfqpoint{1.833246in}{1.076615in}}%
\pgfpathlineto{\pgfqpoint{1.825819in}{1.067542in}}%
\pgfpathlineto{\pgfqpoint{1.813813in}{1.056850in}}%
\pgfpathlineto{\pgfqpoint{1.798819in}{1.046763in}}%
\pgfpathlineto{\pgfqpoint{1.781016in}{1.037462in}}%
\pgfpathlineto{\pgfqpoint{1.758447in}{1.028391in}}%
\pgfpathlineto{\pgfqpoint{1.733203in}{1.020815in}}%
\pgfpathlineto{\pgfqpoint{1.705410in}{1.014872in}}%
\pgfpathlineto{\pgfqpoint{1.675178in}{1.010714in}}%
\pgfpathlineto{\pgfqpoint{1.642610in}{1.008507in}}%
\pgfpathlineto{\pgfqpoint{1.607809in}{1.008432in}}%
\pgfpathlineto{\pgfqpoint{1.570886in}{1.010691in}}%
\pgfpathlineto{\pgfqpoint{1.534118in}{1.015181in}}%
\pgfpathlineto{\pgfqpoint{1.495454in}{1.022233in}}%
\pgfpathlineto{\pgfqpoint{1.457161in}{1.031563in}}%
\pgfpathlineto{\pgfqpoint{1.419337in}{1.043132in}}%
\pgfpathlineto{\pgfqpoint{1.382089in}{1.056929in}}%
\pgfpathlineto{\pgfqpoint{1.347544in}{1.072019in}}%
\pgfpathlineto{\pgfqpoint{1.313727in}{1.089133in}}%
\pgfpathlineto{\pgfqpoint{1.280762in}{1.108299in}}%
\pgfpathlineto{\pgfqpoint{1.248782in}{1.129536in}}%
\pgfpathlineto{\pgfqpoint{1.219708in}{1.151422in}}%
\pgfpathlineto{\pgfqpoint{1.191752in}{1.175138in}}%
\pgfpathlineto{\pgfqpoint{1.165031in}{1.200649in}}%
\pgfpathlineto{\pgfqpoint{1.139653in}{1.227898in}}%
\pgfpathlineto{\pgfqpoint{1.115714in}{1.256800in}}%
\pgfpathlineto{\pgfqpoint{1.093288in}{1.287251in}}%
\pgfpathlineto{\pgfqpoint{1.071178in}{1.321163in}}%
\pgfpathlineto{\pgfqpoint{1.050868in}{1.356520in}}%
\pgfpathlineto{\pgfqpoint{1.032365in}{1.393152in}}%
\pgfpathlineto{\pgfqpoint{1.014718in}{1.433142in}}%
\pgfpathlineto{\pgfqpoint{0.999024in}{1.474185in}}%
\pgfpathlineto{\pgfqpoint{0.984506in}{1.518461in}}%
\pgfpathlineto{\pgfqpoint{0.972010in}{1.563537in}}%
\pgfpathlineto{\pgfqpoint{0.960944in}{1.611678in}}%
\pgfpathlineto{\pgfqpoint{0.951530in}{1.662824in}}%
\pgfpathlineto{\pgfqpoint{0.944286in}{1.714431in}}%
\pgfpathlineto{\pgfqpoint{0.938950in}{1.768847in}}%
\pgfpathlineto{\pgfqpoint{0.935870in}{1.823491in}}%
\pgfpathlineto{\pgfqpoint{0.935034in}{1.878240in}}%
\pgfpathlineto{\pgfqpoint{0.936466in}{1.932973in}}%
\pgfpathlineto{\pgfqpoint{0.940005in}{1.985084in}}%
\pgfpathlineto{\pgfqpoint{0.945759in}{2.036935in}}%
\pgfpathlineto{\pgfqpoint{0.953410in}{2.085938in}}%
\pgfpathlineto{\pgfqpoint{0.962764in}{2.132000in}}%
\pgfpathlineto{\pgfqpoint{0.974287in}{2.177414in}}%
\pgfpathlineto{\pgfqpoint{0.987332in}{2.219653in}}%
\pgfpathlineto{\pgfqpoint{1.001667in}{2.258654in}}%
\pgfpathlineto{\pgfqpoint{1.018051in}{2.296583in}}%
\pgfpathlineto{\pgfqpoint{1.035401in}{2.331101in}}%
\pgfpathlineto{\pgfqpoint{1.054650in}{2.364275in}}%
\pgfpathlineto{\pgfqpoint{1.074406in}{2.393984in}}%
\pgfpathlineto{\pgfqpoint{1.095771in}{2.422197in}}%
\pgfpathlineto{\pgfqpoint{1.118662in}{2.448797in}}%
\pgfpathlineto{\pgfqpoint{1.142967in}{2.473701in}}%
\pgfpathlineto{\pgfqpoint{1.168550in}{2.496867in}}%
\pgfpathlineto{\pgfqpoint{1.197085in}{2.519662in}}%
\pgfpathlineto{\pgfqpoint{1.226727in}{2.540526in}}%
\pgfpathlineto{\pgfqpoint{1.259242in}{2.560673in}}%
\pgfpathlineto{\pgfqpoint{1.294612in}{2.579881in}}%
\pgfpathlineto{\pgfqpoint{1.332792in}{2.597982in}}%
\pgfpathlineto{\pgfqpoint{1.373719in}{2.614859in}}%
\pgfpathlineto{\pgfqpoint{1.417319in}{2.630445in}}%
\pgfpathlineto{\pgfqpoint{1.465632in}{2.645312in}}%
\pgfpathlineto{\pgfqpoint{1.518640in}{2.659204in}}%
\pgfpathlineto{\pgfqpoint{1.576309in}{2.671929in}}%
\pgfpathlineto{\pgfqpoint{1.638597in}{2.683344in}}%
\pgfpathlineto{\pgfqpoint{1.705462in}{2.693343in}}%
\pgfpathlineto{\pgfqpoint{1.779027in}{2.702064in}}%
\pgfpathlineto{\pgfqpoint{1.857097in}{2.709077in}}%
\pgfpathlineto{\pgfqpoint{1.939633in}{2.714280in}}%
\pgfpathlineto{\pgfqpoint{2.026598in}{2.717513in}}%
\pgfpathlineto{\pgfqpoint{2.113605in}{2.718523in}}%
\pgfpathlineto{\pgfqpoint{2.198435in}{2.717303in}}%
\pgfpathlineto{\pgfqpoint{2.278866in}{2.713929in}}%
\pgfpathlineto{\pgfqpoint{2.352678in}{2.708598in}}%
\pgfpathlineto{\pgfqpoint{2.417657in}{2.701709in}}%
\pgfpathlineto{\pgfqpoint{2.473770in}{2.693630in}}%
\pgfpathlineto{\pgfqpoint{2.523140in}{2.684368in}}%
\pgfpathlineto{\pgfqpoint{2.565726in}{2.674202in}}%
\pgfpathlineto{\pgfqpoint{2.601510in}{2.663544in}}%
\pgfpathlineto{\pgfqpoint{2.632577in}{2.652142in}}%
\pgfpathlineto{\pgfqpoint{2.658899in}{2.640331in}}%
\pgfpathlineto{\pgfqpoint{2.682438in}{2.627436in}}%
\pgfpathlineto{\pgfqpoint{2.703062in}{2.613571in}}%
\pgfpathlineto{\pgfqpoint{2.720674in}{2.598978in}}%
\pgfpathlineto{\pgfqpoint{2.735263in}{2.584053in}}%
\pgfpathlineto{\pgfqpoint{2.748320in}{2.567377in}}%
\pgfpathlineto{\pgfqpoint{2.759553in}{2.549046in}}%
\pgfpathlineto{\pgfqpoint{2.768788in}{2.529306in}}%
\pgfpathlineto{\pgfqpoint{2.776017in}{2.508498in}}%
\pgfpathlineto{\pgfqpoint{2.781884in}{2.484540in}}%
\pgfpathlineto{\pgfqpoint{2.786102in}{2.457597in}}%
\pgfpathlineto{\pgfqpoint{2.788720in}{2.425384in}}%
\pgfpathlineto{\pgfqpoint{2.789427in}{2.388061in}}%
\pgfpathlineto{\pgfqpoint{2.787962in}{2.340801in}}%
\pgfpathlineto{\pgfqpoint{2.783672in}{2.278768in}}%
\pgfpathlineto{\pgfqpoint{2.774289in}{2.179783in}}%
\pgfpathlineto{\pgfqpoint{2.743611in}{1.868119in}}%
\pgfpathlineto{\pgfqpoint{2.730112in}{1.702060in}}%
\pgfpathlineto{\pgfqpoint{2.717287in}{1.515949in}}%
\pgfpathlineto{\pgfqpoint{2.702602in}{1.267597in}}%
\pgfpathlineto{\pgfqpoint{2.684434in}{0.964630in}}%
\pgfpathlineto{\pgfqpoint{2.675374in}{0.850600in}}%
\pgfpathlineto{\pgfqpoint{2.667030in}{0.771523in}}%
\pgfpathlineto{\pgfqpoint{2.658752in}{0.712543in}}%
\pgfpathlineto{\pgfqpoint{2.650176in}{0.666284in}}%
\pgfpathlineto{\pgfqpoint{2.640820in}{0.627931in}}%
\pgfpathlineto{\pgfqpoint{2.631145in}{0.597534in}}%
\pgfpathlineto{\pgfqpoint{2.621004in}{0.572745in}}%
\pgfpathlineto{\pgfqpoint{2.609856in}{0.551383in}}%
\pgfpathlineto{\pgfqpoint{2.598042in}{0.533534in}}%
\pgfpathlineto{\pgfqpoint{2.584496in}{0.517378in}}%
\pgfpathlineto{\pgfqpoint{2.571109in}{0.504669in}}%
\pgfpathlineto{\pgfqpoint{2.554789in}{0.492313in}}%
\pgfpathlineto{\pgfqpoint{2.537457in}{0.481914in}}%
\pgfpathlineto{\pgfqpoint{2.517374in}{0.472367in}}%
\pgfpathlineto{\pgfqpoint{2.492542in}{0.463178in}}%
\pgfpathlineto{\pgfqpoint{2.462979in}{0.454833in}}%
\pgfpathlineto{\pgfqpoint{2.428766in}{0.447542in}}%
\pgfpathlineto{\pgfqpoint{2.385671in}{0.440735in}}%
\pgfpathlineto{\pgfqpoint{2.331557in}{0.434581in}}%
\pgfpathlineto{\pgfqpoint{2.262115in}{0.429077in}}%
\pgfpathlineto{\pgfqpoint{2.170851in}{0.424236in}}%
\pgfpathlineto{\pgfqpoint{2.049086in}{0.420134in}}%
\pgfpathlineto{\pgfqpoint{1.879436in}{0.416783in}}%
\pgfpathlineto{\pgfqpoint{1.640159in}{0.414418in}}%
\pgfpathlineto{\pgfqpoint{1.322562in}{0.413569in}}%
\pgfpathlineto{\pgfqpoint{1.020194in}{0.414850in}}%
\pgfpathlineto{\pgfqpoint{0.822256in}{0.417715in}}%
\pgfpathlineto{\pgfqpoint{0.704835in}{0.421430in}}%
\pgfpathlineto{\pgfqpoint{0.630976in}{0.425829in}}%
\pgfpathlineto{\pgfqpoint{0.583316in}{0.430734in}}%
\pgfpathlineto{\pgfqpoint{0.551033in}{0.436123in}}%
\pgfpathlineto{\pgfqpoint{0.527708in}{0.442189in}}%
\pgfpathlineto{\pgfqpoint{0.511250in}{0.448625in}}%
\pgfpathlineto{\pgfqpoint{0.499549in}{0.455216in}}%
\pgfpathlineto{\pgfqpoint{0.488916in}{0.463841in}}%
\pgfpathlineto{\pgfqpoint{0.481322in}{0.472730in}}%
\pgfpathlineto{\pgfqpoint{0.474078in}{0.485127in}}%
\pgfpathlineto{\pgfqpoint{0.468753in}{0.498748in}}%
\pgfpathlineto{\pgfqpoint{0.463870in}{0.517848in}}%
\pgfpathlineto{\pgfqpoint{0.459679in}{0.544796in}}%
\pgfpathlineto{\pgfqpoint{0.456386in}{0.581938in}}%
\pgfpathlineto{\pgfqpoint{0.453731in}{0.639106in}}%
\pgfpathlineto{\pgfqpoint{0.451681in}{0.736155in}}%
\pgfpathlineto{\pgfqpoint{0.450220in}{0.927815in}}%
\pgfpathlineto{\pgfqpoint{0.449345in}{1.403252in}}%
\pgfpathlineto{\pgfqpoint{0.449543in}{2.682703in}}%
\pgfpathlineto{\pgfqpoint{0.451011in}{2.856932in}}%
\pgfpathlineto{\pgfqpoint{0.452802in}{2.879219in}}%
\pgfpathlineto{\pgfqpoint{0.455188in}{2.886108in}}%
\pgfpathlineto{\pgfqpoint{0.458626in}{2.889028in}}%
\pgfpathlineto{\pgfqpoint{0.464996in}{2.890553in}}%
\pgfpathlineto{\pgfqpoint{0.482377in}{2.891423in}}%
\pgfpathlineto{\pgfqpoint{0.565038in}{2.891729in}}%
\pgfpathlineto{\pgfqpoint{2.733842in}{2.891760in}}%
\pgfpathlineto{\pgfqpoint{4.789510in}{2.890885in}}%
\pgfpathlineto{\pgfqpoint{4.793727in}{2.889730in}}%
\pgfpathlineto{\pgfqpoint{4.795481in}{2.888307in}}%
\pgfpathlineto{\pgfqpoint{4.797106in}{2.881145in}}%
\pgfpathlineto{\pgfqpoint{4.797997in}{2.858771in}}%
\pgfpathlineto{\pgfqpoint{4.798039in}{2.856283in}}%
\pgfpathlineto{\pgfqpoint{4.798039in}{2.856283in}}%
\pgfusepath{stroke}%
\end{pgfscope}%
\begin{pgfscope}%
\pgfpathrectangle{\pgfqpoint{0.448634in}{0.402556in}}{\pgfqpoint{4.350661in}{2.489204in}} %
\pgfusepath{clip}%
\pgfsetrectcap%
\pgfsetroundjoin%
\pgfsetlinewidth{1.003750pt}%
\definecolor{currentstroke}{rgb}{1.000000,0.498039,0.054902}%
\pgfsetstrokecolor{currentstroke}%
\pgfsetdash{}{0pt}%
\pgfpathmoveto{\pgfqpoint{3.428772in}{0.402610in}}%
\pgfpathlineto{\pgfqpoint{2.806632in}{0.403760in}}%
\pgfpathlineto{\pgfqpoint{2.769692in}{0.405578in}}%
\pgfpathlineto{\pgfqpoint{2.754632in}{0.408064in}}%
\pgfpathlineto{\pgfqpoint{2.746391in}{0.411198in}}%
\pgfpathlineto{\pgfqpoint{2.740943in}{0.415265in}}%
\pgfpathlineto{\pgfqpoint{2.736784in}{0.420984in}}%
\pgfpathlineto{\pgfqpoint{2.733281in}{0.430071in}}%
\pgfpathlineto{\pgfqpoint{2.730449in}{0.444636in}}%
\pgfpathlineto{\pgfqpoint{2.728238in}{0.469392in}}%
\pgfpathlineto{\pgfqpoint{2.726470in}{0.519131in}}%
\pgfpathlineto{\pgfqpoint{2.725711in}{0.613715in}}%
\pgfpathlineto{\pgfqpoint{2.726842in}{0.768038in}}%
\pgfpathlineto{\pgfqpoint{2.730556in}{0.962148in}}%
\pgfpathlineto{\pgfqpoint{2.736611in}{1.158670in}}%
\pgfpathlineto{\pgfqpoint{2.744092in}{1.327718in}}%
\pgfpathlineto{\pgfqpoint{2.753201in}{1.484189in}}%
\pgfpathlineto{\pgfqpoint{2.763257in}{1.620609in}}%
\pgfpathlineto{\pgfqpoint{2.776118in}{1.764216in}}%
\pgfpathlineto{\pgfqpoint{2.788914in}{1.877776in}}%
\pgfpathlineto{\pgfqpoint{2.805748in}{2.005740in}}%
\pgfpathlineto{\pgfqpoint{2.821176in}{2.101198in}}%
\pgfpathlineto{\pgfqpoint{2.838359in}{2.193718in}}%
\pgfpathlineto{\pgfqpoint{2.859135in}{2.292966in}}%
\pgfpathlineto{\pgfqpoint{2.887209in}{2.425960in}}%
\pgfpathlineto{\pgfqpoint{2.896991in}{2.479559in}}%
\pgfpathlineto{\pgfqpoint{2.901543in}{2.516523in}}%
\pgfpathlineto{\pgfqpoint{2.902849in}{2.543854in}}%
\pgfpathlineto{\pgfqpoint{2.901957in}{2.566223in}}%
\pgfpathlineto{\pgfqpoint{2.899151in}{2.585863in}}%
\pgfpathlineto{\pgfqpoint{2.894794in}{2.602546in}}%
\pgfpathlineto{\pgfqpoint{2.888484in}{2.618388in}}%
\pgfpathlineto{\pgfqpoint{2.880257in}{2.633033in}}%
\pgfpathlineto{\pgfqpoint{2.870348in}{2.646246in}}%
\pgfpathlineto{\pgfqpoint{2.857400in}{2.659530in}}%
\pgfpathlineto{\pgfqpoint{2.843189in}{2.671010in}}%
\pgfpathlineto{\pgfqpoint{2.824237in}{2.683209in}}%
\pgfpathlineto{\pgfqpoint{2.802413in}{2.694418in}}%
\pgfpathlineto{\pgfqpoint{2.775809in}{2.705369in}}%
\pgfpathlineto{\pgfqpoint{2.744461in}{2.715715in}}%
\pgfpathlineto{\pgfqpoint{2.708436in}{2.725252in}}%
\pgfpathlineto{\pgfqpoint{2.665655in}{2.734289in}}%
\pgfpathlineto{\pgfqpoint{2.613991in}{2.742869in}}%
\pgfpathlineto{\pgfqpoint{2.553459in}{2.750589in}}%
\pgfpathlineto{\pgfqpoint{2.481920in}{2.757365in}}%
\pgfpathlineto{\pgfqpoint{2.399398in}{2.762839in}}%
\pgfpathlineto{\pgfqpoint{2.310269in}{2.766482in}}%
\pgfpathlineto{\pgfqpoint{2.175416in}{2.768725in}}%
\pgfpathlineto{\pgfqpoint{2.066653in}{2.767942in}}%
\pgfpathlineto{\pgfqpoint{1.953570in}{2.764859in}}%
\pgfpathlineto{\pgfqpoint{1.851429in}{2.759759in}}%
\pgfpathlineto{\pgfqpoint{1.745051in}{2.752169in}}%
\pgfpathlineto{\pgfqpoint{1.658373in}{2.743453in}}%
\pgfpathlineto{\pgfqpoint{1.580552in}{2.733461in}}%
\pgfpathlineto{\pgfqpoint{1.490057in}{2.719338in}}%
\pgfpathlineto{\pgfqpoint{1.417231in}{2.704698in}}%
\pgfpathlineto{\pgfqpoint{1.361992in}{2.690818in}}%
\pgfpathlineto{\pgfqpoint{1.311460in}{2.675819in}}%
\pgfpathlineto{\pgfqpoint{1.265667in}{2.659924in}}%
\pgfpathlineto{\pgfqpoint{1.222575in}{2.642586in}}%
\pgfpathlineto{\pgfqpoint{1.184324in}{2.624682in}}%
\pgfpathlineto{\pgfqpoint{1.148892in}{2.605623in}}%
\pgfpathlineto{\pgfqpoint{1.116331in}{2.585573in}}%
\pgfpathlineto{\pgfqpoint{1.092327in}{2.568512in}}%
\pgfpathlineto{\pgfqpoint{1.079760in}{2.558686in}}%
\pgfpathlineto{\pgfqpoint{1.051544in}{2.535379in}}%
\pgfpathlineto{\pgfqpoint{1.026312in}{2.511712in}}%
\pgfpathlineto{\pgfqpoint{1.002399in}{2.486318in}}%
\pgfpathlineto{\pgfqpoint{0.979913in}{2.459269in}}%
\pgfpathlineto{\pgfqpoint{0.958934in}{2.430678in}}%
\pgfpathlineto{\pgfqpoint{0.938264in}{2.398643in}}%
\pgfpathlineto{\pgfqpoint{0.923047in}{2.371385in}}%
\pgfpathlineto{\pgfqpoint{0.904513in}{2.334774in}}%
\pgfpathlineto{\pgfqpoint{0.887854in}{2.297001in}}%
\pgfpathlineto{\pgfqpoint{0.872131in}{2.255971in}}%
\pgfpathlineto{\pgfqpoint{0.857508in}{2.211741in}}%
\pgfpathlineto{\pgfqpoint{0.844762in}{2.166757in}}%
\pgfpathlineto{\pgfqpoint{0.838624in}{2.140306in}}%
\pgfpathlineto{\pgfqpoint{0.826982in}{2.087194in}}%
\pgfpathlineto{\pgfqpoint{0.816322in}{2.028715in}}%
\pgfpathlineto{\pgfqpoint{0.810087in}{1.984495in}}%
\pgfpathlineto{\pgfqpoint{0.808026in}{1.967238in}}%
\pgfpathlineto{\pgfqpoint{0.800076in}{1.898140in}}%
\pgfpathlineto{\pgfqpoint{0.793713in}{1.823823in}}%
\pgfpathlineto{\pgfqpoint{0.788799in}{1.741875in}}%
\pgfpathlineto{\pgfqpoint{0.786199in}{1.677225in}}%
\pgfpathlineto{\pgfqpoint{0.776951in}{1.453481in}}%
\pgfpathlineto{\pgfqpoint{0.773280in}{1.418894in}}%
\pgfpathlineto{\pgfqpoint{0.768298in}{1.389582in}}%
\pgfpathlineto{\pgfqpoint{0.762752in}{1.368108in}}%
\pgfpathlineto{\pgfqpoint{0.756722in}{1.352123in}}%
\pgfpathlineto{\pgfqpoint{0.749752in}{1.339519in}}%
\pgfpathlineto{\pgfqpoint{0.742201in}{1.330599in}}%
\pgfpathlineto{\pgfqpoint{0.734854in}{1.325312in}}%
\pgfpathlineto{\pgfqpoint{0.726558in}{1.322419in}}%
\pgfpathlineto{\pgfqpoint{0.717884in}{1.322223in}}%
\pgfpathlineto{\pgfqpoint{0.709412in}{1.324411in}}%
\pgfpathlineto{\pgfqpoint{0.699548in}{1.329604in}}%
\pgfpathlineto{\pgfqpoint{0.688894in}{1.338203in}}%
\pgfpathlineto{\pgfqpoint{0.677907in}{1.350248in}}%
\pgfpathlineto{\pgfqpoint{0.666886in}{1.365647in}}%
\pgfpathlineto{\pgfqpoint{0.654913in}{1.386417in}}%
\pgfpathlineto{\pgfqpoint{0.642574in}{1.412730in}}%
\pgfpathlineto{\pgfqpoint{0.630328in}{1.444629in}}%
\pgfpathlineto{\pgfqpoint{0.618504in}{1.482081in}}%
\pgfpathlineto{\pgfqpoint{0.608613in}{1.520256in}}%
\pgfpathlineto{\pgfqpoint{0.590203in}{1.612445in}}%
\pgfpathlineto{\pgfqpoint{0.581848in}{1.668884in}}%
\pgfpathlineto{\pgfqpoint{0.573137in}{1.740376in}}%
\pgfpathlineto{\pgfqpoint{0.567062in}{1.807213in}}%
\pgfpathlineto{\pgfqpoint{0.560532in}{1.896510in}}%
\pgfpathlineto{\pgfqpoint{0.555526in}{1.995910in}}%
\pgfpathlineto{\pgfqpoint{0.552564in}{2.097908in}}%
\pgfpathlineto{\pgfqpoint{0.551526in}{2.204935in}}%
\pgfpathlineto{\pgfqpoint{0.552728in}{2.309470in}}%
\pgfpathlineto{\pgfqpoint{0.556011in}{2.403981in}}%
\pgfpathlineto{\pgfqpoint{0.560953in}{2.483430in}}%
\pgfpathlineto{\pgfqpoint{0.567303in}{2.550240in}}%
\pgfpathlineto{\pgfqpoint{0.574928in}{2.606817in}}%
\pgfpathlineto{\pgfqpoint{0.582988in}{2.650657in}}%
\pgfpathlineto{\pgfqpoint{0.592756in}{2.691452in}}%
\pgfpathlineto{\pgfqpoint{0.602650in}{2.721756in}}%
\pgfpathlineto{\pgfqpoint{0.612983in}{2.746441in}}%
\pgfpathlineto{\pgfqpoint{0.624292in}{2.767692in}}%
\pgfpathlineto{\pgfqpoint{0.636231in}{2.785433in}}%
\pgfpathlineto{\pgfqpoint{0.649892in}{2.801461in}}%
\pgfpathlineto{\pgfqpoint{0.663386in}{2.814020in}}%
\pgfpathlineto{\pgfqpoint{0.679842in}{2.826135in}}%
\pgfpathlineto{\pgfqpoint{0.697326in}{2.836197in}}%
\pgfpathlineto{\pgfqpoint{0.715574in}{2.844285in}}%
\pgfpathlineto{\pgfqpoint{0.738439in}{2.852335in}}%
\pgfpathlineto{\pgfqpoint{0.765983in}{2.859639in}}%
\pgfpathlineto{\pgfqpoint{0.800300in}{2.866256in}}%
\pgfpathlineto{\pgfqpoint{0.841340in}{2.871832in}}%
\pgfpathlineto{\pgfqpoint{0.895547in}{2.876803in}}%
\pgfpathlineto{\pgfqpoint{0.969413in}{2.881069in}}%
\pgfpathlineto{\pgfqpoint{1.071608in}{2.884501in}}%
\pgfpathlineto{\pgfqpoint{1.219512in}{2.887074in}}%
\pgfpathlineto{\pgfqpoint{1.471844in}{2.889091in}}%
\pgfpathlineto{\pgfqpoint{1.956941in}{2.890384in}}%
\pgfpathlineto{\pgfqpoint{3.096814in}{2.890781in}}%
\pgfpathlineto{\pgfqpoint{3.995224in}{2.889388in}}%
\pgfpathlineto{\pgfqpoint{4.275833in}{2.887011in}}%
\pgfpathlineto{\pgfqpoint{4.412847in}{2.883743in}}%
\pgfpathlineto{\pgfqpoint{4.491081in}{2.879810in}}%
\pgfpathlineto{\pgfqpoint{4.543127in}{2.875163in}}%
\pgfpathlineto{\pgfqpoint{4.579810in}{2.869841in}}%
\pgfpathlineto{\pgfqpoint{4.607580in}{2.863763in}}%
\pgfpathlineto{\pgfqpoint{4.630623in}{2.856424in}}%
\pgfpathlineto{\pgfqpoint{4.648833in}{2.848228in}}%
\pgfpathlineto{\pgfqpoint{4.664136in}{2.838773in}}%
\pgfpathlineto{\pgfqpoint{4.676470in}{2.828576in}}%
\pgfpathlineto{\pgfqpoint{4.687502in}{2.816585in}}%
\pgfpathlineto{\pgfqpoint{4.697051in}{2.803027in}}%
\pgfpathlineto{\pgfqpoint{4.706194in}{2.786098in}}%
\pgfpathlineto{\pgfqpoint{4.714508in}{2.765827in}}%
\pgfpathlineto{\pgfqpoint{4.722462in}{2.740013in}}%
\pgfpathlineto{\pgfqpoint{4.729577in}{2.708703in}}%
\pgfpathlineto{\pgfqpoint{4.736162in}{2.669601in}}%
\pgfpathlineto{\pgfqpoint{4.742419in}{2.617826in}}%
\pgfpathlineto{\pgfqpoint{4.747859in}{2.553410in}}%
\pgfpathlineto{\pgfqpoint{4.752661in}{2.468958in}}%
\pgfpathlineto{\pgfqpoint{4.756610in}{2.359528in}}%
\pgfpathlineto{\pgfqpoint{4.759416in}{2.217681in}}%
\pgfpathlineto{\pgfqpoint{4.760596in}{2.043444in}}%
\pgfpathlineto{\pgfqpoint{4.759662in}{1.851779in}}%
\pgfpathlineto{\pgfqpoint{4.756587in}{1.667613in}}%
\pgfpathlineto{\pgfqpoint{4.751596in}{1.503428in}}%
\pgfpathlineto{\pgfqpoint{4.745410in}{1.374185in}}%
\pgfpathlineto{\pgfqpoint{4.738113in}{1.267479in}}%
\pgfpathlineto{\pgfqpoint{4.729621in}{1.175896in}}%
\pgfpathlineto{\pgfqpoint{4.720762in}{1.104428in}}%
\pgfpathlineto{\pgfqpoint{4.711045in}{1.043204in}}%
\pgfpathlineto{\pgfqpoint{4.700364in}{0.989829in}}%
\pgfpathlineto{\pgfqpoint{4.689055in}{0.944345in}}%
\pgfpathlineto{\pgfqpoint{4.676881in}{0.904394in}}%
\pgfpathlineto{\pgfqpoint{4.676095in}{0.902073in}}%
\pgfpathlineto{\pgfqpoint{4.676095in}{0.902073in}}%
\pgfusepath{stroke}%
\end{pgfscope}%
\begin{pgfscope}%
\pgfpathrectangle{\pgfqpoint{0.448634in}{0.402556in}}{\pgfqpoint{4.350661in}{2.489204in}} %
\pgfusepath{clip}%
\pgfsetrectcap%
\pgfsetroundjoin%
\pgfsetlinewidth{1.003750pt}%
\definecolor{currentstroke}{rgb}{1.000000,0.498039,0.054902}%
\pgfsetstrokecolor{currentstroke}%
\pgfsetdash{}{0pt}%
\pgfpathmoveto{\pgfqpoint{2.795520in}{1.982745in}}%
\pgfpathlineto{\pgfqpoint{2.781780in}{1.874357in}}%
\pgfpathlineto{\pgfqpoint{2.769351in}{1.758234in}}%
\pgfpathlineto{\pgfqpoint{2.758095in}{1.631942in}}%
\pgfpathlineto{\pgfqpoint{2.747786in}{1.490551in}}%
\pgfpathlineto{\pgfqpoint{2.738644in}{1.334082in}}%
\pgfpathlineto{\pgfqpoint{2.730580in}{1.157591in}}%
\pgfpathlineto{\pgfqpoint{2.723334in}{0.948663in}}%
\pgfpathlineto{\pgfqpoint{2.709783in}{0.530788in}}%
\pgfpathlineto{\pgfqpoint{2.705868in}{0.488716in}}%
\pgfpathlineto{\pgfqpoint{2.701769in}{0.464281in}}%
\pgfpathlineto{\pgfqpoint{2.697021in}{0.447744in}}%
\pgfpathlineto{\pgfqpoint{2.691859in}{0.436812in}}%
\pgfpathlineto{\pgfqpoint{2.686245in}{0.429229in}}%
\pgfpathlineto{\pgfqpoint{2.679348in}{0.423188in}}%
\pgfpathlineto{\pgfqpoint{2.669540in}{0.417856in}}%
\pgfpathlineto{\pgfqpoint{2.656987in}{0.413810in}}%
\pgfpathlineto{\pgfqpoint{2.637654in}{0.410337in}}%
\pgfpathlineto{\pgfqpoint{2.607297in}{0.407617in}}%
\pgfpathlineto{\pgfqpoint{2.555121in}{0.405574in}}%
\pgfpathlineto{\pgfqpoint{2.450714in}{0.404139in}}%
\pgfpathlineto{\pgfqpoint{2.176624in}{0.403275in}}%
\pgfpathlineto{\pgfqpoint{1.130290in}{0.402953in}}%
\pgfpathlineto{\pgfqpoint{0.516849in}{0.404175in}}%
\pgfpathlineto{\pgfqpoint{0.466848in}{0.405970in}}%
\pgfpathlineto{\pgfqpoint{0.456130in}{0.407931in}}%
\pgfpathlineto{\pgfqpoint{0.452340in}{0.410303in}}%
\pgfpathlineto{\pgfqpoint{0.450346in}{0.414662in}}%
\pgfpathlineto{\pgfqpoint{0.449266in}{0.424524in}}%
\pgfpathlineto{\pgfqpoint{0.448771in}{0.464344in}}%
\pgfpathlineto{\pgfqpoint{0.448640in}{0.850171in}}%
\pgfpathlineto{\pgfqpoint{0.448679in}{2.891318in}}%
\pgfpathlineto{\pgfqpoint{0.448679in}{2.891318in}}%
\pgfusepath{stroke}%
\end{pgfscope}%
\begin{pgfscope}%
\pgfpathrectangle{\pgfqpoint{0.448634in}{0.402556in}}{\pgfqpoint{4.350661in}{2.489204in}} %
\pgfusepath{clip}%
\pgfsetrectcap%
\pgfsetroundjoin%
\pgfsetlinewidth{1.003750pt}%
\definecolor{currentstroke}{rgb}{1.000000,0.498039,0.054902}%
\pgfsetstrokecolor{currentstroke}%
\pgfsetdash{}{0pt}%
\pgfpathmoveto{\pgfqpoint{3.428189in}{0.402586in}}%
\pgfpathlineto{\pgfqpoint{2.782121in}{0.403701in}}%
\pgfpathlineto{\pgfqpoint{2.753906in}{0.405674in}}%
\pgfpathlineto{\pgfqpoint{2.743328in}{0.408443in}}%
\pgfpathlineto{\pgfqpoint{2.737717in}{0.412188in}}%
\pgfpathlineto{\pgfqpoint{2.733668in}{0.417995in}}%
\pgfpathlineto{\pgfqpoint{2.730649in}{0.427307in}}%
\pgfpathlineto{\pgfqpoint{2.728388in}{0.442004in}}%
\pgfpathlineto{\pgfqpoint{2.726544in}{0.471794in}}%
\pgfpathlineto{\pgfqpoint{2.725216in}{0.534003in}}%
\pgfpathlineto{\pgfqpoint{2.725169in}{0.655973in}}%
\pgfpathlineto{\pgfqpoint{2.727377in}{0.832687in}}%
\pgfpathlineto{\pgfqpoint{2.732259in}{1.041703in}}%
\pgfpathlineto{\pgfqpoint{2.738851in}{1.223257in}}%
\pgfpathlineto{\pgfqpoint{2.747078in}{1.389766in}}%
\pgfpathlineto{\pgfqpoint{2.756608in}{1.538717in}}%
\pgfpathlineto{\pgfqpoint{2.768955in}{1.694887in}}%
\pgfpathlineto{\pgfqpoint{2.781228in}{1.816044in}}%
\pgfpathlineto{\pgfqpoint{2.794401in}{1.924524in}}%
\pgfpathlineto{\pgfqpoint{2.812737in}{2.054722in}}%
\pgfpathlineto{\pgfqpoint{2.828774in}{2.147512in}}%
\pgfpathlineto{\pgfqpoint{2.847382in}{2.242224in}}%
\pgfpathlineto{\pgfqpoint{2.895818in}{2.479699in}}%
\pgfpathlineto{\pgfqpoint{2.900204in}{2.516689in}}%
\pgfpathlineto{\pgfqpoint{2.901346in}{2.544029in}}%
\pgfpathlineto{\pgfqpoint{2.900291in}{2.566388in}}%
\pgfpathlineto{\pgfqpoint{2.897334in}{2.585999in}}%
\pgfpathlineto{\pgfqpoint{2.892836in}{2.602633in}}%
\pgfpathlineto{\pgfqpoint{2.886394in}{2.618405in}}%
\pgfpathlineto{\pgfqpoint{2.878058in}{2.632969in}}%
\pgfpathlineto{\pgfqpoint{2.868065in}{2.646100in}}%
\pgfpathlineto{\pgfqpoint{2.855050in}{2.659300in}}%
\pgfpathlineto{\pgfqpoint{2.840801in}{2.670717in}}%
\pgfpathlineto{\pgfqpoint{2.821822in}{2.682861in}}%
\pgfpathlineto{\pgfqpoint{2.799980in}{2.694026in}}%
\pgfpathlineto{\pgfqpoint{2.773366in}{2.704944in}}%
\pgfpathlineto{\pgfqpoint{2.742012in}{2.715266in}}%
\pgfpathlineto{\pgfqpoint{2.705983in}{2.724785in}}%
\pgfpathlineto{\pgfqpoint{2.663200in}{2.733810in}}%
\pgfpathlineto{\pgfqpoint{2.611535in}{2.742379in}}%
\pgfpathlineto{\pgfqpoint{2.551002in}{2.750090in}}%
\pgfpathlineto{\pgfqpoint{2.481632in}{2.756682in}}%
\pgfpathlineto{\pgfqpoint{2.399112in}{2.762200in}}%
\pgfpathlineto{\pgfqpoint{2.309985in}{2.765886in}}%
\pgfpathlineto{\pgfqpoint{2.188184in}{2.768096in}}%
\pgfpathlineto{\pgfqpoint{2.081595in}{2.767619in}}%
\pgfpathlineto{\pgfqpoint{1.968506in}{2.764840in}}%
\pgfpathlineto{\pgfqpoint{1.864180in}{2.759918in}}%
\pgfpathlineto{\pgfqpoint{1.757786in}{2.752593in}}%
\pgfpathlineto{\pgfqpoint{1.671087in}{2.744171in}}%
\pgfpathlineto{\pgfqpoint{1.591076in}{2.734193in}}%
\pgfpathlineto{\pgfqpoint{1.502689in}{2.720717in}}%
\pgfpathlineto{\pgfqpoint{1.427655in}{2.706083in}}%
\pgfpathlineto{\pgfqpoint{1.372350in}{2.692544in}}%
\pgfpathlineto{\pgfqpoint{1.321734in}{2.677921in}}%
\pgfpathlineto{\pgfqpoint{1.273765in}{2.661664in}}%
\pgfpathlineto{\pgfqpoint{1.230567in}{2.644672in}}%
\pgfpathlineto{\pgfqpoint{1.192197in}{2.627106in}}%
\pgfpathlineto{\pgfqpoint{1.156620in}{2.608403in}}%
\pgfpathlineto{\pgfqpoint{1.123890in}{2.588716in}}%
\pgfpathlineto{\pgfqpoint{1.095883in}{2.569568in}}%
\pgfpathlineto{\pgfqpoint{1.063936in}{2.543701in}}%
\pgfpathlineto{\pgfqpoint{1.038217in}{2.520732in}}%
\pgfpathlineto{\pgfqpoint{1.013766in}{2.496016in}}%
\pgfpathlineto{\pgfqpoint{0.990704in}{2.469610in}}%
\pgfpathlineto{\pgfqpoint{0.969124in}{2.441612in}}%
\pgfpathlineto{\pgfqpoint{0.949083in}{2.412154in}}%
\pgfpathlineto{\pgfqpoint{0.930604in}{2.381387in}}%
\pgfpathlineto{\pgfqpoint{0.906555in}{2.334052in}}%
\pgfpathlineto{\pgfqpoint{0.889925in}{2.296262in}}%
\pgfpathlineto{\pgfqpoint{0.874241in}{2.255213in}}%
\pgfpathlineto{\pgfqpoint{0.859667in}{2.210961in}}%
\pgfpathlineto{\pgfqpoint{0.846986in}{2.165954in}}%
\pgfpathlineto{\pgfqpoint{0.839633in}{2.134715in}}%
\pgfpathlineto{\pgfqpoint{0.828238in}{2.081532in}}%
\pgfpathlineto{\pgfqpoint{0.817866in}{2.022986in}}%
\pgfpathlineto{\pgfqpoint{0.810784in}{1.971352in}}%
\pgfpathlineto{\pgfqpoint{0.802846in}{1.902252in}}%
\pgfpathlineto{\pgfqpoint{0.796554in}{1.827927in}}%
\pgfpathlineto{\pgfqpoint{0.791696in}{1.743480in}}%
\pgfpathlineto{\pgfqpoint{0.787773in}{1.621595in}}%
\pgfpathlineto{\pgfqpoint{0.785408in}{1.522064in}}%
\pgfpathlineto{\pgfqpoint{0.785408in}{1.522064in}}%
\pgfusepath{stroke}%
\end{pgfscope}%
\begin{pgfscope}%
\pgfpathrectangle{\pgfqpoint{0.448634in}{0.402556in}}{\pgfqpoint{4.350661in}{2.489204in}} %
\pgfusepath{clip}%
\pgfsetrectcap%
\pgfsetroundjoin%
\pgfsetlinewidth{1.003750pt}%
\definecolor{currentstroke}{rgb}{0.172549,0.627451,0.172549}%
\pgfsetstrokecolor{currentstroke}%
\pgfsetdash{}{0pt}%
\pgfpathmoveto{\pgfqpoint{0.448634in}{2.896245in}}%
\pgfpathlineto{\pgfqpoint{0.448593in}{0.407043in}}%
\pgfpathlineto{\pgfqpoint{0.448593in}{0.407043in}}%
\pgfusepath{stroke}%
\end{pgfscope}%
\begin{pgfscope}%
\pgfpathrectangle{\pgfqpoint{0.448634in}{0.402556in}}{\pgfqpoint{4.350661in}{2.489204in}} %
\pgfusepath{clip}%
\pgfsetrectcap%
\pgfsetroundjoin%
\pgfsetlinewidth{1.003750pt}%
\definecolor{currentstroke}{rgb}{0.172549,0.627451,0.172549}%
\pgfsetstrokecolor{currentstroke}%
\pgfsetdash{}{0pt}%
\pgfpathmoveto{\pgfqpoint{0.576853in}{1.760817in}}%
\pgfpathlineto{\pgfqpoint{0.569394in}{1.840010in}}%
\pgfpathlineto{\pgfqpoint{0.563209in}{1.929338in}}%
\pgfpathlineto{\pgfqpoint{0.558592in}{2.028764in}}%
\pgfpathlineto{\pgfqpoint{0.555985in}{2.133265in}}%
\pgfpathlineto{\pgfqpoint{0.555566in}{2.237808in}}%
\pgfpathlineto{\pgfqpoint{0.557371in}{2.337352in}}%
\pgfpathlineto{\pgfqpoint{0.561096in}{2.424366in}}%
\pgfpathlineto{\pgfqpoint{0.566403in}{2.498791in}}%
\pgfpathlineto{\pgfqpoint{0.572909in}{2.560570in}}%
\pgfpathlineto{\pgfqpoint{0.580458in}{2.612119in}}%
\pgfpathlineto{\pgfqpoint{0.589086in}{2.655816in}}%
\pgfpathlineto{\pgfqpoint{0.598406in}{2.691589in}}%
\pgfpathlineto{\pgfqpoint{0.608613in}{2.721757in}}%
\pgfpathlineto{\pgfqpoint{0.619241in}{2.746278in}}%
\pgfpathlineto{\pgfqpoint{0.630817in}{2.767339in}}%
\pgfpathlineto{\pgfqpoint{0.642975in}{2.784884in}}%
\pgfpathlineto{\pgfqpoint{0.656813in}{2.800712in}}%
\pgfpathlineto{\pgfqpoint{0.672197in}{2.814549in}}%
\pgfpathlineto{\pgfqpoint{0.688853in}{2.826301in}}%
\pgfpathlineto{\pgfqpoint{0.706461in}{2.836076in}}%
\pgfpathlineto{\pgfqpoint{0.726804in}{2.844875in}}%
\pgfpathlineto{\pgfqpoint{0.751866in}{2.853203in}}%
\pgfpathlineto{\pgfqpoint{0.781631in}{2.860547in}}%
\pgfpathlineto{\pgfqpoint{0.818168in}{2.867054in}}%
\pgfpathlineto{\pgfqpoint{0.863581in}{2.872685in}}%
\pgfpathlineto{\pgfqpoint{0.922161in}{2.877518in}}%
\pgfpathlineto{\pgfqpoint{1.000391in}{2.881567in}}%
\pgfpathlineto{\pgfqpoint{1.111294in}{2.884881in}}%
\pgfpathlineto{\pgfqpoint{1.274428in}{2.887367in}}%
\pgfpathlineto{\pgfqpoint{1.552865in}{2.889263in}}%
\pgfpathlineto{\pgfqpoint{2.107573in}{2.890457in}}%
\pgfpathlineto{\pgfqpoint{3.343161in}{2.890573in}}%
\pgfpathlineto{\pgfqpoint{4.043615in}{2.888941in}}%
\pgfpathlineto{\pgfqpoint{4.289417in}{2.886404in}}%
\pgfpathlineto{\pgfqpoint{4.413375in}{2.883093in}}%
\pgfpathlineto{\pgfqpoint{4.489424in}{2.878997in}}%
\pgfpathlineto{\pgfqpoint{4.541451in}{2.874081in}}%
\pgfpathlineto{\pgfqpoint{4.578100in}{2.868470in}}%
\pgfpathlineto{\pgfqpoint{4.605818in}{2.862092in}}%
\pgfpathlineto{\pgfqpoint{4.626725in}{2.855245in}}%
\pgfpathlineto{\pgfqpoint{4.644925in}{2.847018in}}%
\pgfpathlineto{\pgfqpoint{4.660241in}{2.837590in}}%
\pgfpathlineto{\pgfqpoint{4.672623in}{2.827468in}}%
\pgfpathlineto{\pgfqpoint{4.683751in}{2.815592in}}%
\pgfpathlineto{\pgfqpoint{4.693406in}{2.802135in}}%
\pgfpathlineto{\pgfqpoint{4.702740in}{2.785343in}}%
\pgfpathlineto{\pgfqpoint{4.711277in}{2.765194in}}%
\pgfpathlineto{\pgfqpoint{4.719482in}{2.739484in}}%
\pgfpathlineto{\pgfqpoint{4.726293in}{2.710657in}}%
\pgfpathlineto{\pgfqpoint{4.733259in}{2.671643in}}%
\pgfpathlineto{\pgfqpoint{4.739604in}{2.622396in}}%
\pgfpathlineto{\pgfqpoint{4.745236in}{2.560504in}}%
\pgfpathlineto{\pgfqpoint{4.750164in}{2.481052in}}%
\pgfpathlineto{\pgfqpoint{4.754367in}{2.376618in}}%
\pgfpathlineto{\pgfqpoint{4.757443in}{2.242249in}}%
\pgfpathlineto{\pgfqpoint{4.758977in}{2.075483in}}%
\pgfpathlineto{\pgfqpoint{4.758447in}{1.888795in}}%
\pgfpathlineto{\pgfqpoint{4.755756in}{1.707111in}}%
\pgfpathlineto{\pgfqpoint{4.750925in}{1.532957in}}%
\pgfpathlineto{\pgfqpoint{4.744785in}{1.398726in}}%
\pgfpathlineto{\pgfqpoint{4.737575in}{1.289516in}}%
\pgfpathlineto{\pgfqpoint{4.728714in}{1.190470in}}%
\pgfpathlineto{\pgfqpoint{4.719652in}{1.116521in}}%
\pgfpathlineto{\pgfqpoint{4.710036in}{1.055276in}}%
\pgfpathlineto{\pgfqpoint{4.699503in}{1.001861in}}%
\pgfpathlineto{\pgfqpoint{4.689040in}{0.958690in}}%
\pgfpathlineto{\pgfqpoint{4.677219in}{0.918600in}}%
\pgfpathlineto{\pgfqpoint{4.664034in}{0.881749in}}%
\pgfpathlineto{\pgfqpoint{4.650584in}{0.850492in}}%
\pgfpathlineto{\pgfqpoint{4.636303in}{0.822570in}}%
\pgfpathlineto{\pgfqpoint{4.620207in}{0.795974in}}%
\pgfpathlineto{\pgfqpoint{4.603640in}{0.772901in}}%
\pgfpathlineto{\pgfqpoint{4.585488in}{0.751446in}}%
\pgfpathlineto{\pgfqpoint{4.565874in}{0.731749in}}%
\pgfpathlineto{\pgfqpoint{4.544964in}{0.713879in}}%
\pgfpathlineto{\pgfqpoint{4.522958in}{0.697824in}}%
\pgfpathlineto{\pgfqpoint{4.496157in}{0.681290in}}%
\pgfpathlineto{\pgfqpoint{4.470397in}{0.667953in}}%
\pgfpathlineto{\pgfqpoint{4.439961in}{0.654509in}}%
\pgfpathlineto{\pgfqpoint{4.406841in}{0.642281in}}%
\pgfpathlineto{\pgfqpoint{4.369009in}{0.630748in}}%
\pgfpathlineto{\pgfqpoint{4.326489in}{0.620226in}}%
\pgfpathlineto{\pgfqpoint{4.279327in}{0.610949in}}%
\pgfpathlineto{\pgfqpoint{4.227576in}{0.603085in}}%
\pgfpathlineto{\pgfqpoint{4.173450in}{0.597063in}}%
\pgfpathlineto{\pgfqpoint{4.110511in}{0.592203in}}%
\pgfpathlineto{\pgfqpoint{4.047471in}{0.589537in}}%
\pgfpathlineto{\pgfqpoint{3.977867in}{0.588624in}}%
\pgfpathlineto{\pgfqpoint{3.906093in}{0.589934in}}%
\pgfpathlineto{\pgfqpoint{3.834377in}{0.593496in}}%
\pgfpathlineto{\pgfqpoint{3.767120in}{0.599067in}}%
\pgfpathlineto{\pgfqpoint{3.704364in}{0.606392in}}%
\pgfpathlineto{\pgfqpoint{3.678516in}{0.610510in}}%
\pgfpathlineto{\pgfqpoint{3.620438in}{0.620500in}}%
\pgfpathlineto{\pgfqpoint{3.586319in}{0.628207in}}%
\pgfpathlineto{\pgfqpoint{3.495240in}{0.652428in}}%
\pgfpathlineto{\pgfqpoint{3.451528in}{0.667583in}}%
\pgfpathlineto{\pgfqpoint{3.408538in}{0.685220in}}%
\pgfpathlineto{\pgfqpoint{3.374594in}{0.702001in}}%
\pgfpathlineto{\pgfqpoint{3.345407in}{0.718682in}}%
\pgfpathlineto{\pgfqpoint{3.315236in}{0.738520in}}%
\pgfpathlineto{\pgfqpoint{3.288127in}{0.759290in}}%
\pgfpathlineto{\pgfqpoint{3.264004in}{0.780551in}}%
\pgfpathlineto{\pgfqpoint{3.241208in}{0.803648in}}%
\pgfpathlineto{\pgfqpoint{3.219894in}{0.828530in}}%
\pgfpathlineto{\pgfqpoint{3.200189in}{0.855091in}}%
\pgfpathlineto{\pgfqpoint{3.182177in}{0.883182in}}%
\pgfpathlineto{\pgfqpoint{3.165906in}{0.912633in}}%
\pgfpathlineto{\pgfqpoint{3.150351in}{0.945448in}}%
\pgfpathlineto{\pgfqpoint{3.136682in}{0.979345in}}%
\pgfpathlineto{\pgfqpoint{3.124073in}{1.016460in}}%
\pgfpathlineto{\pgfqpoint{3.112834in}{1.056769in}}%
\pgfpathlineto{\pgfqpoint{3.103046in}{1.100146in}}%
\pgfpathlineto{\pgfqpoint{3.095343in}{1.144071in}}%
\pgfpathlineto{\pgfqpoint{3.089208in}{1.190837in}}%
\pgfpathlineto{\pgfqpoint{3.084595in}{1.242838in}}%
\pgfpathlineto{\pgfqpoint{3.082137in}{1.295031in}}%
\pgfpathlineto{\pgfqpoint{3.081687in}{1.349787in}}%
\pgfpathlineto{\pgfqpoint{3.083451in}{1.406998in}}%
\pgfpathlineto{\pgfqpoint{3.087181in}{1.461589in}}%
\pgfpathlineto{\pgfqpoint{3.093485in}{1.520888in}}%
\pgfpathlineto{\pgfqpoint{3.101823in}{1.577334in}}%
\pgfpathlineto{\pgfqpoint{3.111930in}{1.630856in}}%
\pgfpathlineto{\pgfqpoint{3.124690in}{1.686208in}}%
\pgfpathlineto{\pgfqpoint{3.139178in}{1.738395in}}%
\pgfpathlineto{\pgfqpoint{3.155145in}{1.787366in}}%
\pgfpathlineto{\pgfqpoint{3.172353in}{1.833085in}}%
\pgfpathlineto{\pgfqpoint{3.191618in}{1.877716in}}%
\pgfpathlineto{\pgfqpoint{3.214026in}{1.923261in}}%
\pgfpathlineto{\pgfqpoint{3.236214in}{1.963157in}}%
\pgfpathlineto{\pgfqpoint{3.260178in}{2.001684in}}%
\pgfpathlineto{\pgfqpoint{3.285814in}{2.038776in}}%
\pgfpathlineto{\pgfqpoint{3.314415in}{2.076285in}}%
\pgfpathlineto{\pgfqpoint{3.348944in}{2.117711in}}%
\pgfpathlineto{\pgfqpoint{3.417133in}{2.198022in}}%
\pgfpathlineto{\pgfqpoint{3.426053in}{2.212128in}}%
\pgfpathlineto{\pgfqpoint{3.430798in}{2.223297in}}%
\pgfpathlineto{\pgfqpoint{3.432034in}{2.230603in}}%
\pgfpathlineto{\pgfqpoint{3.430773in}{2.237856in}}%
\pgfpathlineto{\pgfqpoint{3.426621in}{2.243526in}}%
\pgfpathlineto{\pgfqpoint{3.420908in}{2.247084in}}%
\pgfpathlineto{\pgfqpoint{3.412501in}{2.249583in}}%
\pgfpathlineto{\pgfqpoint{3.399499in}{2.250689in}}%
\pgfpathlineto{\pgfqpoint{3.384305in}{2.249671in}}%
\pgfpathlineto{\pgfqpoint{3.364985in}{2.246098in}}%
\pgfpathlineto{\pgfqpoint{3.341804in}{2.239342in}}%
\pgfpathlineto{\pgfqpoint{3.317109in}{2.229682in}}%
\pgfpathlineto{\pgfqpoint{3.291104in}{2.216986in}}%
\pgfpathlineto{\pgfqpoint{3.265928in}{2.202261in}}%
\pgfpathlineto{\pgfqpoint{3.239805in}{2.184361in}}%
\pgfpathlineto{\pgfqpoint{3.214775in}{2.164519in}}%
\pgfpathlineto{\pgfqpoint{3.190900in}{2.142893in}}%
\pgfpathlineto{\pgfqpoint{3.166657in}{2.117912in}}%
\pgfpathlineto{\pgfqpoint{3.143835in}{2.091233in}}%
\pgfpathlineto{\pgfqpoint{3.121079in}{2.061107in}}%
\pgfpathlineto{\pgfqpoint{3.099952in}{2.029463in}}%
\pgfpathlineto{\pgfqpoint{3.079251in}{1.994406in}}%
\pgfpathlineto{\pgfqpoint{3.059218in}{1.955915in}}%
\pgfpathlineto{\pgfqpoint{3.040058in}{1.914015in}}%
\pgfpathlineto{\pgfqpoint{3.022809in}{1.871041in}}%
\pgfpathlineto{\pgfqpoint{3.005790in}{1.822536in}}%
\pgfpathlineto{\pgfqpoint{2.990067in}{1.770819in}}%
\pgfpathlineto{\pgfqpoint{2.975708in}{1.715979in}}%
\pgfpathlineto{\pgfqpoint{2.962284in}{1.655680in}}%
\pgfpathlineto{\pgfqpoint{2.950496in}{1.592386in}}%
\pgfpathlineto{\pgfqpoint{2.940383in}{1.526185in}}%
\pgfpathlineto{\pgfqpoint{2.931745in}{1.454681in}}%
\pgfpathlineto{\pgfqpoint{2.925082in}{1.380399in}}%
\pgfpathlineto{\pgfqpoint{2.920647in}{1.305899in}}%
\pgfpathlineto{\pgfqpoint{2.918444in}{1.231270in}}%
\pgfpathlineto{\pgfqpoint{2.918545in}{1.159087in}}%
\pgfpathlineto{\pgfqpoint{2.920787in}{1.091931in}}%
\pgfpathlineto{\pgfqpoint{2.925177in}{1.027412in}}%
\pgfpathlineto{\pgfqpoint{2.931192in}{0.970580in}}%
\pgfpathlineto{\pgfqpoint{2.938760in}{0.919034in}}%
\pgfpathlineto{\pgfqpoint{2.947651in}{0.872852in}}%
\pgfpathlineto{\pgfqpoint{2.958213in}{0.829714in}}%
\pgfpathlineto{\pgfqpoint{2.969670in}{0.792114in}}%
\pgfpathlineto{\pgfqpoint{2.982463in}{0.757773in}}%
\pgfpathlineto{\pgfqpoint{2.996425in}{0.726812in}}%
\pgfpathlineto{\pgfqpoint{3.011299in}{0.699300in}}%
\pgfpathlineto{\pgfqpoint{3.026739in}{0.675225in}}%
\pgfpathlineto{\pgfqpoint{3.043828in}{0.652656in}}%
\pgfpathlineto{\pgfqpoint{3.062495in}{0.631788in}}%
\pgfpathlineto{\pgfqpoint{3.082602in}{0.612753in}}%
\pgfpathlineto{\pgfqpoint{3.103961in}{0.595592in}}%
\pgfpathlineto{\pgfqpoint{3.128268in}{0.579069in}}%
\pgfpathlineto{\pgfqpoint{3.153537in}{0.564554in}}%
\pgfpathlineto{\pgfqpoint{3.181571in}{0.550952in}}%
\pgfpathlineto{\pgfqpoint{3.214371in}{0.537647in}}%
\pgfpathlineto{\pgfqpoint{3.249846in}{0.525712in}}%
\pgfpathlineto{\pgfqpoint{3.290011in}{0.514571in}}%
\pgfpathlineto{\pgfqpoint{3.334820in}{0.504423in}}%
\pgfpathlineto{\pgfqpoint{3.386372in}{0.494999in}}%
\pgfpathlineto{\pgfqpoint{3.446798in}{0.486257in}}%
\pgfpathlineto{\pgfqpoint{3.518243in}{0.478282in}}%
\pgfpathlineto{\pgfqpoint{3.600685in}{0.471409in}}%
\pgfpathlineto{\pgfqpoint{3.696268in}{0.465713in}}%
\pgfpathlineto{\pgfqpoint{3.807144in}{0.461369in}}%
\pgfpathlineto{\pgfqpoint{3.933291in}{0.458719in}}%
\pgfpathlineto{\pgfqpoint{4.063808in}{0.458211in}}%
\pgfpathlineto{\pgfqpoint{4.187792in}{0.459914in}}%
\pgfpathlineto{\pgfqpoint{4.294335in}{0.463521in}}%
\pgfpathlineto{\pgfqpoint{4.381234in}{0.468574in}}%
\pgfpathlineto{\pgfqpoint{4.450636in}{0.474701in}}%
\pgfpathlineto{\pgfqpoint{4.506850in}{0.481799in}}%
\pgfpathlineto{\pgfqpoint{4.552009in}{0.489658in}}%
\pgfpathlineto{\pgfqpoint{4.588239in}{0.498115in}}%
\pgfpathlineto{\pgfqpoint{4.617656in}{0.507110in}}%
\pgfpathlineto{\pgfqpoint{4.642328in}{0.516843in}}%
\pgfpathlineto{\pgfqpoint{4.664194in}{0.527940in}}%
\pgfpathlineto{\pgfqpoint{4.681238in}{0.538945in}}%
\pgfpathlineto{\pgfqpoint{4.697164in}{0.551953in}}%
\pgfpathlineto{\pgfqpoint{4.710076in}{0.565289in}}%
\pgfpathlineto{\pgfqpoint{4.721578in}{0.580218in}}%
\pgfpathlineto{\pgfqpoint{4.731557in}{0.596521in}}%
\pgfpathlineto{\pgfqpoint{4.741000in}{0.616134in}}%
\pgfpathlineto{\pgfqpoint{4.749521in}{0.639027in}}%
\pgfpathlineto{\pgfqpoint{4.757522in}{0.667450in}}%
\pgfpathlineto{\pgfqpoint{4.764572in}{0.701345in}}%
\pgfpathlineto{\pgfqpoint{4.770840in}{0.743043in}}%
\pgfpathlineto{\pgfqpoint{4.776327in}{0.794934in}}%
\pgfpathlineto{\pgfqpoint{4.781278in}{0.864398in}}%
\pgfpathlineto{\pgfqpoint{4.785468in}{0.956371in}}%
\pgfpathlineto{\pgfqpoint{4.789000in}{1.085745in}}%
\pgfpathlineto{\pgfqpoint{4.791852in}{1.277385in}}%
\pgfpathlineto{\pgfqpoint{4.793959in}{1.581057in}}%
\pgfpathlineto{\pgfqpoint{4.794962in}{2.071429in}}%
\pgfpathlineto{\pgfqpoint{4.793967in}{2.559311in}}%
\pgfpathlineto{\pgfqpoint{4.791733in}{2.745981in}}%
\pgfpathlineto{\pgfqpoint{4.788955in}{2.818091in}}%
\pgfpathlineto{\pgfqpoint{4.785731in}{2.850227in}}%
\pgfpathlineto{\pgfqpoint{4.781879in}{2.867057in}}%
\pgfpathlineto{\pgfqpoint{4.777744in}{2.875780in}}%
\pgfpathlineto{\pgfqpoint{4.773097in}{2.880982in}}%
\pgfpathlineto{\pgfqpoint{4.767363in}{2.884504in}}%
\pgfpathlineto{\pgfqpoint{4.756853in}{2.887622in}}%
\pgfpathlineto{\pgfqpoint{4.739548in}{2.889639in}}%
\pgfpathlineto{\pgfqpoint{4.704762in}{2.890882in}}%
\pgfpathlineto{\pgfqpoint{4.602524in}{2.891538in}}%
\pgfpathlineto{\pgfqpoint{3.952100in}{2.891742in}}%
\pgfpathlineto{\pgfqpoint{0.617321in}{2.890753in}}%
\pgfpathlineto{\pgfqpoint{0.549910in}{2.888858in}}%
\pgfpathlineto{\pgfqpoint{0.521735in}{2.886179in}}%
\pgfpathlineto{\pgfqpoint{0.504666in}{2.882389in}}%
\pgfpathlineto{\pgfqpoint{0.494501in}{2.878011in}}%
\pgfpathlineto{\pgfqpoint{0.487180in}{2.872667in}}%
\pgfpathlineto{\pgfqpoint{0.481152in}{2.865519in}}%
\pgfpathlineto{\pgfqpoint{0.475664in}{2.854804in}}%
\pgfpathlineto{\pgfqpoint{0.471318in}{2.840737in}}%
\pgfpathlineto{\pgfqpoint{0.467301in}{2.818823in}}%
\pgfpathlineto{\pgfqpoint{0.463927in}{2.786700in}}%
\pgfpathlineto{\pgfqpoint{0.460918in}{2.734544in}}%
\pgfpathlineto{\pgfqpoint{0.458363in}{2.647474in}}%
\pgfpathlineto{\pgfqpoint{0.456575in}{2.523031in}}%
\pgfpathlineto{\pgfqpoint{0.456575in}{2.523031in}}%
\pgfusepath{stroke}%
\end{pgfscope}%
\begin{pgfscope}%
\pgfpathrectangle{\pgfqpoint{0.448634in}{0.402556in}}{\pgfqpoint{4.350661in}{2.489204in}} %
\pgfusepath{clip}%
\pgfsetrectcap%
\pgfsetroundjoin%
\pgfsetlinewidth{1.003750pt}%
\definecolor{currentstroke}{rgb}{0.172549,0.627451,0.172549}%
\pgfsetstrokecolor{currentstroke}%
\pgfsetdash{}{0pt}%
\pgfpathmoveto{\pgfqpoint{4.798840in}{2.852369in}}%
\pgfpathlineto{\pgfqpoint{4.797564in}{2.889610in}}%
\pgfpathlineto{\pgfqpoint{4.796215in}{2.891483in}}%
\pgfpathlineto{\pgfqpoint{4.787551in}{2.891760in}}%
\pgfpathlineto{\pgfqpoint{0.452128in}{2.891659in}}%
\pgfpathlineto{\pgfqpoint{0.450530in}{2.890082in}}%
\pgfpathlineto{\pgfqpoint{0.449454in}{2.882763in}}%
\pgfpathlineto{\pgfqpoint{0.448970in}{2.845432in}}%
\pgfpathlineto{\pgfqpoint{0.448743in}{2.494454in}}%
\pgfpathlineto{\pgfqpoint{0.449624in}{0.615107in}}%
\pgfpathlineto{\pgfqpoint{0.451433in}{0.510586in}}%
\pgfpathlineto{\pgfqpoint{0.453993in}{0.473374in}}%
\pgfpathlineto{\pgfqpoint{0.457406in}{0.453868in}}%
\pgfpathlineto{\pgfqpoint{0.461540in}{0.442384in}}%
\pgfpathlineto{\pgfqpoint{0.466739in}{0.434437in}}%
\pgfpathlineto{\pgfqpoint{0.473595in}{0.428350in}}%
\pgfpathlineto{\pgfqpoint{0.483492in}{0.423244in}}%
\pgfpathlineto{\pgfqpoint{0.491854in}{0.420501in}}%
\pgfpathlineto{\pgfqpoint{0.491854in}{0.420501in}}%
\pgfusepath{stroke}%
\end{pgfscope}%
\begin{pgfscope}%
\pgfpathrectangle{\pgfqpoint{0.448634in}{0.402556in}}{\pgfqpoint{4.350661in}{2.489204in}} %
\pgfusepath{clip}%
\pgfsetrectcap%
\pgfsetroundjoin%
\pgfsetlinewidth{1.003750pt}%
\definecolor{currentstroke}{rgb}{0.172549,0.627451,0.172549}%
\pgfsetstrokecolor{currentstroke}%
\pgfsetdash{}{0pt}%
\pgfpathmoveto{\pgfqpoint{0.456424in}{1.370137in}}%
\pgfpathlineto{\pgfqpoint{0.459610in}{1.118755in}}%
\pgfpathlineto{\pgfqpoint{0.463695in}{0.962007in}}%
\pgfpathlineto{\pgfqpoint{0.468519in}{0.857610in}}%
\pgfpathlineto{\pgfqpoint{0.474082in}{0.783210in}}%
\pgfpathlineto{\pgfqpoint{0.480226in}{0.728906in}}%
\pgfpathlineto{\pgfqpoint{0.486970in}{0.687306in}}%
\pgfpathlineto{\pgfqpoint{0.494537in}{0.653558in}}%
\pgfpathlineto{\pgfqpoint{0.503107in}{0.625355in}}%
\pgfpathlineto{\pgfqpoint{0.512193in}{0.602750in}}%
\pgfpathlineto{\pgfqpoint{0.522200in}{0.583508in}}%
\pgfpathlineto{\pgfqpoint{0.534108in}{0.565743in}}%
\pgfpathlineto{\pgfqpoint{0.546263in}{0.551507in}}%
\pgfpathlineto{\pgfqpoint{0.559728in}{0.538907in}}%
\pgfpathlineto{\pgfqpoint{0.576129in}{0.526693in}}%
\pgfpathlineto{\pgfqpoint{0.595483in}{0.515351in}}%
\pgfpathlineto{\pgfqpoint{0.617681in}{0.505147in}}%
\pgfpathlineto{\pgfqpoint{0.642568in}{0.496153in}}%
\pgfpathlineto{\pgfqpoint{0.672126in}{0.487778in}}%
\pgfpathlineto{\pgfqpoint{0.708443in}{0.479824in}}%
\pgfpathlineto{\pgfqpoint{0.753649in}{0.472325in}}%
\pgfpathlineto{\pgfqpoint{0.807717in}{0.465660in}}%
\pgfpathlineto{\pgfqpoint{0.877116in}{0.459475in}}%
\pgfpathlineto{\pgfqpoint{0.961828in}{0.454230in}}%
\pgfpathlineto{\pgfqpoint{1.068351in}{0.449916in}}%
\pgfpathlineto{\pgfqpoint{1.201018in}{0.446839in}}%
\pgfpathlineto{\pgfqpoint{1.357637in}{0.445481in}}%
\pgfpathlineto{\pgfqpoint{1.525135in}{0.446232in}}%
\pgfpathlineto{\pgfqpoint{1.686088in}{0.449142in}}%
\pgfpathlineto{\pgfqpoint{1.823074in}{0.453747in}}%
\pgfpathlineto{\pgfqpoint{1.938245in}{0.459764in}}%
\pgfpathlineto{\pgfqpoint{2.031582in}{0.466759in}}%
\pgfpathlineto{\pgfqpoint{2.109580in}{0.474745in}}%
\pgfpathlineto{\pgfqpoint{2.174384in}{0.483535in}}%
\pgfpathlineto{\pgfqpoint{2.228139in}{0.492940in}}%
\pgfpathlineto{\pgfqpoint{2.275119in}{0.503356in}}%
\pgfpathlineto{\pgfqpoint{2.315282in}{0.514501in}}%
\pgfpathlineto{\pgfqpoint{2.350698in}{0.526659in}}%
\pgfpathlineto{\pgfqpoint{2.381320in}{0.539536in}}%
\pgfpathlineto{\pgfqpoint{2.407164in}{0.552659in}}%
\pgfpathlineto{\pgfqpoint{2.430226in}{0.566639in}}%
\pgfpathlineto{\pgfqpoint{2.452282in}{0.582602in}}%
\pgfpathlineto{\pgfqpoint{2.471391in}{0.599069in}}%
\pgfpathlineto{\pgfqpoint{2.489240in}{0.617293in}}%
\pgfpathlineto{\pgfqpoint{2.505678in}{0.637180in}}%
\pgfpathlineto{\pgfqpoint{2.520620in}{0.658557in}}%
\pgfpathlineto{\pgfqpoint{2.535213in}{0.683314in}}%
\pgfpathlineto{\pgfqpoint{2.549115in}{0.711484in}}%
\pgfpathlineto{\pgfqpoint{2.562091in}{0.743004in}}%
\pgfpathlineto{\pgfqpoint{2.574020in}{0.777751in}}%
\pgfpathlineto{\pgfqpoint{2.585502in}{0.817970in}}%
\pgfpathlineto{\pgfqpoint{2.596809in}{0.866038in}}%
\pgfpathlineto{\pgfqpoint{2.607562in}{0.921948in}}%
\pgfpathlineto{\pgfqpoint{2.617925in}{0.988098in}}%
\pgfpathlineto{\pgfqpoint{2.627958in}{1.066918in}}%
\pgfpathlineto{\pgfqpoint{2.637941in}{1.163320in}}%
\pgfpathlineto{\pgfqpoint{2.648424in}{1.287199in}}%
\pgfpathlineto{\pgfqpoint{2.660103in}{1.453438in}}%
\pgfpathlineto{\pgfqpoint{2.674773in}{1.696801in}}%
\pgfpathlineto{\pgfqpoint{2.687716in}{1.945279in}}%
\pgfpathlineto{\pgfqpoint{2.692670in}{2.079573in}}%
\pgfpathlineto{\pgfqpoint{2.693829in}{2.166682in}}%
\pgfpathlineto{\pgfqpoint{2.692565in}{2.233870in}}%
\pgfpathlineto{\pgfqpoint{2.689436in}{2.286015in}}%
\pgfpathlineto{\pgfqpoint{2.684859in}{2.327999in}}%
\pgfpathlineto{\pgfqpoint{2.678725in}{2.364664in}}%
\pgfpathlineto{\pgfqpoint{2.671356in}{2.395897in}}%
\pgfpathlineto{\pgfqpoint{2.662489in}{2.423981in}}%
\pgfpathlineto{\pgfqpoint{2.652361in}{2.448778in}}%
\pgfpathlineto{\pgfqpoint{2.641365in}{2.470245in}}%
\pgfpathlineto{\pgfqpoint{2.628643in}{2.490425in}}%
\pgfpathlineto{\pgfqpoint{2.614279in}{2.509106in}}%
\pgfpathlineto{\pgfqpoint{2.598443in}{2.526159in}}%
\pgfpathlineto{\pgfqpoint{2.579590in}{2.543005in}}%
\pgfpathlineto{\pgfqpoint{2.559532in}{2.557923in}}%
\pgfpathlineto{\pgfqpoint{2.536602in}{2.572183in}}%
\pgfpathlineto{\pgfqpoint{2.510850in}{2.585538in}}%
\pgfpathlineto{\pgfqpoint{2.482360in}{2.597837in}}%
\pgfpathlineto{\pgfqpoint{2.449134in}{2.609683in}}%
\pgfpathlineto{\pgfqpoint{2.411184in}{2.620696in}}%
\pgfpathlineto{\pgfqpoint{2.368552in}{2.630606in}}%
\pgfpathlineto{\pgfqpoint{2.321294in}{2.639221in}}%
\pgfpathlineto{\pgfqpoint{2.269467in}{2.646399in}}%
\pgfpathlineto{\pgfqpoint{2.210954in}{2.652193in}}%
\pgfpathlineto{\pgfqpoint{2.147967in}{2.656153in}}%
\pgfpathlineto{\pgfqpoint{2.080556in}{2.658135in}}%
\pgfpathlineto{\pgfqpoint{2.010948in}{2.657971in}}%
\pgfpathlineto{\pgfqpoint{1.939195in}{2.655572in}}%
\pgfpathlineto{\pgfqpoint{1.867527in}{2.650913in}}%
\pgfpathlineto{\pgfqpoint{1.798171in}{2.644140in}}%
\pgfpathlineto{\pgfqpoint{1.733341in}{2.635606in}}%
\pgfpathlineto{\pgfqpoint{1.673075in}{2.625521in}}%
\pgfpathlineto{\pgfqpoint{1.615274in}{2.613610in}}%
\pgfpathlineto{\pgfqpoint{1.562133in}{2.600402in}}%
\pgfpathlineto{\pgfqpoint{1.513681in}{2.586139in}}%
\pgfpathlineto{\pgfqpoint{1.467862in}{2.570344in}}%
\pgfpathlineto{\pgfqpoint{1.426794in}{2.553923in}}%
\pgfpathlineto{\pgfqpoint{1.388447in}{2.536289in}}%
\pgfpathlineto{\pgfqpoint{1.352878in}{2.517566in}}%
\pgfpathlineto{\pgfqpoint{1.320128in}{2.497922in}}%
\pgfpathlineto{\pgfqpoint{1.288379in}{2.476236in}}%
\pgfpathlineto{\pgfqpoint{1.259592in}{2.453861in}}%
\pgfpathlineto{\pgfqpoint{1.232050in}{2.429520in}}%
\pgfpathlineto{\pgfqpoint{1.207527in}{2.404898in}}%
\pgfpathlineto{\pgfqpoint{1.184409in}{2.378557in}}%
\pgfpathlineto{\pgfqpoint{1.162828in}{2.350561in}}%
\pgfpathlineto{\pgfqpoint{1.142891in}{2.321011in}}%
\pgfpathlineto{\pgfqpoint{1.124675in}{2.290041in}}%
\pgfpathlineto{\pgfqpoint{1.108225in}{2.257802in}}%
\pgfpathlineto{\pgfqpoint{1.092639in}{2.222199in}}%
\pgfpathlineto{\pgfqpoint{1.079059in}{2.185535in}}%
\pgfpathlineto{\pgfqpoint{1.067443in}{2.147998in}}%
\pgfpathlineto{\pgfqpoint{1.057187in}{2.107348in}}%
\pgfpathlineto{\pgfqpoint{1.049004in}{2.066086in}}%
\pgfpathlineto{\pgfqpoint{1.042513in}{2.021906in}}%
\pgfpathlineto{\pgfqpoint{1.038177in}{1.977382in}}%
\pgfpathlineto{\pgfqpoint{1.035866in}{1.930167in}}%
\pgfpathlineto{\pgfqpoint{1.035826in}{1.882878in}}%
\pgfpathlineto{\pgfqpoint{1.038031in}{1.835656in}}%
\pgfpathlineto{\pgfqpoint{1.042474in}{1.788641in}}%
\pgfpathlineto{\pgfqpoint{1.049176in}{1.741979in}}%
\pgfpathlineto{\pgfqpoint{1.057644in}{1.698239in}}%
\pgfpathlineto{\pgfqpoint{1.068221in}{1.655105in}}%
\pgfpathlineto{\pgfqpoint{1.080962in}{1.612745in}}%
\pgfpathlineto{\pgfqpoint{1.095031in}{1.573617in}}%
\pgfpathlineto{\pgfqpoint{1.111115in}{1.535520in}}%
\pgfpathlineto{\pgfqpoint{1.128118in}{1.500775in}}%
\pgfpathlineto{\pgfqpoint{1.146930in}{1.467274in}}%
\pgfpathlineto{\pgfqpoint{1.167531in}{1.435181in}}%
\pgfpathlineto{\pgfqpoint{1.189874in}{1.404652in}}%
\pgfpathlineto{\pgfqpoint{1.213884in}{1.375828in}}%
\pgfpathlineto{\pgfqpoint{1.237817in}{1.350457in}}%
\pgfpathlineto{\pgfqpoint{1.264748in}{1.325237in}}%
\pgfpathlineto{\pgfqpoint{1.292991in}{1.301972in}}%
\pgfpathlineto{\pgfqpoint{1.322398in}{1.280678in}}%
\pgfpathlineto{\pgfqpoint{1.352820in}{1.261340in}}%
\pgfpathlineto{\pgfqpoint{1.386095in}{1.242889in}}%
\pgfpathlineto{\pgfqpoint{1.420190in}{1.226516in}}%
\pgfpathlineto{\pgfqpoint{1.457024in}{1.211329in}}%
\pgfpathlineto{\pgfqpoint{1.496554in}{1.197536in}}%
\pgfpathlineto{\pgfqpoint{1.538719in}{1.185287in}}%
\pgfpathlineto{\pgfqpoint{1.583441in}{1.174641in}}%
\pgfpathlineto{\pgfqpoint{1.634929in}{1.164775in}}%
\pgfpathlineto{\pgfqpoint{1.706063in}{1.153745in}}%
\pgfpathlineto{\pgfqpoint{1.768492in}{1.143417in}}%
\pgfpathlineto{\pgfqpoint{1.796122in}{1.136567in}}%
\pgfpathlineto{\pgfqpoint{1.812683in}{1.130481in}}%
\pgfpathlineto{\pgfqpoint{1.824471in}{1.124102in}}%
\pgfpathlineto{\pgfqpoint{1.833209in}{1.116741in}}%
\pgfpathlineto{\pgfqpoint{1.838498in}{1.108890in}}%
\pgfpathlineto{\pgfqpoint{1.840588in}{1.101849in}}%
\pgfpathlineto{\pgfqpoint{1.840619in}{1.094412in}}%
\pgfpathlineto{\pgfqpoint{1.837931in}{1.084986in}}%
\pgfpathlineto{\pgfqpoint{1.833246in}{1.076615in}}%
\pgfpathlineto{\pgfqpoint{1.825819in}{1.067542in}}%
\pgfpathlineto{\pgfqpoint{1.813813in}{1.056850in}}%
\pgfpathlineto{\pgfqpoint{1.798819in}{1.046763in}}%
\pgfpathlineto{\pgfqpoint{1.781016in}{1.037462in}}%
\pgfpathlineto{\pgfqpoint{1.758447in}{1.028391in}}%
\pgfpathlineto{\pgfqpoint{1.733203in}{1.020815in}}%
\pgfpathlineto{\pgfqpoint{1.705410in}{1.014872in}}%
\pgfpathlineto{\pgfqpoint{1.675178in}{1.010714in}}%
\pgfpathlineto{\pgfqpoint{1.642610in}{1.008507in}}%
\pgfpathlineto{\pgfqpoint{1.607809in}{1.008432in}}%
\pgfpathlineto{\pgfqpoint{1.570886in}{1.010691in}}%
\pgfpathlineto{\pgfqpoint{1.534118in}{1.015181in}}%
\pgfpathlineto{\pgfqpoint{1.495454in}{1.022233in}}%
\pgfpathlineto{\pgfqpoint{1.457161in}{1.031563in}}%
\pgfpathlineto{\pgfqpoint{1.419337in}{1.043132in}}%
\pgfpathlineto{\pgfqpoint{1.382089in}{1.056929in}}%
\pgfpathlineto{\pgfqpoint{1.347544in}{1.072019in}}%
\pgfpathlineto{\pgfqpoint{1.313727in}{1.089133in}}%
\pgfpathlineto{\pgfqpoint{1.280762in}{1.108299in}}%
\pgfpathlineto{\pgfqpoint{1.248782in}{1.129536in}}%
\pgfpathlineto{\pgfqpoint{1.219708in}{1.151422in}}%
\pgfpathlineto{\pgfqpoint{1.191752in}{1.175138in}}%
\pgfpathlineto{\pgfqpoint{1.165031in}{1.200649in}}%
\pgfpathlineto{\pgfqpoint{1.139653in}{1.227898in}}%
\pgfpathlineto{\pgfqpoint{1.115714in}{1.256800in}}%
\pgfpathlineto{\pgfqpoint{1.093288in}{1.287251in}}%
\pgfpathlineto{\pgfqpoint{1.071178in}{1.321163in}}%
\pgfpathlineto{\pgfqpoint{1.050868in}{1.356520in}}%
\pgfpathlineto{\pgfqpoint{1.032365in}{1.393152in}}%
\pgfpathlineto{\pgfqpoint{1.014718in}{1.433142in}}%
\pgfpathlineto{\pgfqpoint{0.999024in}{1.474185in}}%
\pgfpathlineto{\pgfqpoint{0.984506in}{1.518461in}}%
\pgfpathlineto{\pgfqpoint{0.972010in}{1.563537in}}%
\pgfpathlineto{\pgfqpoint{0.960944in}{1.611678in}}%
\pgfpathlineto{\pgfqpoint{0.951530in}{1.662824in}}%
\pgfpathlineto{\pgfqpoint{0.944286in}{1.714431in}}%
\pgfpathlineto{\pgfqpoint{0.938950in}{1.768847in}}%
\pgfpathlineto{\pgfqpoint{0.935870in}{1.823491in}}%
\pgfpathlineto{\pgfqpoint{0.935034in}{1.878240in}}%
\pgfpathlineto{\pgfqpoint{0.936466in}{1.932973in}}%
\pgfpathlineto{\pgfqpoint{0.940005in}{1.985084in}}%
\pgfpathlineto{\pgfqpoint{0.945759in}{2.036935in}}%
\pgfpathlineto{\pgfqpoint{0.953410in}{2.085938in}}%
\pgfpathlineto{\pgfqpoint{0.962764in}{2.132000in}}%
\pgfpathlineto{\pgfqpoint{0.974287in}{2.177414in}}%
\pgfpathlineto{\pgfqpoint{0.987332in}{2.219653in}}%
\pgfpathlineto{\pgfqpoint{1.001667in}{2.258654in}}%
\pgfpathlineto{\pgfqpoint{1.018051in}{2.296583in}}%
\pgfpathlineto{\pgfqpoint{1.035401in}{2.331101in}}%
\pgfpathlineto{\pgfqpoint{1.054650in}{2.364275in}}%
\pgfpathlineto{\pgfqpoint{1.074406in}{2.393984in}}%
\pgfpathlineto{\pgfqpoint{1.095771in}{2.422197in}}%
\pgfpathlineto{\pgfqpoint{1.118662in}{2.448797in}}%
\pgfpathlineto{\pgfqpoint{1.142967in}{2.473701in}}%
\pgfpathlineto{\pgfqpoint{1.168550in}{2.496867in}}%
\pgfpathlineto{\pgfqpoint{1.197085in}{2.519662in}}%
\pgfpathlineto{\pgfqpoint{1.226727in}{2.540526in}}%
\pgfpathlineto{\pgfqpoint{1.259242in}{2.560673in}}%
\pgfpathlineto{\pgfqpoint{1.294612in}{2.579881in}}%
\pgfpathlineto{\pgfqpoint{1.332792in}{2.597982in}}%
\pgfpathlineto{\pgfqpoint{1.373719in}{2.614859in}}%
\pgfpathlineto{\pgfqpoint{1.417319in}{2.630445in}}%
\pgfpathlineto{\pgfqpoint{1.465632in}{2.645312in}}%
\pgfpathlineto{\pgfqpoint{1.518640in}{2.659204in}}%
\pgfpathlineto{\pgfqpoint{1.576309in}{2.671929in}}%
\pgfpathlineto{\pgfqpoint{1.638597in}{2.683344in}}%
\pgfpathlineto{\pgfqpoint{1.705462in}{2.693343in}}%
\pgfpathlineto{\pgfqpoint{1.779027in}{2.702064in}}%
\pgfpathlineto{\pgfqpoint{1.857097in}{2.709077in}}%
\pgfpathlineto{\pgfqpoint{1.939633in}{2.714280in}}%
\pgfpathlineto{\pgfqpoint{2.026598in}{2.717513in}}%
\pgfpathlineto{\pgfqpoint{2.113605in}{2.718523in}}%
\pgfpathlineto{\pgfqpoint{2.198435in}{2.717303in}}%
\pgfpathlineto{\pgfqpoint{2.278866in}{2.713929in}}%
\pgfpathlineto{\pgfqpoint{2.352678in}{2.708598in}}%
\pgfpathlineto{\pgfqpoint{2.417657in}{2.701709in}}%
\pgfpathlineto{\pgfqpoint{2.473770in}{2.693630in}}%
\pgfpathlineto{\pgfqpoint{2.523140in}{2.684368in}}%
\pgfpathlineto{\pgfqpoint{2.565726in}{2.674202in}}%
\pgfpathlineto{\pgfqpoint{2.601510in}{2.663544in}}%
\pgfpathlineto{\pgfqpoint{2.632577in}{2.652142in}}%
\pgfpathlineto{\pgfqpoint{2.658899in}{2.640331in}}%
\pgfpathlineto{\pgfqpoint{2.682438in}{2.627436in}}%
\pgfpathlineto{\pgfqpoint{2.703062in}{2.613571in}}%
\pgfpathlineto{\pgfqpoint{2.720674in}{2.598978in}}%
\pgfpathlineto{\pgfqpoint{2.735263in}{2.584053in}}%
\pgfpathlineto{\pgfqpoint{2.748320in}{2.567377in}}%
\pgfpathlineto{\pgfqpoint{2.759553in}{2.549046in}}%
\pgfpathlineto{\pgfqpoint{2.768788in}{2.529306in}}%
\pgfpathlineto{\pgfqpoint{2.776017in}{2.508498in}}%
\pgfpathlineto{\pgfqpoint{2.781884in}{2.484540in}}%
\pgfpathlineto{\pgfqpoint{2.786102in}{2.457597in}}%
\pgfpathlineto{\pgfqpoint{2.788720in}{2.425384in}}%
\pgfpathlineto{\pgfqpoint{2.789427in}{2.388061in}}%
\pgfpathlineto{\pgfqpoint{2.787962in}{2.340801in}}%
\pgfpathlineto{\pgfqpoint{2.783672in}{2.278768in}}%
\pgfpathlineto{\pgfqpoint{2.774289in}{2.179783in}}%
\pgfpathlineto{\pgfqpoint{2.743611in}{1.868119in}}%
\pgfpathlineto{\pgfqpoint{2.730112in}{1.702060in}}%
\pgfpathlineto{\pgfqpoint{2.717287in}{1.515949in}}%
\pgfpathlineto{\pgfqpoint{2.702602in}{1.267597in}}%
\pgfpathlineto{\pgfqpoint{2.684434in}{0.964630in}}%
\pgfpathlineto{\pgfqpoint{2.675374in}{0.850600in}}%
\pgfpathlineto{\pgfqpoint{2.667030in}{0.771523in}}%
\pgfpathlineto{\pgfqpoint{2.658752in}{0.712543in}}%
\pgfpathlineto{\pgfqpoint{2.650176in}{0.666284in}}%
\pgfpathlineto{\pgfqpoint{2.640820in}{0.627931in}}%
\pgfpathlineto{\pgfqpoint{2.631145in}{0.597534in}}%
\pgfpathlineto{\pgfqpoint{2.621004in}{0.572745in}}%
\pgfpathlineto{\pgfqpoint{2.609856in}{0.551383in}}%
\pgfpathlineto{\pgfqpoint{2.598042in}{0.533534in}}%
\pgfpathlineto{\pgfqpoint{2.584496in}{0.517378in}}%
\pgfpathlineto{\pgfqpoint{2.571109in}{0.504669in}}%
\pgfpathlineto{\pgfqpoint{2.554789in}{0.492313in}}%
\pgfpathlineto{\pgfqpoint{2.537457in}{0.481914in}}%
\pgfpathlineto{\pgfqpoint{2.517374in}{0.472367in}}%
\pgfpathlineto{\pgfqpoint{2.492542in}{0.463178in}}%
\pgfpathlineto{\pgfqpoint{2.462979in}{0.454833in}}%
\pgfpathlineto{\pgfqpoint{2.428766in}{0.447542in}}%
\pgfpathlineto{\pgfqpoint{2.385671in}{0.440735in}}%
\pgfpathlineto{\pgfqpoint{2.331557in}{0.434581in}}%
\pgfpathlineto{\pgfqpoint{2.262115in}{0.429077in}}%
\pgfpathlineto{\pgfqpoint{2.170851in}{0.424236in}}%
\pgfpathlineto{\pgfqpoint{2.049086in}{0.420134in}}%
\pgfpathlineto{\pgfqpoint{1.879436in}{0.416783in}}%
\pgfpathlineto{\pgfqpoint{1.640159in}{0.414418in}}%
\pgfpathlineto{\pgfqpoint{1.322562in}{0.413569in}}%
\pgfpathlineto{\pgfqpoint{1.020194in}{0.414850in}}%
\pgfpathlineto{\pgfqpoint{0.822256in}{0.417715in}}%
\pgfpathlineto{\pgfqpoint{0.704835in}{0.421430in}}%
\pgfpathlineto{\pgfqpoint{0.630976in}{0.425829in}}%
\pgfpathlineto{\pgfqpoint{0.583316in}{0.430734in}}%
\pgfpathlineto{\pgfqpoint{0.551033in}{0.436123in}}%
\pgfpathlineto{\pgfqpoint{0.527708in}{0.442189in}}%
\pgfpathlineto{\pgfqpoint{0.511250in}{0.448625in}}%
\pgfpathlineto{\pgfqpoint{0.499549in}{0.455216in}}%
\pgfpathlineto{\pgfqpoint{0.488916in}{0.463841in}}%
\pgfpathlineto{\pgfqpoint{0.481322in}{0.472730in}}%
\pgfpathlineto{\pgfqpoint{0.474078in}{0.485127in}}%
\pgfpathlineto{\pgfqpoint{0.468753in}{0.498748in}}%
\pgfpathlineto{\pgfqpoint{0.463870in}{0.517848in}}%
\pgfpathlineto{\pgfqpoint{0.459679in}{0.544796in}}%
\pgfpathlineto{\pgfqpoint{0.456386in}{0.581938in}}%
\pgfpathlineto{\pgfqpoint{0.453731in}{0.639106in}}%
\pgfpathlineto{\pgfqpoint{0.451681in}{0.736155in}}%
\pgfpathlineto{\pgfqpoint{0.450220in}{0.927815in}}%
\pgfpathlineto{\pgfqpoint{0.449345in}{1.403252in}}%
\pgfpathlineto{\pgfqpoint{0.449543in}{2.682703in}}%
\pgfpathlineto{\pgfqpoint{0.451011in}{2.856932in}}%
\pgfpathlineto{\pgfqpoint{0.452802in}{2.879219in}}%
\pgfpathlineto{\pgfqpoint{0.455188in}{2.886108in}}%
\pgfpathlineto{\pgfqpoint{0.458626in}{2.889028in}}%
\pgfpathlineto{\pgfqpoint{0.464996in}{2.890553in}}%
\pgfpathlineto{\pgfqpoint{0.482377in}{2.891423in}}%
\pgfpathlineto{\pgfqpoint{0.565038in}{2.891729in}}%
\pgfpathlineto{\pgfqpoint{2.733842in}{2.891760in}}%
\pgfpathlineto{\pgfqpoint{4.789510in}{2.890885in}}%
\pgfpathlineto{\pgfqpoint{4.793727in}{2.889730in}}%
\pgfpathlineto{\pgfqpoint{4.795481in}{2.888307in}}%
\pgfpathlineto{\pgfqpoint{4.797106in}{2.881145in}}%
\pgfpathlineto{\pgfqpoint{4.797997in}{2.858771in}}%
\pgfpathlineto{\pgfqpoint{4.798039in}{2.856283in}}%
\pgfpathlineto{\pgfqpoint{4.798039in}{2.856283in}}%
\pgfusepath{stroke}%
\end{pgfscope}%
\begin{pgfscope}%
\pgfpathrectangle{\pgfqpoint{0.448634in}{0.402556in}}{\pgfqpoint{4.350661in}{2.489204in}} %
\pgfusepath{clip}%
\pgfsetrectcap%
\pgfsetroundjoin%
\pgfsetlinewidth{1.003750pt}%
\definecolor{currentstroke}{rgb}{0.172549,0.627451,0.172549}%
\pgfsetstrokecolor{currentstroke}%
\pgfsetdash{}{0pt}%
\pgfpathmoveto{\pgfqpoint{3.428772in}{0.402610in}}%
\pgfpathlineto{\pgfqpoint{2.806632in}{0.403760in}}%
\pgfpathlineto{\pgfqpoint{2.769692in}{0.405578in}}%
\pgfpathlineto{\pgfqpoint{2.754632in}{0.408064in}}%
\pgfpathlineto{\pgfqpoint{2.746391in}{0.411198in}}%
\pgfpathlineto{\pgfqpoint{2.740943in}{0.415265in}}%
\pgfpathlineto{\pgfqpoint{2.736784in}{0.420984in}}%
\pgfpathlineto{\pgfqpoint{2.733281in}{0.430071in}}%
\pgfpathlineto{\pgfqpoint{2.730449in}{0.444636in}}%
\pgfpathlineto{\pgfqpoint{2.728238in}{0.469392in}}%
\pgfpathlineto{\pgfqpoint{2.726470in}{0.519131in}}%
\pgfpathlineto{\pgfqpoint{2.725711in}{0.613715in}}%
\pgfpathlineto{\pgfqpoint{2.726842in}{0.768038in}}%
\pgfpathlineto{\pgfqpoint{2.730556in}{0.962148in}}%
\pgfpathlineto{\pgfqpoint{2.736611in}{1.158670in}}%
\pgfpathlineto{\pgfqpoint{2.744092in}{1.327718in}}%
\pgfpathlineto{\pgfqpoint{2.753201in}{1.484189in}}%
\pgfpathlineto{\pgfqpoint{2.763257in}{1.620609in}}%
\pgfpathlineto{\pgfqpoint{2.776118in}{1.764216in}}%
\pgfpathlineto{\pgfqpoint{2.788914in}{1.877776in}}%
\pgfpathlineto{\pgfqpoint{2.805748in}{2.005740in}}%
\pgfpathlineto{\pgfqpoint{2.821176in}{2.101198in}}%
\pgfpathlineto{\pgfqpoint{2.838359in}{2.193718in}}%
\pgfpathlineto{\pgfqpoint{2.859135in}{2.292966in}}%
\pgfpathlineto{\pgfqpoint{2.887209in}{2.425960in}}%
\pgfpathlineto{\pgfqpoint{2.896991in}{2.479559in}}%
\pgfpathlineto{\pgfqpoint{2.901543in}{2.516523in}}%
\pgfpathlineto{\pgfqpoint{2.902849in}{2.543854in}}%
\pgfpathlineto{\pgfqpoint{2.901957in}{2.566223in}}%
\pgfpathlineto{\pgfqpoint{2.899151in}{2.585863in}}%
\pgfpathlineto{\pgfqpoint{2.894794in}{2.602546in}}%
\pgfpathlineto{\pgfqpoint{2.888484in}{2.618388in}}%
\pgfpathlineto{\pgfqpoint{2.880257in}{2.633033in}}%
\pgfpathlineto{\pgfqpoint{2.870348in}{2.646246in}}%
\pgfpathlineto{\pgfqpoint{2.857399in}{2.659530in}}%
\pgfpathlineto{\pgfqpoint{2.843189in}{2.671010in}}%
\pgfpathlineto{\pgfqpoint{2.824237in}{2.683209in}}%
\pgfpathlineto{\pgfqpoint{2.802413in}{2.694418in}}%
\pgfpathlineto{\pgfqpoint{2.775809in}{2.705369in}}%
\pgfpathlineto{\pgfqpoint{2.744461in}{2.715715in}}%
\pgfpathlineto{\pgfqpoint{2.708436in}{2.725252in}}%
\pgfpathlineto{\pgfqpoint{2.665655in}{2.734289in}}%
\pgfpathlineto{\pgfqpoint{2.613991in}{2.742869in}}%
\pgfpathlineto{\pgfqpoint{2.553459in}{2.750589in}}%
\pgfpathlineto{\pgfqpoint{2.481920in}{2.757365in}}%
\pgfpathlineto{\pgfqpoint{2.399398in}{2.762839in}}%
\pgfpathlineto{\pgfqpoint{2.310269in}{2.766482in}}%
\pgfpathlineto{\pgfqpoint{2.175416in}{2.768725in}}%
\pgfpathlineto{\pgfqpoint{2.066653in}{2.767942in}}%
\pgfpathlineto{\pgfqpoint{1.953570in}{2.764859in}}%
\pgfpathlineto{\pgfqpoint{1.851429in}{2.759759in}}%
\pgfpathlineto{\pgfqpoint{1.745051in}{2.752169in}}%
\pgfpathlineto{\pgfqpoint{1.658373in}{2.743453in}}%
\pgfpathlineto{\pgfqpoint{1.580552in}{2.733461in}}%
\pgfpathlineto{\pgfqpoint{1.490057in}{2.719338in}}%
\pgfpathlineto{\pgfqpoint{1.417231in}{2.704698in}}%
\pgfpathlineto{\pgfqpoint{1.361992in}{2.690818in}}%
\pgfpathlineto{\pgfqpoint{1.311460in}{2.675819in}}%
\pgfpathlineto{\pgfqpoint{1.265667in}{2.659924in}}%
\pgfpathlineto{\pgfqpoint{1.222575in}{2.642586in}}%
\pgfpathlineto{\pgfqpoint{1.184324in}{2.624682in}}%
\pgfpathlineto{\pgfqpoint{1.148892in}{2.605623in}}%
\pgfpathlineto{\pgfqpoint{1.116331in}{2.585573in}}%
\pgfpathlineto{\pgfqpoint{1.092327in}{2.568512in}}%
\pgfpathlineto{\pgfqpoint{1.079760in}{2.558686in}}%
\pgfpathlineto{\pgfqpoint{1.051544in}{2.535379in}}%
\pgfpathlineto{\pgfqpoint{1.026312in}{2.511712in}}%
\pgfpathlineto{\pgfqpoint{1.002399in}{2.486318in}}%
\pgfpathlineto{\pgfqpoint{0.979913in}{2.459269in}}%
\pgfpathlineto{\pgfqpoint{0.958934in}{2.430678in}}%
\pgfpathlineto{\pgfqpoint{0.938264in}{2.398643in}}%
\pgfpathlineto{\pgfqpoint{0.923047in}{2.371385in}}%
\pgfpathlineto{\pgfqpoint{0.904513in}{2.334774in}}%
\pgfpathlineto{\pgfqpoint{0.887854in}{2.297001in}}%
\pgfpathlineto{\pgfqpoint{0.872131in}{2.255971in}}%
\pgfpathlineto{\pgfqpoint{0.857508in}{2.211741in}}%
\pgfpathlineto{\pgfqpoint{0.844762in}{2.166757in}}%
\pgfpathlineto{\pgfqpoint{0.838624in}{2.140306in}}%
\pgfpathlineto{\pgfqpoint{0.826982in}{2.087194in}}%
\pgfpathlineto{\pgfqpoint{0.816322in}{2.028715in}}%
\pgfpathlineto{\pgfqpoint{0.810087in}{1.984495in}}%
\pgfpathlineto{\pgfqpoint{0.808026in}{1.967238in}}%
\pgfpathlineto{\pgfqpoint{0.800076in}{1.898140in}}%
\pgfpathlineto{\pgfqpoint{0.793713in}{1.823823in}}%
\pgfpathlineto{\pgfqpoint{0.788799in}{1.741875in}}%
\pgfpathlineto{\pgfqpoint{0.786199in}{1.677225in}}%
\pgfpathlineto{\pgfqpoint{0.776951in}{1.453481in}}%
\pgfpathlineto{\pgfqpoint{0.773280in}{1.418894in}}%
\pgfpathlineto{\pgfqpoint{0.768298in}{1.389582in}}%
\pgfpathlineto{\pgfqpoint{0.762752in}{1.368108in}}%
\pgfpathlineto{\pgfqpoint{0.756722in}{1.352123in}}%
\pgfpathlineto{\pgfqpoint{0.749752in}{1.339519in}}%
\pgfpathlineto{\pgfqpoint{0.742201in}{1.330599in}}%
\pgfpathlineto{\pgfqpoint{0.734854in}{1.325312in}}%
\pgfpathlineto{\pgfqpoint{0.726558in}{1.322419in}}%
\pgfpathlineto{\pgfqpoint{0.717884in}{1.322223in}}%
\pgfpathlineto{\pgfqpoint{0.709412in}{1.324411in}}%
\pgfpathlineto{\pgfqpoint{0.699548in}{1.329604in}}%
\pgfpathlineto{\pgfqpoint{0.688894in}{1.338203in}}%
\pgfpathlineto{\pgfqpoint{0.677907in}{1.350248in}}%
\pgfpathlineto{\pgfqpoint{0.666886in}{1.365647in}}%
\pgfpathlineto{\pgfqpoint{0.654913in}{1.386417in}}%
\pgfpathlineto{\pgfqpoint{0.642574in}{1.412730in}}%
\pgfpathlineto{\pgfqpoint{0.630328in}{1.444629in}}%
\pgfpathlineto{\pgfqpoint{0.618504in}{1.482081in}}%
\pgfpathlineto{\pgfqpoint{0.608613in}{1.520256in}}%
\pgfpathlineto{\pgfqpoint{0.590203in}{1.612445in}}%
\pgfpathlineto{\pgfqpoint{0.581848in}{1.668884in}}%
\pgfpathlineto{\pgfqpoint{0.573137in}{1.740376in}}%
\pgfpathlineto{\pgfqpoint{0.567062in}{1.807213in}}%
\pgfpathlineto{\pgfqpoint{0.560532in}{1.896510in}}%
\pgfpathlineto{\pgfqpoint{0.555526in}{1.995910in}}%
\pgfpathlineto{\pgfqpoint{0.552564in}{2.097908in}}%
\pgfpathlineto{\pgfqpoint{0.551526in}{2.204935in}}%
\pgfpathlineto{\pgfqpoint{0.552728in}{2.309470in}}%
\pgfpathlineto{\pgfqpoint{0.556011in}{2.403981in}}%
\pgfpathlineto{\pgfqpoint{0.560953in}{2.483430in}}%
\pgfpathlineto{\pgfqpoint{0.567303in}{2.550240in}}%
\pgfpathlineto{\pgfqpoint{0.574928in}{2.606817in}}%
\pgfpathlineto{\pgfqpoint{0.582988in}{2.650657in}}%
\pgfpathlineto{\pgfqpoint{0.592756in}{2.691452in}}%
\pgfpathlineto{\pgfqpoint{0.602650in}{2.721756in}}%
\pgfpathlineto{\pgfqpoint{0.612983in}{2.746441in}}%
\pgfpathlineto{\pgfqpoint{0.624292in}{2.767692in}}%
\pgfpathlineto{\pgfqpoint{0.636231in}{2.785433in}}%
\pgfpathlineto{\pgfqpoint{0.649892in}{2.801461in}}%
\pgfpathlineto{\pgfqpoint{0.663386in}{2.814020in}}%
\pgfpathlineto{\pgfqpoint{0.679842in}{2.826135in}}%
\pgfpathlineto{\pgfqpoint{0.697326in}{2.836197in}}%
\pgfpathlineto{\pgfqpoint{0.715574in}{2.844285in}}%
\pgfpathlineto{\pgfqpoint{0.738439in}{2.852335in}}%
\pgfpathlineto{\pgfqpoint{0.765983in}{2.859639in}}%
\pgfpathlineto{\pgfqpoint{0.800300in}{2.866256in}}%
\pgfpathlineto{\pgfqpoint{0.841340in}{2.871832in}}%
\pgfpathlineto{\pgfqpoint{0.895547in}{2.876803in}}%
\pgfpathlineto{\pgfqpoint{0.969413in}{2.881069in}}%
\pgfpathlineto{\pgfqpoint{1.071608in}{2.884501in}}%
\pgfpathlineto{\pgfqpoint{1.219512in}{2.887074in}}%
\pgfpathlineto{\pgfqpoint{1.471844in}{2.889091in}}%
\pgfpathlineto{\pgfqpoint{1.956941in}{2.890384in}}%
\pgfpathlineto{\pgfqpoint{3.096814in}{2.890781in}}%
\pgfpathlineto{\pgfqpoint{3.995224in}{2.889388in}}%
\pgfpathlineto{\pgfqpoint{4.275833in}{2.887011in}}%
\pgfpathlineto{\pgfqpoint{4.412847in}{2.883743in}}%
\pgfpathlineto{\pgfqpoint{4.491081in}{2.879810in}}%
\pgfpathlineto{\pgfqpoint{4.543127in}{2.875163in}}%
\pgfpathlineto{\pgfqpoint{4.579810in}{2.869841in}}%
\pgfpathlineto{\pgfqpoint{4.607580in}{2.863763in}}%
\pgfpathlineto{\pgfqpoint{4.630623in}{2.856424in}}%
\pgfpathlineto{\pgfqpoint{4.648833in}{2.848228in}}%
\pgfpathlineto{\pgfqpoint{4.664136in}{2.838773in}}%
\pgfpathlineto{\pgfqpoint{4.676470in}{2.828576in}}%
\pgfpathlineto{\pgfqpoint{4.687502in}{2.816585in}}%
\pgfpathlineto{\pgfqpoint{4.697051in}{2.803027in}}%
\pgfpathlineto{\pgfqpoint{4.706194in}{2.786098in}}%
\pgfpathlineto{\pgfqpoint{4.714508in}{2.765827in}}%
\pgfpathlineto{\pgfqpoint{4.722462in}{2.740013in}}%
\pgfpathlineto{\pgfqpoint{4.729577in}{2.708703in}}%
\pgfpathlineto{\pgfqpoint{4.736162in}{2.669601in}}%
\pgfpathlineto{\pgfqpoint{4.742419in}{2.617826in}}%
\pgfpathlineto{\pgfqpoint{4.747859in}{2.553410in}}%
\pgfpathlineto{\pgfqpoint{4.752661in}{2.468958in}}%
\pgfpathlineto{\pgfqpoint{4.756610in}{2.359528in}}%
\pgfpathlineto{\pgfqpoint{4.759416in}{2.217681in}}%
\pgfpathlineto{\pgfqpoint{4.760596in}{2.043444in}}%
\pgfpathlineto{\pgfqpoint{4.759662in}{1.851779in}}%
\pgfpathlineto{\pgfqpoint{4.756587in}{1.667613in}}%
\pgfpathlineto{\pgfqpoint{4.751596in}{1.503428in}}%
\pgfpathlineto{\pgfqpoint{4.745410in}{1.374185in}}%
\pgfpathlineto{\pgfqpoint{4.738113in}{1.267479in}}%
\pgfpathlineto{\pgfqpoint{4.729621in}{1.175896in}}%
\pgfpathlineto{\pgfqpoint{4.720762in}{1.104428in}}%
\pgfpathlineto{\pgfqpoint{4.711045in}{1.043204in}}%
\pgfpathlineto{\pgfqpoint{4.700364in}{0.989829in}}%
\pgfpathlineto{\pgfqpoint{4.689055in}{0.944345in}}%
\pgfpathlineto{\pgfqpoint{4.676881in}{0.904394in}}%
\pgfpathlineto{\pgfqpoint{4.676095in}{0.902073in}}%
\pgfpathlineto{\pgfqpoint{4.676095in}{0.902073in}}%
\pgfusepath{stroke}%
\end{pgfscope}%
\begin{pgfscope}%
\pgfpathrectangle{\pgfqpoint{0.448634in}{0.402556in}}{\pgfqpoint{4.350661in}{2.489204in}} %
\pgfusepath{clip}%
\pgfsetrectcap%
\pgfsetroundjoin%
\pgfsetlinewidth{1.003750pt}%
\definecolor{currentstroke}{rgb}{0.172549,0.627451,0.172549}%
\pgfsetstrokecolor{currentstroke}%
\pgfsetdash{}{0pt}%
\pgfpathmoveto{\pgfqpoint{2.795520in}{1.982745in}}%
\pgfpathlineto{\pgfqpoint{2.781780in}{1.874357in}}%
\pgfpathlineto{\pgfqpoint{2.769351in}{1.758234in}}%
\pgfpathlineto{\pgfqpoint{2.758095in}{1.631942in}}%
\pgfpathlineto{\pgfqpoint{2.747786in}{1.490551in}}%
\pgfpathlineto{\pgfqpoint{2.738644in}{1.334082in}}%
\pgfpathlineto{\pgfqpoint{2.730580in}{1.157591in}}%
\pgfpathlineto{\pgfqpoint{2.723334in}{0.948663in}}%
\pgfpathlineto{\pgfqpoint{2.709783in}{0.530788in}}%
\pgfpathlineto{\pgfqpoint{2.705868in}{0.488716in}}%
\pgfpathlineto{\pgfqpoint{2.701769in}{0.464281in}}%
\pgfpathlineto{\pgfqpoint{2.697021in}{0.447744in}}%
\pgfpathlineto{\pgfqpoint{2.691859in}{0.436812in}}%
\pgfpathlineto{\pgfqpoint{2.686245in}{0.429229in}}%
\pgfpathlineto{\pgfqpoint{2.679348in}{0.423188in}}%
\pgfpathlineto{\pgfqpoint{2.669540in}{0.417856in}}%
\pgfpathlineto{\pgfqpoint{2.656987in}{0.413810in}}%
\pgfpathlineto{\pgfqpoint{2.637654in}{0.410337in}}%
\pgfpathlineto{\pgfqpoint{2.607297in}{0.407617in}}%
\pgfpathlineto{\pgfqpoint{2.555121in}{0.405574in}}%
\pgfpathlineto{\pgfqpoint{2.450714in}{0.404139in}}%
\pgfpathlineto{\pgfqpoint{2.176624in}{0.403275in}}%
\pgfpathlineto{\pgfqpoint{1.130290in}{0.402953in}}%
\pgfpathlineto{\pgfqpoint{0.516849in}{0.404175in}}%
\pgfpathlineto{\pgfqpoint{0.466848in}{0.405970in}}%
\pgfpathlineto{\pgfqpoint{0.456130in}{0.407931in}}%
\pgfpathlineto{\pgfqpoint{0.452340in}{0.410303in}}%
\pgfpathlineto{\pgfqpoint{0.450346in}{0.414662in}}%
\pgfpathlineto{\pgfqpoint{0.449266in}{0.424524in}}%
\pgfpathlineto{\pgfqpoint{0.448771in}{0.464344in}}%
\pgfpathlineto{\pgfqpoint{0.448640in}{0.850171in}}%
\pgfpathlineto{\pgfqpoint{0.448679in}{2.891318in}}%
\pgfpathlineto{\pgfqpoint{0.448679in}{2.891318in}}%
\pgfusepath{stroke}%
\end{pgfscope}%
\begin{pgfscope}%
\pgfpathrectangle{\pgfqpoint{0.448634in}{0.402556in}}{\pgfqpoint{4.350661in}{2.489204in}} %
\pgfusepath{clip}%
\pgfsetrectcap%
\pgfsetroundjoin%
\pgfsetlinewidth{1.003750pt}%
\definecolor{currentstroke}{rgb}{0.172549,0.627451,0.172549}%
\pgfsetstrokecolor{currentstroke}%
\pgfsetdash{}{0pt}%
\pgfpathmoveto{\pgfqpoint{3.428189in}{0.402586in}}%
\pgfpathlineto{\pgfqpoint{2.782121in}{0.403701in}}%
\pgfpathlineto{\pgfqpoint{2.753906in}{0.405674in}}%
\pgfpathlineto{\pgfqpoint{2.743328in}{0.408443in}}%
\pgfpathlineto{\pgfqpoint{2.737717in}{0.412188in}}%
\pgfpathlineto{\pgfqpoint{2.733668in}{0.417995in}}%
\pgfpathlineto{\pgfqpoint{2.730649in}{0.427307in}}%
\pgfpathlineto{\pgfqpoint{2.728388in}{0.442004in}}%
\pgfpathlineto{\pgfqpoint{2.726544in}{0.471794in}}%
\pgfpathlineto{\pgfqpoint{2.725216in}{0.534003in}}%
\pgfpathlineto{\pgfqpoint{2.725169in}{0.655973in}}%
\pgfpathlineto{\pgfqpoint{2.727377in}{0.832687in}}%
\pgfpathlineto{\pgfqpoint{2.732259in}{1.041703in}}%
\pgfpathlineto{\pgfqpoint{2.738851in}{1.223257in}}%
\pgfpathlineto{\pgfqpoint{2.747078in}{1.389766in}}%
\pgfpathlineto{\pgfqpoint{2.756608in}{1.538717in}}%
\pgfpathlineto{\pgfqpoint{2.768955in}{1.694887in}}%
\pgfpathlineto{\pgfqpoint{2.781228in}{1.816044in}}%
\pgfpathlineto{\pgfqpoint{2.794401in}{1.924524in}}%
\pgfpathlineto{\pgfqpoint{2.812737in}{2.054722in}}%
\pgfpathlineto{\pgfqpoint{2.828774in}{2.147512in}}%
\pgfpathlineto{\pgfqpoint{2.847382in}{2.242224in}}%
\pgfpathlineto{\pgfqpoint{2.895818in}{2.479699in}}%
\pgfpathlineto{\pgfqpoint{2.900204in}{2.516689in}}%
\pgfpathlineto{\pgfqpoint{2.901346in}{2.544029in}}%
\pgfpathlineto{\pgfqpoint{2.900291in}{2.566388in}}%
\pgfpathlineto{\pgfqpoint{2.897334in}{2.585999in}}%
\pgfpathlineto{\pgfqpoint{2.892836in}{2.602633in}}%
\pgfpathlineto{\pgfqpoint{2.886394in}{2.618405in}}%
\pgfpathlineto{\pgfqpoint{2.878058in}{2.632969in}}%
\pgfpathlineto{\pgfqpoint{2.868065in}{2.646100in}}%
\pgfpathlineto{\pgfqpoint{2.855050in}{2.659300in}}%
\pgfpathlineto{\pgfqpoint{2.840801in}{2.670717in}}%
\pgfpathlineto{\pgfqpoint{2.821822in}{2.682861in}}%
\pgfpathlineto{\pgfqpoint{2.799980in}{2.694026in}}%
\pgfpathlineto{\pgfqpoint{2.773366in}{2.704944in}}%
\pgfpathlineto{\pgfqpoint{2.742012in}{2.715266in}}%
\pgfpathlineto{\pgfqpoint{2.705983in}{2.724785in}}%
\pgfpathlineto{\pgfqpoint{2.663200in}{2.733810in}}%
\pgfpathlineto{\pgfqpoint{2.611535in}{2.742379in}}%
\pgfpathlineto{\pgfqpoint{2.551002in}{2.750090in}}%
\pgfpathlineto{\pgfqpoint{2.481632in}{2.756682in}}%
\pgfpathlineto{\pgfqpoint{2.399112in}{2.762200in}}%
\pgfpathlineto{\pgfqpoint{2.309985in}{2.765886in}}%
\pgfpathlineto{\pgfqpoint{2.188184in}{2.768096in}}%
\pgfpathlineto{\pgfqpoint{2.081595in}{2.767619in}}%
\pgfpathlineto{\pgfqpoint{1.968506in}{2.764840in}}%
\pgfpathlineto{\pgfqpoint{1.864180in}{2.759918in}}%
\pgfpathlineto{\pgfqpoint{1.757786in}{2.752593in}}%
\pgfpathlineto{\pgfqpoint{1.671087in}{2.744171in}}%
\pgfpathlineto{\pgfqpoint{1.591076in}{2.734193in}}%
\pgfpathlineto{\pgfqpoint{1.502689in}{2.720717in}}%
\pgfpathlineto{\pgfqpoint{1.427655in}{2.706083in}}%
\pgfpathlineto{\pgfqpoint{1.372350in}{2.692544in}}%
\pgfpathlineto{\pgfqpoint{1.321734in}{2.677921in}}%
\pgfpathlineto{\pgfqpoint{1.273765in}{2.661664in}}%
\pgfpathlineto{\pgfqpoint{1.230567in}{2.644672in}}%
\pgfpathlineto{\pgfqpoint{1.192197in}{2.627106in}}%
\pgfpathlineto{\pgfqpoint{1.156620in}{2.608403in}}%
\pgfpathlineto{\pgfqpoint{1.123890in}{2.588716in}}%
\pgfpathlineto{\pgfqpoint{1.095883in}{2.569568in}}%
\pgfpathlineto{\pgfqpoint{1.063936in}{2.543701in}}%
\pgfpathlineto{\pgfqpoint{1.038217in}{2.520732in}}%
\pgfpathlineto{\pgfqpoint{1.013766in}{2.496016in}}%
\pgfpathlineto{\pgfqpoint{0.990704in}{2.469610in}}%
\pgfpathlineto{\pgfqpoint{0.969124in}{2.441612in}}%
\pgfpathlineto{\pgfqpoint{0.949083in}{2.412154in}}%
\pgfpathlineto{\pgfqpoint{0.930604in}{2.381387in}}%
\pgfpathlineto{\pgfqpoint{0.906555in}{2.334052in}}%
\pgfpathlineto{\pgfqpoint{0.889925in}{2.296262in}}%
\pgfpathlineto{\pgfqpoint{0.874241in}{2.255213in}}%
\pgfpathlineto{\pgfqpoint{0.859667in}{2.210961in}}%
\pgfpathlineto{\pgfqpoint{0.846986in}{2.165954in}}%
\pgfpathlineto{\pgfqpoint{0.839633in}{2.134715in}}%
\pgfpathlineto{\pgfqpoint{0.828238in}{2.081532in}}%
\pgfpathlineto{\pgfqpoint{0.817866in}{2.022986in}}%
\pgfpathlineto{\pgfqpoint{0.810784in}{1.971352in}}%
\pgfpathlineto{\pgfqpoint{0.802846in}{1.902252in}}%
\pgfpathlineto{\pgfqpoint{0.796554in}{1.827927in}}%
\pgfpathlineto{\pgfqpoint{0.791696in}{1.743480in}}%
\pgfpathlineto{\pgfqpoint{0.787773in}{1.621595in}}%
\pgfpathlineto{\pgfqpoint{0.785408in}{1.522064in}}%
\pgfpathlineto{\pgfqpoint{0.785408in}{1.522064in}}%
\pgfusepath{stroke}%
\end{pgfscope}%
\begin{pgfscope}%
\pgfpathrectangle{\pgfqpoint{0.448634in}{0.402556in}}{\pgfqpoint{4.350661in}{2.489204in}} %
\pgfusepath{clip}%
\pgfsetrectcap%
\pgfsetroundjoin%
\pgfsetlinewidth{1.003750pt}%
\definecolor{currentstroke}{rgb}{0.839216,0.152941,0.156863}%
\pgfsetstrokecolor{currentstroke}%
\pgfsetdash{}{0pt}%
\pgfpathmoveto{\pgfqpoint{0.448634in}{2.896245in}}%
\pgfpathlineto{\pgfqpoint{0.448593in}{0.407043in}}%
\pgfpathlineto{\pgfqpoint{0.448593in}{0.407043in}}%
\pgfusepath{stroke}%
\end{pgfscope}%
\begin{pgfscope}%
\pgfpathrectangle{\pgfqpoint{0.448634in}{0.402556in}}{\pgfqpoint{4.350661in}{2.489204in}} %
\pgfusepath{clip}%
\pgfsetrectcap%
\pgfsetroundjoin%
\pgfsetlinewidth{1.003750pt}%
\definecolor{currentstroke}{rgb}{0.839216,0.152941,0.156863}%
\pgfsetstrokecolor{currentstroke}%
\pgfsetdash{}{0pt}%
\pgfpathmoveto{\pgfqpoint{0.576852in}{1.760819in}}%
\pgfpathlineto{\pgfqpoint{0.569393in}{1.840012in}}%
\pgfpathlineto{\pgfqpoint{0.563208in}{1.929340in}}%
\pgfpathlineto{\pgfqpoint{0.558592in}{2.028766in}}%
\pgfpathlineto{\pgfqpoint{0.555985in}{2.133267in}}%
\pgfpathlineto{\pgfqpoint{0.555565in}{2.237810in}}%
\pgfpathlineto{\pgfqpoint{0.557371in}{2.337354in}}%
\pgfpathlineto{\pgfqpoint{0.561095in}{2.424368in}}%
\pgfpathlineto{\pgfqpoint{0.566403in}{2.498793in}}%
\pgfpathlineto{\pgfqpoint{0.572908in}{2.560572in}}%
\pgfpathlineto{\pgfqpoint{0.580458in}{2.612121in}}%
\pgfpathlineto{\pgfqpoint{0.589086in}{2.655818in}}%
\pgfpathlineto{\pgfqpoint{0.598406in}{2.691591in}}%
\pgfpathlineto{\pgfqpoint{0.608613in}{2.721759in}}%
\pgfpathlineto{\pgfqpoint{0.619241in}{2.746280in}}%
\pgfpathlineto{\pgfqpoint{0.630817in}{2.767341in}}%
\pgfpathlineto{\pgfqpoint{0.642976in}{2.784885in}}%
\pgfpathlineto{\pgfqpoint{0.656814in}{2.800714in}}%
\pgfpathlineto{\pgfqpoint{0.672197in}{2.814550in}}%
\pgfpathlineto{\pgfqpoint{0.688854in}{2.826302in}}%
\pgfpathlineto{\pgfqpoint{0.706462in}{2.836077in}}%
\pgfpathlineto{\pgfqpoint{0.726805in}{2.844876in}}%
\pgfpathlineto{\pgfqpoint{0.751867in}{2.853203in}}%
\pgfpathlineto{\pgfqpoint{0.781633in}{2.860548in}}%
\pgfpathlineto{\pgfqpoint{0.818169in}{2.867054in}}%
\pgfpathlineto{\pgfqpoint{0.863582in}{2.872685in}}%
\pgfpathlineto{\pgfqpoint{0.922162in}{2.877518in}}%
\pgfpathlineto{\pgfqpoint{1.000392in}{2.881567in}}%
\pgfpathlineto{\pgfqpoint{1.111295in}{2.884881in}}%
\pgfpathlineto{\pgfqpoint{1.274429in}{2.887367in}}%
\pgfpathlineto{\pgfqpoint{1.552866in}{2.889263in}}%
\pgfpathlineto{\pgfqpoint{2.107574in}{2.890457in}}%
\pgfpathlineto{\pgfqpoint{3.343162in}{2.890573in}}%
\pgfpathlineto{\pgfqpoint{4.043616in}{2.888941in}}%
\pgfpathlineto{\pgfqpoint{4.289418in}{2.886404in}}%
\pgfpathlineto{\pgfqpoint{4.413376in}{2.883093in}}%
\pgfpathlineto{\pgfqpoint{4.489426in}{2.878997in}}%
\pgfpathlineto{\pgfqpoint{4.541452in}{2.874081in}}%
\pgfpathlineto{\pgfqpoint{4.578101in}{2.868470in}}%
\pgfpathlineto{\pgfqpoint{4.605820in}{2.862092in}}%
\pgfpathlineto{\pgfqpoint{4.626727in}{2.855245in}}%
\pgfpathlineto{\pgfqpoint{4.644926in}{2.847018in}}%
\pgfpathlineto{\pgfqpoint{4.660242in}{2.837589in}}%
\pgfpathlineto{\pgfqpoint{4.672624in}{2.827468in}}%
\pgfpathlineto{\pgfqpoint{4.683752in}{2.815592in}}%
\pgfpathlineto{\pgfqpoint{4.693407in}{2.802134in}}%
\pgfpathlineto{\pgfqpoint{4.702741in}{2.785342in}}%
\pgfpathlineto{\pgfqpoint{4.711278in}{2.765193in}}%
\pgfpathlineto{\pgfqpoint{4.719483in}{2.739483in}}%
\pgfpathlineto{\pgfqpoint{4.726294in}{2.710657in}}%
\pgfpathlineto{\pgfqpoint{4.733260in}{2.671642in}}%
\pgfpathlineto{\pgfqpoint{4.739604in}{2.622395in}}%
\pgfpathlineto{\pgfqpoint{4.745236in}{2.560503in}}%
\pgfpathlineto{\pgfqpoint{4.750164in}{2.481051in}}%
\pgfpathlineto{\pgfqpoint{4.754367in}{2.376617in}}%
\pgfpathlineto{\pgfqpoint{4.757443in}{2.242248in}}%
\pgfpathlineto{\pgfqpoint{4.758977in}{2.075482in}}%
\pgfpathlineto{\pgfqpoint{4.758447in}{1.888794in}}%
\pgfpathlineto{\pgfqpoint{4.755756in}{1.707110in}}%
\pgfpathlineto{\pgfqpoint{4.750925in}{1.532956in}}%
\pgfpathlineto{\pgfqpoint{4.744786in}{1.398726in}}%
\pgfpathlineto{\pgfqpoint{4.737575in}{1.289515in}}%
\pgfpathlineto{\pgfqpoint{4.728714in}{1.190469in}}%
\pgfpathlineto{\pgfqpoint{4.719653in}{1.116520in}}%
\pgfpathlineto{\pgfqpoint{4.710036in}{1.055275in}}%
\pgfpathlineto{\pgfqpoint{4.699504in}{1.001860in}}%
\pgfpathlineto{\pgfqpoint{4.689040in}{0.958689in}}%
\pgfpathlineto{\pgfqpoint{4.677220in}{0.918599in}}%
\pgfpathlineto{\pgfqpoint{4.664034in}{0.881748in}}%
\pgfpathlineto{\pgfqpoint{4.650584in}{0.850491in}}%
\pgfpathlineto{\pgfqpoint{4.636303in}{0.822569in}}%
\pgfpathlineto{\pgfqpoint{4.620207in}{0.795974in}}%
\pgfpathlineto{\pgfqpoint{4.603640in}{0.772901in}}%
\pgfpathlineto{\pgfqpoint{4.585488in}{0.751445in}}%
\pgfpathlineto{\pgfqpoint{4.565874in}{0.731748in}}%
\pgfpathlineto{\pgfqpoint{4.544964in}{0.713878in}}%
\pgfpathlineto{\pgfqpoint{4.522958in}{0.697823in}}%
\pgfpathlineto{\pgfqpoint{4.496157in}{0.681289in}}%
\pgfpathlineto{\pgfqpoint{4.470397in}{0.667952in}}%
\pgfpathlineto{\pgfqpoint{4.439961in}{0.654509in}}%
\pgfpathlineto{\pgfqpoint{4.406841in}{0.642281in}}%
\pgfpathlineto{\pgfqpoint{4.369009in}{0.630748in}}%
\pgfpathlineto{\pgfqpoint{4.326489in}{0.620226in}}%
\pgfpathlineto{\pgfqpoint{4.279327in}{0.610948in}}%
\pgfpathlineto{\pgfqpoint{4.227576in}{0.603084in}}%
\pgfpathlineto{\pgfqpoint{4.173450in}{0.597062in}}%
\pgfpathlineto{\pgfqpoint{4.110511in}{0.592202in}}%
\pgfpathlineto{\pgfqpoint{4.047471in}{0.589536in}}%
\pgfpathlineto{\pgfqpoint{3.977867in}{0.588623in}}%
\pgfpathlineto{\pgfqpoint{3.906093in}{0.589933in}}%
\pgfpathlineto{\pgfqpoint{3.834377in}{0.593496in}}%
\pgfpathlineto{\pgfqpoint{3.767120in}{0.599067in}}%
\pgfpathlineto{\pgfqpoint{3.704364in}{0.606392in}}%
\pgfpathlineto{\pgfqpoint{3.678516in}{0.610510in}}%
\pgfpathlineto{\pgfqpoint{3.620438in}{0.620500in}}%
\pgfpathlineto{\pgfqpoint{3.586319in}{0.628207in}}%
\pgfpathlineto{\pgfqpoint{3.495241in}{0.652428in}}%
\pgfpathlineto{\pgfqpoint{3.451528in}{0.667583in}}%
\pgfpathlineto{\pgfqpoint{3.408538in}{0.685220in}}%
\pgfpathlineto{\pgfqpoint{3.374594in}{0.702000in}}%
\pgfpathlineto{\pgfqpoint{3.345407in}{0.718682in}}%
\pgfpathlineto{\pgfqpoint{3.315236in}{0.738520in}}%
\pgfpathlineto{\pgfqpoint{3.288127in}{0.759289in}}%
\pgfpathlineto{\pgfqpoint{3.264004in}{0.780550in}}%
\pgfpathlineto{\pgfqpoint{3.241208in}{0.803648in}}%
\pgfpathlineto{\pgfqpoint{3.219894in}{0.828529in}}%
\pgfpathlineto{\pgfqpoint{3.200189in}{0.855091in}}%
\pgfpathlineto{\pgfqpoint{3.182177in}{0.883182in}}%
\pgfpathlineto{\pgfqpoint{3.165906in}{0.912633in}}%
\pgfpathlineto{\pgfqpoint{3.150351in}{0.945447in}}%
\pgfpathlineto{\pgfqpoint{3.136682in}{0.979344in}}%
\pgfpathlineto{\pgfqpoint{3.124073in}{1.016460in}}%
\pgfpathlineto{\pgfqpoint{3.112834in}{1.056769in}}%
\pgfpathlineto{\pgfqpoint{3.103046in}{1.100146in}}%
\pgfpathlineto{\pgfqpoint{3.095343in}{1.144071in}}%
\pgfpathlineto{\pgfqpoint{3.089208in}{1.190837in}}%
\pgfpathlineto{\pgfqpoint{3.084595in}{1.242838in}}%
\pgfpathlineto{\pgfqpoint{3.082136in}{1.295031in}}%
\pgfpathlineto{\pgfqpoint{3.081687in}{1.349786in}}%
\pgfpathlineto{\pgfqpoint{3.083450in}{1.406998in}}%
\pgfpathlineto{\pgfqpoint{3.087181in}{1.461589in}}%
\pgfpathlineto{\pgfqpoint{3.093485in}{1.520887in}}%
\pgfpathlineto{\pgfqpoint{3.101823in}{1.577334in}}%
\pgfpathlineto{\pgfqpoint{3.111930in}{1.630856in}}%
\pgfpathlineto{\pgfqpoint{3.124690in}{1.686208in}}%
\pgfpathlineto{\pgfqpoint{3.139178in}{1.738395in}}%
\pgfpathlineto{\pgfqpoint{3.155145in}{1.787365in}}%
\pgfpathlineto{\pgfqpoint{3.172353in}{1.833084in}}%
\pgfpathlineto{\pgfqpoint{3.191618in}{1.877716in}}%
\pgfpathlineto{\pgfqpoint{3.214025in}{1.923261in}}%
\pgfpathlineto{\pgfqpoint{3.236214in}{1.963157in}}%
\pgfpathlineto{\pgfqpoint{3.260177in}{2.001684in}}%
\pgfpathlineto{\pgfqpoint{3.285813in}{2.038776in}}%
\pgfpathlineto{\pgfqpoint{3.314414in}{2.076285in}}%
\pgfpathlineto{\pgfqpoint{3.348944in}{2.117711in}}%
\pgfpathlineto{\pgfqpoint{3.417133in}{2.198022in}}%
\pgfpathlineto{\pgfqpoint{3.426053in}{2.212128in}}%
\pgfpathlineto{\pgfqpoint{3.430798in}{2.223297in}}%
\pgfpathlineto{\pgfqpoint{3.432034in}{2.230603in}}%
\pgfpathlineto{\pgfqpoint{3.430772in}{2.237856in}}%
\pgfpathlineto{\pgfqpoint{3.426621in}{2.243526in}}%
\pgfpathlineto{\pgfqpoint{3.420908in}{2.247084in}}%
\pgfpathlineto{\pgfqpoint{3.412501in}{2.249583in}}%
\pgfpathlineto{\pgfqpoint{3.399499in}{2.250689in}}%
\pgfpathlineto{\pgfqpoint{3.384305in}{2.249671in}}%
\pgfpathlineto{\pgfqpoint{3.364985in}{2.246098in}}%
\pgfpathlineto{\pgfqpoint{3.341804in}{2.239342in}}%
\pgfpathlineto{\pgfqpoint{3.317109in}{2.229682in}}%
\pgfpathlineto{\pgfqpoint{3.291104in}{2.216986in}}%
\pgfpathlineto{\pgfqpoint{3.265928in}{2.202261in}}%
\pgfpathlineto{\pgfqpoint{3.239805in}{2.184361in}}%
\pgfpathlineto{\pgfqpoint{3.214775in}{2.164519in}}%
\pgfpathlineto{\pgfqpoint{3.190900in}{2.142893in}}%
\pgfpathlineto{\pgfqpoint{3.166656in}{2.117912in}}%
\pgfpathlineto{\pgfqpoint{3.143835in}{2.091233in}}%
\pgfpathlineto{\pgfqpoint{3.121079in}{2.061107in}}%
\pgfpathlineto{\pgfqpoint{3.099952in}{2.029463in}}%
\pgfpathlineto{\pgfqpoint{3.079250in}{1.994406in}}%
\pgfpathlineto{\pgfqpoint{3.059218in}{1.955915in}}%
\pgfpathlineto{\pgfqpoint{3.040058in}{1.914015in}}%
\pgfpathlineto{\pgfqpoint{3.022809in}{1.871041in}}%
\pgfpathlineto{\pgfqpoint{3.005790in}{1.822536in}}%
\pgfpathlineto{\pgfqpoint{2.990067in}{1.770819in}}%
\pgfpathlineto{\pgfqpoint{2.975708in}{1.715979in}}%
\pgfpathlineto{\pgfqpoint{2.962284in}{1.655680in}}%
\pgfpathlineto{\pgfqpoint{2.950496in}{1.592386in}}%
\pgfpathlineto{\pgfqpoint{2.940383in}{1.526185in}}%
\pgfpathlineto{\pgfqpoint{2.931745in}{1.454681in}}%
\pgfpathlineto{\pgfqpoint{2.925082in}{1.380399in}}%
\pgfpathlineto{\pgfqpoint{2.920647in}{1.305899in}}%
\pgfpathlineto{\pgfqpoint{2.918444in}{1.231270in}}%
\pgfpathlineto{\pgfqpoint{2.918545in}{1.159087in}}%
\pgfpathlineto{\pgfqpoint{2.920787in}{1.091931in}}%
\pgfpathlineto{\pgfqpoint{2.925177in}{1.027412in}}%
\pgfpathlineto{\pgfqpoint{2.931192in}{0.970580in}}%
\pgfpathlineto{\pgfqpoint{2.938760in}{0.919034in}}%
\pgfpathlineto{\pgfqpoint{2.947651in}{0.872852in}}%
\pgfpathlineto{\pgfqpoint{2.958213in}{0.829714in}}%
\pgfpathlineto{\pgfqpoint{2.969670in}{0.792114in}}%
\pgfpathlineto{\pgfqpoint{2.982463in}{0.757773in}}%
\pgfpathlineto{\pgfqpoint{2.996425in}{0.726812in}}%
\pgfpathlineto{\pgfqpoint{3.011299in}{0.699300in}}%
\pgfpathlineto{\pgfqpoint{3.026739in}{0.675225in}}%
\pgfpathlineto{\pgfqpoint{3.043828in}{0.652656in}}%
\pgfpathlineto{\pgfqpoint{3.062495in}{0.631788in}}%
\pgfpathlineto{\pgfqpoint{3.082602in}{0.612753in}}%
\pgfpathlineto{\pgfqpoint{3.103961in}{0.595592in}}%
\pgfpathlineto{\pgfqpoint{3.128268in}{0.579069in}}%
\pgfpathlineto{\pgfqpoint{3.153537in}{0.564554in}}%
\pgfpathlineto{\pgfqpoint{3.181571in}{0.550952in}}%
\pgfpathlineto{\pgfqpoint{3.214371in}{0.537647in}}%
\pgfpathlineto{\pgfqpoint{3.249846in}{0.525712in}}%
\pgfpathlineto{\pgfqpoint{3.290011in}{0.514571in}}%
\pgfpathlineto{\pgfqpoint{3.334820in}{0.504423in}}%
\pgfpathlineto{\pgfqpoint{3.386372in}{0.494999in}}%
\pgfpathlineto{\pgfqpoint{3.446798in}{0.486257in}}%
\pgfpathlineto{\pgfqpoint{3.518243in}{0.478282in}}%
\pgfpathlineto{\pgfqpoint{3.600685in}{0.471409in}}%
\pgfpathlineto{\pgfqpoint{3.696268in}{0.465713in}}%
\pgfpathlineto{\pgfqpoint{3.807144in}{0.461369in}}%
\pgfpathlineto{\pgfqpoint{3.933291in}{0.458719in}}%
\pgfpathlineto{\pgfqpoint{4.063808in}{0.458211in}}%
\pgfpathlineto{\pgfqpoint{4.187792in}{0.459914in}}%
\pgfpathlineto{\pgfqpoint{4.294335in}{0.463521in}}%
\pgfpathlineto{\pgfqpoint{4.381234in}{0.468574in}}%
\pgfpathlineto{\pgfqpoint{4.450636in}{0.474701in}}%
\pgfpathlineto{\pgfqpoint{4.506850in}{0.481799in}}%
\pgfpathlineto{\pgfqpoint{4.552009in}{0.489658in}}%
\pgfpathlineto{\pgfqpoint{4.588239in}{0.498115in}}%
\pgfpathlineto{\pgfqpoint{4.617656in}{0.507110in}}%
\pgfpathlineto{\pgfqpoint{4.642328in}{0.516843in}}%
\pgfpathlineto{\pgfqpoint{4.664194in}{0.527940in}}%
\pgfpathlineto{\pgfqpoint{4.681238in}{0.538945in}}%
\pgfpathlineto{\pgfqpoint{4.697164in}{0.551954in}}%
\pgfpathlineto{\pgfqpoint{4.710076in}{0.565289in}}%
\pgfpathlineto{\pgfqpoint{4.721578in}{0.580218in}}%
\pgfpathlineto{\pgfqpoint{4.731557in}{0.596521in}}%
\pgfpathlineto{\pgfqpoint{4.741000in}{0.616134in}}%
\pgfpathlineto{\pgfqpoint{4.749521in}{0.639027in}}%
\pgfpathlineto{\pgfqpoint{4.757522in}{0.667450in}}%
\pgfpathlineto{\pgfqpoint{4.764572in}{0.701345in}}%
\pgfpathlineto{\pgfqpoint{4.770840in}{0.743043in}}%
\pgfpathlineto{\pgfqpoint{4.776327in}{0.794934in}}%
\pgfpathlineto{\pgfqpoint{4.781278in}{0.864398in}}%
\pgfpathlineto{\pgfqpoint{4.785468in}{0.956371in}}%
\pgfpathlineto{\pgfqpoint{4.789000in}{1.085745in}}%
\pgfpathlineto{\pgfqpoint{4.791852in}{1.277385in}}%
\pgfpathlineto{\pgfqpoint{4.793959in}{1.581058in}}%
\pgfpathlineto{\pgfqpoint{4.794962in}{2.071429in}}%
\pgfpathlineto{\pgfqpoint{4.793967in}{2.559311in}}%
\pgfpathlineto{\pgfqpoint{4.791733in}{2.745981in}}%
\pgfpathlineto{\pgfqpoint{4.788955in}{2.818091in}}%
\pgfpathlineto{\pgfqpoint{4.785731in}{2.850227in}}%
\pgfpathlineto{\pgfqpoint{4.781879in}{2.867057in}}%
\pgfpathlineto{\pgfqpoint{4.777744in}{2.875780in}}%
\pgfpathlineto{\pgfqpoint{4.773097in}{2.880982in}}%
\pgfpathlineto{\pgfqpoint{4.767363in}{2.884504in}}%
\pgfpathlineto{\pgfqpoint{4.756853in}{2.887622in}}%
\pgfpathlineto{\pgfqpoint{4.739548in}{2.889639in}}%
\pgfpathlineto{\pgfqpoint{4.704762in}{2.890882in}}%
\pgfpathlineto{\pgfqpoint{4.602524in}{2.891538in}}%
\pgfpathlineto{\pgfqpoint{3.952100in}{2.891742in}}%
\pgfpathlineto{\pgfqpoint{0.617321in}{2.890753in}}%
\pgfpathlineto{\pgfqpoint{0.549910in}{2.888858in}}%
\pgfpathlineto{\pgfqpoint{0.521735in}{2.886179in}}%
\pgfpathlineto{\pgfqpoint{0.504666in}{2.882389in}}%
\pgfpathlineto{\pgfqpoint{0.494501in}{2.878011in}}%
\pgfpathlineto{\pgfqpoint{0.487180in}{2.872667in}}%
\pgfpathlineto{\pgfqpoint{0.481152in}{2.865519in}}%
\pgfpathlineto{\pgfqpoint{0.475664in}{2.854804in}}%
\pgfpathlineto{\pgfqpoint{0.471318in}{2.840737in}}%
\pgfpathlineto{\pgfqpoint{0.467301in}{2.818823in}}%
\pgfpathlineto{\pgfqpoint{0.463927in}{2.786700in}}%
\pgfpathlineto{\pgfqpoint{0.460918in}{2.734544in}}%
\pgfpathlineto{\pgfqpoint{0.458363in}{2.647473in}}%
\pgfpathlineto{\pgfqpoint{0.456575in}{2.523031in}}%
\pgfpathlineto{\pgfqpoint{0.456575in}{2.523031in}}%
\pgfusepath{stroke}%
\end{pgfscope}%
\begin{pgfscope}%
\pgfpathrectangle{\pgfqpoint{0.448634in}{0.402556in}}{\pgfqpoint{4.350661in}{2.489204in}} %
\pgfusepath{clip}%
\pgfsetrectcap%
\pgfsetroundjoin%
\pgfsetlinewidth{1.003750pt}%
\definecolor{currentstroke}{rgb}{0.839216,0.152941,0.156863}%
\pgfsetstrokecolor{currentstroke}%
\pgfsetdash{}{0pt}%
\pgfpathmoveto{\pgfqpoint{4.798840in}{2.852369in}}%
\pgfpathlineto{\pgfqpoint{4.797564in}{2.889610in}}%
\pgfpathlineto{\pgfqpoint{4.796215in}{2.891483in}}%
\pgfpathlineto{\pgfqpoint{4.787551in}{2.891760in}}%
\pgfpathlineto{\pgfqpoint{0.452128in}{2.891659in}}%
\pgfpathlineto{\pgfqpoint{0.450530in}{2.890082in}}%
\pgfpathlineto{\pgfqpoint{0.449454in}{2.882763in}}%
\pgfpathlineto{\pgfqpoint{0.448970in}{2.845432in}}%
\pgfpathlineto{\pgfqpoint{0.448743in}{2.494454in}}%
\pgfpathlineto{\pgfqpoint{0.449624in}{0.615107in}}%
\pgfpathlineto{\pgfqpoint{0.451433in}{0.510586in}}%
\pgfpathlineto{\pgfqpoint{0.453993in}{0.473374in}}%
\pgfpathlineto{\pgfqpoint{0.457406in}{0.453868in}}%
\pgfpathlineto{\pgfqpoint{0.461540in}{0.442384in}}%
\pgfpathlineto{\pgfqpoint{0.466739in}{0.434437in}}%
\pgfpathlineto{\pgfqpoint{0.473595in}{0.428350in}}%
\pgfpathlineto{\pgfqpoint{0.483492in}{0.423244in}}%
\pgfpathlineto{\pgfqpoint{0.491854in}{0.420501in}}%
\pgfpathlineto{\pgfqpoint{0.491854in}{0.420501in}}%
\pgfusepath{stroke}%
\end{pgfscope}%
\begin{pgfscope}%
\pgfpathrectangle{\pgfqpoint{0.448634in}{0.402556in}}{\pgfqpoint{4.350661in}{2.489204in}} %
\pgfusepath{clip}%
\pgfsetrectcap%
\pgfsetroundjoin%
\pgfsetlinewidth{1.003750pt}%
\definecolor{currentstroke}{rgb}{0.839216,0.152941,0.156863}%
\pgfsetstrokecolor{currentstroke}%
\pgfsetdash{}{0pt}%
\pgfpathmoveto{\pgfqpoint{0.456424in}{1.370138in}}%
\pgfpathlineto{\pgfqpoint{0.459610in}{1.118756in}}%
\pgfpathlineto{\pgfqpoint{0.463695in}{0.962008in}}%
\pgfpathlineto{\pgfqpoint{0.468519in}{0.857611in}}%
\pgfpathlineto{\pgfqpoint{0.474082in}{0.783211in}}%
\pgfpathlineto{\pgfqpoint{0.480226in}{0.728907in}}%
\pgfpathlineto{\pgfqpoint{0.486970in}{0.687307in}}%
\pgfpathlineto{\pgfqpoint{0.494537in}{0.653559in}}%
\pgfpathlineto{\pgfqpoint{0.503107in}{0.625356in}}%
\pgfpathlineto{\pgfqpoint{0.512193in}{0.602750in}}%
\pgfpathlineto{\pgfqpoint{0.522200in}{0.583508in}}%
\pgfpathlineto{\pgfqpoint{0.534107in}{0.565744in}}%
\pgfpathlineto{\pgfqpoint{0.546263in}{0.551508in}}%
\pgfpathlineto{\pgfqpoint{0.559728in}{0.538907in}}%
\pgfpathlineto{\pgfqpoint{0.576129in}{0.526694in}}%
\pgfpathlineto{\pgfqpoint{0.595483in}{0.515351in}}%
\pgfpathlineto{\pgfqpoint{0.617680in}{0.505147in}}%
\pgfpathlineto{\pgfqpoint{0.642567in}{0.496153in}}%
\pgfpathlineto{\pgfqpoint{0.672125in}{0.487778in}}%
\pgfpathlineto{\pgfqpoint{0.708443in}{0.479824in}}%
\pgfpathlineto{\pgfqpoint{0.753649in}{0.472325in}}%
\pgfpathlineto{\pgfqpoint{0.807717in}{0.465660in}}%
\pgfpathlineto{\pgfqpoint{0.877115in}{0.459475in}}%
\pgfpathlineto{\pgfqpoint{0.961828in}{0.454230in}}%
\pgfpathlineto{\pgfqpoint{1.068351in}{0.449916in}}%
\pgfpathlineto{\pgfqpoint{1.201018in}{0.446839in}}%
\pgfpathlineto{\pgfqpoint{1.357636in}{0.445481in}}%
\pgfpathlineto{\pgfqpoint{1.525134in}{0.446232in}}%
\pgfpathlineto{\pgfqpoint{1.686088in}{0.449142in}}%
\pgfpathlineto{\pgfqpoint{1.823073in}{0.453747in}}%
\pgfpathlineto{\pgfqpoint{1.938244in}{0.459764in}}%
\pgfpathlineto{\pgfqpoint{2.031581in}{0.466759in}}%
\pgfpathlineto{\pgfqpoint{2.109580in}{0.474745in}}%
\pgfpathlineto{\pgfqpoint{2.174383in}{0.483535in}}%
\pgfpathlineto{\pgfqpoint{2.228139in}{0.492940in}}%
\pgfpathlineto{\pgfqpoint{2.275118in}{0.503356in}}%
\pgfpathlineto{\pgfqpoint{2.315281in}{0.514501in}}%
\pgfpathlineto{\pgfqpoint{2.350697in}{0.526659in}}%
\pgfpathlineto{\pgfqpoint{2.381320in}{0.539535in}}%
\pgfpathlineto{\pgfqpoint{2.407164in}{0.552658in}}%
\pgfpathlineto{\pgfqpoint{2.430226in}{0.566639in}}%
\pgfpathlineto{\pgfqpoint{2.452281in}{0.582601in}}%
\pgfpathlineto{\pgfqpoint{2.471390in}{0.599069in}}%
\pgfpathlineto{\pgfqpoint{2.489240in}{0.617292in}}%
\pgfpathlineto{\pgfqpoint{2.505677in}{0.637180in}}%
\pgfpathlineto{\pgfqpoint{2.520620in}{0.658556in}}%
\pgfpathlineto{\pgfqpoint{2.535213in}{0.683313in}}%
\pgfpathlineto{\pgfqpoint{2.549115in}{0.711484in}}%
\pgfpathlineto{\pgfqpoint{2.562090in}{0.743003in}}%
\pgfpathlineto{\pgfqpoint{2.574020in}{0.777750in}}%
\pgfpathlineto{\pgfqpoint{2.585502in}{0.817969in}}%
\pgfpathlineto{\pgfqpoint{2.596809in}{0.866038in}}%
\pgfpathlineto{\pgfqpoint{2.607562in}{0.921947in}}%
\pgfpathlineto{\pgfqpoint{2.617924in}{0.988097in}}%
\pgfpathlineto{\pgfqpoint{2.627958in}{1.066917in}}%
\pgfpathlineto{\pgfqpoint{2.637941in}{1.163319in}}%
\pgfpathlineto{\pgfqpoint{2.648424in}{1.287198in}}%
\pgfpathlineto{\pgfqpoint{2.660103in}{1.453437in}}%
\pgfpathlineto{\pgfqpoint{2.674773in}{1.696800in}}%
\pgfpathlineto{\pgfqpoint{2.687716in}{1.945278in}}%
\pgfpathlineto{\pgfqpoint{2.692670in}{2.079573in}}%
\pgfpathlineto{\pgfqpoint{2.693829in}{2.166681in}}%
\pgfpathlineto{\pgfqpoint{2.692565in}{2.233869in}}%
\pgfpathlineto{\pgfqpoint{2.689436in}{2.286014in}}%
\pgfpathlineto{\pgfqpoint{2.684859in}{2.327999in}}%
\pgfpathlineto{\pgfqpoint{2.678725in}{2.364663in}}%
\pgfpathlineto{\pgfqpoint{2.671356in}{2.395896in}}%
\pgfpathlineto{\pgfqpoint{2.662489in}{2.423981in}}%
\pgfpathlineto{\pgfqpoint{2.652361in}{2.448778in}}%
\pgfpathlineto{\pgfqpoint{2.641365in}{2.470245in}}%
\pgfpathlineto{\pgfqpoint{2.628643in}{2.490424in}}%
\pgfpathlineto{\pgfqpoint{2.614279in}{2.509105in}}%
\pgfpathlineto{\pgfqpoint{2.598443in}{2.526159in}}%
\pgfpathlineto{\pgfqpoint{2.579590in}{2.543005in}}%
\pgfpathlineto{\pgfqpoint{2.559533in}{2.557923in}}%
\pgfpathlineto{\pgfqpoint{2.536602in}{2.572183in}}%
\pgfpathlineto{\pgfqpoint{2.510850in}{2.585538in}}%
\pgfpathlineto{\pgfqpoint{2.482360in}{2.597837in}}%
\pgfpathlineto{\pgfqpoint{2.449135in}{2.609683in}}%
\pgfpathlineto{\pgfqpoint{2.411185in}{2.620696in}}%
\pgfpathlineto{\pgfqpoint{2.368552in}{2.630606in}}%
\pgfpathlineto{\pgfqpoint{2.321294in}{2.639221in}}%
\pgfpathlineto{\pgfqpoint{2.269467in}{2.646399in}}%
\pgfpathlineto{\pgfqpoint{2.210955in}{2.652193in}}%
\pgfpathlineto{\pgfqpoint{2.147968in}{2.656153in}}%
\pgfpathlineto{\pgfqpoint{2.080557in}{2.658135in}}%
\pgfpathlineto{\pgfqpoint{2.010949in}{2.657971in}}%
\pgfpathlineto{\pgfqpoint{1.939195in}{2.655572in}}%
\pgfpathlineto{\pgfqpoint{1.867527in}{2.650913in}}%
\pgfpathlineto{\pgfqpoint{1.798171in}{2.644140in}}%
\pgfpathlineto{\pgfqpoint{1.733342in}{2.635606in}}%
\pgfpathlineto{\pgfqpoint{1.673076in}{2.625521in}}%
\pgfpathlineto{\pgfqpoint{1.615275in}{2.613610in}}%
\pgfpathlineto{\pgfqpoint{1.562134in}{2.600402in}}%
\pgfpathlineto{\pgfqpoint{1.513682in}{2.586139in}}%
\pgfpathlineto{\pgfqpoint{1.467862in}{2.570344in}}%
\pgfpathlineto{\pgfqpoint{1.426794in}{2.553923in}}%
\pgfpathlineto{\pgfqpoint{1.388447in}{2.536289in}}%
\pgfpathlineto{\pgfqpoint{1.352878in}{2.517566in}}%
\pgfpathlineto{\pgfqpoint{1.320129in}{2.497922in}}%
\pgfpathlineto{\pgfqpoint{1.288380in}{2.476236in}}%
\pgfpathlineto{\pgfqpoint{1.259592in}{2.453861in}}%
\pgfpathlineto{\pgfqpoint{1.232051in}{2.429520in}}%
\pgfpathlineto{\pgfqpoint{1.207527in}{2.404899in}}%
\pgfpathlineto{\pgfqpoint{1.184409in}{2.378558in}}%
\pgfpathlineto{\pgfqpoint{1.162828in}{2.350561in}}%
\pgfpathlineto{\pgfqpoint{1.142891in}{2.321012in}}%
\pgfpathlineto{\pgfqpoint{1.124675in}{2.290042in}}%
\pgfpathlineto{\pgfqpoint{1.108225in}{2.257803in}}%
\pgfpathlineto{\pgfqpoint{1.092639in}{2.222200in}}%
\pgfpathlineto{\pgfqpoint{1.079060in}{2.185536in}}%
\pgfpathlineto{\pgfqpoint{1.067443in}{2.147998in}}%
\pgfpathlineto{\pgfqpoint{1.057188in}{2.107348in}}%
\pgfpathlineto{\pgfqpoint{1.049004in}{2.066087in}}%
\pgfpathlineto{\pgfqpoint{1.042513in}{2.021907in}}%
\pgfpathlineto{\pgfqpoint{1.038177in}{1.977382in}}%
\pgfpathlineto{\pgfqpoint{1.035865in}{1.930168in}}%
\pgfpathlineto{\pgfqpoint{1.035826in}{1.882879in}}%
\pgfpathlineto{\pgfqpoint{1.038031in}{1.835657in}}%
\pgfpathlineto{\pgfqpoint{1.042474in}{1.788642in}}%
\pgfpathlineto{\pgfqpoint{1.049175in}{1.741979in}}%
\pgfpathlineto{\pgfqpoint{1.057644in}{1.698240in}}%
\pgfpathlineto{\pgfqpoint{1.068221in}{1.655106in}}%
\pgfpathlineto{\pgfqpoint{1.080962in}{1.612746in}}%
\pgfpathlineto{\pgfqpoint{1.095030in}{1.573617in}}%
\pgfpathlineto{\pgfqpoint{1.111115in}{1.535520in}}%
\pgfpathlineto{\pgfqpoint{1.128117in}{1.500775in}}%
\pgfpathlineto{\pgfqpoint{1.146929in}{1.467274in}}%
\pgfpathlineto{\pgfqpoint{1.167530in}{1.435181in}}%
\pgfpathlineto{\pgfqpoint{1.189873in}{1.404652in}}%
\pgfpathlineto{\pgfqpoint{1.213883in}{1.375828in}}%
\pgfpathlineto{\pgfqpoint{1.237816in}{1.350457in}}%
\pgfpathlineto{\pgfqpoint{1.264747in}{1.325238in}}%
\pgfpathlineto{\pgfqpoint{1.292990in}{1.301972in}}%
\pgfpathlineto{\pgfqpoint{1.322397in}{1.280678in}}%
\pgfpathlineto{\pgfqpoint{1.352819in}{1.261340in}}%
\pgfpathlineto{\pgfqpoint{1.386094in}{1.242889in}}%
\pgfpathlineto{\pgfqpoint{1.420190in}{1.226516in}}%
\pgfpathlineto{\pgfqpoint{1.457023in}{1.211329in}}%
\pgfpathlineto{\pgfqpoint{1.496554in}{1.197536in}}%
\pgfpathlineto{\pgfqpoint{1.538719in}{1.185287in}}%
\pgfpathlineto{\pgfqpoint{1.583440in}{1.174641in}}%
\pgfpathlineto{\pgfqpoint{1.634928in}{1.164775in}}%
\pgfpathlineto{\pgfqpoint{1.706062in}{1.153745in}}%
\pgfpathlineto{\pgfqpoint{1.768492in}{1.143417in}}%
\pgfpathlineto{\pgfqpoint{1.796122in}{1.136567in}}%
\pgfpathlineto{\pgfqpoint{1.812682in}{1.130481in}}%
\pgfpathlineto{\pgfqpoint{1.824470in}{1.124102in}}%
\pgfpathlineto{\pgfqpoint{1.833209in}{1.116741in}}%
\pgfpathlineto{\pgfqpoint{1.838498in}{1.108890in}}%
\pgfpathlineto{\pgfqpoint{1.840588in}{1.101849in}}%
\pgfpathlineto{\pgfqpoint{1.840619in}{1.094412in}}%
\pgfpathlineto{\pgfqpoint{1.837931in}{1.084986in}}%
\pgfpathlineto{\pgfqpoint{1.833246in}{1.076615in}}%
\pgfpathlineto{\pgfqpoint{1.825818in}{1.067543in}}%
\pgfpathlineto{\pgfqpoint{1.813813in}{1.056850in}}%
\pgfpathlineto{\pgfqpoint{1.798819in}{1.046763in}}%
\pgfpathlineto{\pgfqpoint{1.781016in}{1.037462in}}%
\pgfpathlineto{\pgfqpoint{1.758447in}{1.028391in}}%
\pgfpathlineto{\pgfqpoint{1.733203in}{1.020815in}}%
\pgfpathlineto{\pgfqpoint{1.705410in}{1.014872in}}%
\pgfpathlineto{\pgfqpoint{1.675178in}{1.010714in}}%
\pgfpathlineto{\pgfqpoint{1.642610in}{1.008507in}}%
\pgfpathlineto{\pgfqpoint{1.607809in}{1.008432in}}%
\pgfpathlineto{\pgfqpoint{1.570886in}{1.010691in}}%
\pgfpathlineto{\pgfqpoint{1.534118in}{1.015181in}}%
\pgfpathlineto{\pgfqpoint{1.495454in}{1.022233in}}%
\pgfpathlineto{\pgfqpoint{1.457161in}{1.031563in}}%
\pgfpathlineto{\pgfqpoint{1.419337in}{1.043132in}}%
\pgfpathlineto{\pgfqpoint{1.382089in}{1.056929in}}%
\pgfpathlineto{\pgfqpoint{1.347544in}{1.072019in}}%
\pgfpathlineto{\pgfqpoint{1.313727in}{1.089133in}}%
\pgfpathlineto{\pgfqpoint{1.280762in}{1.108299in}}%
\pgfpathlineto{\pgfqpoint{1.248782in}{1.129536in}}%
\pgfpathlineto{\pgfqpoint{1.219708in}{1.151422in}}%
\pgfpathlineto{\pgfqpoint{1.191752in}{1.175138in}}%
\pgfpathlineto{\pgfqpoint{1.165031in}{1.200649in}}%
\pgfpathlineto{\pgfqpoint{1.139653in}{1.227898in}}%
\pgfpathlineto{\pgfqpoint{1.115714in}{1.256800in}}%
\pgfpathlineto{\pgfqpoint{1.093288in}{1.287251in}}%
\pgfpathlineto{\pgfqpoint{1.071178in}{1.321163in}}%
\pgfpathlineto{\pgfqpoint{1.050868in}{1.356520in}}%
\pgfpathlineto{\pgfqpoint{1.032365in}{1.393152in}}%
\pgfpathlineto{\pgfqpoint{1.014718in}{1.433142in}}%
\pgfpathlineto{\pgfqpoint{0.999024in}{1.474185in}}%
\pgfpathlineto{\pgfqpoint{0.984506in}{1.518461in}}%
\pgfpathlineto{\pgfqpoint{0.972010in}{1.563537in}}%
\pgfpathlineto{\pgfqpoint{0.960944in}{1.611678in}}%
\pgfpathlineto{\pgfqpoint{0.951530in}{1.662824in}}%
\pgfpathlineto{\pgfqpoint{0.944286in}{1.714431in}}%
\pgfpathlineto{\pgfqpoint{0.938950in}{1.768847in}}%
\pgfpathlineto{\pgfqpoint{0.935870in}{1.823491in}}%
\pgfpathlineto{\pgfqpoint{0.935034in}{1.878240in}}%
\pgfpathlineto{\pgfqpoint{0.936466in}{1.932973in}}%
\pgfpathlineto{\pgfqpoint{0.940005in}{1.985084in}}%
\pgfpathlineto{\pgfqpoint{0.945759in}{2.036935in}}%
\pgfpathlineto{\pgfqpoint{0.953410in}{2.085938in}}%
\pgfpathlineto{\pgfqpoint{0.962764in}{2.132000in}}%
\pgfpathlineto{\pgfqpoint{0.974287in}{2.177414in}}%
\pgfpathlineto{\pgfqpoint{0.987332in}{2.219653in}}%
\pgfpathlineto{\pgfqpoint{1.001667in}{2.258654in}}%
\pgfpathlineto{\pgfqpoint{1.018051in}{2.296583in}}%
\pgfpathlineto{\pgfqpoint{1.035401in}{2.331101in}}%
\pgfpathlineto{\pgfqpoint{1.054650in}{2.364275in}}%
\pgfpathlineto{\pgfqpoint{1.074406in}{2.393984in}}%
\pgfpathlineto{\pgfqpoint{1.095771in}{2.422197in}}%
\pgfpathlineto{\pgfqpoint{1.118662in}{2.448797in}}%
\pgfpathlineto{\pgfqpoint{1.142967in}{2.473701in}}%
\pgfpathlineto{\pgfqpoint{1.168550in}{2.496867in}}%
\pgfpathlineto{\pgfqpoint{1.197085in}{2.519662in}}%
\pgfpathlineto{\pgfqpoint{1.226727in}{2.540526in}}%
\pgfpathlineto{\pgfqpoint{1.259242in}{2.560673in}}%
\pgfpathlineto{\pgfqpoint{1.294612in}{2.579881in}}%
\pgfpathlineto{\pgfqpoint{1.332792in}{2.597982in}}%
\pgfpathlineto{\pgfqpoint{1.373719in}{2.614859in}}%
\pgfpathlineto{\pgfqpoint{1.417319in}{2.630445in}}%
\pgfpathlineto{\pgfqpoint{1.465632in}{2.645312in}}%
\pgfpathlineto{\pgfqpoint{1.518640in}{2.659204in}}%
\pgfpathlineto{\pgfqpoint{1.576309in}{2.671929in}}%
\pgfpathlineto{\pgfqpoint{1.638597in}{2.683344in}}%
\pgfpathlineto{\pgfqpoint{1.705462in}{2.693343in}}%
\pgfpathlineto{\pgfqpoint{1.779027in}{2.702064in}}%
\pgfpathlineto{\pgfqpoint{1.857097in}{2.709077in}}%
\pgfpathlineto{\pgfqpoint{1.939633in}{2.714280in}}%
\pgfpathlineto{\pgfqpoint{2.026598in}{2.717513in}}%
\pgfpathlineto{\pgfqpoint{2.113605in}{2.718523in}}%
\pgfpathlineto{\pgfqpoint{2.198435in}{2.717303in}}%
\pgfpathlineto{\pgfqpoint{2.278866in}{2.713929in}}%
\pgfpathlineto{\pgfqpoint{2.352678in}{2.708598in}}%
\pgfpathlineto{\pgfqpoint{2.417657in}{2.701709in}}%
\pgfpathlineto{\pgfqpoint{2.473770in}{2.693630in}}%
\pgfpathlineto{\pgfqpoint{2.523140in}{2.684368in}}%
\pgfpathlineto{\pgfqpoint{2.565726in}{2.674202in}}%
\pgfpathlineto{\pgfqpoint{2.601510in}{2.663544in}}%
\pgfpathlineto{\pgfqpoint{2.632577in}{2.652142in}}%
\pgfpathlineto{\pgfqpoint{2.658899in}{2.640331in}}%
\pgfpathlineto{\pgfqpoint{2.682438in}{2.627436in}}%
\pgfpathlineto{\pgfqpoint{2.703062in}{2.613571in}}%
\pgfpathlineto{\pgfqpoint{2.720674in}{2.598978in}}%
\pgfpathlineto{\pgfqpoint{2.735263in}{2.584053in}}%
\pgfpathlineto{\pgfqpoint{2.748320in}{2.567377in}}%
\pgfpathlineto{\pgfqpoint{2.759553in}{2.549046in}}%
\pgfpathlineto{\pgfqpoint{2.768788in}{2.529306in}}%
\pgfpathlineto{\pgfqpoint{2.776017in}{2.508498in}}%
\pgfpathlineto{\pgfqpoint{2.781884in}{2.484540in}}%
\pgfpathlineto{\pgfqpoint{2.786102in}{2.457597in}}%
\pgfpathlineto{\pgfqpoint{2.788720in}{2.425384in}}%
\pgfpathlineto{\pgfqpoint{2.789427in}{2.388061in}}%
\pgfpathlineto{\pgfqpoint{2.787962in}{2.340801in}}%
\pgfpathlineto{\pgfqpoint{2.783672in}{2.278768in}}%
\pgfpathlineto{\pgfqpoint{2.774289in}{2.179783in}}%
\pgfpathlineto{\pgfqpoint{2.743611in}{1.868119in}}%
\pgfpathlineto{\pgfqpoint{2.730112in}{1.702060in}}%
\pgfpathlineto{\pgfqpoint{2.717287in}{1.515949in}}%
\pgfpathlineto{\pgfqpoint{2.702602in}{1.267597in}}%
\pgfpathlineto{\pgfqpoint{2.684434in}{0.964630in}}%
\pgfpathlineto{\pgfqpoint{2.675374in}{0.850600in}}%
\pgfpathlineto{\pgfqpoint{2.667030in}{0.771523in}}%
\pgfpathlineto{\pgfqpoint{2.658752in}{0.712543in}}%
\pgfpathlineto{\pgfqpoint{2.650176in}{0.666284in}}%
\pgfpathlineto{\pgfqpoint{2.640820in}{0.627931in}}%
\pgfpathlineto{\pgfqpoint{2.631145in}{0.597534in}}%
\pgfpathlineto{\pgfqpoint{2.621004in}{0.572745in}}%
\pgfpathlineto{\pgfqpoint{2.609856in}{0.551383in}}%
\pgfpathlineto{\pgfqpoint{2.598042in}{0.533534in}}%
\pgfpathlineto{\pgfqpoint{2.584495in}{0.517378in}}%
\pgfpathlineto{\pgfqpoint{2.571109in}{0.504669in}}%
\pgfpathlineto{\pgfqpoint{2.554789in}{0.492313in}}%
\pgfpathlineto{\pgfqpoint{2.537456in}{0.481914in}}%
\pgfpathlineto{\pgfqpoint{2.517374in}{0.472367in}}%
\pgfpathlineto{\pgfqpoint{2.492542in}{0.463178in}}%
\pgfpathlineto{\pgfqpoint{2.462979in}{0.454833in}}%
\pgfpathlineto{\pgfqpoint{2.428766in}{0.447542in}}%
\pgfpathlineto{\pgfqpoint{2.385671in}{0.440735in}}%
\pgfpathlineto{\pgfqpoint{2.331557in}{0.434581in}}%
\pgfpathlineto{\pgfqpoint{2.262115in}{0.429077in}}%
\pgfpathlineto{\pgfqpoint{2.170851in}{0.424236in}}%
\pgfpathlineto{\pgfqpoint{2.049086in}{0.420134in}}%
\pgfpathlineto{\pgfqpoint{1.879436in}{0.416783in}}%
\pgfpathlineto{\pgfqpoint{1.640159in}{0.414418in}}%
\pgfpathlineto{\pgfqpoint{1.322562in}{0.413569in}}%
\pgfpathlineto{\pgfqpoint{1.020194in}{0.414850in}}%
\pgfpathlineto{\pgfqpoint{0.822256in}{0.417715in}}%
\pgfpathlineto{\pgfqpoint{0.704835in}{0.421430in}}%
\pgfpathlineto{\pgfqpoint{0.630976in}{0.425829in}}%
\pgfpathlineto{\pgfqpoint{0.583316in}{0.430734in}}%
\pgfpathlineto{\pgfqpoint{0.551033in}{0.436124in}}%
\pgfpathlineto{\pgfqpoint{0.527708in}{0.442189in}}%
\pgfpathlineto{\pgfqpoint{0.511250in}{0.448625in}}%
\pgfpathlineto{\pgfqpoint{0.499549in}{0.455216in}}%
\pgfpathlineto{\pgfqpoint{0.488916in}{0.463842in}}%
\pgfpathlineto{\pgfqpoint{0.481322in}{0.472731in}}%
\pgfpathlineto{\pgfqpoint{0.474078in}{0.485127in}}%
\pgfpathlineto{\pgfqpoint{0.468753in}{0.498748in}}%
\pgfpathlineto{\pgfqpoint{0.463869in}{0.517849in}}%
\pgfpathlineto{\pgfqpoint{0.459679in}{0.544797in}}%
\pgfpathlineto{\pgfqpoint{0.456386in}{0.581939in}}%
\pgfpathlineto{\pgfqpoint{0.453731in}{0.639106in}}%
\pgfpathlineto{\pgfqpoint{0.451681in}{0.736155in}}%
\pgfpathlineto{\pgfqpoint{0.450220in}{0.927816in}}%
\pgfpathlineto{\pgfqpoint{0.449345in}{1.403252in}}%
\pgfpathlineto{\pgfqpoint{0.449543in}{2.682703in}}%
\pgfpathlineto{\pgfqpoint{0.451011in}{2.856932in}}%
\pgfpathlineto{\pgfqpoint{0.452802in}{2.879220in}}%
\pgfpathlineto{\pgfqpoint{0.455188in}{2.886108in}}%
\pgfpathlineto{\pgfqpoint{0.458626in}{2.889029in}}%
\pgfpathlineto{\pgfqpoint{0.464996in}{2.890553in}}%
\pgfpathlineto{\pgfqpoint{0.482377in}{2.891423in}}%
\pgfpathlineto{\pgfqpoint{0.565038in}{2.891729in}}%
\pgfpathlineto{\pgfqpoint{2.733843in}{2.891760in}}%
\pgfpathlineto{\pgfqpoint{4.789510in}{2.890885in}}%
\pgfpathlineto{\pgfqpoint{4.793727in}{2.889730in}}%
\pgfpathlineto{\pgfqpoint{4.795481in}{2.888306in}}%
\pgfpathlineto{\pgfqpoint{4.797106in}{2.881145in}}%
\pgfpathlineto{\pgfqpoint{4.797997in}{2.858771in}}%
\pgfpathlineto{\pgfqpoint{4.798039in}{2.856282in}}%
\pgfpathlineto{\pgfqpoint{4.798039in}{2.856282in}}%
\pgfusepath{stroke}%
\end{pgfscope}%
\begin{pgfscope}%
\pgfpathrectangle{\pgfqpoint{0.448634in}{0.402556in}}{\pgfqpoint{4.350661in}{2.489204in}} %
\pgfusepath{clip}%
\pgfsetrectcap%
\pgfsetroundjoin%
\pgfsetlinewidth{1.003750pt}%
\definecolor{currentstroke}{rgb}{0.839216,0.152941,0.156863}%
\pgfsetstrokecolor{currentstroke}%
\pgfsetdash{}{0pt}%
\pgfpathmoveto{\pgfqpoint{3.428769in}{0.402610in}}%
\pgfpathlineto{\pgfqpoint{2.806628in}{0.403760in}}%
\pgfpathlineto{\pgfqpoint{2.769688in}{0.405579in}}%
\pgfpathlineto{\pgfqpoint{2.754629in}{0.408066in}}%
\pgfpathlineto{\pgfqpoint{2.746388in}{0.411201in}}%
\pgfpathlineto{\pgfqpoint{2.740941in}{0.415269in}}%
\pgfpathlineto{\pgfqpoint{2.736783in}{0.420989in}}%
\pgfpathlineto{\pgfqpoint{2.733281in}{0.430077in}}%
\pgfpathlineto{\pgfqpoint{2.730449in}{0.444642in}}%
\pgfpathlineto{\pgfqpoint{2.728238in}{0.469398in}}%
\pgfpathlineto{\pgfqpoint{2.726470in}{0.519137in}}%
\pgfpathlineto{\pgfqpoint{2.725711in}{0.613720in}}%
\pgfpathlineto{\pgfqpoint{2.726842in}{0.768044in}}%
\pgfpathlineto{\pgfqpoint{2.730557in}{0.962154in}}%
\pgfpathlineto{\pgfqpoint{2.736611in}{1.158676in}}%
\pgfpathlineto{\pgfqpoint{2.744092in}{1.327724in}}%
\pgfpathlineto{\pgfqpoint{2.753202in}{1.484195in}}%
\pgfpathlineto{\pgfqpoint{2.763257in}{1.620615in}}%
\pgfpathlineto{\pgfqpoint{2.776119in}{1.764221in}}%
\pgfpathlineto{\pgfqpoint{2.788915in}{1.877782in}}%
\pgfpathlineto{\pgfqpoint{2.805749in}{2.005746in}}%
\pgfpathlineto{\pgfqpoint{2.821177in}{2.101203in}}%
\pgfpathlineto{\pgfqpoint{2.838361in}{2.193724in}}%
\pgfpathlineto{\pgfqpoint{2.859137in}{2.292971in}}%
\pgfpathlineto{\pgfqpoint{2.887211in}{2.425966in}}%
\pgfpathlineto{\pgfqpoint{2.896993in}{2.479565in}}%
\pgfpathlineto{\pgfqpoint{2.901544in}{2.516529in}}%
\pgfpathlineto{\pgfqpoint{2.902850in}{2.543860in}}%
\pgfpathlineto{\pgfqpoint{2.901958in}{2.566228in}}%
\pgfpathlineto{\pgfqpoint{2.899152in}{2.585868in}}%
\pgfpathlineto{\pgfqpoint{2.894793in}{2.602551in}}%
\pgfpathlineto{\pgfqpoint{2.888483in}{2.618393in}}%
\pgfpathlineto{\pgfqpoint{2.880256in}{2.633037in}}%
\pgfpathlineto{\pgfqpoint{2.870346in}{2.646250in}}%
\pgfpathlineto{\pgfqpoint{2.857397in}{2.659534in}}%
\pgfpathlineto{\pgfqpoint{2.843187in}{2.671013in}}%
\pgfpathlineto{\pgfqpoint{2.824235in}{2.683212in}}%
\pgfpathlineto{\pgfqpoint{2.802410in}{2.694421in}}%
\pgfpathlineto{\pgfqpoint{2.775805in}{2.705371in}}%
\pgfpathlineto{\pgfqpoint{2.744458in}{2.715717in}}%
\pgfpathlineto{\pgfqpoint{2.708433in}{2.725254in}}%
\pgfpathlineto{\pgfqpoint{2.665652in}{2.734291in}}%
\pgfpathlineto{\pgfqpoint{2.613988in}{2.742870in}}%
\pgfpathlineto{\pgfqpoint{2.553456in}{2.750589in}}%
\pgfpathlineto{\pgfqpoint{2.481917in}{2.757366in}}%
\pgfpathlineto{\pgfqpoint{2.399395in}{2.762840in}}%
\pgfpathlineto{\pgfqpoint{2.310266in}{2.766482in}}%
\pgfpathlineto{\pgfqpoint{2.175413in}{2.768726in}}%
\pgfpathlineto{\pgfqpoint{2.066650in}{2.767942in}}%
\pgfpathlineto{\pgfqpoint{1.953567in}{2.764860in}}%
\pgfpathlineto{\pgfqpoint{1.851425in}{2.759759in}}%
\pgfpathlineto{\pgfqpoint{1.745047in}{2.752169in}}%
\pgfpathlineto{\pgfqpoint{1.658370in}{2.743454in}}%
\pgfpathlineto{\pgfqpoint{1.580549in}{2.733461in}}%
\pgfpathlineto{\pgfqpoint{1.490054in}{2.719338in}}%
\pgfpathlineto{\pgfqpoint{1.417228in}{2.704697in}}%
\pgfpathlineto{\pgfqpoint{1.361989in}{2.690816in}}%
\pgfpathlineto{\pgfqpoint{1.311457in}{2.675817in}}%
\pgfpathlineto{\pgfqpoint{1.265664in}{2.659921in}}%
\pgfpathlineto{\pgfqpoint{1.222572in}{2.642584in}}%
\pgfpathlineto{\pgfqpoint{1.184321in}{2.624680in}}%
\pgfpathlineto{\pgfqpoint{1.148889in}{2.605620in}}%
\pgfpathlineto{\pgfqpoint{1.116329in}{2.585570in}}%
\pgfpathlineto{\pgfqpoint{1.092324in}{2.568509in}}%
\pgfpathlineto{\pgfqpoint{1.079757in}{2.558684in}}%
\pgfpathlineto{\pgfqpoint{1.051541in}{2.535376in}}%
\pgfpathlineto{\pgfqpoint{1.026309in}{2.511710in}}%
\pgfpathlineto{\pgfqpoint{1.002396in}{2.486315in}}%
\pgfpathlineto{\pgfqpoint{0.979910in}{2.459266in}}%
\pgfpathlineto{\pgfqpoint{0.958932in}{2.430675in}}%
\pgfpathlineto{\pgfqpoint{0.938262in}{2.398640in}}%
\pgfpathlineto{\pgfqpoint{0.923045in}{2.371382in}}%
\pgfpathlineto{\pgfqpoint{0.904511in}{2.334770in}}%
\pgfpathlineto{\pgfqpoint{0.887852in}{2.296997in}}%
\pgfpathlineto{\pgfqpoint{0.872130in}{2.255968in}}%
\pgfpathlineto{\pgfqpoint{0.857506in}{2.211737in}}%
\pgfpathlineto{\pgfqpoint{0.844761in}{2.166753in}}%
\pgfpathlineto{\pgfqpoint{0.838623in}{2.140302in}}%
\pgfpathlineto{\pgfqpoint{0.826981in}{2.087189in}}%
\pgfpathlineto{\pgfqpoint{0.816321in}{2.028711in}}%
\pgfpathlineto{\pgfqpoint{0.810087in}{1.984490in}}%
\pgfpathlineto{\pgfqpoint{0.808026in}{1.967234in}}%
\pgfpathlineto{\pgfqpoint{0.800076in}{1.898136in}}%
\pgfpathlineto{\pgfqpoint{0.793713in}{1.823819in}}%
\pgfpathlineto{\pgfqpoint{0.788799in}{1.741870in}}%
\pgfpathlineto{\pgfqpoint{0.786200in}{1.677221in}}%
\pgfpathlineto{\pgfqpoint{0.776951in}{1.453477in}}%
\pgfpathlineto{\pgfqpoint{0.773280in}{1.418890in}}%
\pgfpathlineto{\pgfqpoint{0.768298in}{1.389578in}}%
\pgfpathlineto{\pgfqpoint{0.762752in}{1.368104in}}%
\pgfpathlineto{\pgfqpoint{0.756721in}{1.352119in}}%
\pgfpathlineto{\pgfqpoint{0.749751in}{1.339515in}}%
\pgfpathlineto{\pgfqpoint{0.742200in}{1.330596in}}%
\pgfpathlineto{\pgfqpoint{0.734853in}{1.325309in}}%
\pgfpathlineto{\pgfqpoint{0.726556in}{1.322417in}}%
\pgfpathlineto{\pgfqpoint{0.717882in}{1.322221in}}%
\pgfpathlineto{\pgfqpoint{0.709411in}{1.324410in}}%
\pgfpathlineto{\pgfqpoint{0.699546in}{1.329604in}}%
\pgfpathlineto{\pgfqpoint{0.688893in}{1.338202in}}%
\pgfpathlineto{\pgfqpoint{0.677906in}{1.350247in}}%
\pgfpathlineto{\pgfqpoint{0.666885in}{1.365646in}}%
\pgfpathlineto{\pgfqpoint{0.654912in}{1.386416in}}%
\pgfpathlineto{\pgfqpoint{0.642573in}{1.412729in}}%
\pgfpathlineto{\pgfqpoint{0.630328in}{1.444628in}}%
\pgfpathlineto{\pgfqpoint{0.618504in}{1.482080in}}%
\pgfpathlineto{\pgfqpoint{0.608612in}{1.520256in}}%
\pgfpathlineto{\pgfqpoint{0.590202in}{1.612445in}}%
\pgfpathlineto{\pgfqpoint{0.581848in}{1.668884in}}%
\pgfpathlineto{\pgfqpoint{0.573137in}{1.740376in}}%
\pgfpathlineto{\pgfqpoint{0.567061in}{1.807213in}}%
\pgfpathlineto{\pgfqpoint{0.560532in}{1.896509in}}%
\pgfpathlineto{\pgfqpoint{0.555526in}{1.995910in}}%
\pgfpathlineto{\pgfqpoint{0.552564in}{2.097908in}}%
\pgfpathlineto{\pgfqpoint{0.551525in}{2.204935in}}%
\pgfpathlineto{\pgfqpoint{0.552727in}{2.309470in}}%
\pgfpathlineto{\pgfqpoint{0.556011in}{2.403981in}}%
\pgfpathlineto{\pgfqpoint{0.560953in}{2.483430in}}%
\pgfpathlineto{\pgfqpoint{0.567303in}{2.550240in}}%
\pgfpathlineto{\pgfqpoint{0.574928in}{2.606817in}}%
\pgfpathlineto{\pgfqpoint{0.582987in}{2.650657in}}%
\pgfpathlineto{\pgfqpoint{0.592756in}{2.691452in}}%
\pgfpathlineto{\pgfqpoint{0.602650in}{2.721756in}}%
\pgfpathlineto{\pgfqpoint{0.612984in}{2.746441in}}%
\pgfpathlineto{\pgfqpoint{0.624292in}{2.767692in}}%
\pgfpathlineto{\pgfqpoint{0.636231in}{2.785432in}}%
\pgfpathlineto{\pgfqpoint{0.649892in}{2.801461in}}%
\pgfpathlineto{\pgfqpoint{0.663386in}{2.814020in}}%
\pgfpathlineto{\pgfqpoint{0.679842in}{2.826135in}}%
\pgfpathlineto{\pgfqpoint{0.697326in}{2.836197in}}%
\pgfpathlineto{\pgfqpoint{0.715574in}{2.844285in}}%
\pgfpathlineto{\pgfqpoint{0.738439in}{2.852335in}}%
\pgfpathlineto{\pgfqpoint{0.765983in}{2.859639in}}%
\pgfpathlineto{\pgfqpoint{0.800300in}{2.866256in}}%
\pgfpathlineto{\pgfqpoint{0.841340in}{2.871832in}}%
\pgfpathlineto{\pgfqpoint{0.895547in}{2.876803in}}%
\pgfpathlineto{\pgfqpoint{0.969413in}{2.881069in}}%
\pgfpathlineto{\pgfqpoint{1.071608in}{2.884501in}}%
\pgfpathlineto{\pgfqpoint{1.219512in}{2.887074in}}%
\pgfpathlineto{\pgfqpoint{1.471844in}{2.889091in}}%
\pgfpathlineto{\pgfqpoint{1.956941in}{2.890384in}}%
\pgfpathlineto{\pgfqpoint{3.096814in}{2.890781in}}%
\pgfpathlineto{\pgfqpoint{3.995224in}{2.889388in}}%
\pgfpathlineto{\pgfqpoint{4.275833in}{2.887011in}}%
\pgfpathlineto{\pgfqpoint{4.412847in}{2.883743in}}%
\pgfpathlineto{\pgfqpoint{4.491081in}{2.879810in}}%
\pgfpathlineto{\pgfqpoint{4.543127in}{2.875163in}}%
\pgfpathlineto{\pgfqpoint{4.579810in}{2.869841in}}%
\pgfpathlineto{\pgfqpoint{4.607580in}{2.863763in}}%
\pgfpathlineto{\pgfqpoint{4.630623in}{2.856424in}}%
\pgfpathlineto{\pgfqpoint{4.648833in}{2.848228in}}%
\pgfpathlineto{\pgfqpoint{4.664136in}{2.838773in}}%
\pgfpathlineto{\pgfqpoint{4.676470in}{2.828576in}}%
\pgfpathlineto{\pgfqpoint{4.687502in}{2.816585in}}%
\pgfpathlineto{\pgfqpoint{4.697051in}{2.803027in}}%
\pgfpathlineto{\pgfqpoint{4.706194in}{2.786098in}}%
\pgfpathlineto{\pgfqpoint{4.714508in}{2.765827in}}%
\pgfpathlineto{\pgfqpoint{4.722462in}{2.740013in}}%
\pgfpathlineto{\pgfqpoint{4.729577in}{2.708703in}}%
\pgfpathlineto{\pgfqpoint{4.736162in}{2.669601in}}%
\pgfpathlineto{\pgfqpoint{4.742419in}{2.617826in}}%
\pgfpathlineto{\pgfqpoint{4.747859in}{2.553410in}}%
\pgfpathlineto{\pgfqpoint{4.752661in}{2.468958in}}%
\pgfpathlineto{\pgfqpoint{4.756610in}{2.359528in}}%
\pgfpathlineto{\pgfqpoint{4.759416in}{2.217681in}}%
\pgfpathlineto{\pgfqpoint{4.760596in}{2.043444in}}%
\pgfpathlineto{\pgfqpoint{4.759662in}{1.851779in}}%
\pgfpathlineto{\pgfqpoint{4.756587in}{1.667613in}}%
\pgfpathlineto{\pgfqpoint{4.751596in}{1.503428in}}%
\pgfpathlineto{\pgfqpoint{4.745410in}{1.374185in}}%
\pgfpathlineto{\pgfqpoint{4.738113in}{1.267479in}}%
\pgfpathlineto{\pgfqpoint{4.729621in}{1.175896in}}%
\pgfpathlineto{\pgfqpoint{4.720762in}{1.104428in}}%
\pgfpathlineto{\pgfqpoint{4.711045in}{1.043204in}}%
\pgfpathlineto{\pgfqpoint{4.700364in}{0.989829in}}%
\pgfpathlineto{\pgfqpoint{4.689055in}{0.944345in}}%
\pgfpathlineto{\pgfqpoint{4.676881in}{0.904394in}}%
\pgfpathlineto{\pgfqpoint{4.676095in}{0.902073in}}%
\pgfpathlineto{\pgfqpoint{4.676095in}{0.902073in}}%
\pgfusepath{stroke}%
\end{pgfscope}%
\begin{pgfscope}%
\pgfpathrectangle{\pgfqpoint{0.448634in}{0.402556in}}{\pgfqpoint{4.350661in}{2.489204in}} %
\pgfusepath{clip}%
\pgfsetrectcap%
\pgfsetroundjoin%
\pgfsetlinewidth{1.003750pt}%
\definecolor{currentstroke}{rgb}{0.839216,0.152941,0.156863}%
\pgfsetstrokecolor{currentstroke}%
\pgfsetdash{}{0pt}%
\pgfpathmoveto{\pgfqpoint{2.795520in}{1.982745in}}%
\pgfpathlineto{\pgfqpoint{2.781780in}{1.874357in}}%
\pgfpathlineto{\pgfqpoint{2.769351in}{1.758234in}}%
\pgfpathlineto{\pgfqpoint{2.758095in}{1.631942in}}%
\pgfpathlineto{\pgfqpoint{2.747786in}{1.490551in}}%
\pgfpathlineto{\pgfqpoint{2.738644in}{1.334082in}}%
\pgfpathlineto{\pgfqpoint{2.730580in}{1.157591in}}%
\pgfpathlineto{\pgfqpoint{2.723334in}{0.948663in}}%
\pgfpathlineto{\pgfqpoint{2.709783in}{0.530788in}}%
\pgfpathlineto{\pgfqpoint{2.705868in}{0.488716in}}%
\pgfpathlineto{\pgfqpoint{2.701769in}{0.464281in}}%
\pgfpathlineto{\pgfqpoint{2.697021in}{0.447744in}}%
\pgfpathlineto{\pgfqpoint{2.691859in}{0.436812in}}%
\pgfpathlineto{\pgfqpoint{2.686245in}{0.429229in}}%
\pgfpathlineto{\pgfqpoint{2.679348in}{0.423188in}}%
\pgfpathlineto{\pgfqpoint{2.669540in}{0.417856in}}%
\pgfpathlineto{\pgfqpoint{2.656987in}{0.413810in}}%
\pgfpathlineto{\pgfqpoint{2.637654in}{0.410337in}}%
\pgfpathlineto{\pgfqpoint{2.607297in}{0.407617in}}%
\pgfpathlineto{\pgfqpoint{2.555121in}{0.405574in}}%
\pgfpathlineto{\pgfqpoint{2.450714in}{0.404139in}}%
\pgfpathlineto{\pgfqpoint{2.176624in}{0.403275in}}%
\pgfpathlineto{\pgfqpoint{1.130290in}{0.402953in}}%
\pgfpathlineto{\pgfqpoint{0.516850in}{0.404175in}}%
\pgfpathlineto{\pgfqpoint{0.466848in}{0.405970in}}%
\pgfpathlineto{\pgfqpoint{0.456130in}{0.407931in}}%
\pgfpathlineto{\pgfqpoint{0.452340in}{0.410303in}}%
\pgfpathlineto{\pgfqpoint{0.450346in}{0.414662in}}%
\pgfpathlineto{\pgfqpoint{0.449266in}{0.424524in}}%
\pgfpathlineto{\pgfqpoint{0.448771in}{0.464344in}}%
\pgfpathlineto{\pgfqpoint{0.448640in}{0.850171in}}%
\pgfpathlineto{\pgfqpoint{0.448679in}{2.891318in}}%
\pgfpathlineto{\pgfqpoint{0.448679in}{2.891318in}}%
\pgfusepath{stroke}%
\end{pgfscope}%
\begin{pgfscope}%
\pgfpathrectangle{\pgfqpoint{0.448634in}{0.402556in}}{\pgfqpoint{4.350661in}{2.489204in}} %
\pgfusepath{clip}%
\pgfsetrectcap%
\pgfsetroundjoin%
\pgfsetlinewidth{1.003750pt}%
\definecolor{currentstroke}{rgb}{0.839216,0.152941,0.156863}%
\pgfsetstrokecolor{currentstroke}%
\pgfsetdash{}{0pt}%
\pgfpathmoveto{\pgfqpoint{3.428195in}{0.402586in}}%
\pgfpathlineto{\pgfqpoint{2.782127in}{0.403703in}}%
\pgfpathlineto{\pgfqpoint{2.753912in}{0.405677in}}%
\pgfpathlineto{\pgfqpoint{2.743335in}{0.408449in}}%
\pgfpathlineto{\pgfqpoint{2.737724in}{0.412194in}}%
\pgfpathlineto{\pgfqpoint{2.733674in}{0.418000in}}%
\pgfpathlineto{\pgfqpoint{2.730653in}{0.427312in}}%
\pgfpathlineto{\pgfqpoint{2.728391in}{0.442009in}}%
\pgfpathlineto{\pgfqpoint{2.726546in}{0.471799in}}%
\pgfpathlineto{\pgfqpoint{2.725218in}{0.534007in}}%
\pgfpathlineto{\pgfqpoint{2.725170in}{0.655977in}}%
\pgfpathlineto{\pgfqpoint{2.727378in}{0.832691in}}%
\pgfpathlineto{\pgfqpoint{2.732260in}{1.041707in}}%
\pgfpathlineto{\pgfqpoint{2.738852in}{1.223261in}}%
\pgfpathlineto{\pgfqpoint{2.747079in}{1.389770in}}%
\pgfpathlineto{\pgfqpoint{2.756609in}{1.538722in}}%
\pgfpathlineto{\pgfqpoint{2.768956in}{1.694891in}}%
\pgfpathlineto{\pgfqpoint{2.781229in}{1.816048in}}%
\pgfpathlineto{\pgfqpoint{2.794403in}{1.924529in}}%
\pgfpathlineto{\pgfqpoint{2.812739in}{2.054726in}}%
\pgfpathlineto{\pgfqpoint{2.828776in}{2.147516in}}%
\pgfpathlineto{\pgfqpoint{2.847384in}{2.242228in}}%
\pgfpathlineto{\pgfqpoint{2.895820in}{2.479703in}}%
\pgfpathlineto{\pgfqpoint{2.900206in}{2.516693in}}%
\pgfpathlineto{\pgfqpoint{2.901348in}{2.544033in}}%
\pgfpathlineto{\pgfqpoint{2.900293in}{2.566392in}}%
\pgfpathlineto{\pgfqpoint{2.897336in}{2.586003in}}%
\pgfpathlineto{\pgfqpoint{2.892838in}{2.602637in}}%
\pgfpathlineto{\pgfqpoint{2.886396in}{2.618409in}}%
\pgfpathlineto{\pgfqpoint{2.878059in}{2.632973in}}%
\pgfpathlineto{\pgfqpoint{2.868065in}{2.646103in}}%
\pgfpathlineto{\pgfqpoint{2.855051in}{2.659303in}}%
\pgfpathlineto{\pgfqpoint{2.840801in}{2.670719in}}%
\pgfpathlineto{\pgfqpoint{2.821822in}{2.682863in}}%
\pgfpathlineto{\pgfqpoint{2.799981in}{2.694028in}}%
\pgfpathlineto{\pgfqpoint{2.773366in}{2.704946in}}%
\pgfpathlineto{\pgfqpoint{2.742012in}{2.715268in}}%
\pgfpathlineto{\pgfqpoint{2.705983in}{2.724787in}}%
\pgfpathlineto{\pgfqpoint{2.663200in}{2.733812in}}%
\pgfpathlineto{\pgfqpoint{2.611535in}{2.742380in}}%
\pgfpathlineto{\pgfqpoint{2.551002in}{2.750091in}}%
\pgfpathlineto{\pgfqpoint{2.481632in}{2.756683in}}%
\pgfpathlineto{\pgfqpoint{2.399112in}{2.762201in}}%
\pgfpathlineto{\pgfqpoint{2.309985in}{2.765886in}}%
\pgfpathlineto{\pgfqpoint{2.188184in}{2.768097in}}%
\pgfpathlineto{\pgfqpoint{2.081595in}{2.767620in}}%
\pgfpathlineto{\pgfqpoint{1.968506in}{2.764841in}}%
\pgfpathlineto{\pgfqpoint{1.864180in}{2.759919in}}%
\pgfpathlineto{\pgfqpoint{1.757786in}{2.752594in}}%
\pgfpathlineto{\pgfqpoint{1.671087in}{2.744172in}}%
\pgfpathlineto{\pgfqpoint{1.591076in}{2.734194in}}%
\pgfpathlineto{\pgfqpoint{1.502689in}{2.720719in}}%
\pgfpathlineto{\pgfqpoint{1.427655in}{2.706083in}}%
\pgfpathlineto{\pgfqpoint{1.372350in}{2.692544in}}%
\pgfpathlineto{\pgfqpoint{1.321734in}{2.677921in}}%
\pgfpathlineto{\pgfqpoint{1.273765in}{2.661664in}}%
\pgfpathlineto{\pgfqpoint{1.230567in}{2.644672in}}%
\pgfpathlineto{\pgfqpoint{1.192197in}{2.627106in}}%
\pgfpathlineto{\pgfqpoint{1.156620in}{2.608403in}}%
\pgfpathlineto{\pgfqpoint{1.123890in}{2.588717in}}%
\pgfpathlineto{\pgfqpoint{1.095883in}{2.569568in}}%
\pgfpathlineto{\pgfqpoint{1.062184in}{2.542226in}}%
\pgfpathlineto{\pgfqpoint{1.036544in}{2.519140in}}%
\pgfpathlineto{\pgfqpoint{1.012183in}{2.494309in}}%
\pgfpathlineto{\pgfqpoint{0.989217in}{2.467794in}}%
\pgfpathlineto{\pgfqpoint{0.967738in}{2.439694in}}%
\pgfpathlineto{\pgfqpoint{0.946527in}{2.408126in}}%
\pgfpathlineto{\pgfqpoint{0.929469in}{2.379265in}}%
\pgfpathlineto{\pgfqpoint{0.888086in}{2.291751in}}%
\pgfpathlineto{\pgfqpoint{0.872613in}{2.250598in}}%
\pgfpathlineto{\pgfqpoint{0.858243in}{2.206259in}}%
\pgfpathlineto{\pgfqpoint{0.846499in}{2.163529in}}%
\pgfpathlineto{\pgfqpoint{0.841347in}{2.141922in}}%
\pgfpathlineto{\pgfqpoint{0.829682in}{2.088816in}}%
\pgfpathlineto{\pgfqpoint{0.819051in}{2.030331in}}%
\pgfpathlineto{\pgfqpoint{0.811767in}{1.978736in}}%
\pgfpathlineto{\pgfqpoint{0.803595in}{1.909672in}}%
\pgfpathlineto{\pgfqpoint{0.797091in}{1.835371in}}%
\pgfpathlineto{\pgfqpoint{0.792158in}{1.753425in}}%
\pgfpathlineto{\pgfqpoint{0.788708in}{1.658919in}}%
\pgfpathlineto{\pgfqpoint{0.785405in}{1.522066in}}%
\pgfpathlineto{\pgfqpoint{0.785405in}{1.522066in}}%
\pgfusepath{stroke}%
\end{pgfscope}%
\begin{pgfscope}%
\pgfpathrectangle{\pgfqpoint{0.448634in}{0.402556in}}{\pgfqpoint{4.350661in}{2.489204in}} %
\pgfusepath{clip}%
\pgfsetrectcap%
\pgfsetroundjoin%
\pgfsetlinewidth{1.003750pt}%
\definecolor{currentstroke}{rgb}{0.580392,0.403922,0.741176}%
\pgfsetstrokecolor{currentstroke}%
\pgfsetdash{}{0pt}%
\pgfpathmoveto{\pgfqpoint{0.448634in}{2.896245in}}%
\pgfpathlineto{\pgfqpoint{0.448593in}{0.407043in}}%
\pgfpathlineto{\pgfqpoint{0.448593in}{0.407043in}}%
\pgfusepath{stroke}%
\end{pgfscope}%
\begin{pgfscope}%
\pgfpathrectangle{\pgfqpoint{0.448634in}{0.402556in}}{\pgfqpoint{4.350661in}{2.489204in}} %
\pgfusepath{clip}%
\pgfsetrectcap%
\pgfsetroundjoin%
\pgfsetlinewidth{1.003750pt}%
\definecolor{currentstroke}{rgb}{0.580392,0.403922,0.741176}%
\pgfsetstrokecolor{currentstroke}%
\pgfsetdash{}{0pt}%
\pgfpathmoveto{\pgfqpoint{0.785144in}{1.518061in}}%
\pgfpathlineto{\pgfqpoint{0.794911in}{1.804074in}}%
\pgfpathlineto{\pgfqpoint{0.801199in}{1.885898in}}%
\pgfpathlineto{\pgfqpoint{0.808945in}{1.957535in}}%
\pgfpathlineto{\pgfqpoint{0.819856in}{2.036188in}}%
\pgfpathlineto{\pgfqpoint{0.831200in}{2.097044in}}%
\pgfpathlineto{\pgfqpoint{0.843757in}{2.152457in}}%
\pgfpathlineto{\pgfqpoint{0.861506in}{2.219026in}}%
\pgfpathlineto{\pgfqpoint{0.876438in}{2.263121in}}%
\pgfpathlineto{\pgfqpoint{0.892489in}{2.303983in}}%
\pgfpathlineto{\pgfqpoint{0.909488in}{2.341557in}}%
\pgfpathlineto{\pgfqpoint{0.935782in}{2.393067in}}%
\pgfpathlineto{\pgfqpoint{0.956161in}{2.425346in}}%
\pgfpathlineto{\pgfqpoint{0.976869in}{2.454193in}}%
\pgfpathlineto{\pgfqpoint{0.999096in}{2.481521in}}%
\pgfpathlineto{\pgfqpoint{1.022768in}{2.507211in}}%
\pgfpathlineto{\pgfqpoint{1.047779in}{2.531181in}}%
\pgfpathlineto{\pgfqpoint{1.075788in}{2.554814in}}%
\pgfpathlineto{\pgfqpoint{1.092310in}{2.566728in}}%
\pgfpathlineto{\pgfqpoint{1.096462in}{2.568146in}}%
\pgfpathlineto{\pgfqpoint{1.141660in}{2.598014in}}%
\pgfpathlineto{\pgfqpoint{1.174827in}{2.616719in}}%
\pgfpathlineto{\pgfqpoint{1.210775in}{2.634470in}}%
\pgfpathlineto{\pgfqpoint{1.247452in}{2.650138in}}%
\pgfpathlineto{\pgfqpoint{1.288769in}{2.665721in}}%
\pgfpathlineto{\pgfqpoint{1.339004in}{2.681972in}}%
\pgfpathlineto{\pgfqpoint{1.389762in}{2.695942in}}%
\pgfpathlineto{\pgfqpoint{1.445175in}{2.708884in}}%
\pgfpathlineto{\pgfqpoint{1.479315in}{2.716375in}}%
\pgfpathlineto{\pgfqpoint{1.482943in}{2.719063in}}%
\pgfpathlineto{\pgfqpoint{1.493637in}{2.721271in}}%
\pgfpathlineto{\pgfqpoint{1.543173in}{2.729177in}}%
\pgfpathlineto{\pgfqpoint{1.616562in}{2.739657in}}%
\pgfpathlineto{\pgfqpoint{1.696651in}{2.748797in}}%
\pgfpathlineto{\pgfqpoint{1.785573in}{2.756645in}}%
\pgfpathlineto{\pgfqpoint{1.872451in}{2.762129in}}%
\pgfpathlineto{\pgfqpoint{1.976785in}{2.766824in}}%
\pgfpathlineto{\pgfqpoint{2.081177in}{2.769277in}}%
\pgfpathlineto{\pgfqpoint{2.189942in}{2.769778in}}%
\pgfpathlineto{\pgfqpoint{2.296520in}{2.768041in}}%
\pgfpathlineto{\pgfqpoint{2.398700in}{2.764121in}}%
\pgfpathlineto{\pgfqpoint{2.483398in}{2.758593in}}%
\pgfpathlineto{\pgfqpoint{2.557108in}{2.751658in}}%
\pgfpathlineto{\pgfqpoint{2.619801in}{2.743646in}}%
\pgfpathlineto{\pgfqpoint{2.673602in}{2.734594in}}%
\pgfpathlineto{\pgfqpoint{2.718481in}{2.724859in}}%
\pgfpathlineto{\pgfqpoint{2.756536in}{2.714334in}}%
\pgfpathlineto{\pgfqpoint{2.787736in}{2.703424in}}%
\pgfpathlineto{\pgfqpoint{2.812116in}{2.692766in}}%
\pgfpathlineto{\pgfqpoint{2.833704in}{2.680975in}}%
\pgfpathlineto{\pgfqpoint{2.850505in}{2.669491in}}%
\pgfpathlineto{\pgfqpoint{2.864416in}{2.657543in}}%
\pgfpathlineto{\pgfqpoint{2.876929in}{2.643726in}}%
\pgfpathlineto{\pgfqpoint{2.886320in}{2.630027in}}%
\pgfpathlineto{\pgfqpoint{2.893913in}{2.614941in}}%
\pgfpathlineto{\pgfqpoint{2.899528in}{2.598760in}}%
\pgfpathlineto{\pgfqpoint{2.903581in}{2.579411in}}%
\pgfpathlineto{\pgfqpoint{2.905441in}{2.559623in}}%
\pgfpathlineto{\pgfqpoint{2.905382in}{2.534742in}}%
\pgfpathlineto{\pgfqpoint{2.902842in}{2.505022in}}%
\pgfpathlineto{\pgfqpoint{2.897320in}{2.468229in}}%
\pgfpathlineto{\pgfqpoint{2.887276in}{2.417238in}}%
\pgfpathlineto{\pgfqpoint{2.828360in}{2.136541in}}%
\pgfpathlineto{\pgfqpoint{2.812393in}{2.043735in}}%
\pgfpathlineto{\pgfqpoint{2.797826in}{1.945575in}}%
\pgfpathlineto{\pgfqpoint{2.796735in}{1.940784in}}%
\pgfpathlineto{\pgfqpoint{2.794383in}{1.936630in}}%
\pgfpathlineto{\pgfqpoint{2.792540in}{1.924378in}}%
\pgfpathlineto{\pgfqpoint{2.779525in}{1.815873in}}%
\pgfpathlineto{\pgfqpoint{2.767601in}{1.697175in}}%
\pgfpathlineto{\pgfqpoint{2.758758in}{1.588127in}}%
\pgfpathlineto{\pgfqpoint{2.748954in}{1.446689in}}%
\pgfpathlineto{\pgfqpoint{2.740383in}{1.290178in}}%
\pgfpathlineto{\pgfqpoint{2.733108in}{1.116134in}}%
\pgfpathlineto{\pgfqpoint{2.727696in}{0.934530in}}%
\pgfpathlineto{\pgfqpoint{2.723317in}{0.718028in}}%
\pgfpathlineto{\pgfqpoint{2.717843in}{0.424413in}}%
\pgfpathlineto{\pgfqpoint{2.715541in}{0.412291in}}%
\pgfpathlineto{\pgfqpoint{2.713161in}{0.408158in}}%
\pgfpathlineto{\pgfqpoint{2.709489in}{0.405591in}}%
\pgfpathlineto{\pgfqpoint{2.700923in}{0.403957in}}%
\pgfpathlineto{\pgfqpoint{2.679188in}{0.403057in}}%
\pgfpathlineto{\pgfqpoint{2.585650in}{0.402654in}}%
\pgfpathlineto{\pgfqpoint{1.347887in}{0.402565in}}%
\pgfpathlineto{\pgfqpoint{0.449476in}{0.402709in}}%
\pgfpathlineto{\pgfqpoint{0.449476in}{0.402709in}}%
\pgfusepath{stroke}%
\end{pgfscope}%
\begin{pgfscope}%
\pgfpathrectangle{\pgfqpoint{0.448634in}{0.402556in}}{\pgfqpoint{4.350661in}{2.489204in}} %
\pgfusepath{clip}%
\pgfsetrectcap%
\pgfsetroundjoin%
\pgfsetlinewidth{1.003750pt}%
\definecolor{currentstroke}{rgb}{0.580392,0.403922,0.741176}%
\pgfsetstrokecolor{currentstroke}%
\pgfsetdash{}{0pt}%
\pgfpathmoveto{\pgfqpoint{0.585555in}{1.685427in}}%
\pgfpathlineto{\pgfqpoint{0.576735in}{1.756901in}}%
\pgfpathlineto{\pgfqpoint{0.569225in}{1.836088in}}%
\pgfpathlineto{\pgfqpoint{0.562989in}{1.925412in}}%
\pgfpathlineto{\pgfqpoint{0.558320in}{2.024834in}}%
\pgfpathlineto{\pgfqpoint{0.555656in}{2.129334in}}%
\pgfpathlineto{\pgfqpoint{0.555164in}{2.233876in}}%
\pgfpathlineto{\pgfqpoint{0.556874in}{2.333422in}}%
\pgfpathlineto{\pgfqpoint{0.560491in}{2.420443in}}%
\pgfpathlineto{\pgfqpoint{0.565537in}{2.492396in}}%
\pgfpathlineto{\pgfqpoint{0.572099in}{2.556673in}}%
\pgfpathlineto{\pgfqpoint{0.579462in}{2.608257in}}%
\pgfpathlineto{\pgfqpoint{0.587866in}{2.652012in}}%
\pgfpathlineto{\pgfqpoint{0.596928in}{2.687873in}}%
\pgfpathlineto{\pgfqpoint{0.606844in}{2.718168in}}%
\pgfpathlineto{\pgfqpoint{0.617165in}{2.742861in}}%
\pgfpathlineto{\pgfqpoint{0.628418in}{2.764150in}}%
\pgfpathlineto{\pgfqpoint{0.640267in}{2.781971in}}%
\pgfpathlineto{\pgfqpoint{0.653805in}{2.798137in}}%
\pgfpathlineto{\pgfqpoint{0.667175in}{2.810869in}}%
\pgfpathlineto{\pgfqpoint{0.683494in}{2.823223in}}%
\pgfpathlineto{\pgfqpoint{0.700860in}{2.833549in}}%
\pgfpathlineto{\pgfqpoint{0.721024in}{2.842866in}}%
\pgfpathlineto{\pgfqpoint{0.743860in}{2.851023in}}%
\pgfpathlineto{\pgfqpoint{0.771379in}{2.858453in}}%
\pgfpathlineto{\pgfqpoint{0.805674in}{2.865220in}}%
\pgfpathlineto{\pgfqpoint{0.846697in}{2.870951in}}%
\pgfpathlineto{\pgfqpoint{0.898722in}{2.875916in}}%
\pgfpathlineto{\pgfqpoint{0.968230in}{2.880205in}}%
\pgfpathlineto{\pgfqpoint{1.063893in}{2.883744in}}%
\pgfpathlineto{\pgfqpoint{1.203092in}{2.886501in}}%
\pgfpathlineto{\pgfqpoint{1.424968in}{2.888625in}}%
\pgfpathlineto{\pgfqpoint{1.840453in}{2.890092in}}%
\pgfpathlineto{\pgfqpoint{2.780196in}{2.890757in}}%
\pgfpathlineto{\pgfqpoint{3.880912in}{2.889675in}}%
\pgfpathlineto{\pgfqpoint{4.218082in}{2.887460in}}%
\pgfpathlineto{\pgfqpoint{4.376857in}{2.884398in}}%
\pgfpathlineto{\pgfqpoint{4.465982in}{2.880605in}}%
\pgfpathlineto{\pgfqpoint{4.524578in}{2.876065in}}%
\pgfpathlineto{\pgfqpoint{4.565646in}{2.870776in}}%
\pgfpathlineto{\pgfqpoint{4.595655in}{2.864871in}}%
\pgfpathlineto{\pgfqpoint{4.618871in}{2.858272in}}%
\pgfpathlineto{\pgfqpoint{4.637375in}{2.850980in}}%
\pgfpathlineto{\pgfqpoint{4.653145in}{2.842584in}}%
\pgfpathlineto{\pgfqpoint{4.666106in}{2.833456in}}%
\pgfpathlineto{\pgfqpoint{4.677964in}{2.822545in}}%
\pgfpathlineto{\pgfqpoint{4.688463in}{2.809942in}}%
\pgfpathlineto{\pgfqpoint{4.698627in}{2.793792in}}%
\pgfpathlineto{\pgfqpoint{4.707069in}{2.776390in}}%
\pgfpathlineto{\pgfqpoint{4.715539in}{2.753474in}}%
\pgfpathlineto{\pgfqpoint{4.723397in}{2.724999in}}%
\pgfpathlineto{\pgfqpoint{4.730281in}{2.691057in}}%
\pgfpathlineto{\pgfqpoint{4.736902in}{2.646902in}}%
\pgfpathlineto{\pgfqpoint{4.742743in}{2.592553in}}%
\pgfpathlineto{\pgfqpoint{4.748081in}{2.520629in}}%
\pgfpathlineto{\pgfqpoint{4.752651in}{2.428679in}}%
\pgfpathlineto{\pgfqpoint{4.756274in}{2.309271in}}%
\pgfpathlineto{\pgfqpoint{4.758596in}{2.157455in}}%
\pgfpathlineto{\pgfqpoint{4.759083in}{1.978234in}}%
\pgfpathlineto{\pgfqpoint{4.757403in}{1.791556in}}%
\pgfpathlineto{\pgfqpoint{4.753608in}{1.614877in}}%
\pgfpathlineto{\pgfqpoint{4.747931in}{1.458197in}}%
\pgfpathlineto{\pgfqpoint{4.741281in}{1.338961in}}%
\pgfpathlineto{\pgfqpoint{4.733373in}{1.237308in}}%
\pgfpathlineto{\pgfqpoint{4.724293in}{1.150814in}}%
\pgfpathlineto{\pgfqpoint{4.715058in}{1.084445in}}%
\pgfpathlineto{\pgfqpoint{4.704710in}{1.025895in}}%
\pgfpathlineto{\pgfqpoint{4.693491in}{0.975229in}}%
\pgfpathlineto{\pgfqpoint{4.681793in}{0.932476in}}%
\pgfpathlineto{\pgfqpoint{4.669413in}{0.895261in}}%
\pgfpathlineto{\pgfqpoint{4.655797in}{0.861337in}}%
\pgfpathlineto{\pgfqpoint{4.641088in}{0.830832in}}%
\pgfpathlineto{\pgfqpoint{4.625573in}{0.803787in}}%
\pgfpathlineto{\pgfqpoint{4.609552in}{0.780214in}}%
\pgfpathlineto{\pgfqpoint{4.591923in}{0.758195in}}%
\pgfpathlineto{\pgfqpoint{4.572784in}{0.737894in}}%
\pgfpathlineto{\pgfqpoint{4.552289in}{0.719408in}}%
\pgfpathlineto{\pgfqpoint{4.530628in}{0.702748in}}%
\pgfpathlineto{\pgfqpoint{4.506075in}{0.686707in}}%
\pgfpathlineto{\pgfqpoint{4.480639in}{0.672578in}}%
\pgfpathlineto{\pgfqpoint{4.446321in}{0.656828in}}%
\pgfpathlineto{\pgfqpoint{4.413312in}{0.644213in}}%
\pgfpathlineto{\pgfqpoint{4.375568in}{0.632309in}}%
\pgfpathlineto{\pgfqpoint{4.333113in}{0.621452in}}%
\pgfpathlineto{\pgfqpoint{4.285997in}{0.611870in}}%
\pgfpathlineto{\pgfqpoint{4.234279in}{0.603735in}}%
\pgfpathlineto{\pgfqpoint{4.180171in}{0.597503in}}%
\pgfpathlineto{\pgfqpoint{4.119419in}{0.592545in}}%
\pgfpathlineto{\pgfqpoint{4.056389in}{0.589608in}}%
\pgfpathlineto{\pgfqpoint{3.986788in}{0.588410in}}%
\pgfpathlineto{\pgfqpoint{3.915010in}{0.589433in}}%
\pgfpathlineto{\pgfqpoint{3.843283in}{0.592698in}}%
\pgfpathlineto{\pgfqpoint{3.776007in}{0.597965in}}%
\pgfpathlineto{\pgfqpoint{3.708897in}{0.605500in}}%
\pgfpathlineto{\pgfqpoint{3.667972in}{0.612012in}}%
\pgfpathlineto{\pgfqpoint{3.616381in}{0.621147in}}%
\pgfpathlineto{\pgfqpoint{3.577983in}{0.629825in}}%
\pgfpathlineto{\pgfqpoint{3.531266in}{0.641698in}}%
\pgfpathlineto{\pgfqpoint{3.487108in}{0.655066in}}%
\pgfpathlineto{\pgfqpoint{3.443510in}{0.670637in}}%
\pgfpathlineto{\pgfqpoint{3.402737in}{0.687969in}}%
\pgfpathlineto{\pgfqpoint{3.368933in}{0.705117in}}%
\pgfpathlineto{\pgfqpoint{3.336054in}{0.724474in}}%
\pgfpathlineto{\pgfqpoint{3.308104in}{0.743731in}}%
\pgfpathlineto{\pgfqpoint{3.283063in}{0.763554in}}%
\pgfpathlineto{\pgfqpoint{3.259197in}{0.785191in}}%
\pgfpathlineto{\pgfqpoint{3.236695in}{0.808662in}}%
\pgfpathlineto{\pgfqpoint{3.215704in}{0.833901in}}%
\pgfpathlineto{\pgfqpoint{3.196340in}{0.860789in}}%
\pgfpathlineto{\pgfqpoint{3.178679in}{0.889170in}}%
\pgfpathlineto{\pgfqpoint{3.162749in}{0.918865in}}%
\pgfpathlineto{\pgfqpoint{3.146610in}{0.954143in}}%
\pgfpathlineto{\pgfqpoint{3.133417in}{0.988287in}}%
\pgfpathlineto{\pgfqpoint{3.121328in}{1.025627in}}%
\pgfpathlineto{\pgfqpoint{3.110548in}{1.066100in}}%
\pgfpathlineto{\pgfqpoint{3.101223in}{1.109610in}}%
\pgfpathlineto{\pgfqpoint{3.092187in}{1.165913in}}%
\pgfpathlineto{\pgfqpoint{3.088061in}{1.200438in}}%
\pgfpathlineto{\pgfqpoint{3.083854in}{1.252484in}}%
\pgfpathlineto{\pgfqpoint{3.081735in}{1.307188in}}%
\pgfpathlineto{\pgfqpoint{3.081804in}{1.364435in}}%
\pgfpathlineto{\pgfqpoint{3.084147in}{1.421619in}}%
\pgfpathlineto{\pgfqpoint{3.088726in}{1.478624in}}%
\pgfpathlineto{\pgfqpoint{3.095419in}{1.535357in}}%
\pgfpathlineto{\pgfqpoint{3.104323in}{1.591689in}}%
\pgfpathlineto{\pgfqpoint{3.114979in}{1.645072in}}%
\pgfpathlineto{\pgfqpoint{3.127729in}{1.697850in}}%
\pgfpathlineto{\pgfqpoint{3.142000in}{1.747502in}}%
\pgfpathlineto{\pgfqpoint{3.158344in}{1.796310in}}%
\pgfpathlineto{\pgfqpoint{3.175943in}{1.841834in}}%
\pgfpathlineto{\pgfqpoint{3.195553in}{1.886267in}}%
\pgfpathlineto{\pgfqpoint{3.216084in}{1.927308in}}%
\pgfpathlineto{\pgfqpoint{3.238450in}{1.967073in}}%
\pgfpathlineto{\pgfqpoint{3.262582in}{2.005464in}}%
\pgfpathlineto{\pgfqpoint{3.288377in}{2.042411in}}%
\pgfpathlineto{\pgfqpoint{3.317118in}{2.079780in}}%
\pgfpathlineto{\pgfqpoint{3.351759in}{2.121084in}}%
\pgfpathlineto{\pgfqpoint{3.415500in}{2.195942in}}%
\pgfpathlineto{\pgfqpoint{3.425818in}{2.211957in}}%
\pgfpathlineto{\pgfqpoint{3.430593in}{2.223110in}}%
\pgfpathlineto{\pgfqpoint{3.431872in}{2.230407in}}%
\pgfpathlineto{\pgfqpoint{3.430682in}{2.237676in}}%
\pgfpathlineto{\pgfqpoint{3.426585in}{2.243397in}}%
\pgfpathlineto{\pgfqpoint{3.420892in}{2.246996in}}%
\pgfpathlineto{\pgfqpoint{3.412492in}{2.249527in}}%
\pgfpathlineto{\pgfqpoint{3.399492in}{2.250656in}}%
\pgfpathlineto{\pgfqpoint{3.384298in}{2.249652in}}%
\pgfpathlineto{\pgfqpoint{3.364976in}{2.246088in}}%
\pgfpathlineto{\pgfqpoint{3.341794in}{2.239339in}}%
\pgfpathlineto{\pgfqpoint{3.317098in}{2.229682in}}%
\pgfpathlineto{\pgfqpoint{3.291092in}{2.216989in}}%
\pgfpathlineto{\pgfqpoint{3.265915in}{2.202265in}}%
\pgfpathlineto{\pgfqpoint{3.239792in}{2.184366in}}%
\pgfpathlineto{\pgfqpoint{3.214762in}{2.164524in}}%
\pgfpathlineto{\pgfqpoint{3.190887in}{2.142898in}}%
\pgfpathlineto{\pgfqpoint{3.166643in}{2.117917in}}%
\pgfpathlineto{\pgfqpoint{3.143822in}{2.091238in}}%
\pgfpathlineto{\pgfqpoint{3.121066in}{2.061112in}}%
\pgfpathlineto{\pgfqpoint{3.099940in}{2.029468in}}%
\pgfpathlineto{\pgfqpoint{3.079239in}{1.994409in}}%
\pgfpathlineto{\pgfqpoint{3.059207in}{1.955918in}}%
\pgfpathlineto{\pgfqpoint{3.040048in}{1.914018in}}%
\pgfpathlineto{\pgfqpoint{3.022800in}{1.871043in}}%
\pgfpathlineto{\pgfqpoint{3.005782in}{1.822538in}}%
\pgfpathlineto{\pgfqpoint{2.990059in}{1.770820in}}%
\pgfpathlineto{\pgfqpoint{2.975702in}{1.715981in}}%
\pgfpathlineto{\pgfqpoint{2.962279in}{1.655681in}}%
\pgfpathlineto{\pgfqpoint{2.950492in}{1.592386in}}%
\pgfpathlineto{\pgfqpoint{2.940380in}{1.526185in}}%
\pgfpathlineto{\pgfqpoint{2.931744in}{1.454681in}}%
\pgfpathlineto{\pgfqpoint{2.925082in}{1.380399in}}%
\pgfpathlineto{\pgfqpoint{2.920649in}{1.305899in}}%
\pgfpathlineto{\pgfqpoint{2.918447in}{1.231269in}}%
\pgfpathlineto{\pgfqpoint{2.918549in}{1.159087in}}%
\pgfpathlineto{\pgfqpoint{2.920792in}{1.091931in}}%
\pgfpathlineto{\pgfqpoint{2.925183in}{1.027412in}}%
\pgfpathlineto{\pgfqpoint{2.931199in}{0.970580in}}%
\pgfpathlineto{\pgfqpoint{2.938768in}{0.919034in}}%
\pgfpathlineto{\pgfqpoint{2.947660in}{0.872853in}}%
\pgfpathlineto{\pgfqpoint{2.958224in}{0.829715in}}%
\pgfpathlineto{\pgfqpoint{2.969681in}{0.792115in}}%
\pgfpathlineto{\pgfqpoint{2.982475in}{0.757775in}}%
\pgfpathlineto{\pgfqpoint{2.996437in}{0.726814in}}%
\pgfpathlineto{\pgfqpoint{3.011312in}{0.699303in}}%
\pgfpathlineto{\pgfqpoint{3.026753in}{0.675228in}}%
\pgfpathlineto{\pgfqpoint{3.043842in}{0.652659in}}%
\pgfpathlineto{\pgfqpoint{3.062510in}{0.631793in}}%
\pgfpathlineto{\pgfqpoint{3.082618in}{0.612757in}}%
\pgfpathlineto{\pgfqpoint{3.103977in}{0.595598in}}%
\pgfpathlineto{\pgfqpoint{3.128284in}{0.579075in}}%
\pgfpathlineto{\pgfqpoint{3.153553in}{0.564560in}}%
\pgfpathlineto{\pgfqpoint{3.181587in}{0.550959in}}%
\pgfpathlineto{\pgfqpoint{3.214388in}{0.537654in}}%
\pgfpathlineto{\pgfqpoint{3.249863in}{0.525720in}}%
\pgfpathlineto{\pgfqpoint{3.290027in}{0.514578in}}%
\pgfpathlineto{\pgfqpoint{3.334837in}{0.504431in}}%
\pgfpathlineto{\pgfqpoint{3.386389in}{0.495007in}}%
\pgfpathlineto{\pgfqpoint{3.446815in}{0.486266in}}%
\pgfpathlineto{\pgfqpoint{3.518259in}{0.478291in}}%
\pgfpathlineto{\pgfqpoint{3.600702in}{0.471418in}}%
\pgfpathlineto{\pgfqpoint{3.696285in}{0.465722in}}%
\pgfpathlineto{\pgfqpoint{3.807161in}{0.461379in}}%
\pgfpathlineto{\pgfqpoint{3.933308in}{0.458730in}}%
\pgfpathlineto{\pgfqpoint{4.063825in}{0.458223in}}%
\pgfpathlineto{\pgfqpoint{4.187809in}{0.459927in}}%
\pgfpathlineto{\pgfqpoint{4.294352in}{0.463536in}}%
\pgfpathlineto{\pgfqpoint{4.381251in}{0.468591in}}%
\pgfpathlineto{\pgfqpoint{4.450653in}{0.474720in}}%
\pgfpathlineto{\pgfqpoint{4.506867in}{0.481820in}}%
\pgfpathlineto{\pgfqpoint{4.552025in}{0.489681in}}%
\pgfpathlineto{\pgfqpoint{4.588254in}{0.498141in}}%
\pgfpathlineto{\pgfqpoint{4.617671in}{0.507138in}}%
\pgfpathlineto{\pgfqpoint{4.642343in}{0.516873in}}%
\pgfpathlineto{\pgfqpoint{4.664207in}{0.527974in}}%
\pgfpathlineto{\pgfqpoint{4.681250in}{0.538982in}}%
\pgfpathlineto{\pgfqpoint{4.697174in}{0.551993in}}%
\pgfpathlineto{\pgfqpoint{4.710084in}{0.565332in}}%
\pgfpathlineto{\pgfqpoint{4.721584in}{0.580263in}}%
\pgfpathlineto{\pgfqpoint{4.731561in}{0.596567in}}%
\pgfpathlineto{\pgfqpoint{4.741002in}{0.616181in}}%
\pgfpathlineto{\pgfqpoint{4.749522in}{0.639074in}}%
\pgfpathlineto{\pgfqpoint{4.757522in}{0.667498in}}%
\pgfpathlineto{\pgfqpoint{4.764570in}{0.701393in}}%
\pgfpathlineto{\pgfqpoint{4.770838in}{0.743092in}}%
\pgfpathlineto{\pgfqpoint{4.776324in}{0.794983in}}%
\pgfpathlineto{\pgfqpoint{4.781275in}{0.864446in}}%
\pgfpathlineto{\pgfqpoint{4.785466in}{0.956420in}}%
\pgfpathlineto{\pgfqpoint{4.788998in}{1.085794in}}%
\pgfpathlineto{\pgfqpoint{4.791851in}{1.277433in}}%
\pgfpathlineto{\pgfqpoint{4.793958in}{1.581106in}}%
\pgfpathlineto{\pgfqpoint{4.794961in}{2.071478in}}%
\pgfpathlineto{\pgfqpoint{4.793965in}{2.559359in}}%
\pgfpathlineto{\pgfqpoint{4.791729in}{2.746029in}}%
\pgfpathlineto{\pgfqpoint{4.788949in}{2.818140in}}%
\pgfpathlineto{\pgfqpoint{4.785716in}{2.850274in}}%
\pgfpathlineto{\pgfqpoint{4.781855in}{2.867101in}}%
\pgfpathlineto{\pgfqpoint{4.777709in}{2.875817in}}%
\pgfpathlineto{\pgfqpoint{4.773053in}{2.881008in}}%
\pgfpathlineto{\pgfqpoint{4.767314in}{2.884520in}}%
\pgfpathlineto{\pgfqpoint{4.756802in}{2.887629in}}%
\pgfpathlineto{\pgfqpoint{4.739496in}{2.889641in}}%
\pgfpathlineto{\pgfqpoint{4.704710in}{2.890882in}}%
\pgfpathlineto{\pgfqpoint{4.602472in}{2.891538in}}%
\pgfpathlineto{\pgfqpoint{3.952048in}{2.891742in}}%
\pgfpathlineto{\pgfqpoint{0.617269in}{2.890751in}}%
\pgfpathlineto{\pgfqpoint{0.549858in}{2.888853in}}%
\pgfpathlineto{\pgfqpoint{0.521684in}{2.886167in}}%
\pgfpathlineto{\pgfqpoint{0.504617in}{2.882365in}}%
\pgfpathlineto{\pgfqpoint{0.494456in}{2.877974in}}%
\pgfpathlineto{\pgfqpoint{0.487143in}{2.872616in}}%
\pgfpathlineto{\pgfqpoint{0.481125in}{2.865456in}}%
\pgfpathlineto{\pgfqpoint{0.475649in}{2.854734in}}%
\pgfpathlineto{\pgfqpoint{0.471311in}{2.840663in}}%
\pgfpathlineto{\pgfqpoint{0.467300in}{2.818747in}}%
\pgfpathlineto{\pgfqpoint{0.463929in}{2.786625in}}%
\pgfpathlineto{\pgfqpoint{0.460921in}{2.734469in}}%
\pgfpathlineto{\pgfqpoint{0.458366in}{2.647398in}}%
\pgfpathlineto{\pgfqpoint{0.456578in}{2.522955in}}%
\pgfpathlineto{\pgfqpoint{0.456578in}{2.522955in}}%
\pgfusepath{stroke}%
\end{pgfscope}%
\begin{pgfscope}%
\pgfpathrectangle{\pgfqpoint{0.448634in}{0.402556in}}{\pgfqpoint{4.350661in}{2.489204in}} %
\pgfusepath{clip}%
\pgfsetrectcap%
\pgfsetroundjoin%
\pgfsetlinewidth{1.003750pt}%
\definecolor{currentstroke}{rgb}{0.580392,0.403922,0.741176}%
\pgfsetstrokecolor{currentstroke}%
\pgfsetdash{}{0pt}%
\pgfpathmoveto{\pgfqpoint{4.798853in}{2.849880in}}%
\pgfpathlineto{\pgfqpoint{4.797564in}{2.889610in}}%
\pgfpathlineto{\pgfqpoint{4.796215in}{2.891483in}}%
\pgfpathlineto{\pgfqpoint{4.787551in}{2.891760in}}%
\pgfpathlineto{\pgfqpoint{0.452128in}{2.891659in}}%
\pgfpathlineto{\pgfqpoint{0.450530in}{2.890082in}}%
\pgfpathlineto{\pgfqpoint{0.449454in}{2.882763in}}%
\pgfpathlineto{\pgfqpoint{0.448970in}{2.845432in}}%
\pgfpathlineto{\pgfqpoint{0.448743in}{2.494454in}}%
\pgfpathlineto{\pgfqpoint{0.449624in}{0.615107in}}%
\pgfpathlineto{\pgfqpoint{0.451434in}{0.510586in}}%
\pgfpathlineto{\pgfqpoint{0.453994in}{0.473374in}}%
\pgfpathlineto{\pgfqpoint{0.457408in}{0.453868in}}%
\pgfpathlineto{\pgfqpoint{0.461542in}{0.442385in}}%
\pgfpathlineto{\pgfqpoint{0.466741in}{0.434438in}}%
\pgfpathlineto{\pgfqpoint{0.473597in}{0.428351in}}%
\pgfpathlineto{\pgfqpoint{0.483495in}{0.423245in}}%
\pgfpathlineto{\pgfqpoint{0.489746in}{0.421106in}}%
\pgfpathlineto{\pgfqpoint{0.489746in}{0.421106in}}%
\pgfusepath{stroke}%
\end{pgfscope}%
\begin{pgfscope}%
\pgfpathrectangle{\pgfqpoint{0.448634in}{0.402556in}}{\pgfqpoint{4.350661in}{2.489204in}} %
\pgfusepath{clip}%
\pgfsetrectcap%
\pgfsetroundjoin%
\pgfsetlinewidth{1.003750pt}%
\definecolor{currentstroke}{rgb}{0.580392,0.403922,0.741176}%
\pgfsetstrokecolor{currentstroke}%
\pgfsetdash{}{0pt}%
\pgfpathmoveto{\pgfqpoint{0.456440in}{1.368390in}}%
\pgfpathlineto{\pgfqpoint{0.459645in}{1.117009in}}%
\pgfpathlineto{\pgfqpoint{0.463671in}{0.962749in}}%
\pgfpathlineto{\pgfqpoint{0.468478in}{0.858351in}}%
\pgfpathlineto{\pgfqpoint{0.474018in}{0.783949in}}%
\pgfpathlineto{\pgfqpoint{0.480132in}{0.729640in}}%
\pgfpathlineto{\pgfqpoint{0.486839in}{0.688032in}}%
\pgfpathlineto{\pgfqpoint{0.494360in}{0.654271in}}%
\pgfpathlineto{\pgfqpoint{0.502877in}{0.626046in}}%
\pgfpathlineto{\pgfqpoint{0.511905in}{0.603410in}}%
\pgfpathlineto{\pgfqpoint{0.521853in}{0.584128in}}%
\pgfpathlineto{\pgfqpoint{0.533698in}{0.566309in}}%
\pgfpathlineto{\pgfqpoint{0.545802in}{0.552016in}}%
\pgfpathlineto{\pgfqpoint{0.559223in}{0.539354in}}%
\pgfpathlineto{\pgfqpoint{0.575587in}{0.527075in}}%
\pgfpathlineto{\pgfqpoint{0.594911in}{0.515668in}}%
\pgfpathlineto{\pgfqpoint{0.617088in}{0.505407in}}%
\pgfpathlineto{\pgfqpoint{0.641962in}{0.496364in}}%
\pgfpathlineto{\pgfqpoint{0.671511in}{0.487947in}}%
\pgfpathlineto{\pgfqpoint{0.707822in}{0.479956in}}%
\pgfpathlineto{\pgfqpoint{0.753024in}{0.472427in}}%
\pgfpathlineto{\pgfqpoint{0.807090in}{0.465738in}}%
\pgfpathlineto{\pgfqpoint{0.874316in}{0.459696in}}%
\pgfpathlineto{\pgfqpoint{0.959025in}{0.454382in}}%
\pgfpathlineto{\pgfqpoint{1.065547in}{0.450013in}}%
\pgfpathlineto{\pgfqpoint{1.198213in}{0.446891in}}%
\pgfpathlineto{\pgfqpoint{1.354831in}{0.445493in}}%
\pgfpathlineto{\pgfqpoint{1.522329in}{0.446206in}}%
\pgfpathlineto{\pgfqpoint{1.683283in}{0.449075in}}%
\pgfpathlineto{\pgfqpoint{1.822444in}{0.453725in}}%
\pgfpathlineto{\pgfqpoint{1.937615in}{0.459730in}}%
\pgfpathlineto{\pgfqpoint{2.030954in}{0.466709in}}%
\pgfpathlineto{\pgfqpoint{2.108953in}{0.474677in}}%
\pgfpathlineto{\pgfqpoint{2.173759in}{0.483445in}}%
\pgfpathlineto{\pgfqpoint{2.227518in}{0.492825in}}%
\pgfpathlineto{\pgfqpoint{2.274503in}{0.503212in}}%
\pgfpathlineto{\pgfqpoint{2.314673in}{0.514324in}}%
\pgfpathlineto{\pgfqpoint{2.350099in}{0.526443in}}%
\pgfpathlineto{\pgfqpoint{2.380735in}{0.539278in}}%
\pgfpathlineto{\pgfqpoint{2.406595in}{0.552358in}}%
\pgfpathlineto{\pgfqpoint{2.431559in}{0.567543in}}%
\pgfpathlineto{\pgfqpoint{2.453546in}{0.583629in}}%
\pgfpathlineto{\pgfqpoint{2.472579in}{0.600212in}}%
\pgfpathlineto{\pgfqpoint{2.490340in}{0.618548in}}%
\pgfpathlineto{\pgfqpoint{2.506681in}{0.638539in}}%
\pgfpathlineto{\pgfqpoint{2.521525in}{0.660005in}}%
\pgfpathlineto{\pgfqpoint{2.536014in}{0.684843in}}%
\pgfpathlineto{\pgfqpoint{2.549812in}{0.713080in}}%
\pgfpathlineto{\pgfqpoint{2.562689in}{0.744653in}}%
\pgfpathlineto{\pgfqpoint{2.575255in}{0.781786in}}%
\pgfpathlineto{\pgfqpoint{2.587150in}{0.824468in}}%
\pgfpathlineto{\pgfqpoint{2.598177in}{0.872622in}}%
\pgfpathlineto{\pgfqpoint{2.608689in}{0.928591in}}%
\pgfpathlineto{\pgfqpoint{2.619186in}{0.997242in}}%
\pgfpathlineto{\pgfqpoint{2.629266in}{1.078569in}}%
\pgfpathlineto{\pgfqpoint{2.639465in}{1.179955in}}%
\pgfpathlineto{\pgfqpoint{2.650055in}{1.308823in}}%
\pgfpathlineto{\pgfqpoint{2.661960in}{1.482533in}}%
\pgfpathlineto{\pgfqpoint{2.677229in}{1.740819in}}%
\pgfpathlineto{\pgfqpoint{2.688664in}{1.966957in}}%
\pgfpathlineto{\pgfqpoint{2.692946in}{2.091317in}}%
\pgfpathlineto{\pgfqpoint{2.693786in}{2.175941in}}%
\pgfpathlineto{\pgfqpoint{2.692278in}{2.240632in}}%
\pgfpathlineto{\pgfqpoint{2.688840in}{2.292751in}}%
\pgfpathlineto{\pgfqpoint{2.683904in}{2.334682in}}%
\pgfpathlineto{\pgfqpoint{2.677351in}{2.371251in}}%
\pgfpathlineto{\pgfqpoint{2.669528in}{2.402338in}}%
\pgfpathlineto{\pgfqpoint{2.660167in}{2.430211in}}%
\pgfpathlineto{\pgfqpoint{2.649539in}{2.454731in}}%
\pgfpathlineto{\pgfqpoint{2.638071in}{2.475872in}}%
\pgfpathlineto{\pgfqpoint{2.624890in}{2.495662in}}%
\pgfpathlineto{\pgfqpoint{2.610106in}{2.513909in}}%
\pgfpathlineto{\pgfqpoint{2.593907in}{2.530511in}}%
\pgfpathlineto{\pgfqpoint{2.574728in}{2.546869in}}%
\pgfpathlineto{\pgfqpoint{2.554417in}{2.561332in}}%
\pgfpathlineto{\pgfqpoint{2.531280in}{2.575148in}}%
\pgfpathlineto{\pgfqpoint{2.505365in}{2.588086in}}%
\pgfpathlineto{\pgfqpoint{2.474686in}{2.600787in}}%
\pgfpathlineto{\pgfqpoint{2.441330in}{2.612144in}}%
\pgfpathlineto{\pgfqpoint{2.403283in}{2.622710in}}%
\pgfpathlineto{\pgfqpoint{2.360580in}{2.632218in}}%
\pgfpathlineto{\pgfqpoint{2.313273in}{2.640474in}}%
\pgfpathlineto{\pgfqpoint{2.259248in}{2.647574in}}%
\pgfpathlineto{\pgfqpoint{2.200707in}{2.652988in}}%
\pgfpathlineto{\pgfqpoint{2.137704in}{2.656600in}}%
\pgfpathlineto{\pgfqpoint{2.070286in}{2.658250in}}%
\pgfpathlineto{\pgfqpoint{2.000679in}{2.657768in}}%
\pgfpathlineto{\pgfqpoint{1.928935in}{2.655050in}}%
\pgfpathlineto{\pgfqpoint{1.859453in}{2.650248in}}%
\pgfpathlineto{\pgfqpoint{1.792281in}{2.643463in}}%
\pgfpathlineto{\pgfqpoint{1.727472in}{2.634728in}}%
\pgfpathlineto{\pgfqpoint{1.667233in}{2.624433in}}%
\pgfpathlineto{\pgfqpoint{1.609469in}{2.612291in}}%
\pgfpathlineto{\pgfqpoint{1.556375in}{2.598839in}}%
\pgfpathlineto{\pgfqpoint{1.507981in}{2.584321in}}%
\pgfpathlineto{\pgfqpoint{1.462235in}{2.568248in}}%
\pgfpathlineto{\pgfqpoint{1.421255in}{2.551543in}}%
\pgfpathlineto{\pgfqpoint{1.383014in}{2.533610in}}%
\pgfpathlineto{\pgfqpoint{1.347572in}{2.514577in}}%
\pgfpathlineto{\pgfqpoint{1.314969in}{2.494616in}}%
\pgfpathlineto{\pgfqpoint{1.285222in}{2.473948in}}%
\pgfpathlineto{\pgfqpoint{1.256559in}{2.451366in}}%
\pgfpathlineto{\pgfqpoint{1.230833in}{2.428406in}}%
\pgfpathlineto{\pgfqpoint{1.206373in}{2.403702in}}%
\pgfpathlineto{\pgfqpoint{1.183324in}{2.377282in}}%
\pgfpathlineto{\pgfqpoint{1.161818in}{2.349210in}}%
\pgfpathlineto{\pgfqpoint{1.141961in}{2.319590in}}%
\pgfpathlineto{\pgfqpoint{1.123828in}{2.288556in}}%
\pgfpathlineto{\pgfqpoint{1.107461in}{2.256262in}}%
\pgfpathlineto{\pgfqpoint{1.091964in}{2.220608in}}%
\pgfpathlineto{\pgfqpoint{1.078472in}{2.183902in}}%
\pgfpathlineto{\pgfqpoint{1.066941in}{2.146330in}}%
\pgfpathlineto{\pgfqpoint{1.056772in}{2.105651in}}%
\pgfpathlineto{\pgfqpoint{1.048672in}{2.064368in}}%
\pgfpathlineto{\pgfqpoint{1.042266in}{2.020172in}}%
\pgfpathlineto{\pgfqpoint{1.038011in}{1.975637in}}%
\pgfpathlineto{\pgfqpoint{1.035783in}{1.928417in}}%
\pgfpathlineto{\pgfqpoint{1.035826in}{1.881128in}}%
\pgfpathlineto{\pgfqpoint{1.038111in}{1.833911in}}%
\pgfpathlineto{\pgfqpoint{1.042635in}{1.786907in}}%
\pgfpathlineto{\pgfqpoint{1.049418in}{1.740259in}}%
\pgfpathlineto{\pgfqpoint{1.057964in}{1.696540in}}%
\pgfpathlineto{\pgfqpoint{1.068622in}{1.653432in}}%
\pgfpathlineto{\pgfqpoint{1.081445in}{1.611104in}}%
\pgfpathlineto{\pgfqpoint{1.095592in}{1.572013in}}%
\pgfpathlineto{\pgfqpoint{1.111757in}{1.533961in}}%
\pgfpathlineto{\pgfqpoint{1.128835in}{1.499264in}}%
\pgfpathlineto{\pgfqpoint{1.147723in}{1.465819in}}%
\pgfpathlineto{\pgfqpoint{1.168398in}{1.433788in}}%
\pgfpathlineto{\pgfqpoint{1.190812in}{1.403327in}}%
\pgfpathlineto{\pgfqpoint{1.214888in}{1.374576in}}%
\pgfpathlineto{\pgfqpoint{1.238880in}{1.349277in}}%
\pgfpathlineto{\pgfqpoint{1.265868in}{1.324138in}}%
\pgfpathlineto{\pgfqpoint{1.294161in}{1.300953in}}%
\pgfpathlineto{\pgfqpoint{1.323612in}{1.279739in}}%
\pgfpathlineto{\pgfqpoint{1.354072in}{1.260480in}}%
\pgfpathlineto{\pgfqpoint{1.387381in}{1.242108in}}%
\pgfpathlineto{\pgfqpoint{1.421504in}{1.225811in}}%
\pgfpathlineto{\pgfqpoint{1.458361in}{1.210698in}}%
\pgfpathlineto{\pgfqpoint{1.497911in}{1.196976in}}%
\pgfpathlineto{\pgfqpoint{1.540090in}{1.184793in}}%
\pgfpathlineto{\pgfqpoint{1.584822in}{1.174203in}}%
\pgfpathlineto{\pgfqpoint{1.636316in}{1.164382in}}%
\pgfpathlineto{\pgfqpoint{1.707452in}{1.153357in}}%
\pgfpathlineto{\pgfqpoint{1.767723in}{1.143346in}}%
\pgfpathlineto{\pgfqpoint{1.795359in}{1.136527in}}%
\pgfpathlineto{\pgfqpoint{1.811934in}{1.130489in}}%
\pgfpathlineto{\pgfqpoint{1.823749in}{1.124177in}}%
\pgfpathlineto{\pgfqpoint{1.832549in}{1.116909in}}%
\pgfpathlineto{\pgfqpoint{1.837946in}{1.109152in}}%
\pgfpathlineto{\pgfqpoint{1.840163in}{1.102161in}}%
\pgfpathlineto{\pgfqpoint{1.840322in}{1.094728in}}%
\pgfpathlineto{\pgfqpoint{1.837755in}{1.085259in}}%
\pgfpathlineto{\pgfqpoint{1.833137in}{1.076841in}}%
\pgfpathlineto{\pgfqpoint{1.825752in}{1.067723in}}%
\pgfpathlineto{\pgfqpoint{1.813776in}{1.056987in}}%
\pgfpathlineto{\pgfqpoint{1.798799in}{1.046868in}}%
\pgfpathlineto{\pgfqpoint{1.781005in}{1.037544in}}%
\pgfpathlineto{\pgfqpoint{1.758442in}{1.028453in}}%
\pgfpathlineto{\pgfqpoint{1.733202in}{1.020862in}}%
\pgfpathlineto{\pgfqpoint{1.705411in}{1.014908in}}%
\pgfpathlineto{\pgfqpoint{1.675180in}{1.010741in}}%
\pgfpathlineto{\pgfqpoint{1.642612in}{1.008528in}}%
\pgfpathlineto{\pgfqpoint{1.607811in}{1.008448in}}%
\pgfpathlineto{\pgfqpoint{1.570888in}{1.010703in}}%
\pgfpathlineto{\pgfqpoint{1.534120in}{1.015190in}}%
\pgfpathlineto{\pgfqpoint{1.495456in}{1.022239in}}%
\pgfpathlineto{\pgfqpoint{1.457163in}{1.031568in}}%
\pgfpathlineto{\pgfqpoint{1.419338in}{1.043134in}}%
\pgfpathlineto{\pgfqpoint{1.382089in}{1.056930in}}%
\pgfpathlineto{\pgfqpoint{1.347544in}{1.072019in}}%
\pgfpathlineto{\pgfqpoint{1.313727in}{1.089133in}}%
\pgfpathlineto{\pgfqpoint{1.280762in}{1.108298in}}%
\pgfpathlineto{\pgfqpoint{1.248781in}{1.129535in}}%
\pgfpathlineto{\pgfqpoint{1.219707in}{1.151421in}}%
\pgfpathlineto{\pgfqpoint{1.191751in}{1.175136in}}%
\pgfpathlineto{\pgfqpoint{1.165029in}{1.200647in}}%
\pgfpathlineto{\pgfqpoint{1.139651in}{1.227895in}}%
\pgfpathlineto{\pgfqpoint{1.115712in}{1.256797in}}%
\pgfpathlineto{\pgfqpoint{1.093286in}{1.287248in}}%
\pgfpathlineto{\pgfqpoint{1.071176in}{1.321160in}}%
\pgfpathlineto{\pgfqpoint{1.050867in}{1.356516in}}%
\pgfpathlineto{\pgfqpoint{1.032363in}{1.393148in}}%
\pgfpathlineto{\pgfqpoint{1.014716in}{1.433139in}}%
\pgfpathlineto{\pgfqpoint{0.999023in}{1.474183in}}%
\pgfpathlineto{\pgfqpoint{0.984506in}{1.518458in}}%
\pgfpathlineto{\pgfqpoint{0.972009in}{1.563534in}}%
\pgfpathlineto{\pgfqpoint{0.960944in}{1.611676in}}%
\pgfpathlineto{\pgfqpoint{0.951531in}{1.662822in}}%
\pgfpathlineto{\pgfqpoint{0.944288in}{1.714429in}}%
\pgfpathlineto{\pgfqpoint{0.938952in}{1.768845in}}%
\pgfpathlineto{\pgfqpoint{0.935874in}{1.823489in}}%
\pgfpathlineto{\pgfqpoint{0.935038in}{1.878238in}}%
\pgfpathlineto{\pgfqpoint{0.936471in}{1.932971in}}%
\pgfpathlineto{\pgfqpoint{0.940011in}{1.985082in}}%
\pgfpathlineto{\pgfqpoint{0.945765in}{2.036934in}}%
\pgfpathlineto{\pgfqpoint{0.953417in}{2.085936in}}%
\pgfpathlineto{\pgfqpoint{0.962772in}{2.131998in}}%
\pgfpathlineto{\pgfqpoint{0.974295in}{2.177411in}}%
\pgfpathlineto{\pgfqpoint{0.987341in}{2.219650in}}%
\pgfpathlineto{\pgfqpoint{1.001676in}{2.258651in}}%
\pgfpathlineto{\pgfqpoint{1.018060in}{2.296580in}}%
\pgfpathlineto{\pgfqpoint{1.035411in}{2.331098in}}%
\pgfpathlineto{\pgfqpoint{1.054659in}{2.364272in}}%
\pgfpathlineto{\pgfqpoint{1.074416in}{2.393980in}}%
\pgfpathlineto{\pgfqpoint{1.095781in}{2.422193in}}%
\pgfpathlineto{\pgfqpoint{1.118672in}{2.448793in}}%
\pgfpathlineto{\pgfqpoint{1.142977in}{2.473696in}}%
\pgfpathlineto{\pgfqpoint{1.168561in}{2.496862in}}%
\pgfpathlineto{\pgfqpoint{1.197096in}{2.519657in}}%
\pgfpathlineto{\pgfqpoint{1.226738in}{2.540521in}}%
\pgfpathlineto{\pgfqpoint{1.259253in}{2.560667in}}%
\pgfpathlineto{\pgfqpoint{1.294624in}{2.579875in}}%
\pgfpathlineto{\pgfqpoint{1.332804in}{2.597975in}}%
\pgfpathlineto{\pgfqpoint{1.373731in}{2.614852in}}%
\pgfpathlineto{\pgfqpoint{1.417332in}{2.630437in}}%
\pgfpathlineto{\pgfqpoint{1.465645in}{2.645303in}}%
\pgfpathlineto{\pgfqpoint{1.518653in}{2.659195in}}%
\pgfpathlineto{\pgfqpoint{1.576322in}{2.671919in}}%
\pgfpathlineto{\pgfqpoint{1.638610in}{2.683333in}}%
\pgfpathlineto{\pgfqpoint{1.705475in}{2.693330in}}%
\pgfpathlineto{\pgfqpoint{1.779040in}{2.702051in}}%
\pgfpathlineto{\pgfqpoint{1.857110in}{2.709062in}}%
\pgfpathlineto{\pgfqpoint{1.939646in}{2.714264in}}%
\pgfpathlineto{\pgfqpoint{2.026612in}{2.717496in}}%
\pgfpathlineto{\pgfqpoint{2.113619in}{2.718505in}}%
\pgfpathlineto{\pgfqpoint{2.198448in}{2.717283in}}%
\pgfpathlineto{\pgfqpoint{2.278879in}{2.713907in}}%
\pgfpathlineto{\pgfqpoint{2.352691in}{2.708574in}}%
\pgfpathlineto{\pgfqpoint{2.417669in}{2.701683in}}%
\pgfpathlineto{\pgfqpoint{2.473783in}{2.693602in}}%
\pgfpathlineto{\pgfqpoint{2.523152in}{2.684337in}}%
\pgfpathlineto{\pgfqpoint{2.565738in}{2.674168in}}%
\pgfpathlineto{\pgfqpoint{2.601521in}{2.663506in}}%
\pgfpathlineto{\pgfqpoint{2.632586in}{2.652101in}}%
\pgfpathlineto{\pgfqpoint{2.658907in}{2.640285in}}%
\pgfpathlineto{\pgfqpoint{2.682444in}{2.627386in}}%
\pgfpathlineto{\pgfqpoint{2.703066in}{2.613517in}}%
\pgfpathlineto{\pgfqpoint{2.720675in}{2.598919in}}%
\pgfpathlineto{\pgfqpoint{2.735260in}{2.583989in}}%
\pgfpathlineto{\pgfqpoint{2.748312in}{2.567308in}}%
\pgfpathlineto{\pgfqpoint{2.759541in}{2.548973in}}%
\pgfpathlineto{\pgfqpoint{2.768770in}{2.529230in}}%
\pgfpathlineto{\pgfqpoint{2.775994in}{2.508420in}}%
\pgfpathlineto{\pgfqpoint{2.781857in}{2.484460in}}%
\pgfpathlineto{\pgfqpoint{2.786071in}{2.457516in}}%
\pgfpathlineto{\pgfqpoint{2.788685in}{2.425303in}}%
\pgfpathlineto{\pgfqpoint{2.789390in}{2.387980in}}%
\pgfpathlineto{\pgfqpoint{2.787924in}{2.340720in}}%
\pgfpathlineto{\pgfqpoint{2.783633in}{2.278687in}}%
\pgfpathlineto{\pgfqpoint{2.774251in}{2.179701in}}%
\pgfpathlineto{\pgfqpoint{2.743580in}{1.868036in}}%
\pgfpathlineto{\pgfqpoint{2.730084in}{1.701978in}}%
\pgfpathlineto{\pgfqpoint{2.717261in}{1.515866in}}%
\pgfpathlineto{\pgfqpoint{2.702576in}{1.267515in}}%
\pgfpathlineto{\pgfqpoint{2.684402in}{0.964548in}}%
\pgfpathlineto{\pgfqpoint{2.675337in}{0.850518in}}%
\pgfpathlineto{\pgfqpoint{2.666987in}{0.771443in}}%
\pgfpathlineto{\pgfqpoint{2.658702in}{0.712464in}}%
\pgfpathlineto{\pgfqpoint{2.650119in}{0.666206in}}%
\pgfpathlineto{\pgfqpoint{2.640753in}{0.627856in}}%
\pgfpathlineto{\pgfqpoint{2.631068in}{0.597464in}}%
\pgfpathlineto{\pgfqpoint{2.620918in}{0.572679in}}%
\pgfpathlineto{\pgfqpoint{2.609760in}{0.551325in}}%
\pgfpathlineto{\pgfqpoint{2.597937in}{0.533483in}}%
\pgfpathlineto{\pgfqpoint{2.584382in}{0.517337in}}%
\pgfpathlineto{\pgfqpoint{2.569236in}{0.503162in}}%
\pgfpathlineto{\pgfqpoint{2.552782in}{0.491041in}}%
\pgfpathlineto{\pgfqpoint{2.535353in}{0.480854in}}%
\pgfpathlineto{\pgfqpoint{2.515199in}{0.471505in}}%
\pgfpathlineto{\pgfqpoint{2.490315in}{0.462502in}}%
\pgfpathlineto{\pgfqpoint{2.460718in}{0.454315in}}%
\pgfpathlineto{\pgfqpoint{2.424335in}{0.446765in}}%
\pgfpathlineto{\pgfqpoint{2.381216in}{0.440166in}}%
\pgfpathlineto{\pgfqpoint{2.327087in}{0.434178in}}%
\pgfpathlineto{\pgfqpoint{2.257637in}{0.428802in}}%
\pgfpathlineto{\pgfqpoint{2.166369in}{0.424058in}}%
\pgfpathlineto{\pgfqpoint{2.042427in}{0.419971in}}%
\pgfpathlineto{\pgfqpoint{1.870601in}{0.416665in}}%
\pgfpathlineto{\pgfqpoint{1.626973in}{0.414345in}}%
\pgfpathlineto{\pgfqpoint{1.307201in}{0.413588in}}%
\pgfpathlineto{\pgfqpoint{1.009184in}{0.414958in}}%
\pgfpathlineto{\pgfqpoint{0.815597in}{0.417877in}}%
\pgfpathlineto{\pgfqpoint{0.700354in}{0.421645in}}%
\pgfpathlineto{\pgfqpoint{0.628673in}{0.426031in}}%
\pgfpathlineto{\pgfqpoint{0.581023in}{0.431064in}}%
\pgfpathlineto{\pgfqpoint{0.548764in}{0.436633in}}%
\pgfpathlineto{\pgfqpoint{0.527581in}{0.442258in}}%
\pgfpathlineto{\pgfqpoint{0.511129in}{0.448715in}}%
\pgfpathlineto{\pgfqpoint{0.499437in}{0.455327in}}%
\pgfpathlineto{\pgfqpoint{0.488820in}{0.463979in}}%
\pgfpathlineto{\pgfqpoint{0.481245in}{0.472888in}}%
\pgfpathlineto{\pgfqpoint{0.474023in}{0.485301in}}%
\pgfpathlineto{\pgfqpoint{0.468716in}{0.498932in}}%
\pgfpathlineto{\pgfqpoint{0.463848in}{0.518037in}}%
\pgfpathlineto{\pgfqpoint{0.459668in}{0.544988in}}%
\pgfpathlineto{\pgfqpoint{0.456382in}{0.582131in}}%
\pgfpathlineto{\pgfqpoint{0.453730in}{0.639299in}}%
\pgfpathlineto{\pgfqpoint{0.451649in}{0.738836in}}%
\pgfpathlineto{\pgfqpoint{0.450200in}{0.932986in}}%
\pgfpathlineto{\pgfqpoint{0.449337in}{1.415891in}}%
\pgfpathlineto{\pgfqpoint{0.449567in}{2.692852in}}%
\pgfpathlineto{\pgfqpoint{0.451020in}{2.857124in}}%
\pgfpathlineto{\pgfqpoint{0.452843in}{2.879407in}}%
\pgfpathlineto{\pgfqpoint{0.455293in}{2.886261in}}%
\pgfpathlineto{\pgfqpoint{0.458791in}{2.889095in}}%
\pgfpathlineto{\pgfqpoint{0.465170in}{2.890572in}}%
\pgfpathlineto{\pgfqpoint{0.482552in}{2.891425in}}%
\pgfpathlineto{\pgfqpoint{0.567389in}{2.891731in}}%
\pgfpathlineto{\pgfqpoint{2.862363in}{2.891760in}}%
\pgfpathlineto{\pgfqpoint{4.789684in}{2.890857in}}%
\pgfpathlineto{\pgfqpoint{4.793885in}{2.889641in}}%
\pgfpathlineto{\pgfqpoint{4.795589in}{2.888145in}}%
\pgfpathlineto{\pgfqpoint{4.797122in}{2.880943in}}%
\pgfpathlineto{\pgfqpoint{4.798041in}{2.856080in}}%
\pgfpathlineto{\pgfqpoint{4.798041in}{2.856080in}}%
\pgfusepath{stroke}%
\end{pgfscope}%
\begin{pgfscope}%
\pgfpathrectangle{\pgfqpoint{0.448634in}{0.402556in}}{\pgfqpoint{4.350661in}{2.489204in}} %
\pgfusepath{clip}%
\pgfsetrectcap%
\pgfsetroundjoin%
\pgfsetlinewidth{1.003750pt}%
\definecolor{currentstroke}{rgb}{0.580392,0.403922,0.741176}%
\pgfsetstrokecolor{currentstroke}%
\pgfsetdash{}{0pt}%
\pgfpathmoveto{\pgfqpoint{0.446896in}{0.402556in}}%
\pgfpathlineto{\pgfqpoint{3.427099in}{0.402556in}}%
\pgfpathlineto{\pgfqpoint{3.427099in}{0.402556in}}%
\pgfusepath{stroke}%
\end{pgfscope}%
\begin{pgfscope}%
\pgfpathrectangle{\pgfqpoint{0.448634in}{0.402556in}}{\pgfqpoint{4.350661in}{2.489204in}} %
\pgfusepath{clip}%
\pgfsetrectcap%
\pgfsetroundjoin%
\pgfsetlinewidth{1.003750pt}%
\definecolor{currentstroke}{rgb}{0.580392,0.403922,0.741176}%
\pgfsetstrokecolor{currentstroke}%
\pgfsetdash{}{0pt}%
\pgfpathmoveto{\pgfqpoint{2.795518in}{1.982745in}}%
\pgfpathlineto{\pgfqpoint{2.781777in}{1.874357in}}%
\pgfpathlineto{\pgfqpoint{2.769349in}{1.758235in}}%
\pgfpathlineto{\pgfqpoint{2.758093in}{1.631942in}}%
\pgfpathlineto{\pgfqpoint{2.747784in}{1.490551in}}%
\pgfpathlineto{\pgfqpoint{2.738643in}{1.334082in}}%
\pgfpathlineto{\pgfqpoint{2.730579in}{1.157591in}}%
\pgfpathlineto{\pgfqpoint{2.723333in}{0.948663in}}%
\pgfpathlineto{\pgfqpoint{2.709613in}{0.528306in}}%
\pgfpathlineto{\pgfqpoint{2.705550in}{0.486254in}}%
\pgfpathlineto{\pgfqpoint{2.701225in}{0.461873in}}%
\pgfpathlineto{\pgfqpoint{2.696142in}{0.445470in}}%
\pgfpathlineto{\pgfqpoint{2.690596in}{0.434791in}}%
\pgfpathlineto{\pgfqpoint{2.684641in}{0.427558in}}%
\pgfpathlineto{\pgfqpoint{2.677480in}{0.421932in}}%
\pgfpathlineto{\pgfqpoint{2.667499in}{0.417029in}}%
\pgfpathlineto{\pgfqpoint{2.654869in}{0.413300in}}%
\pgfpathlineto{\pgfqpoint{2.635504in}{0.410061in}}%
\pgfpathlineto{\pgfqpoint{2.605136in}{0.407484in}}%
\pgfpathlineto{\pgfqpoint{2.550783in}{0.405467in}}%
\pgfpathlineto{\pgfqpoint{2.442025in}{0.404077in}}%
\pgfpathlineto{\pgfqpoint{2.146181in}{0.403235in}}%
\pgfpathlineto{\pgfqpoint{1.012834in}{0.402983in}}%
\pgfpathlineto{\pgfqpoint{0.512511in}{0.404246in}}%
\pgfpathlineto{\pgfqpoint{0.464694in}{0.406190in}}%
\pgfpathlineto{\pgfqpoint{0.456140in}{0.407921in}}%
\pgfpathlineto{\pgfqpoint{0.452347in}{0.410288in}}%
\pgfpathlineto{\pgfqpoint{0.450349in}{0.414643in}}%
\pgfpathlineto{\pgfqpoint{0.449266in}{0.424504in}}%
\pgfpathlineto{\pgfqpoint{0.448771in}{0.464325in}}%
\pgfpathlineto{\pgfqpoint{0.448640in}{0.850151in}}%
\pgfpathlineto{\pgfqpoint{0.448679in}{2.891298in}}%
\pgfpathlineto{\pgfqpoint{0.448679in}{2.891298in}}%
\pgfusepath{stroke}%
\end{pgfscope}%
\begin{pgfscope}%
\pgfpathrectangle{\pgfqpoint{0.448634in}{0.402556in}}{\pgfqpoint{4.350661in}{2.489204in}} %
\pgfusepath{clip}%
\pgfsetrectcap%
\pgfsetroundjoin%
\pgfsetlinewidth{1.003750pt}%
\definecolor{currentstroke}{rgb}{0.580392,0.403922,0.741176}%
\pgfsetstrokecolor{currentstroke}%
\pgfsetdash{}{0pt}%
\pgfpathmoveto{\pgfqpoint{3.385051in}{2.228612in}}%
\pgfpathlineto{\pgfqpoint{3.393310in}{2.225596in}}%
\pgfpathlineto{\pgfqpoint{3.398404in}{2.221016in}}%
\pgfpathlineto{\pgfqpoint{3.400443in}{2.216649in}}%
\pgfpathlineto{\pgfqpoint{3.400915in}{2.209259in}}%
\pgfpathlineto{\pgfqpoint{3.398469in}{2.199738in}}%
\pgfpathlineto{\pgfqpoint{3.392083in}{2.186734in}}%
\pgfpathlineto{\pgfqpoint{3.380299in}{2.168856in}}%
\pgfpathlineto{\pgfqpoint{3.357303in}{2.138965in}}%
\pgfpathlineto{\pgfqpoint{3.296378in}{2.061070in}}%
\pgfpathlineto{\pgfqpoint{3.267755in}{2.020335in}}%
\pgfpathlineto{\pgfqpoint{3.242198in}{1.980053in}}%
\pgfpathlineto{\pgfqpoint{3.219630in}{1.940437in}}%
\pgfpathlineto{\pgfqpoint{3.198824in}{1.899577in}}%
\pgfpathlineto{\pgfqpoint{3.178872in}{1.855343in}}%
\pgfpathlineto{\pgfqpoint{3.160909in}{1.810006in}}%
\pgfpathlineto{\pgfqpoint{3.144912in}{1.763715in}}%
\pgfpathlineto{\pgfqpoint{3.130141in}{1.714255in}}%
\pgfpathlineto{\pgfqpoint{3.116844in}{1.661654in}}%
\pgfpathlineto{\pgfqpoint{3.104763in}{1.603540in}}%
\pgfpathlineto{\pgfqpoint{3.095494in}{1.547284in}}%
\pgfpathlineto{\pgfqpoint{3.088175in}{1.488139in}}%
\pgfpathlineto{\pgfqpoint{3.083160in}{1.428680in}}%
\pgfpathlineto{\pgfqpoint{3.080606in}{1.371507in}}%
\pgfpathlineto{\pgfqpoint{3.080298in}{1.314262in}}%
\pgfpathlineto{\pgfqpoint{3.082177in}{1.259546in}}%
\pgfpathlineto{\pgfqpoint{3.085870in}{1.209946in}}%
\pgfpathlineto{\pgfqpoint{3.091696in}{1.160618in}}%
\pgfpathlineto{\pgfqpoint{3.099593in}{1.111667in}}%
\pgfpathlineto{\pgfqpoint{3.107681in}{1.072935in}}%
\pgfpathlineto{\pgfqpoint{3.118796in}{1.029979in}}%
\pgfpathlineto{\pgfqpoint{3.131462in}{0.990230in}}%
\pgfpathlineto{\pgfqpoint{3.145475in}{0.953781in}}%
\pgfpathlineto{\pgfqpoint{3.160547in}{0.920672in}}%
\pgfpathlineto{\pgfqpoint{3.176350in}{0.890889in}}%
\pgfpathlineto{\pgfqpoint{3.193886in}{0.862406in}}%
\pgfpathlineto{\pgfqpoint{3.213128in}{0.835403in}}%
\pgfpathlineto{\pgfqpoint{3.234000in}{0.810036in}}%
\pgfpathlineto{\pgfqpoint{3.256392in}{0.786427in}}%
\pgfpathlineto{\pgfqpoint{3.280160in}{0.764650in}}%
\pgfpathlineto{\pgfqpoint{3.306945in}{0.743336in}}%
\pgfpathlineto{\pgfqpoint{3.334890in}{0.724073in}}%
\pgfpathlineto{\pgfqpoint{3.363845in}{0.706865in}}%
\pgfpathlineto{\pgfqpoint{3.395539in}{0.690418in}}%
\pgfpathlineto{\pgfqpoint{3.427962in}{0.675963in}}%
\pgfpathlineto{\pgfqpoint{3.490108in}{0.653245in}}%
\pgfpathlineto{\pgfqpoint{3.532199in}{0.640666in}}%
\pgfpathlineto{\pgfqpoint{3.581067in}{0.628400in}}%
\pgfpathlineto{\pgfqpoint{3.643115in}{0.615521in}}%
\pgfpathlineto{\pgfqpoint{3.712050in}{0.604545in}}%
\pgfpathlineto{\pgfqpoint{3.772668in}{0.597764in}}%
\pgfpathlineto{\pgfqpoint{3.839939in}{0.592410in}}%
\pgfpathlineto{\pgfqpoint{3.909488in}{0.589119in}}%
\pgfpathlineto{\pgfqpoint{3.981264in}{0.587933in}}%
\pgfpathlineto{\pgfqpoint{4.050867in}{0.588961in}}%
\pgfpathlineto{\pgfqpoint{4.122591in}{0.592294in}}%
\pgfpathlineto{\pgfqpoint{4.185503in}{0.597577in}}%
\pgfpathlineto{\pgfqpoint{4.246068in}{0.604933in}}%
\pgfpathlineto{\pgfqpoint{4.297725in}{0.613564in}}%
\pgfpathlineto{\pgfqpoint{4.344752in}{0.623697in}}%
\pgfpathlineto{\pgfqpoint{4.384979in}{0.634539in}}%
\pgfpathlineto{\pgfqpoint{4.422589in}{0.646984in}}%
\pgfpathlineto{\pgfqpoint{4.455427in}{0.660168in}}%
\pgfpathlineto{\pgfqpoint{4.485536in}{0.674544in}}%
\pgfpathlineto{\pgfqpoint{4.512740in}{0.690195in}}%
\pgfpathlineto{\pgfqpoint{4.537027in}{0.706758in}}%
\pgfpathlineto{\pgfqpoint{4.558383in}{0.723922in}}%
\pgfpathlineto{\pgfqpoint{4.578519in}{0.742918in}}%
\pgfpathlineto{\pgfqpoint{4.597249in}{0.763713in}}%
\pgfpathlineto{\pgfqpoint{4.614436in}{0.786184in}}%
\pgfpathlineto{\pgfqpoint{4.629992in}{0.810162in}}%
\pgfpathlineto{\pgfqpoint{4.646118in}{0.839714in}}%
\pgfpathlineto{\pgfqpoint{4.660195in}{0.870607in}}%
\pgfpathlineto{\pgfqpoint{4.673208in}{0.904840in}}%
\pgfpathlineto{\pgfqpoint{4.685035in}{0.942291in}}%
\pgfpathlineto{\pgfqpoint{4.696217in}{0.985225in}}%
\pgfpathlineto{\pgfqpoint{4.706492in}{1.033596in}}%
\pgfpathlineto{\pgfqpoint{4.716133in}{1.089770in}}%
\pgfpathlineto{\pgfqpoint{4.725166in}{1.156175in}}%
\pgfpathlineto{\pgfqpoint{4.732692in}{1.227843in}}%
\pgfpathlineto{\pgfqpoint{4.739776in}{1.314585in}}%
\pgfpathlineto{\pgfqpoint{4.746304in}{1.421357in}}%
\pgfpathlineto{\pgfqpoint{4.751545in}{1.543178in}}%
\pgfpathlineto{\pgfqpoint{4.753172in}{1.595417in}}%
\pgfpathlineto{\pgfqpoint{4.756888in}{1.752178in}}%
\pgfpathlineto{\pgfqpoint{4.759013in}{1.933872in}}%
\pgfpathlineto{\pgfqpoint{4.759004in}{2.120561in}}%
\pgfpathlineto{\pgfqpoint{4.756830in}{2.289807in}}%
\pgfpathlineto{\pgfqpoint{4.752979in}{2.424149in}}%
\pgfpathlineto{\pgfqpoint{4.747910in}{2.526038in}}%
\pgfpathlineto{\pgfqpoint{4.741967in}{2.602898in}}%
\pgfpathlineto{\pgfqpoint{4.735414in}{2.659651in}}%
\pgfpathlineto{\pgfqpoint{4.728166in}{2.703675in}}%
\pgfpathlineto{\pgfqpoint{4.720521in}{2.737398in}}%
\pgfpathlineto{\pgfqpoint{4.712527in}{2.763196in}}%
\pgfpathlineto{\pgfqpoint{4.704212in}{2.783467in}}%
\pgfpathlineto{\pgfqpoint{4.695109in}{2.800424in}}%
\pgfpathlineto{\pgfqpoint{4.684207in}{2.815924in}}%
\pgfpathlineto{\pgfqpoint{4.673076in}{2.827795in}}%
\pgfpathlineto{\pgfqpoint{4.660686in}{2.837904in}}%
\pgfpathlineto{\pgfqpoint{4.645360in}{2.847311in}}%
\pgfpathlineto{\pgfqpoint{4.627151in}{2.855509in}}%
\pgfpathlineto{\pgfqpoint{4.606236in}{2.862325in}}%
\pgfpathlineto{\pgfqpoint{4.580657in}{2.868255in}}%
\pgfpathlineto{\pgfqpoint{4.548344in}{2.873419in}}%
\pgfpathlineto{\pgfqpoint{4.505020in}{2.877932in}}%
\pgfpathlineto{\pgfqpoint{4.446383in}{2.881737in}}%
\pgfpathlineto{\pgfqpoint{4.361591in}{2.884881in}}%
\pgfpathlineto{\pgfqpoint{4.231089in}{2.887321in}}%
\pgfpathlineto{\pgfqpoint{4.002686in}{2.889193in}}%
\pgfpathlineto{\pgfqpoint{3.537167in}{2.890399in}}%
\pgfpathlineto{\pgfqpoint{2.410345in}{2.890680in}}%
\pgfpathlineto{\pgfqpoint{1.527163in}{2.889181in}}%
\pgfpathlineto{\pgfqpoint{1.213924in}{2.886687in}}%
\pgfpathlineto{\pgfqpoint{1.046451in}{2.883288in}}%
\pgfpathlineto{\pgfqpoint{0.946453in}{2.879152in}}%
\pgfpathlineto{\pgfqpoint{0.876980in}{2.874197in}}%
\pgfpathlineto{\pgfqpoint{0.825028in}{2.868334in}}%
\pgfpathlineto{\pgfqpoint{0.786286in}{2.861871in}}%
\pgfpathlineto{\pgfqpoint{0.754329in}{2.854360in}}%
\pgfpathlineto{\pgfqpoint{0.729190in}{2.846336in}}%
\pgfpathlineto{\pgfqpoint{0.706731in}{2.836912in}}%
\pgfpathlineto{\pgfqpoint{0.687157in}{2.826078in}}%
\pgfpathlineto{\pgfqpoint{0.670551in}{2.814234in}}%
\pgfpathlineto{\pgfqpoint{0.656865in}{2.801948in}}%
\pgfpathlineto{\pgfqpoint{0.644395in}{2.788071in}}%
\pgfpathlineto{\pgfqpoint{0.631952in}{2.770790in}}%
\pgfpathlineto{\pgfqpoint{0.621182in}{2.752093in}}%
\pgfpathlineto{\pgfqpoint{0.611013in}{2.730098in}}%
\pgfpathlineto{\pgfqpoint{0.600936in}{2.702554in}}%
\pgfpathlineto{\pgfqpoint{0.592113in}{2.671816in}}%
\pgfpathlineto{\pgfqpoint{0.583068in}{2.630793in}}%
\pgfpathlineto{\pgfqpoint{0.575364in}{2.584333in}}%
\pgfpathlineto{\pgfqpoint{0.568713in}{2.530106in}}%
\pgfpathlineto{\pgfqpoint{0.563081in}{2.465712in}}%
\pgfpathlineto{\pgfqpoint{0.558287in}{2.383757in}}%
\pgfpathlineto{\pgfqpoint{0.555532in}{2.294204in}}%
\pgfpathlineto{\pgfqpoint{0.554688in}{2.194643in}}%
\pgfpathlineto{\pgfqpoint{0.555989in}{2.090109in}}%
\pgfpathlineto{\pgfqpoint{0.559455in}{1.985641in}}%
\pgfpathlineto{\pgfqpoint{0.564977in}{1.886276in}}%
\pgfpathlineto{\pgfqpoint{0.572131in}{1.797043in}}%
\pgfpathlineto{\pgfqpoint{0.580652in}{1.717991in}}%
\pgfpathlineto{\pgfqpoint{0.590274in}{1.649173in}}%
\pgfpathlineto{\pgfqpoint{0.600605in}{1.590618in}}%
\pgfpathlineto{\pgfqpoint{0.611709in}{1.539918in}}%
\pgfpathlineto{\pgfqpoint{0.623236in}{1.497104in}}%
\pgfpathlineto{\pgfqpoint{0.635535in}{1.459855in}}%
\pgfpathlineto{\pgfqpoint{0.647411in}{1.430496in}}%
\pgfpathlineto{\pgfqpoint{0.659211in}{1.406686in}}%
\pgfpathlineto{\pgfqpoint{0.670453in}{1.388357in}}%
\pgfpathlineto{\pgfqpoint{0.682043in}{1.373522in}}%
\pgfpathlineto{\pgfqpoint{0.692023in}{1.363918in}}%
\pgfpathlineto{\pgfqpoint{0.701373in}{1.357594in}}%
\pgfpathlineto{\pgfqpoint{0.709564in}{1.354285in}}%
\pgfpathlineto{\pgfqpoint{0.718193in}{1.353262in}}%
\pgfpathlineto{\pgfqpoint{0.724634in}{1.354360in}}%
\pgfpathlineto{\pgfqpoint{0.732587in}{1.358313in}}%
\pgfpathlineto{\pgfqpoint{0.739387in}{1.364490in}}%
\pgfpathlineto{\pgfqpoint{0.746235in}{1.374132in}}%
\pgfpathlineto{\pgfqpoint{0.752543in}{1.387190in}}%
\pgfpathlineto{\pgfqpoint{0.758728in}{1.405789in}}%
\pgfpathlineto{\pgfqpoint{0.764192in}{1.429874in}}%
\pgfpathlineto{\pgfqpoint{0.769028in}{1.461750in}}%
\pgfpathlineto{\pgfqpoint{0.773352in}{1.506276in}}%
\pgfpathlineto{\pgfqpoint{0.777372in}{1.573324in}}%
\pgfpathlineto{\pgfqpoint{0.790423in}{1.821781in}}%
\pgfpathlineto{\pgfqpoint{0.797798in}{1.903486in}}%
\pgfpathlineto{\pgfqpoint{0.806442in}{1.974988in}}%
\pgfpathlineto{\pgfqpoint{0.818016in}{2.050992in}}%
\pgfpathlineto{\pgfqpoint{0.830041in}{2.111677in}}%
\pgfpathlineto{\pgfqpoint{0.844798in}{2.174139in}}%
\pgfpathlineto{\pgfqpoint{0.859371in}{2.223675in}}%
\pgfpathlineto{\pgfqpoint{0.874528in}{2.267670in}}%
\pgfpathlineto{\pgfqpoint{0.890793in}{2.308421in}}%
\pgfpathlineto{\pgfqpoint{0.907994in}{2.345874in}}%
\pgfpathlineto{\pgfqpoint{0.930313in}{2.388591in}}%
\pgfpathlineto{\pgfqpoint{0.950423in}{2.421089in}}%
\pgfpathlineto{\pgfqpoint{0.972297in}{2.452060in}}%
\pgfpathlineto{\pgfqpoint{0.994376in}{2.479544in}}%
\pgfpathlineto{\pgfqpoint{1.017907in}{2.505402in}}%
\pgfpathlineto{\pgfqpoint{1.042788in}{2.529549in}}%
\pgfpathlineto{\pgfqpoint{1.068893in}{2.551942in}}%
\pgfpathlineto{\pgfqpoint{1.090645in}{2.568427in}}%
\pgfpathlineto{\pgfqpoint{1.098483in}{2.572655in}}%
\pgfpathlineto{\pgfqpoint{1.128532in}{2.592743in}}%
\pgfpathlineto{\pgfqpoint{1.161427in}{2.612066in}}%
\pgfpathlineto{\pgfqpoint{1.197144in}{2.630415in}}%
\pgfpathlineto{\pgfqpoint{1.235649in}{2.647586in}}%
\pgfpathlineto{\pgfqpoint{1.324905in}{2.679506in}}%
\pgfpathlineto{\pgfqpoint{1.375562in}{2.693945in}}%
\pgfpathlineto{\pgfqpoint{1.430898in}{2.707315in}}%
\pgfpathlineto{\pgfqpoint{1.481774in}{2.719607in}}%
\pgfpathlineto{\pgfqpoint{1.490283in}{2.721581in}}%
\pgfpathlineto{\pgfqpoint{1.544114in}{2.730334in}}%
\pgfpathlineto{\pgfqpoint{1.617514in}{2.740721in}}%
\pgfpathlineto{\pgfqpoint{1.697610in}{2.749780in}}%
\pgfpathlineto{\pgfqpoint{1.786537in}{2.757558in}}%
\pgfpathlineto{\pgfqpoint{1.875589in}{2.763160in}}%
\pgfpathlineto{\pgfqpoint{1.977751in}{2.767692in}}%
\pgfpathlineto{\pgfqpoint{2.082144in}{2.770143in}}%
\pgfpathlineto{\pgfqpoint{2.190908in}{2.770639in}}%
\pgfpathlineto{\pgfqpoint{2.297487in}{2.768916in}}%
\pgfpathlineto{\pgfqpoint{2.399667in}{2.765008in}}%
\pgfpathlineto{\pgfqpoint{2.486538in}{2.759361in}}%
\pgfpathlineto{\pgfqpoint{2.560249in}{2.752428in}}%
\pgfpathlineto{\pgfqpoint{2.622942in}{2.744422in}}%
\pgfpathlineto{\pgfqpoint{2.676745in}{2.735380in}}%
\pgfpathlineto{\pgfqpoint{2.721624in}{2.725651in}}%
\pgfpathlineto{\pgfqpoint{2.759680in}{2.715130in}}%
\pgfpathlineto{\pgfqpoint{2.790879in}{2.704215in}}%
\pgfpathlineto{\pgfqpoint{2.817252in}{2.692558in}}%
\pgfpathlineto{\pgfqpoint{2.838742in}{2.680539in}}%
\pgfpathlineto{\pgfqpoint{2.855405in}{2.668795in}}%
\pgfpathlineto{\pgfqpoint{2.869135in}{2.656577in}}%
\pgfpathlineto{\pgfqpoint{2.879951in}{2.644327in}}%
\pgfpathlineto{\pgfqpoint{2.889268in}{2.630564in}}%
\pgfpathlineto{\pgfqpoint{2.896761in}{2.615413in}}%
\pgfpathlineto{\pgfqpoint{2.902253in}{2.599177in}}%
\pgfpathlineto{\pgfqpoint{2.905794in}{2.582242in}}%
\pgfpathlineto{\pgfqpoint{2.907758in}{2.562467in}}%
\pgfpathlineto{\pgfqpoint{2.907762in}{2.537587in}}%
\pgfpathlineto{\pgfqpoint{2.905230in}{2.507867in}}%
\pgfpathlineto{\pgfqpoint{2.899651in}{2.471085in}}%
\pgfpathlineto{\pgfqpoint{2.889484in}{2.420127in}}%
\pgfpathlineto{\pgfqpoint{2.827855in}{2.129912in}}%
\pgfpathlineto{\pgfqpoint{2.811968in}{2.037088in}}%
\pgfpathlineto{\pgfqpoint{2.795269in}{1.937271in}}%
\pgfpathlineto{\pgfqpoint{2.793842in}{1.930026in}}%
\pgfpathlineto{\pgfqpoint{2.780940in}{1.824015in}}%
\pgfpathlineto{\pgfqpoint{2.769057in}{1.707818in}}%
\pgfpathlineto{\pgfqpoint{2.760675in}{1.606225in}}%
\pgfpathlineto{\pgfqpoint{2.750415in}{1.462331in}}%
\pgfpathlineto{\pgfqpoint{2.741748in}{1.308321in}}%
\pgfpathlineto{\pgfqpoint{2.734365in}{1.136775in}}%
\pgfpathlineto{\pgfqpoint{2.728938in}{0.960154in}}%
\pgfpathlineto{\pgfqpoint{2.724623in}{0.753611in}}%
\pgfpathlineto{\pgfqpoint{2.722416in}{0.542045in}}%
\pgfpathlineto{\pgfqpoint{2.721262in}{0.405332in}}%
\pgfpathlineto{\pgfqpoint{2.717323in}{0.403443in}}%
\pgfpathlineto{\pgfqpoint{2.706474in}{0.402733in}}%
\pgfpathlineto{\pgfqpoint{2.615110in}{0.402567in}}%
\pgfpathlineto{\pgfqpoint{0.450656in}{0.402571in}}%
\pgfpathlineto{\pgfqpoint{0.450656in}{0.402571in}}%
\pgfusepath{stroke}%
\end{pgfscope}%
\begin{pgfscope}%
\pgfsetrectcap%
\pgfsetmiterjoin%
\pgfsetlinewidth{0.803000pt}%
\definecolor{currentstroke}{rgb}{0.000000,0.000000,0.000000}%
\pgfsetstrokecolor{currentstroke}%
\pgfsetdash{}{0pt}%
\pgfpathmoveto{\pgfqpoint{0.448634in}{0.402556in}}%
\pgfpathlineto{\pgfqpoint{0.448634in}{2.891760in}}%
\pgfusepath{stroke}%
\end{pgfscope}%
\begin{pgfscope}%
\pgfsetrectcap%
\pgfsetmiterjoin%
\pgfsetlinewidth{0.803000pt}%
\definecolor{currentstroke}{rgb}{0.000000,0.000000,0.000000}%
\pgfsetstrokecolor{currentstroke}%
\pgfsetdash{}{0pt}%
\pgfpathmoveto{\pgfqpoint{4.799294in}{0.402556in}}%
\pgfpathlineto{\pgfqpoint{4.799294in}{2.891760in}}%
\pgfusepath{stroke}%
\end{pgfscope}%
\begin{pgfscope}%
\pgfsetrectcap%
\pgfsetmiterjoin%
\pgfsetlinewidth{0.803000pt}%
\definecolor{currentstroke}{rgb}{0.000000,0.000000,0.000000}%
\pgfsetstrokecolor{currentstroke}%
\pgfsetdash{}{0pt}%
\pgfpathmoveto{\pgfqpoint{0.448634in}{0.402556in}}%
\pgfpathlineto{\pgfqpoint{4.799294in}{0.402556in}}%
\pgfusepath{stroke}%
\end{pgfscope}%
\begin{pgfscope}%
\pgfsetrectcap%
\pgfsetmiterjoin%
\pgfsetlinewidth{0.803000pt}%
\definecolor{currentstroke}{rgb}{0.000000,0.000000,0.000000}%
\pgfsetstrokecolor{currentstroke}%
\pgfsetdash{}{0pt}%
\pgfpathmoveto{\pgfqpoint{0.448634in}{2.891760in}}%
\pgfpathlineto{\pgfqpoint{4.799294in}{2.891760in}}%
\pgfusepath{stroke}%
\end{pgfscope}%
\begin{pgfscope}%
\pgfsetbuttcap%
\pgfsetmiterjoin%
\definecolor{currentfill}{rgb}{1.000000,1.000000,1.000000}%
\pgfsetfillcolor{currentfill}%
\pgfsetfillopacity{0.500000}%
\pgfsetlinewidth{1.003750pt}%
\definecolor{currentstroke}{rgb}{0.800000,0.800000,0.800000}%
\pgfsetstrokecolor{currentstroke}%
\pgfsetstrokeopacity{0.500000}%
\pgfsetdash{}{0pt}%
\pgfpathmoveto{\pgfqpoint{3.700085in}{1.210602in}}%
\pgfpathlineto{\pgfqpoint{4.494389in}{1.210602in}}%
\pgfpathquadraticcurveto{\pgfqpoint{4.522167in}{1.210602in}}{\pgfqpoint{4.522167in}{1.238379in}}%
\pgfpathlineto{\pgfqpoint{4.522167in}{2.421157in}}%
\pgfpathquadraticcurveto{\pgfqpoint{4.522167in}{2.448935in}}{\pgfqpoint{4.494389in}{2.448935in}}%
\pgfpathlineto{\pgfqpoint{3.700085in}{2.448935in}}%
\pgfpathquadraticcurveto{\pgfqpoint{3.672307in}{2.448935in}}{\pgfqpoint{3.672307in}{2.421157in}}%
\pgfpathlineto{\pgfqpoint{3.672307in}{1.238379in}}%
\pgfpathquadraticcurveto{\pgfqpoint{3.672307in}{1.210602in}}{\pgfqpoint{3.700085in}{1.210602in}}%
\pgfpathclose%
\pgfusepath{stroke,fill}%
\end{pgfscope}%
\begin{pgfscope}%
\pgfsetrectcap%
\pgfsetroundjoin%
\pgfsetlinewidth{1.003750pt}%
\definecolor{currentstroke}{rgb}{1.000000,0.388235,0.278431}%
\pgfsetstrokecolor{currentstroke}%
\pgfsetdash{}{0pt}%
\pgfpathmoveto{\pgfqpoint{3.727863in}{2.344768in}}%
\pgfpathlineto{\pgfqpoint{3.797307in}{2.344768in}}%
\pgfusepath{stroke}%
\end{pgfscope}%
\begin{pgfscope}%
\pgftext[x=3.908418in,y=2.296157in,left,base]{\rmfamily\fontsize{10.000000}{12.000000}\selectfont \textnormal{Reference}}%
\end{pgfscope}%
\begin{pgfscope}%
\pgfsetrectcap%
\pgfsetroundjoin%
\pgfsetlinewidth{1.003750pt}%
\definecolor{currentstroke}{rgb}{0.121569,0.466667,0.705882}%
\pgfsetstrokecolor{currentstroke}%
\pgfsetdash{}{0pt}%
\pgfpathmoveto{\pgfqpoint{3.727863in}{2.145324in}}%
\pgfpathlineto{\pgfqpoint{3.797307in}{2.145324in}}%
\pgfusepath{stroke}%
\end{pgfscope}%
\begin{pgfscope}%
\pgftext[x=3.908418in,y=2.096713in,left,base]{\rmfamily\fontsize{10.000000}{12.000000}\selectfont \(\displaystyle h=10^{{-}5}\)}%
\end{pgfscope}%
\begin{pgfscope}%
\pgfsetrectcap%
\pgfsetroundjoin%
\pgfsetlinewidth{1.003750pt}%
\definecolor{currentstroke}{rgb}{1.000000,0.498039,0.054902}%
\pgfsetstrokecolor{currentstroke}%
\pgfsetdash{}{0pt}%
\pgfpathmoveto{\pgfqpoint{3.727863in}{1.945879in}}%
\pgfpathlineto{\pgfqpoint{3.797307in}{1.945879in}}%
\pgfusepath{stroke}%
\end{pgfscope}%
\begin{pgfscope}%
\pgftext[x=3.908418in,y=1.897268in,left,base]{\rmfamily\fontsize{10.000000}{12.000000}\selectfont \(\displaystyle h=10^{{-}4}\)}%
\end{pgfscope}%
\begin{pgfscope}%
\pgfsetrectcap%
\pgfsetroundjoin%
\pgfsetlinewidth{1.003750pt}%
\definecolor{currentstroke}{rgb}{0.172549,0.627451,0.172549}%
\pgfsetstrokecolor{currentstroke}%
\pgfsetdash{}{0pt}%
\pgfpathmoveto{\pgfqpoint{3.727863in}{1.746435in}}%
\pgfpathlineto{\pgfqpoint{3.797307in}{1.746435in}}%
\pgfusepath{stroke}%
\end{pgfscope}%
\begin{pgfscope}%
\pgftext[x=3.908418in,y=1.697824in,left,base]{\rmfamily\fontsize{10.000000}{12.000000}\selectfont \(\displaystyle h=10^{{-}3}\)}%
\end{pgfscope}%
\begin{pgfscope}%
\pgfsetrectcap%
\pgfsetroundjoin%
\pgfsetlinewidth{1.003750pt}%
\definecolor{currentstroke}{rgb}{0.839216,0.152941,0.156863}%
\pgfsetstrokecolor{currentstroke}%
\pgfsetdash{}{0pt}%
\pgfpathmoveto{\pgfqpoint{3.727863in}{1.546991in}}%
\pgfpathlineto{\pgfqpoint{3.797307in}{1.546991in}}%
\pgfusepath{stroke}%
\end{pgfscope}%
\begin{pgfscope}%
\pgftext[x=3.908418in,y=1.498379in,left,base]{\rmfamily\fontsize{10.000000}{12.000000}\selectfont \(\displaystyle h=10^{{-}2}\)}%
\end{pgfscope}%
\begin{pgfscope}%
\pgfsetrectcap%
\pgfsetroundjoin%
\pgfsetlinewidth{1.003750pt}%
\definecolor{currentstroke}{rgb}{0.580392,0.403922,0.741176}%
\pgfsetstrokecolor{currentstroke}%
\pgfsetdash{}{0pt}%
\pgfpathmoveto{\pgfqpoint{3.727863in}{1.347546in}}%
\pgfpathlineto{\pgfqpoint{3.797307in}{1.347546in}}%
\pgfusepath{stroke}%
\end{pgfscope}%
\begin{pgfscope}%
\pgftext[x=3.908418in,y=1.298935in,left,base]{\rmfamily\fontsize{10.000000}{12.000000}\selectfont \(\displaystyle h=10^{{-}1}\)}%
\end{pgfscope}%
\end{pgfpicture}%
\makeatother%
\endgroup%

    \caption[LCS curves found by means of the Kutta integration scheme]{
        LCS curves found by means of the Kutta integration scheme. The
        reference LCS, as shown by itself in figure~\ref{fig:referencelcs},
        is dashed on the top layer. There are some disparities with
        regards to the reference both for the two largest numerical
        time step lengths considered. These are most pronounced in the lower
        left corner, near the leftmost `U' shape by $y=0.4$ and the `$\cap$'
        shape near $x=1.25$.}
    \label{fig:lcs_rk3}
\end{figure}

\clearpage
\begin{figure}[htpb]
    \centering
    \includegraphics[width=0.8\linewidth]{figures/lcs_figures/rk4.pdf}
    \caption[LCS curves found by means of the classical Runge-Kutta integration scheme]{
        LCS curves found by means of the classical Runge-Kutta integration scheme. The
        reference LCS, as shown by itself in figure~\ref{fig:referencelcs},
        is plotted on the bottom layer. The only visible discrepancy belongs
        to the second largest numerical time step length considered, and is
        located in the lower left corner.}
    \label{fig:lcs_rk4}
\end{figure}


\subsubsection{Adaptive stepsize methods}
\label{ssub:adaptive_stepsize_methods}

The adaptive stepsize methods under consideration are the Bogacki-Shampine
3(2) and 5(4) methods, and the Dormand-Prince 5(4) and 8(7) methods. The digit
outside of the parentheses indicates the order of the solution which is used
to continue the integration, while the digit within the parentheses indicates
the order of the interpolant solution. Note that the concept of \emph{order}
does not translate directly from singlestep methods, as a direct consequence
of the adaptive time step. Although the \emph{local} errors of each integration
step scale as per~\cref{eq:rungekuttaorder}, the bound on the \emph{global}
(i.e., observable) error suggested in~\cref{eq:rungekuttaglobalorderestimate}
is invalid, as the time step is, in principle, different for each integration
step. Generally, lower order methods are more suitable than higher order methods
for cases where crude approximations of the solution are sufficient.
\citeauthor{bogacki1989pair} argue that their methods
outperform other methods of the same order
\parencite{bogacki1989pair,bogacki1996efficient}, a notion which, for the 5(4)
method, is supported by
\textcite[p.194]{hairer1993solving}.

Butcher tableau representations of the aforementioned adaptive stepsize methods
can be found in
\cref{tab:butcherbs32,tab:butcherbs54,tab:butcherdopri54,tab:butcherdopri87},
the latter of which has been typeset in landscape orientation for the reader's
convenience. Three of the methods, namely the Bogacki-Shampine 3(2) and 5(4)
methods, in addition to the Dormand-Prince 5(4) method, possess the so-called
\emph{First Same As Last} property. This means that the last function evaluation
of an accepted step is exactly the same as the first function evaluation of the
next step. The notions of accepted and rejected integration steps will be
elaborated upon in
\cref{sub:on_the_implementation_of_embedded_runge_kutta_methods}. The
\emph{First Same As Last} property is readily apparent from their Butcher
tableaus, where the $b$ coefficients correspond exactly with the last row of the
Runge-Kutta matrix. This property reduces the computational cost of a successive
step. Moreover, the Bogacki-Shampine 5(4) method yields \emph{two} interpolant
solutions. The details on how these were used, will be presented in
\cref{sub:on_the_implementation_of_embedded_runge_kutta_methods}.
%\clearpage
\begin{figure}[htpb]
    \centering
    %% Creator: Matplotlib, PGF backend
%%
%% To include the figure in your LaTeX document, write
%%   \input{<filename>.pgf}
%%
%% Make sure the required packages are loaded in your preamble
%%   \usepackage{pgf}
%%
%% Figures using additional raster images can only be included by \input if
%% they are in the same directory as the main LaTeX file. For loading figures
%% from other directories you can use the `import` package
%%   \usepackage{import}
%% and then include the figures with
%%   \import{<path to file>}{<filename>.pgf}
%%
%% Matplotlib used the following preamble
%%   \usepackage[utf8x]{inputenc}
%%   \usepackage[T1]{fontenc}
%%   \usepackage[]{libertine}\usepackage[libertine]{newtxmath}
%%
\begingroup%
\makeatletter%
\begin{pgfpicture}%
\pgfpathrectangle{\pgfpointorigin}{\pgfqpoint{5.050000in}{3.100000in}}%
\pgfusepath{use as bounding box, clip}%
\begin{pgfscope}%
\pgfsetbuttcap%
\pgfsetmiterjoin%
\definecolor{currentfill}{rgb}{1.000000,1.000000,1.000000}%
\pgfsetfillcolor{currentfill}%
\pgfsetlinewidth{0.000000pt}%
\definecolor{currentstroke}{rgb}{1.000000,1.000000,1.000000}%
\pgfsetstrokecolor{currentstroke}%
\pgfsetdash{}{0pt}%
\pgfpathmoveto{\pgfqpoint{0.000000in}{0.000000in}}%
\pgfpathlineto{\pgfqpoint{5.050000in}{0.000000in}}%
\pgfpathlineto{\pgfqpoint{5.050000in}{3.100000in}}%
\pgfpathlineto{\pgfqpoint{0.000000in}{3.100000in}}%
\pgfpathclose%
\pgfusepath{fill}%
\end{pgfscope}%
\begin{pgfscope}%
\pgfsetbuttcap%
\pgfsetmiterjoin%
\definecolor{currentfill}{rgb}{1.000000,1.000000,1.000000}%
\pgfsetfillcolor{currentfill}%
\pgfsetlinewidth{0.000000pt}%
\definecolor{currentstroke}{rgb}{0.000000,0.000000,0.000000}%
\pgfsetstrokecolor{currentstroke}%
\pgfsetstrokeopacity{0.000000}%
\pgfsetdash{}{0pt}%
\pgfpathmoveto{\pgfqpoint{0.448634in}{0.402556in}}%
\pgfpathlineto{\pgfqpoint{4.799294in}{0.402556in}}%
\pgfpathlineto{\pgfqpoint{4.799294in}{2.891760in}}%
\pgfpathlineto{\pgfqpoint{0.448634in}{2.891760in}}%
\pgfpathclose%
\pgfusepath{fill}%
\end{pgfscope}%
\begin{pgfscope}%
\pgfsetbuttcap%
\pgfsetroundjoin%
\definecolor{currentfill}{rgb}{0.000000,0.000000,0.000000}%
\pgfsetfillcolor{currentfill}%
\pgfsetlinewidth{0.803000pt}%
\definecolor{currentstroke}{rgb}{0.000000,0.000000,0.000000}%
\pgfsetstrokecolor{currentstroke}%
\pgfsetdash{}{0pt}%
\pgfsys@defobject{currentmarker}{\pgfqpoint{0.000000in}{-0.048611in}}{\pgfqpoint{0.000000in}{0.000000in}}{%
\pgfpathmoveto{\pgfqpoint{0.000000in}{0.000000in}}%
\pgfpathlineto{\pgfqpoint{0.000000in}{-0.048611in}}%
\pgfusepath{stroke,fill}%
}%
\begin{pgfscope}%
\pgfsys@transformshift{0.448634in}{0.402556in}%
\pgfsys@useobject{currentmarker}{}%
\end{pgfscope}%
\end{pgfscope}%
\begin{pgfscope}%
\pgftext[x=0.448634in,y=0.305334in,,top]{\rmfamily\fontsize{12.000000}{14.400000}\selectfont \(\displaystyle 0.00\)}%
\end{pgfscope}%
\begin{pgfscope}%
\pgfsetbuttcap%
\pgfsetroundjoin%
\definecolor{currentfill}{rgb}{0.000000,0.000000,0.000000}%
\pgfsetfillcolor{currentfill}%
\pgfsetlinewidth{0.803000pt}%
\definecolor{currentstroke}{rgb}{0.000000,0.000000,0.000000}%
\pgfsetstrokecolor{currentstroke}%
\pgfsetdash{}{0pt}%
\pgfsys@defobject{currentmarker}{\pgfqpoint{0.000000in}{-0.048611in}}{\pgfqpoint{0.000000in}{0.000000in}}{%
\pgfpathmoveto{\pgfqpoint{0.000000in}{0.000000in}}%
\pgfpathlineto{\pgfqpoint{0.000000in}{-0.048611in}}%
\pgfusepath{stroke,fill}%
}%
\begin{pgfscope}%
\pgfsys@transformshift{0.992466in}{0.402556in}%
\pgfsys@useobject{currentmarker}{}%
\end{pgfscope}%
\end{pgfscope}%
\begin{pgfscope}%
\pgftext[x=0.992466in,y=0.305334in,,top]{\rmfamily\fontsize{12.000000}{14.400000}\selectfont \(\displaystyle 0.25\)}%
\end{pgfscope}%
\begin{pgfscope}%
\pgfsetbuttcap%
\pgfsetroundjoin%
\definecolor{currentfill}{rgb}{0.000000,0.000000,0.000000}%
\pgfsetfillcolor{currentfill}%
\pgfsetlinewidth{0.803000pt}%
\definecolor{currentstroke}{rgb}{0.000000,0.000000,0.000000}%
\pgfsetstrokecolor{currentstroke}%
\pgfsetdash{}{0pt}%
\pgfsys@defobject{currentmarker}{\pgfqpoint{0.000000in}{-0.048611in}}{\pgfqpoint{0.000000in}{0.000000in}}{%
\pgfpathmoveto{\pgfqpoint{0.000000in}{0.000000in}}%
\pgfpathlineto{\pgfqpoint{0.000000in}{-0.048611in}}%
\pgfusepath{stroke,fill}%
}%
\begin{pgfscope}%
\pgfsys@transformshift{1.536299in}{0.402556in}%
\pgfsys@useobject{currentmarker}{}%
\end{pgfscope}%
\end{pgfscope}%
\begin{pgfscope}%
\pgftext[x=1.536299in,y=0.305334in,,top]{\rmfamily\fontsize{12.000000}{14.400000}\selectfont \(\displaystyle 0.50\)}%
\end{pgfscope}%
\begin{pgfscope}%
\pgfsetbuttcap%
\pgfsetroundjoin%
\definecolor{currentfill}{rgb}{0.000000,0.000000,0.000000}%
\pgfsetfillcolor{currentfill}%
\pgfsetlinewidth{0.803000pt}%
\definecolor{currentstroke}{rgb}{0.000000,0.000000,0.000000}%
\pgfsetstrokecolor{currentstroke}%
\pgfsetdash{}{0pt}%
\pgfsys@defobject{currentmarker}{\pgfqpoint{0.000000in}{-0.048611in}}{\pgfqpoint{0.000000in}{0.000000in}}{%
\pgfpathmoveto{\pgfqpoint{0.000000in}{0.000000in}}%
\pgfpathlineto{\pgfqpoint{0.000000in}{-0.048611in}}%
\pgfusepath{stroke,fill}%
}%
\begin{pgfscope}%
\pgfsys@transformshift{2.080131in}{0.402556in}%
\pgfsys@useobject{currentmarker}{}%
\end{pgfscope}%
\end{pgfscope}%
\begin{pgfscope}%
\pgftext[x=2.080131in,y=0.305334in,,top]{\rmfamily\fontsize{12.000000}{14.400000}\selectfont \(\displaystyle 0.75\)}%
\end{pgfscope}%
\begin{pgfscope}%
\pgfsetbuttcap%
\pgfsetroundjoin%
\definecolor{currentfill}{rgb}{0.000000,0.000000,0.000000}%
\pgfsetfillcolor{currentfill}%
\pgfsetlinewidth{0.803000pt}%
\definecolor{currentstroke}{rgb}{0.000000,0.000000,0.000000}%
\pgfsetstrokecolor{currentstroke}%
\pgfsetdash{}{0pt}%
\pgfsys@defobject{currentmarker}{\pgfqpoint{0.000000in}{-0.048611in}}{\pgfqpoint{0.000000in}{0.000000in}}{%
\pgfpathmoveto{\pgfqpoint{0.000000in}{0.000000in}}%
\pgfpathlineto{\pgfqpoint{0.000000in}{-0.048611in}}%
\pgfusepath{stroke,fill}%
}%
\begin{pgfscope}%
\pgfsys@transformshift{2.623964in}{0.402556in}%
\pgfsys@useobject{currentmarker}{}%
\end{pgfscope}%
\end{pgfscope}%
\begin{pgfscope}%
\pgftext[x=2.623964in,y=0.305334in,,top]{\rmfamily\fontsize{12.000000}{14.400000}\selectfont \(\displaystyle 1.00\)}%
\end{pgfscope}%
\begin{pgfscope}%
\pgfsetbuttcap%
\pgfsetroundjoin%
\definecolor{currentfill}{rgb}{0.000000,0.000000,0.000000}%
\pgfsetfillcolor{currentfill}%
\pgfsetlinewidth{0.803000pt}%
\definecolor{currentstroke}{rgb}{0.000000,0.000000,0.000000}%
\pgfsetstrokecolor{currentstroke}%
\pgfsetdash{}{0pt}%
\pgfsys@defobject{currentmarker}{\pgfqpoint{0.000000in}{-0.048611in}}{\pgfqpoint{0.000000in}{0.000000in}}{%
\pgfpathmoveto{\pgfqpoint{0.000000in}{0.000000in}}%
\pgfpathlineto{\pgfqpoint{0.000000in}{-0.048611in}}%
\pgfusepath{stroke,fill}%
}%
\begin{pgfscope}%
\pgfsys@transformshift{3.167797in}{0.402556in}%
\pgfsys@useobject{currentmarker}{}%
\end{pgfscope}%
\end{pgfscope}%
\begin{pgfscope}%
\pgftext[x=3.167797in,y=0.305334in,,top]{\rmfamily\fontsize{12.000000}{14.400000}\selectfont \(\displaystyle 1.25\)}%
\end{pgfscope}%
\begin{pgfscope}%
\pgfsetbuttcap%
\pgfsetroundjoin%
\definecolor{currentfill}{rgb}{0.000000,0.000000,0.000000}%
\pgfsetfillcolor{currentfill}%
\pgfsetlinewidth{0.803000pt}%
\definecolor{currentstroke}{rgb}{0.000000,0.000000,0.000000}%
\pgfsetstrokecolor{currentstroke}%
\pgfsetdash{}{0pt}%
\pgfsys@defobject{currentmarker}{\pgfqpoint{0.000000in}{-0.048611in}}{\pgfqpoint{0.000000in}{0.000000in}}{%
\pgfpathmoveto{\pgfqpoint{0.000000in}{0.000000in}}%
\pgfpathlineto{\pgfqpoint{0.000000in}{-0.048611in}}%
\pgfusepath{stroke,fill}%
}%
\begin{pgfscope}%
\pgfsys@transformshift{3.711629in}{0.402556in}%
\pgfsys@useobject{currentmarker}{}%
\end{pgfscope}%
\end{pgfscope}%
\begin{pgfscope}%
\pgftext[x=3.711629in,y=0.305334in,,top]{\rmfamily\fontsize{12.000000}{14.400000}\selectfont \(\displaystyle 1.50\)}%
\end{pgfscope}%
\begin{pgfscope}%
\pgfsetbuttcap%
\pgfsetroundjoin%
\definecolor{currentfill}{rgb}{0.000000,0.000000,0.000000}%
\pgfsetfillcolor{currentfill}%
\pgfsetlinewidth{0.803000pt}%
\definecolor{currentstroke}{rgb}{0.000000,0.000000,0.000000}%
\pgfsetstrokecolor{currentstroke}%
\pgfsetdash{}{0pt}%
\pgfsys@defobject{currentmarker}{\pgfqpoint{0.000000in}{-0.048611in}}{\pgfqpoint{0.000000in}{0.000000in}}{%
\pgfpathmoveto{\pgfqpoint{0.000000in}{0.000000in}}%
\pgfpathlineto{\pgfqpoint{0.000000in}{-0.048611in}}%
\pgfusepath{stroke,fill}%
}%
\begin{pgfscope}%
\pgfsys@transformshift{4.255462in}{0.402556in}%
\pgfsys@useobject{currentmarker}{}%
\end{pgfscope}%
\end{pgfscope}%
\begin{pgfscope}%
\pgftext[x=4.255462in,y=0.305334in,,top]{\rmfamily\fontsize{12.000000}{14.400000}\selectfont \(\displaystyle 1.75\)}%
\end{pgfscope}%
\begin{pgfscope}%
\pgfsetbuttcap%
\pgfsetroundjoin%
\definecolor{currentfill}{rgb}{0.000000,0.000000,0.000000}%
\pgfsetfillcolor{currentfill}%
\pgfsetlinewidth{0.803000pt}%
\definecolor{currentstroke}{rgb}{0.000000,0.000000,0.000000}%
\pgfsetstrokecolor{currentstroke}%
\pgfsetdash{}{0pt}%
\pgfsys@defobject{currentmarker}{\pgfqpoint{0.000000in}{-0.048611in}}{\pgfqpoint{0.000000in}{0.000000in}}{%
\pgfpathmoveto{\pgfqpoint{0.000000in}{0.000000in}}%
\pgfpathlineto{\pgfqpoint{0.000000in}{-0.048611in}}%
\pgfusepath{stroke,fill}%
}%
\begin{pgfscope}%
\pgfsys@transformshift{4.799294in}{0.402556in}%
\pgfsys@useobject{currentmarker}{}%
\end{pgfscope}%
\end{pgfscope}%
\begin{pgfscope}%
\pgftext[x=4.799294in,y=0.305334in,,top]{\rmfamily\fontsize{12.000000}{14.400000}\selectfont \(\displaystyle 2.00\)}%
\end{pgfscope}%
\begin{pgfscope}%
\pgfsetbuttcap%
\pgfsetroundjoin%
\definecolor{currentfill}{rgb}{0.000000,0.000000,0.000000}%
\pgfsetfillcolor{currentfill}%
\pgfsetlinewidth{0.803000pt}%
\definecolor{currentstroke}{rgb}{0.000000,0.000000,0.000000}%
\pgfsetstrokecolor{currentstroke}%
\pgfsetdash{}{0pt}%
\pgfsys@defobject{currentmarker}{\pgfqpoint{-0.048611in}{0.000000in}}{\pgfqpoint{0.000000in}{0.000000in}}{%
\pgfpathmoveto{\pgfqpoint{0.000000in}{0.000000in}}%
\pgfpathlineto{\pgfqpoint{-0.048611in}{0.000000in}}%
\pgfusepath{stroke,fill}%
}%
\begin{pgfscope}%
\pgfsys@transformshift{0.448634in}{0.402556in}%
\pgfsys@useobject{currentmarker}{}%
\end{pgfscope}%
\end{pgfscope}%
\begin{pgfscope}%
\pgftext[x=0.149245in,y=0.345015in,left,base]{\rmfamily\fontsize{12.000000}{14.400000}\selectfont \(\displaystyle 0.0\)}%
\end{pgfscope}%
\begin{pgfscope}%
\pgfsetbuttcap%
\pgfsetroundjoin%
\definecolor{currentfill}{rgb}{0.000000,0.000000,0.000000}%
\pgfsetfillcolor{currentfill}%
\pgfsetlinewidth{0.803000pt}%
\definecolor{currentstroke}{rgb}{0.000000,0.000000,0.000000}%
\pgfsetstrokecolor{currentstroke}%
\pgfsetdash{}{0pt}%
\pgfsys@defobject{currentmarker}{\pgfqpoint{-0.048611in}{0.000000in}}{\pgfqpoint{0.000000in}{0.000000in}}{%
\pgfpathmoveto{\pgfqpoint{0.000000in}{0.000000in}}%
\pgfpathlineto{\pgfqpoint{-0.048611in}{0.000000in}}%
\pgfusepath{stroke,fill}%
}%
\begin{pgfscope}%
\pgfsys@transformshift{0.448634in}{0.900397in}%
\pgfsys@useobject{currentmarker}{}%
\end{pgfscope}%
\end{pgfscope}%
\begin{pgfscope}%
\pgftext[x=0.149245in,y=0.842855in,left,base]{\rmfamily\fontsize{12.000000}{14.400000}\selectfont \(\displaystyle 0.2\)}%
\end{pgfscope}%
\begin{pgfscope}%
\pgfsetbuttcap%
\pgfsetroundjoin%
\definecolor{currentfill}{rgb}{0.000000,0.000000,0.000000}%
\pgfsetfillcolor{currentfill}%
\pgfsetlinewidth{0.803000pt}%
\definecolor{currentstroke}{rgb}{0.000000,0.000000,0.000000}%
\pgfsetstrokecolor{currentstroke}%
\pgfsetdash{}{0pt}%
\pgfsys@defobject{currentmarker}{\pgfqpoint{-0.048611in}{0.000000in}}{\pgfqpoint{0.000000in}{0.000000in}}{%
\pgfpathmoveto{\pgfqpoint{0.000000in}{0.000000in}}%
\pgfpathlineto{\pgfqpoint{-0.048611in}{0.000000in}}%
\pgfusepath{stroke,fill}%
}%
\begin{pgfscope}%
\pgfsys@transformshift{0.448634in}{1.398238in}%
\pgfsys@useobject{currentmarker}{}%
\end{pgfscope}%
\end{pgfscope}%
\begin{pgfscope}%
\pgftext[x=0.149245in,y=1.340696in,left,base]{\rmfamily\fontsize{12.000000}{14.400000}\selectfont \(\displaystyle 0.4\)}%
\end{pgfscope}%
\begin{pgfscope}%
\pgfsetbuttcap%
\pgfsetroundjoin%
\definecolor{currentfill}{rgb}{0.000000,0.000000,0.000000}%
\pgfsetfillcolor{currentfill}%
\pgfsetlinewidth{0.803000pt}%
\definecolor{currentstroke}{rgb}{0.000000,0.000000,0.000000}%
\pgfsetstrokecolor{currentstroke}%
\pgfsetdash{}{0pt}%
\pgfsys@defobject{currentmarker}{\pgfqpoint{-0.048611in}{0.000000in}}{\pgfqpoint{0.000000in}{0.000000in}}{%
\pgfpathmoveto{\pgfqpoint{0.000000in}{0.000000in}}%
\pgfpathlineto{\pgfqpoint{-0.048611in}{0.000000in}}%
\pgfusepath{stroke,fill}%
}%
\begin{pgfscope}%
\pgfsys@transformshift{0.448634in}{1.896079in}%
\pgfsys@useobject{currentmarker}{}%
\end{pgfscope}%
\end{pgfscope}%
\begin{pgfscope}%
\pgftext[x=0.149245in,y=1.838537in,left,base]{\rmfamily\fontsize{12.000000}{14.400000}\selectfont \(\displaystyle 0.6\)}%
\end{pgfscope}%
\begin{pgfscope}%
\pgfsetbuttcap%
\pgfsetroundjoin%
\definecolor{currentfill}{rgb}{0.000000,0.000000,0.000000}%
\pgfsetfillcolor{currentfill}%
\pgfsetlinewidth{0.803000pt}%
\definecolor{currentstroke}{rgb}{0.000000,0.000000,0.000000}%
\pgfsetstrokecolor{currentstroke}%
\pgfsetdash{}{0pt}%
\pgfsys@defobject{currentmarker}{\pgfqpoint{-0.048611in}{0.000000in}}{\pgfqpoint{0.000000in}{0.000000in}}{%
\pgfpathmoveto{\pgfqpoint{0.000000in}{0.000000in}}%
\pgfpathlineto{\pgfqpoint{-0.048611in}{0.000000in}}%
\pgfusepath{stroke,fill}%
}%
\begin{pgfscope}%
\pgfsys@transformshift{0.448634in}{2.393919in}%
\pgfsys@useobject{currentmarker}{}%
\end{pgfscope}%
\end{pgfscope}%
\begin{pgfscope}%
\pgftext[x=0.149245in,y=2.336378in,left,base]{\rmfamily\fontsize{12.000000}{14.400000}\selectfont \(\displaystyle 0.8\)}%
\end{pgfscope}%
\begin{pgfscope}%
\pgfsetbuttcap%
\pgfsetroundjoin%
\definecolor{currentfill}{rgb}{0.000000,0.000000,0.000000}%
\pgfsetfillcolor{currentfill}%
\pgfsetlinewidth{0.803000pt}%
\definecolor{currentstroke}{rgb}{0.000000,0.000000,0.000000}%
\pgfsetstrokecolor{currentstroke}%
\pgfsetdash{}{0pt}%
\pgfsys@defobject{currentmarker}{\pgfqpoint{-0.048611in}{0.000000in}}{\pgfqpoint{0.000000in}{0.000000in}}{%
\pgfpathmoveto{\pgfqpoint{0.000000in}{0.000000in}}%
\pgfpathlineto{\pgfqpoint{-0.048611in}{0.000000in}}%
\pgfusepath{stroke,fill}%
}%
\begin{pgfscope}%
\pgfsys@transformshift{0.448634in}{2.891760in}%
\pgfsys@useobject{currentmarker}{}%
\end{pgfscope}%
\end{pgfscope}%
\begin{pgfscope}%
\pgftext[x=0.149245in,y=2.834219in,left,base]{\rmfamily\fontsize{12.000000}{14.400000}\selectfont \(\displaystyle 1.0\)}%
\end{pgfscope}%
\begin{pgfscope}%
\pgfpathrectangle{\pgfqpoint{0.448634in}{0.402556in}}{\pgfqpoint{4.350661in}{2.489204in}} %
\pgfusepath{clip}%
\pgfsetrectcap%
\pgfsetroundjoin%
\pgfsetlinewidth{1.003750pt}%
\definecolor{currentstroke}{rgb}{0.121569,0.466667,0.705882}%
\pgfsetstrokecolor{currentstroke}%
\pgfsetdash{}{0pt}%
\pgfpathmoveto{\pgfqpoint{0.448634in}{2.896245in}}%
\pgfpathlineto{\pgfqpoint{0.448593in}{0.407043in}}%
\pgfpathlineto{\pgfqpoint{0.448593in}{0.407043in}}%
\pgfusepath{stroke}%
\end{pgfscope}%
\begin{pgfscope}%
\pgfpathrectangle{\pgfqpoint{0.448634in}{0.402556in}}{\pgfqpoint{4.350661in}{2.489204in}} %
\pgfusepath{clip}%
\pgfsetrectcap%
\pgfsetroundjoin%
\pgfsetlinewidth{1.003750pt}%
\definecolor{currentstroke}{rgb}{0.121569,0.466667,0.705882}%
\pgfsetstrokecolor{currentstroke}%
\pgfsetdash{}{0pt}%
\pgfpathmoveto{\pgfqpoint{0.576853in}{1.760817in}}%
\pgfpathlineto{\pgfqpoint{0.569394in}{1.840010in}}%
\pgfpathlineto{\pgfqpoint{0.563209in}{1.929338in}}%
\pgfpathlineto{\pgfqpoint{0.558592in}{2.028764in}}%
\pgfpathlineto{\pgfqpoint{0.555985in}{2.133265in}}%
\pgfpathlineto{\pgfqpoint{0.555566in}{2.237808in}}%
\pgfpathlineto{\pgfqpoint{0.557371in}{2.337352in}}%
\pgfpathlineto{\pgfqpoint{0.561096in}{2.424366in}}%
\pgfpathlineto{\pgfqpoint{0.566403in}{2.498791in}}%
\pgfpathlineto{\pgfqpoint{0.572909in}{2.560570in}}%
\pgfpathlineto{\pgfqpoint{0.580458in}{2.612119in}}%
\pgfpathlineto{\pgfqpoint{0.589086in}{2.655816in}}%
\pgfpathlineto{\pgfqpoint{0.598406in}{2.691589in}}%
\pgfpathlineto{\pgfqpoint{0.608613in}{2.721757in}}%
\pgfpathlineto{\pgfqpoint{0.619241in}{2.746278in}}%
\pgfpathlineto{\pgfqpoint{0.630817in}{2.767339in}}%
\pgfpathlineto{\pgfqpoint{0.642975in}{2.784884in}}%
\pgfpathlineto{\pgfqpoint{0.656813in}{2.800712in}}%
\pgfpathlineto{\pgfqpoint{0.672197in}{2.814549in}}%
\pgfpathlineto{\pgfqpoint{0.688853in}{2.826301in}}%
\pgfpathlineto{\pgfqpoint{0.706461in}{2.836076in}}%
\pgfpathlineto{\pgfqpoint{0.726804in}{2.844875in}}%
\pgfpathlineto{\pgfqpoint{0.751866in}{2.853203in}}%
\pgfpathlineto{\pgfqpoint{0.781631in}{2.860547in}}%
\pgfpathlineto{\pgfqpoint{0.818168in}{2.867054in}}%
\pgfpathlineto{\pgfqpoint{0.863581in}{2.872685in}}%
\pgfpathlineto{\pgfqpoint{0.922160in}{2.877518in}}%
\pgfpathlineto{\pgfqpoint{1.000391in}{2.881567in}}%
\pgfpathlineto{\pgfqpoint{1.111294in}{2.884881in}}%
\pgfpathlineto{\pgfqpoint{1.274428in}{2.887367in}}%
\pgfpathlineto{\pgfqpoint{1.552865in}{2.889263in}}%
\pgfpathlineto{\pgfqpoint{2.107573in}{2.890457in}}%
\pgfpathlineto{\pgfqpoint{3.343161in}{2.890573in}}%
\pgfpathlineto{\pgfqpoint{4.043615in}{2.888941in}}%
\pgfpathlineto{\pgfqpoint{4.289417in}{2.886404in}}%
\pgfpathlineto{\pgfqpoint{4.413375in}{2.883093in}}%
\pgfpathlineto{\pgfqpoint{4.489424in}{2.878997in}}%
\pgfpathlineto{\pgfqpoint{4.541451in}{2.874081in}}%
\pgfpathlineto{\pgfqpoint{4.578100in}{2.868470in}}%
\pgfpathlineto{\pgfqpoint{4.605818in}{2.862092in}}%
\pgfpathlineto{\pgfqpoint{4.626725in}{2.855245in}}%
\pgfpathlineto{\pgfqpoint{4.644925in}{2.847018in}}%
\pgfpathlineto{\pgfqpoint{4.660241in}{2.837590in}}%
\pgfpathlineto{\pgfqpoint{4.672623in}{2.827468in}}%
\pgfpathlineto{\pgfqpoint{4.683751in}{2.815592in}}%
\pgfpathlineto{\pgfqpoint{4.693406in}{2.802135in}}%
\pgfpathlineto{\pgfqpoint{4.702740in}{2.785343in}}%
\pgfpathlineto{\pgfqpoint{4.711277in}{2.765194in}}%
\pgfpathlineto{\pgfqpoint{4.719482in}{2.739484in}}%
\pgfpathlineto{\pgfqpoint{4.726293in}{2.710657in}}%
\pgfpathlineto{\pgfqpoint{4.733259in}{2.671643in}}%
\pgfpathlineto{\pgfqpoint{4.739604in}{2.622396in}}%
\pgfpathlineto{\pgfqpoint{4.745236in}{2.560504in}}%
\pgfpathlineto{\pgfqpoint{4.750164in}{2.481052in}}%
\pgfpathlineto{\pgfqpoint{4.754367in}{2.376618in}}%
\pgfpathlineto{\pgfqpoint{4.757443in}{2.242249in}}%
\pgfpathlineto{\pgfqpoint{4.758977in}{2.075483in}}%
\pgfpathlineto{\pgfqpoint{4.758447in}{1.888795in}}%
\pgfpathlineto{\pgfqpoint{4.755756in}{1.707111in}}%
\pgfpathlineto{\pgfqpoint{4.750925in}{1.532957in}}%
\pgfpathlineto{\pgfqpoint{4.744785in}{1.398727in}}%
\pgfpathlineto{\pgfqpoint{4.737575in}{1.289516in}}%
\pgfpathlineto{\pgfqpoint{4.728714in}{1.190470in}}%
\pgfpathlineto{\pgfqpoint{4.719652in}{1.116521in}}%
\pgfpathlineto{\pgfqpoint{4.710036in}{1.055276in}}%
\pgfpathlineto{\pgfqpoint{4.699503in}{1.001861in}}%
\pgfpathlineto{\pgfqpoint{4.689040in}{0.958690in}}%
\pgfpathlineto{\pgfqpoint{4.677219in}{0.918600in}}%
\pgfpathlineto{\pgfqpoint{4.664034in}{0.881749in}}%
\pgfpathlineto{\pgfqpoint{4.650584in}{0.850492in}}%
\pgfpathlineto{\pgfqpoint{4.636303in}{0.822570in}}%
\pgfpathlineto{\pgfqpoint{4.620207in}{0.795974in}}%
\pgfpathlineto{\pgfqpoint{4.603640in}{0.772901in}}%
\pgfpathlineto{\pgfqpoint{4.585488in}{0.751446in}}%
\pgfpathlineto{\pgfqpoint{4.565874in}{0.731749in}}%
\pgfpathlineto{\pgfqpoint{4.544964in}{0.713879in}}%
\pgfpathlineto{\pgfqpoint{4.522958in}{0.697824in}}%
\pgfpathlineto{\pgfqpoint{4.496157in}{0.681290in}}%
\pgfpathlineto{\pgfqpoint{4.470397in}{0.667953in}}%
\pgfpathlineto{\pgfqpoint{4.439961in}{0.654509in}}%
\pgfpathlineto{\pgfqpoint{4.406841in}{0.642281in}}%
\pgfpathlineto{\pgfqpoint{4.369009in}{0.630748in}}%
\pgfpathlineto{\pgfqpoint{4.326489in}{0.620226in}}%
\pgfpathlineto{\pgfqpoint{4.279327in}{0.610949in}}%
\pgfpathlineto{\pgfqpoint{4.227576in}{0.603085in}}%
\pgfpathlineto{\pgfqpoint{4.173450in}{0.597063in}}%
\pgfpathlineto{\pgfqpoint{4.110511in}{0.592203in}}%
\pgfpathlineto{\pgfqpoint{4.047471in}{0.589537in}}%
\pgfpathlineto{\pgfqpoint{3.977867in}{0.588624in}}%
\pgfpathlineto{\pgfqpoint{3.906093in}{0.589934in}}%
\pgfpathlineto{\pgfqpoint{3.834377in}{0.593496in}}%
\pgfpathlineto{\pgfqpoint{3.767120in}{0.599067in}}%
\pgfpathlineto{\pgfqpoint{3.704364in}{0.606392in}}%
\pgfpathlineto{\pgfqpoint{3.678516in}{0.610510in}}%
\pgfpathlineto{\pgfqpoint{3.620438in}{0.620500in}}%
\pgfpathlineto{\pgfqpoint{3.586319in}{0.628207in}}%
\pgfpathlineto{\pgfqpoint{3.495240in}{0.652428in}}%
\pgfpathlineto{\pgfqpoint{3.451528in}{0.667583in}}%
\pgfpathlineto{\pgfqpoint{3.408538in}{0.685220in}}%
\pgfpathlineto{\pgfqpoint{3.374594in}{0.702001in}}%
\pgfpathlineto{\pgfqpoint{3.345407in}{0.718682in}}%
\pgfpathlineto{\pgfqpoint{3.315236in}{0.738520in}}%
\pgfpathlineto{\pgfqpoint{3.288127in}{0.759290in}}%
\pgfpathlineto{\pgfqpoint{3.264004in}{0.780551in}}%
\pgfpathlineto{\pgfqpoint{3.241208in}{0.803648in}}%
\pgfpathlineto{\pgfqpoint{3.219894in}{0.828530in}}%
\pgfpathlineto{\pgfqpoint{3.200189in}{0.855091in}}%
\pgfpathlineto{\pgfqpoint{3.182177in}{0.883182in}}%
\pgfpathlineto{\pgfqpoint{3.165906in}{0.912633in}}%
\pgfpathlineto{\pgfqpoint{3.150351in}{0.945448in}}%
\pgfpathlineto{\pgfqpoint{3.136682in}{0.979345in}}%
\pgfpathlineto{\pgfqpoint{3.124073in}{1.016460in}}%
\pgfpathlineto{\pgfqpoint{3.112834in}{1.056769in}}%
\pgfpathlineto{\pgfqpoint{3.103046in}{1.100146in}}%
\pgfpathlineto{\pgfqpoint{3.095343in}{1.144071in}}%
\pgfpathlineto{\pgfqpoint{3.089208in}{1.190837in}}%
\pgfpathlineto{\pgfqpoint{3.084595in}{1.242838in}}%
\pgfpathlineto{\pgfqpoint{3.082137in}{1.295031in}}%
\pgfpathlineto{\pgfqpoint{3.081687in}{1.349787in}}%
\pgfpathlineto{\pgfqpoint{3.083451in}{1.406998in}}%
\pgfpathlineto{\pgfqpoint{3.087181in}{1.461589in}}%
\pgfpathlineto{\pgfqpoint{3.093485in}{1.520888in}}%
\pgfpathlineto{\pgfqpoint{3.101823in}{1.577334in}}%
\pgfpathlineto{\pgfqpoint{3.111930in}{1.630856in}}%
\pgfpathlineto{\pgfqpoint{3.124690in}{1.686208in}}%
\pgfpathlineto{\pgfqpoint{3.139178in}{1.738395in}}%
\pgfpathlineto{\pgfqpoint{3.155145in}{1.787366in}}%
\pgfpathlineto{\pgfqpoint{3.172353in}{1.833085in}}%
\pgfpathlineto{\pgfqpoint{3.191618in}{1.877716in}}%
\pgfpathlineto{\pgfqpoint{3.214026in}{1.923261in}}%
\pgfpathlineto{\pgfqpoint{3.236214in}{1.963157in}}%
\pgfpathlineto{\pgfqpoint{3.260178in}{2.001684in}}%
\pgfpathlineto{\pgfqpoint{3.285814in}{2.038776in}}%
\pgfpathlineto{\pgfqpoint{3.314415in}{2.076285in}}%
\pgfpathlineto{\pgfqpoint{3.348944in}{2.117711in}}%
\pgfpathlineto{\pgfqpoint{3.417133in}{2.198022in}}%
\pgfpathlineto{\pgfqpoint{3.426053in}{2.212128in}}%
\pgfpathlineto{\pgfqpoint{3.430798in}{2.223297in}}%
\pgfpathlineto{\pgfqpoint{3.432034in}{2.230603in}}%
\pgfpathlineto{\pgfqpoint{3.430773in}{2.237856in}}%
\pgfpathlineto{\pgfqpoint{3.426621in}{2.243526in}}%
\pgfpathlineto{\pgfqpoint{3.420908in}{2.247084in}}%
\pgfpathlineto{\pgfqpoint{3.412501in}{2.249583in}}%
\pgfpathlineto{\pgfqpoint{3.399499in}{2.250689in}}%
\pgfpathlineto{\pgfqpoint{3.384305in}{2.249671in}}%
\pgfpathlineto{\pgfqpoint{3.364985in}{2.246098in}}%
\pgfpathlineto{\pgfqpoint{3.341804in}{2.239342in}}%
\pgfpathlineto{\pgfqpoint{3.317109in}{2.229682in}}%
\pgfpathlineto{\pgfqpoint{3.291104in}{2.216986in}}%
\pgfpathlineto{\pgfqpoint{3.265928in}{2.202261in}}%
\pgfpathlineto{\pgfqpoint{3.239805in}{2.184361in}}%
\pgfpathlineto{\pgfqpoint{3.214775in}{2.164519in}}%
\pgfpathlineto{\pgfqpoint{3.190900in}{2.142893in}}%
\pgfpathlineto{\pgfqpoint{3.166657in}{2.117912in}}%
\pgfpathlineto{\pgfqpoint{3.143835in}{2.091233in}}%
\pgfpathlineto{\pgfqpoint{3.121079in}{2.061107in}}%
\pgfpathlineto{\pgfqpoint{3.099952in}{2.029463in}}%
\pgfpathlineto{\pgfqpoint{3.079251in}{1.994406in}}%
\pgfpathlineto{\pgfqpoint{3.059218in}{1.955915in}}%
\pgfpathlineto{\pgfqpoint{3.040058in}{1.914015in}}%
\pgfpathlineto{\pgfqpoint{3.022809in}{1.871041in}}%
\pgfpathlineto{\pgfqpoint{3.005790in}{1.822536in}}%
\pgfpathlineto{\pgfqpoint{2.990067in}{1.770819in}}%
\pgfpathlineto{\pgfqpoint{2.975708in}{1.715979in}}%
\pgfpathlineto{\pgfqpoint{2.962284in}{1.655680in}}%
\pgfpathlineto{\pgfqpoint{2.950496in}{1.592386in}}%
\pgfpathlineto{\pgfqpoint{2.940383in}{1.526185in}}%
\pgfpathlineto{\pgfqpoint{2.931745in}{1.454681in}}%
\pgfpathlineto{\pgfqpoint{2.925082in}{1.380399in}}%
\pgfpathlineto{\pgfqpoint{2.920647in}{1.305899in}}%
\pgfpathlineto{\pgfqpoint{2.918444in}{1.231270in}}%
\pgfpathlineto{\pgfqpoint{2.918545in}{1.159087in}}%
\pgfpathlineto{\pgfqpoint{2.920787in}{1.091931in}}%
\pgfpathlineto{\pgfqpoint{2.925177in}{1.027412in}}%
\pgfpathlineto{\pgfqpoint{2.931192in}{0.970580in}}%
\pgfpathlineto{\pgfqpoint{2.938760in}{0.919034in}}%
\pgfpathlineto{\pgfqpoint{2.947651in}{0.872852in}}%
\pgfpathlineto{\pgfqpoint{2.958213in}{0.829714in}}%
\pgfpathlineto{\pgfqpoint{2.969670in}{0.792114in}}%
\pgfpathlineto{\pgfqpoint{2.982463in}{0.757773in}}%
\pgfpathlineto{\pgfqpoint{2.996425in}{0.726812in}}%
\pgfpathlineto{\pgfqpoint{3.011299in}{0.699300in}}%
\pgfpathlineto{\pgfqpoint{3.026739in}{0.675225in}}%
\pgfpathlineto{\pgfqpoint{3.043828in}{0.652656in}}%
\pgfpathlineto{\pgfqpoint{3.062495in}{0.631788in}}%
\pgfpathlineto{\pgfqpoint{3.082602in}{0.612753in}}%
\pgfpathlineto{\pgfqpoint{3.103961in}{0.595592in}}%
\pgfpathlineto{\pgfqpoint{3.128268in}{0.579069in}}%
\pgfpathlineto{\pgfqpoint{3.153537in}{0.564554in}}%
\pgfpathlineto{\pgfqpoint{3.181571in}{0.550952in}}%
\pgfpathlineto{\pgfqpoint{3.214371in}{0.537647in}}%
\pgfpathlineto{\pgfqpoint{3.249846in}{0.525712in}}%
\pgfpathlineto{\pgfqpoint{3.290011in}{0.514571in}}%
\pgfpathlineto{\pgfqpoint{3.334820in}{0.504423in}}%
\pgfpathlineto{\pgfqpoint{3.386372in}{0.494999in}}%
\pgfpathlineto{\pgfqpoint{3.446798in}{0.486257in}}%
\pgfpathlineto{\pgfqpoint{3.518243in}{0.478282in}}%
\pgfpathlineto{\pgfqpoint{3.600685in}{0.471409in}}%
\pgfpathlineto{\pgfqpoint{3.696268in}{0.465713in}}%
\pgfpathlineto{\pgfqpoint{3.807144in}{0.461369in}}%
\pgfpathlineto{\pgfqpoint{3.933291in}{0.458719in}}%
\pgfpathlineto{\pgfqpoint{4.063808in}{0.458211in}}%
\pgfpathlineto{\pgfqpoint{4.187792in}{0.459914in}}%
\pgfpathlineto{\pgfqpoint{4.294335in}{0.463521in}}%
\pgfpathlineto{\pgfqpoint{4.381234in}{0.468574in}}%
\pgfpathlineto{\pgfqpoint{4.450636in}{0.474701in}}%
\pgfpathlineto{\pgfqpoint{4.506850in}{0.481799in}}%
\pgfpathlineto{\pgfqpoint{4.552009in}{0.489658in}}%
\pgfpathlineto{\pgfqpoint{4.588239in}{0.498115in}}%
\pgfpathlineto{\pgfqpoint{4.617656in}{0.507110in}}%
\pgfpathlineto{\pgfqpoint{4.642328in}{0.516843in}}%
\pgfpathlineto{\pgfqpoint{4.664194in}{0.527940in}}%
\pgfpathlineto{\pgfqpoint{4.681238in}{0.538945in}}%
\pgfpathlineto{\pgfqpoint{4.697164in}{0.551953in}}%
\pgfpathlineto{\pgfqpoint{4.710076in}{0.565289in}}%
\pgfpathlineto{\pgfqpoint{4.721578in}{0.580218in}}%
\pgfpathlineto{\pgfqpoint{4.731557in}{0.596521in}}%
\pgfpathlineto{\pgfqpoint{4.741000in}{0.616134in}}%
\pgfpathlineto{\pgfqpoint{4.749521in}{0.639027in}}%
\pgfpathlineto{\pgfqpoint{4.757522in}{0.667450in}}%
\pgfpathlineto{\pgfqpoint{4.764572in}{0.701345in}}%
\pgfpathlineto{\pgfqpoint{4.770840in}{0.743043in}}%
\pgfpathlineto{\pgfqpoint{4.776327in}{0.794934in}}%
\pgfpathlineto{\pgfqpoint{4.781278in}{0.864398in}}%
\pgfpathlineto{\pgfqpoint{4.785468in}{0.956371in}}%
\pgfpathlineto{\pgfqpoint{4.789000in}{1.085745in}}%
\pgfpathlineto{\pgfqpoint{4.791852in}{1.277385in}}%
\pgfpathlineto{\pgfqpoint{4.793959in}{1.581057in}}%
\pgfpathlineto{\pgfqpoint{4.794962in}{2.071429in}}%
\pgfpathlineto{\pgfqpoint{4.793967in}{2.559311in}}%
\pgfpathlineto{\pgfqpoint{4.791733in}{2.745981in}}%
\pgfpathlineto{\pgfqpoint{4.788955in}{2.818091in}}%
\pgfpathlineto{\pgfqpoint{4.785731in}{2.850227in}}%
\pgfpathlineto{\pgfqpoint{4.781879in}{2.867057in}}%
\pgfpathlineto{\pgfqpoint{4.777744in}{2.875780in}}%
\pgfpathlineto{\pgfqpoint{4.773097in}{2.880982in}}%
\pgfpathlineto{\pgfqpoint{4.767363in}{2.884504in}}%
\pgfpathlineto{\pgfqpoint{4.756853in}{2.887622in}}%
\pgfpathlineto{\pgfqpoint{4.739548in}{2.889639in}}%
\pgfpathlineto{\pgfqpoint{4.704762in}{2.890882in}}%
\pgfpathlineto{\pgfqpoint{4.602524in}{2.891538in}}%
\pgfpathlineto{\pgfqpoint{3.952100in}{2.891742in}}%
\pgfpathlineto{\pgfqpoint{0.617321in}{2.890753in}}%
\pgfpathlineto{\pgfqpoint{0.549910in}{2.888858in}}%
\pgfpathlineto{\pgfqpoint{0.521735in}{2.886179in}}%
\pgfpathlineto{\pgfqpoint{0.504666in}{2.882389in}}%
\pgfpathlineto{\pgfqpoint{0.494501in}{2.878011in}}%
\pgfpathlineto{\pgfqpoint{0.487180in}{2.872667in}}%
\pgfpathlineto{\pgfqpoint{0.481152in}{2.865519in}}%
\pgfpathlineto{\pgfqpoint{0.475664in}{2.854804in}}%
\pgfpathlineto{\pgfqpoint{0.471318in}{2.840737in}}%
\pgfpathlineto{\pgfqpoint{0.467301in}{2.818823in}}%
\pgfpathlineto{\pgfqpoint{0.463927in}{2.786700in}}%
\pgfpathlineto{\pgfqpoint{0.460918in}{2.734544in}}%
\pgfpathlineto{\pgfqpoint{0.458363in}{2.647473in}}%
\pgfpathlineto{\pgfqpoint{0.456575in}{2.523031in}}%
\pgfpathlineto{\pgfqpoint{0.456575in}{2.523031in}}%
\pgfusepath{stroke}%
\end{pgfscope}%
\begin{pgfscope}%
\pgfpathrectangle{\pgfqpoint{0.448634in}{0.402556in}}{\pgfqpoint{4.350661in}{2.489204in}} %
\pgfusepath{clip}%
\pgfsetrectcap%
\pgfsetroundjoin%
\pgfsetlinewidth{1.003750pt}%
\definecolor{currentstroke}{rgb}{0.121569,0.466667,0.705882}%
\pgfsetstrokecolor{currentstroke}%
\pgfsetdash{}{0pt}%
\pgfpathmoveto{\pgfqpoint{4.798840in}{2.852369in}}%
\pgfpathlineto{\pgfqpoint{4.797564in}{2.889610in}}%
\pgfpathlineto{\pgfqpoint{4.796215in}{2.891483in}}%
\pgfpathlineto{\pgfqpoint{4.787551in}{2.891760in}}%
\pgfpathlineto{\pgfqpoint{0.452128in}{2.891659in}}%
\pgfpathlineto{\pgfqpoint{0.450530in}{2.890082in}}%
\pgfpathlineto{\pgfqpoint{0.449454in}{2.882763in}}%
\pgfpathlineto{\pgfqpoint{0.448970in}{2.845432in}}%
\pgfpathlineto{\pgfqpoint{0.448743in}{2.494454in}}%
\pgfpathlineto{\pgfqpoint{0.449624in}{0.615107in}}%
\pgfpathlineto{\pgfqpoint{0.451433in}{0.510586in}}%
\pgfpathlineto{\pgfqpoint{0.453993in}{0.473374in}}%
\pgfpathlineto{\pgfqpoint{0.457406in}{0.453868in}}%
\pgfpathlineto{\pgfqpoint{0.461540in}{0.442384in}}%
\pgfpathlineto{\pgfqpoint{0.466739in}{0.434437in}}%
\pgfpathlineto{\pgfqpoint{0.473595in}{0.428350in}}%
\pgfpathlineto{\pgfqpoint{0.483492in}{0.423244in}}%
\pgfpathlineto{\pgfqpoint{0.491854in}{0.420501in}}%
\pgfpathlineto{\pgfqpoint{0.491854in}{0.420501in}}%
\pgfusepath{stroke}%
\end{pgfscope}%
\begin{pgfscope}%
\pgfpathrectangle{\pgfqpoint{0.448634in}{0.402556in}}{\pgfqpoint{4.350661in}{2.489204in}} %
\pgfusepath{clip}%
\pgfsetrectcap%
\pgfsetroundjoin%
\pgfsetlinewidth{1.003750pt}%
\definecolor{currentstroke}{rgb}{0.121569,0.466667,0.705882}%
\pgfsetstrokecolor{currentstroke}%
\pgfsetdash{}{0pt}%
\pgfpathmoveto{\pgfqpoint{0.456424in}{1.370137in}}%
\pgfpathlineto{\pgfqpoint{0.459610in}{1.118755in}}%
\pgfpathlineto{\pgfqpoint{0.463695in}{0.962007in}}%
\pgfpathlineto{\pgfqpoint{0.468519in}{0.857610in}}%
\pgfpathlineto{\pgfqpoint{0.474082in}{0.783210in}}%
\pgfpathlineto{\pgfqpoint{0.480226in}{0.728906in}}%
\pgfpathlineto{\pgfqpoint{0.486970in}{0.687306in}}%
\pgfpathlineto{\pgfqpoint{0.494537in}{0.653558in}}%
\pgfpathlineto{\pgfqpoint{0.503107in}{0.625355in}}%
\pgfpathlineto{\pgfqpoint{0.512193in}{0.602749in}}%
\pgfpathlineto{\pgfqpoint{0.522200in}{0.583508in}}%
\pgfpathlineto{\pgfqpoint{0.534108in}{0.565743in}}%
\pgfpathlineto{\pgfqpoint{0.546263in}{0.551507in}}%
\pgfpathlineto{\pgfqpoint{0.559728in}{0.538907in}}%
\pgfpathlineto{\pgfqpoint{0.576129in}{0.526693in}}%
\pgfpathlineto{\pgfqpoint{0.595483in}{0.515351in}}%
\pgfpathlineto{\pgfqpoint{0.617681in}{0.505147in}}%
\pgfpathlineto{\pgfqpoint{0.642568in}{0.496153in}}%
\pgfpathlineto{\pgfqpoint{0.672126in}{0.487778in}}%
\pgfpathlineto{\pgfqpoint{0.708443in}{0.479824in}}%
\pgfpathlineto{\pgfqpoint{0.753649in}{0.472325in}}%
\pgfpathlineto{\pgfqpoint{0.807717in}{0.465660in}}%
\pgfpathlineto{\pgfqpoint{0.877116in}{0.459475in}}%
\pgfpathlineto{\pgfqpoint{0.961828in}{0.454230in}}%
\pgfpathlineto{\pgfqpoint{1.068351in}{0.449916in}}%
\pgfpathlineto{\pgfqpoint{1.201018in}{0.446839in}}%
\pgfpathlineto{\pgfqpoint{1.357637in}{0.445481in}}%
\pgfpathlineto{\pgfqpoint{1.525135in}{0.446232in}}%
\pgfpathlineto{\pgfqpoint{1.686088in}{0.449142in}}%
\pgfpathlineto{\pgfqpoint{1.823074in}{0.453747in}}%
\pgfpathlineto{\pgfqpoint{1.938245in}{0.459764in}}%
\pgfpathlineto{\pgfqpoint{2.031582in}{0.466759in}}%
\pgfpathlineto{\pgfqpoint{2.109580in}{0.474745in}}%
\pgfpathlineto{\pgfqpoint{2.174384in}{0.483535in}}%
\pgfpathlineto{\pgfqpoint{2.228139in}{0.492940in}}%
\pgfpathlineto{\pgfqpoint{2.275119in}{0.503356in}}%
\pgfpathlineto{\pgfqpoint{2.315282in}{0.514501in}}%
\pgfpathlineto{\pgfqpoint{2.350698in}{0.526659in}}%
\pgfpathlineto{\pgfqpoint{2.381320in}{0.539536in}}%
\pgfpathlineto{\pgfqpoint{2.407164in}{0.552659in}}%
\pgfpathlineto{\pgfqpoint{2.430226in}{0.566639in}}%
\pgfpathlineto{\pgfqpoint{2.452282in}{0.582602in}}%
\pgfpathlineto{\pgfqpoint{2.471391in}{0.599069in}}%
\pgfpathlineto{\pgfqpoint{2.489240in}{0.617293in}}%
\pgfpathlineto{\pgfqpoint{2.505678in}{0.637180in}}%
\pgfpathlineto{\pgfqpoint{2.520620in}{0.658557in}}%
\pgfpathlineto{\pgfqpoint{2.535213in}{0.683314in}}%
\pgfpathlineto{\pgfqpoint{2.549115in}{0.711484in}}%
\pgfpathlineto{\pgfqpoint{2.562091in}{0.743004in}}%
\pgfpathlineto{\pgfqpoint{2.574020in}{0.777751in}}%
\pgfpathlineto{\pgfqpoint{2.585502in}{0.817970in}}%
\pgfpathlineto{\pgfqpoint{2.596809in}{0.866038in}}%
\pgfpathlineto{\pgfqpoint{2.607562in}{0.921948in}}%
\pgfpathlineto{\pgfqpoint{2.617925in}{0.988098in}}%
\pgfpathlineto{\pgfqpoint{2.627958in}{1.066918in}}%
\pgfpathlineto{\pgfqpoint{2.637941in}{1.163320in}}%
\pgfpathlineto{\pgfqpoint{2.648424in}{1.287199in}}%
\pgfpathlineto{\pgfqpoint{2.660103in}{1.453438in}}%
\pgfpathlineto{\pgfqpoint{2.674773in}{1.696801in}}%
\pgfpathlineto{\pgfqpoint{2.687716in}{1.945279in}}%
\pgfpathlineto{\pgfqpoint{2.692670in}{2.079573in}}%
\pgfpathlineto{\pgfqpoint{2.693829in}{2.166682in}}%
\pgfpathlineto{\pgfqpoint{2.692565in}{2.233870in}}%
\pgfpathlineto{\pgfqpoint{2.689436in}{2.286015in}}%
\pgfpathlineto{\pgfqpoint{2.684859in}{2.327999in}}%
\pgfpathlineto{\pgfqpoint{2.678725in}{2.364664in}}%
\pgfpathlineto{\pgfqpoint{2.671356in}{2.395897in}}%
\pgfpathlineto{\pgfqpoint{2.662489in}{2.423981in}}%
\pgfpathlineto{\pgfqpoint{2.652361in}{2.448778in}}%
\pgfpathlineto{\pgfqpoint{2.641365in}{2.470245in}}%
\pgfpathlineto{\pgfqpoint{2.628643in}{2.490425in}}%
\pgfpathlineto{\pgfqpoint{2.614279in}{2.509106in}}%
\pgfpathlineto{\pgfqpoint{2.598443in}{2.526159in}}%
\pgfpathlineto{\pgfqpoint{2.579590in}{2.543005in}}%
\pgfpathlineto{\pgfqpoint{2.559532in}{2.557923in}}%
\pgfpathlineto{\pgfqpoint{2.536602in}{2.572183in}}%
\pgfpathlineto{\pgfqpoint{2.510850in}{2.585538in}}%
\pgfpathlineto{\pgfqpoint{2.482360in}{2.597837in}}%
\pgfpathlineto{\pgfqpoint{2.449134in}{2.609683in}}%
\pgfpathlineto{\pgfqpoint{2.411184in}{2.620696in}}%
\pgfpathlineto{\pgfqpoint{2.368552in}{2.630606in}}%
\pgfpathlineto{\pgfqpoint{2.321294in}{2.639221in}}%
\pgfpathlineto{\pgfqpoint{2.269467in}{2.646399in}}%
\pgfpathlineto{\pgfqpoint{2.210954in}{2.652193in}}%
\pgfpathlineto{\pgfqpoint{2.147967in}{2.656153in}}%
\pgfpathlineto{\pgfqpoint{2.080556in}{2.658135in}}%
\pgfpathlineto{\pgfqpoint{2.010948in}{2.657971in}}%
\pgfpathlineto{\pgfqpoint{1.939195in}{2.655572in}}%
\pgfpathlineto{\pgfqpoint{1.867527in}{2.650913in}}%
\pgfpathlineto{\pgfqpoint{1.798171in}{2.644140in}}%
\pgfpathlineto{\pgfqpoint{1.733341in}{2.635606in}}%
\pgfpathlineto{\pgfqpoint{1.673075in}{2.625521in}}%
\pgfpathlineto{\pgfqpoint{1.615274in}{2.613610in}}%
\pgfpathlineto{\pgfqpoint{1.562133in}{2.600402in}}%
\pgfpathlineto{\pgfqpoint{1.513681in}{2.586139in}}%
\pgfpathlineto{\pgfqpoint{1.467862in}{2.570344in}}%
\pgfpathlineto{\pgfqpoint{1.426794in}{2.553923in}}%
\pgfpathlineto{\pgfqpoint{1.388447in}{2.536289in}}%
\pgfpathlineto{\pgfqpoint{1.352878in}{2.517566in}}%
\pgfpathlineto{\pgfqpoint{1.320128in}{2.497922in}}%
\pgfpathlineto{\pgfqpoint{1.288379in}{2.476236in}}%
\pgfpathlineto{\pgfqpoint{1.259592in}{2.453861in}}%
\pgfpathlineto{\pgfqpoint{1.232050in}{2.429520in}}%
\pgfpathlineto{\pgfqpoint{1.207527in}{2.404898in}}%
\pgfpathlineto{\pgfqpoint{1.184409in}{2.378557in}}%
\pgfpathlineto{\pgfqpoint{1.162828in}{2.350561in}}%
\pgfpathlineto{\pgfqpoint{1.142891in}{2.321011in}}%
\pgfpathlineto{\pgfqpoint{1.124675in}{2.290041in}}%
\pgfpathlineto{\pgfqpoint{1.108225in}{2.257802in}}%
\pgfpathlineto{\pgfqpoint{1.092639in}{2.222199in}}%
\pgfpathlineto{\pgfqpoint{1.079059in}{2.185535in}}%
\pgfpathlineto{\pgfqpoint{1.067443in}{2.147998in}}%
\pgfpathlineto{\pgfqpoint{1.057187in}{2.107348in}}%
\pgfpathlineto{\pgfqpoint{1.049004in}{2.066086in}}%
\pgfpathlineto{\pgfqpoint{1.042513in}{2.021906in}}%
\pgfpathlineto{\pgfqpoint{1.038177in}{1.977382in}}%
\pgfpathlineto{\pgfqpoint{1.035866in}{1.930167in}}%
\pgfpathlineto{\pgfqpoint{1.035826in}{1.882878in}}%
\pgfpathlineto{\pgfqpoint{1.038031in}{1.835656in}}%
\pgfpathlineto{\pgfqpoint{1.042474in}{1.788641in}}%
\pgfpathlineto{\pgfqpoint{1.049176in}{1.741979in}}%
\pgfpathlineto{\pgfqpoint{1.057644in}{1.698239in}}%
\pgfpathlineto{\pgfqpoint{1.068221in}{1.655105in}}%
\pgfpathlineto{\pgfqpoint{1.080962in}{1.612745in}}%
\pgfpathlineto{\pgfqpoint{1.095031in}{1.573617in}}%
\pgfpathlineto{\pgfqpoint{1.111115in}{1.535520in}}%
\pgfpathlineto{\pgfqpoint{1.128118in}{1.500775in}}%
\pgfpathlineto{\pgfqpoint{1.146930in}{1.467274in}}%
\pgfpathlineto{\pgfqpoint{1.167531in}{1.435181in}}%
\pgfpathlineto{\pgfqpoint{1.189874in}{1.404652in}}%
\pgfpathlineto{\pgfqpoint{1.213884in}{1.375828in}}%
\pgfpathlineto{\pgfqpoint{1.237817in}{1.350457in}}%
\pgfpathlineto{\pgfqpoint{1.264748in}{1.325237in}}%
\pgfpathlineto{\pgfqpoint{1.292991in}{1.301972in}}%
\pgfpathlineto{\pgfqpoint{1.322398in}{1.280678in}}%
\pgfpathlineto{\pgfqpoint{1.352820in}{1.261340in}}%
\pgfpathlineto{\pgfqpoint{1.386095in}{1.242889in}}%
\pgfpathlineto{\pgfqpoint{1.420190in}{1.226516in}}%
\pgfpathlineto{\pgfqpoint{1.457024in}{1.211329in}}%
\pgfpathlineto{\pgfqpoint{1.496554in}{1.197536in}}%
\pgfpathlineto{\pgfqpoint{1.538719in}{1.185287in}}%
\pgfpathlineto{\pgfqpoint{1.583441in}{1.174641in}}%
\pgfpathlineto{\pgfqpoint{1.634929in}{1.164775in}}%
\pgfpathlineto{\pgfqpoint{1.706063in}{1.153745in}}%
\pgfpathlineto{\pgfqpoint{1.768492in}{1.143417in}}%
\pgfpathlineto{\pgfqpoint{1.796122in}{1.136567in}}%
\pgfpathlineto{\pgfqpoint{1.812683in}{1.130481in}}%
\pgfpathlineto{\pgfqpoint{1.824471in}{1.124102in}}%
\pgfpathlineto{\pgfqpoint{1.833209in}{1.116741in}}%
\pgfpathlineto{\pgfqpoint{1.838498in}{1.108890in}}%
\pgfpathlineto{\pgfqpoint{1.840588in}{1.101849in}}%
\pgfpathlineto{\pgfqpoint{1.840619in}{1.094412in}}%
\pgfpathlineto{\pgfqpoint{1.837931in}{1.084986in}}%
\pgfpathlineto{\pgfqpoint{1.833246in}{1.076615in}}%
\pgfpathlineto{\pgfqpoint{1.825819in}{1.067542in}}%
\pgfpathlineto{\pgfqpoint{1.813813in}{1.056850in}}%
\pgfpathlineto{\pgfqpoint{1.798819in}{1.046763in}}%
\pgfpathlineto{\pgfqpoint{1.781016in}{1.037462in}}%
\pgfpathlineto{\pgfqpoint{1.758447in}{1.028391in}}%
\pgfpathlineto{\pgfqpoint{1.733203in}{1.020815in}}%
\pgfpathlineto{\pgfqpoint{1.705410in}{1.014872in}}%
\pgfpathlineto{\pgfqpoint{1.675178in}{1.010714in}}%
\pgfpathlineto{\pgfqpoint{1.642610in}{1.008507in}}%
\pgfpathlineto{\pgfqpoint{1.607809in}{1.008432in}}%
\pgfpathlineto{\pgfqpoint{1.570886in}{1.010691in}}%
\pgfpathlineto{\pgfqpoint{1.534118in}{1.015181in}}%
\pgfpathlineto{\pgfqpoint{1.495454in}{1.022233in}}%
\pgfpathlineto{\pgfqpoint{1.457161in}{1.031563in}}%
\pgfpathlineto{\pgfqpoint{1.419337in}{1.043132in}}%
\pgfpathlineto{\pgfqpoint{1.382089in}{1.056929in}}%
\pgfpathlineto{\pgfqpoint{1.347544in}{1.072019in}}%
\pgfpathlineto{\pgfqpoint{1.313727in}{1.089133in}}%
\pgfpathlineto{\pgfqpoint{1.280762in}{1.108299in}}%
\pgfpathlineto{\pgfqpoint{1.248782in}{1.129536in}}%
\pgfpathlineto{\pgfqpoint{1.219708in}{1.151422in}}%
\pgfpathlineto{\pgfqpoint{1.191752in}{1.175138in}}%
\pgfpathlineto{\pgfqpoint{1.165031in}{1.200649in}}%
\pgfpathlineto{\pgfqpoint{1.139653in}{1.227898in}}%
\pgfpathlineto{\pgfqpoint{1.115714in}{1.256800in}}%
\pgfpathlineto{\pgfqpoint{1.093288in}{1.287251in}}%
\pgfpathlineto{\pgfqpoint{1.071178in}{1.321163in}}%
\pgfpathlineto{\pgfqpoint{1.050868in}{1.356520in}}%
\pgfpathlineto{\pgfqpoint{1.032365in}{1.393152in}}%
\pgfpathlineto{\pgfqpoint{1.014718in}{1.433142in}}%
\pgfpathlineto{\pgfqpoint{0.999024in}{1.474185in}}%
\pgfpathlineto{\pgfqpoint{0.984506in}{1.518461in}}%
\pgfpathlineto{\pgfqpoint{0.972010in}{1.563537in}}%
\pgfpathlineto{\pgfqpoint{0.960944in}{1.611678in}}%
\pgfpathlineto{\pgfqpoint{0.951530in}{1.662824in}}%
\pgfpathlineto{\pgfqpoint{0.944286in}{1.714431in}}%
\pgfpathlineto{\pgfqpoint{0.938950in}{1.768847in}}%
\pgfpathlineto{\pgfqpoint{0.935870in}{1.823491in}}%
\pgfpathlineto{\pgfqpoint{0.935034in}{1.878240in}}%
\pgfpathlineto{\pgfqpoint{0.936466in}{1.932973in}}%
\pgfpathlineto{\pgfqpoint{0.940005in}{1.985084in}}%
\pgfpathlineto{\pgfqpoint{0.945759in}{2.036935in}}%
\pgfpathlineto{\pgfqpoint{0.953410in}{2.085938in}}%
\pgfpathlineto{\pgfqpoint{0.962764in}{2.132000in}}%
\pgfpathlineto{\pgfqpoint{0.974287in}{2.177414in}}%
\pgfpathlineto{\pgfqpoint{0.987332in}{2.219653in}}%
\pgfpathlineto{\pgfqpoint{1.001667in}{2.258654in}}%
\pgfpathlineto{\pgfqpoint{1.018051in}{2.296583in}}%
\pgfpathlineto{\pgfqpoint{1.035401in}{2.331101in}}%
\pgfpathlineto{\pgfqpoint{1.054650in}{2.364275in}}%
\pgfpathlineto{\pgfqpoint{1.074406in}{2.393984in}}%
\pgfpathlineto{\pgfqpoint{1.095771in}{2.422197in}}%
\pgfpathlineto{\pgfqpoint{1.118662in}{2.448797in}}%
\pgfpathlineto{\pgfqpoint{1.142967in}{2.473701in}}%
\pgfpathlineto{\pgfqpoint{1.168550in}{2.496867in}}%
\pgfpathlineto{\pgfqpoint{1.197085in}{2.519662in}}%
\pgfpathlineto{\pgfqpoint{1.226727in}{2.540526in}}%
\pgfpathlineto{\pgfqpoint{1.259242in}{2.560673in}}%
\pgfpathlineto{\pgfqpoint{1.294612in}{2.579881in}}%
\pgfpathlineto{\pgfqpoint{1.332792in}{2.597982in}}%
\pgfpathlineto{\pgfqpoint{1.373719in}{2.614859in}}%
\pgfpathlineto{\pgfqpoint{1.417319in}{2.630445in}}%
\pgfpathlineto{\pgfqpoint{1.465632in}{2.645312in}}%
\pgfpathlineto{\pgfqpoint{1.518640in}{2.659204in}}%
\pgfpathlineto{\pgfqpoint{1.576309in}{2.671929in}}%
\pgfpathlineto{\pgfqpoint{1.638597in}{2.683344in}}%
\pgfpathlineto{\pgfqpoint{1.705462in}{2.693343in}}%
\pgfpathlineto{\pgfqpoint{1.779027in}{2.702064in}}%
\pgfpathlineto{\pgfqpoint{1.857097in}{2.709077in}}%
\pgfpathlineto{\pgfqpoint{1.939633in}{2.714280in}}%
\pgfpathlineto{\pgfqpoint{2.026598in}{2.717513in}}%
\pgfpathlineto{\pgfqpoint{2.113605in}{2.718523in}}%
\pgfpathlineto{\pgfqpoint{2.198435in}{2.717303in}}%
\pgfpathlineto{\pgfqpoint{2.278866in}{2.713929in}}%
\pgfpathlineto{\pgfqpoint{2.352678in}{2.708598in}}%
\pgfpathlineto{\pgfqpoint{2.417657in}{2.701709in}}%
\pgfpathlineto{\pgfqpoint{2.473770in}{2.693630in}}%
\pgfpathlineto{\pgfqpoint{2.523140in}{2.684368in}}%
\pgfpathlineto{\pgfqpoint{2.565726in}{2.674202in}}%
\pgfpathlineto{\pgfqpoint{2.601510in}{2.663544in}}%
\pgfpathlineto{\pgfqpoint{2.632577in}{2.652142in}}%
\pgfpathlineto{\pgfqpoint{2.658899in}{2.640331in}}%
\pgfpathlineto{\pgfqpoint{2.682438in}{2.627436in}}%
\pgfpathlineto{\pgfqpoint{2.703062in}{2.613571in}}%
\pgfpathlineto{\pgfqpoint{2.720674in}{2.598978in}}%
\pgfpathlineto{\pgfqpoint{2.735263in}{2.584053in}}%
\pgfpathlineto{\pgfqpoint{2.748320in}{2.567377in}}%
\pgfpathlineto{\pgfqpoint{2.759553in}{2.549046in}}%
\pgfpathlineto{\pgfqpoint{2.768788in}{2.529306in}}%
\pgfpathlineto{\pgfqpoint{2.776017in}{2.508498in}}%
\pgfpathlineto{\pgfqpoint{2.781884in}{2.484540in}}%
\pgfpathlineto{\pgfqpoint{2.786102in}{2.457597in}}%
\pgfpathlineto{\pgfqpoint{2.788720in}{2.425384in}}%
\pgfpathlineto{\pgfqpoint{2.789427in}{2.388061in}}%
\pgfpathlineto{\pgfqpoint{2.787962in}{2.340801in}}%
\pgfpathlineto{\pgfqpoint{2.783672in}{2.278768in}}%
\pgfpathlineto{\pgfqpoint{2.774289in}{2.179783in}}%
\pgfpathlineto{\pgfqpoint{2.743611in}{1.868119in}}%
\pgfpathlineto{\pgfqpoint{2.730112in}{1.702060in}}%
\pgfpathlineto{\pgfqpoint{2.717287in}{1.515949in}}%
\pgfpathlineto{\pgfqpoint{2.702602in}{1.267597in}}%
\pgfpathlineto{\pgfqpoint{2.684434in}{0.964630in}}%
\pgfpathlineto{\pgfqpoint{2.675374in}{0.850600in}}%
\pgfpathlineto{\pgfqpoint{2.667030in}{0.771523in}}%
\pgfpathlineto{\pgfqpoint{2.658752in}{0.712543in}}%
\pgfpathlineto{\pgfqpoint{2.650176in}{0.666284in}}%
\pgfpathlineto{\pgfqpoint{2.640820in}{0.627931in}}%
\pgfpathlineto{\pgfqpoint{2.631145in}{0.597534in}}%
\pgfpathlineto{\pgfqpoint{2.621004in}{0.572745in}}%
\pgfpathlineto{\pgfqpoint{2.609856in}{0.551383in}}%
\pgfpathlineto{\pgfqpoint{2.598042in}{0.533534in}}%
\pgfpathlineto{\pgfqpoint{2.584496in}{0.517378in}}%
\pgfpathlineto{\pgfqpoint{2.571109in}{0.504669in}}%
\pgfpathlineto{\pgfqpoint{2.554789in}{0.492313in}}%
\pgfpathlineto{\pgfqpoint{2.537456in}{0.481914in}}%
\pgfpathlineto{\pgfqpoint{2.517374in}{0.472367in}}%
\pgfpathlineto{\pgfqpoint{2.492542in}{0.463178in}}%
\pgfpathlineto{\pgfqpoint{2.462979in}{0.454833in}}%
\pgfpathlineto{\pgfqpoint{2.428766in}{0.447542in}}%
\pgfpathlineto{\pgfqpoint{2.385671in}{0.440735in}}%
\pgfpathlineto{\pgfqpoint{2.331557in}{0.434581in}}%
\pgfpathlineto{\pgfqpoint{2.262115in}{0.429077in}}%
\pgfpathlineto{\pgfqpoint{2.170851in}{0.424236in}}%
\pgfpathlineto{\pgfqpoint{2.049086in}{0.420134in}}%
\pgfpathlineto{\pgfqpoint{1.879436in}{0.416783in}}%
\pgfpathlineto{\pgfqpoint{1.640159in}{0.414418in}}%
\pgfpathlineto{\pgfqpoint{1.322562in}{0.413569in}}%
\pgfpathlineto{\pgfqpoint{1.020194in}{0.414850in}}%
\pgfpathlineto{\pgfqpoint{0.822256in}{0.417715in}}%
\pgfpathlineto{\pgfqpoint{0.704835in}{0.421430in}}%
\pgfpathlineto{\pgfqpoint{0.630976in}{0.425829in}}%
\pgfpathlineto{\pgfqpoint{0.583316in}{0.430734in}}%
\pgfpathlineto{\pgfqpoint{0.551033in}{0.436123in}}%
\pgfpathlineto{\pgfqpoint{0.527708in}{0.442189in}}%
\pgfpathlineto{\pgfqpoint{0.511250in}{0.448625in}}%
\pgfpathlineto{\pgfqpoint{0.499549in}{0.455216in}}%
\pgfpathlineto{\pgfqpoint{0.488916in}{0.463841in}}%
\pgfpathlineto{\pgfqpoint{0.481322in}{0.472730in}}%
\pgfpathlineto{\pgfqpoint{0.474078in}{0.485127in}}%
\pgfpathlineto{\pgfqpoint{0.468753in}{0.498748in}}%
\pgfpathlineto{\pgfqpoint{0.463870in}{0.517848in}}%
\pgfpathlineto{\pgfqpoint{0.459679in}{0.544796in}}%
\pgfpathlineto{\pgfqpoint{0.456386in}{0.581938in}}%
\pgfpathlineto{\pgfqpoint{0.453731in}{0.639106in}}%
\pgfpathlineto{\pgfqpoint{0.451681in}{0.736155in}}%
\pgfpathlineto{\pgfqpoint{0.450220in}{0.927815in}}%
\pgfpathlineto{\pgfqpoint{0.449345in}{1.403252in}}%
\pgfpathlineto{\pgfqpoint{0.449543in}{2.682703in}}%
\pgfpathlineto{\pgfqpoint{0.451011in}{2.856932in}}%
\pgfpathlineto{\pgfqpoint{0.452802in}{2.879219in}}%
\pgfpathlineto{\pgfqpoint{0.455188in}{2.886108in}}%
\pgfpathlineto{\pgfqpoint{0.458626in}{2.889028in}}%
\pgfpathlineto{\pgfqpoint{0.464996in}{2.890553in}}%
\pgfpathlineto{\pgfqpoint{0.482377in}{2.891423in}}%
\pgfpathlineto{\pgfqpoint{0.565038in}{2.891729in}}%
\pgfpathlineto{\pgfqpoint{2.733842in}{2.891760in}}%
\pgfpathlineto{\pgfqpoint{4.789510in}{2.890885in}}%
\pgfpathlineto{\pgfqpoint{4.793727in}{2.889730in}}%
\pgfpathlineto{\pgfqpoint{4.795481in}{2.888307in}}%
\pgfpathlineto{\pgfqpoint{4.797106in}{2.881145in}}%
\pgfpathlineto{\pgfqpoint{4.797997in}{2.858771in}}%
\pgfpathlineto{\pgfqpoint{4.798039in}{2.856283in}}%
\pgfpathlineto{\pgfqpoint{4.798039in}{2.856283in}}%
\pgfusepath{stroke}%
\end{pgfscope}%
\begin{pgfscope}%
\pgfpathrectangle{\pgfqpoint{0.448634in}{0.402556in}}{\pgfqpoint{4.350661in}{2.489204in}} %
\pgfusepath{clip}%
\pgfsetrectcap%
\pgfsetroundjoin%
\pgfsetlinewidth{1.003750pt}%
\definecolor{currentstroke}{rgb}{0.121569,0.466667,0.705882}%
\pgfsetstrokecolor{currentstroke}%
\pgfsetdash{}{0pt}%
\pgfpathmoveto{\pgfqpoint{3.428772in}{0.402610in}}%
\pgfpathlineto{\pgfqpoint{2.806632in}{0.403760in}}%
\pgfpathlineto{\pgfqpoint{2.769691in}{0.405578in}}%
\pgfpathlineto{\pgfqpoint{2.754632in}{0.408064in}}%
\pgfpathlineto{\pgfqpoint{2.746391in}{0.411198in}}%
\pgfpathlineto{\pgfqpoint{2.740942in}{0.415265in}}%
\pgfpathlineto{\pgfqpoint{2.736784in}{0.420984in}}%
\pgfpathlineto{\pgfqpoint{2.733281in}{0.430071in}}%
\pgfpathlineto{\pgfqpoint{2.730449in}{0.444636in}}%
\pgfpathlineto{\pgfqpoint{2.728238in}{0.469392in}}%
\pgfpathlineto{\pgfqpoint{2.726470in}{0.519131in}}%
\pgfpathlineto{\pgfqpoint{2.725711in}{0.613715in}}%
\pgfpathlineto{\pgfqpoint{2.726842in}{0.768038in}}%
\pgfpathlineto{\pgfqpoint{2.730556in}{0.962148in}}%
\pgfpathlineto{\pgfqpoint{2.736611in}{1.158670in}}%
\pgfpathlineto{\pgfqpoint{2.744092in}{1.327718in}}%
\pgfpathlineto{\pgfqpoint{2.753201in}{1.484189in}}%
\pgfpathlineto{\pgfqpoint{2.763257in}{1.620609in}}%
\pgfpathlineto{\pgfqpoint{2.776118in}{1.764216in}}%
\pgfpathlineto{\pgfqpoint{2.788914in}{1.877776in}}%
\pgfpathlineto{\pgfqpoint{2.805748in}{2.005740in}}%
\pgfpathlineto{\pgfqpoint{2.821176in}{2.101198in}}%
\pgfpathlineto{\pgfqpoint{2.838359in}{2.193718in}}%
\pgfpathlineto{\pgfqpoint{2.859135in}{2.292966in}}%
\pgfpathlineto{\pgfqpoint{2.887209in}{2.425960in}}%
\pgfpathlineto{\pgfqpoint{2.896991in}{2.479559in}}%
\pgfpathlineto{\pgfqpoint{2.901543in}{2.516523in}}%
\pgfpathlineto{\pgfqpoint{2.902849in}{2.543854in}}%
\pgfpathlineto{\pgfqpoint{2.901957in}{2.566223in}}%
\pgfpathlineto{\pgfqpoint{2.899151in}{2.585863in}}%
\pgfpathlineto{\pgfqpoint{2.894794in}{2.602546in}}%
\pgfpathlineto{\pgfqpoint{2.888484in}{2.618388in}}%
\pgfpathlineto{\pgfqpoint{2.880257in}{2.633033in}}%
\pgfpathlineto{\pgfqpoint{2.870348in}{2.646246in}}%
\pgfpathlineto{\pgfqpoint{2.857399in}{2.659530in}}%
\pgfpathlineto{\pgfqpoint{2.843189in}{2.671010in}}%
\pgfpathlineto{\pgfqpoint{2.824237in}{2.683209in}}%
\pgfpathlineto{\pgfqpoint{2.802413in}{2.694418in}}%
\pgfpathlineto{\pgfqpoint{2.775809in}{2.705369in}}%
\pgfpathlineto{\pgfqpoint{2.744461in}{2.715715in}}%
\pgfpathlineto{\pgfqpoint{2.708436in}{2.725252in}}%
\pgfpathlineto{\pgfqpoint{2.665655in}{2.734289in}}%
\pgfpathlineto{\pgfqpoint{2.613991in}{2.742869in}}%
\pgfpathlineto{\pgfqpoint{2.553459in}{2.750589in}}%
\pgfpathlineto{\pgfqpoint{2.481920in}{2.757365in}}%
\pgfpathlineto{\pgfqpoint{2.399398in}{2.762839in}}%
\pgfpathlineto{\pgfqpoint{2.310269in}{2.766482in}}%
\pgfpathlineto{\pgfqpoint{2.175416in}{2.768725in}}%
\pgfpathlineto{\pgfqpoint{2.066653in}{2.767942in}}%
\pgfpathlineto{\pgfqpoint{1.953570in}{2.764859in}}%
\pgfpathlineto{\pgfqpoint{1.851429in}{2.759759in}}%
\pgfpathlineto{\pgfqpoint{1.745051in}{2.752168in}}%
\pgfpathlineto{\pgfqpoint{1.658373in}{2.743453in}}%
\pgfpathlineto{\pgfqpoint{1.580552in}{2.733461in}}%
\pgfpathlineto{\pgfqpoint{1.490057in}{2.719338in}}%
\pgfpathlineto{\pgfqpoint{1.417231in}{2.704698in}}%
\pgfpathlineto{\pgfqpoint{1.361992in}{2.690818in}}%
\pgfpathlineto{\pgfqpoint{1.311460in}{2.675819in}}%
\pgfpathlineto{\pgfqpoint{1.265667in}{2.659924in}}%
\pgfpathlineto{\pgfqpoint{1.222575in}{2.642586in}}%
\pgfpathlineto{\pgfqpoint{1.184324in}{2.624682in}}%
\pgfpathlineto{\pgfqpoint{1.148892in}{2.605623in}}%
\pgfpathlineto{\pgfqpoint{1.116331in}{2.585573in}}%
\pgfpathlineto{\pgfqpoint{1.092327in}{2.568512in}}%
\pgfpathlineto{\pgfqpoint{1.079760in}{2.558686in}}%
\pgfpathlineto{\pgfqpoint{1.051544in}{2.535379in}}%
\pgfpathlineto{\pgfqpoint{1.026312in}{2.511712in}}%
\pgfpathlineto{\pgfqpoint{1.002399in}{2.486318in}}%
\pgfpathlineto{\pgfqpoint{0.979913in}{2.459269in}}%
\pgfpathlineto{\pgfqpoint{0.958934in}{2.430678in}}%
\pgfpathlineto{\pgfqpoint{0.938264in}{2.398643in}}%
\pgfpathlineto{\pgfqpoint{0.923047in}{2.371385in}}%
\pgfpathlineto{\pgfqpoint{0.904513in}{2.334774in}}%
\pgfpathlineto{\pgfqpoint{0.887854in}{2.297001in}}%
\pgfpathlineto{\pgfqpoint{0.872131in}{2.255971in}}%
\pgfpathlineto{\pgfqpoint{0.857508in}{2.211741in}}%
\pgfpathlineto{\pgfqpoint{0.844762in}{2.166757in}}%
\pgfpathlineto{\pgfqpoint{0.838624in}{2.140306in}}%
\pgfpathlineto{\pgfqpoint{0.826982in}{2.087193in}}%
\pgfpathlineto{\pgfqpoint{0.816322in}{2.028715in}}%
\pgfpathlineto{\pgfqpoint{0.810087in}{1.984495in}}%
\pgfpathlineto{\pgfqpoint{0.808026in}{1.967238in}}%
\pgfpathlineto{\pgfqpoint{0.800076in}{1.898140in}}%
\pgfpathlineto{\pgfqpoint{0.793713in}{1.823823in}}%
\pgfpathlineto{\pgfqpoint{0.788799in}{1.741875in}}%
\pgfpathlineto{\pgfqpoint{0.786199in}{1.677225in}}%
\pgfpathlineto{\pgfqpoint{0.776951in}{1.453481in}}%
\pgfpathlineto{\pgfqpoint{0.773280in}{1.418894in}}%
\pgfpathlineto{\pgfqpoint{0.768298in}{1.389582in}}%
\pgfpathlineto{\pgfqpoint{0.762752in}{1.368108in}}%
\pgfpathlineto{\pgfqpoint{0.756722in}{1.352123in}}%
\pgfpathlineto{\pgfqpoint{0.749752in}{1.339519in}}%
\pgfpathlineto{\pgfqpoint{0.742201in}{1.330599in}}%
\pgfpathlineto{\pgfqpoint{0.734854in}{1.325312in}}%
\pgfpathlineto{\pgfqpoint{0.726558in}{1.322419in}}%
\pgfpathlineto{\pgfqpoint{0.717884in}{1.322223in}}%
\pgfpathlineto{\pgfqpoint{0.709412in}{1.324411in}}%
\pgfpathlineto{\pgfqpoint{0.699548in}{1.329604in}}%
\pgfpathlineto{\pgfqpoint{0.688894in}{1.338203in}}%
\pgfpathlineto{\pgfqpoint{0.677907in}{1.350248in}}%
\pgfpathlineto{\pgfqpoint{0.666886in}{1.365647in}}%
\pgfpathlineto{\pgfqpoint{0.654913in}{1.386417in}}%
\pgfpathlineto{\pgfqpoint{0.642574in}{1.412730in}}%
\pgfpathlineto{\pgfqpoint{0.630328in}{1.444629in}}%
\pgfpathlineto{\pgfqpoint{0.618504in}{1.482081in}}%
\pgfpathlineto{\pgfqpoint{0.608613in}{1.520256in}}%
\pgfpathlineto{\pgfqpoint{0.590203in}{1.612445in}}%
\pgfpathlineto{\pgfqpoint{0.581848in}{1.668884in}}%
\pgfpathlineto{\pgfqpoint{0.573137in}{1.740376in}}%
\pgfpathlineto{\pgfqpoint{0.567062in}{1.807213in}}%
\pgfpathlineto{\pgfqpoint{0.560532in}{1.896510in}}%
\pgfpathlineto{\pgfqpoint{0.555526in}{1.995910in}}%
\pgfpathlineto{\pgfqpoint{0.552564in}{2.097908in}}%
\pgfpathlineto{\pgfqpoint{0.551526in}{2.204935in}}%
\pgfpathlineto{\pgfqpoint{0.552728in}{2.309470in}}%
\pgfpathlineto{\pgfqpoint{0.556011in}{2.403981in}}%
\pgfpathlineto{\pgfqpoint{0.560953in}{2.483430in}}%
\pgfpathlineto{\pgfqpoint{0.567303in}{2.550240in}}%
\pgfpathlineto{\pgfqpoint{0.574928in}{2.606817in}}%
\pgfpathlineto{\pgfqpoint{0.582988in}{2.650657in}}%
\pgfpathlineto{\pgfqpoint{0.592756in}{2.691452in}}%
\pgfpathlineto{\pgfqpoint{0.602650in}{2.721756in}}%
\pgfpathlineto{\pgfqpoint{0.612983in}{2.746441in}}%
\pgfpathlineto{\pgfqpoint{0.624292in}{2.767692in}}%
\pgfpathlineto{\pgfqpoint{0.636231in}{2.785433in}}%
\pgfpathlineto{\pgfqpoint{0.649892in}{2.801461in}}%
\pgfpathlineto{\pgfqpoint{0.663386in}{2.814020in}}%
\pgfpathlineto{\pgfqpoint{0.679842in}{2.826135in}}%
\pgfpathlineto{\pgfqpoint{0.697326in}{2.836197in}}%
\pgfpathlineto{\pgfqpoint{0.715574in}{2.844285in}}%
\pgfpathlineto{\pgfqpoint{0.738439in}{2.852335in}}%
\pgfpathlineto{\pgfqpoint{0.765983in}{2.859639in}}%
\pgfpathlineto{\pgfqpoint{0.800300in}{2.866256in}}%
\pgfpathlineto{\pgfqpoint{0.841340in}{2.871832in}}%
\pgfpathlineto{\pgfqpoint{0.895547in}{2.876803in}}%
\pgfpathlineto{\pgfqpoint{0.969413in}{2.881069in}}%
\pgfpathlineto{\pgfqpoint{1.071608in}{2.884501in}}%
\pgfpathlineto{\pgfqpoint{1.219512in}{2.887074in}}%
\pgfpathlineto{\pgfqpoint{1.471844in}{2.889091in}}%
\pgfpathlineto{\pgfqpoint{1.956941in}{2.890384in}}%
\pgfpathlineto{\pgfqpoint{3.096814in}{2.890781in}}%
\pgfpathlineto{\pgfqpoint{3.995224in}{2.889388in}}%
\pgfpathlineto{\pgfqpoint{4.275833in}{2.887011in}}%
\pgfpathlineto{\pgfqpoint{4.412847in}{2.883743in}}%
\pgfpathlineto{\pgfqpoint{4.491081in}{2.879810in}}%
\pgfpathlineto{\pgfqpoint{4.543127in}{2.875163in}}%
\pgfpathlineto{\pgfqpoint{4.579810in}{2.869841in}}%
\pgfpathlineto{\pgfqpoint{4.607580in}{2.863763in}}%
\pgfpathlineto{\pgfqpoint{4.630623in}{2.856424in}}%
\pgfpathlineto{\pgfqpoint{4.648833in}{2.848228in}}%
\pgfpathlineto{\pgfqpoint{4.664136in}{2.838773in}}%
\pgfpathlineto{\pgfqpoint{4.676470in}{2.828576in}}%
\pgfpathlineto{\pgfqpoint{4.687502in}{2.816585in}}%
\pgfpathlineto{\pgfqpoint{4.697051in}{2.803027in}}%
\pgfpathlineto{\pgfqpoint{4.706194in}{2.786098in}}%
\pgfpathlineto{\pgfqpoint{4.714508in}{2.765827in}}%
\pgfpathlineto{\pgfqpoint{4.722462in}{2.740013in}}%
\pgfpathlineto{\pgfqpoint{4.729577in}{2.708703in}}%
\pgfpathlineto{\pgfqpoint{4.736162in}{2.669601in}}%
\pgfpathlineto{\pgfqpoint{4.742419in}{2.617826in}}%
\pgfpathlineto{\pgfqpoint{4.747859in}{2.553410in}}%
\pgfpathlineto{\pgfqpoint{4.752661in}{2.468958in}}%
\pgfpathlineto{\pgfqpoint{4.756610in}{2.359528in}}%
\pgfpathlineto{\pgfqpoint{4.759416in}{2.217681in}}%
\pgfpathlineto{\pgfqpoint{4.760596in}{2.043444in}}%
\pgfpathlineto{\pgfqpoint{4.759662in}{1.851779in}}%
\pgfpathlineto{\pgfqpoint{4.756587in}{1.667613in}}%
\pgfpathlineto{\pgfqpoint{4.751596in}{1.503428in}}%
\pgfpathlineto{\pgfqpoint{4.745410in}{1.374185in}}%
\pgfpathlineto{\pgfqpoint{4.738113in}{1.267479in}}%
\pgfpathlineto{\pgfqpoint{4.729621in}{1.175896in}}%
\pgfpathlineto{\pgfqpoint{4.720762in}{1.104428in}}%
\pgfpathlineto{\pgfqpoint{4.711045in}{1.043204in}}%
\pgfpathlineto{\pgfqpoint{4.700364in}{0.989829in}}%
\pgfpathlineto{\pgfqpoint{4.689055in}{0.944345in}}%
\pgfpathlineto{\pgfqpoint{4.676881in}{0.904394in}}%
\pgfpathlineto{\pgfqpoint{4.676095in}{0.902073in}}%
\pgfpathlineto{\pgfqpoint{4.676095in}{0.902073in}}%
\pgfusepath{stroke}%
\end{pgfscope}%
\begin{pgfscope}%
\pgfpathrectangle{\pgfqpoint{0.448634in}{0.402556in}}{\pgfqpoint{4.350661in}{2.489204in}} %
\pgfusepath{clip}%
\pgfsetrectcap%
\pgfsetroundjoin%
\pgfsetlinewidth{1.003750pt}%
\definecolor{currentstroke}{rgb}{0.121569,0.466667,0.705882}%
\pgfsetstrokecolor{currentstroke}%
\pgfsetdash{}{0pt}%
\pgfpathmoveto{\pgfqpoint{2.795520in}{1.982745in}}%
\pgfpathlineto{\pgfqpoint{2.781780in}{1.874357in}}%
\pgfpathlineto{\pgfqpoint{2.769351in}{1.758234in}}%
\pgfpathlineto{\pgfqpoint{2.758095in}{1.631942in}}%
\pgfpathlineto{\pgfqpoint{2.747786in}{1.490551in}}%
\pgfpathlineto{\pgfqpoint{2.738644in}{1.334082in}}%
\pgfpathlineto{\pgfqpoint{2.730580in}{1.157591in}}%
\pgfpathlineto{\pgfqpoint{2.723334in}{0.948663in}}%
\pgfpathlineto{\pgfqpoint{2.709783in}{0.530788in}}%
\pgfpathlineto{\pgfqpoint{2.705868in}{0.488716in}}%
\pgfpathlineto{\pgfqpoint{2.701769in}{0.464281in}}%
\pgfpathlineto{\pgfqpoint{2.697021in}{0.447744in}}%
\pgfpathlineto{\pgfqpoint{2.691859in}{0.436812in}}%
\pgfpathlineto{\pgfqpoint{2.686245in}{0.429229in}}%
\pgfpathlineto{\pgfqpoint{2.679348in}{0.423188in}}%
\pgfpathlineto{\pgfqpoint{2.669540in}{0.417856in}}%
\pgfpathlineto{\pgfqpoint{2.656987in}{0.413810in}}%
\pgfpathlineto{\pgfqpoint{2.637654in}{0.410337in}}%
\pgfpathlineto{\pgfqpoint{2.607297in}{0.407617in}}%
\pgfpathlineto{\pgfqpoint{2.555121in}{0.405574in}}%
\pgfpathlineto{\pgfqpoint{2.450714in}{0.404139in}}%
\pgfpathlineto{\pgfqpoint{2.176624in}{0.403275in}}%
\pgfpathlineto{\pgfqpoint{1.130290in}{0.402953in}}%
\pgfpathlineto{\pgfqpoint{0.516849in}{0.404175in}}%
\pgfpathlineto{\pgfqpoint{0.466848in}{0.405970in}}%
\pgfpathlineto{\pgfqpoint{0.456130in}{0.407931in}}%
\pgfpathlineto{\pgfqpoint{0.452340in}{0.410303in}}%
\pgfpathlineto{\pgfqpoint{0.450346in}{0.414662in}}%
\pgfpathlineto{\pgfqpoint{0.449266in}{0.424524in}}%
\pgfpathlineto{\pgfqpoint{0.448771in}{0.464344in}}%
\pgfpathlineto{\pgfqpoint{0.448640in}{0.850171in}}%
\pgfpathlineto{\pgfqpoint{0.448679in}{2.891318in}}%
\pgfpathlineto{\pgfqpoint{0.448679in}{2.891318in}}%
\pgfusepath{stroke}%
\end{pgfscope}%
\begin{pgfscope}%
\pgfpathrectangle{\pgfqpoint{0.448634in}{0.402556in}}{\pgfqpoint{4.350661in}{2.489204in}} %
\pgfusepath{clip}%
\pgfsetrectcap%
\pgfsetroundjoin%
\pgfsetlinewidth{1.003750pt}%
\definecolor{currentstroke}{rgb}{0.121569,0.466667,0.705882}%
\pgfsetstrokecolor{currentstroke}%
\pgfsetdash{}{0pt}%
\pgfpathmoveto{\pgfqpoint{3.428189in}{0.402586in}}%
\pgfpathlineto{\pgfqpoint{2.782121in}{0.403701in}}%
\pgfpathlineto{\pgfqpoint{2.753906in}{0.405674in}}%
\pgfpathlineto{\pgfqpoint{2.743328in}{0.408443in}}%
\pgfpathlineto{\pgfqpoint{2.737717in}{0.412188in}}%
\pgfpathlineto{\pgfqpoint{2.733668in}{0.417995in}}%
\pgfpathlineto{\pgfqpoint{2.730649in}{0.427307in}}%
\pgfpathlineto{\pgfqpoint{2.728388in}{0.442004in}}%
\pgfpathlineto{\pgfqpoint{2.726544in}{0.471794in}}%
\pgfpathlineto{\pgfqpoint{2.725216in}{0.534003in}}%
\pgfpathlineto{\pgfqpoint{2.725169in}{0.655972in}}%
\pgfpathlineto{\pgfqpoint{2.727377in}{0.832687in}}%
\pgfpathlineto{\pgfqpoint{2.732259in}{1.041703in}}%
\pgfpathlineto{\pgfqpoint{2.738851in}{1.223257in}}%
\pgfpathlineto{\pgfqpoint{2.747078in}{1.389766in}}%
\pgfpathlineto{\pgfqpoint{2.756608in}{1.538717in}}%
\pgfpathlineto{\pgfqpoint{2.768955in}{1.694887in}}%
\pgfpathlineto{\pgfqpoint{2.781228in}{1.816044in}}%
\pgfpathlineto{\pgfqpoint{2.794401in}{1.924524in}}%
\pgfpathlineto{\pgfqpoint{2.812737in}{2.054722in}}%
\pgfpathlineto{\pgfqpoint{2.828774in}{2.147512in}}%
\pgfpathlineto{\pgfqpoint{2.847382in}{2.242224in}}%
\pgfpathlineto{\pgfqpoint{2.895817in}{2.479699in}}%
\pgfpathlineto{\pgfqpoint{2.900204in}{2.516689in}}%
\pgfpathlineto{\pgfqpoint{2.901346in}{2.544029in}}%
\pgfpathlineto{\pgfqpoint{2.900291in}{2.566388in}}%
\pgfpathlineto{\pgfqpoint{2.897334in}{2.585999in}}%
\pgfpathlineto{\pgfqpoint{2.892836in}{2.602633in}}%
\pgfpathlineto{\pgfqpoint{2.886394in}{2.618405in}}%
\pgfpathlineto{\pgfqpoint{2.878058in}{2.632969in}}%
\pgfpathlineto{\pgfqpoint{2.868065in}{2.646100in}}%
\pgfpathlineto{\pgfqpoint{2.855050in}{2.659300in}}%
\pgfpathlineto{\pgfqpoint{2.840801in}{2.670717in}}%
\pgfpathlineto{\pgfqpoint{2.821822in}{2.682861in}}%
\pgfpathlineto{\pgfqpoint{2.799980in}{2.694026in}}%
\pgfpathlineto{\pgfqpoint{2.773366in}{2.704944in}}%
\pgfpathlineto{\pgfqpoint{2.742012in}{2.715266in}}%
\pgfpathlineto{\pgfqpoint{2.705983in}{2.724785in}}%
\pgfpathlineto{\pgfqpoint{2.663200in}{2.733810in}}%
\pgfpathlineto{\pgfqpoint{2.611535in}{2.742379in}}%
\pgfpathlineto{\pgfqpoint{2.551002in}{2.750090in}}%
\pgfpathlineto{\pgfqpoint{2.481632in}{2.756682in}}%
\pgfpathlineto{\pgfqpoint{2.399112in}{2.762200in}}%
\pgfpathlineto{\pgfqpoint{2.309985in}{2.765886in}}%
\pgfpathlineto{\pgfqpoint{2.188184in}{2.768096in}}%
\pgfpathlineto{\pgfqpoint{2.081595in}{2.767619in}}%
\pgfpathlineto{\pgfqpoint{1.968506in}{2.764840in}}%
\pgfpathlineto{\pgfqpoint{1.864180in}{2.759918in}}%
\pgfpathlineto{\pgfqpoint{1.757786in}{2.752593in}}%
\pgfpathlineto{\pgfqpoint{1.671087in}{2.744171in}}%
\pgfpathlineto{\pgfqpoint{1.591076in}{2.734193in}}%
\pgfpathlineto{\pgfqpoint{1.502689in}{2.720717in}}%
\pgfpathlineto{\pgfqpoint{1.427655in}{2.706083in}}%
\pgfpathlineto{\pgfqpoint{1.372350in}{2.692544in}}%
\pgfpathlineto{\pgfqpoint{1.321734in}{2.677921in}}%
\pgfpathlineto{\pgfqpoint{1.273765in}{2.661664in}}%
\pgfpathlineto{\pgfqpoint{1.230567in}{2.644672in}}%
\pgfpathlineto{\pgfqpoint{1.192197in}{2.627106in}}%
\pgfpathlineto{\pgfqpoint{1.156620in}{2.608403in}}%
\pgfpathlineto{\pgfqpoint{1.123890in}{2.588716in}}%
\pgfpathlineto{\pgfqpoint{1.095883in}{2.569568in}}%
\pgfpathlineto{\pgfqpoint{1.063936in}{2.543701in}}%
\pgfpathlineto{\pgfqpoint{1.038217in}{2.520732in}}%
\pgfpathlineto{\pgfqpoint{1.013766in}{2.496016in}}%
\pgfpathlineto{\pgfqpoint{0.990704in}{2.469610in}}%
\pgfpathlineto{\pgfqpoint{0.969124in}{2.441612in}}%
\pgfpathlineto{\pgfqpoint{0.949083in}{2.412154in}}%
\pgfpathlineto{\pgfqpoint{0.930604in}{2.381387in}}%
\pgfpathlineto{\pgfqpoint{0.906555in}{2.334052in}}%
\pgfpathlineto{\pgfqpoint{0.889925in}{2.296262in}}%
\pgfpathlineto{\pgfqpoint{0.874241in}{2.255213in}}%
\pgfpathlineto{\pgfqpoint{0.859667in}{2.210961in}}%
\pgfpathlineto{\pgfqpoint{0.846986in}{2.165954in}}%
\pgfpathlineto{\pgfqpoint{0.839633in}{2.134715in}}%
\pgfpathlineto{\pgfqpoint{0.828238in}{2.081532in}}%
\pgfpathlineto{\pgfqpoint{0.817866in}{2.022986in}}%
\pgfpathlineto{\pgfqpoint{0.810784in}{1.971352in}}%
\pgfpathlineto{\pgfqpoint{0.802846in}{1.902252in}}%
\pgfpathlineto{\pgfqpoint{0.796554in}{1.827927in}}%
\pgfpathlineto{\pgfqpoint{0.791696in}{1.743480in}}%
\pgfpathlineto{\pgfqpoint{0.787773in}{1.621595in}}%
\pgfpathlineto{\pgfqpoint{0.785408in}{1.522064in}}%
\pgfpathlineto{\pgfqpoint{0.785408in}{1.522064in}}%
\pgfusepath{stroke}%
\end{pgfscope}%
\begin{pgfscope}%
\pgfpathrectangle{\pgfqpoint{0.448634in}{0.402556in}}{\pgfqpoint{4.350661in}{2.489204in}} %
\pgfusepath{clip}%
\pgfsetrectcap%
\pgfsetroundjoin%
\pgfsetlinewidth{1.003750pt}%
\definecolor{currentstroke}{rgb}{1.000000,0.498039,0.054902}%
\pgfsetstrokecolor{currentstroke}%
\pgfsetdash{}{0pt}%
\pgfpathmoveto{\pgfqpoint{0.448634in}{2.896245in}}%
\pgfpathlineto{\pgfqpoint{0.448593in}{0.407043in}}%
\pgfpathlineto{\pgfqpoint{0.448593in}{0.407043in}}%
\pgfusepath{stroke}%
\end{pgfscope}%
\begin{pgfscope}%
\pgfpathrectangle{\pgfqpoint{0.448634in}{0.402556in}}{\pgfqpoint{4.350661in}{2.489204in}} %
\pgfusepath{clip}%
\pgfsetrectcap%
\pgfsetroundjoin%
\pgfsetlinewidth{1.003750pt}%
\definecolor{currentstroke}{rgb}{1.000000,0.498039,0.054902}%
\pgfsetstrokecolor{currentstroke}%
\pgfsetdash{}{0pt}%
\pgfpathmoveto{\pgfqpoint{0.576853in}{1.760816in}}%
\pgfpathlineto{\pgfqpoint{0.569394in}{1.840009in}}%
\pgfpathlineto{\pgfqpoint{0.563209in}{1.929338in}}%
\pgfpathlineto{\pgfqpoint{0.558592in}{2.028763in}}%
\pgfpathlineto{\pgfqpoint{0.555985in}{2.133265in}}%
\pgfpathlineto{\pgfqpoint{0.555566in}{2.237808in}}%
\pgfpathlineto{\pgfqpoint{0.557371in}{2.337351in}}%
\pgfpathlineto{\pgfqpoint{0.561096in}{2.424366in}}%
\pgfpathlineto{\pgfqpoint{0.566403in}{2.498791in}}%
\pgfpathlineto{\pgfqpoint{0.572909in}{2.560570in}}%
\pgfpathlineto{\pgfqpoint{0.580458in}{2.612118in}}%
\pgfpathlineto{\pgfqpoint{0.589086in}{2.655815in}}%
\pgfpathlineto{\pgfqpoint{0.598406in}{2.691589in}}%
\pgfpathlineto{\pgfqpoint{0.608613in}{2.721757in}}%
\pgfpathlineto{\pgfqpoint{0.619241in}{2.746278in}}%
\pgfpathlineto{\pgfqpoint{0.630816in}{2.767339in}}%
\pgfpathlineto{\pgfqpoint{0.642975in}{2.784883in}}%
\pgfpathlineto{\pgfqpoint{0.656813in}{2.800712in}}%
\pgfpathlineto{\pgfqpoint{0.672196in}{2.814549in}}%
\pgfpathlineto{\pgfqpoint{0.688853in}{2.826301in}}%
\pgfpathlineto{\pgfqpoint{0.706461in}{2.836076in}}%
\pgfpathlineto{\pgfqpoint{0.726804in}{2.844875in}}%
\pgfpathlineto{\pgfqpoint{0.751866in}{2.853203in}}%
\pgfpathlineto{\pgfqpoint{0.781631in}{2.860547in}}%
\pgfpathlineto{\pgfqpoint{0.818168in}{2.867054in}}%
\pgfpathlineto{\pgfqpoint{0.863581in}{2.872685in}}%
\pgfpathlineto{\pgfqpoint{0.922160in}{2.877518in}}%
\pgfpathlineto{\pgfqpoint{1.000391in}{2.881567in}}%
\pgfpathlineto{\pgfqpoint{1.111293in}{2.884881in}}%
\pgfpathlineto{\pgfqpoint{1.274428in}{2.887367in}}%
\pgfpathlineto{\pgfqpoint{1.552865in}{2.889263in}}%
\pgfpathlineto{\pgfqpoint{2.107573in}{2.890457in}}%
\pgfpathlineto{\pgfqpoint{3.343161in}{2.890573in}}%
\pgfpathlineto{\pgfqpoint{4.043615in}{2.888941in}}%
\pgfpathlineto{\pgfqpoint{4.289416in}{2.886404in}}%
\pgfpathlineto{\pgfqpoint{4.413375in}{2.883093in}}%
\pgfpathlineto{\pgfqpoint{4.489424in}{2.878997in}}%
\pgfpathlineto{\pgfqpoint{4.541451in}{2.874081in}}%
\pgfpathlineto{\pgfqpoint{4.578100in}{2.868470in}}%
\pgfpathlineto{\pgfqpoint{4.605818in}{2.862092in}}%
\pgfpathlineto{\pgfqpoint{4.626725in}{2.855245in}}%
\pgfpathlineto{\pgfqpoint{4.644925in}{2.847018in}}%
\pgfpathlineto{\pgfqpoint{4.660241in}{2.837590in}}%
\pgfpathlineto{\pgfqpoint{4.672623in}{2.827469in}}%
\pgfpathlineto{\pgfqpoint{4.683751in}{2.815592in}}%
\pgfpathlineto{\pgfqpoint{4.693406in}{2.802135in}}%
\pgfpathlineto{\pgfqpoint{4.702740in}{2.785343in}}%
\pgfpathlineto{\pgfqpoint{4.711277in}{2.765194in}}%
\pgfpathlineto{\pgfqpoint{4.719482in}{2.739484in}}%
\pgfpathlineto{\pgfqpoint{4.726293in}{2.710658in}}%
\pgfpathlineto{\pgfqpoint{4.733259in}{2.671643in}}%
\pgfpathlineto{\pgfqpoint{4.739604in}{2.622396in}}%
\pgfpathlineto{\pgfqpoint{4.745236in}{2.560504in}}%
\pgfpathlineto{\pgfqpoint{4.750164in}{2.481052in}}%
\pgfpathlineto{\pgfqpoint{4.754367in}{2.376618in}}%
\pgfpathlineto{\pgfqpoint{4.757443in}{2.242249in}}%
\pgfpathlineto{\pgfqpoint{4.758977in}{2.075483in}}%
\pgfpathlineto{\pgfqpoint{4.758447in}{1.888795in}}%
\pgfpathlineto{\pgfqpoint{4.755756in}{1.707111in}}%
\pgfpathlineto{\pgfqpoint{4.750925in}{1.532958in}}%
\pgfpathlineto{\pgfqpoint{4.744785in}{1.398727in}}%
\pgfpathlineto{\pgfqpoint{4.737575in}{1.289516in}}%
\pgfpathlineto{\pgfqpoint{4.728714in}{1.190470in}}%
\pgfpathlineto{\pgfqpoint{4.719652in}{1.116521in}}%
\pgfpathlineto{\pgfqpoint{4.710036in}{1.055276in}}%
\pgfpathlineto{\pgfqpoint{4.699503in}{1.001861in}}%
\pgfpathlineto{\pgfqpoint{4.689040in}{0.958690in}}%
\pgfpathlineto{\pgfqpoint{4.677219in}{0.918600in}}%
\pgfpathlineto{\pgfqpoint{4.664034in}{0.881749in}}%
\pgfpathlineto{\pgfqpoint{4.650584in}{0.850492in}}%
\pgfpathlineto{\pgfqpoint{4.636303in}{0.822570in}}%
\pgfpathlineto{\pgfqpoint{4.620207in}{0.795975in}}%
\pgfpathlineto{\pgfqpoint{4.603640in}{0.772902in}}%
\pgfpathlineto{\pgfqpoint{4.585488in}{0.751446in}}%
\pgfpathlineto{\pgfqpoint{4.565874in}{0.731749in}}%
\pgfpathlineto{\pgfqpoint{4.544964in}{0.713879in}}%
\pgfpathlineto{\pgfqpoint{4.522958in}{0.697824in}}%
\pgfpathlineto{\pgfqpoint{4.496157in}{0.681290in}}%
\pgfpathlineto{\pgfqpoint{4.470397in}{0.667953in}}%
\pgfpathlineto{\pgfqpoint{4.439961in}{0.654510in}}%
\pgfpathlineto{\pgfqpoint{4.406841in}{0.642282in}}%
\pgfpathlineto{\pgfqpoint{4.369009in}{0.630749in}}%
\pgfpathlineto{\pgfqpoint{4.326489in}{0.620226in}}%
\pgfpathlineto{\pgfqpoint{4.279327in}{0.610949in}}%
\pgfpathlineto{\pgfqpoint{4.227576in}{0.603085in}}%
\pgfpathlineto{\pgfqpoint{4.173450in}{0.597063in}}%
\pgfpathlineto{\pgfqpoint{4.110511in}{0.592203in}}%
\pgfpathlineto{\pgfqpoint{4.047471in}{0.589537in}}%
\pgfpathlineto{\pgfqpoint{3.977867in}{0.588624in}}%
\pgfpathlineto{\pgfqpoint{3.906093in}{0.589934in}}%
\pgfpathlineto{\pgfqpoint{3.834377in}{0.593496in}}%
\pgfpathlineto{\pgfqpoint{3.767120in}{0.599067in}}%
\pgfpathlineto{\pgfqpoint{3.704364in}{0.606392in}}%
\pgfpathlineto{\pgfqpoint{3.678516in}{0.610510in}}%
\pgfpathlineto{\pgfqpoint{3.620438in}{0.620500in}}%
\pgfpathlineto{\pgfqpoint{3.586319in}{0.628207in}}%
\pgfpathlineto{\pgfqpoint{3.495240in}{0.652428in}}%
\pgfpathlineto{\pgfqpoint{3.451528in}{0.667583in}}%
\pgfpathlineto{\pgfqpoint{3.408538in}{0.685220in}}%
\pgfpathlineto{\pgfqpoint{3.374594in}{0.702001in}}%
\pgfpathlineto{\pgfqpoint{3.345407in}{0.718682in}}%
\pgfpathlineto{\pgfqpoint{3.315236in}{0.738520in}}%
\pgfpathlineto{\pgfqpoint{3.288127in}{0.759290in}}%
\pgfpathlineto{\pgfqpoint{3.264004in}{0.780551in}}%
\pgfpathlineto{\pgfqpoint{3.241208in}{0.803648in}}%
\pgfpathlineto{\pgfqpoint{3.219894in}{0.828530in}}%
\pgfpathlineto{\pgfqpoint{3.200189in}{0.855091in}}%
\pgfpathlineto{\pgfqpoint{3.182177in}{0.883182in}}%
\pgfpathlineto{\pgfqpoint{3.165906in}{0.912633in}}%
\pgfpathlineto{\pgfqpoint{3.150351in}{0.945448in}}%
\pgfpathlineto{\pgfqpoint{3.136682in}{0.979345in}}%
\pgfpathlineto{\pgfqpoint{3.124073in}{1.016460in}}%
\pgfpathlineto{\pgfqpoint{3.112834in}{1.056769in}}%
\pgfpathlineto{\pgfqpoint{3.103046in}{1.100146in}}%
\pgfpathlineto{\pgfqpoint{3.095343in}{1.144071in}}%
\pgfpathlineto{\pgfqpoint{3.089208in}{1.190837in}}%
\pgfpathlineto{\pgfqpoint{3.084595in}{1.242838in}}%
\pgfpathlineto{\pgfqpoint{3.082137in}{1.295031in}}%
\pgfpathlineto{\pgfqpoint{3.081687in}{1.349787in}}%
\pgfpathlineto{\pgfqpoint{3.083451in}{1.406998in}}%
\pgfpathlineto{\pgfqpoint{3.087181in}{1.461589in}}%
\pgfpathlineto{\pgfqpoint{3.093485in}{1.520888in}}%
\pgfpathlineto{\pgfqpoint{3.101823in}{1.577334in}}%
\pgfpathlineto{\pgfqpoint{3.111930in}{1.630856in}}%
\pgfpathlineto{\pgfqpoint{3.124690in}{1.686208in}}%
\pgfpathlineto{\pgfqpoint{3.139178in}{1.738395in}}%
\pgfpathlineto{\pgfqpoint{3.155145in}{1.787366in}}%
\pgfpathlineto{\pgfqpoint{3.172353in}{1.833085in}}%
\pgfpathlineto{\pgfqpoint{3.191618in}{1.877716in}}%
\pgfpathlineto{\pgfqpoint{3.214026in}{1.923261in}}%
\pgfpathlineto{\pgfqpoint{3.236214in}{1.963157in}}%
\pgfpathlineto{\pgfqpoint{3.260178in}{2.001684in}}%
\pgfpathlineto{\pgfqpoint{3.285814in}{2.038776in}}%
\pgfpathlineto{\pgfqpoint{3.314415in}{2.076285in}}%
\pgfpathlineto{\pgfqpoint{3.348944in}{2.117711in}}%
\pgfpathlineto{\pgfqpoint{3.417133in}{2.198022in}}%
\pgfpathlineto{\pgfqpoint{3.426053in}{2.212128in}}%
\pgfpathlineto{\pgfqpoint{3.430798in}{2.223297in}}%
\pgfpathlineto{\pgfqpoint{3.432034in}{2.230603in}}%
\pgfpathlineto{\pgfqpoint{3.430773in}{2.237856in}}%
\pgfpathlineto{\pgfqpoint{3.426621in}{2.243526in}}%
\pgfpathlineto{\pgfqpoint{3.420908in}{2.247084in}}%
\pgfpathlineto{\pgfqpoint{3.412501in}{2.249583in}}%
\pgfpathlineto{\pgfqpoint{3.399499in}{2.250689in}}%
\pgfpathlineto{\pgfqpoint{3.384305in}{2.249671in}}%
\pgfpathlineto{\pgfqpoint{3.364985in}{2.246098in}}%
\pgfpathlineto{\pgfqpoint{3.341804in}{2.239342in}}%
\pgfpathlineto{\pgfqpoint{3.317109in}{2.229682in}}%
\pgfpathlineto{\pgfqpoint{3.291104in}{2.216986in}}%
\pgfpathlineto{\pgfqpoint{3.265928in}{2.202261in}}%
\pgfpathlineto{\pgfqpoint{3.239805in}{2.184361in}}%
\pgfpathlineto{\pgfqpoint{3.214775in}{2.164519in}}%
\pgfpathlineto{\pgfqpoint{3.190900in}{2.142893in}}%
\pgfpathlineto{\pgfqpoint{3.166657in}{2.117912in}}%
\pgfpathlineto{\pgfqpoint{3.143835in}{2.091233in}}%
\pgfpathlineto{\pgfqpoint{3.121079in}{2.061107in}}%
\pgfpathlineto{\pgfqpoint{3.099952in}{2.029463in}}%
\pgfpathlineto{\pgfqpoint{3.079251in}{1.994406in}}%
\pgfpathlineto{\pgfqpoint{3.059218in}{1.955915in}}%
\pgfpathlineto{\pgfqpoint{3.040058in}{1.914015in}}%
\pgfpathlineto{\pgfqpoint{3.022809in}{1.871041in}}%
\pgfpathlineto{\pgfqpoint{3.005790in}{1.822536in}}%
\pgfpathlineto{\pgfqpoint{2.990067in}{1.770819in}}%
\pgfpathlineto{\pgfqpoint{2.975708in}{1.715979in}}%
\pgfpathlineto{\pgfqpoint{2.962284in}{1.655680in}}%
\pgfpathlineto{\pgfqpoint{2.950496in}{1.592386in}}%
\pgfpathlineto{\pgfqpoint{2.940383in}{1.526185in}}%
\pgfpathlineto{\pgfqpoint{2.931745in}{1.454681in}}%
\pgfpathlineto{\pgfqpoint{2.925082in}{1.380399in}}%
\pgfpathlineto{\pgfqpoint{2.920647in}{1.305899in}}%
\pgfpathlineto{\pgfqpoint{2.918444in}{1.231270in}}%
\pgfpathlineto{\pgfqpoint{2.918545in}{1.159087in}}%
\pgfpathlineto{\pgfqpoint{2.920787in}{1.091931in}}%
\pgfpathlineto{\pgfqpoint{2.925177in}{1.027412in}}%
\pgfpathlineto{\pgfqpoint{2.931192in}{0.970580in}}%
\pgfpathlineto{\pgfqpoint{2.938760in}{0.919034in}}%
\pgfpathlineto{\pgfqpoint{2.947651in}{0.872852in}}%
\pgfpathlineto{\pgfqpoint{2.958213in}{0.829714in}}%
\pgfpathlineto{\pgfqpoint{2.969670in}{0.792114in}}%
\pgfpathlineto{\pgfqpoint{2.982463in}{0.757773in}}%
\pgfpathlineto{\pgfqpoint{2.996425in}{0.726812in}}%
\pgfpathlineto{\pgfqpoint{3.011299in}{0.699300in}}%
\pgfpathlineto{\pgfqpoint{3.026739in}{0.675225in}}%
\pgfpathlineto{\pgfqpoint{3.043828in}{0.652656in}}%
\pgfpathlineto{\pgfqpoint{3.062495in}{0.631788in}}%
\pgfpathlineto{\pgfqpoint{3.082602in}{0.612753in}}%
\pgfpathlineto{\pgfqpoint{3.103961in}{0.595592in}}%
\pgfpathlineto{\pgfqpoint{3.128268in}{0.579069in}}%
\pgfpathlineto{\pgfqpoint{3.153537in}{0.564554in}}%
\pgfpathlineto{\pgfqpoint{3.181571in}{0.550952in}}%
\pgfpathlineto{\pgfqpoint{3.214371in}{0.537647in}}%
\pgfpathlineto{\pgfqpoint{3.249846in}{0.525712in}}%
\pgfpathlineto{\pgfqpoint{3.290011in}{0.514571in}}%
\pgfpathlineto{\pgfqpoint{3.334820in}{0.504423in}}%
\pgfpathlineto{\pgfqpoint{3.386372in}{0.494999in}}%
\pgfpathlineto{\pgfqpoint{3.446798in}{0.486257in}}%
\pgfpathlineto{\pgfqpoint{3.518243in}{0.478282in}}%
\pgfpathlineto{\pgfqpoint{3.600685in}{0.471409in}}%
\pgfpathlineto{\pgfqpoint{3.696268in}{0.465713in}}%
\pgfpathlineto{\pgfqpoint{3.807144in}{0.461369in}}%
\pgfpathlineto{\pgfqpoint{3.933291in}{0.458719in}}%
\pgfpathlineto{\pgfqpoint{4.063808in}{0.458211in}}%
\pgfpathlineto{\pgfqpoint{4.187792in}{0.459914in}}%
\pgfpathlineto{\pgfqpoint{4.294335in}{0.463521in}}%
\pgfpathlineto{\pgfqpoint{4.381234in}{0.468574in}}%
\pgfpathlineto{\pgfqpoint{4.450636in}{0.474701in}}%
\pgfpathlineto{\pgfqpoint{4.506850in}{0.481799in}}%
\pgfpathlineto{\pgfqpoint{4.552009in}{0.489658in}}%
\pgfpathlineto{\pgfqpoint{4.588239in}{0.498115in}}%
\pgfpathlineto{\pgfqpoint{4.617656in}{0.507110in}}%
\pgfpathlineto{\pgfqpoint{4.642328in}{0.516843in}}%
\pgfpathlineto{\pgfqpoint{4.664194in}{0.527940in}}%
\pgfpathlineto{\pgfqpoint{4.681238in}{0.538945in}}%
\pgfpathlineto{\pgfqpoint{4.697164in}{0.551953in}}%
\pgfpathlineto{\pgfqpoint{4.710076in}{0.565289in}}%
\pgfpathlineto{\pgfqpoint{4.721578in}{0.580218in}}%
\pgfpathlineto{\pgfqpoint{4.731557in}{0.596521in}}%
\pgfpathlineto{\pgfqpoint{4.741000in}{0.616134in}}%
\pgfpathlineto{\pgfqpoint{4.749521in}{0.639027in}}%
\pgfpathlineto{\pgfqpoint{4.757522in}{0.667450in}}%
\pgfpathlineto{\pgfqpoint{4.764572in}{0.701345in}}%
\pgfpathlineto{\pgfqpoint{4.770840in}{0.743043in}}%
\pgfpathlineto{\pgfqpoint{4.776327in}{0.794934in}}%
\pgfpathlineto{\pgfqpoint{4.781278in}{0.864398in}}%
\pgfpathlineto{\pgfqpoint{4.785468in}{0.956371in}}%
\pgfpathlineto{\pgfqpoint{4.789000in}{1.085745in}}%
\pgfpathlineto{\pgfqpoint{4.791852in}{1.277385in}}%
\pgfpathlineto{\pgfqpoint{4.793959in}{1.581057in}}%
\pgfpathlineto{\pgfqpoint{4.794962in}{2.071429in}}%
\pgfpathlineto{\pgfqpoint{4.793967in}{2.559311in}}%
\pgfpathlineto{\pgfqpoint{4.791733in}{2.745981in}}%
\pgfpathlineto{\pgfqpoint{4.788955in}{2.818091in}}%
\pgfpathlineto{\pgfqpoint{4.785731in}{2.850227in}}%
\pgfpathlineto{\pgfqpoint{4.781879in}{2.867057in}}%
\pgfpathlineto{\pgfqpoint{4.777744in}{2.875780in}}%
\pgfpathlineto{\pgfqpoint{4.773097in}{2.880982in}}%
\pgfpathlineto{\pgfqpoint{4.767363in}{2.884504in}}%
\pgfpathlineto{\pgfqpoint{4.756853in}{2.887622in}}%
\pgfpathlineto{\pgfqpoint{4.739548in}{2.889639in}}%
\pgfpathlineto{\pgfqpoint{4.704762in}{2.890882in}}%
\pgfpathlineto{\pgfqpoint{4.602524in}{2.891538in}}%
\pgfpathlineto{\pgfqpoint{3.952100in}{2.891742in}}%
\pgfpathlineto{\pgfqpoint{0.617321in}{2.890753in}}%
\pgfpathlineto{\pgfqpoint{0.549910in}{2.888858in}}%
\pgfpathlineto{\pgfqpoint{0.521735in}{2.886179in}}%
\pgfpathlineto{\pgfqpoint{0.504666in}{2.882389in}}%
\pgfpathlineto{\pgfqpoint{0.494501in}{2.878011in}}%
\pgfpathlineto{\pgfqpoint{0.487180in}{2.872667in}}%
\pgfpathlineto{\pgfqpoint{0.481152in}{2.865519in}}%
\pgfpathlineto{\pgfqpoint{0.475664in}{2.854804in}}%
\pgfpathlineto{\pgfqpoint{0.471318in}{2.840737in}}%
\pgfpathlineto{\pgfqpoint{0.467301in}{2.818823in}}%
\pgfpathlineto{\pgfqpoint{0.463927in}{2.786700in}}%
\pgfpathlineto{\pgfqpoint{0.460918in}{2.734544in}}%
\pgfpathlineto{\pgfqpoint{0.458363in}{2.647473in}}%
\pgfpathlineto{\pgfqpoint{0.456575in}{2.523031in}}%
\pgfpathlineto{\pgfqpoint{0.456575in}{2.523031in}}%
\pgfusepath{stroke}%
\end{pgfscope}%
\begin{pgfscope}%
\pgfpathrectangle{\pgfqpoint{0.448634in}{0.402556in}}{\pgfqpoint{4.350661in}{2.489204in}} %
\pgfusepath{clip}%
\pgfsetrectcap%
\pgfsetroundjoin%
\pgfsetlinewidth{1.003750pt}%
\definecolor{currentstroke}{rgb}{1.000000,0.498039,0.054902}%
\pgfsetstrokecolor{currentstroke}%
\pgfsetdash{}{0pt}%
\pgfpathmoveto{\pgfqpoint{4.798840in}{2.852369in}}%
\pgfpathlineto{\pgfqpoint{4.797564in}{2.889610in}}%
\pgfpathlineto{\pgfqpoint{4.796215in}{2.891483in}}%
\pgfpathlineto{\pgfqpoint{4.787551in}{2.891760in}}%
\pgfpathlineto{\pgfqpoint{0.452128in}{2.891659in}}%
\pgfpathlineto{\pgfqpoint{0.450530in}{2.890082in}}%
\pgfpathlineto{\pgfqpoint{0.449454in}{2.882763in}}%
\pgfpathlineto{\pgfqpoint{0.448970in}{2.845432in}}%
\pgfpathlineto{\pgfqpoint{0.448743in}{2.494454in}}%
\pgfpathlineto{\pgfqpoint{0.449624in}{0.615107in}}%
\pgfpathlineto{\pgfqpoint{0.451433in}{0.510586in}}%
\pgfpathlineto{\pgfqpoint{0.453993in}{0.473374in}}%
\pgfpathlineto{\pgfqpoint{0.457406in}{0.453868in}}%
\pgfpathlineto{\pgfqpoint{0.461540in}{0.442384in}}%
\pgfpathlineto{\pgfqpoint{0.466739in}{0.434437in}}%
\pgfpathlineto{\pgfqpoint{0.473595in}{0.428350in}}%
\pgfpathlineto{\pgfqpoint{0.483492in}{0.423244in}}%
\pgfpathlineto{\pgfqpoint{0.491854in}{0.420501in}}%
\pgfpathlineto{\pgfqpoint{0.491854in}{0.420501in}}%
\pgfusepath{stroke}%
\end{pgfscope}%
\begin{pgfscope}%
\pgfpathrectangle{\pgfqpoint{0.448634in}{0.402556in}}{\pgfqpoint{4.350661in}{2.489204in}} %
\pgfusepath{clip}%
\pgfsetrectcap%
\pgfsetroundjoin%
\pgfsetlinewidth{1.003750pt}%
\definecolor{currentstroke}{rgb}{1.000000,0.498039,0.054902}%
\pgfsetstrokecolor{currentstroke}%
\pgfsetdash{}{0pt}%
\pgfpathmoveto{\pgfqpoint{0.456424in}{1.370137in}}%
\pgfpathlineto{\pgfqpoint{0.459610in}{1.118755in}}%
\pgfpathlineto{\pgfqpoint{0.463695in}{0.962007in}}%
\pgfpathlineto{\pgfqpoint{0.468519in}{0.857610in}}%
\pgfpathlineto{\pgfqpoint{0.474082in}{0.783210in}}%
\pgfpathlineto{\pgfqpoint{0.480226in}{0.728906in}}%
\pgfpathlineto{\pgfqpoint{0.486970in}{0.687306in}}%
\pgfpathlineto{\pgfqpoint{0.494537in}{0.653558in}}%
\pgfpathlineto{\pgfqpoint{0.503107in}{0.625355in}}%
\pgfpathlineto{\pgfqpoint{0.512193in}{0.602749in}}%
\pgfpathlineto{\pgfqpoint{0.522200in}{0.583508in}}%
\pgfpathlineto{\pgfqpoint{0.534108in}{0.565743in}}%
\pgfpathlineto{\pgfqpoint{0.546263in}{0.551507in}}%
\pgfpathlineto{\pgfqpoint{0.559728in}{0.538907in}}%
\pgfpathlineto{\pgfqpoint{0.576129in}{0.526693in}}%
\pgfpathlineto{\pgfqpoint{0.595483in}{0.515351in}}%
\pgfpathlineto{\pgfqpoint{0.617681in}{0.505147in}}%
\pgfpathlineto{\pgfqpoint{0.642568in}{0.496153in}}%
\pgfpathlineto{\pgfqpoint{0.672126in}{0.487778in}}%
\pgfpathlineto{\pgfqpoint{0.708443in}{0.479824in}}%
\pgfpathlineto{\pgfqpoint{0.753649in}{0.472325in}}%
\pgfpathlineto{\pgfqpoint{0.807717in}{0.465660in}}%
\pgfpathlineto{\pgfqpoint{0.877116in}{0.459475in}}%
\pgfpathlineto{\pgfqpoint{0.961828in}{0.454230in}}%
\pgfpathlineto{\pgfqpoint{1.068351in}{0.449916in}}%
\pgfpathlineto{\pgfqpoint{1.201018in}{0.446839in}}%
\pgfpathlineto{\pgfqpoint{1.357637in}{0.445481in}}%
\pgfpathlineto{\pgfqpoint{1.525135in}{0.446232in}}%
\pgfpathlineto{\pgfqpoint{1.686088in}{0.449142in}}%
\pgfpathlineto{\pgfqpoint{1.823074in}{0.453747in}}%
\pgfpathlineto{\pgfqpoint{1.938245in}{0.459764in}}%
\pgfpathlineto{\pgfqpoint{2.031582in}{0.466759in}}%
\pgfpathlineto{\pgfqpoint{2.109580in}{0.474745in}}%
\pgfpathlineto{\pgfqpoint{2.174384in}{0.483535in}}%
\pgfpathlineto{\pgfqpoint{2.228139in}{0.492940in}}%
\pgfpathlineto{\pgfqpoint{2.275119in}{0.503356in}}%
\pgfpathlineto{\pgfqpoint{2.315282in}{0.514501in}}%
\pgfpathlineto{\pgfqpoint{2.350698in}{0.526659in}}%
\pgfpathlineto{\pgfqpoint{2.381320in}{0.539536in}}%
\pgfpathlineto{\pgfqpoint{2.407164in}{0.552659in}}%
\pgfpathlineto{\pgfqpoint{2.430226in}{0.566639in}}%
\pgfpathlineto{\pgfqpoint{2.452282in}{0.582602in}}%
\pgfpathlineto{\pgfqpoint{2.471391in}{0.599069in}}%
\pgfpathlineto{\pgfqpoint{2.489240in}{0.617293in}}%
\pgfpathlineto{\pgfqpoint{2.505678in}{0.637180in}}%
\pgfpathlineto{\pgfqpoint{2.520620in}{0.658557in}}%
\pgfpathlineto{\pgfqpoint{2.535213in}{0.683314in}}%
\pgfpathlineto{\pgfqpoint{2.549115in}{0.711484in}}%
\pgfpathlineto{\pgfqpoint{2.562091in}{0.743004in}}%
\pgfpathlineto{\pgfqpoint{2.574020in}{0.777751in}}%
\pgfpathlineto{\pgfqpoint{2.585502in}{0.817970in}}%
\pgfpathlineto{\pgfqpoint{2.596809in}{0.866038in}}%
\pgfpathlineto{\pgfqpoint{2.607562in}{0.921948in}}%
\pgfpathlineto{\pgfqpoint{2.617925in}{0.988098in}}%
\pgfpathlineto{\pgfqpoint{2.627958in}{1.066918in}}%
\pgfpathlineto{\pgfqpoint{2.637941in}{1.163320in}}%
\pgfpathlineto{\pgfqpoint{2.648424in}{1.287199in}}%
\pgfpathlineto{\pgfqpoint{2.660103in}{1.453438in}}%
\pgfpathlineto{\pgfqpoint{2.674773in}{1.696801in}}%
\pgfpathlineto{\pgfqpoint{2.687716in}{1.945279in}}%
\pgfpathlineto{\pgfqpoint{2.692670in}{2.079573in}}%
\pgfpathlineto{\pgfqpoint{2.693829in}{2.166682in}}%
\pgfpathlineto{\pgfqpoint{2.692565in}{2.233870in}}%
\pgfpathlineto{\pgfqpoint{2.689436in}{2.286015in}}%
\pgfpathlineto{\pgfqpoint{2.684859in}{2.327999in}}%
\pgfpathlineto{\pgfqpoint{2.678725in}{2.364664in}}%
\pgfpathlineto{\pgfqpoint{2.671356in}{2.395897in}}%
\pgfpathlineto{\pgfqpoint{2.662489in}{2.423981in}}%
\pgfpathlineto{\pgfqpoint{2.652361in}{2.448778in}}%
\pgfpathlineto{\pgfqpoint{2.641365in}{2.470245in}}%
\pgfpathlineto{\pgfqpoint{2.628643in}{2.490425in}}%
\pgfpathlineto{\pgfqpoint{2.614279in}{2.509106in}}%
\pgfpathlineto{\pgfqpoint{2.598443in}{2.526159in}}%
\pgfpathlineto{\pgfqpoint{2.579590in}{2.543005in}}%
\pgfpathlineto{\pgfqpoint{2.559532in}{2.557923in}}%
\pgfpathlineto{\pgfqpoint{2.536602in}{2.572183in}}%
\pgfpathlineto{\pgfqpoint{2.510850in}{2.585538in}}%
\pgfpathlineto{\pgfqpoint{2.482360in}{2.597837in}}%
\pgfpathlineto{\pgfqpoint{2.449134in}{2.609683in}}%
\pgfpathlineto{\pgfqpoint{2.411184in}{2.620696in}}%
\pgfpathlineto{\pgfqpoint{2.368552in}{2.630606in}}%
\pgfpathlineto{\pgfqpoint{2.321294in}{2.639221in}}%
\pgfpathlineto{\pgfqpoint{2.269467in}{2.646399in}}%
\pgfpathlineto{\pgfqpoint{2.210954in}{2.652193in}}%
\pgfpathlineto{\pgfqpoint{2.147967in}{2.656153in}}%
\pgfpathlineto{\pgfqpoint{2.080556in}{2.658135in}}%
\pgfpathlineto{\pgfqpoint{2.010948in}{2.657971in}}%
\pgfpathlineto{\pgfqpoint{1.939195in}{2.655572in}}%
\pgfpathlineto{\pgfqpoint{1.867527in}{2.650913in}}%
\pgfpathlineto{\pgfqpoint{1.798171in}{2.644140in}}%
\pgfpathlineto{\pgfqpoint{1.733341in}{2.635606in}}%
\pgfpathlineto{\pgfqpoint{1.673075in}{2.625521in}}%
\pgfpathlineto{\pgfqpoint{1.615274in}{2.613610in}}%
\pgfpathlineto{\pgfqpoint{1.562133in}{2.600402in}}%
\pgfpathlineto{\pgfqpoint{1.513681in}{2.586139in}}%
\pgfpathlineto{\pgfqpoint{1.467862in}{2.570344in}}%
\pgfpathlineto{\pgfqpoint{1.426794in}{2.553923in}}%
\pgfpathlineto{\pgfqpoint{1.388447in}{2.536289in}}%
\pgfpathlineto{\pgfqpoint{1.352878in}{2.517566in}}%
\pgfpathlineto{\pgfqpoint{1.320128in}{2.497922in}}%
\pgfpathlineto{\pgfqpoint{1.288379in}{2.476236in}}%
\pgfpathlineto{\pgfqpoint{1.259592in}{2.453861in}}%
\pgfpathlineto{\pgfqpoint{1.232050in}{2.429520in}}%
\pgfpathlineto{\pgfqpoint{1.207527in}{2.404898in}}%
\pgfpathlineto{\pgfqpoint{1.184409in}{2.378557in}}%
\pgfpathlineto{\pgfqpoint{1.162828in}{2.350561in}}%
\pgfpathlineto{\pgfqpoint{1.142891in}{2.321011in}}%
\pgfpathlineto{\pgfqpoint{1.124675in}{2.290041in}}%
\pgfpathlineto{\pgfqpoint{1.108225in}{2.257802in}}%
\pgfpathlineto{\pgfqpoint{1.092639in}{2.222199in}}%
\pgfpathlineto{\pgfqpoint{1.079059in}{2.185535in}}%
\pgfpathlineto{\pgfqpoint{1.067443in}{2.147998in}}%
\pgfpathlineto{\pgfqpoint{1.057187in}{2.107348in}}%
\pgfpathlineto{\pgfqpoint{1.049004in}{2.066086in}}%
\pgfpathlineto{\pgfqpoint{1.042513in}{2.021906in}}%
\pgfpathlineto{\pgfqpoint{1.038177in}{1.977382in}}%
\pgfpathlineto{\pgfqpoint{1.035866in}{1.930167in}}%
\pgfpathlineto{\pgfqpoint{1.035826in}{1.882878in}}%
\pgfpathlineto{\pgfqpoint{1.038031in}{1.835656in}}%
\pgfpathlineto{\pgfqpoint{1.042474in}{1.788641in}}%
\pgfpathlineto{\pgfqpoint{1.049176in}{1.741979in}}%
\pgfpathlineto{\pgfqpoint{1.057644in}{1.698239in}}%
\pgfpathlineto{\pgfqpoint{1.068221in}{1.655105in}}%
\pgfpathlineto{\pgfqpoint{1.080962in}{1.612745in}}%
\pgfpathlineto{\pgfqpoint{1.095031in}{1.573617in}}%
\pgfpathlineto{\pgfqpoint{1.111115in}{1.535520in}}%
\pgfpathlineto{\pgfqpoint{1.128118in}{1.500775in}}%
\pgfpathlineto{\pgfqpoint{1.146930in}{1.467274in}}%
\pgfpathlineto{\pgfqpoint{1.167531in}{1.435181in}}%
\pgfpathlineto{\pgfqpoint{1.189874in}{1.404652in}}%
\pgfpathlineto{\pgfqpoint{1.213884in}{1.375828in}}%
\pgfpathlineto{\pgfqpoint{1.237817in}{1.350457in}}%
\pgfpathlineto{\pgfqpoint{1.264748in}{1.325237in}}%
\pgfpathlineto{\pgfqpoint{1.292991in}{1.301972in}}%
\pgfpathlineto{\pgfqpoint{1.322398in}{1.280678in}}%
\pgfpathlineto{\pgfqpoint{1.352820in}{1.261340in}}%
\pgfpathlineto{\pgfqpoint{1.386095in}{1.242889in}}%
\pgfpathlineto{\pgfqpoint{1.420190in}{1.226516in}}%
\pgfpathlineto{\pgfqpoint{1.457024in}{1.211329in}}%
\pgfpathlineto{\pgfqpoint{1.496554in}{1.197536in}}%
\pgfpathlineto{\pgfqpoint{1.538719in}{1.185287in}}%
\pgfpathlineto{\pgfqpoint{1.583441in}{1.174641in}}%
\pgfpathlineto{\pgfqpoint{1.634929in}{1.164775in}}%
\pgfpathlineto{\pgfqpoint{1.706063in}{1.153745in}}%
\pgfpathlineto{\pgfqpoint{1.768492in}{1.143417in}}%
\pgfpathlineto{\pgfqpoint{1.796122in}{1.136567in}}%
\pgfpathlineto{\pgfqpoint{1.812683in}{1.130481in}}%
\pgfpathlineto{\pgfqpoint{1.824471in}{1.124102in}}%
\pgfpathlineto{\pgfqpoint{1.833209in}{1.116741in}}%
\pgfpathlineto{\pgfqpoint{1.838498in}{1.108890in}}%
\pgfpathlineto{\pgfqpoint{1.840588in}{1.101849in}}%
\pgfpathlineto{\pgfqpoint{1.840619in}{1.094412in}}%
\pgfpathlineto{\pgfqpoint{1.837931in}{1.084986in}}%
\pgfpathlineto{\pgfqpoint{1.833246in}{1.076615in}}%
\pgfpathlineto{\pgfqpoint{1.825819in}{1.067542in}}%
\pgfpathlineto{\pgfqpoint{1.813813in}{1.056850in}}%
\pgfpathlineto{\pgfqpoint{1.798819in}{1.046763in}}%
\pgfpathlineto{\pgfqpoint{1.781016in}{1.037462in}}%
\pgfpathlineto{\pgfqpoint{1.758447in}{1.028391in}}%
\pgfpathlineto{\pgfqpoint{1.733203in}{1.020815in}}%
\pgfpathlineto{\pgfqpoint{1.705410in}{1.014872in}}%
\pgfpathlineto{\pgfqpoint{1.675178in}{1.010714in}}%
\pgfpathlineto{\pgfqpoint{1.642610in}{1.008507in}}%
\pgfpathlineto{\pgfqpoint{1.607809in}{1.008432in}}%
\pgfpathlineto{\pgfqpoint{1.570886in}{1.010691in}}%
\pgfpathlineto{\pgfqpoint{1.534118in}{1.015181in}}%
\pgfpathlineto{\pgfqpoint{1.495454in}{1.022233in}}%
\pgfpathlineto{\pgfqpoint{1.457161in}{1.031563in}}%
\pgfpathlineto{\pgfqpoint{1.419337in}{1.043132in}}%
\pgfpathlineto{\pgfqpoint{1.382089in}{1.056929in}}%
\pgfpathlineto{\pgfqpoint{1.347544in}{1.072019in}}%
\pgfpathlineto{\pgfqpoint{1.313727in}{1.089133in}}%
\pgfpathlineto{\pgfqpoint{1.280762in}{1.108299in}}%
\pgfpathlineto{\pgfqpoint{1.248782in}{1.129536in}}%
\pgfpathlineto{\pgfqpoint{1.219708in}{1.151422in}}%
\pgfpathlineto{\pgfqpoint{1.191752in}{1.175138in}}%
\pgfpathlineto{\pgfqpoint{1.165031in}{1.200649in}}%
\pgfpathlineto{\pgfqpoint{1.139653in}{1.227898in}}%
\pgfpathlineto{\pgfqpoint{1.115714in}{1.256800in}}%
\pgfpathlineto{\pgfqpoint{1.093288in}{1.287251in}}%
\pgfpathlineto{\pgfqpoint{1.071178in}{1.321163in}}%
\pgfpathlineto{\pgfqpoint{1.050868in}{1.356520in}}%
\pgfpathlineto{\pgfqpoint{1.032365in}{1.393152in}}%
\pgfpathlineto{\pgfqpoint{1.014718in}{1.433142in}}%
\pgfpathlineto{\pgfqpoint{0.999024in}{1.474185in}}%
\pgfpathlineto{\pgfqpoint{0.984506in}{1.518461in}}%
\pgfpathlineto{\pgfqpoint{0.972010in}{1.563537in}}%
\pgfpathlineto{\pgfqpoint{0.960944in}{1.611678in}}%
\pgfpathlineto{\pgfqpoint{0.951530in}{1.662824in}}%
\pgfpathlineto{\pgfqpoint{0.944286in}{1.714431in}}%
\pgfpathlineto{\pgfqpoint{0.938950in}{1.768847in}}%
\pgfpathlineto{\pgfqpoint{0.935870in}{1.823491in}}%
\pgfpathlineto{\pgfqpoint{0.935034in}{1.878240in}}%
\pgfpathlineto{\pgfqpoint{0.936466in}{1.932973in}}%
\pgfpathlineto{\pgfqpoint{0.940005in}{1.985084in}}%
\pgfpathlineto{\pgfqpoint{0.945759in}{2.036935in}}%
\pgfpathlineto{\pgfqpoint{0.953410in}{2.085938in}}%
\pgfpathlineto{\pgfqpoint{0.962764in}{2.132000in}}%
\pgfpathlineto{\pgfqpoint{0.974287in}{2.177414in}}%
\pgfpathlineto{\pgfqpoint{0.987332in}{2.219653in}}%
\pgfpathlineto{\pgfqpoint{1.001667in}{2.258654in}}%
\pgfpathlineto{\pgfqpoint{1.018051in}{2.296583in}}%
\pgfpathlineto{\pgfqpoint{1.035401in}{2.331101in}}%
\pgfpathlineto{\pgfqpoint{1.054650in}{2.364275in}}%
\pgfpathlineto{\pgfqpoint{1.074406in}{2.393984in}}%
\pgfpathlineto{\pgfqpoint{1.095771in}{2.422197in}}%
\pgfpathlineto{\pgfqpoint{1.118662in}{2.448797in}}%
\pgfpathlineto{\pgfqpoint{1.142967in}{2.473701in}}%
\pgfpathlineto{\pgfqpoint{1.168550in}{2.496867in}}%
\pgfpathlineto{\pgfqpoint{1.197085in}{2.519662in}}%
\pgfpathlineto{\pgfqpoint{1.226727in}{2.540526in}}%
\pgfpathlineto{\pgfqpoint{1.259242in}{2.560673in}}%
\pgfpathlineto{\pgfqpoint{1.294612in}{2.579881in}}%
\pgfpathlineto{\pgfqpoint{1.332792in}{2.597982in}}%
\pgfpathlineto{\pgfqpoint{1.373719in}{2.614859in}}%
\pgfpathlineto{\pgfqpoint{1.417319in}{2.630445in}}%
\pgfpathlineto{\pgfqpoint{1.465632in}{2.645312in}}%
\pgfpathlineto{\pgfqpoint{1.518640in}{2.659204in}}%
\pgfpathlineto{\pgfqpoint{1.576309in}{2.671929in}}%
\pgfpathlineto{\pgfqpoint{1.638597in}{2.683344in}}%
\pgfpathlineto{\pgfqpoint{1.705462in}{2.693343in}}%
\pgfpathlineto{\pgfqpoint{1.779027in}{2.702064in}}%
\pgfpathlineto{\pgfqpoint{1.857097in}{2.709077in}}%
\pgfpathlineto{\pgfqpoint{1.939633in}{2.714280in}}%
\pgfpathlineto{\pgfqpoint{2.026598in}{2.717513in}}%
\pgfpathlineto{\pgfqpoint{2.113605in}{2.718523in}}%
\pgfpathlineto{\pgfqpoint{2.198435in}{2.717303in}}%
\pgfpathlineto{\pgfqpoint{2.278866in}{2.713929in}}%
\pgfpathlineto{\pgfqpoint{2.352678in}{2.708598in}}%
\pgfpathlineto{\pgfqpoint{2.417657in}{2.701709in}}%
\pgfpathlineto{\pgfqpoint{2.473770in}{2.693630in}}%
\pgfpathlineto{\pgfqpoint{2.523140in}{2.684368in}}%
\pgfpathlineto{\pgfqpoint{2.565726in}{2.674202in}}%
\pgfpathlineto{\pgfqpoint{2.601510in}{2.663544in}}%
\pgfpathlineto{\pgfqpoint{2.632577in}{2.652142in}}%
\pgfpathlineto{\pgfqpoint{2.658899in}{2.640331in}}%
\pgfpathlineto{\pgfqpoint{2.682438in}{2.627436in}}%
\pgfpathlineto{\pgfqpoint{2.703062in}{2.613571in}}%
\pgfpathlineto{\pgfqpoint{2.720674in}{2.598978in}}%
\pgfpathlineto{\pgfqpoint{2.735263in}{2.584053in}}%
\pgfpathlineto{\pgfqpoint{2.748320in}{2.567377in}}%
\pgfpathlineto{\pgfqpoint{2.759553in}{2.549046in}}%
\pgfpathlineto{\pgfqpoint{2.768788in}{2.529306in}}%
\pgfpathlineto{\pgfqpoint{2.776017in}{2.508498in}}%
\pgfpathlineto{\pgfqpoint{2.781884in}{2.484540in}}%
\pgfpathlineto{\pgfqpoint{2.786102in}{2.457597in}}%
\pgfpathlineto{\pgfqpoint{2.788720in}{2.425384in}}%
\pgfpathlineto{\pgfqpoint{2.789427in}{2.388061in}}%
\pgfpathlineto{\pgfqpoint{2.787962in}{2.340801in}}%
\pgfpathlineto{\pgfqpoint{2.783672in}{2.278768in}}%
\pgfpathlineto{\pgfqpoint{2.774289in}{2.179783in}}%
\pgfpathlineto{\pgfqpoint{2.743611in}{1.868119in}}%
\pgfpathlineto{\pgfqpoint{2.730112in}{1.702060in}}%
\pgfpathlineto{\pgfqpoint{2.717287in}{1.515949in}}%
\pgfpathlineto{\pgfqpoint{2.702602in}{1.267597in}}%
\pgfpathlineto{\pgfqpoint{2.684434in}{0.964630in}}%
\pgfpathlineto{\pgfqpoint{2.675374in}{0.850600in}}%
\pgfpathlineto{\pgfqpoint{2.667030in}{0.771523in}}%
\pgfpathlineto{\pgfqpoint{2.658752in}{0.712543in}}%
\pgfpathlineto{\pgfqpoint{2.650176in}{0.666284in}}%
\pgfpathlineto{\pgfqpoint{2.640820in}{0.627931in}}%
\pgfpathlineto{\pgfqpoint{2.631145in}{0.597534in}}%
\pgfpathlineto{\pgfqpoint{2.621004in}{0.572745in}}%
\pgfpathlineto{\pgfqpoint{2.609856in}{0.551383in}}%
\pgfpathlineto{\pgfqpoint{2.598042in}{0.533534in}}%
\pgfpathlineto{\pgfqpoint{2.584496in}{0.517378in}}%
\pgfpathlineto{\pgfqpoint{2.571109in}{0.504669in}}%
\pgfpathlineto{\pgfqpoint{2.554789in}{0.492313in}}%
\pgfpathlineto{\pgfqpoint{2.537456in}{0.481914in}}%
\pgfpathlineto{\pgfqpoint{2.517374in}{0.472367in}}%
\pgfpathlineto{\pgfqpoint{2.492542in}{0.463178in}}%
\pgfpathlineto{\pgfqpoint{2.462979in}{0.454833in}}%
\pgfpathlineto{\pgfqpoint{2.428766in}{0.447542in}}%
\pgfpathlineto{\pgfqpoint{2.385671in}{0.440735in}}%
\pgfpathlineto{\pgfqpoint{2.331557in}{0.434581in}}%
\pgfpathlineto{\pgfqpoint{2.262115in}{0.429077in}}%
\pgfpathlineto{\pgfqpoint{2.170851in}{0.424236in}}%
\pgfpathlineto{\pgfqpoint{2.049086in}{0.420134in}}%
\pgfpathlineto{\pgfqpoint{1.879436in}{0.416783in}}%
\pgfpathlineto{\pgfqpoint{1.640159in}{0.414418in}}%
\pgfpathlineto{\pgfqpoint{1.322562in}{0.413569in}}%
\pgfpathlineto{\pgfqpoint{1.020194in}{0.414850in}}%
\pgfpathlineto{\pgfqpoint{0.822256in}{0.417715in}}%
\pgfpathlineto{\pgfqpoint{0.704835in}{0.421430in}}%
\pgfpathlineto{\pgfqpoint{0.630976in}{0.425829in}}%
\pgfpathlineto{\pgfqpoint{0.583316in}{0.430734in}}%
\pgfpathlineto{\pgfqpoint{0.551033in}{0.436124in}}%
\pgfpathlineto{\pgfqpoint{0.527708in}{0.442189in}}%
\pgfpathlineto{\pgfqpoint{0.511250in}{0.448625in}}%
\pgfpathlineto{\pgfqpoint{0.499549in}{0.455216in}}%
\pgfpathlineto{\pgfqpoint{0.488916in}{0.463841in}}%
\pgfpathlineto{\pgfqpoint{0.481322in}{0.472730in}}%
\pgfpathlineto{\pgfqpoint{0.474078in}{0.485127in}}%
\pgfpathlineto{\pgfqpoint{0.468753in}{0.498748in}}%
\pgfpathlineto{\pgfqpoint{0.463870in}{0.517848in}}%
\pgfpathlineto{\pgfqpoint{0.459679in}{0.544797in}}%
\pgfpathlineto{\pgfqpoint{0.456386in}{0.581938in}}%
\pgfpathlineto{\pgfqpoint{0.453731in}{0.639106in}}%
\pgfpathlineto{\pgfqpoint{0.451681in}{0.736155in}}%
\pgfpathlineto{\pgfqpoint{0.450220in}{0.927815in}}%
\pgfpathlineto{\pgfqpoint{0.449345in}{1.403252in}}%
\pgfpathlineto{\pgfqpoint{0.449543in}{2.682703in}}%
\pgfpathlineto{\pgfqpoint{0.451011in}{2.856932in}}%
\pgfpathlineto{\pgfqpoint{0.452802in}{2.879219in}}%
\pgfpathlineto{\pgfqpoint{0.455188in}{2.886108in}}%
\pgfpathlineto{\pgfqpoint{0.458626in}{2.889028in}}%
\pgfpathlineto{\pgfqpoint{0.464996in}{2.890553in}}%
\pgfpathlineto{\pgfqpoint{0.482377in}{2.891423in}}%
\pgfpathlineto{\pgfqpoint{0.565038in}{2.891729in}}%
\pgfpathlineto{\pgfqpoint{2.733843in}{2.891760in}}%
\pgfpathlineto{\pgfqpoint{4.789510in}{2.890885in}}%
\pgfpathlineto{\pgfqpoint{4.793727in}{2.889730in}}%
\pgfpathlineto{\pgfqpoint{4.795481in}{2.888307in}}%
\pgfpathlineto{\pgfqpoint{4.797106in}{2.881145in}}%
\pgfpathlineto{\pgfqpoint{4.797997in}{2.858771in}}%
\pgfpathlineto{\pgfqpoint{4.798039in}{2.856283in}}%
\pgfpathlineto{\pgfqpoint{4.798039in}{2.856283in}}%
\pgfusepath{stroke}%
\end{pgfscope}%
\begin{pgfscope}%
\pgfpathrectangle{\pgfqpoint{0.448634in}{0.402556in}}{\pgfqpoint{4.350661in}{2.489204in}} %
\pgfusepath{clip}%
\pgfsetrectcap%
\pgfsetroundjoin%
\pgfsetlinewidth{1.003750pt}%
\definecolor{currentstroke}{rgb}{1.000000,0.498039,0.054902}%
\pgfsetstrokecolor{currentstroke}%
\pgfsetdash{}{0pt}%
\pgfpathmoveto{\pgfqpoint{3.428771in}{0.402610in}}%
\pgfpathlineto{\pgfqpoint{2.806630in}{0.403760in}}%
\pgfpathlineto{\pgfqpoint{2.769690in}{0.405578in}}%
\pgfpathlineto{\pgfqpoint{2.754631in}{0.408064in}}%
\pgfpathlineto{\pgfqpoint{2.746390in}{0.411197in}}%
\pgfpathlineto{\pgfqpoint{2.740941in}{0.415265in}}%
\pgfpathlineto{\pgfqpoint{2.736783in}{0.420984in}}%
\pgfpathlineto{\pgfqpoint{2.733280in}{0.430071in}}%
\pgfpathlineto{\pgfqpoint{2.730448in}{0.444636in}}%
\pgfpathlineto{\pgfqpoint{2.728237in}{0.469392in}}%
\pgfpathlineto{\pgfqpoint{2.726470in}{0.519131in}}%
\pgfpathlineto{\pgfqpoint{2.725711in}{0.613715in}}%
\pgfpathlineto{\pgfqpoint{2.726842in}{0.768038in}}%
\pgfpathlineto{\pgfqpoint{2.730556in}{0.962148in}}%
\pgfpathlineto{\pgfqpoint{2.736611in}{1.158670in}}%
\pgfpathlineto{\pgfqpoint{2.744092in}{1.327718in}}%
\pgfpathlineto{\pgfqpoint{2.753201in}{1.484189in}}%
\pgfpathlineto{\pgfqpoint{2.763256in}{1.620609in}}%
\pgfpathlineto{\pgfqpoint{2.776118in}{1.764216in}}%
\pgfpathlineto{\pgfqpoint{2.788914in}{1.877776in}}%
\pgfpathlineto{\pgfqpoint{2.805748in}{2.005740in}}%
\pgfpathlineto{\pgfqpoint{2.821176in}{2.101198in}}%
\pgfpathlineto{\pgfqpoint{2.838359in}{2.193718in}}%
\pgfpathlineto{\pgfqpoint{2.859135in}{2.292966in}}%
\pgfpathlineto{\pgfqpoint{2.887209in}{2.425960in}}%
\pgfpathlineto{\pgfqpoint{2.896991in}{2.479560in}}%
\pgfpathlineto{\pgfqpoint{2.901542in}{2.516523in}}%
\pgfpathlineto{\pgfqpoint{2.902849in}{2.543854in}}%
\pgfpathlineto{\pgfqpoint{2.901957in}{2.566223in}}%
\pgfpathlineto{\pgfqpoint{2.899151in}{2.585863in}}%
\pgfpathlineto{\pgfqpoint{2.894793in}{2.602546in}}%
\pgfpathlineto{\pgfqpoint{2.888483in}{2.618388in}}%
\pgfpathlineto{\pgfqpoint{2.880257in}{2.633033in}}%
\pgfpathlineto{\pgfqpoint{2.870347in}{2.646246in}}%
\pgfpathlineto{\pgfqpoint{2.857399in}{2.659530in}}%
\pgfpathlineto{\pgfqpoint{2.843188in}{2.671010in}}%
\pgfpathlineto{\pgfqpoint{2.824237in}{2.683209in}}%
\pgfpathlineto{\pgfqpoint{2.802412in}{2.694418in}}%
\pgfpathlineto{\pgfqpoint{2.775808in}{2.705369in}}%
\pgfpathlineto{\pgfqpoint{2.744461in}{2.715715in}}%
\pgfpathlineto{\pgfqpoint{2.708435in}{2.725252in}}%
\pgfpathlineto{\pgfqpoint{2.665654in}{2.734289in}}%
\pgfpathlineto{\pgfqpoint{2.613991in}{2.742869in}}%
\pgfpathlineto{\pgfqpoint{2.553458in}{2.750588in}}%
\pgfpathlineto{\pgfqpoint{2.481919in}{2.757365in}}%
\pgfpathlineto{\pgfqpoint{2.399397in}{2.762839in}}%
\pgfpathlineto{\pgfqpoint{2.310268in}{2.766482in}}%
\pgfpathlineto{\pgfqpoint{2.175415in}{2.768725in}}%
\pgfpathlineto{\pgfqpoint{2.066652in}{2.767941in}}%
\pgfpathlineto{\pgfqpoint{1.953570in}{2.764859in}}%
\pgfpathlineto{\pgfqpoint{1.851428in}{2.759758in}}%
\pgfpathlineto{\pgfqpoint{1.745050in}{2.752168in}}%
\pgfpathlineto{\pgfqpoint{1.658373in}{2.743453in}}%
\pgfpathlineto{\pgfqpoint{1.580551in}{2.733461in}}%
\pgfpathlineto{\pgfqpoint{1.490057in}{2.719337in}}%
\pgfpathlineto{\pgfqpoint{1.417231in}{2.704698in}}%
\pgfpathlineto{\pgfqpoint{1.361991in}{2.690818in}}%
\pgfpathlineto{\pgfqpoint{1.311459in}{2.675818in}}%
\pgfpathlineto{\pgfqpoint{1.265666in}{2.659923in}}%
\pgfpathlineto{\pgfqpoint{1.222574in}{2.642586in}}%
\pgfpathlineto{\pgfqpoint{1.184323in}{2.624682in}}%
\pgfpathlineto{\pgfqpoint{1.148891in}{2.605623in}}%
\pgfpathlineto{\pgfqpoint{1.116331in}{2.585573in}}%
\pgfpathlineto{\pgfqpoint{1.092327in}{2.568511in}}%
\pgfpathlineto{\pgfqpoint{1.079760in}{2.558685in}}%
\pgfpathlineto{\pgfqpoint{1.051543in}{2.535378in}}%
\pgfpathlineto{\pgfqpoint{1.026312in}{2.511712in}}%
\pgfpathlineto{\pgfqpoint{1.002398in}{2.486317in}}%
\pgfpathlineto{\pgfqpoint{0.979912in}{2.459268in}}%
\pgfpathlineto{\pgfqpoint{0.958934in}{2.430677in}}%
\pgfpathlineto{\pgfqpoint{0.938264in}{2.398643in}}%
\pgfpathlineto{\pgfqpoint{0.923047in}{2.371384in}}%
\pgfpathlineto{\pgfqpoint{0.904513in}{2.334773in}}%
\pgfpathlineto{\pgfqpoint{0.887853in}{2.297000in}}%
\pgfpathlineto{\pgfqpoint{0.872131in}{2.255971in}}%
\pgfpathlineto{\pgfqpoint{0.857507in}{2.211740in}}%
\pgfpathlineto{\pgfqpoint{0.844761in}{2.166756in}}%
\pgfpathlineto{\pgfqpoint{0.838623in}{2.140305in}}%
\pgfpathlineto{\pgfqpoint{0.826982in}{2.087193in}}%
\pgfpathlineto{\pgfqpoint{0.816322in}{2.028715in}}%
\pgfpathlineto{\pgfqpoint{0.810087in}{1.984494in}}%
\pgfpathlineto{\pgfqpoint{0.808026in}{1.967237in}}%
\pgfpathlineto{\pgfqpoint{0.800076in}{1.898140in}}%
\pgfpathlineto{\pgfqpoint{0.793713in}{1.823822in}}%
\pgfpathlineto{\pgfqpoint{0.788799in}{1.741874in}}%
\pgfpathlineto{\pgfqpoint{0.786200in}{1.677224in}}%
\pgfpathlineto{\pgfqpoint{0.776951in}{1.453480in}}%
\pgfpathlineto{\pgfqpoint{0.773280in}{1.418893in}}%
\pgfpathlineto{\pgfqpoint{0.768298in}{1.389581in}}%
\pgfpathlineto{\pgfqpoint{0.762752in}{1.368107in}}%
\pgfpathlineto{\pgfqpoint{0.756722in}{1.352122in}}%
\pgfpathlineto{\pgfqpoint{0.749752in}{1.339518in}}%
\pgfpathlineto{\pgfqpoint{0.742201in}{1.330599in}}%
\pgfpathlineto{\pgfqpoint{0.734854in}{1.325311in}}%
\pgfpathlineto{\pgfqpoint{0.726557in}{1.322418in}}%
\pgfpathlineto{\pgfqpoint{0.717883in}{1.322223in}}%
\pgfpathlineto{\pgfqpoint{0.709412in}{1.324411in}}%
\pgfpathlineto{\pgfqpoint{0.699548in}{1.329604in}}%
\pgfpathlineto{\pgfqpoint{0.688894in}{1.338202in}}%
\pgfpathlineto{\pgfqpoint{0.677907in}{1.350248in}}%
\pgfpathlineto{\pgfqpoint{0.666886in}{1.365646in}}%
\pgfpathlineto{\pgfqpoint{0.654912in}{1.386417in}}%
\pgfpathlineto{\pgfqpoint{0.642574in}{1.412730in}}%
\pgfpathlineto{\pgfqpoint{0.630328in}{1.444629in}}%
\pgfpathlineto{\pgfqpoint{0.618504in}{1.482080in}}%
\pgfpathlineto{\pgfqpoint{0.608613in}{1.520256in}}%
\pgfpathlineto{\pgfqpoint{0.590203in}{1.612445in}}%
\pgfpathlineto{\pgfqpoint{0.581848in}{1.668884in}}%
\pgfpathlineto{\pgfqpoint{0.573137in}{1.740376in}}%
\pgfpathlineto{\pgfqpoint{0.567062in}{1.807213in}}%
\pgfpathlineto{\pgfqpoint{0.560532in}{1.896510in}}%
\pgfpathlineto{\pgfqpoint{0.555526in}{1.995910in}}%
\pgfpathlineto{\pgfqpoint{0.552564in}{2.097908in}}%
\pgfpathlineto{\pgfqpoint{0.551526in}{2.204935in}}%
\pgfpathlineto{\pgfqpoint{0.552727in}{2.309470in}}%
\pgfpathlineto{\pgfqpoint{0.556011in}{2.403981in}}%
\pgfpathlineto{\pgfqpoint{0.560953in}{2.483430in}}%
\pgfpathlineto{\pgfqpoint{0.567303in}{2.550240in}}%
\pgfpathlineto{\pgfqpoint{0.574928in}{2.606817in}}%
\pgfpathlineto{\pgfqpoint{0.582988in}{2.650657in}}%
\pgfpathlineto{\pgfqpoint{0.592756in}{2.691452in}}%
\pgfpathlineto{\pgfqpoint{0.602650in}{2.721756in}}%
\pgfpathlineto{\pgfqpoint{0.612983in}{2.746441in}}%
\pgfpathlineto{\pgfqpoint{0.624292in}{2.767692in}}%
\pgfpathlineto{\pgfqpoint{0.636231in}{2.785433in}}%
\pgfpathlineto{\pgfqpoint{0.649892in}{2.801461in}}%
\pgfpathlineto{\pgfqpoint{0.663386in}{2.814020in}}%
\pgfpathlineto{\pgfqpoint{0.679842in}{2.826135in}}%
\pgfpathlineto{\pgfqpoint{0.697326in}{2.836197in}}%
\pgfpathlineto{\pgfqpoint{0.715574in}{2.844285in}}%
\pgfpathlineto{\pgfqpoint{0.738439in}{2.852335in}}%
\pgfpathlineto{\pgfqpoint{0.765983in}{2.859639in}}%
\pgfpathlineto{\pgfqpoint{0.800300in}{2.866256in}}%
\pgfpathlineto{\pgfqpoint{0.841340in}{2.871832in}}%
\pgfpathlineto{\pgfqpoint{0.895547in}{2.876803in}}%
\pgfpathlineto{\pgfqpoint{0.969413in}{2.881069in}}%
\pgfpathlineto{\pgfqpoint{1.071608in}{2.884501in}}%
\pgfpathlineto{\pgfqpoint{1.219512in}{2.887074in}}%
\pgfpathlineto{\pgfqpoint{1.471844in}{2.889091in}}%
\pgfpathlineto{\pgfqpoint{1.956941in}{2.890384in}}%
\pgfpathlineto{\pgfqpoint{3.096814in}{2.890781in}}%
\pgfpathlineto{\pgfqpoint{3.995224in}{2.889388in}}%
\pgfpathlineto{\pgfqpoint{4.275833in}{2.887011in}}%
\pgfpathlineto{\pgfqpoint{4.412847in}{2.883743in}}%
\pgfpathlineto{\pgfqpoint{4.491081in}{2.879810in}}%
\pgfpathlineto{\pgfqpoint{4.543127in}{2.875163in}}%
\pgfpathlineto{\pgfqpoint{4.579810in}{2.869841in}}%
\pgfpathlineto{\pgfqpoint{4.607580in}{2.863763in}}%
\pgfpathlineto{\pgfqpoint{4.630623in}{2.856424in}}%
\pgfpathlineto{\pgfqpoint{4.648833in}{2.848228in}}%
\pgfpathlineto{\pgfqpoint{4.664136in}{2.838773in}}%
\pgfpathlineto{\pgfqpoint{4.676470in}{2.828576in}}%
\pgfpathlineto{\pgfqpoint{4.687502in}{2.816585in}}%
\pgfpathlineto{\pgfqpoint{4.697051in}{2.803027in}}%
\pgfpathlineto{\pgfqpoint{4.706194in}{2.786098in}}%
\pgfpathlineto{\pgfqpoint{4.714508in}{2.765827in}}%
\pgfpathlineto{\pgfqpoint{4.722462in}{2.740013in}}%
\pgfpathlineto{\pgfqpoint{4.729577in}{2.708703in}}%
\pgfpathlineto{\pgfqpoint{4.736162in}{2.669601in}}%
\pgfpathlineto{\pgfqpoint{4.742419in}{2.617826in}}%
\pgfpathlineto{\pgfqpoint{4.747859in}{2.553410in}}%
\pgfpathlineto{\pgfqpoint{4.752661in}{2.468958in}}%
\pgfpathlineto{\pgfqpoint{4.756610in}{2.359528in}}%
\pgfpathlineto{\pgfqpoint{4.759416in}{2.217681in}}%
\pgfpathlineto{\pgfqpoint{4.760596in}{2.043444in}}%
\pgfpathlineto{\pgfqpoint{4.759662in}{1.851779in}}%
\pgfpathlineto{\pgfqpoint{4.756587in}{1.667613in}}%
\pgfpathlineto{\pgfqpoint{4.751596in}{1.503428in}}%
\pgfpathlineto{\pgfqpoint{4.745410in}{1.374185in}}%
\pgfpathlineto{\pgfqpoint{4.738113in}{1.267479in}}%
\pgfpathlineto{\pgfqpoint{4.729621in}{1.175896in}}%
\pgfpathlineto{\pgfqpoint{4.720762in}{1.104428in}}%
\pgfpathlineto{\pgfqpoint{4.711045in}{1.043204in}}%
\pgfpathlineto{\pgfqpoint{4.700364in}{0.989829in}}%
\pgfpathlineto{\pgfqpoint{4.689055in}{0.944345in}}%
\pgfpathlineto{\pgfqpoint{4.676881in}{0.904394in}}%
\pgfpathlineto{\pgfqpoint{4.676095in}{0.902073in}}%
\pgfpathlineto{\pgfqpoint{4.676095in}{0.902073in}}%
\pgfusepath{stroke}%
\end{pgfscope}%
\begin{pgfscope}%
\pgfpathrectangle{\pgfqpoint{0.448634in}{0.402556in}}{\pgfqpoint{4.350661in}{2.489204in}} %
\pgfusepath{clip}%
\pgfsetrectcap%
\pgfsetroundjoin%
\pgfsetlinewidth{1.003750pt}%
\definecolor{currentstroke}{rgb}{1.000000,0.498039,0.054902}%
\pgfsetstrokecolor{currentstroke}%
\pgfsetdash{}{0pt}%
\pgfpathmoveto{\pgfqpoint{2.795520in}{1.982745in}}%
\pgfpathlineto{\pgfqpoint{2.781780in}{1.874357in}}%
\pgfpathlineto{\pgfqpoint{2.769351in}{1.758234in}}%
\pgfpathlineto{\pgfqpoint{2.758095in}{1.631942in}}%
\pgfpathlineto{\pgfqpoint{2.747786in}{1.490551in}}%
\pgfpathlineto{\pgfqpoint{2.738644in}{1.334082in}}%
\pgfpathlineto{\pgfqpoint{2.730580in}{1.157591in}}%
\pgfpathlineto{\pgfqpoint{2.723334in}{0.948663in}}%
\pgfpathlineto{\pgfqpoint{2.709783in}{0.530788in}}%
\pgfpathlineto{\pgfqpoint{2.705868in}{0.488716in}}%
\pgfpathlineto{\pgfqpoint{2.701769in}{0.464281in}}%
\pgfpathlineto{\pgfqpoint{2.697021in}{0.447744in}}%
\pgfpathlineto{\pgfqpoint{2.691859in}{0.436812in}}%
\pgfpathlineto{\pgfqpoint{2.686245in}{0.429229in}}%
\pgfpathlineto{\pgfqpoint{2.679348in}{0.423188in}}%
\pgfpathlineto{\pgfqpoint{2.669540in}{0.417856in}}%
\pgfpathlineto{\pgfqpoint{2.656987in}{0.413810in}}%
\pgfpathlineto{\pgfqpoint{2.637654in}{0.410337in}}%
\pgfpathlineto{\pgfqpoint{2.607297in}{0.407617in}}%
\pgfpathlineto{\pgfqpoint{2.555121in}{0.405574in}}%
\pgfpathlineto{\pgfqpoint{2.450714in}{0.404139in}}%
\pgfpathlineto{\pgfqpoint{2.176624in}{0.403275in}}%
\pgfpathlineto{\pgfqpoint{1.130290in}{0.402953in}}%
\pgfpathlineto{\pgfqpoint{0.516849in}{0.404175in}}%
\pgfpathlineto{\pgfqpoint{0.466848in}{0.405970in}}%
\pgfpathlineto{\pgfqpoint{0.456130in}{0.407931in}}%
\pgfpathlineto{\pgfqpoint{0.452340in}{0.410303in}}%
\pgfpathlineto{\pgfqpoint{0.450346in}{0.414662in}}%
\pgfpathlineto{\pgfqpoint{0.449266in}{0.424524in}}%
\pgfpathlineto{\pgfqpoint{0.448771in}{0.464344in}}%
\pgfpathlineto{\pgfqpoint{0.448640in}{0.850171in}}%
\pgfpathlineto{\pgfqpoint{0.448679in}{2.891318in}}%
\pgfpathlineto{\pgfqpoint{0.448679in}{2.891318in}}%
\pgfusepath{stroke}%
\end{pgfscope}%
\begin{pgfscope}%
\pgfpathrectangle{\pgfqpoint{0.448634in}{0.402556in}}{\pgfqpoint{4.350661in}{2.489204in}} %
\pgfusepath{clip}%
\pgfsetrectcap%
\pgfsetroundjoin%
\pgfsetlinewidth{1.003750pt}%
\definecolor{currentstroke}{rgb}{1.000000,0.498039,0.054902}%
\pgfsetstrokecolor{currentstroke}%
\pgfsetdash{}{0pt}%
\pgfpathmoveto{\pgfqpoint{3.428188in}{0.402586in}}%
\pgfpathlineto{\pgfqpoint{2.782120in}{0.403701in}}%
\pgfpathlineto{\pgfqpoint{2.753905in}{0.405673in}}%
\pgfpathlineto{\pgfqpoint{2.743327in}{0.408443in}}%
\pgfpathlineto{\pgfqpoint{2.737716in}{0.412187in}}%
\pgfpathlineto{\pgfqpoint{2.733667in}{0.417994in}}%
\pgfpathlineto{\pgfqpoint{2.730648in}{0.427307in}}%
\pgfpathlineto{\pgfqpoint{2.728387in}{0.442004in}}%
\pgfpathlineto{\pgfqpoint{2.726543in}{0.471794in}}%
\pgfpathlineto{\pgfqpoint{2.725216in}{0.534002in}}%
\pgfpathlineto{\pgfqpoint{2.725169in}{0.655972in}}%
\pgfpathlineto{\pgfqpoint{2.727377in}{0.832686in}}%
\pgfpathlineto{\pgfqpoint{2.732259in}{1.041702in}}%
\pgfpathlineto{\pgfqpoint{2.738851in}{1.223256in}}%
\pgfpathlineto{\pgfqpoint{2.747078in}{1.389765in}}%
\pgfpathlineto{\pgfqpoint{2.756608in}{1.538717in}}%
\pgfpathlineto{\pgfqpoint{2.768954in}{1.694886in}}%
\pgfpathlineto{\pgfqpoint{2.781227in}{1.816044in}}%
\pgfpathlineto{\pgfqpoint{2.794401in}{1.924524in}}%
\pgfpathlineto{\pgfqpoint{2.812737in}{2.054722in}}%
\pgfpathlineto{\pgfqpoint{2.828774in}{2.147511in}}%
\pgfpathlineto{\pgfqpoint{2.847382in}{2.242224in}}%
\pgfpathlineto{\pgfqpoint{2.895817in}{2.479699in}}%
\pgfpathlineto{\pgfqpoint{2.900203in}{2.516688in}}%
\pgfpathlineto{\pgfqpoint{2.901345in}{2.544029in}}%
\pgfpathlineto{\pgfqpoint{2.900291in}{2.566388in}}%
\pgfpathlineto{\pgfqpoint{2.897334in}{2.585999in}}%
\pgfpathlineto{\pgfqpoint{2.892836in}{2.602633in}}%
\pgfpathlineto{\pgfqpoint{2.886394in}{2.618405in}}%
\pgfpathlineto{\pgfqpoint{2.878058in}{2.632969in}}%
\pgfpathlineto{\pgfqpoint{2.868065in}{2.646100in}}%
\pgfpathlineto{\pgfqpoint{2.855050in}{2.659300in}}%
\pgfpathlineto{\pgfqpoint{2.840801in}{2.670716in}}%
\pgfpathlineto{\pgfqpoint{2.821822in}{2.682860in}}%
\pgfpathlineto{\pgfqpoint{2.799980in}{2.694025in}}%
\pgfpathlineto{\pgfqpoint{2.773366in}{2.704944in}}%
\pgfpathlineto{\pgfqpoint{2.742012in}{2.715266in}}%
\pgfpathlineto{\pgfqpoint{2.705983in}{2.724785in}}%
\pgfpathlineto{\pgfqpoint{2.663200in}{2.733810in}}%
\pgfpathlineto{\pgfqpoint{2.611535in}{2.742379in}}%
\pgfpathlineto{\pgfqpoint{2.551002in}{2.750090in}}%
\pgfpathlineto{\pgfqpoint{2.481632in}{2.756682in}}%
\pgfpathlineto{\pgfqpoint{2.399112in}{2.762200in}}%
\pgfpathlineto{\pgfqpoint{2.309985in}{2.765885in}}%
\pgfpathlineto{\pgfqpoint{2.188184in}{2.768096in}}%
\pgfpathlineto{\pgfqpoint{2.081595in}{2.767619in}}%
\pgfpathlineto{\pgfqpoint{1.968506in}{2.764840in}}%
\pgfpathlineto{\pgfqpoint{1.864180in}{2.759918in}}%
\pgfpathlineto{\pgfqpoint{1.757786in}{2.752593in}}%
\pgfpathlineto{\pgfqpoint{1.671087in}{2.744171in}}%
\pgfpathlineto{\pgfqpoint{1.591076in}{2.734193in}}%
\pgfpathlineto{\pgfqpoint{1.502689in}{2.720717in}}%
\pgfpathlineto{\pgfqpoint{1.427655in}{2.706083in}}%
\pgfpathlineto{\pgfqpoint{1.372350in}{2.692544in}}%
\pgfpathlineto{\pgfqpoint{1.321734in}{2.677921in}}%
\pgfpathlineto{\pgfqpoint{1.273765in}{2.661664in}}%
\pgfpathlineto{\pgfqpoint{1.230567in}{2.644672in}}%
\pgfpathlineto{\pgfqpoint{1.192197in}{2.627106in}}%
\pgfpathlineto{\pgfqpoint{1.156620in}{2.608403in}}%
\pgfpathlineto{\pgfqpoint{1.123890in}{2.588716in}}%
\pgfpathlineto{\pgfqpoint{1.095883in}{2.569568in}}%
\pgfpathlineto{\pgfqpoint{1.063936in}{2.543701in}}%
\pgfpathlineto{\pgfqpoint{1.038217in}{2.520732in}}%
\pgfpathlineto{\pgfqpoint{1.013766in}{2.496016in}}%
\pgfpathlineto{\pgfqpoint{0.990704in}{2.469610in}}%
\pgfpathlineto{\pgfqpoint{0.969124in}{2.441612in}}%
\pgfpathlineto{\pgfqpoint{0.949083in}{2.412154in}}%
\pgfpathlineto{\pgfqpoint{0.930604in}{2.381387in}}%
\pgfpathlineto{\pgfqpoint{0.906555in}{2.334052in}}%
\pgfpathlineto{\pgfqpoint{0.889925in}{2.296262in}}%
\pgfpathlineto{\pgfqpoint{0.874241in}{2.255213in}}%
\pgfpathlineto{\pgfqpoint{0.859667in}{2.210961in}}%
\pgfpathlineto{\pgfqpoint{0.846986in}{2.165954in}}%
\pgfpathlineto{\pgfqpoint{0.839633in}{2.134715in}}%
\pgfpathlineto{\pgfqpoint{0.828238in}{2.081532in}}%
\pgfpathlineto{\pgfqpoint{0.817866in}{2.022986in}}%
\pgfpathlineto{\pgfqpoint{0.810784in}{1.971352in}}%
\pgfpathlineto{\pgfqpoint{0.802846in}{1.902252in}}%
\pgfpathlineto{\pgfqpoint{0.796554in}{1.827927in}}%
\pgfpathlineto{\pgfqpoint{0.791696in}{1.743480in}}%
\pgfpathlineto{\pgfqpoint{0.787773in}{1.621595in}}%
\pgfpathlineto{\pgfqpoint{0.785408in}{1.522064in}}%
\pgfpathlineto{\pgfqpoint{0.785408in}{1.522064in}}%
\pgfusepath{stroke}%
\end{pgfscope}%
\begin{pgfscope}%
\pgfpathrectangle{\pgfqpoint{0.448634in}{0.402556in}}{\pgfqpoint{4.350661in}{2.489204in}} %
\pgfusepath{clip}%
\pgfsetrectcap%
\pgfsetroundjoin%
\pgfsetlinewidth{1.003750pt}%
\definecolor{currentstroke}{rgb}{0.172549,0.627451,0.172549}%
\pgfsetstrokecolor{currentstroke}%
\pgfsetdash{}{0pt}%
\pgfpathmoveto{\pgfqpoint{1.127319in}{2.572074in}}%
\pgfpathlineto{\pgfqpoint{1.159575in}{2.592758in}}%
\pgfpathlineto{\pgfqpoint{1.192763in}{2.611414in}}%
\pgfpathlineto{\pgfqpoint{1.228726in}{2.629126in}}%
\pgfpathlineto{\pgfqpoint{1.267413in}{2.645758in}}%
\pgfpathlineto{\pgfqpoint{1.310846in}{2.661945in}}%
\pgfpathlineto{\pgfqpoint{1.356920in}{2.676740in}}%
\pgfpathlineto{\pgfqpoint{1.407680in}{2.690702in}}%
\pgfpathlineto{\pgfqpoint{1.463094in}{2.703640in}}%
\pgfpathlineto{\pgfqpoint{1.525273in}{2.715813in}}%
\pgfpathlineto{\pgfqpoint{1.594199in}{2.726937in}}%
\pgfpathlineto{\pgfqpoint{1.669843in}{2.736808in}}%
\pgfpathlineto{\pgfqpoint{1.752172in}{2.745271in}}%
\pgfpathlineto{\pgfqpoint{1.843325in}{2.752344in}}%
\pgfpathlineto{\pgfqpoint{1.941103in}{2.757656in}}%
\pgfpathlineto{\pgfqpoint{2.043301in}{2.760987in}}%
\pgfpathlineto{\pgfqpoint{2.147710in}{2.762199in}}%
\pgfpathlineto{\pgfqpoint{2.249945in}{2.761215in}}%
\pgfpathlineto{\pgfqpoint{2.345620in}{2.758145in}}%
\pgfpathlineto{\pgfqpoint{2.432525in}{2.753210in}}%
\pgfpathlineto{\pgfqpoint{2.508451in}{2.746766in}}%
\pgfpathlineto{\pgfqpoint{2.573368in}{2.739156in}}%
\pgfpathlineto{\pgfqpoint{2.629410in}{2.730451in}}%
\pgfpathlineto{\pgfqpoint{2.676543in}{2.720985in}}%
\pgfpathlineto{\pgfqpoint{2.716874in}{2.710666in}}%
\pgfpathlineto{\pgfqpoint{2.750366in}{2.699848in}}%
\pgfpathlineto{\pgfqpoint{2.779059in}{2.688192in}}%
\pgfpathlineto{\pgfqpoint{2.802882in}{2.676004in}}%
\pgfpathlineto{\pgfqpoint{2.821842in}{2.663819in}}%
\pgfpathlineto{\pgfqpoint{2.837815in}{2.650886in}}%
\pgfpathlineto{\pgfqpoint{2.850736in}{2.637564in}}%
\pgfpathlineto{\pgfqpoint{2.860694in}{2.624398in}}%
\pgfpathlineto{\pgfqpoint{2.869084in}{2.609873in}}%
\pgfpathlineto{\pgfqpoint{2.875698in}{2.594192in}}%
\pgfpathlineto{\pgfqpoint{2.881035in}{2.575255in}}%
\pgfpathlineto{\pgfqpoint{2.884200in}{2.555685in}}%
\pgfpathlineto{\pgfqpoint{2.885619in}{2.533351in}}%
\pgfpathlineto{\pgfqpoint{2.885038in}{2.505987in}}%
\pgfpathlineto{\pgfqpoint{2.882112in}{2.473807in}}%
\pgfpathlineto{\pgfqpoint{2.875657in}{2.429620in}}%
\pgfpathlineto{\pgfqpoint{2.863489in}{2.363873in}}%
\pgfpathlineto{\pgfqpoint{2.821102in}{2.142619in}}%
\pgfpathlineto{\pgfqpoint{2.804859in}{2.042271in}}%
\pgfpathlineto{\pgfqpoint{2.790421in}{1.939040in}}%
\pgfpathlineto{\pgfqpoint{2.777207in}{1.828054in}}%
\pgfpathlineto{\pgfqpoint{2.765338in}{1.709349in}}%
\pgfpathlineto{\pgfqpoint{2.754471in}{1.578010in}}%
\pgfpathlineto{\pgfqpoint{2.744640in}{1.431580in}}%
\pgfpathlineto{\pgfqpoint{2.735914in}{1.267598in}}%
\pgfpathlineto{\pgfqpoint{2.728277in}{1.081114in}}%
\pgfpathlineto{\pgfqpoint{2.721437in}{0.857223in}}%
\pgfpathlineto{\pgfqpoint{2.711961in}{0.541290in}}%
\pgfpathlineto{\pgfqpoint{2.708250in}{0.491694in}}%
\pgfpathlineto{\pgfqpoint{2.703951in}{0.462246in}}%
\pgfpathlineto{\pgfqpoint{2.699504in}{0.445599in}}%
\pgfpathlineto{\pgfqpoint{2.694517in}{0.434563in}}%
\pgfpathlineto{\pgfqpoint{2.688942in}{0.426947in}}%
\pgfpathlineto{\pgfqpoint{2.681980in}{0.421009in}}%
\pgfpathlineto{\pgfqpoint{2.672064in}{0.415948in}}%
\pgfpathlineto{\pgfqpoint{2.659429in}{0.412247in}}%
\pgfpathlineto{\pgfqpoint{2.640044in}{0.409163in}}%
\pgfpathlineto{\pgfqpoint{2.607490in}{0.406692in}}%
\pgfpathlineto{\pgfqpoint{2.548779in}{0.404894in}}%
\pgfpathlineto{\pgfqpoint{2.422615in}{0.403701in}}%
\pgfpathlineto{\pgfqpoint{2.026705in}{0.403016in}}%
\pgfpathlineto{\pgfqpoint{0.623617in}{0.403253in}}%
\pgfpathlineto{\pgfqpoint{0.477880in}{0.404742in}}%
\pgfpathlineto{\pgfqpoint{0.458368in}{0.406382in}}%
\pgfpathlineto{\pgfqpoint{0.452304in}{0.408937in}}%
\pgfpathlineto{\pgfqpoint{0.450213in}{0.413215in}}%
\pgfpathlineto{\pgfqpoint{0.449165in}{0.423080in}}%
\pgfpathlineto{\pgfqpoint{0.448735in}{0.465392in}}%
\pgfpathlineto{\pgfqpoint{0.448637in}{0.983146in}}%
\pgfpathlineto{\pgfqpoint{0.448652in}{2.889876in}}%
\pgfpathlineto{\pgfqpoint{0.448652in}{2.889876in}}%
\pgfusepath{stroke}%
\end{pgfscope}%
\begin{pgfscope}%
\pgfpathrectangle{\pgfqpoint{0.448634in}{0.402556in}}{\pgfqpoint{4.350661in}{2.489204in}} %
\pgfusepath{clip}%
\pgfsetrectcap%
\pgfsetroundjoin%
\pgfsetlinewidth{1.003750pt}%
\definecolor{currentstroke}{rgb}{0.172549,0.627451,0.172549}%
\pgfsetstrokecolor{currentstroke}%
\pgfsetdash{}{0pt}%
\pgfpathmoveto{\pgfqpoint{0.448634in}{2.896245in}}%
\pgfpathlineto{\pgfqpoint{0.448593in}{0.407043in}}%
\pgfpathlineto{\pgfqpoint{0.448593in}{0.407043in}}%
\pgfusepath{stroke}%
\end{pgfscope}%
\begin{pgfscope}%
\pgfpathrectangle{\pgfqpoint{0.448634in}{0.402556in}}{\pgfqpoint{4.350661in}{2.489204in}} %
\pgfusepath{clip}%
\pgfsetrectcap%
\pgfsetroundjoin%
\pgfsetlinewidth{1.003750pt}%
\definecolor{currentstroke}{rgb}{0.172549,0.627451,0.172549}%
\pgfsetstrokecolor{currentstroke}%
\pgfsetdash{}{0pt}%
\pgfpathmoveto{\pgfqpoint{0.576854in}{1.760814in}}%
\pgfpathlineto{\pgfqpoint{0.569395in}{1.840007in}}%
\pgfpathlineto{\pgfqpoint{0.563209in}{1.929336in}}%
\pgfpathlineto{\pgfqpoint{0.558593in}{2.028761in}}%
\pgfpathlineto{\pgfqpoint{0.555986in}{2.133263in}}%
\pgfpathlineto{\pgfqpoint{0.555566in}{2.237805in}}%
\pgfpathlineto{\pgfqpoint{0.557372in}{2.337349in}}%
\pgfpathlineto{\pgfqpoint{0.561096in}{2.424364in}}%
\pgfpathlineto{\pgfqpoint{0.566403in}{2.498789in}}%
\pgfpathlineto{\pgfqpoint{0.572909in}{2.560567in}}%
\pgfpathlineto{\pgfqpoint{0.580458in}{2.612116in}}%
\pgfpathlineto{\pgfqpoint{0.589086in}{2.655813in}}%
\pgfpathlineto{\pgfqpoint{0.598406in}{2.691587in}}%
\pgfpathlineto{\pgfqpoint{0.608613in}{2.721755in}}%
\pgfpathlineto{\pgfqpoint{0.619240in}{2.746276in}}%
\pgfpathlineto{\pgfqpoint{0.630816in}{2.767337in}}%
\pgfpathlineto{\pgfqpoint{0.642974in}{2.784882in}}%
\pgfpathlineto{\pgfqpoint{0.656812in}{2.800711in}}%
\pgfpathlineto{\pgfqpoint{0.672195in}{2.814547in}}%
\pgfpathlineto{\pgfqpoint{0.688852in}{2.826300in}}%
\pgfpathlineto{\pgfqpoint{0.706460in}{2.836075in}}%
\pgfpathlineto{\pgfqpoint{0.726802in}{2.844874in}}%
\pgfpathlineto{\pgfqpoint{0.751865in}{2.853202in}}%
\pgfpathlineto{\pgfqpoint{0.781630in}{2.860547in}}%
\pgfpathlineto{\pgfqpoint{0.818167in}{2.867053in}}%
\pgfpathlineto{\pgfqpoint{0.863580in}{2.872684in}}%
\pgfpathlineto{\pgfqpoint{0.922159in}{2.877517in}}%
\pgfpathlineto{\pgfqpoint{1.000389in}{2.881567in}}%
\pgfpathlineto{\pgfqpoint{1.111292in}{2.884881in}}%
\pgfpathlineto{\pgfqpoint{1.274427in}{2.887367in}}%
\pgfpathlineto{\pgfqpoint{1.552864in}{2.889263in}}%
\pgfpathlineto{\pgfqpoint{2.107572in}{2.890457in}}%
\pgfpathlineto{\pgfqpoint{3.343159in}{2.890573in}}%
\pgfpathlineto{\pgfqpoint{4.043614in}{2.888941in}}%
\pgfpathlineto{\pgfqpoint{4.289415in}{2.886404in}}%
\pgfpathlineto{\pgfqpoint{4.413373in}{2.883093in}}%
\pgfpathlineto{\pgfqpoint{4.489423in}{2.878997in}}%
\pgfpathlineto{\pgfqpoint{4.541450in}{2.874081in}}%
\pgfpathlineto{\pgfqpoint{4.578099in}{2.868470in}}%
\pgfpathlineto{\pgfqpoint{4.605817in}{2.862093in}}%
\pgfpathlineto{\pgfqpoint{4.626724in}{2.855245in}}%
\pgfpathlineto{\pgfqpoint{4.644924in}{2.847019in}}%
\pgfpathlineto{\pgfqpoint{4.660240in}{2.837590in}}%
\pgfpathlineto{\pgfqpoint{4.672622in}{2.827469in}}%
\pgfpathlineto{\pgfqpoint{4.683749in}{2.815593in}}%
\pgfpathlineto{\pgfqpoint{4.693405in}{2.802136in}}%
\pgfpathlineto{\pgfqpoint{4.702739in}{2.785344in}}%
\pgfpathlineto{\pgfqpoint{4.711276in}{2.765195in}}%
\pgfpathlineto{\pgfqpoint{4.719482in}{2.739485in}}%
\pgfpathlineto{\pgfqpoint{4.726293in}{2.710659in}}%
\pgfpathlineto{\pgfqpoint{4.733259in}{2.671644in}}%
\pgfpathlineto{\pgfqpoint{4.739604in}{2.622397in}}%
\pgfpathlineto{\pgfqpoint{4.745236in}{2.560505in}}%
\pgfpathlineto{\pgfqpoint{4.750164in}{2.481053in}}%
\pgfpathlineto{\pgfqpoint{4.754367in}{2.376619in}}%
\pgfpathlineto{\pgfqpoint{4.757443in}{2.242250in}}%
\pgfpathlineto{\pgfqpoint{4.758977in}{2.075484in}}%
\pgfpathlineto{\pgfqpoint{4.758447in}{1.888796in}}%
\pgfpathlineto{\pgfqpoint{4.755756in}{1.707112in}}%
\pgfpathlineto{\pgfqpoint{4.750925in}{1.532959in}}%
\pgfpathlineto{\pgfqpoint{4.744785in}{1.398728in}}%
\pgfpathlineto{\pgfqpoint{4.737575in}{1.289517in}}%
\pgfpathlineto{\pgfqpoint{4.728713in}{1.190471in}}%
\pgfpathlineto{\pgfqpoint{4.719652in}{1.116522in}}%
\pgfpathlineto{\pgfqpoint{4.710035in}{1.055277in}}%
\pgfpathlineto{\pgfqpoint{4.699503in}{1.001862in}}%
\pgfpathlineto{\pgfqpoint{4.689040in}{0.958691in}}%
\pgfpathlineto{\pgfqpoint{4.677219in}{0.918601in}}%
\pgfpathlineto{\pgfqpoint{4.664034in}{0.881751in}}%
\pgfpathlineto{\pgfqpoint{4.650583in}{0.850493in}}%
\pgfpathlineto{\pgfqpoint{4.636303in}{0.822571in}}%
\pgfpathlineto{\pgfqpoint{4.620207in}{0.795976in}}%
\pgfpathlineto{\pgfqpoint{4.603640in}{0.772903in}}%
\pgfpathlineto{\pgfqpoint{4.585488in}{0.751447in}}%
\pgfpathlineto{\pgfqpoint{4.565874in}{0.731750in}}%
\pgfpathlineto{\pgfqpoint{4.544964in}{0.713880in}}%
\pgfpathlineto{\pgfqpoint{4.522958in}{0.697825in}}%
\pgfpathlineto{\pgfqpoint{4.496157in}{0.681291in}}%
\pgfpathlineto{\pgfqpoint{4.470398in}{0.667954in}}%
\pgfpathlineto{\pgfqpoint{4.439961in}{0.654510in}}%
\pgfpathlineto{\pgfqpoint{4.406841in}{0.642282in}}%
\pgfpathlineto{\pgfqpoint{4.369009in}{0.630749in}}%
\pgfpathlineto{\pgfqpoint{4.326489in}{0.620227in}}%
\pgfpathlineto{\pgfqpoint{4.279327in}{0.610950in}}%
\pgfpathlineto{\pgfqpoint{4.227576in}{0.603086in}}%
\pgfpathlineto{\pgfqpoint{4.173450in}{0.597064in}}%
\pgfpathlineto{\pgfqpoint{4.110511in}{0.592204in}}%
\pgfpathlineto{\pgfqpoint{4.047471in}{0.589537in}}%
\pgfpathlineto{\pgfqpoint{3.977867in}{0.588625in}}%
\pgfpathlineto{\pgfqpoint{3.906093in}{0.589934in}}%
\pgfpathlineto{\pgfqpoint{3.834377in}{0.593497in}}%
\pgfpathlineto{\pgfqpoint{3.767120in}{0.599068in}}%
\pgfpathlineto{\pgfqpoint{3.704364in}{0.606393in}}%
\pgfpathlineto{\pgfqpoint{3.678516in}{0.610511in}}%
\pgfpathlineto{\pgfqpoint{3.620438in}{0.620501in}}%
\pgfpathlineto{\pgfqpoint{3.586319in}{0.628207in}}%
\pgfpathlineto{\pgfqpoint{3.495240in}{0.652428in}}%
\pgfpathlineto{\pgfqpoint{3.451528in}{0.667583in}}%
\pgfpathlineto{\pgfqpoint{3.408538in}{0.685220in}}%
\pgfpathlineto{\pgfqpoint{3.374594in}{0.702001in}}%
\pgfpathlineto{\pgfqpoint{3.345407in}{0.718682in}}%
\pgfpathlineto{\pgfqpoint{3.315236in}{0.738520in}}%
\pgfpathlineto{\pgfqpoint{3.288127in}{0.759290in}}%
\pgfpathlineto{\pgfqpoint{3.264004in}{0.780551in}}%
\pgfpathlineto{\pgfqpoint{3.241208in}{0.803648in}}%
\pgfpathlineto{\pgfqpoint{3.219894in}{0.828530in}}%
\pgfpathlineto{\pgfqpoint{3.200189in}{0.855091in}}%
\pgfpathlineto{\pgfqpoint{3.182177in}{0.883183in}}%
\pgfpathlineto{\pgfqpoint{3.165906in}{0.912633in}}%
\pgfpathlineto{\pgfqpoint{3.150351in}{0.945448in}}%
\pgfpathlineto{\pgfqpoint{3.136682in}{0.979345in}}%
\pgfpathlineto{\pgfqpoint{3.124073in}{1.016460in}}%
\pgfpathlineto{\pgfqpoint{3.112834in}{1.056769in}}%
\pgfpathlineto{\pgfqpoint{3.103046in}{1.100146in}}%
\pgfpathlineto{\pgfqpoint{3.095343in}{1.144071in}}%
\pgfpathlineto{\pgfqpoint{3.089208in}{1.190837in}}%
\pgfpathlineto{\pgfqpoint{3.084595in}{1.242838in}}%
\pgfpathlineto{\pgfqpoint{3.082136in}{1.295031in}}%
\pgfpathlineto{\pgfqpoint{3.081687in}{1.349787in}}%
\pgfpathlineto{\pgfqpoint{3.083451in}{1.406998in}}%
\pgfpathlineto{\pgfqpoint{3.087181in}{1.461589in}}%
\pgfpathlineto{\pgfqpoint{3.093485in}{1.520888in}}%
\pgfpathlineto{\pgfqpoint{3.101823in}{1.577334in}}%
\pgfpathlineto{\pgfqpoint{3.111930in}{1.630856in}}%
\pgfpathlineto{\pgfqpoint{3.124690in}{1.686208in}}%
\pgfpathlineto{\pgfqpoint{3.139178in}{1.738395in}}%
\pgfpathlineto{\pgfqpoint{3.155145in}{1.787366in}}%
\pgfpathlineto{\pgfqpoint{3.172353in}{1.833085in}}%
\pgfpathlineto{\pgfqpoint{3.191618in}{1.877716in}}%
\pgfpathlineto{\pgfqpoint{3.214026in}{1.923261in}}%
\pgfpathlineto{\pgfqpoint{3.236214in}{1.963157in}}%
\pgfpathlineto{\pgfqpoint{3.260178in}{2.001684in}}%
\pgfpathlineto{\pgfqpoint{3.285814in}{2.038776in}}%
\pgfpathlineto{\pgfqpoint{3.314415in}{2.076285in}}%
\pgfpathlineto{\pgfqpoint{3.348944in}{2.117711in}}%
\pgfpathlineto{\pgfqpoint{3.417133in}{2.198022in}}%
\pgfpathlineto{\pgfqpoint{3.426053in}{2.212128in}}%
\pgfpathlineto{\pgfqpoint{3.430798in}{2.223297in}}%
\pgfpathlineto{\pgfqpoint{3.432034in}{2.230603in}}%
\pgfpathlineto{\pgfqpoint{3.430773in}{2.237856in}}%
\pgfpathlineto{\pgfqpoint{3.426621in}{2.243526in}}%
\pgfpathlineto{\pgfqpoint{3.420908in}{2.247084in}}%
\pgfpathlineto{\pgfqpoint{3.412501in}{2.249583in}}%
\pgfpathlineto{\pgfqpoint{3.399499in}{2.250689in}}%
\pgfpathlineto{\pgfqpoint{3.384305in}{2.249671in}}%
\pgfpathlineto{\pgfqpoint{3.364985in}{2.246098in}}%
\pgfpathlineto{\pgfqpoint{3.341804in}{2.239342in}}%
\pgfpathlineto{\pgfqpoint{3.317109in}{2.229682in}}%
\pgfpathlineto{\pgfqpoint{3.291104in}{2.216986in}}%
\pgfpathlineto{\pgfqpoint{3.265928in}{2.202261in}}%
\pgfpathlineto{\pgfqpoint{3.239805in}{2.184361in}}%
\pgfpathlineto{\pgfqpoint{3.214775in}{2.164519in}}%
\pgfpathlineto{\pgfqpoint{3.190900in}{2.142893in}}%
\pgfpathlineto{\pgfqpoint{3.166657in}{2.117912in}}%
\pgfpathlineto{\pgfqpoint{3.143835in}{2.091233in}}%
\pgfpathlineto{\pgfqpoint{3.121079in}{2.061107in}}%
\pgfpathlineto{\pgfqpoint{3.099952in}{2.029463in}}%
\pgfpathlineto{\pgfqpoint{3.079250in}{1.994406in}}%
\pgfpathlineto{\pgfqpoint{3.059218in}{1.955915in}}%
\pgfpathlineto{\pgfqpoint{3.040058in}{1.914015in}}%
\pgfpathlineto{\pgfqpoint{3.022809in}{1.871041in}}%
\pgfpathlineto{\pgfqpoint{3.005790in}{1.822536in}}%
\pgfpathlineto{\pgfqpoint{2.990067in}{1.770819in}}%
\pgfpathlineto{\pgfqpoint{2.975708in}{1.715979in}}%
\pgfpathlineto{\pgfqpoint{2.962284in}{1.655680in}}%
\pgfpathlineto{\pgfqpoint{2.950496in}{1.592386in}}%
\pgfpathlineto{\pgfqpoint{2.940383in}{1.526185in}}%
\pgfpathlineto{\pgfqpoint{2.931745in}{1.454681in}}%
\pgfpathlineto{\pgfqpoint{2.925082in}{1.380399in}}%
\pgfpathlineto{\pgfqpoint{2.920647in}{1.305899in}}%
\pgfpathlineto{\pgfqpoint{2.918444in}{1.231270in}}%
\pgfpathlineto{\pgfqpoint{2.918545in}{1.159087in}}%
\pgfpathlineto{\pgfqpoint{2.920787in}{1.091931in}}%
\pgfpathlineto{\pgfqpoint{2.925177in}{1.027412in}}%
\pgfpathlineto{\pgfqpoint{2.931192in}{0.970580in}}%
\pgfpathlineto{\pgfqpoint{2.938760in}{0.919034in}}%
\pgfpathlineto{\pgfqpoint{2.947651in}{0.872852in}}%
\pgfpathlineto{\pgfqpoint{2.958213in}{0.829714in}}%
\pgfpathlineto{\pgfqpoint{2.969670in}{0.792114in}}%
\pgfpathlineto{\pgfqpoint{2.982463in}{0.757773in}}%
\pgfpathlineto{\pgfqpoint{2.996425in}{0.726812in}}%
\pgfpathlineto{\pgfqpoint{3.011299in}{0.699300in}}%
\pgfpathlineto{\pgfqpoint{3.026739in}{0.675225in}}%
\pgfpathlineto{\pgfqpoint{3.043828in}{0.652656in}}%
\pgfpathlineto{\pgfqpoint{3.062495in}{0.631788in}}%
\pgfpathlineto{\pgfqpoint{3.082602in}{0.612753in}}%
\pgfpathlineto{\pgfqpoint{3.103961in}{0.595592in}}%
\pgfpathlineto{\pgfqpoint{3.128268in}{0.579069in}}%
\pgfpathlineto{\pgfqpoint{3.153537in}{0.564554in}}%
\pgfpathlineto{\pgfqpoint{3.181571in}{0.550952in}}%
\pgfpathlineto{\pgfqpoint{3.214371in}{0.537647in}}%
\pgfpathlineto{\pgfqpoint{3.249846in}{0.525712in}}%
\pgfpathlineto{\pgfqpoint{3.290011in}{0.514571in}}%
\pgfpathlineto{\pgfqpoint{3.334820in}{0.504423in}}%
\pgfpathlineto{\pgfqpoint{3.386372in}{0.494999in}}%
\pgfpathlineto{\pgfqpoint{3.446798in}{0.486257in}}%
\pgfpathlineto{\pgfqpoint{3.518243in}{0.478282in}}%
\pgfpathlineto{\pgfqpoint{3.600685in}{0.471409in}}%
\pgfpathlineto{\pgfqpoint{3.696268in}{0.465713in}}%
\pgfpathlineto{\pgfqpoint{3.807144in}{0.461369in}}%
\pgfpathlineto{\pgfqpoint{3.933291in}{0.458719in}}%
\pgfpathlineto{\pgfqpoint{4.063808in}{0.458211in}}%
\pgfpathlineto{\pgfqpoint{4.187792in}{0.459914in}}%
\pgfpathlineto{\pgfqpoint{4.294335in}{0.463521in}}%
\pgfpathlineto{\pgfqpoint{4.381234in}{0.468574in}}%
\pgfpathlineto{\pgfqpoint{4.450636in}{0.474701in}}%
\pgfpathlineto{\pgfqpoint{4.506850in}{0.481799in}}%
\pgfpathlineto{\pgfqpoint{4.552009in}{0.489658in}}%
\pgfpathlineto{\pgfqpoint{4.588239in}{0.498115in}}%
\pgfpathlineto{\pgfqpoint{4.617656in}{0.507110in}}%
\pgfpathlineto{\pgfqpoint{4.642328in}{0.516843in}}%
\pgfpathlineto{\pgfqpoint{4.664194in}{0.527940in}}%
\pgfpathlineto{\pgfqpoint{4.681238in}{0.538945in}}%
\pgfpathlineto{\pgfqpoint{4.697164in}{0.551953in}}%
\pgfpathlineto{\pgfqpoint{4.710076in}{0.565289in}}%
\pgfpathlineto{\pgfqpoint{4.721578in}{0.580218in}}%
\pgfpathlineto{\pgfqpoint{4.731557in}{0.596521in}}%
\pgfpathlineto{\pgfqpoint{4.741000in}{0.616134in}}%
\pgfpathlineto{\pgfqpoint{4.749521in}{0.639027in}}%
\pgfpathlineto{\pgfqpoint{4.757522in}{0.667450in}}%
\pgfpathlineto{\pgfqpoint{4.764572in}{0.701345in}}%
\pgfpathlineto{\pgfqpoint{4.770840in}{0.743043in}}%
\pgfpathlineto{\pgfqpoint{4.776327in}{0.794934in}}%
\pgfpathlineto{\pgfqpoint{4.781278in}{0.864398in}}%
\pgfpathlineto{\pgfqpoint{4.785468in}{0.956371in}}%
\pgfpathlineto{\pgfqpoint{4.789000in}{1.085745in}}%
\pgfpathlineto{\pgfqpoint{4.791852in}{1.277385in}}%
\pgfpathlineto{\pgfqpoint{4.793959in}{1.581058in}}%
\pgfpathlineto{\pgfqpoint{4.794962in}{2.071429in}}%
\pgfpathlineto{\pgfqpoint{4.793967in}{2.559311in}}%
\pgfpathlineto{\pgfqpoint{4.791733in}{2.745981in}}%
\pgfpathlineto{\pgfqpoint{4.788955in}{2.818091in}}%
\pgfpathlineto{\pgfqpoint{4.785731in}{2.850227in}}%
\pgfpathlineto{\pgfqpoint{4.781879in}{2.867057in}}%
\pgfpathlineto{\pgfqpoint{4.777744in}{2.875780in}}%
\pgfpathlineto{\pgfqpoint{4.773097in}{2.880982in}}%
\pgfpathlineto{\pgfqpoint{4.767363in}{2.884504in}}%
\pgfpathlineto{\pgfqpoint{4.756853in}{2.887622in}}%
\pgfpathlineto{\pgfqpoint{4.739548in}{2.889639in}}%
\pgfpathlineto{\pgfqpoint{4.704762in}{2.890882in}}%
\pgfpathlineto{\pgfqpoint{4.602524in}{2.891538in}}%
\pgfpathlineto{\pgfqpoint{3.952100in}{2.891742in}}%
\pgfpathlineto{\pgfqpoint{0.617321in}{2.890753in}}%
\pgfpathlineto{\pgfqpoint{0.549910in}{2.888858in}}%
\pgfpathlineto{\pgfqpoint{0.521735in}{2.886179in}}%
\pgfpathlineto{\pgfqpoint{0.504666in}{2.882389in}}%
\pgfpathlineto{\pgfqpoint{0.494501in}{2.878011in}}%
\pgfpathlineto{\pgfqpoint{0.487180in}{2.872667in}}%
\pgfpathlineto{\pgfqpoint{0.481152in}{2.865519in}}%
\pgfpathlineto{\pgfqpoint{0.475664in}{2.854804in}}%
\pgfpathlineto{\pgfqpoint{0.471318in}{2.840737in}}%
\pgfpathlineto{\pgfqpoint{0.467301in}{2.818823in}}%
\pgfpathlineto{\pgfqpoint{0.463927in}{2.786700in}}%
\pgfpathlineto{\pgfqpoint{0.460918in}{2.734544in}}%
\pgfpathlineto{\pgfqpoint{0.458363in}{2.647473in}}%
\pgfpathlineto{\pgfqpoint{0.456575in}{2.523031in}}%
\pgfpathlineto{\pgfqpoint{0.456575in}{2.523031in}}%
\pgfusepath{stroke}%
\end{pgfscope}%
\begin{pgfscope}%
\pgfpathrectangle{\pgfqpoint{0.448634in}{0.402556in}}{\pgfqpoint{4.350661in}{2.489204in}} %
\pgfusepath{clip}%
\pgfsetrectcap%
\pgfsetroundjoin%
\pgfsetlinewidth{1.003750pt}%
\definecolor{currentstroke}{rgb}{0.172549,0.627451,0.172549}%
\pgfsetstrokecolor{currentstroke}%
\pgfsetdash{}{0pt}%
\pgfpathmoveto{\pgfqpoint{0.456424in}{1.370137in}}%
\pgfpathlineto{\pgfqpoint{0.459610in}{1.118755in}}%
\pgfpathlineto{\pgfqpoint{0.463695in}{0.962007in}}%
\pgfpathlineto{\pgfqpoint{0.468519in}{0.857610in}}%
\pgfpathlineto{\pgfqpoint{0.474082in}{0.783210in}}%
\pgfpathlineto{\pgfqpoint{0.480226in}{0.728906in}}%
\pgfpathlineto{\pgfqpoint{0.486970in}{0.687306in}}%
\pgfpathlineto{\pgfqpoint{0.494537in}{0.653558in}}%
\pgfpathlineto{\pgfqpoint{0.503107in}{0.625355in}}%
\pgfpathlineto{\pgfqpoint{0.512193in}{0.602749in}}%
\pgfpathlineto{\pgfqpoint{0.522200in}{0.583508in}}%
\pgfpathlineto{\pgfqpoint{0.534108in}{0.565743in}}%
\pgfpathlineto{\pgfqpoint{0.546263in}{0.551507in}}%
\pgfpathlineto{\pgfqpoint{0.559728in}{0.538907in}}%
\pgfpathlineto{\pgfqpoint{0.576129in}{0.526693in}}%
\pgfpathlineto{\pgfqpoint{0.595483in}{0.515351in}}%
\pgfpathlineto{\pgfqpoint{0.617681in}{0.505147in}}%
\pgfpathlineto{\pgfqpoint{0.642568in}{0.496153in}}%
\pgfpathlineto{\pgfqpoint{0.672126in}{0.487778in}}%
\pgfpathlineto{\pgfqpoint{0.708443in}{0.479824in}}%
\pgfpathlineto{\pgfqpoint{0.753649in}{0.472325in}}%
\pgfpathlineto{\pgfqpoint{0.807718in}{0.465660in}}%
\pgfpathlineto{\pgfqpoint{0.877116in}{0.459475in}}%
\pgfpathlineto{\pgfqpoint{0.961828in}{0.454230in}}%
\pgfpathlineto{\pgfqpoint{1.068351in}{0.449916in}}%
\pgfpathlineto{\pgfqpoint{1.201018in}{0.446839in}}%
\pgfpathlineto{\pgfqpoint{1.357637in}{0.445481in}}%
\pgfpathlineto{\pgfqpoint{1.525135in}{0.446232in}}%
\pgfpathlineto{\pgfqpoint{1.686088in}{0.449142in}}%
\pgfpathlineto{\pgfqpoint{1.823074in}{0.453747in}}%
\pgfpathlineto{\pgfqpoint{1.938245in}{0.459764in}}%
\pgfpathlineto{\pgfqpoint{2.031582in}{0.466759in}}%
\pgfpathlineto{\pgfqpoint{2.109580in}{0.474745in}}%
\pgfpathlineto{\pgfqpoint{2.174384in}{0.483535in}}%
\pgfpathlineto{\pgfqpoint{2.228139in}{0.492940in}}%
\pgfpathlineto{\pgfqpoint{2.275119in}{0.503356in}}%
\pgfpathlineto{\pgfqpoint{2.315282in}{0.514501in}}%
\pgfpathlineto{\pgfqpoint{2.350698in}{0.526659in}}%
\pgfpathlineto{\pgfqpoint{2.381320in}{0.539536in}}%
\pgfpathlineto{\pgfqpoint{2.407164in}{0.552659in}}%
\pgfpathlineto{\pgfqpoint{2.430226in}{0.566639in}}%
\pgfpathlineto{\pgfqpoint{2.452282in}{0.582602in}}%
\pgfpathlineto{\pgfqpoint{2.471391in}{0.599069in}}%
\pgfpathlineto{\pgfqpoint{2.489240in}{0.617293in}}%
\pgfpathlineto{\pgfqpoint{2.505678in}{0.637180in}}%
\pgfpathlineto{\pgfqpoint{2.520620in}{0.658557in}}%
\pgfpathlineto{\pgfqpoint{2.535213in}{0.683314in}}%
\pgfpathlineto{\pgfqpoint{2.549115in}{0.711484in}}%
\pgfpathlineto{\pgfqpoint{2.562091in}{0.743004in}}%
\pgfpathlineto{\pgfqpoint{2.574020in}{0.777751in}}%
\pgfpathlineto{\pgfqpoint{2.585502in}{0.817970in}}%
\pgfpathlineto{\pgfqpoint{2.596809in}{0.866038in}}%
\pgfpathlineto{\pgfqpoint{2.607562in}{0.921948in}}%
\pgfpathlineto{\pgfqpoint{2.617925in}{0.988098in}}%
\pgfpathlineto{\pgfqpoint{2.627958in}{1.066918in}}%
\pgfpathlineto{\pgfqpoint{2.637941in}{1.163320in}}%
\pgfpathlineto{\pgfqpoint{2.648424in}{1.287199in}}%
\pgfpathlineto{\pgfqpoint{2.660103in}{1.453438in}}%
\pgfpathlineto{\pgfqpoint{2.674773in}{1.696801in}}%
\pgfpathlineto{\pgfqpoint{2.687716in}{1.945279in}}%
\pgfpathlineto{\pgfqpoint{2.692670in}{2.079573in}}%
\pgfpathlineto{\pgfqpoint{2.693829in}{2.166682in}}%
\pgfpathlineto{\pgfqpoint{2.692565in}{2.233870in}}%
\pgfpathlineto{\pgfqpoint{2.689436in}{2.286015in}}%
\pgfpathlineto{\pgfqpoint{2.684859in}{2.327999in}}%
\pgfpathlineto{\pgfqpoint{2.678725in}{2.364664in}}%
\pgfpathlineto{\pgfqpoint{2.671356in}{2.395897in}}%
\pgfpathlineto{\pgfqpoint{2.662489in}{2.423981in}}%
\pgfpathlineto{\pgfqpoint{2.652361in}{2.448778in}}%
\pgfpathlineto{\pgfqpoint{2.641365in}{2.470245in}}%
\pgfpathlineto{\pgfqpoint{2.628643in}{2.490425in}}%
\pgfpathlineto{\pgfqpoint{2.614278in}{2.509106in}}%
\pgfpathlineto{\pgfqpoint{2.598443in}{2.526159in}}%
\pgfpathlineto{\pgfqpoint{2.579590in}{2.543005in}}%
\pgfpathlineto{\pgfqpoint{2.559532in}{2.557923in}}%
\pgfpathlineto{\pgfqpoint{2.536602in}{2.572183in}}%
\pgfpathlineto{\pgfqpoint{2.510849in}{2.585538in}}%
\pgfpathlineto{\pgfqpoint{2.482360in}{2.597837in}}%
\pgfpathlineto{\pgfqpoint{2.449134in}{2.609683in}}%
\pgfpathlineto{\pgfqpoint{2.411184in}{2.620696in}}%
\pgfpathlineto{\pgfqpoint{2.368552in}{2.630606in}}%
\pgfpathlineto{\pgfqpoint{2.321294in}{2.639221in}}%
\pgfpathlineto{\pgfqpoint{2.269467in}{2.646399in}}%
\pgfpathlineto{\pgfqpoint{2.210954in}{2.652193in}}%
\pgfpathlineto{\pgfqpoint{2.147967in}{2.656153in}}%
\pgfpathlineto{\pgfqpoint{2.080556in}{2.658135in}}%
\pgfpathlineto{\pgfqpoint{2.010948in}{2.657971in}}%
\pgfpathlineto{\pgfqpoint{1.939195in}{2.655572in}}%
\pgfpathlineto{\pgfqpoint{1.867527in}{2.650913in}}%
\pgfpathlineto{\pgfqpoint{1.798171in}{2.644140in}}%
\pgfpathlineto{\pgfqpoint{1.733341in}{2.635606in}}%
\pgfpathlineto{\pgfqpoint{1.673075in}{2.625521in}}%
\pgfpathlineto{\pgfqpoint{1.615274in}{2.613610in}}%
\pgfpathlineto{\pgfqpoint{1.562133in}{2.600402in}}%
\pgfpathlineto{\pgfqpoint{1.513681in}{2.586139in}}%
\pgfpathlineto{\pgfqpoint{1.467862in}{2.570344in}}%
\pgfpathlineto{\pgfqpoint{1.426794in}{2.553923in}}%
\pgfpathlineto{\pgfqpoint{1.388447in}{2.536289in}}%
\pgfpathlineto{\pgfqpoint{1.352878in}{2.517566in}}%
\pgfpathlineto{\pgfqpoint{1.320128in}{2.497922in}}%
\pgfpathlineto{\pgfqpoint{1.288379in}{2.476236in}}%
\pgfpathlineto{\pgfqpoint{1.259592in}{2.453861in}}%
\pgfpathlineto{\pgfqpoint{1.232050in}{2.429520in}}%
\pgfpathlineto{\pgfqpoint{1.207527in}{2.404898in}}%
\pgfpathlineto{\pgfqpoint{1.184409in}{2.378557in}}%
\pgfpathlineto{\pgfqpoint{1.162828in}{2.350561in}}%
\pgfpathlineto{\pgfqpoint{1.142891in}{2.321011in}}%
\pgfpathlineto{\pgfqpoint{1.124675in}{2.290041in}}%
\pgfpathlineto{\pgfqpoint{1.108225in}{2.257802in}}%
\pgfpathlineto{\pgfqpoint{1.092639in}{2.222199in}}%
\pgfpathlineto{\pgfqpoint{1.079059in}{2.185535in}}%
\pgfpathlineto{\pgfqpoint{1.067443in}{2.147998in}}%
\pgfpathlineto{\pgfqpoint{1.057187in}{2.107347in}}%
\pgfpathlineto{\pgfqpoint{1.049004in}{2.066086in}}%
\pgfpathlineto{\pgfqpoint{1.042513in}{2.021906in}}%
\pgfpathlineto{\pgfqpoint{1.038177in}{1.977382in}}%
\pgfpathlineto{\pgfqpoint{1.035865in}{1.930167in}}%
\pgfpathlineto{\pgfqpoint{1.035826in}{1.882878in}}%
\pgfpathlineto{\pgfqpoint{1.038031in}{1.835656in}}%
\pgfpathlineto{\pgfqpoint{1.042474in}{1.788641in}}%
\pgfpathlineto{\pgfqpoint{1.049176in}{1.741978in}}%
\pgfpathlineto{\pgfqpoint{1.057644in}{1.698239in}}%
\pgfpathlineto{\pgfqpoint{1.068221in}{1.655105in}}%
\pgfpathlineto{\pgfqpoint{1.080962in}{1.612745in}}%
\pgfpathlineto{\pgfqpoint{1.095031in}{1.573617in}}%
\pgfpathlineto{\pgfqpoint{1.111115in}{1.535520in}}%
\pgfpathlineto{\pgfqpoint{1.128118in}{1.500775in}}%
\pgfpathlineto{\pgfqpoint{1.146930in}{1.467274in}}%
\pgfpathlineto{\pgfqpoint{1.167531in}{1.435181in}}%
\pgfpathlineto{\pgfqpoint{1.189874in}{1.404652in}}%
\pgfpathlineto{\pgfqpoint{1.213884in}{1.375828in}}%
\pgfpathlineto{\pgfqpoint{1.237817in}{1.350457in}}%
\pgfpathlineto{\pgfqpoint{1.264748in}{1.325237in}}%
\pgfpathlineto{\pgfqpoint{1.292991in}{1.301972in}}%
\pgfpathlineto{\pgfqpoint{1.322398in}{1.280678in}}%
\pgfpathlineto{\pgfqpoint{1.352820in}{1.261340in}}%
\pgfpathlineto{\pgfqpoint{1.386095in}{1.242889in}}%
\pgfpathlineto{\pgfqpoint{1.420190in}{1.226516in}}%
\pgfpathlineto{\pgfqpoint{1.457024in}{1.211329in}}%
\pgfpathlineto{\pgfqpoint{1.496554in}{1.197536in}}%
\pgfpathlineto{\pgfqpoint{1.538719in}{1.185287in}}%
\pgfpathlineto{\pgfqpoint{1.583441in}{1.174641in}}%
\pgfpathlineto{\pgfqpoint{1.634929in}{1.164775in}}%
\pgfpathlineto{\pgfqpoint{1.706063in}{1.153745in}}%
\pgfpathlineto{\pgfqpoint{1.768492in}{1.143417in}}%
\pgfpathlineto{\pgfqpoint{1.796122in}{1.136567in}}%
\pgfpathlineto{\pgfqpoint{1.812683in}{1.130481in}}%
\pgfpathlineto{\pgfqpoint{1.824471in}{1.124102in}}%
\pgfpathlineto{\pgfqpoint{1.833209in}{1.116741in}}%
\pgfpathlineto{\pgfqpoint{1.838498in}{1.108890in}}%
\pgfpathlineto{\pgfqpoint{1.840588in}{1.101849in}}%
\pgfpathlineto{\pgfqpoint{1.840619in}{1.094412in}}%
\pgfpathlineto{\pgfqpoint{1.837931in}{1.084986in}}%
\pgfpathlineto{\pgfqpoint{1.833246in}{1.076615in}}%
\pgfpathlineto{\pgfqpoint{1.825819in}{1.067542in}}%
\pgfpathlineto{\pgfqpoint{1.813813in}{1.056850in}}%
\pgfpathlineto{\pgfqpoint{1.798819in}{1.046763in}}%
\pgfpathlineto{\pgfqpoint{1.781016in}{1.037462in}}%
\pgfpathlineto{\pgfqpoint{1.758447in}{1.028391in}}%
\pgfpathlineto{\pgfqpoint{1.733203in}{1.020815in}}%
\pgfpathlineto{\pgfqpoint{1.705410in}{1.014872in}}%
\pgfpathlineto{\pgfqpoint{1.675178in}{1.010714in}}%
\pgfpathlineto{\pgfqpoint{1.642610in}{1.008507in}}%
\pgfpathlineto{\pgfqpoint{1.607809in}{1.008432in}}%
\pgfpathlineto{\pgfqpoint{1.570886in}{1.010691in}}%
\pgfpathlineto{\pgfqpoint{1.534118in}{1.015181in}}%
\pgfpathlineto{\pgfqpoint{1.495454in}{1.022233in}}%
\pgfpathlineto{\pgfqpoint{1.457161in}{1.031563in}}%
\pgfpathlineto{\pgfqpoint{1.419337in}{1.043132in}}%
\pgfpathlineto{\pgfqpoint{1.382089in}{1.056929in}}%
\pgfpathlineto{\pgfqpoint{1.347544in}{1.072019in}}%
\pgfpathlineto{\pgfqpoint{1.313727in}{1.089133in}}%
\pgfpathlineto{\pgfqpoint{1.280762in}{1.108299in}}%
\pgfpathlineto{\pgfqpoint{1.248782in}{1.129536in}}%
\pgfpathlineto{\pgfqpoint{1.219708in}{1.151422in}}%
\pgfpathlineto{\pgfqpoint{1.191752in}{1.175138in}}%
\pgfpathlineto{\pgfqpoint{1.165031in}{1.200649in}}%
\pgfpathlineto{\pgfqpoint{1.139653in}{1.227898in}}%
\pgfpathlineto{\pgfqpoint{1.115714in}{1.256800in}}%
\pgfpathlineto{\pgfqpoint{1.093288in}{1.287251in}}%
\pgfpathlineto{\pgfqpoint{1.071178in}{1.321163in}}%
\pgfpathlineto{\pgfqpoint{1.050868in}{1.356520in}}%
\pgfpathlineto{\pgfqpoint{1.032365in}{1.393152in}}%
\pgfpathlineto{\pgfqpoint{1.014718in}{1.433142in}}%
\pgfpathlineto{\pgfqpoint{0.999024in}{1.474185in}}%
\pgfpathlineto{\pgfqpoint{0.984506in}{1.518461in}}%
\pgfpathlineto{\pgfqpoint{0.972010in}{1.563537in}}%
\pgfpathlineto{\pgfqpoint{0.960944in}{1.611678in}}%
\pgfpathlineto{\pgfqpoint{0.951530in}{1.662824in}}%
\pgfpathlineto{\pgfqpoint{0.944286in}{1.714431in}}%
\pgfpathlineto{\pgfqpoint{0.938950in}{1.768847in}}%
\pgfpathlineto{\pgfqpoint{0.935870in}{1.823491in}}%
\pgfpathlineto{\pgfqpoint{0.935034in}{1.878240in}}%
\pgfpathlineto{\pgfqpoint{0.936466in}{1.932973in}}%
\pgfpathlineto{\pgfqpoint{0.940005in}{1.985084in}}%
\pgfpathlineto{\pgfqpoint{0.945759in}{2.036935in}}%
\pgfpathlineto{\pgfqpoint{0.953410in}{2.085938in}}%
\pgfpathlineto{\pgfqpoint{0.962764in}{2.132000in}}%
\pgfpathlineto{\pgfqpoint{0.974287in}{2.177414in}}%
\pgfpathlineto{\pgfqpoint{0.987332in}{2.219653in}}%
\pgfpathlineto{\pgfqpoint{1.001667in}{2.258654in}}%
\pgfpathlineto{\pgfqpoint{1.018051in}{2.296583in}}%
\pgfpathlineto{\pgfqpoint{1.035401in}{2.331101in}}%
\pgfpathlineto{\pgfqpoint{1.054650in}{2.364275in}}%
\pgfpathlineto{\pgfqpoint{1.074407in}{2.393984in}}%
\pgfpathlineto{\pgfqpoint{1.095771in}{2.422197in}}%
\pgfpathlineto{\pgfqpoint{1.118662in}{2.448797in}}%
\pgfpathlineto{\pgfqpoint{1.142967in}{2.473701in}}%
\pgfpathlineto{\pgfqpoint{1.168551in}{2.496867in}}%
\pgfpathlineto{\pgfqpoint{1.197085in}{2.519662in}}%
\pgfpathlineto{\pgfqpoint{1.226727in}{2.540526in}}%
\pgfpathlineto{\pgfqpoint{1.259242in}{2.560673in}}%
\pgfpathlineto{\pgfqpoint{1.294612in}{2.579881in}}%
\pgfpathlineto{\pgfqpoint{1.332793in}{2.597982in}}%
\pgfpathlineto{\pgfqpoint{1.373719in}{2.614859in}}%
\pgfpathlineto{\pgfqpoint{1.417320in}{2.630445in}}%
\pgfpathlineto{\pgfqpoint{1.465633in}{2.645312in}}%
\pgfpathlineto{\pgfqpoint{1.518640in}{2.659204in}}%
\pgfpathlineto{\pgfqpoint{1.576309in}{2.671929in}}%
\pgfpathlineto{\pgfqpoint{1.638598in}{2.683344in}}%
\pgfpathlineto{\pgfqpoint{1.705462in}{2.693343in}}%
\pgfpathlineto{\pgfqpoint{1.779028in}{2.702064in}}%
\pgfpathlineto{\pgfqpoint{1.857097in}{2.709077in}}%
\pgfpathlineto{\pgfqpoint{1.939633in}{2.714280in}}%
\pgfpathlineto{\pgfqpoint{2.026598in}{2.717513in}}%
\pgfpathlineto{\pgfqpoint{2.113605in}{2.718523in}}%
\pgfpathlineto{\pgfqpoint{2.198435in}{2.717303in}}%
\pgfpathlineto{\pgfqpoint{2.278866in}{2.713929in}}%
\pgfpathlineto{\pgfqpoint{2.352678in}{2.708598in}}%
\pgfpathlineto{\pgfqpoint{2.417657in}{2.701709in}}%
\pgfpathlineto{\pgfqpoint{2.473770in}{2.693630in}}%
\pgfpathlineto{\pgfqpoint{2.523140in}{2.684368in}}%
\pgfpathlineto{\pgfqpoint{2.565726in}{2.674202in}}%
\pgfpathlineto{\pgfqpoint{2.601510in}{2.663544in}}%
\pgfpathlineto{\pgfqpoint{2.632577in}{2.652142in}}%
\pgfpathlineto{\pgfqpoint{2.658899in}{2.640331in}}%
\pgfpathlineto{\pgfqpoint{2.682438in}{2.627436in}}%
\pgfpathlineto{\pgfqpoint{2.703062in}{2.613571in}}%
\pgfpathlineto{\pgfqpoint{2.720674in}{2.598978in}}%
\pgfpathlineto{\pgfqpoint{2.735263in}{2.584053in}}%
\pgfpathlineto{\pgfqpoint{2.748320in}{2.567377in}}%
\pgfpathlineto{\pgfqpoint{2.759553in}{2.549045in}}%
\pgfpathlineto{\pgfqpoint{2.768788in}{2.529306in}}%
\pgfpathlineto{\pgfqpoint{2.776017in}{2.508498in}}%
\pgfpathlineto{\pgfqpoint{2.781884in}{2.484540in}}%
\pgfpathlineto{\pgfqpoint{2.786102in}{2.457596in}}%
\pgfpathlineto{\pgfqpoint{2.788720in}{2.425384in}}%
\pgfpathlineto{\pgfqpoint{2.789427in}{2.388061in}}%
\pgfpathlineto{\pgfqpoint{2.787962in}{2.340801in}}%
\pgfpathlineto{\pgfqpoint{2.783672in}{2.278768in}}%
\pgfpathlineto{\pgfqpoint{2.774288in}{2.179783in}}%
\pgfpathlineto{\pgfqpoint{2.743611in}{1.868118in}}%
\pgfpathlineto{\pgfqpoint{2.730112in}{1.702060in}}%
\pgfpathlineto{\pgfqpoint{2.717287in}{1.515949in}}%
\pgfpathlineto{\pgfqpoint{2.702602in}{1.267597in}}%
\pgfpathlineto{\pgfqpoint{2.684434in}{0.964630in}}%
\pgfpathlineto{\pgfqpoint{2.675374in}{0.850599in}}%
\pgfpathlineto{\pgfqpoint{2.667030in}{0.771523in}}%
\pgfpathlineto{\pgfqpoint{2.658752in}{0.712543in}}%
\pgfpathlineto{\pgfqpoint{2.650176in}{0.666284in}}%
\pgfpathlineto{\pgfqpoint{2.640820in}{0.627931in}}%
\pgfpathlineto{\pgfqpoint{2.631145in}{0.597534in}}%
\pgfpathlineto{\pgfqpoint{2.621004in}{0.572744in}}%
\pgfpathlineto{\pgfqpoint{2.609856in}{0.551383in}}%
\pgfpathlineto{\pgfqpoint{2.598042in}{0.533534in}}%
\pgfpathlineto{\pgfqpoint{2.584495in}{0.517378in}}%
\pgfpathlineto{\pgfqpoint{2.571109in}{0.504669in}}%
\pgfpathlineto{\pgfqpoint{2.554789in}{0.492313in}}%
\pgfpathlineto{\pgfqpoint{2.537456in}{0.481914in}}%
\pgfpathlineto{\pgfqpoint{2.517373in}{0.472367in}}%
\pgfpathlineto{\pgfqpoint{2.492542in}{0.463178in}}%
\pgfpathlineto{\pgfqpoint{2.462979in}{0.454833in}}%
\pgfpathlineto{\pgfqpoint{2.428766in}{0.447542in}}%
\pgfpathlineto{\pgfqpoint{2.385671in}{0.440735in}}%
\pgfpathlineto{\pgfqpoint{2.331557in}{0.434581in}}%
\pgfpathlineto{\pgfqpoint{2.262115in}{0.429077in}}%
\pgfpathlineto{\pgfqpoint{2.170850in}{0.424236in}}%
\pgfpathlineto{\pgfqpoint{2.049086in}{0.420134in}}%
\pgfpathlineto{\pgfqpoint{1.879436in}{0.416783in}}%
\pgfpathlineto{\pgfqpoint{1.640159in}{0.414418in}}%
\pgfpathlineto{\pgfqpoint{1.322562in}{0.413569in}}%
\pgfpathlineto{\pgfqpoint{1.020194in}{0.414850in}}%
\pgfpathlineto{\pgfqpoint{0.822256in}{0.417715in}}%
\pgfpathlineto{\pgfqpoint{0.704835in}{0.421430in}}%
\pgfpathlineto{\pgfqpoint{0.630976in}{0.425829in}}%
\pgfpathlineto{\pgfqpoint{0.583315in}{0.430734in}}%
\pgfpathlineto{\pgfqpoint{0.551033in}{0.436124in}}%
\pgfpathlineto{\pgfqpoint{0.527708in}{0.442189in}}%
\pgfpathlineto{\pgfqpoint{0.511250in}{0.448626in}}%
\pgfpathlineto{\pgfqpoint{0.499549in}{0.455216in}}%
\pgfpathlineto{\pgfqpoint{0.488916in}{0.463842in}}%
\pgfpathlineto{\pgfqpoint{0.481322in}{0.472731in}}%
\pgfpathlineto{\pgfqpoint{0.474078in}{0.485127in}}%
\pgfpathlineto{\pgfqpoint{0.468753in}{0.498749in}}%
\pgfpathlineto{\pgfqpoint{0.463869in}{0.517849in}}%
\pgfpathlineto{\pgfqpoint{0.459679in}{0.544797in}}%
\pgfpathlineto{\pgfqpoint{0.456386in}{0.581939in}}%
\pgfpathlineto{\pgfqpoint{0.453731in}{0.639107in}}%
\pgfpathlineto{\pgfqpoint{0.451681in}{0.736155in}}%
\pgfpathlineto{\pgfqpoint{0.450220in}{0.927816in}}%
\pgfpathlineto{\pgfqpoint{0.449345in}{1.403252in}}%
\pgfpathlineto{\pgfqpoint{0.449543in}{2.682703in}}%
\pgfpathlineto{\pgfqpoint{0.451011in}{2.856933in}}%
\pgfpathlineto{\pgfqpoint{0.452802in}{2.879220in}}%
\pgfpathlineto{\pgfqpoint{0.455188in}{2.886108in}}%
\pgfpathlineto{\pgfqpoint{0.458626in}{2.889029in}}%
\pgfpathlineto{\pgfqpoint{0.464996in}{2.890553in}}%
\pgfpathlineto{\pgfqpoint{0.482377in}{2.891423in}}%
\pgfpathlineto{\pgfqpoint{0.565038in}{2.891729in}}%
\pgfpathlineto{\pgfqpoint{2.733843in}{2.891760in}}%
\pgfpathlineto{\pgfqpoint{4.789510in}{2.890885in}}%
\pgfpathlineto{\pgfqpoint{4.793727in}{2.889730in}}%
\pgfpathlineto{\pgfqpoint{4.795481in}{2.888306in}}%
\pgfpathlineto{\pgfqpoint{4.797106in}{2.881145in}}%
\pgfpathlineto{\pgfqpoint{4.797997in}{2.858771in}}%
\pgfpathlineto{\pgfqpoint{4.798039in}{2.856282in}}%
\pgfpathlineto{\pgfqpoint{4.798039in}{2.856282in}}%
\pgfusepath{stroke}%
\end{pgfscope}%
\begin{pgfscope}%
\pgfpathrectangle{\pgfqpoint{0.448634in}{0.402556in}}{\pgfqpoint{4.350661in}{2.489204in}} %
\pgfusepath{clip}%
\pgfsetrectcap%
\pgfsetroundjoin%
\pgfsetlinewidth{1.003750pt}%
\definecolor{currentstroke}{rgb}{0.172549,0.627451,0.172549}%
\pgfsetstrokecolor{currentstroke}%
\pgfsetdash{}{0pt}%
\pgfpathmoveto{\pgfqpoint{3.428760in}{0.402610in}}%
\pgfpathlineto{\pgfqpoint{2.806619in}{0.403759in}}%
\pgfpathlineto{\pgfqpoint{2.769679in}{0.405576in}}%
\pgfpathlineto{\pgfqpoint{2.754620in}{0.408061in}}%
\pgfpathlineto{\pgfqpoint{2.746378in}{0.411194in}}%
\pgfpathlineto{\pgfqpoint{2.740931in}{0.415262in}}%
\pgfpathlineto{\pgfqpoint{2.736773in}{0.420983in}}%
\pgfpathlineto{\pgfqpoint{2.733272in}{0.430071in}}%
\pgfpathlineto{\pgfqpoint{2.730442in}{0.444636in}}%
\pgfpathlineto{\pgfqpoint{2.728233in}{0.469392in}}%
\pgfpathlineto{\pgfqpoint{2.726467in}{0.519131in}}%
\pgfpathlineto{\pgfqpoint{2.725709in}{0.613715in}}%
\pgfpathlineto{\pgfqpoint{2.726840in}{0.768039in}}%
\pgfpathlineto{\pgfqpoint{2.730555in}{0.962149in}}%
\pgfpathlineto{\pgfqpoint{2.736610in}{1.158671in}}%
\pgfpathlineto{\pgfqpoint{2.744091in}{1.327719in}}%
\pgfpathlineto{\pgfqpoint{2.753200in}{1.484190in}}%
\pgfpathlineto{\pgfqpoint{2.763256in}{1.620610in}}%
\pgfpathlineto{\pgfqpoint{2.776117in}{1.764216in}}%
\pgfpathlineto{\pgfqpoint{2.788913in}{1.877777in}}%
\pgfpathlineto{\pgfqpoint{2.805746in}{2.005741in}}%
\pgfpathlineto{\pgfqpoint{2.821174in}{2.101198in}}%
\pgfpathlineto{\pgfqpoint{2.838357in}{2.193719in}}%
\pgfpathlineto{\pgfqpoint{2.859133in}{2.292966in}}%
\pgfpathlineto{\pgfqpoint{2.887206in}{2.425961in}}%
\pgfpathlineto{\pgfqpoint{2.896988in}{2.479561in}}%
\pgfpathlineto{\pgfqpoint{2.901539in}{2.516524in}}%
\pgfpathlineto{\pgfqpoint{2.902845in}{2.543855in}}%
\pgfpathlineto{\pgfqpoint{2.901952in}{2.566224in}}%
\pgfpathlineto{\pgfqpoint{2.899146in}{2.585864in}}%
\pgfpathlineto{\pgfqpoint{2.894788in}{2.602547in}}%
\pgfpathlineto{\pgfqpoint{2.888477in}{2.618388in}}%
\pgfpathlineto{\pgfqpoint{2.880250in}{2.633033in}}%
\pgfpathlineto{\pgfqpoint{2.870341in}{2.646246in}}%
\pgfpathlineto{\pgfqpoint{2.857392in}{2.659530in}}%
\pgfpathlineto{\pgfqpoint{2.843182in}{2.671009in}}%
\pgfpathlineto{\pgfqpoint{2.824230in}{2.683208in}}%
\pgfpathlineto{\pgfqpoint{2.802405in}{2.694417in}}%
\pgfpathlineto{\pgfqpoint{2.775801in}{2.705368in}}%
\pgfpathlineto{\pgfqpoint{2.744453in}{2.715714in}}%
\pgfpathlineto{\pgfqpoint{2.708428in}{2.725251in}}%
\pgfpathlineto{\pgfqpoint{2.665647in}{2.734288in}}%
\pgfpathlineto{\pgfqpoint{2.613983in}{2.742868in}}%
\pgfpathlineto{\pgfqpoint{2.553451in}{2.750587in}}%
\pgfpathlineto{\pgfqpoint{2.481912in}{2.757364in}}%
\pgfpathlineto{\pgfqpoint{2.399390in}{2.762838in}}%
\pgfpathlineto{\pgfqpoint{2.310261in}{2.766480in}}%
\pgfpathlineto{\pgfqpoint{2.177584in}{2.768716in}}%
\pgfpathlineto{\pgfqpoint{2.068820in}{2.767977in}}%
\pgfpathlineto{\pgfqpoint{1.955737in}{2.764942in}}%
\pgfpathlineto{\pgfqpoint{1.853593in}{2.759888in}}%
\pgfpathlineto{\pgfqpoint{1.747212in}{2.752355in}}%
\pgfpathlineto{\pgfqpoint{1.660530in}{2.743697in}}%
\pgfpathlineto{\pgfqpoint{1.582703in}{2.733766in}}%
\pgfpathlineto{\pgfqpoint{1.492197in}{2.719732in}}%
\pgfpathlineto{\pgfqpoint{1.417223in}{2.704697in}}%
\pgfpathlineto{\pgfqpoint{1.361984in}{2.690816in}}%
\pgfpathlineto{\pgfqpoint{1.311452in}{2.675816in}}%
\pgfpathlineto{\pgfqpoint{1.265659in}{2.659921in}}%
\pgfpathlineto{\pgfqpoint{1.222567in}{2.642583in}}%
\pgfpathlineto{\pgfqpoint{1.184317in}{2.624679in}}%
\pgfpathlineto{\pgfqpoint{1.148885in}{2.605619in}}%
\pgfpathlineto{\pgfqpoint{1.116324in}{2.585569in}}%
\pgfpathlineto{\pgfqpoint{1.092321in}{2.568506in}}%
\pgfpathlineto{\pgfqpoint{1.079754in}{2.558680in}}%
\pgfpathlineto{\pgfqpoint{1.051538in}{2.535373in}}%
\pgfpathlineto{\pgfqpoint{1.026306in}{2.511706in}}%
\pgfpathlineto{\pgfqpoint{1.002393in}{2.486311in}}%
\pgfpathlineto{\pgfqpoint{0.979908in}{2.459262in}}%
\pgfpathlineto{\pgfqpoint{0.958930in}{2.430671in}}%
\pgfpathlineto{\pgfqpoint{0.938260in}{2.398635in}}%
\pgfpathlineto{\pgfqpoint{0.923043in}{2.371377in}}%
\pgfpathlineto{\pgfqpoint{0.904510in}{2.334766in}}%
\pgfpathlineto{\pgfqpoint{0.887851in}{2.296992in}}%
\pgfpathlineto{\pgfqpoint{0.872129in}{2.255963in}}%
\pgfpathlineto{\pgfqpoint{0.857505in}{2.211732in}}%
\pgfpathlineto{\pgfqpoint{0.844760in}{2.166748in}}%
\pgfpathlineto{\pgfqpoint{0.838622in}{2.140297in}}%
\pgfpathlineto{\pgfqpoint{0.826981in}{2.087184in}}%
\pgfpathlineto{\pgfqpoint{0.816321in}{2.028706in}}%
\pgfpathlineto{\pgfqpoint{0.810087in}{1.984485in}}%
\pgfpathlineto{\pgfqpoint{0.808025in}{1.967229in}}%
\pgfpathlineto{\pgfqpoint{0.800076in}{1.898131in}}%
\pgfpathlineto{\pgfqpoint{0.793713in}{1.823813in}}%
\pgfpathlineto{\pgfqpoint{0.788799in}{1.741865in}}%
\pgfpathlineto{\pgfqpoint{0.786200in}{1.677216in}}%
\pgfpathlineto{\pgfqpoint{0.776952in}{1.453472in}}%
\pgfpathlineto{\pgfqpoint{0.773280in}{1.418885in}}%
\pgfpathlineto{\pgfqpoint{0.768298in}{1.389573in}}%
\pgfpathlineto{\pgfqpoint{0.762751in}{1.368099in}}%
\pgfpathlineto{\pgfqpoint{0.756721in}{1.352113in}}%
\pgfpathlineto{\pgfqpoint{0.749750in}{1.339511in}}%
\pgfpathlineto{\pgfqpoint{0.742198in}{1.330592in}}%
\pgfpathlineto{\pgfqpoint{0.734851in}{1.325305in}}%
\pgfpathlineto{\pgfqpoint{0.726554in}{1.322414in}}%
\pgfpathlineto{\pgfqpoint{0.717880in}{1.322219in}}%
\pgfpathlineto{\pgfqpoint{0.709409in}{1.324408in}}%
\pgfpathlineto{\pgfqpoint{0.699545in}{1.329602in}}%
\pgfpathlineto{\pgfqpoint{0.688891in}{1.338201in}}%
\pgfpathlineto{\pgfqpoint{0.677905in}{1.350247in}}%
\pgfpathlineto{\pgfqpoint{0.666884in}{1.365646in}}%
\pgfpathlineto{\pgfqpoint{0.654911in}{1.386416in}}%
\pgfpathlineto{\pgfqpoint{0.642573in}{1.412729in}}%
\pgfpathlineto{\pgfqpoint{0.630327in}{1.444628in}}%
\pgfpathlineto{\pgfqpoint{0.618503in}{1.482080in}}%
\pgfpathlineto{\pgfqpoint{0.608612in}{1.520256in}}%
\pgfpathlineto{\pgfqpoint{0.590202in}{1.612445in}}%
\pgfpathlineto{\pgfqpoint{0.581847in}{1.668884in}}%
\pgfpathlineto{\pgfqpoint{0.573137in}{1.740376in}}%
\pgfpathlineto{\pgfqpoint{0.567061in}{1.807213in}}%
\pgfpathlineto{\pgfqpoint{0.560531in}{1.896509in}}%
\pgfpathlineto{\pgfqpoint{0.555525in}{1.995910in}}%
\pgfpathlineto{\pgfqpoint{0.552563in}{2.097908in}}%
\pgfpathlineto{\pgfqpoint{0.551525in}{2.204935in}}%
\pgfpathlineto{\pgfqpoint{0.552727in}{2.309470in}}%
\pgfpathlineto{\pgfqpoint{0.556011in}{2.403981in}}%
\pgfpathlineto{\pgfqpoint{0.560952in}{2.483430in}}%
\pgfpathlineto{\pgfqpoint{0.567303in}{2.550240in}}%
\pgfpathlineto{\pgfqpoint{0.574928in}{2.606817in}}%
\pgfpathlineto{\pgfqpoint{0.582987in}{2.650657in}}%
\pgfpathlineto{\pgfqpoint{0.592756in}{2.691452in}}%
\pgfpathlineto{\pgfqpoint{0.602650in}{2.721756in}}%
\pgfpathlineto{\pgfqpoint{0.612984in}{2.746441in}}%
\pgfpathlineto{\pgfqpoint{0.624292in}{2.767692in}}%
\pgfpathlineto{\pgfqpoint{0.636231in}{2.785432in}}%
\pgfpathlineto{\pgfqpoint{0.649892in}{2.801461in}}%
\pgfpathlineto{\pgfqpoint{0.663386in}{2.814020in}}%
\pgfpathlineto{\pgfqpoint{0.679842in}{2.826135in}}%
\pgfpathlineto{\pgfqpoint{0.697326in}{2.836197in}}%
\pgfpathlineto{\pgfqpoint{0.715574in}{2.844285in}}%
\pgfpathlineto{\pgfqpoint{0.738439in}{2.852335in}}%
\pgfpathlineto{\pgfqpoint{0.765983in}{2.859639in}}%
\pgfpathlineto{\pgfqpoint{0.800300in}{2.866256in}}%
\pgfpathlineto{\pgfqpoint{0.841340in}{2.871832in}}%
\pgfpathlineto{\pgfqpoint{0.895547in}{2.876803in}}%
\pgfpathlineto{\pgfqpoint{0.969413in}{2.881069in}}%
\pgfpathlineto{\pgfqpoint{1.071608in}{2.884501in}}%
\pgfpathlineto{\pgfqpoint{1.219512in}{2.887074in}}%
\pgfpathlineto{\pgfqpoint{1.471844in}{2.889091in}}%
\pgfpathlineto{\pgfqpoint{1.956941in}{2.890384in}}%
\pgfpathlineto{\pgfqpoint{3.096814in}{2.890781in}}%
\pgfpathlineto{\pgfqpoint{3.995224in}{2.889388in}}%
\pgfpathlineto{\pgfqpoint{4.275833in}{2.887011in}}%
\pgfpathlineto{\pgfqpoint{4.412847in}{2.883743in}}%
\pgfpathlineto{\pgfqpoint{4.491081in}{2.879810in}}%
\pgfpathlineto{\pgfqpoint{4.543127in}{2.875163in}}%
\pgfpathlineto{\pgfqpoint{4.579810in}{2.869841in}}%
\pgfpathlineto{\pgfqpoint{4.607580in}{2.863763in}}%
\pgfpathlineto{\pgfqpoint{4.630623in}{2.856424in}}%
\pgfpathlineto{\pgfqpoint{4.648833in}{2.848228in}}%
\pgfpathlineto{\pgfqpoint{4.664136in}{2.838773in}}%
\pgfpathlineto{\pgfqpoint{4.676470in}{2.828576in}}%
\pgfpathlineto{\pgfqpoint{4.687502in}{2.816585in}}%
\pgfpathlineto{\pgfqpoint{4.697051in}{2.803027in}}%
\pgfpathlineto{\pgfqpoint{4.706194in}{2.786098in}}%
\pgfpathlineto{\pgfqpoint{4.714508in}{2.765827in}}%
\pgfpathlineto{\pgfqpoint{4.722462in}{2.740013in}}%
\pgfpathlineto{\pgfqpoint{4.729577in}{2.708703in}}%
\pgfpathlineto{\pgfqpoint{4.736162in}{2.669601in}}%
\pgfpathlineto{\pgfqpoint{4.742419in}{2.617826in}}%
\pgfpathlineto{\pgfqpoint{4.747859in}{2.553410in}}%
\pgfpathlineto{\pgfqpoint{4.752661in}{2.468958in}}%
\pgfpathlineto{\pgfqpoint{4.756610in}{2.359528in}}%
\pgfpathlineto{\pgfqpoint{4.759416in}{2.217681in}}%
\pgfpathlineto{\pgfqpoint{4.760596in}{2.043444in}}%
\pgfpathlineto{\pgfqpoint{4.759662in}{1.851779in}}%
\pgfpathlineto{\pgfqpoint{4.756587in}{1.667613in}}%
\pgfpathlineto{\pgfqpoint{4.751596in}{1.503428in}}%
\pgfpathlineto{\pgfqpoint{4.745410in}{1.374185in}}%
\pgfpathlineto{\pgfqpoint{4.738113in}{1.267479in}}%
\pgfpathlineto{\pgfqpoint{4.729621in}{1.175896in}}%
\pgfpathlineto{\pgfqpoint{4.720762in}{1.104428in}}%
\pgfpathlineto{\pgfqpoint{4.711045in}{1.043204in}}%
\pgfpathlineto{\pgfqpoint{4.700364in}{0.989829in}}%
\pgfpathlineto{\pgfqpoint{4.689055in}{0.944345in}}%
\pgfpathlineto{\pgfqpoint{4.676881in}{0.904394in}}%
\pgfpathlineto{\pgfqpoint{4.676095in}{0.902073in}}%
\pgfpathlineto{\pgfqpoint{4.676095in}{0.902073in}}%
\pgfusepath{stroke}%
\end{pgfscope}%
\begin{pgfscope}%
\pgfpathrectangle{\pgfqpoint{0.448634in}{0.402556in}}{\pgfqpoint{4.350661in}{2.489204in}} %
\pgfusepath{clip}%
\pgfsetrectcap%
\pgfsetroundjoin%
\pgfsetlinewidth{1.003750pt}%
\definecolor{currentstroke}{rgb}{0.172549,0.627451,0.172549}%
\pgfsetstrokecolor{currentstroke}%
\pgfsetdash{}{0pt}%
\pgfpathmoveto{\pgfqpoint{2.795520in}{1.982745in}}%
\pgfpathlineto{\pgfqpoint{2.781780in}{1.874357in}}%
\pgfpathlineto{\pgfqpoint{2.769351in}{1.758234in}}%
\pgfpathlineto{\pgfqpoint{2.758095in}{1.631942in}}%
\pgfpathlineto{\pgfqpoint{2.747786in}{1.490551in}}%
\pgfpathlineto{\pgfqpoint{2.738644in}{1.334082in}}%
\pgfpathlineto{\pgfqpoint{2.730580in}{1.157591in}}%
\pgfpathlineto{\pgfqpoint{2.723334in}{0.948663in}}%
\pgfpathlineto{\pgfqpoint{2.709783in}{0.530788in}}%
\pgfpathlineto{\pgfqpoint{2.705868in}{0.488716in}}%
\pgfpathlineto{\pgfqpoint{2.701769in}{0.464281in}}%
\pgfpathlineto{\pgfqpoint{2.697021in}{0.447744in}}%
\pgfpathlineto{\pgfqpoint{2.691859in}{0.436812in}}%
\pgfpathlineto{\pgfqpoint{2.686245in}{0.429229in}}%
\pgfpathlineto{\pgfqpoint{2.679348in}{0.423188in}}%
\pgfpathlineto{\pgfqpoint{2.669540in}{0.417856in}}%
\pgfpathlineto{\pgfqpoint{2.656987in}{0.413810in}}%
\pgfpathlineto{\pgfqpoint{2.637654in}{0.410337in}}%
\pgfpathlineto{\pgfqpoint{2.607297in}{0.407617in}}%
\pgfpathlineto{\pgfqpoint{2.555121in}{0.405574in}}%
\pgfpathlineto{\pgfqpoint{2.450714in}{0.404139in}}%
\pgfpathlineto{\pgfqpoint{2.176624in}{0.403275in}}%
\pgfpathlineto{\pgfqpoint{1.130290in}{0.402953in}}%
\pgfpathlineto{\pgfqpoint{0.516849in}{0.404175in}}%
\pgfpathlineto{\pgfqpoint{0.466848in}{0.405970in}}%
\pgfpathlineto{\pgfqpoint{0.456130in}{0.407931in}}%
\pgfpathlineto{\pgfqpoint{0.452340in}{0.410303in}}%
\pgfpathlineto{\pgfqpoint{0.450346in}{0.414662in}}%
\pgfpathlineto{\pgfqpoint{0.449266in}{0.424524in}}%
\pgfpathlineto{\pgfqpoint{0.448771in}{0.464344in}}%
\pgfpathlineto{\pgfqpoint{0.448640in}{0.850171in}}%
\pgfpathlineto{\pgfqpoint{0.448653in}{2.891318in}}%
\pgfpathlineto{\pgfqpoint{0.448653in}{2.891318in}}%
\pgfusepath{stroke}%
\end{pgfscope}%
\begin{pgfscope}%
\pgfpathrectangle{\pgfqpoint{0.448634in}{0.402556in}}{\pgfqpoint{4.350661in}{2.489204in}} %
\pgfusepath{clip}%
\pgfsetrectcap%
\pgfsetroundjoin%
\pgfsetlinewidth{1.003750pt}%
\definecolor{currentstroke}{rgb}{0.172549,0.627451,0.172549}%
\pgfsetstrokecolor{currentstroke}%
\pgfsetdash{}{0pt}%
\pgfpathmoveto{\pgfqpoint{3.428180in}{0.402586in}}%
\pgfpathlineto{\pgfqpoint{2.782112in}{0.403699in}}%
\pgfpathlineto{\pgfqpoint{2.753896in}{0.405668in}}%
\pgfpathlineto{\pgfqpoint{2.743318in}{0.408435in}}%
\pgfpathlineto{\pgfqpoint{2.737706in}{0.412179in}}%
\pgfpathlineto{\pgfqpoint{2.733658in}{0.417987in}}%
\pgfpathlineto{\pgfqpoint{2.730641in}{0.427300in}}%
\pgfpathlineto{\pgfqpoint{2.728382in}{0.441998in}}%
\pgfpathlineto{\pgfqpoint{2.726540in}{0.471788in}}%
\pgfpathlineto{\pgfqpoint{2.725214in}{0.533997in}}%
\pgfpathlineto{\pgfqpoint{2.725168in}{0.655966in}}%
\pgfpathlineto{\pgfqpoint{2.727376in}{0.832680in}}%
\pgfpathlineto{\pgfqpoint{2.732258in}{1.041697in}}%
\pgfpathlineto{\pgfqpoint{2.738850in}{1.223250in}}%
\pgfpathlineto{\pgfqpoint{2.747077in}{1.389759in}}%
\pgfpathlineto{\pgfqpoint{2.756607in}{1.538711in}}%
\pgfpathlineto{\pgfqpoint{2.768953in}{1.694881in}}%
\pgfpathlineto{\pgfqpoint{2.781226in}{1.816038in}}%
\pgfpathlineto{\pgfqpoint{2.794399in}{1.924518in}}%
\pgfpathlineto{\pgfqpoint{2.812734in}{2.054716in}}%
\pgfpathlineto{\pgfqpoint{2.828771in}{2.147506in}}%
\pgfpathlineto{\pgfqpoint{2.847379in}{2.242218in}}%
\pgfpathlineto{\pgfqpoint{2.895814in}{2.479693in}}%
\pgfpathlineto{\pgfqpoint{2.900200in}{2.516683in}}%
\pgfpathlineto{\pgfqpoint{2.901342in}{2.544023in}}%
\pgfpathlineto{\pgfqpoint{2.900288in}{2.566382in}}%
\pgfpathlineto{\pgfqpoint{2.897332in}{2.585993in}}%
\pgfpathlineto{\pgfqpoint{2.892834in}{2.602628in}}%
\pgfpathlineto{\pgfqpoint{2.886392in}{2.618400in}}%
\pgfpathlineto{\pgfqpoint{2.878057in}{2.632964in}}%
\pgfpathlineto{\pgfqpoint{2.868064in}{2.646095in}}%
\pgfpathlineto{\pgfqpoint{2.855049in}{2.659296in}}%
\pgfpathlineto{\pgfqpoint{2.840800in}{2.670713in}}%
\pgfpathlineto{\pgfqpoint{2.821822in}{2.682857in}}%
\pgfpathlineto{\pgfqpoint{2.799980in}{2.694023in}}%
\pgfpathlineto{\pgfqpoint{2.773366in}{2.704941in}}%
\pgfpathlineto{\pgfqpoint{2.742012in}{2.715264in}}%
\pgfpathlineto{\pgfqpoint{2.705983in}{2.724783in}}%
\pgfpathlineto{\pgfqpoint{2.663201in}{2.733808in}}%
\pgfpathlineto{\pgfqpoint{2.611535in}{2.742377in}}%
\pgfpathlineto{\pgfqpoint{2.551002in}{2.750088in}}%
\pgfpathlineto{\pgfqpoint{2.481632in}{2.756681in}}%
\pgfpathlineto{\pgfqpoint{2.399113in}{2.762199in}}%
\pgfpathlineto{\pgfqpoint{2.309985in}{2.765884in}}%
\pgfpathlineto{\pgfqpoint{2.188184in}{2.768095in}}%
\pgfpathlineto{\pgfqpoint{2.081595in}{2.767618in}}%
\pgfpathlineto{\pgfqpoint{1.968506in}{2.764839in}}%
\pgfpathlineto{\pgfqpoint{1.864180in}{2.759917in}}%
\pgfpathlineto{\pgfqpoint{1.757787in}{2.752592in}}%
\pgfpathlineto{\pgfqpoint{1.671087in}{2.744169in}}%
\pgfpathlineto{\pgfqpoint{1.591076in}{2.734191in}}%
\pgfpathlineto{\pgfqpoint{1.502690in}{2.720715in}}%
\pgfpathlineto{\pgfqpoint{1.427655in}{2.706083in}}%
\pgfpathlineto{\pgfqpoint{1.372350in}{2.692544in}}%
\pgfpathlineto{\pgfqpoint{1.321734in}{2.677921in}}%
\pgfpathlineto{\pgfqpoint{1.273765in}{2.661664in}}%
\pgfpathlineto{\pgfqpoint{1.230567in}{2.644672in}}%
\pgfpathlineto{\pgfqpoint{1.192197in}{2.627106in}}%
\pgfpathlineto{\pgfqpoint{1.156620in}{2.608403in}}%
\pgfpathlineto{\pgfqpoint{1.123890in}{2.588716in}}%
\pgfpathlineto{\pgfqpoint{1.095884in}{2.569568in}}%
\pgfpathlineto{\pgfqpoint{1.063936in}{2.543700in}}%
\pgfpathlineto{\pgfqpoint{1.038217in}{2.520732in}}%
\pgfpathlineto{\pgfqpoint{1.013766in}{2.496015in}}%
\pgfpathlineto{\pgfqpoint{0.990704in}{2.469610in}}%
\pgfpathlineto{\pgfqpoint{0.969124in}{2.441611in}}%
\pgfpathlineto{\pgfqpoint{0.949083in}{2.412153in}}%
\pgfpathlineto{\pgfqpoint{0.930604in}{2.381386in}}%
\pgfpathlineto{\pgfqpoint{0.906556in}{2.334051in}}%
\pgfpathlineto{\pgfqpoint{0.889925in}{2.296261in}}%
\pgfpathlineto{\pgfqpoint{0.874241in}{2.255213in}}%
\pgfpathlineto{\pgfqpoint{0.859668in}{2.210961in}}%
\pgfpathlineto{\pgfqpoint{0.846986in}{2.165953in}}%
\pgfpathlineto{\pgfqpoint{0.839633in}{2.134714in}}%
\pgfpathlineto{\pgfqpoint{0.828238in}{2.081531in}}%
\pgfpathlineto{\pgfqpoint{0.817867in}{2.022985in}}%
\pgfpathlineto{\pgfqpoint{0.810784in}{1.971351in}}%
\pgfpathlineto{\pgfqpoint{0.802846in}{1.902252in}}%
\pgfpathlineto{\pgfqpoint{0.796555in}{1.827927in}}%
\pgfpathlineto{\pgfqpoint{0.791696in}{1.743480in}}%
\pgfpathlineto{\pgfqpoint{0.787774in}{1.621594in}}%
\pgfpathlineto{\pgfqpoint{0.785408in}{1.522063in}}%
\pgfpathlineto{\pgfqpoint{0.785408in}{1.522063in}}%
\pgfusepath{stroke}%
\end{pgfscope}%
\begin{pgfscope}%
\pgfpathrectangle{\pgfqpoint{0.448634in}{0.402556in}}{\pgfqpoint{4.350661in}{2.489204in}} %
\pgfusepath{clip}%
\pgfsetrectcap%
\pgfsetroundjoin%
\pgfsetlinewidth{1.003750pt}%
\definecolor{currentstroke}{rgb}{0.172549,0.627451,0.172549}%
\pgfsetstrokecolor{currentstroke}%
\pgfsetdash{}{0pt}%
\pgfpathmoveto{\pgfqpoint{2.028735in}{0.425754in}}%
\pgfpathlineto{\pgfqpoint{1.878677in}{0.421879in}}%
\pgfpathlineto{\pgfqpoint{1.676387in}{0.418997in}}%
\pgfpathlineto{\pgfqpoint{1.413176in}{0.417558in}}%
\pgfpathlineto{\pgfqpoint{1.134735in}{0.418204in}}%
\pgfpathlineto{\pgfqpoint{0.921565in}{0.420769in}}%
\pgfpathlineto{\pgfqpoint{0.782384in}{0.424523in}}%
\pgfpathlineto{\pgfqpoint{0.693283in}{0.428974in}}%
\pgfpathlineto{\pgfqpoint{0.632541in}{0.434091in}}%
\pgfpathlineto{\pgfqpoint{0.591492in}{0.439564in}}%
\pgfpathlineto{\pgfqpoint{0.561503in}{0.445595in}}%
\pgfpathlineto{\pgfqpoint{0.538349in}{0.452466in}}%
\pgfpathlineto{\pgfqpoint{0.522042in}{0.459394in}}%
\pgfpathlineto{\pgfqpoint{0.508540in}{0.467420in}}%
\pgfpathlineto{\pgfqpoint{0.497973in}{0.476161in}}%
\pgfpathlineto{\pgfqpoint{0.488790in}{0.486749in}}%
\pgfpathlineto{\pgfqpoint{0.481284in}{0.498948in}}%
\pgfpathlineto{\pgfqpoint{0.474590in}{0.514580in}}%
\pgfpathlineto{\pgfqpoint{0.469106in}{0.533467in}}%
\pgfpathlineto{\pgfqpoint{0.464439in}{0.557771in}}%
\pgfpathlineto{\pgfqpoint{0.460297in}{0.592289in}}%
\pgfpathlineto{\pgfqpoint{0.456855in}{0.641912in}}%
\pgfpathlineto{\pgfqpoint{0.454122in}{0.716520in}}%
\pgfpathlineto{\pgfqpoint{0.451978in}{0.843444in}}%
\pgfpathlineto{\pgfqpoint{0.450459in}{1.087380in}}%
\pgfpathlineto{\pgfqpoint{0.449596in}{1.657406in}}%
\pgfpathlineto{\pgfqpoint{0.450150in}{2.687936in}}%
\pgfpathlineto{\pgfqpoint{0.451781in}{2.839761in}}%
\pgfpathlineto{\pgfqpoint{0.453975in}{2.872003in}}%
\pgfpathlineto{\pgfqpoint{0.456339in}{2.881553in}}%
\pgfpathlineto{\pgfqpoint{0.458888in}{2.885549in}}%
\pgfpathlineto{\pgfqpoint{0.462554in}{2.888171in}}%
\pgfpathlineto{\pgfqpoint{0.471046in}{2.890205in}}%
\pgfpathlineto{\pgfqpoint{0.490597in}{2.891263in}}%
\pgfpathlineto{\pgfqpoint{0.564556in}{2.891692in}}%
\pgfpathlineto{\pgfqpoint{1.569559in}{2.891759in}}%
\pgfpathlineto{\pgfqpoint{4.784679in}{2.890785in}}%
\pgfpathlineto{\pgfqpoint{4.791005in}{2.889098in}}%
\pgfpathlineto{\pgfqpoint{4.793910in}{2.885555in}}%
\pgfpathlineto{\pgfqpoint{4.795579in}{2.878366in}}%
\pgfpathlineto{\pgfqpoint{4.796850in}{2.858513in}}%
\pgfpathlineto{\pgfqpoint{4.796850in}{2.858513in}}%
\pgfusepath{stroke}%
\end{pgfscope}%
\begin{pgfscope}%
\pgfpathrectangle{\pgfqpoint{0.448634in}{0.402556in}}{\pgfqpoint{4.350661in}{2.489204in}} %
\pgfusepath{clip}%
\pgfsetrectcap%
\pgfsetroundjoin%
\pgfsetlinewidth{1.003750pt}%
\definecolor{currentstroke}{rgb}{0.839216,0.152941,0.156863}%
\pgfsetstrokecolor{currentstroke}%
\pgfsetdash{}{0pt}%
\pgfpathmoveto{\pgfqpoint{1.127319in}{2.572074in}}%
\pgfpathlineto{\pgfqpoint{1.159575in}{2.592758in}}%
\pgfpathlineto{\pgfqpoint{1.192763in}{2.611414in}}%
\pgfpathlineto{\pgfqpoint{1.228726in}{2.629126in}}%
\pgfpathlineto{\pgfqpoint{1.267413in}{2.645758in}}%
\pgfpathlineto{\pgfqpoint{1.310846in}{2.661946in}}%
\pgfpathlineto{\pgfqpoint{1.356920in}{2.676740in}}%
\pgfpathlineto{\pgfqpoint{1.407680in}{2.690702in}}%
\pgfpathlineto{\pgfqpoint{1.463094in}{2.703640in}}%
\pgfpathlineto{\pgfqpoint{1.525273in}{2.715813in}}%
\pgfpathlineto{\pgfqpoint{1.594199in}{2.726937in}}%
\pgfpathlineto{\pgfqpoint{1.669843in}{2.736808in}}%
\pgfpathlineto{\pgfqpoint{1.752172in}{2.745271in}}%
\pgfpathlineto{\pgfqpoint{1.843325in}{2.752344in}}%
\pgfpathlineto{\pgfqpoint{1.941103in}{2.757656in}}%
\pgfpathlineto{\pgfqpoint{2.043301in}{2.760987in}}%
\pgfpathlineto{\pgfqpoint{2.147710in}{2.762199in}}%
\pgfpathlineto{\pgfqpoint{2.249945in}{2.761215in}}%
\pgfpathlineto{\pgfqpoint{2.345620in}{2.758145in}}%
\pgfpathlineto{\pgfqpoint{2.432525in}{2.753210in}}%
\pgfpathlineto{\pgfqpoint{2.508451in}{2.746766in}}%
\pgfpathlineto{\pgfqpoint{2.573368in}{2.739156in}}%
\pgfpathlineto{\pgfqpoint{2.629410in}{2.730451in}}%
\pgfpathlineto{\pgfqpoint{2.676543in}{2.720985in}}%
\pgfpathlineto{\pgfqpoint{2.716874in}{2.710666in}}%
\pgfpathlineto{\pgfqpoint{2.750366in}{2.699848in}}%
\pgfpathlineto{\pgfqpoint{2.779059in}{2.688192in}}%
\pgfpathlineto{\pgfqpoint{2.802882in}{2.676004in}}%
\pgfpathlineto{\pgfqpoint{2.821842in}{2.663819in}}%
\pgfpathlineto{\pgfqpoint{2.837815in}{2.650886in}}%
\pgfpathlineto{\pgfqpoint{2.850736in}{2.637564in}}%
\pgfpathlineto{\pgfqpoint{2.860694in}{2.624398in}}%
\pgfpathlineto{\pgfqpoint{2.869084in}{2.609873in}}%
\pgfpathlineto{\pgfqpoint{2.875698in}{2.594192in}}%
\pgfpathlineto{\pgfqpoint{2.881035in}{2.575255in}}%
\pgfpathlineto{\pgfqpoint{2.884200in}{2.555685in}}%
\pgfpathlineto{\pgfqpoint{2.885619in}{2.533351in}}%
\pgfpathlineto{\pgfqpoint{2.885038in}{2.505987in}}%
\pgfpathlineto{\pgfqpoint{2.882112in}{2.473807in}}%
\pgfpathlineto{\pgfqpoint{2.875657in}{2.429620in}}%
\pgfpathlineto{\pgfqpoint{2.863489in}{2.363873in}}%
\pgfpathlineto{\pgfqpoint{2.821102in}{2.142619in}}%
\pgfpathlineto{\pgfqpoint{2.804859in}{2.042271in}}%
\pgfpathlineto{\pgfqpoint{2.790421in}{1.939040in}}%
\pgfpathlineto{\pgfqpoint{2.777207in}{1.828054in}}%
\pgfpathlineto{\pgfqpoint{2.765338in}{1.709349in}}%
\pgfpathlineto{\pgfqpoint{2.754471in}{1.578010in}}%
\pgfpathlineto{\pgfqpoint{2.744640in}{1.431580in}}%
\pgfpathlineto{\pgfqpoint{2.735914in}{1.267598in}}%
\pgfpathlineto{\pgfqpoint{2.728277in}{1.081114in}}%
\pgfpathlineto{\pgfqpoint{2.721436in}{0.857223in}}%
\pgfpathlineto{\pgfqpoint{2.711961in}{0.541290in}}%
\pgfpathlineto{\pgfqpoint{2.708250in}{0.491694in}}%
\pgfpathlineto{\pgfqpoint{2.703951in}{0.462246in}}%
\pgfpathlineto{\pgfqpoint{2.699504in}{0.445599in}}%
\pgfpathlineto{\pgfqpoint{2.694517in}{0.434563in}}%
\pgfpathlineto{\pgfqpoint{2.688941in}{0.426947in}}%
\pgfpathlineto{\pgfqpoint{2.681980in}{0.421009in}}%
\pgfpathlineto{\pgfqpoint{2.672064in}{0.415948in}}%
\pgfpathlineto{\pgfqpoint{2.659429in}{0.412247in}}%
\pgfpathlineto{\pgfqpoint{2.640044in}{0.409163in}}%
\pgfpathlineto{\pgfqpoint{2.607490in}{0.406692in}}%
\pgfpathlineto{\pgfqpoint{2.548779in}{0.404894in}}%
\pgfpathlineto{\pgfqpoint{2.422614in}{0.403701in}}%
\pgfpathlineto{\pgfqpoint{2.026705in}{0.403016in}}%
\pgfpathlineto{\pgfqpoint{0.623617in}{0.403253in}}%
\pgfpathlineto{\pgfqpoint{0.477880in}{0.404742in}}%
\pgfpathlineto{\pgfqpoint{0.458367in}{0.406382in}}%
\pgfpathlineto{\pgfqpoint{0.452304in}{0.408938in}}%
\pgfpathlineto{\pgfqpoint{0.450213in}{0.413215in}}%
\pgfpathlineto{\pgfqpoint{0.449165in}{0.423081in}}%
\pgfpathlineto{\pgfqpoint{0.448735in}{0.465392in}}%
\pgfpathlineto{\pgfqpoint{0.448637in}{0.983146in}}%
\pgfpathlineto{\pgfqpoint{0.448652in}{2.889877in}}%
\pgfpathlineto{\pgfqpoint{0.448652in}{2.889877in}}%
\pgfusepath{stroke}%
\end{pgfscope}%
\begin{pgfscope}%
\pgfpathrectangle{\pgfqpoint{0.448634in}{0.402556in}}{\pgfqpoint{4.350661in}{2.489204in}} %
\pgfusepath{clip}%
\pgfsetrectcap%
\pgfsetroundjoin%
\pgfsetlinewidth{1.003750pt}%
\definecolor{currentstroke}{rgb}{0.839216,0.152941,0.156863}%
\pgfsetstrokecolor{currentstroke}%
\pgfsetdash{}{0pt}%
\pgfpathmoveto{\pgfqpoint{0.448634in}{2.896245in}}%
\pgfpathlineto{\pgfqpoint{0.448593in}{0.407043in}}%
\pgfpathlineto{\pgfqpoint{0.448593in}{0.407043in}}%
\pgfusepath{stroke}%
\end{pgfscope}%
\begin{pgfscope}%
\pgfpathrectangle{\pgfqpoint{0.448634in}{0.402556in}}{\pgfqpoint{4.350661in}{2.489204in}} %
\pgfusepath{clip}%
\pgfsetrectcap%
\pgfsetroundjoin%
\pgfsetlinewidth{1.003750pt}%
\definecolor{currentstroke}{rgb}{0.839216,0.152941,0.156863}%
\pgfsetstrokecolor{currentstroke}%
\pgfsetdash{}{0pt}%
\pgfpathmoveto{\pgfqpoint{0.576862in}{1.760791in}}%
\pgfpathlineto{\pgfqpoint{0.569402in}{1.839984in}}%
\pgfpathlineto{\pgfqpoint{0.563215in}{1.929312in}}%
\pgfpathlineto{\pgfqpoint{0.558598in}{2.028737in}}%
\pgfpathlineto{\pgfqpoint{0.555990in}{2.133239in}}%
\pgfpathlineto{\pgfqpoint{0.555570in}{2.237782in}}%
\pgfpathlineto{\pgfqpoint{0.557375in}{2.337325in}}%
\pgfpathlineto{\pgfqpoint{0.561099in}{2.424340in}}%
\pgfpathlineto{\pgfqpoint{0.566406in}{2.498765in}}%
\pgfpathlineto{\pgfqpoint{0.572911in}{2.560544in}}%
\pgfpathlineto{\pgfqpoint{0.580459in}{2.612093in}}%
\pgfpathlineto{\pgfqpoint{0.589086in}{2.655790in}}%
\pgfpathlineto{\pgfqpoint{0.598405in}{2.691564in}}%
\pgfpathlineto{\pgfqpoint{0.608611in}{2.721732in}}%
\pgfpathlineto{\pgfqpoint{0.619236in}{2.746254in}}%
\pgfpathlineto{\pgfqpoint{0.630810in}{2.767317in}}%
\pgfpathlineto{\pgfqpoint{0.642967in}{2.784863in}}%
\pgfpathlineto{\pgfqpoint{0.656804in}{2.800694in}}%
\pgfpathlineto{\pgfqpoint{0.672185in}{2.814532in}}%
\pgfpathlineto{\pgfqpoint{0.688840in}{2.826287in}}%
\pgfpathlineto{\pgfqpoint{0.706448in}{2.836064in}}%
\pgfpathlineto{\pgfqpoint{0.726789in}{2.844866in}}%
\pgfpathlineto{\pgfqpoint{0.751851in}{2.853195in}}%
\pgfpathlineto{\pgfqpoint{0.781616in}{2.860541in}}%
\pgfpathlineto{\pgfqpoint{0.818153in}{2.867049in}}%
\pgfpathlineto{\pgfqpoint{0.863566in}{2.872681in}}%
\pgfpathlineto{\pgfqpoint{0.922145in}{2.877515in}}%
\pgfpathlineto{\pgfqpoint{1.000375in}{2.881565in}}%
\pgfpathlineto{\pgfqpoint{1.111278in}{2.884880in}}%
\pgfpathlineto{\pgfqpoint{1.274413in}{2.887367in}}%
\pgfpathlineto{\pgfqpoint{1.552850in}{2.889263in}}%
\pgfpathlineto{\pgfqpoint{2.107558in}{2.890456in}}%
\pgfpathlineto{\pgfqpoint{3.343145in}{2.890573in}}%
\pgfpathlineto{\pgfqpoint{4.043600in}{2.888941in}}%
\pgfpathlineto{\pgfqpoint{4.289401in}{2.886404in}}%
\pgfpathlineto{\pgfqpoint{4.413359in}{2.883093in}}%
\pgfpathlineto{\pgfqpoint{4.489409in}{2.878997in}}%
\pgfpathlineto{\pgfqpoint{4.541436in}{2.874081in}}%
\pgfpathlineto{\pgfqpoint{4.578085in}{2.868471in}}%
\pgfpathlineto{\pgfqpoint{4.605803in}{2.862094in}}%
\pgfpathlineto{\pgfqpoint{4.626710in}{2.855248in}}%
\pgfpathlineto{\pgfqpoint{4.644911in}{2.847023in}}%
\pgfpathlineto{\pgfqpoint{4.660227in}{2.837595in}}%
\pgfpathlineto{\pgfqpoint{4.672610in}{2.827476in}}%
\pgfpathlineto{\pgfqpoint{4.683739in}{2.815601in}}%
\pgfpathlineto{\pgfqpoint{4.693395in}{2.802144in}}%
\pgfpathlineto{\pgfqpoint{4.702731in}{2.785353in}}%
\pgfpathlineto{\pgfqpoint{4.711269in}{2.765205in}}%
\pgfpathlineto{\pgfqpoint{4.719476in}{2.739495in}}%
\pgfpathlineto{\pgfqpoint{4.726288in}{2.710670in}}%
\pgfpathlineto{\pgfqpoint{4.733255in}{2.671655in}}%
\pgfpathlineto{\pgfqpoint{4.739600in}{2.622408in}}%
\pgfpathlineto{\pgfqpoint{4.745233in}{2.560516in}}%
\pgfpathlineto{\pgfqpoint{4.750161in}{2.481064in}}%
\pgfpathlineto{\pgfqpoint{4.754365in}{2.376631in}}%
\pgfpathlineto{\pgfqpoint{4.757441in}{2.242261in}}%
\pgfpathlineto{\pgfqpoint{4.758975in}{2.075495in}}%
\pgfpathlineto{\pgfqpoint{4.758445in}{1.888808in}}%
\pgfpathlineto{\pgfqpoint{4.755755in}{1.707123in}}%
\pgfpathlineto{\pgfqpoint{4.750924in}{1.532970in}}%
\pgfpathlineto{\pgfqpoint{4.744784in}{1.398739in}}%
\pgfpathlineto{\pgfqpoint{4.737574in}{1.289528in}}%
\pgfpathlineto{\pgfqpoint{4.728712in}{1.190482in}}%
\pgfpathlineto{\pgfqpoint{4.719651in}{1.116534in}}%
\pgfpathlineto{\pgfqpoint{4.710034in}{1.055289in}}%
\pgfpathlineto{\pgfqpoint{4.699502in}{1.001874in}}%
\pgfpathlineto{\pgfqpoint{4.689039in}{0.958703in}}%
\pgfpathlineto{\pgfqpoint{4.677218in}{0.918613in}}%
\pgfpathlineto{\pgfqpoint{4.664033in}{0.881762in}}%
\pgfpathlineto{\pgfqpoint{4.650582in}{0.850504in}}%
\pgfpathlineto{\pgfqpoint{4.636302in}{0.822582in}}%
\pgfpathlineto{\pgfqpoint{4.620206in}{0.795987in}}%
\pgfpathlineto{\pgfqpoint{4.603639in}{0.772913in}}%
\pgfpathlineto{\pgfqpoint{4.585488in}{0.751458in}}%
\pgfpathlineto{\pgfqpoint{4.565874in}{0.731760in}}%
\pgfpathlineto{\pgfqpoint{4.544964in}{0.713890in}}%
\pgfpathlineto{\pgfqpoint{4.522958in}{0.697834in}}%
\pgfpathlineto{\pgfqpoint{4.496157in}{0.681300in}}%
\pgfpathlineto{\pgfqpoint{4.470398in}{0.667963in}}%
\pgfpathlineto{\pgfqpoint{4.439962in}{0.654519in}}%
\pgfpathlineto{\pgfqpoint{4.406842in}{0.642291in}}%
\pgfpathlineto{\pgfqpoint{4.369010in}{0.630757in}}%
\pgfpathlineto{\pgfqpoint{4.326490in}{0.620235in}}%
\pgfpathlineto{\pgfqpoint{4.279328in}{0.610957in}}%
\pgfpathlineto{\pgfqpoint{4.227577in}{0.603093in}}%
\pgfpathlineto{\pgfqpoint{4.173451in}{0.597071in}}%
\pgfpathlineto{\pgfqpoint{4.110512in}{0.592211in}}%
\pgfpathlineto{\pgfqpoint{4.047472in}{0.589546in}}%
\pgfpathlineto{\pgfqpoint{3.977868in}{0.588633in}}%
\pgfpathlineto{\pgfqpoint{3.906094in}{0.589943in}}%
\pgfpathlineto{\pgfqpoint{3.834378in}{0.593506in}}%
\pgfpathlineto{\pgfqpoint{3.767121in}{0.599077in}}%
\pgfpathlineto{\pgfqpoint{3.704366in}{0.606402in}}%
\pgfpathlineto{\pgfqpoint{3.678517in}{0.610514in}}%
\pgfpathlineto{\pgfqpoint{3.620438in}{0.620504in}}%
\pgfpathlineto{\pgfqpoint{3.586319in}{0.628210in}}%
\pgfpathlineto{\pgfqpoint{3.495240in}{0.652428in}}%
\pgfpathlineto{\pgfqpoint{3.451527in}{0.667584in}}%
\pgfpathlineto{\pgfqpoint{3.408538in}{0.685220in}}%
\pgfpathlineto{\pgfqpoint{3.374593in}{0.702000in}}%
\pgfpathlineto{\pgfqpoint{3.345406in}{0.718682in}}%
\pgfpathlineto{\pgfqpoint{3.315235in}{0.738520in}}%
\pgfpathlineto{\pgfqpoint{3.288127in}{0.759290in}}%
\pgfpathlineto{\pgfqpoint{3.264004in}{0.780551in}}%
\pgfpathlineto{\pgfqpoint{3.241208in}{0.803648in}}%
\pgfpathlineto{\pgfqpoint{3.219894in}{0.828530in}}%
\pgfpathlineto{\pgfqpoint{3.200189in}{0.855091in}}%
\pgfpathlineto{\pgfqpoint{3.182177in}{0.883183in}}%
\pgfpathlineto{\pgfqpoint{3.165905in}{0.912633in}}%
\pgfpathlineto{\pgfqpoint{3.150350in}{0.945448in}}%
\pgfpathlineto{\pgfqpoint{3.136682in}{0.979345in}}%
\pgfpathlineto{\pgfqpoint{3.124072in}{1.016460in}}%
\pgfpathlineto{\pgfqpoint{3.112834in}{1.056769in}}%
\pgfpathlineto{\pgfqpoint{3.103045in}{1.100146in}}%
\pgfpathlineto{\pgfqpoint{3.095343in}{1.144071in}}%
\pgfpathlineto{\pgfqpoint{3.089208in}{1.190837in}}%
\pgfpathlineto{\pgfqpoint{3.084594in}{1.242838in}}%
\pgfpathlineto{\pgfqpoint{3.082136in}{1.295031in}}%
\pgfpathlineto{\pgfqpoint{3.081686in}{1.349787in}}%
\pgfpathlineto{\pgfqpoint{3.083450in}{1.406998in}}%
\pgfpathlineto{\pgfqpoint{3.087180in}{1.461589in}}%
\pgfpathlineto{\pgfqpoint{3.093485in}{1.520888in}}%
\pgfpathlineto{\pgfqpoint{3.101823in}{1.577334in}}%
\pgfpathlineto{\pgfqpoint{3.111930in}{1.630856in}}%
\pgfpathlineto{\pgfqpoint{3.124690in}{1.686208in}}%
\pgfpathlineto{\pgfqpoint{3.139178in}{1.738395in}}%
\pgfpathlineto{\pgfqpoint{3.155145in}{1.787366in}}%
\pgfpathlineto{\pgfqpoint{3.172353in}{1.833085in}}%
\pgfpathlineto{\pgfqpoint{3.191618in}{1.877716in}}%
\pgfpathlineto{\pgfqpoint{3.214025in}{1.923261in}}%
\pgfpathlineto{\pgfqpoint{3.236214in}{1.963157in}}%
\pgfpathlineto{\pgfqpoint{3.260177in}{2.001684in}}%
\pgfpathlineto{\pgfqpoint{3.285813in}{2.038776in}}%
\pgfpathlineto{\pgfqpoint{3.314414in}{2.076285in}}%
\pgfpathlineto{\pgfqpoint{3.348944in}{2.117711in}}%
\pgfpathlineto{\pgfqpoint{3.417133in}{2.198022in}}%
\pgfpathlineto{\pgfqpoint{3.426053in}{2.212128in}}%
\pgfpathlineto{\pgfqpoint{3.430798in}{2.223297in}}%
\pgfpathlineto{\pgfqpoint{3.432033in}{2.230604in}}%
\pgfpathlineto{\pgfqpoint{3.430772in}{2.237856in}}%
\pgfpathlineto{\pgfqpoint{3.426621in}{2.243526in}}%
\pgfpathlineto{\pgfqpoint{3.420908in}{2.247084in}}%
\pgfpathlineto{\pgfqpoint{3.412500in}{2.249583in}}%
\pgfpathlineto{\pgfqpoint{3.399499in}{2.250689in}}%
\pgfpathlineto{\pgfqpoint{3.384305in}{2.249671in}}%
\pgfpathlineto{\pgfqpoint{3.364985in}{2.246098in}}%
\pgfpathlineto{\pgfqpoint{3.341804in}{2.239343in}}%
\pgfpathlineto{\pgfqpoint{3.317109in}{2.229682in}}%
\pgfpathlineto{\pgfqpoint{3.291104in}{2.216986in}}%
\pgfpathlineto{\pgfqpoint{3.265928in}{2.202261in}}%
\pgfpathlineto{\pgfqpoint{3.239804in}{2.184361in}}%
\pgfpathlineto{\pgfqpoint{3.214775in}{2.164519in}}%
\pgfpathlineto{\pgfqpoint{3.190900in}{2.142893in}}%
\pgfpathlineto{\pgfqpoint{3.166656in}{2.117912in}}%
\pgfpathlineto{\pgfqpoint{3.143835in}{2.091233in}}%
\pgfpathlineto{\pgfqpoint{3.121079in}{2.061107in}}%
\pgfpathlineto{\pgfqpoint{3.099952in}{2.029463in}}%
\pgfpathlineto{\pgfqpoint{3.079250in}{1.994406in}}%
\pgfpathlineto{\pgfqpoint{3.059218in}{1.955915in}}%
\pgfpathlineto{\pgfqpoint{3.040058in}{1.914015in}}%
\pgfpathlineto{\pgfqpoint{3.022809in}{1.871041in}}%
\pgfpathlineto{\pgfqpoint{3.005790in}{1.822536in}}%
\pgfpathlineto{\pgfqpoint{2.990067in}{1.770819in}}%
\pgfpathlineto{\pgfqpoint{2.975708in}{1.715979in}}%
\pgfpathlineto{\pgfqpoint{2.962284in}{1.655680in}}%
\pgfpathlineto{\pgfqpoint{2.950496in}{1.592386in}}%
\pgfpathlineto{\pgfqpoint{2.940383in}{1.526185in}}%
\pgfpathlineto{\pgfqpoint{2.931745in}{1.454681in}}%
\pgfpathlineto{\pgfqpoint{2.925082in}{1.380399in}}%
\pgfpathlineto{\pgfqpoint{2.920647in}{1.305899in}}%
\pgfpathlineto{\pgfqpoint{2.918444in}{1.231270in}}%
\pgfpathlineto{\pgfqpoint{2.918545in}{1.159087in}}%
\pgfpathlineto{\pgfqpoint{2.920787in}{1.091931in}}%
\pgfpathlineto{\pgfqpoint{2.925177in}{1.027412in}}%
\pgfpathlineto{\pgfqpoint{2.931192in}{0.970580in}}%
\pgfpathlineto{\pgfqpoint{2.938760in}{0.919034in}}%
\pgfpathlineto{\pgfqpoint{2.947651in}{0.872852in}}%
\pgfpathlineto{\pgfqpoint{2.958213in}{0.829714in}}%
\pgfpathlineto{\pgfqpoint{2.969670in}{0.792114in}}%
\pgfpathlineto{\pgfqpoint{2.982463in}{0.757773in}}%
\pgfpathlineto{\pgfqpoint{2.996425in}{0.726812in}}%
\pgfpathlineto{\pgfqpoint{3.011299in}{0.699300in}}%
\pgfpathlineto{\pgfqpoint{3.026739in}{0.675225in}}%
\pgfpathlineto{\pgfqpoint{3.043828in}{0.652656in}}%
\pgfpathlineto{\pgfqpoint{3.062495in}{0.631788in}}%
\pgfpathlineto{\pgfqpoint{3.082602in}{0.612753in}}%
\pgfpathlineto{\pgfqpoint{3.103961in}{0.595592in}}%
\pgfpathlineto{\pgfqpoint{3.128268in}{0.579069in}}%
\pgfpathlineto{\pgfqpoint{3.153537in}{0.564554in}}%
\pgfpathlineto{\pgfqpoint{3.181571in}{0.550952in}}%
\pgfpathlineto{\pgfqpoint{3.214371in}{0.537647in}}%
\pgfpathlineto{\pgfqpoint{3.249846in}{0.525712in}}%
\pgfpathlineto{\pgfqpoint{3.290011in}{0.514571in}}%
\pgfpathlineto{\pgfqpoint{3.334820in}{0.504423in}}%
\pgfpathlineto{\pgfqpoint{3.386372in}{0.494999in}}%
\pgfpathlineto{\pgfqpoint{3.446798in}{0.486257in}}%
\pgfpathlineto{\pgfqpoint{3.518243in}{0.478282in}}%
\pgfpathlineto{\pgfqpoint{3.600685in}{0.471409in}}%
\pgfpathlineto{\pgfqpoint{3.696268in}{0.465713in}}%
\pgfpathlineto{\pgfqpoint{3.807144in}{0.461369in}}%
\pgfpathlineto{\pgfqpoint{3.933291in}{0.458719in}}%
\pgfpathlineto{\pgfqpoint{4.063808in}{0.458211in}}%
\pgfpathlineto{\pgfqpoint{4.187792in}{0.459914in}}%
\pgfpathlineto{\pgfqpoint{4.294335in}{0.463521in}}%
\pgfpathlineto{\pgfqpoint{4.381234in}{0.468574in}}%
\pgfpathlineto{\pgfqpoint{4.450636in}{0.474702in}}%
\pgfpathlineto{\pgfqpoint{4.506850in}{0.481799in}}%
\pgfpathlineto{\pgfqpoint{4.552009in}{0.489659in}}%
\pgfpathlineto{\pgfqpoint{4.588239in}{0.498115in}}%
\pgfpathlineto{\pgfqpoint{4.617656in}{0.507110in}}%
\pgfpathlineto{\pgfqpoint{4.642328in}{0.516843in}}%
\pgfpathlineto{\pgfqpoint{4.664194in}{0.527940in}}%
\pgfpathlineto{\pgfqpoint{4.681238in}{0.538945in}}%
\pgfpathlineto{\pgfqpoint{4.697164in}{0.551954in}}%
\pgfpathlineto{\pgfqpoint{4.710076in}{0.565290in}}%
\pgfpathlineto{\pgfqpoint{4.721578in}{0.580219in}}%
\pgfpathlineto{\pgfqpoint{4.731557in}{0.596522in}}%
\pgfpathlineto{\pgfqpoint{4.741000in}{0.616135in}}%
\pgfpathlineto{\pgfqpoint{4.749521in}{0.639027in}}%
\pgfpathlineto{\pgfqpoint{4.757522in}{0.667450in}}%
\pgfpathlineto{\pgfqpoint{4.764572in}{0.701345in}}%
\pgfpathlineto{\pgfqpoint{4.770840in}{0.743044in}}%
\pgfpathlineto{\pgfqpoint{4.776327in}{0.794934in}}%
\pgfpathlineto{\pgfqpoint{4.781278in}{0.864398in}}%
\pgfpathlineto{\pgfqpoint{4.785468in}{0.956371in}}%
\pgfpathlineto{\pgfqpoint{4.789000in}{1.085745in}}%
\pgfpathlineto{\pgfqpoint{4.791852in}{1.277385in}}%
\pgfpathlineto{\pgfqpoint{4.793959in}{1.581058in}}%
\pgfpathlineto{\pgfqpoint{4.794962in}{2.071429in}}%
\pgfpathlineto{\pgfqpoint{4.793967in}{2.559311in}}%
\pgfpathlineto{\pgfqpoint{4.791733in}{2.745981in}}%
\pgfpathlineto{\pgfqpoint{4.788955in}{2.818091in}}%
\pgfpathlineto{\pgfqpoint{4.785731in}{2.850227in}}%
\pgfpathlineto{\pgfqpoint{4.781879in}{2.867057in}}%
\pgfpathlineto{\pgfqpoint{4.777744in}{2.875781in}}%
\pgfpathlineto{\pgfqpoint{4.773097in}{2.880982in}}%
\pgfpathlineto{\pgfqpoint{4.767362in}{2.884504in}}%
\pgfpathlineto{\pgfqpoint{4.756853in}{2.887622in}}%
\pgfpathlineto{\pgfqpoint{4.739548in}{2.889639in}}%
\pgfpathlineto{\pgfqpoint{4.704762in}{2.890882in}}%
\pgfpathlineto{\pgfqpoint{4.602524in}{2.891538in}}%
\pgfpathlineto{\pgfqpoint{3.952100in}{2.891742in}}%
\pgfpathlineto{\pgfqpoint{0.617320in}{2.890753in}}%
\pgfpathlineto{\pgfqpoint{0.549910in}{2.888858in}}%
\pgfpathlineto{\pgfqpoint{0.521735in}{2.886179in}}%
\pgfpathlineto{\pgfqpoint{0.504665in}{2.882389in}}%
\pgfpathlineto{\pgfqpoint{0.494500in}{2.878011in}}%
\pgfpathlineto{\pgfqpoint{0.487180in}{2.872666in}}%
\pgfpathlineto{\pgfqpoint{0.481152in}{2.865518in}}%
\pgfpathlineto{\pgfqpoint{0.475664in}{2.854804in}}%
\pgfpathlineto{\pgfqpoint{0.471318in}{2.840737in}}%
\pgfpathlineto{\pgfqpoint{0.467301in}{2.818822in}}%
\pgfpathlineto{\pgfqpoint{0.463927in}{2.786700in}}%
\pgfpathlineto{\pgfqpoint{0.460918in}{2.734544in}}%
\pgfpathlineto{\pgfqpoint{0.458363in}{2.647473in}}%
\pgfpathlineto{\pgfqpoint{0.456575in}{2.523030in}}%
\pgfpathlineto{\pgfqpoint{0.456575in}{2.523030in}}%
\pgfusepath{stroke}%
\end{pgfscope}%
\begin{pgfscope}%
\pgfpathrectangle{\pgfqpoint{0.448634in}{0.402556in}}{\pgfqpoint{4.350661in}{2.489204in}} %
\pgfusepath{clip}%
\pgfsetrectcap%
\pgfsetroundjoin%
\pgfsetlinewidth{1.003750pt}%
\definecolor{currentstroke}{rgb}{0.839216,0.152941,0.156863}%
\pgfsetstrokecolor{currentstroke}%
\pgfsetdash{}{0pt}%
\pgfpathmoveto{\pgfqpoint{0.456424in}{1.370136in}}%
\pgfpathlineto{\pgfqpoint{0.459610in}{1.118754in}}%
\pgfpathlineto{\pgfqpoint{0.463695in}{0.962007in}}%
\pgfpathlineto{\pgfqpoint{0.468519in}{0.857609in}}%
\pgfpathlineto{\pgfqpoint{0.474082in}{0.783209in}}%
\pgfpathlineto{\pgfqpoint{0.480226in}{0.728905in}}%
\pgfpathlineto{\pgfqpoint{0.486970in}{0.687305in}}%
\pgfpathlineto{\pgfqpoint{0.494537in}{0.653558in}}%
\pgfpathlineto{\pgfqpoint{0.503107in}{0.625354in}}%
\pgfpathlineto{\pgfqpoint{0.512193in}{0.602749in}}%
\pgfpathlineto{\pgfqpoint{0.522200in}{0.583507in}}%
\pgfpathlineto{\pgfqpoint{0.534108in}{0.565742in}}%
\pgfpathlineto{\pgfqpoint{0.546264in}{0.551507in}}%
\pgfpathlineto{\pgfqpoint{0.559728in}{0.538906in}}%
\pgfpathlineto{\pgfqpoint{0.576130in}{0.526693in}}%
\pgfpathlineto{\pgfqpoint{0.595483in}{0.515350in}}%
\pgfpathlineto{\pgfqpoint{0.617681in}{0.505146in}}%
\pgfpathlineto{\pgfqpoint{0.642568in}{0.496153in}}%
\pgfpathlineto{\pgfqpoint{0.672126in}{0.487777in}}%
\pgfpathlineto{\pgfqpoint{0.708443in}{0.479823in}}%
\pgfpathlineto{\pgfqpoint{0.753650in}{0.472325in}}%
\pgfpathlineto{\pgfqpoint{0.807718in}{0.465660in}}%
\pgfpathlineto{\pgfqpoint{0.877116in}{0.459475in}}%
\pgfpathlineto{\pgfqpoint{0.961829in}{0.454230in}}%
\pgfpathlineto{\pgfqpoint{1.068352in}{0.449916in}}%
\pgfpathlineto{\pgfqpoint{1.201019in}{0.446839in}}%
\pgfpathlineto{\pgfqpoint{1.357637in}{0.445481in}}%
\pgfpathlineto{\pgfqpoint{1.525135in}{0.446232in}}%
\pgfpathlineto{\pgfqpoint{1.686088in}{0.449142in}}%
\pgfpathlineto{\pgfqpoint{1.823074in}{0.453747in}}%
\pgfpathlineto{\pgfqpoint{1.938245in}{0.459765in}}%
\pgfpathlineto{\pgfqpoint{2.031582in}{0.466759in}}%
\pgfpathlineto{\pgfqpoint{2.109580in}{0.474745in}}%
\pgfpathlineto{\pgfqpoint{2.174384in}{0.483535in}}%
\pgfpathlineto{\pgfqpoint{2.228140in}{0.492940in}}%
\pgfpathlineto{\pgfqpoint{2.275119in}{0.503357in}}%
\pgfpathlineto{\pgfqpoint{2.315282in}{0.514502in}}%
\pgfpathlineto{\pgfqpoint{2.350698in}{0.526659in}}%
\pgfpathlineto{\pgfqpoint{2.381321in}{0.539536in}}%
\pgfpathlineto{\pgfqpoint{2.407164in}{0.552659in}}%
\pgfpathlineto{\pgfqpoint{2.430226in}{0.566639in}}%
\pgfpathlineto{\pgfqpoint{2.452282in}{0.582602in}}%
\pgfpathlineto{\pgfqpoint{2.471391in}{0.599070in}}%
\pgfpathlineto{\pgfqpoint{2.489240in}{0.617293in}}%
\pgfpathlineto{\pgfqpoint{2.505678in}{0.637181in}}%
\pgfpathlineto{\pgfqpoint{2.520620in}{0.658557in}}%
\pgfpathlineto{\pgfqpoint{2.535214in}{0.683314in}}%
\pgfpathlineto{\pgfqpoint{2.549115in}{0.711485in}}%
\pgfpathlineto{\pgfqpoint{2.562091in}{0.743004in}}%
\pgfpathlineto{\pgfqpoint{2.574020in}{0.777751in}}%
\pgfpathlineto{\pgfqpoint{2.585502in}{0.817971in}}%
\pgfpathlineto{\pgfqpoint{2.596809in}{0.866039in}}%
\pgfpathlineto{\pgfqpoint{2.607562in}{0.921948in}}%
\pgfpathlineto{\pgfqpoint{2.617925in}{0.988099in}}%
\pgfpathlineto{\pgfqpoint{2.627958in}{1.066919in}}%
\pgfpathlineto{\pgfqpoint{2.637941in}{1.163321in}}%
\pgfpathlineto{\pgfqpoint{2.648424in}{1.287200in}}%
\pgfpathlineto{\pgfqpoint{2.660103in}{1.453439in}}%
\pgfpathlineto{\pgfqpoint{2.674773in}{1.696802in}}%
\pgfpathlineto{\pgfqpoint{2.687716in}{1.945279in}}%
\pgfpathlineto{\pgfqpoint{2.692670in}{2.079574in}}%
\pgfpathlineto{\pgfqpoint{2.693829in}{2.166682in}}%
\pgfpathlineto{\pgfqpoint{2.692565in}{2.233870in}}%
\pgfpathlineto{\pgfqpoint{2.689436in}{2.286015in}}%
\pgfpathlineto{\pgfqpoint{2.684858in}{2.328000in}}%
\pgfpathlineto{\pgfqpoint{2.678724in}{2.364664in}}%
\pgfpathlineto{\pgfqpoint{2.671355in}{2.395898in}}%
\pgfpathlineto{\pgfqpoint{2.662489in}{2.423982in}}%
\pgfpathlineto{\pgfqpoint{2.652361in}{2.448779in}}%
\pgfpathlineto{\pgfqpoint{2.641364in}{2.470246in}}%
\pgfpathlineto{\pgfqpoint{2.628642in}{2.490425in}}%
\pgfpathlineto{\pgfqpoint{2.614278in}{2.509106in}}%
\pgfpathlineto{\pgfqpoint{2.598442in}{2.526160in}}%
\pgfpathlineto{\pgfqpoint{2.579589in}{2.543005in}}%
\pgfpathlineto{\pgfqpoint{2.559531in}{2.557923in}}%
\pgfpathlineto{\pgfqpoint{2.536601in}{2.572183in}}%
\pgfpathlineto{\pgfqpoint{2.510849in}{2.585539in}}%
\pgfpathlineto{\pgfqpoint{2.482359in}{2.597838in}}%
\pgfpathlineto{\pgfqpoint{2.449134in}{2.609683in}}%
\pgfpathlineto{\pgfqpoint{2.411183in}{2.620696in}}%
\pgfpathlineto{\pgfqpoint{2.368551in}{2.630606in}}%
\pgfpathlineto{\pgfqpoint{2.321293in}{2.639221in}}%
\pgfpathlineto{\pgfqpoint{2.269466in}{2.646399in}}%
\pgfpathlineto{\pgfqpoint{2.210953in}{2.652193in}}%
\pgfpathlineto{\pgfqpoint{2.147966in}{2.656153in}}%
\pgfpathlineto{\pgfqpoint{2.080556in}{2.658135in}}%
\pgfpathlineto{\pgfqpoint{2.010947in}{2.657971in}}%
\pgfpathlineto{\pgfqpoint{1.939194in}{2.655572in}}%
\pgfpathlineto{\pgfqpoint{1.867526in}{2.650913in}}%
\pgfpathlineto{\pgfqpoint{1.798170in}{2.644140in}}%
\pgfpathlineto{\pgfqpoint{1.733340in}{2.635606in}}%
\pgfpathlineto{\pgfqpoint{1.673074in}{2.625521in}}%
\pgfpathlineto{\pgfqpoint{1.615273in}{2.613610in}}%
\pgfpathlineto{\pgfqpoint{1.562132in}{2.600401in}}%
\pgfpathlineto{\pgfqpoint{1.513680in}{2.586139in}}%
\pgfpathlineto{\pgfqpoint{1.467861in}{2.570343in}}%
\pgfpathlineto{\pgfqpoint{1.426793in}{2.553922in}}%
\pgfpathlineto{\pgfqpoint{1.388446in}{2.536289in}}%
\pgfpathlineto{\pgfqpoint{1.352877in}{2.517566in}}%
\pgfpathlineto{\pgfqpoint{1.320128in}{2.497922in}}%
\pgfpathlineto{\pgfqpoint{1.288378in}{2.476235in}}%
\pgfpathlineto{\pgfqpoint{1.259591in}{2.453860in}}%
\pgfpathlineto{\pgfqpoint{1.232050in}{2.429519in}}%
\pgfpathlineto{\pgfqpoint{1.207526in}{2.404898in}}%
\pgfpathlineto{\pgfqpoint{1.184408in}{2.378557in}}%
\pgfpathlineto{\pgfqpoint{1.162827in}{2.350560in}}%
\pgfpathlineto{\pgfqpoint{1.142890in}{2.321011in}}%
\pgfpathlineto{\pgfqpoint{1.124675in}{2.290040in}}%
\pgfpathlineto{\pgfqpoint{1.108225in}{2.257801in}}%
\pgfpathlineto{\pgfqpoint{1.092639in}{2.222198in}}%
\pgfpathlineto{\pgfqpoint{1.079059in}{2.185535in}}%
\pgfpathlineto{\pgfqpoint{1.067443in}{2.147997in}}%
\pgfpathlineto{\pgfqpoint{1.057187in}{2.107347in}}%
\pgfpathlineto{\pgfqpoint{1.049004in}{2.066085in}}%
\pgfpathlineto{\pgfqpoint{1.042513in}{2.021906in}}%
\pgfpathlineto{\pgfqpoint{1.038176in}{1.977381in}}%
\pgfpathlineto{\pgfqpoint{1.035865in}{1.930166in}}%
\pgfpathlineto{\pgfqpoint{1.035826in}{1.882877in}}%
\pgfpathlineto{\pgfqpoint{1.038031in}{1.835656in}}%
\pgfpathlineto{\pgfqpoint{1.042474in}{1.788641in}}%
\pgfpathlineto{\pgfqpoint{1.049175in}{1.741978in}}%
\pgfpathlineto{\pgfqpoint{1.057644in}{1.698238in}}%
\pgfpathlineto{\pgfqpoint{1.068221in}{1.655105in}}%
\pgfpathlineto{\pgfqpoint{1.080962in}{1.612744in}}%
\pgfpathlineto{\pgfqpoint{1.095030in}{1.573616in}}%
\pgfpathlineto{\pgfqpoint{1.111115in}{1.535519in}}%
\pgfpathlineto{\pgfqpoint{1.128117in}{1.500774in}}%
\pgfpathlineto{\pgfqpoint{1.146930in}{1.467273in}}%
\pgfpathlineto{\pgfqpoint{1.167531in}{1.435180in}}%
\pgfpathlineto{\pgfqpoint{1.189874in}{1.404651in}}%
\pgfpathlineto{\pgfqpoint{1.213884in}{1.375827in}}%
\pgfpathlineto{\pgfqpoint{1.237817in}{1.350456in}}%
\pgfpathlineto{\pgfqpoint{1.264748in}{1.325237in}}%
\pgfpathlineto{\pgfqpoint{1.292991in}{1.301971in}}%
\pgfpathlineto{\pgfqpoint{1.322397in}{1.280677in}}%
\pgfpathlineto{\pgfqpoint{1.352820in}{1.261340in}}%
\pgfpathlineto{\pgfqpoint{1.386094in}{1.242888in}}%
\pgfpathlineto{\pgfqpoint{1.420190in}{1.226515in}}%
\pgfpathlineto{\pgfqpoint{1.457024in}{1.211328in}}%
\pgfpathlineto{\pgfqpoint{1.496554in}{1.197536in}}%
\pgfpathlineto{\pgfqpoint{1.538719in}{1.185286in}}%
\pgfpathlineto{\pgfqpoint{1.583441in}{1.174640in}}%
\pgfpathlineto{\pgfqpoint{1.634929in}{1.164774in}}%
\pgfpathlineto{\pgfqpoint{1.706063in}{1.153744in}}%
\pgfpathlineto{\pgfqpoint{1.768492in}{1.143416in}}%
\pgfpathlineto{\pgfqpoint{1.796122in}{1.136566in}}%
\pgfpathlineto{\pgfqpoint{1.812683in}{1.130480in}}%
\pgfpathlineto{\pgfqpoint{1.824471in}{1.124101in}}%
\pgfpathlineto{\pgfqpoint{1.833209in}{1.116740in}}%
\pgfpathlineto{\pgfqpoint{1.838498in}{1.108889in}}%
\pgfpathlineto{\pgfqpoint{1.840588in}{1.101848in}}%
\pgfpathlineto{\pgfqpoint{1.840619in}{1.094411in}}%
\pgfpathlineto{\pgfqpoint{1.837931in}{1.084985in}}%
\pgfpathlineto{\pgfqpoint{1.833246in}{1.076614in}}%
\pgfpathlineto{\pgfqpoint{1.825818in}{1.067542in}}%
\pgfpathlineto{\pgfqpoint{1.813813in}{1.056849in}}%
\pgfpathlineto{\pgfqpoint{1.798819in}{1.046762in}}%
\pgfpathlineto{\pgfqpoint{1.781016in}{1.037462in}}%
\pgfpathlineto{\pgfqpoint{1.758446in}{1.028390in}}%
\pgfpathlineto{\pgfqpoint{1.733203in}{1.020814in}}%
\pgfpathlineto{\pgfqpoint{1.705410in}{1.014871in}}%
\pgfpathlineto{\pgfqpoint{1.675178in}{1.010713in}}%
\pgfpathlineto{\pgfqpoint{1.642610in}{1.008506in}}%
\pgfpathlineto{\pgfqpoint{1.607809in}{1.008432in}}%
\pgfpathlineto{\pgfqpoint{1.570886in}{1.010691in}}%
\pgfpathlineto{\pgfqpoint{1.534118in}{1.015181in}}%
\pgfpathlineto{\pgfqpoint{1.495454in}{1.022232in}}%
\pgfpathlineto{\pgfqpoint{1.457161in}{1.031563in}}%
\pgfpathlineto{\pgfqpoint{1.419337in}{1.043131in}}%
\pgfpathlineto{\pgfqpoint{1.382089in}{1.056928in}}%
\pgfpathlineto{\pgfqpoint{1.347544in}{1.072019in}}%
\pgfpathlineto{\pgfqpoint{1.313727in}{1.089133in}}%
\pgfpathlineto{\pgfqpoint{1.280762in}{1.108299in}}%
\pgfpathlineto{\pgfqpoint{1.248782in}{1.129536in}}%
\pgfpathlineto{\pgfqpoint{1.219708in}{1.151423in}}%
\pgfpathlineto{\pgfqpoint{1.191752in}{1.175138in}}%
\pgfpathlineto{\pgfqpoint{1.165031in}{1.200649in}}%
\pgfpathlineto{\pgfqpoint{1.139653in}{1.227898in}}%
\pgfpathlineto{\pgfqpoint{1.115714in}{1.256800in}}%
\pgfpathlineto{\pgfqpoint{1.093288in}{1.287251in}}%
\pgfpathlineto{\pgfqpoint{1.071178in}{1.321163in}}%
\pgfpathlineto{\pgfqpoint{1.050869in}{1.356520in}}%
\pgfpathlineto{\pgfqpoint{1.032365in}{1.393152in}}%
\pgfpathlineto{\pgfqpoint{1.014718in}{1.433142in}}%
\pgfpathlineto{\pgfqpoint{0.999024in}{1.474186in}}%
\pgfpathlineto{\pgfqpoint{0.984507in}{1.518461in}}%
\pgfpathlineto{\pgfqpoint{0.972010in}{1.563537in}}%
\pgfpathlineto{\pgfqpoint{0.960944in}{1.611679in}}%
\pgfpathlineto{\pgfqpoint{0.951530in}{1.662825in}}%
\pgfpathlineto{\pgfqpoint{0.944287in}{1.714432in}}%
\pgfpathlineto{\pgfqpoint{0.938950in}{1.768847in}}%
\pgfpathlineto{\pgfqpoint{0.935871in}{1.823491in}}%
\pgfpathlineto{\pgfqpoint{0.935034in}{1.878240in}}%
\pgfpathlineto{\pgfqpoint{0.936466in}{1.932973in}}%
\pgfpathlineto{\pgfqpoint{0.940005in}{1.985084in}}%
\pgfpathlineto{\pgfqpoint{0.945759in}{2.036936in}}%
\pgfpathlineto{\pgfqpoint{0.953410in}{2.085938in}}%
\pgfpathlineto{\pgfqpoint{0.962765in}{2.132001in}}%
\pgfpathlineto{\pgfqpoint{0.974287in}{2.177414in}}%
\pgfpathlineto{\pgfqpoint{0.987333in}{2.219653in}}%
\pgfpathlineto{\pgfqpoint{1.001668in}{2.258654in}}%
\pgfpathlineto{\pgfqpoint{1.018051in}{2.296583in}}%
\pgfpathlineto{\pgfqpoint{1.035402in}{2.331102in}}%
\pgfpathlineto{\pgfqpoint{1.054650in}{2.364276in}}%
\pgfpathlineto{\pgfqpoint{1.074407in}{2.393984in}}%
\pgfpathlineto{\pgfqpoint{1.095772in}{2.422197in}}%
\pgfpathlineto{\pgfqpoint{1.118663in}{2.448797in}}%
\pgfpathlineto{\pgfqpoint{1.142967in}{2.473701in}}%
\pgfpathlineto{\pgfqpoint{1.168551in}{2.496867in}}%
\pgfpathlineto{\pgfqpoint{1.197085in}{2.519662in}}%
\pgfpathlineto{\pgfqpoint{1.226727in}{2.540526in}}%
\pgfpathlineto{\pgfqpoint{1.259243in}{2.560673in}}%
\pgfpathlineto{\pgfqpoint{1.294613in}{2.579881in}}%
\pgfpathlineto{\pgfqpoint{1.332793in}{2.597981in}}%
\pgfpathlineto{\pgfqpoint{1.373719in}{2.614859in}}%
\pgfpathlineto{\pgfqpoint{1.417320in}{2.630445in}}%
\pgfpathlineto{\pgfqpoint{1.465633in}{2.645312in}}%
\pgfpathlineto{\pgfqpoint{1.518641in}{2.659204in}}%
\pgfpathlineto{\pgfqpoint{1.576310in}{2.671929in}}%
\pgfpathlineto{\pgfqpoint{1.638598in}{2.683344in}}%
\pgfpathlineto{\pgfqpoint{1.705463in}{2.693342in}}%
\pgfpathlineto{\pgfqpoint{1.779028in}{2.702064in}}%
\pgfpathlineto{\pgfqpoint{1.857098in}{2.709076in}}%
\pgfpathlineto{\pgfqpoint{1.939634in}{2.714279in}}%
\pgfpathlineto{\pgfqpoint{2.026599in}{2.717513in}}%
\pgfpathlineto{\pgfqpoint{2.113606in}{2.718523in}}%
\pgfpathlineto{\pgfqpoint{2.198435in}{2.717302in}}%
\pgfpathlineto{\pgfqpoint{2.278866in}{2.713928in}}%
\pgfpathlineto{\pgfqpoint{2.352678in}{2.708597in}}%
\pgfpathlineto{\pgfqpoint{2.417657in}{2.701708in}}%
\pgfpathlineto{\pgfqpoint{2.473771in}{2.693629in}}%
\pgfpathlineto{\pgfqpoint{2.523141in}{2.684367in}}%
\pgfpathlineto{\pgfqpoint{2.565727in}{2.674202in}}%
\pgfpathlineto{\pgfqpoint{2.601511in}{2.663543in}}%
\pgfpathlineto{\pgfqpoint{2.632577in}{2.652141in}}%
\pgfpathlineto{\pgfqpoint{2.658900in}{2.640329in}}%
\pgfpathlineto{\pgfqpoint{2.682438in}{2.627435in}}%
\pgfpathlineto{\pgfqpoint{2.703063in}{2.613570in}}%
\pgfpathlineto{\pgfqpoint{2.720675in}{2.598977in}}%
\pgfpathlineto{\pgfqpoint{2.735263in}{2.584051in}}%
\pgfpathlineto{\pgfqpoint{2.748320in}{2.567375in}}%
\pgfpathlineto{\pgfqpoint{2.759553in}{2.549044in}}%
\pgfpathlineto{\pgfqpoint{2.768788in}{2.529304in}}%
\pgfpathlineto{\pgfqpoint{2.776016in}{2.508496in}}%
\pgfpathlineto{\pgfqpoint{2.781884in}{2.484537in}}%
\pgfpathlineto{\pgfqpoint{2.786102in}{2.457594in}}%
\pgfpathlineto{\pgfqpoint{2.788719in}{2.425382in}}%
\pgfpathlineto{\pgfqpoint{2.789427in}{2.388059in}}%
\pgfpathlineto{\pgfqpoint{2.787962in}{2.340799in}}%
\pgfpathlineto{\pgfqpoint{2.783671in}{2.278766in}}%
\pgfpathlineto{\pgfqpoint{2.774288in}{2.179780in}}%
\pgfpathlineto{\pgfqpoint{2.743610in}{1.868116in}}%
\pgfpathlineto{\pgfqpoint{2.730111in}{1.702058in}}%
\pgfpathlineto{\pgfqpoint{2.717286in}{1.515947in}}%
\pgfpathlineto{\pgfqpoint{2.702602in}{1.267595in}}%
\pgfpathlineto{\pgfqpoint{2.684433in}{0.964628in}}%
\pgfpathlineto{\pgfqpoint{2.675374in}{0.850597in}}%
\pgfpathlineto{\pgfqpoint{2.667029in}{0.771521in}}%
\pgfpathlineto{\pgfqpoint{2.658751in}{0.712541in}}%
\pgfpathlineto{\pgfqpoint{2.650175in}{0.666282in}}%
\pgfpathlineto{\pgfqpoint{2.640819in}{0.627929in}}%
\pgfpathlineto{\pgfqpoint{2.631143in}{0.597532in}}%
\pgfpathlineto{\pgfqpoint{2.621002in}{0.572743in}}%
\pgfpathlineto{\pgfqpoint{2.609854in}{0.551382in}}%
\pgfpathlineto{\pgfqpoint{2.598039in}{0.533532in}}%
\pgfpathlineto{\pgfqpoint{2.584493in}{0.517377in}}%
\pgfpathlineto{\pgfqpoint{2.571106in}{0.504668in}}%
\pgfpathlineto{\pgfqpoint{2.554786in}{0.492312in}}%
\pgfpathlineto{\pgfqpoint{2.537453in}{0.481913in}}%
\pgfpathlineto{\pgfqpoint{2.517370in}{0.472366in}}%
\pgfpathlineto{\pgfqpoint{2.492539in}{0.463178in}}%
\pgfpathlineto{\pgfqpoint{2.462976in}{0.454833in}}%
\pgfpathlineto{\pgfqpoint{2.428763in}{0.447542in}}%
\pgfpathlineto{\pgfqpoint{2.385668in}{0.440735in}}%
\pgfpathlineto{\pgfqpoint{2.331554in}{0.434582in}}%
\pgfpathlineto{\pgfqpoint{2.262112in}{0.429077in}}%
\pgfpathlineto{\pgfqpoint{2.170847in}{0.424236in}}%
\pgfpathlineto{\pgfqpoint{2.049082in}{0.420134in}}%
\pgfpathlineto{\pgfqpoint{1.879433in}{0.416783in}}%
\pgfpathlineto{\pgfqpoint{1.640156in}{0.414418in}}%
\pgfpathlineto{\pgfqpoint{1.322559in}{0.413569in}}%
\pgfpathlineto{\pgfqpoint{1.020191in}{0.414850in}}%
\pgfpathlineto{\pgfqpoint{0.822253in}{0.417715in}}%
\pgfpathlineto{\pgfqpoint{0.704831in}{0.421430in}}%
\pgfpathlineto{\pgfqpoint{0.630973in}{0.425829in}}%
\pgfpathlineto{\pgfqpoint{0.583312in}{0.430735in}}%
\pgfpathlineto{\pgfqpoint{0.551030in}{0.436125in}}%
\pgfpathlineto{\pgfqpoint{0.527705in}{0.442190in}}%
\pgfpathlineto{\pgfqpoint{0.511247in}{0.448627in}}%
\pgfpathlineto{\pgfqpoint{0.499546in}{0.455218in}}%
\pgfpathlineto{\pgfqpoint{0.488913in}{0.463844in}}%
\pgfpathlineto{\pgfqpoint{0.481320in}{0.472734in}}%
\pgfpathlineto{\pgfqpoint{0.474076in}{0.485131in}}%
\pgfpathlineto{\pgfqpoint{0.468752in}{0.498752in}}%
\pgfpathlineto{\pgfqpoint{0.463869in}{0.517853in}}%
\pgfpathlineto{\pgfqpoint{0.459678in}{0.544801in}}%
\pgfpathlineto{\pgfqpoint{0.456386in}{0.581943in}}%
\pgfpathlineto{\pgfqpoint{0.453731in}{0.639111in}}%
\pgfpathlineto{\pgfqpoint{0.451681in}{0.736159in}}%
\pgfpathlineto{\pgfqpoint{0.450220in}{0.927820in}}%
\pgfpathlineto{\pgfqpoint{0.449345in}{1.403257in}}%
\pgfpathlineto{\pgfqpoint{0.449543in}{2.682707in}}%
\pgfpathlineto{\pgfqpoint{0.451011in}{2.856937in}}%
\pgfpathlineto{\pgfqpoint{0.452803in}{2.879224in}}%
\pgfpathlineto{\pgfqpoint{0.455190in}{2.886111in}}%
\pgfpathlineto{\pgfqpoint{0.458630in}{2.889030in}}%
\pgfpathlineto{\pgfqpoint{0.465000in}{2.890553in}}%
\pgfpathlineto{\pgfqpoint{0.482381in}{2.891423in}}%
\pgfpathlineto{\pgfqpoint{0.565042in}{2.891729in}}%
\pgfpathlineto{\pgfqpoint{2.733846in}{2.891760in}}%
\pgfpathlineto{\pgfqpoint{4.789514in}{2.890885in}}%
\pgfpathlineto{\pgfqpoint{4.793730in}{2.889728in}}%
\pgfpathlineto{\pgfqpoint{4.795483in}{2.888303in}}%
\pgfpathlineto{\pgfqpoint{4.797106in}{2.881140in}}%
\pgfpathlineto{\pgfqpoint{4.797997in}{2.858767in}}%
\pgfpathlineto{\pgfqpoint{4.798039in}{2.856278in}}%
\pgfpathlineto{\pgfqpoint{4.798039in}{2.856278in}}%
\pgfusepath{stroke}%
\end{pgfscope}%
\begin{pgfscope}%
\pgfpathrectangle{\pgfqpoint{0.448634in}{0.402556in}}{\pgfqpoint{4.350661in}{2.489204in}} %
\pgfusepath{clip}%
\pgfsetrectcap%
\pgfsetroundjoin%
\pgfsetlinewidth{1.003750pt}%
\definecolor{currentstroke}{rgb}{0.839216,0.152941,0.156863}%
\pgfsetstrokecolor{currentstroke}%
\pgfsetdash{}{0pt}%
\pgfpathmoveto{\pgfqpoint{3.428651in}{0.402609in}}%
\pgfpathlineto{\pgfqpoint{2.806511in}{0.403751in}}%
\pgfpathlineto{\pgfqpoint{2.769570in}{0.405560in}}%
\pgfpathlineto{\pgfqpoint{2.754510in}{0.408041in}}%
\pgfpathlineto{\pgfqpoint{2.746270in}{0.411177in}}%
\pgfpathlineto{\pgfqpoint{2.740828in}{0.415255in}}%
\pgfpathlineto{\pgfqpoint{2.736685in}{0.420988in}}%
\pgfpathlineto{\pgfqpoint{2.733203in}{0.430086in}}%
\pgfpathlineto{\pgfqpoint{2.730390in}{0.444656in}}%
\pgfpathlineto{\pgfqpoint{2.728198in}{0.469414in}}%
\pgfpathlineto{\pgfqpoint{2.726444in}{0.519154in}}%
\pgfpathlineto{\pgfqpoint{2.725694in}{0.613738in}}%
\pgfpathlineto{\pgfqpoint{2.726830in}{0.768062in}}%
\pgfpathlineto{\pgfqpoint{2.730547in}{0.962172in}}%
\pgfpathlineto{\pgfqpoint{2.736603in}{1.158694in}}%
\pgfpathlineto{\pgfqpoint{2.744085in}{1.327741in}}%
\pgfpathlineto{\pgfqpoint{2.753195in}{1.484212in}}%
\pgfpathlineto{\pgfqpoint{2.763250in}{1.620632in}}%
\pgfpathlineto{\pgfqpoint{2.776110in}{1.764239in}}%
\pgfpathlineto{\pgfqpoint{2.788906in}{1.877800in}}%
\pgfpathlineto{\pgfqpoint{2.805737in}{2.005764in}}%
\pgfpathlineto{\pgfqpoint{2.821163in}{2.101221in}}%
\pgfpathlineto{\pgfqpoint{2.838345in}{2.193743in}}%
\pgfpathlineto{\pgfqpoint{2.859118in}{2.292991in}}%
\pgfpathlineto{\pgfqpoint{2.887190in}{2.425987in}}%
\pgfpathlineto{\pgfqpoint{2.896967in}{2.479587in}}%
\pgfpathlineto{\pgfqpoint{2.901513in}{2.516552in}}%
\pgfpathlineto{\pgfqpoint{2.902813in}{2.543883in}}%
\pgfpathlineto{\pgfqpoint{2.901915in}{2.566252in}}%
\pgfpathlineto{\pgfqpoint{2.899103in}{2.585890in}}%
\pgfpathlineto{\pgfqpoint{2.894739in}{2.602571in}}%
\pgfpathlineto{\pgfqpoint{2.888423in}{2.618410in}}%
\pgfpathlineto{\pgfqpoint{2.880191in}{2.633051in}}%
\pgfpathlineto{\pgfqpoint{2.870278in}{2.646260in}}%
\pgfpathlineto{\pgfqpoint{2.857326in}{2.659540in}}%
\pgfpathlineto{\pgfqpoint{2.843113in}{2.671016in}}%
\pgfpathlineto{\pgfqpoint{2.824160in}{2.683211in}}%
\pgfpathlineto{\pgfqpoint{2.802334in}{2.694418in}}%
\pgfpathlineto{\pgfqpoint{2.775729in}{2.705366in}}%
\pgfpathlineto{\pgfqpoint{2.744381in}{2.715710in}}%
\pgfpathlineto{\pgfqpoint{2.708355in}{2.725246in}}%
\pgfpathlineto{\pgfqpoint{2.665574in}{2.734281in}}%
\pgfpathlineto{\pgfqpoint{2.613910in}{2.742860in}}%
\pgfpathlineto{\pgfqpoint{2.553378in}{2.750579in}}%
\pgfpathlineto{\pgfqpoint{2.481839in}{2.757354in}}%
\pgfpathlineto{\pgfqpoint{2.399317in}{2.762827in}}%
\pgfpathlineto{\pgfqpoint{2.310188in}{2.766469in}}%
\pgfpathlineto{\pgfqpoint{2.179686in}{2.768696in}}%
\pgfpathlineto{\pgfqpoint{2.070922in}{2.768001in}}%
\pgfpathlineto{\pgfqpoint{1.957837in}{2.765011in}}%
\pgfpathlineto{\pgfqpoint{1.855692in}{2.760002in}}%
\pgfpathlineto{\pgfqpoint{1.749308in}{2.752521in}}%
\pgfpathlineto{\pgfqpoint{1.662622in}{2.743917in}}%
\pgfpathlineto{\pgfqpoint{1.584789in}{2.734045in}}%
\pgfpathlineto{\pgfqpoint{1.494273in}{2.720097in}}%
\pgfpathlineto{\pgfqpoint{1.419283in}{2.705171in}}%
\pgfpathlineto{\pgfqpoint{1.364028in}{2.691373in}}%
\pgfpathlineto{\pgfqpoint{1.313475in}{2.676465in}}%
\pgfpathlineto{\pgfqpoint{1.267656in}{2.660668in}}%
\pgfpathlineto{\pgfqpoint{1.224531in}{2.643437in}}%
\pgfpathlineto{\pgfqpoint{1.186241in}{2.625642in}}%
\pgfpathlineto{\pgfqpoint{1.150763in}{2.606697in}}%
\pgfpathlineto{\pgfqpoint{1.118147in}{2.586764in}}%
\pgfpathlineto{\pgfqpoint{1.093965in}{2.570002in}}%
\pgfpathlineto{\pgfqpoint{1.081497in}{2.560021in}}%
\pgfpathlineto{\pgfqpoint{1.053206in}{2.536834in}}%
\pgfpathlineto{\pgfqpoint{1.027894in}{2.513279in}}%
\pgfpathlineto{\pgfqpoint{1.003893in}{2.487994in}}%
\pgfpathlineto{\pgfqpoint{0.981313in}{2.461048in}}%
\pgfpathlineto{\pgfqpoint{0.960236in}{2.432552in}}%
\pgfpathlineto{\pgfqpoint{0.939459in}{2.400607in}}%
\pgfpathlineto{\pgfqpoint{0.924155in}{2.373415in}}%
\pgfpathlineto{\pgfqpoint{0.905513in}{2.336876in}}%
\pgfpathlineto{\pgfqpoint{0.888751in}{2.299162in}}%
\pgfpathlineto{\pgfqpoint{0.872927in}{2.258184in}}%
\pgfpathlineto{\pgfqpoint{0.858205in}{2.213996in}}%
\pgfpathlineto{\pgfqpoint{0.844745in}{2.166661in}}%
\pgfpathlineto{\pgfqpoint{0.839185in}{2.142610in}}%
\pgfpathlineto{\pgfqpoint{0.827457in}{2.089523in}}%
\pgfpathlineto{\pgfqpoint{0.816714in}{2.031064in}}%
\pgfpathlineto{\pgfqpoint{0.810087in}{1.984396in}}%
\pgfpathlineto{\pgfqpoint{0.808019in}{1.967140in}}%
\pgfpathlineto{\pgfqpoint{0.800072in}{1.898042in}}%
\pgfpathlineto{\pgfqpoint{0.793712in}{1.823724in}}%
\pgfpathlineto{\pgfqpoint{0.788803in}{1.741776in}}%
\pgfpathlineto{\pgfqpoint{0.786295in}{1.679613in}}%
\pgfpathlineto{\pgfqpoint{0.776768in}{1.450902in}}%
\pgfpathlineto{\pgfqpoint{0.773612in}{1.421256in}}%
\pgfpathlineto{\pgfqpoint{0.768297in}{1.389484in}}%
\pgfpathlineto{\pgfqpoint{0.762746in}{1.368012in}}%
\pgfpathlineto{\pgfqpoint{0.756710in}{1.352029in}}%
\pgfpathlineto{\pgfqpoint{0.749733in}{1.339431in}}%
\pgfpathlineto{\pgfqpoint{0.742174in}{1.330520in}}%
\pgfpathlineto{\pgfqpoint{0.734821in}{1.325244in}}%
\pgfpathlineto{\pgfqpoint{0.726520in}{1.322365in}}%
\pgfpathlineto{\pgfqpoint{0.717846in}{1.322183in}}%
\pgfpathlineto{\pgfqpoint{0.709377in}{1.324382in}}%
\pgfpathlineto{\pgfqpoint{0.699516in}{1.329584in}}%
\pgfpathlineto{\pgfqpoint{0.688866in}{1.338189in}}%
\pgfpathlineto{\pgfqpoint{0.677883in}{1.350238in}}%
\pgfpathlineto{\pgfqpoint{0.666865in}{1.365640in}}%
\pgfpathlineto{\pgfqpoint{0.654894in}{1.386412in}}%
\pgfpathlineto{\pgfqpoint{0.642558in}{1.412726in}}%
\pgfpathlineto{\pgfqpoint{0.630314in}{1.444626in}}%
\pgfpathlineto{\pgfqpoint{0.618492in}{1.482079in}}%
\pgfpathlineto{\pgfqpoint{0.608601in}{1.520255in}}%
\pgfpathlineto{\pgfqpoint{0.590193in}{1.612444in}}%
\pgfpathlineto{\pgfqpoint{0.581839in}{1.668884in}}%
\pgfpathlineto{\pgfqpoint{0.573129in}{1.740376in}}%
\pgfpathlineto{\pgfqpoint{0.567053in}{1.807213in}}%
\pgfpathlineto{\pgfqpoint{0.560524in}{1.896509in}}%
\pgfpathlineto{\pgfqpoint{0.555519in}{1.995910in}}%
\pgfpathlineto{\pgfqpoint{0.552557in}{2.097908in}}%
\pgfpathlineto{\pgfqpoint{0.551518in}{2.204935in}}%
\pgfpathlineto{\pgfqpoint{0.552720in}{2.309469in}}%
\pgfpathlineto{\pgfqpoint{0.556004in}{2.403981in}}%
\pgfpathlineto{\pgfqpoint{0.560951in}{2.483430in}}%
\pgfpathlineto{\pgfqpoint{0.567302in}{2.550240in}}%
\pgfpathlineto{\pgfqpoint{0.574926in}{2.606816in}}%
\pgfpathlineto{\pgfqpoint{0.582986in}{2.650656in}}%
\pgfpathlineto{\pgfqpoint{0.592756in}{2.691451in}}%
\pgfpathlineto{\pgfqpoint{0.602651in}{2.721755in}}%
\pgfpathlineto{\pgfqpoint{0.612984in}{2.746440in}}%
\pgfpathlineto{\pgfqpoint{0.624292in}{2.767691in}}%
\pgfpathlineto{\pgfqpoint{0.636231in}{2.785432in}}%
\pgfpathlineto{\pgfqpoint{0.649892in}{2.801460in}}%
\pgfpathlineto{\pgfqpoint{0.663387in}{2.814019in}}%
\pgfpathlineto{\pgfqpoint{0.679843in}{2.826134in}}%
\pgfpathlineto{\pgfqpoint{0.697326in}{2.836196in}}%
\pgfpathlineto{\pgfqpoint{0.715574in}{2.844285in}}%
\pgfpathlineto{\pgfqpoint{0.738439in}{2.852335in}}%
\pgfpathlineto{\pgfqpoint{0.765983in}{2.859639in}}%
\pgfpathlineto{\pgfqpoint{0.800300in}{2.866256in}}%
\pgfpathlineto{\pgfqpoint{0.841340in}{2.871832in}}%
\pgfpathlineto{\pgfqpoint{0.895547in}{2.876802in}}%
\pgfpathlineto{\pgfqpoint{0.969412in}{2.881069in}}%
\pgfpathlineto{\pgfqpoint{1.071608in}{2.884501in}}%
\pgfpathlineto{\pgfqpoint{1.219512in}{2.887074in}}%
\pgfpathlineto{\pgfqpoint{1.471844in}{2.889091in}}%
\pgfpathlineto{\pgfqpoint{1.956941in}{2.890384in}}%
\pgfpathlineto{\pgfqpoint{3.096814in}{2.890781in}}%
\pgfpathlineto{\pgfqpoint{3.995224in}{2.889388in}}%
\pgfpathlineto{\pgfqpoint{4.275833in}{2.887011in}}%
\pgfpathlineto{\pgfqpoint{4.412847in}{2.883743in}}%
\pgfpathlineto{\pgfqpoint{4.491081in}{2.879810in}}%
\pgfpathlineto{\pgfqpoint{4.543127in}{2.875163in}}%
\pgfpathlineto{\pgfqpoint{4.579810in}{2.869841in}}%
\pgfpathlineto{\pgfqpoint{4.607579in}{2.863763in}}%
\pgfpathlineto{\pgfqpoint{4.630623in}{2.856423in}}%
\pgfpathlineto{\pgfqpoint{4.648833in}{2.848228in}}%
\pgfpathlineto{\pgfqpoint{4.664136in}{2.838773in}}%
\pgfpathlineto{\pgfqpoint{4.676470in}{2.828576in}}%
\pgfpathlineto{\pgfqpoint{4.687502in}{2.816585in}}%
\pgfpathlineto{\pgfqpoint{4.697050in}{2.803027in}}%
\pgfpathlineto{\pgfqpoint{4.706194in}{2.786098in}}%
\pgfpathlineto{\pgfqpoint{4.714508in}{2.765827in}}%
\pgfpathlineto{\pgfqpoint{4.722461in}{2.740013in}}%
\pgfpathlineto{\pgfqpoint{4.729577in}{2.708703in}}%
\pgfpathlineto{\pgfqpoint{4.736162in}{2.669601in}}%
\pgfpathlineto{\pgfqpoint{4.742419in}{2.617826in}}%
\pgfpathlineto{\pgfqpoint{4.747859in}{2.553410in}}%
\pgfpathlineto{\pgfqpoint{4.752661in}{2.468958in}}%
\pgfpathlineto{\pgfqpoint{4.756610in}{2.359528in}}%
\pgfpathlineto{\pgfqpoint{4.759416in}{2.217681in}}%
\pgfpathlineto{\pgfqpoint{4.760596in}{2.043444in}}%
\pgfpathlineto{\pgfqpoint{4.759662in}{1.851779in}}%
\pgfpathlineto{\pgfqpoint{4.756587in}{1.667613in}}%
\pgfpathlineto{\pgfqpoint{4.751596in}{1.503428in}}%
\pgfpathlineto{\pgfqpoint{4.745410in}{1.374185in}}%
\pgfpathlineto{\pgfqpoint{4.738113in}{1.267479in}}%
\pgfpathlineto{\pgfqpoint{4.729621in}{1.175896in}}%
\pgfpathlineto{\pgfqpoint{4.720762in}{1.104428in}}%
\pgfpathlineto{\pgfqpoint{4.711044in}{1.043205in}}%
\pgfpathlineto{\pgfqpoint{4.700364in}{0.989829in}}%
\pgfpathlineto{\pgfqpoint{4.689055in}{0.944345in}}%
\pgfpathlineto{\pgfqpoint{4.676881in}{0.904394in}}%
\pgfpathlineto{\pgfqpoint{4.676095in}{0.902073in}}%
\pgfpathlineto{\pgfqpoint{4.676095in}{0.902073in}}%
\pgfusepath{stroke}%
\end{pgfscope}%
\begin{pgfscope}%
\pgfpathrectangle{\pgfqpoint{0.448634in}{0.402556in}}{\pgfqpoint{4.350661in}{2.489204in}} %
\pgfusepath{clip}%
\pgfsetrectcap%
\pgfsetroundjoin%
\pgfsetlinewidth{1.003750pt}%
\definecolor{currentstroke}{rgb}{0.839216,0.152941,0.156863}%
\pgfsetstrokecolor{currentstroke}%
\pgfsetdash{}{0pt}%
\pgfpathmoveto{\pgfqpoint{2.795520in}{1.982745in}}%
\pgfpathlineto{\pgfqpoint{2.781780in}{1.874357in}}%
\pgfpathlineto{\pgfqpoint{2.769351in}{1.758234in}}%
\pgfpathlineto{\pgfqpoint{2.758095in}{1.631942in}}%
\pgfpathlineto{\pgfqpoint{2.747786in}{1.490551in}}%
\pgfpathlineto{\pgfqpoint{2.738644in}{1.334082in}}%
\pgfpathlineto{\pgfqpoint{2.730580in}{1.157591in}}%
\pgfpathlineto{\pgfqpoint{2.723334in}{0.948663in}}%
\pgfpathlineto{\pgfqpoint{2.709783in}{0.530788in}}%
\pgfpathlineto{\pgfqpoint{2.705868in}{0.488716in}}%
\pgfpathlineto{\pgfqpoint{2.701769in}{0.464281in}}%
\pgfpathlineto{\pgfqpoint{2.697021in}{0.447744in}}%
\pgfpathlineto{\pgfqpoint{2.691859in}{0.436812in}}%
\pgfpathlineto{\pgfqpoint{2.686245in}{0.429229in}}%
\pgfpathlineto{\pgfqpoint{2.679348in}{0.423188in}}%
\pgfpathlineto{\pgfqpoint{2.669540in}{0.417856in}}%
\pgfpathlineto{\pgfqpoint{2.656987in}{0.413810in}}%
\pgfpathlineto{\pgfqpoint{2.637654in}{0.410337in}}%
\pgfpathlineto{\pgfqpoint{2.607297in}{0.407617in}}%
\pgfpathlineto{\pgfqpoint{2.555121in}{0.405574in}}%
\pgfpathlineto{\pgfqpoint{2.450714in}{0.404139in}}%
\pgfpathlineto{\pgfqpoint{2.176624in}{0.403275in}}%
\pgfpathlineto{\pgfqpoint{1.130290in}{0.402953in}}%
\pgfpathlineto{\pgfqpoint{0.516850in}{0.404175in}}%
\pgfpathlineto{\pgfqpoint{0.466848in}{0.405970in}}%
\pgfpathlineto{\pgfqpoint{0.456130in}{0.407931in}}%
\pgfpathlineto{\pgfqpoint{0.452340in}{0.410303in}}%
\pgfpathlineto{\pgfqpoint{0.450346in}{0.414662in}}%
\pgfpathlineto{\pgfqpoint{0.449266in}{0.424524in}}%
\pgfpathlineto{\pgfqpoint{0.448771in}{0.464344in}}%
\pgfpathlineto{\pgfqpoint{0.448640in}{0.850171in}}%
\pgfpathlineto{\pgfqpoint{0.448653in}{2.891318in}}%
\pgfpathlineto{\pgfqpoint{0.448653in}{2.891318in}}%
\pgfusepath{stroke}%
\end{pgfscope}%
\begin{pgfscope}%
\pgfpathrectangle{\pgfqpoint{0.448634in}{0.402556in}}{\pgfqpoint{4.350661in}{2.489204in}} %
\pgfusepath{clip}%
\pgfsetrectcap%
\pgfsetroundjoin%
\pgfsetlinewidth{1.003750pt}%
\definecolor{currentstroke}{rgb}{0.839216,0.152941,0.156863}%
\pgfsetstrokecolor{currentstroke}%
\pgfsetdash{}{0pt}%
\pgfpathmoveto{\pgfqpoint{3.428108in}{0.402585in}}%
\pgfpathlineto{\pgfqpoint{2.782040in}{0.403682in}}%
\pgfpathlineto{\pgfqpoint{2.753823in}{0.405625in}}%
\pgfpathlineto{\pgfqpoint{2.743240in}{0.408368in}}%
\pgfpathlineto{\pgfqpoint{2.737627in}{0.412108in}}%
\pgfpathlineto{\pgfqpoint{2.733587in}{0.417923in}}%
\pgfpathlineto{\pgfqpoint{2.730585in}{0.427243in}}%
\pgfpathlineto{\pgfqpoint{2.728341in}{0.441944in}}%
\pgfpathlineto{\pgfqpoint{2.726513in}{0.471735in}}%
\pgfpathlineto{\pgfqpoint{2.725172in}{0.536433in}}%
\pgfpathlineto{\pgfqpoint{2.725187in}{0.660892in}}%
\pgfpathlineto{\pgfqpoint{2.727502in}{0.840094in}}%
\pgfpathlineto{\pgfqpoint{2.732400in}{1.046619in}}%
\pgfpathlineto{\pgfqpoint{2.739055in}{1.228170in}}%
\pgfpathlineto{\pgfqpoint{2.747349in}{1.394675in}}%
\pgfpathlineto{\pgfqpoint{2.756955in}{1.543620in}}%
\pgfpathlineto{\pgfqpoint{2.769168in}{1.697304in}}%
\pgfpathlineto{\pgfqpoint{2.781487in}{1.818455in}}%
\pgfpathlineto{\pgfqpoint{2.794709in}{1.926927in}}%
\pgfpathlineto{\pgfqpoint{2.813109in}{2.057113in}}%
\pgfpathlineto{\pgfqpoint{2.829200in}{2.149890in}}%
\pgfpathlineto{\pgfqpoint{2.847858in}{2.244590in}}%
\pgfpathlineto{\pgfqpoint{2.895785in}{2.479645in}}%
\pgfpathlineto{\pgfqpoint{2.900171in}{2.516635in}}%
\pgfpathlineto{\pgfqpoint{2.901315in}{2.543975in}}%
\pgfpathlineto{\pgfqpoint{2.900263in}{2.566334in}}%
\pgfpathlineto{\pgfqpoint{2.897309in}{2.585946in}}%
\pgfpathlineto{\pgfqpoint{2.892815in}{2.602581in}}%
\pgfpathlineto{\pgfqpoint{2.886377in}{2.618355in}}%
\pgfpathlineto{\pgfqpoint{2.878044in}{2.632922in}}%
\pgfpathlineto{\pgfqpoint{2.868054in}{2.646056in}}%
\pgfpathlineto{\pgfqpoint{2.855043in}{2.659260in}}%
\pgfpathlineto{\pgfqpoint{2.840796in}{2.670680in}}%
\pgfpathlineto{\pgfqpoint{2.821819in}{2.682828in}}%
\pgfpathlineto{\pgfqpoint{2.799979in}{2.693996in}}%
\pgfpathlineto{\pgfqpoint{2.773365in}{2.704918in}}%
\pgfpathlineto{\pgfqpoint{2.742012in}{2.715243in}}%
\pgfpathlineto{\pgfqpoint{2.705983in}{2.724764in}}%
\pgfpathlineto{\pgfqpoint{2.663201in}{2.733792in}}%
\pgfpathlineto{\pgfqpoint{2.611536in}{2.742362in}}%
\pgfpathlineto{\pgfqpoint{2.551003in}{2.750075in}}%
\pgfpathlineto{\pgfqpoint{2.481634in}{2.756668in}}%
\pgfpathlineto{\pgfqpoint{2.399114in}{2.762187in}}%
\pgfpathlineto{\pgfqpoint{2.309986in}{2.765874in}}%
\pgfpathlineto{\pgfqpoint{2.188185in}{2.768085in}}%
\pgfpathlineto{\pgfqpoint{2.081596in}{2.767608in}}%
\pgfpathlineto{\pgfqpoint{1.968507in}{2.764828in}}%
\pgfpathlineto{\pgfqpoint{1.864182in}{2.759906in}}%
\pgfpathlineto{\pgfqpoint{1.757788in}{2.752579in}}%
\pgfpathlineto{\pgfqpoint{1.671088in}{2.744156in}}%
\pgfpathlineto{\pgfqpoint{1.591077in}{2.734177in}}%
\pgfpathlineto{\pgfqpoint{1.502691in}{2.720697in}}%
\pgfpathlineto{\pgfqpoint{1.427655in}{2.706083in}}%
\pgfpathlineto{\pgfqpoint{1.372350in}{2.692544in}}%
\pgfpathlineto{\pgfqpoint{1.321734in}{2.677921in}}%
\pgfpathlineto{\pgfqpoint{1.273765in}{2.661664in}}%
\pgfpathlineto{\pgfqpoint{1.230567in}{2.644671in}}%
\pgfpathlineto{\pgfqpoint{1.192197in}{2.627105in}}%
\pgfpathlineto{\pgfqpoint{1.156620in}{2.608402in}}%
\pgfpathlineto{\pgfqpoint{1.123890in}{2.588716in}}%
\pgfpathlineto{\pgfqpoint{1.095884in}{2.569567in}}%
\pgfpathlineto{\pgfqpoint{1.065695in}{2.545165in}}%
\pgfpathlineto{\pgfqpoint{1.039897in}{2.522313in}}%
\pgfpathlineto{\pgfqpoint{1.015357in}{2.497712in}}%
\pgfpathlineto{\pgfqpoint{0.992200in}{2.471416in}}%
\pgfpathlineto{\pgfqpoint{0.970519in}{2.443520in}}%
\pgfpathlineto{\pgfqpoint{0.950374in}{2.414154in}}%
\pgfpathlineto{\pgfqpoint{0.930608in}{2.381381in}}%
\pgfpathlineto{\pgfqpoint{0.906559in}{2.334046in}}%
\pgfpathlineto{\pgfqpoint{0.889929in}{2.296256in}}%
\pgfpathlineto{\pgfqpoint{0.874245in}{2.255207in}}%
\pgfpathlineto{\pgfqpoint{0.859672in}{2.210955in}}%
\pgfpathlineto{\pgfqpoint{0.846991in}{2.165947in}}%
\pgfpathlineto{\pgfqpoint{0.839637in}{2.134709in}}%
\pgfpathlineto{\pgfqpoint{0.828242in}{2.081526in}}%
\pgfpathlineto{\pgfqpoint{0.817871in}{2.022980in}}%
\pgfpathlineto{\pgfqpoint{0.810788in}{1.971346in}}%
\pgfpathlineto{\pgfqpoint{0.802850in}{1.902246in}}%
\pgfpathlineto{\pgfqpoint{0.796560in}{1.827921in}}%
\pgfpathlineto{\pgfqpoint{0.791702in}{1.743474in}}%
\pgfpathlineto{\pgfqpoint{0.787780in}{1.621588in}}%
\pgfpathlineto{\pgfqpoint{0.785417in}{1.522057in}}%
\pgfpathlineto{\pgfqpoint{0.785417in}{1.522057in}}%
\pgfusepath{stroke}%
\end{pgfscope}%
\begin{pgfscope}%
\pgfpathrectangle{\pgfqpoint{0.448634in}{0.402556in}}{\pgfqpoint{4.350661in}{2.489204in}} %
\pgfusepath{clip}%
\pgfsetrectcap%
\pgfsetroundjoin%
\pgfsetlinewidth{1.003750pt}%
\definecolor{currentstroke}{rgb}{0.839216,0.152941,0.156863}%
\pgfsetstrokecolor{currentstroke}%
\pgfsetdash{}{0pt}%
\pgfpathmoveto{\pgfqpoint{2.028735in}{0.425754in}}%
\pgfpathlineto{\pgfqpoint{1.878677in}{0.421879in}}%
\pgfpathlineto{\pgfqpoint{1.676387in}{0.418997in}}%
\pgfpathlineto{\pgfqpoint{1.413176in}{0.417558in}}%
\pgfpathlineto{\pgfqpoint{1.134735in}{0.418204in}}%
\pgfpathlineto{\pgfqpoint{0.921565in}{0.420769in}}%
\pgfpathlineto{\pgfqpoint{0.782384in}{0.424523in}}%
\pgfpathlineto{\pgfqpoint{0.693283in}{0.428974in}}%
\pgfpathlineto{\pgfqpoint{0.632541in}{0.434091in}}%
\pgfpathlineto{\pgfqpoint{0.591492in}{0.439563in}}%
\pgfpathlineto{\pgfqpoint{0.561503in}{0.445595in}}%
\pgfpathlineto{\pgfqpoint{0.538349in}{0.452466in}}%
\pgfpathlineto{\pgfqpoint{0.522042in}{0.459394in}}%
\pgfpathlineto{\pgfqpoint{0.508540in}{0.467420in}}%
\pgfpathlineto{\pgfqpoint{0.497973in}{0.476161in}}%
\pgfpathlineto{\pgfqpoint{0.488790in}{0.486749in}}%
\pgfpathlineto{\pgfqpoint{0.481284in}{0.498948in}}%
\pgfpathlineto{\pgfqpoint{0.474590in}{0.514580in}}%
\pgfpathlineto{\pgfqpoint{0.469106in}{0.533467in}}%
\pgfpathlineto{\pgfqpoint{0.464439in}{0.557770in}}%
\pgfpathlineto{\pgfqpoint{0.460297in}{0.592288in}}%
\pgfpathlineto{\pgfqpoint{0.456855in}{0.641912in}}%
\pgfpathlineto{\pgfqpoint{0.454122in}{0.716520in}}%
\pgfpathlineto{\pgfqpoint{0.451978in}{0.843444in}}%
\pgfpathlineto{\pgfqpoint{0.450459in}{1.087379in}}%
\pgfpathlineto{\pgfqpoint{0.449596in}{1.657406in}}%
\pgfpathlineto{\pgfqpoint{0.450150in}{2.687936in}}%
\pgfpathlineto{\pgfqpoint{0.451781in}{2.839761in}}%
\pgfpathlineto{\pgfqpoint{0.453975in}{2.872003in}}%
\pgfpathlineto{\pgfqpoint{0.456339in}{2.881553in}}%
\pgfpathlineto{\pgfqpoint{0.458888in}{2.885549in}}%
\pgfpathlineto{\pgfqpoint{0.462554in}{2.888171in}}%
\pgfpathlineto{\pgfqpoint{0.471046in}{2.890205in}}%
\pgfpathlineto{\pgfqpoint{0.490597in}{2.891263in}}%
\pgfpathlineto{\pgfqpoint{0.564556in}{2.891692in}}%
\pgfpathlineto{\pgfqpoint{1.569559in}{2.891759in}}%
\pgfpathlineto{\pgfqpoint{4.784679in}{2.890785in}}%
\pgfpathlineto{\pgfqpoint{4.791004in}{2.889098in}}%
\pgfpathlineto{\pgfqpoint{4.793910in}{2.885555in}}%
\pgfpathlineto{\pgfqpoint{4.795579in}{2.878366in}}%
\pgfpathlineto{\pgfqpoint{4.796850in}{2.858514in}}%
\pgfpathlineto{\pgfqpoint{4.796850in}{2.858514in}}%
\pgfusepath{stroke}%
\end{pgfscope}%
\begin{pgfscope}%
\pgfpathrectangle{\pgfqpoint{0.448634in}{0.402556in}}{\pgfqpoint{4.350661in}{2.489204in}} %
\pgfusepath{clip}%
\pgfsetrectcap%
\pgfsetroundjoin%
\pgfsetlinewidth{1.003750pt}%
\definecolor{currentstroke}{rgb}{0.580392,0.403922,0.741176}%
\pgfsetstrokecolor{currentstroke}%
\pgfsetdash{}{0pt}%
\pgfpathmoveto{\pgfqpoint{0.448634in}{2.896245in}}%
\pgfpathlineto{\pgfqpoint{0.448593in}{0.407043in}}%
\pgfpathlineto{\pgfqpoint{0.448593in}{0.407043in}}%
\pgfusepath{stroke}%
\end{pgfscope}%
\begin{pgfscope}%
\pgfpathrectangle{\pgfqpoint{0.448634in}{0.402556in}}{\pgfqpoint{4.350661in}{2.489204in}} %
\pgfusepath{clip}%
\pgfsetrectcap%
\pgfsetroundjoin%
\pgfsetlinewidth{1.003750pt}%
\definecolor{currentstroke}{rgb}{0.580392,0.403922,0.741176}%
\pgfsetstrokecolor{currentstroke}%
\pgfsetdash{}{0pt}%
\pgfpathmoveto{\pgfqpoint{0.576951in}{1.760538in}}%
\pgfpathlineto{\pgfqpoint{0.569478in}{1.839729in}}%
\pgfpathlineto{\pgfqpoint{0.563281in}{1.929056in}}%
\pgfpathlineto{\pgfqpoint{0.558655in}{2.028481in}}%
\pgfpathlineto{\pgfqpoint{0.556039in}{2.132982in}}%
\pgfpathlineto{\pgfqpoint{0.555612in}{2.237525in}}%
\pgfpathlineto{\pgfqpoint{0.557411in}{2.337069in}}%
\pgfpathlineto{\pgfqpoint{0.561129in}{2.424084in}}%
\pgfpathlineto{\pgfqpoint{0.566433in}{2.498509in}}%
\pgfpathlineto{\pgfqpoint{0.572931in}{2.560289in}}%
\pgfpathlineto{\pgfqpoint{0.580470in}{2.611840in}}%
\pgfpathlineto{\pgfqpoint{0.589087in}{2.655540in}}%
\pgfpathlineto{\pgfqpoint{0.598393in}{2.691318in}}%
\pgfpathlineto{\pgfqpoint{0.608584in}{2.721493in}}%
\pgfpathlineto{\pgfqpoint{0.619193in}{2.746024in}}%
\pgfpathlineto{\pgfqpoint{0.630750in}{2.767100in}}%
\pgfpathlineto{\pgfqpoint{0.642889in}{2.784662in}}%
\pgfpathlineto{\pgfqpoint{0.656708in}{2.800512in}}%
\pgfpathlineto{\pgfqpoint{0.672074in}{2.814373in}}%
\pgfpathlineto{\pgfqpoint{0.688717in}{2.826151in}}%
\pgfpathlineto{\pgfqpoint{0.706316in}{2.835949in}}%
\pgfpathlineto{\pgfqpoint{0.726650in}{2.844773in}}%
\pgfpathlineto{\pgfqpoint{0.751707in}{2.853121in}}%
\pgfpathlineto{\pgfqpoint{0.781469in}{2.860484in}}%
\pgfpathlineto{\pgfqpoint{0.818003in}{2.867006in}}%
\pgfpathlineto{\pgfqpoint{0.863415in}{2.872650in}}%
\pgfpathlineto{\pgfqpoint{0.921994in}{2.877494in}}%
\pgfpathlineto{\pgfqpoint{1.000223in}{2.881551in}}%
\pgfpathlineto{\pgfqpoint{1.111126in}{2.884871in}}%
\pgfpathlineto{\pgfqpoint{1.274261in}{2.887361in}}%
\pgfpathlineto{\pgfqpoint{1.552698in}{2.889260in}}%
\pgfpathlineto{\pgfqpoint{2.107406in}{2.890455in}}%
\pgfpathlineto{\pgfqpoint{3.342993in}{2.890572in}}%
\pgfpathlineto{\pgfqpoint{4.043448in}{2.888939in}}%
\pgfpathlineto{\pgfqpoint{4.289249in}{2.886402in}}%
\pgfpathlineto{\pgfqpoint{4.413207in}{2.883092in}}%
\pgfpathlineto{\pgfqpoint{4.489257in}{2.878998in}}%
\pgfpathlineto{\pgfqpoint{4.541284in}{2.874086in}}%
\pgfpathlineto{\pgfqpoint{4.577934in}{2.868482in}}%
\pgfpathlineto{\pgfqpoint{4.605654in}{2.862113in}}%
\pgfpathlineto{\pgfqpoint{4.626564in}{2.855276in}}%
\pgfpathlineto{\pgfqpoint{4.644768in}{2.847064in}}%
\pgfpathlineto{\pgfqpoint{4.660091in}{2.837651in}}%
\pgfpathlineto{\pgfqpoint{4.672483in}{2.827545in}}%
\pgfpathlineto{\pgfqpoint{4.683623in}{2.815683in}}%
\pgfpathlineto{\pgfqpoint{4.693290in}{2.802237in}}%
\pgfpathlineto{\pgfqpoint{4.702639in}{2.785456in}}%
\pgfpathlineto{\pgfqpoint{4.711190in}{2.765315in}}%
\pgfpathlineto{\pgfqpoint{4.719409in}{2.739611in}}%
\pgfpathlineto{\pgfqpoint{4.726237in}{2.710790in}}%
\pgfpathlineto{\pgfqpoint{4.733213in}{2.671777in}}%
\pgfpathlineto{\pgfqpoint{4.739565in}{2.622531in}}%
\pgfpathlineto{\pgfqpoint{4.745204in}{2.560640in}}%
\pgfpathlineto{\pgfqpoint{4.750137in}{2.481189in}}%
\pgfpathlineto{\pgfqpoint{4.754344in}{2.376755in}}%
\pgfpathlineto{\pgfqpoint{4.757424in}{2.242386in}}%
\pgfpathlineto{\pgfqpoint{4.758960in}{2.075620in}}%
\pgfpathlineto{\pgfqpoint{4.758432in}{1.888932in}}%
\pgfpathlineto{\pgfqpoint{4.755742in}{1.707248in}}%
\pgfpathlineto{\pgfqpoint{4.750913in}{1.533095in}}%
\pgfpathlineto{\pgfqpoint{4.744772in}{1.398864in}}%
\pgfpathlineto{\pgfqpoint{4.737562in}{1.289653in}}%
\pgfpathlineto{\pgfqpoint{4.728958in}{1.193079in}}%
\pgfpathlineto{\pgfqpoint{4.719974in}{1.119117in}}%
\pgfpathlineto{\pgfqpoint{4.710015in}{1.055414in}}%
\pgfpathlineto{\pgfqpoint{4.699483in}{1.001999in}}%
\pgfpathlineto{\pgfqpoint{4.689021in}{0.958828in}}%
\pgfpathlineto{\pgfqpoint{4.677201in}{0.918737in}}%
\pgfpathlineto{\pgfqpoint{4.664018in}{0.881886in}}%
\pgfpathlineto{\pgfqpoint{4.650570in}{0.850627in}}%
\pgfpathlineto{\pgfqpoint{4.636291in}{0.822703in}}%
\pgfpathlineto{\pgfqpoint{4.620197in}{0.796107in}}%
\pgfpathlineto{\pgfqpoint{4.603634in}{0.773030in}}%
\pgfpathlineto{\pgfqpoint{4.585485in}{0.751572in}}%
\pgfpathlineto{\pgfqpoint{4.565874in}{0.731870in}}%
\pgfpathlineto{\pgfqpoint{4.544966in}{0.713997in}}%
\pgfpathlineto{\pgfqpoint{4.522963in}{0.697937in}}%
\pgfpathlineto{\pgfqpoint{4.496164in}{0.681397in}}%
\pgfpathlineto{\pgfqpoint{4.470404in}{0.668064in}}%
\pgfpathlineto{\pgfqpoint{4.439969in}{0.654616in}}%
\pgfpathlineto{\pgfqpoint{4.406851in}{0.642383in}}%
\pgfpathlineto{\pgfqpoint{4.369019in}{0.630845in}}%
\pgfpathlineto{\pgfqpoint{4.326500in}{0.620319in}}%
\pgfpathlineto{\pgfqpoint{4.279339in}{0.611038in}}%
\pgfpathlineto{\pgfqpoint{4.227588in}{0.603171in}}%
\pgfpathlineto{\pgfqpoint{4.173462in}{0.597155in}}%
\pgfpathlineto{\pgfqpoint{4.110523in}{0.592293in}}%
\pgfpathlineto{\pgfqpoint{4.047483in}{0.589634in}}%
\pgfpathlineto{\pgfqpoint{3.977879in}{0.588722in}}%
\pgfpathlineto{\pgfqpoint{3.906104in}{0.590034in}}%
\pgfpathlineto{\pgfqpoint{3.834388in}{0.593598in}}%
\pgfpathlineto{\pgfqpoint{3.767132in}{0.599173in}}%
\pgfpathlineto{\pgfqpoint{3.704376in}{0.606499in}}%
\pgfpathlineto{\pgfqpoint{3.672051in}{0.611550in}}%
\pgfpathlineto{\pgfqpoint{3.616152in}{0.621383in}}%
\pgfpathlineto{\pgfqpoint{3.582059in}{0.629255in}}%
\pgfpathlineto{\pgfqpoint{3.526793in}{0.642946in}}%
\pgfpathlineto{\pgfqpoint{3.484777in}{0.655851in}}%
\pgfpathlineto{\pgfqpoint{3.445342in}{0.669986in}}%
\pgfpathlineto{\pgfqpoint{3.402485in}{0.688029in}}%
\pgfpathlineto{\pgfqpoint{3.368683in}{0.705182in}}%
\pgfpathlineto{\pgfqpoint{3.337727in}{0.723377in}}%
\pgfpathlineto{\pgfqpoint{3.309712in}{0.742510in}}%
\pgfpathlineto{\pgfqpoint{3.284602in}{0.762217in}}%
\pgfpathlineto{\pgfqpoint{3.260657in}{0.783741in}}%
\pgfpathlineto{\pgfqpoint{3.238063in}{0.807097in}}%
\pgfpathlineto{\pgfqpoint{3.216972in}{0.832226in}}%
\pgfpathlineto{\pgfqpoint{3.197504in}{0.859015in}}%
\pgfpathlineto{\pgfqpoint{3.179739in}{0.887312in}}%
\pgfpathlineto{\pgfqpoint{3.163711in}{0.916936in}}%
\pgfpathlineto{\pgfqpoint{3.148413in}{0.949910in}}%
\pgfpathlineto{\pgfqpoint{3.134992in}{0.983937in}}%
\pgfpathlineto{\pgfqpoint{3.122644in}{1.021166in}}%
\pgfpathlineto{\pgfqpoint{3.111045in}{1.063955in}}%
\pgfpathlineto{\pgfqpoint{3.101611in}{1.107434in}}%
\pgfpathlineto{\pgfqpoint{3.093519in}{1.156344in}}%
\pgfpathlineto{\pgfqpoint{3.087658in}{1.205669in}}%
\pgfpathlineto{\pgfqpoint{3.083671in}{1.257738in}}%
\pgfpathlineto{\pgfqpoint{3.081720in}{1.314941in}}%
\pgfpathlineto{\pgfqpoint{3.082104in}{1.372186in}}%
\pgfpathlineto{\pgfqpoint{3.084714in}{1.429355in}}%
\pgfpathlineto{\pgfqpoint{3.089507in}{1.486340in}}%
\pgfpathlineto{\pgfqpoint{3.096482in}{1.543027in}}%
\pgfpathlineto{\pgfqpoint{3.105243in}{1.596860in}}%
\pgfpathlineto{\pgfqpoint{3.116109in}{1.650187in}}%
\pgfpathlineto{\pgfqpoint{3.129049in}{1.702904in}}%
\pgfpathlineto{\pgfqpoint{3.143526in}{1.752478in}}%
\pgfpathlineto{\pgfqpoint{3.159251in}{1.798892in}}%
\pgfpathlineto{\pgfqpoint{3.177011in}{1.844332in}}%
\pgfpathlineto{\pgfqpoint{3.196758in}{1.888686in}}%
\pgfpathlineto{\pgfqpoint{3.217405in}{1.929650in}}%
\pgfpathlineto{\pgfqpoint{3.239880in}{1.969335in}}%
\pgfpathlineto{\pgfqpoint{3.264115in}{2.007640in}}%
\pgfpathlineto{\pgfqpoint{3.291407in}{2.046404in}}%
\pgfpathlineto{\pgfqpoint{3.321788in}{2.085434in}}%
\pgfpathlineto{\pgfqpoint{3.358154in}{2.128294in}}%
\pgfpathlineto{\pgfqpoint{3.412930in}{2.192310in}}%
\pgfpathlineto{\pgfqpoint{3.424890in}{2.210025in}}%
\pgfpathlineto{\pgfqpoint{3.430043in}{2.220965in}}%
\pgfpathlineto{\pgfqpoint{3.432032in}{2.230605in}}%
\pgfpathlineto{\pgfqpoint{3.430771in}{2.237858in}}%
\pgfpathlineto{\pgfqpoint{3.426620in}{2.243528in}}%
\pgfpathlineto{\pgfqpoint{3.420906in}{2.247086in}}%
\pgfpathlineto{\pgfqpoint{3.412499in}{2.249585in}}%
\pgfpathlineto{\pgfqpoint{3.399497in}{2.250691in}}%
\pgfpathlineto{\pgfqpoint{3.384304in}{2.249673in}}%
\pgfpathlineto{\pgfqpoint{3.364984in}{2.246100in}}%
\pgfpathlineto{\pgfqpoint{3.341802in}{2.239344in}}%
\pgfpathlineto{\pgfqpoint{3.317108in}{2.229683in}}%
\pgfpathlineto{\pgfqpoint{3.291103in}{2.216987in}}%
\pgfpathlineto{\pgfqpoint{3.265927in}{2.202263in}}%
\pgfpathlineto{\pgfqpoint{3.239803in}{2.184362in}}%
\pgfpathlineto{\pgfqpoint{3.214774in}{2.164520in}}%
\pgfpathlineto{\pgfqpoint{3.190899in}{2.142894in}}%
\pgfpathlineto{\pgfqpoint{3.166656in}{2.117913in}}%
\pgfpathlineto{\pgfqpoint{3.143835in}{2.091234in}}%
\pgfpathlineto{\pgfqpoint{3.121078in}{2.061107in}}%
\pgfpathlineto{\pgfqpoint{3.099951in}{2.029464in}}%
\pgfpathlineto{\pgfqpoint{3.079250in}{1.994406in}}%
\pgfpathlineto{\pgfqpoint{3.059217in}{1.955915in}}%
\pgfpathlineto{\pgfqpoint{3.040058in}{1.914015in}}%
\pgfpathlineto{\pgfqpoint{3.022809in}{1.871041in}}%
\pgfpathlineto{\pgfqpoint{3.005790in}{1.822536in}}%
\pgfpathlineto{\pgfqpoint{2.990067in}{1.770819in}}%
\pgfpathlineto{\pgfqpoint{2.975708in}{1.715980in}}%
\pgfpathlineto{\pgfqpoint{2.962284in}{1.655680in}}%
\pgfpathlineto{\pgfqpoint{2.950495in}{1.592386in}}%
\pgfpathlineto{\pgfqpoint{2.940383in}{1.526185in}}%
\pgfpathlineto{\pgfqpoint{2.931745in}{1.454681in}}%
\pgfpathlineto{\pgfqpoint{2.925082in}{1.380399in}}%
\pgfpathlineto{\pgfqpoint{2.920647in}{1.305899in}}%
\pgfpathlineto{\pgfqpoint{2.918444in}{1.231270in}}%
\pgfpathlineto{\pgfqpoint{2.918545in}{1.159087in}}%
\pgfpathlineto{\pgfqpoint{2.920787in}{1.091931in}}%
\pgfpathlineto{\pgfqpoint{2.925177in}{1.027412in}}%
\pgfpathlineto{\pgfqpoint{2.931192in}{0.970580in}}%
\pgfpathlineto{\pgfqpoint{2.938760in}{0.919034in}}%
\pgfpathlineto{\pgfqpoint{2.947651in}{0.872852in}}%
\pgfpathlineto{\pgfqpoint{2.958213in}{0.829714in}}%
\pgfpathlineto{\pgfqpoint{2.969670in}{0.792114in}}%
\pgfpathlineto{\pgfqpoint{2.982463in}{0.757773in}}%
\pgfpathlineto{\pgfqpoint{2.996425in}{0.726812in}}%
\pgfpathlineto{\pgfqpoint{3.011299in}{0.699300in}}%
\pgfpathlineto{\pgfqpoint{3.026739in}{0.675225in}}%
\pgfpathlineto{\pgfqpoint{3.043828in}{0.652656in}}%
\pgfpathlineto{\pgfqpoint{3.062495in}{0.631788in}}%
\pgfpathlineto{\pgfqpoint{3.082602in}{0.612753in}}%
\pgfpathlineto{\pgfqpoint{3.103961in}{0.595592in}}%
\pgfpathlineto{\pgfqpoint{3.128268in}{0.579069in}}%
\pgfpathlineto{\pgfqpoint{3.153537in}{0.564554in}}%
\pgfpathlineto{\pgfqpoint{3.181571in}{0.550952in}}%
\pgfpathlineto{\pgfqpoint{3.214371in}{0.537647in}}%
\pgfpathlineto{\pgfqpoint{3.249846in}{0.525713in}}%
\pgfpathlineto{\pgfqpoint{3.290011in}{0.514571in}}%
\pgfpathlineto{\pgfqpoint{3.334821in}{0.504423in}}%
\pgfpathlineto{\pgfqpoint{3.386372in}{0.494999in}}%
\pgfpathlineto{\pgfqpoint{3.446798in}{0.486258in}}%
\pgfpathlineto{\pgfqpoint{3.518243in}{0.478283in}}%
\pgfpathlineto{\pgfqpoint{3.600685in}{0.471410in}}%
\pgfpathlineto{\pgfqpoint{3.696269in}{0.465714in}}%
\pgfpathlineto{\pgfqpoint{3.807144in}{0.461370in}}%
\pgfpathlineto{\pgfqpoint{3.933291in}{0.458720in}}%
\pgfpathlineto{\pgfqpoint{4.063809in}{0.458212in}}%
\pgfpathlineto{\pgfqpoint{4.187792in}{0.459915in}}%
\pgfpathlineto{\pgfqpoint{4.294335in}{0.463523in}}%
\pgfpathlineto{\pgfqpoint{4.381234in}{0.468576in}}%
\pgfpathlineto{\pgfqpoint{4.450636in}{0.474703in}}%
\pgfpathlineto{\pgfqpoint{4.506850in}{0.481801in}}%
\pgfpathlineto{\pgfqpoint{4.552009in}{0.489661in}}%
\pgfpathlineto{\pgfqpoint{4.588239in}{0.498117in}}%
\pgfpathlineto{\pgfqpoint{4.617656in}{0.507113in}}%
\pgfpathlineto{\pgfqpoint{4.642328in}{0.516845in}}%
\pgfpathlineto{\pgfqpoint{4.664194in}{0.527943in}}%
\pgfpathlineto{\pgfqpoint{4.681238in}{0.538948in}}%
\pgfpathlineto{\pgfqpoint{4.697164in}{0.551957in}}%
\pgfpathlineto{\pgfqpoint{4.710075in}{0.565293in}}%
\pgfpathlineto{\pgfqpoint{4.721578in}{0.580222in}}%
\pgfpathlineto{\pgfqpoint{4.731557in}{0.596525in}}%
\pgfpathlineto{\pgfqpoint{4.740999in}{0.616138in}}%
\pgfpathlineto{\pgfqpoint{4.749521in}{0.639030in}}%
\pgfpathlineto{\pgfqpoint{4.757522in}{0.667453in}}%
\pgfpathlineto{\pgfqpoint{4.764571in}{0.701348in}}%
\pgfpathlineto{\pgfqpoint{4.770839in}{0.743047in}}%
\pgfpathlineto{\pgfqpoint{4.776326in}{0.794937in}}%
\pgfpathlineto{\pgfqpoint{4.781277in}{0.864401in}}%
\pgfpathlineto{\pgfqpoint{4.785468in}{0.956374in}}%
\pgfpathlineto{\pgfqpoint{4.789000in}{1.085748in}}%
\pgfpathlineto{\pgfqpoint{4.791852in}{1.277388in}}%
\pgfpathlineto{\pgfqpoint{4.793959in}{1.581061in}}%
\pgfpathlineto{\pgfqpoint{4.794962in}{2.071432in}}%
\pgfpathlineto{\pgfqpoint{4.793967in}{2.559314in}}%
\pgfpathlineto{\pgfqpoint{4.791733in}{2.745984in}}%
\pgfpathlineto{\pgfqpoint{4.788955in}{2.818095in}}%
\pgfpathlineto{\pgfqpoint{4.785730in}{2.850230in}}%
\pgfpathlineto{\pgfqpoint{4.781877in}{2.867060in}}%
\pgfpathlineto{\pgfqpoint{4.777742in}{2.875783in}}%
\pgfpathlineto{\pgfqpoint{4.773094in}{2.880984in}}%
\pgfpathlineto{\pgfqpoint{4.767359in}{2.884505in}}%
\pgfpathlineto{\pgfqpoint{4.756850in}{2.887623in}}%
\pgfpathlineto{\pgfqpoint{4.739544in}{2.889639in}}%
\pgfpathlineto{\pgfqpoint{4.704759in}{2.890882in}}%
\pgfpathlineto{\pgfqpoint{4.602520in}{2.891538in}}%
\pgfpathlineto{\pgfqpoint{3.952097in}{2.891742in}}%
\pgfpathlineto{\pgfqpoint{0.617317in}{2.890753in}}%
\pgfpathlineto{\pgfqpoint{0.549906in}{2.888859in}}%
\pgfpathlineto{\pgfqpoint{0.521731in}{2.886180in}}%
\pgfpathlineto{\pgfqpoint{0.504662in}{2.882390in}}%
\pgfpathlineto{\pgfqpoint{0.494497in}{2.878012in}}%
\pgfpathlineto{\pgfqpoint{0.487176in}{2.872667in}}%
\pgfpathlineto{\pgfqpoint{0.481148in}{2.865519in}}%
\pgfpathlineto{\pgfqpoint{0.475661in}{2.854804in}}%
\pgfpathlineto{\pgfqpoint{0.471316in}{2.840737in}}%
\pgfpathlineto{\pgfqpoint{0.467299in}{2.818822in}}%
\pgfpathlineto{\pgfqpoint{0.463926in}{2.786700in}}%
\pgfpathlineto{\pgfqpoint{0.460917in}{2.734544in}}%
\pgfpathlineto{\pgfqpoint{0.458362in}{2.647473in}}%
\pgfpathlineto{\pgfqpoint{0.456574in}{2.523030in}}%
\pgfpathlineto{\pgfqpoint{0.456574in}{2.523030in}}%
\pgfusepath{stroke}%
\end{pgfscope}%
\begin{pgfscope}%
\pgfpathrectangle{\pgfqpoint{0.448634in}{0.402556in}}{\pgfqpoint{4.350661in}{2.489204in}} %
\pgfusepath{clip}%
\pgfsetrectcap%
\pgfsetroundjoin%
\pgfsetlinewidth{1.003750pt}%
\definecolor{currentstroke}{rgb}{0.580392,0.403922,0.741176}%
\pgfsetstrokecolor{currentstroke}%
\pgfsetdash{}{0pt}%
\pgfpathmoveto{\pgfqpoint{4.798840in}{2.852369in}}%
\pgfpathlineto{\pgfqpoint{4.797564in}{2.889610in}}%
\pgfpathlineto{\pgfqpoint{4.796215in}{2.891483in}}%
\pgfpathlineto{\pgfqpoint{4.787551in}{2.891760in}}%
\pgfpathlineto{\pgfqpoint{0.452128in}{2.891659in}}%
\pgfpathlineto{\pgfqpoint{0.450530in}{2.890082in}}%
\pgfpathlineto{\pgfqpoint{0.449454in}{2.882763in}}%
\pgfpathlineto{\pgfqpoint{0.448970in}{2.845432in}}%
\pgfpathlineto{\pgfqpoint{0.448743in}{2.494454in}}%
\pgfpathlineto{\pgfqpoint{0.449624in}{0.615107in}}%
\pgfpathlineto{\pgfqpoint{0.451433in}{0.510586in}}%
\pgfpathlineto{\pgfqpoint{0.453993in}{0.473374in}}%
\pgfpathlineto{\pgfqpoint{0.457406in}{0.453868in}}%
\pgfpathlineto{\pgfqpoint{0.461540in}{0.442384in}}%
\pgfpathlineto{\pgfqpoint{0.466739in}{0.434437in}}%
\pgfpathlineto{\pgfqpoint{0.473595in}{0.428350in}}%
\pgfpathlineto{\pgfqpoint{0.483493in}{0.423244in}}%
\pgfpathlineto{\pgfqpoint{0.491854in}{0.420501in}}%
\pgfpathlineto{\pgfqpoint{0.491854in}{0.420501in}}%
\pgfusepath{stroke}%
\end{pgfscope}%
\begin{pgfscope}%
\pgfpathrectangle{\pgfqpoint{0.448634in}{0.402556in}}{\pgfqpoint{4.350661in}{2.489204in}} %
\pgfusepath{clip}%
\pgfsetrectcap%
\pgfsetroundjoin%
\pgfsetlinewidth{1.003750pt}%
\definecolor{currentstroke}{rgb}{0.580392,0.403922,0.741176}%
\pgfsetstrokecolor{currentstroke}%
\pgfsetdash{}{0pt}%
\pgfpathmoveto{\pgfqpoint{0.456423in}{1.370144in}}%
\pgfpathlineto{\pgfqpoint{0.459610in}{1.118762in}}%
\pgfpathlineto{\pgfqpoint{0.463694in}{0.962014in}}%
\pgfpathlineto{\pgfqpoint{0.468517in}{0.857617in}}%
\pgfpathlineto{\pgfqpoint{0.474080in}{0.783217in}}%
\pgfpathlineto{\pgfqpoint{0.480223in}{0.728913in}}%
\pgfpathlineto{\pgfqpoint{0.486967in}{0.687313in}}%
\pgfpathlineto{\pgfqpoint{0.494533in}{0.653565in}}%
\pgfpathlineto{\pgfqpoint{0.503103in}{0.625361in}}%
\pgfpathlineto{\pgfqpoint{0.512187in}{0.602755in}}%
\pgfpathlineto{\pgfqpoint{0.522194in}{0.583513in}}%
\pgfpathlineto{\pgfqpoint{0.534101in}{0.565747in}}%
\pgfpathlineto{\pgfqpoint{0.546256in}{0.551511in}}%
\pgfpathlineto{\pgfqpoint{0.559720in}{0.538910in}}%
\pgfpathlineto{\pgfqpoint{0.576121in}{0.526696in}}%
\pgfpathlineto{\pgfqpoint{0.595475in}{0.515353in}}%
\pgfpathlineto{\pgfqpoint{0.617672in}{0.505148in}}%
\pgfpathlineto{\pgfqpoint{0.642559in}{0.496154in}}%
\pgfpathlineto{\pgfqpoint{0.672117in}{0.487778in}}%
\pgfpathlineto{\pgfqpoint{0.708434in}{0.479824in}}%
\pgfpathlineto{\pgfqpoint{0.753640in}{0.472326in}}%
\pgfpathlineto{\pgfqpoint{0.807708in}{0.465661in}}%
\pgfpathlineto{\pgfqpoint{0.877107in}{0.459475in}}%
\pgfpathlineto{\pgfqpoint{0.961819in}{0.454230in}}%
\pgfpathlineto{\pgfqpoint{1.068342in}{0.449916in}}%
\pgfpathlineto{\pgfqpoint{1.201009in}{0.446840in}}%
\pgfpathlineto{\pgfqpoint{1.357628in}{0.445481in}}%
\pgfpathlineto{\pgfqpoint{1.525126in}{0.446232in}}%
\pgfpathlineto{\pgfqpoint{1.686079in}{0.449142in}}%
\pgfpathlineto{\pgfqpoint{1.823065in}{0.453747in}}%
\pgfpathlineto{\pgfqpoint{1.938236in}{0.459764in}}%
\pgfpathlineto{\pgfqpoint{2.031573in}{0.466759in}}%
\pgfpathlineto{\pgfqpoint{2.109571in}{0.474744in}}%
\pgfpathlineto{\pgfqpoint{2.174375in}{0.483534in}}%
\pgfpathlineto{\pgfqpoint{2.228130in}{0.492939in}}%
\pgfpathlineto{\pgfqpoint{2.275110in}{0.503355in}}%
\pgfpathlineto{\pgfqpoint{2.315273in}{0.514500in}}%
\pgfpathlineto{\pgfqpoint{2.350689in}{0.526657in}}%
\pgfpathlineto{\pgfqpoint{2.381312in}{0.539533in}}%
\pgfpathlineto{\pgfqpoint{2.407156in}{0.552655in}}%
\pgfpathlineto{\pgfqpoint{2.430218in}{0.566635in}}%
\pgfpathlineto{\pgfqpoint{2.452274in}{0.582597in}}%
\pgfpathlineto{\pgfqpoint{2.471383in}{0.599064in}}%
\pgfpathlineto{\pgfqpoint{2.489233in}{0.617287in}}%
\pgfpathlineto{\pgfqpoint{2.505671in}{0.637174in}}%
\pgfpathlineto{\pgfqpoint{2.520614in}{0.658551in}}%
\pgfpathlineto{\pgfqpoint{2.535208in}{0.683307in}}%
\pgfpathlineto{\pgfqpoint{2.549110in}{0.711477in}}%
\pgfpathlineto{\pgfqpoint{2.562086in}{0.742996in}}%
\pgfpathlineto{\pgfqpoint{2.574016in}{0.777743in}}%
\pgfpathlineto{\pgfqpoint{2.585499in}{0.817962in}}%
\pgfpathlineto{\pgfqpoint{2.596806in}{0.866030in}}%
\pgfpathlineto{\pgfqpoint{2.607559in}{0.921940in}}%
\pgfpathlineto{\pgfqpoint{2.617922in}{0.988090in}}%
\pgfpathlineto{\pgfqpoint{2.627956in}{1.066910in}}%
\pgfpathlineto{\pgfqpoint{2.637940in}{1.163312in}}%
\pgfpathlineto{\pgfqpoint{2.648423in}{1.287191in}}%
\pgfpathlineto{\pgfqpoint{2.660102in}{1.453430in}}%
\pgfpathlineto{\pgfqpoint{2.674772in}{1.696793in}}%
\pgfpathlineto{\pgfqpoint{2.687715in}{1.945270in}}%
\pgfpathlineto{\pgfqpoint{2.692669in}{2.079565in}}%
\pgfpathlineto{\pgfqpoint{2.693828in}{2.166674in}}%
\pgfpathlineto{\pgfqpoint{2.692564in}{2.233861in}}%
\pgfpathlineto{\pgfqpoint{2.689436in}{2.286006in}}%
\pgfpathlineto{\pgfqpoint{2.684858in}{2.327991in}}%
\pgfpathlineto{\pgfqpoint{2.678725in}{2.364656in}}%
\pgfpathlineto{\pgfqpoint{2.671357in}{2.395889in}}%
\pgfpathlineto{\pgfqpoint{2.662490in}{2.423974in}}%
\pgfpathlineto{\pgfqpoint{2.652363in}{2.448771in}}%
\pgfpathlineto{\pgfqpoint{2.641367in}{2.470238in}}%
\pgfpathlineto{\pgfqpoint{2.628646in}{2.490418in}}%
\pgfpathlineto{\pgfqpoint{2.614282in}{2.509099in}}%
\pgfpathlineto{\pgfqpoint{2.598447in}{2.526153in}}%
\pgfpathlineto{\pgfqpoint{2.579594in}{2.543000in}}%
\pgfpathlineto{\pgfqpoint{2.559536in}{2.557918in}}%
\pgfpathlineto{\pgfqpoint{2.536606in}{2.572179in}}%
\pgfpathlineto{\pgfqpoint{2.510854in}{2.585535in}}%
\pgfpathlineto{\pgfqpoint{2.482365in}{2.597834in}}%
\pgfpathlineto{\pgfqpoint{2.449139in}{2.609680in}}%
\pgfpathlineto{\pgfqpoint{2.411189in}{2.620694in}}%
\pgfpathlineto{\pgfqpoint{2.368557in}{2.630604in}}%
\pgfpathlineto{\pgfqpoint{2.321299in}{2.639219in}}%
\pgfpathlineto{\pgfqpoint{2.269472in}{2.646397in}}%
\pgfpathlineto{\pgfqpoint{2.210959in}{2.652192in}}%
\pgfpathlineto{\pgfqpoint{2.147972in}{2.656152in}}%
\pgfpathlineto{\pgfqpoint{2.080562in}{2.658134in}}%
\pgfpathlineto{\pgfqpoint{2.010953in}{2.657971in}}%
\pgfpathlineto{\pgfqpoint{1.939200in}{2.655572in}}%
\pgfpathlineto{\pgfqpoint{1.867532in}{2.650913in}}%
\pgfpathlineto{\pgfqpoint{1.798176in}{2.644141in}}%
\pgfpathlineto{\pgfqpoint{1.733346in}{2.635606in}}%
\pgfpathlineto{\pgfqpoint{1.673080in}{2.625522in}}%
\pgfpathlineto{\pgfqpoint{1.615279in}{2.613611in}}%
\pgfpathlineto{\pgfqpoint{1.562138in}{2.600403in}}%
\pgfpathlineto{\pgfqpoint{1.513686in}{2.586141in}}%
\pgfpathlineto{\pgfqpoint{1.467867in}{2.570346in}}%
\pgfpathlineto{\pgfqpoint{1.426798in}{2.553925in}}%
\pgfpathlineto{\pgfqpoint{1.388452in}{2.536292in}}%
\pgfpathlineto{\pgfqpoint{1.352882in}{2.517569in}}%
\pgfpathlineto{\pgfqpoint{1.320133in}{2.497925in}}%
\pgfpathlineto{\pgfqpoint{1.288383in}{2.476240in}}%
\pgfpathlineto{\pgfqpoint{1.259596in}{2.453865in}}%
\pgfpathlineto{\pgfqpoint{1.232054in}{2.429524in}}%
\pgfpathlineto{\pgfqpoint{1.207530in}{2.404903in}}%
\pgfpathlineto{\pgfqpoint{1.184412in}{2.378563in}}%
\pgfpathlineto{\pgfqpoint{1.162830in}{2.350567in}}%
\pgfpathlineto{\pgfqpoint{1.142893in}{2.321018in}}%
\pgfpathlineto{\pgfqpoint{1.124677in}{2.290047in}}%
\pgfpathlineto{\pgfqpoint{1.108227in}{2.257809in}}%
\pgfpathlineto{\pgfqpoint{1.092640in}{2.222206in}}%
\pgfpathlineto{\pgfqpoint{1.079060in}{2.185542in}}%
\pgfpathlineto{\pgfqpoint{1.067444in}{2.148005in}}%
\pgfpathlineto{\pgfqpoint{1.057187in}{2.107355in}}%
\pgfpathlineto{\pgfqpoint{1.049004in}{2.066093in}}%
\pgfpathlineto{\pgfqpoint{1.042512in}{2.021914in}}%
\pgfpathlineto{\pgfqpoint{1.038175in}{1.977390in}}%
\pgfpathlineto{\pgfqpoint{1.035863in}{1.930175in}}%
\pgfpathlineto{\pgfqpoint{1.035824in}{1.882886in}}%
\pgfpathlineto{\pgfqpoint{1.038028in}{1.835664in}}%
\pgfpathlineto{\pgfqpoint{1.042470in}{1.788649in}}%
\pgfpathlineto{\pgfqpoint{1.049171in}{1.741986in}}%
\pgfpathlineto{\pgfqpoint{1.057639in}{1.698246in}}%
\pgfpathlineto{\pgfqpoint{1.068216in}{1.655113in}}%
\pgfpathlineto{\pgfqpoint{1.080956in}{1.612752in}}%
\pgfpathlineto{\pgfqpoint{1.095024in}{1.573623in}}%
\pgfpathlineto{\pgfqpoint{1.111108in}{1.535526in}}%
\pgfpathlineto{\pgfqpoint{1.128110in}{1.500781in}}%
\pgfpathlineto{\pgfqpoint{1.146921in}{1.467279in}}%
\pgfpathlineto{\pgfqpoint{1.167522in}{1.435186in}}%
\pgfpathlineto{\pgfqpoint{1.189865in}{1.404657in}}%
\pgfpathlineto{\pgfqpoint{1.213874in}{1.375832in}}%
\pgfpathlineto{\pgfqpoint{1.237807in}{1.350460in}}%
\pgfpathlineto{\pgfqpoint{1.264737in}{1.325240in}}%
\pgfpathlineto{\pgfqpoint{1.292979in}{1.301974in}}%
\pgfpathlineto{\pgfqpoint{1.322386in}{1.280679in}}%
\pgfpathlineto{\pgfqpoint{1.352808in}{1.261341in}}%
\pgfpathlineto{\pgfqpoint{1.386082in}{1.242889in}}%
\pgfpathlineto{\pgfqpoint{1.420178in}{1.226515in}}%
\pgfpathlineto{\pgfqpoint{1.457011in}{1.211327in}}%
\pgfpathlineto{\pgfqpoint{1.496541in}{1.197533in}}%
\pgfpathlineto{\pgfqpoint{1.538706in}{1.185283in}}%
\pgfpathlineto{\pgfqpoint{1.583427in}{1.174636in}}%
\pgfpathlineto{\pgfqpoint{1.634915in}{1.164769in}}%
\pgfpathlineto{\pgfqpoint{1.706049in}{1.153738in}}%
\pgfpathlineto{\pgfqpoint{1.768479in}{1.143410in}}%
\pgfpathlineto{\pgfqpoint{1.796109in}{1.136561in}}%
\pgfpathlineto{\pgfqpoint{1.812670in}{1.130476in}}%
\pgfpathlineto{\pgfqpoint{1.824458in}{1.124098in}}%
\pgfpathlineto{\pgfqpoint{1.833198in}{1.116738in}}%
\pgfpathlineto{\pgfqpoint{1.838488in}{1.108888in}}%
\pgfpathlineto{\pgfqpoint{1.840581in}{1.101848in}}%
\pgfpathlineto{\pgfqpoint{1.840614in}{1.094412in}}%
\pgfpathlineto{\pgfqpoint{1.837927in}{1.084985in}}%
\pgfpathlineto{\pgfqpoint{1.833244in}{1.076613in}}%
\pgfpathlineto{\pgfqpoint{1.825817in}{1.067540in}}%
\pgfpathlineto{\pgfqpoint{1.813812in}{1.056847in}}%
\pgfpathlineto{\pgfqpoint{1.798818in}{1.046759in}}%
\pgfpathlineto{\pgfqpoint{1.781015in}{1.037459in}}%
\pgfpathlineto{\pgfqpoint{1.758446in}{1.028387in}}%
\pgfpathlineto{\pgfqpoint{1.733202in}{1.020811in}}%
\pgfpathlineto{\pgfqpoint{1.705409in}{1.014868in}}%
\pgfpathlineto{\pgfqpoint{1.675177in}{1.010710in}}%
\pgfpathlineto{\pgfqpoint{1.642609in}{1.008503in}}%
\pgfpathlineto{\pgfqpoint{1.607808in}{1.008428in}}%
\pgfpathlineto{\pgfqpoint{1.570885in}{1.010688in}}%
\pgfpathlineto{\pgfqpoint{1.534117in}{1.015177in}}%
\pgfpathlineto{\pgfqpoint{1.495454in}{1.022230in}}%
\pgfpathlineto{\pgfqpoint{1.457161in}{1.031561in}}%
\pgfpathlineto{\pgfqpoint{1.419337in}{1.043130in}}%
\pgfpathlineto{\pgfqpoint{1.382089in}{1.056928in}}%
\pgfpathlineto{\pgfqpoint{1.347544in}{1.072018in}}%
\pgfpathlineto{\pgfqpoint{1.313727in}{1.089133in}}%
\pgfpathlineto{\pgfqpoint{1.280762in}{1.108299in}}%
\pgfpathlineto{\pgfqpoint{1.248782in}{1.129537in}}%
\pgfpathlineto{\pgfqpoint{1.219709in}{1.151423in}}%
\pgfpathlineto{\pgfqpoint{1.191752in}{1.175139in}}%
\pgfpathlineto{\pgfqpoint{1.165031in}{1.200650in}}%
\pgfpathlineto{\pgfqpoint{1.139654in}{1.227899in}}%
\pgfpathlineto{\pgfqpoint{1.115714in}{1.256801in}}%
\pgfpathlineto{\pgfqpoint{1.093289in}{1.287252in}}%
\pgfpathlineto{\pgfqpoint{1.071178in}{1.321164in}}%
\pgfpathlineto{\pgfqpoint{1.050869in}{1.356521in}}%
\pgfpathlineto{\pgfqpoint{1.032365in}{1.393153in}}%
\pgfpathlineto{\pgfqpoint{1.014718in}{1.433143in}}%
\pgfpathlineto{\pgfqpoint{0.999024in}{1.474186in}}%
\pgfpathlineto{\pgfqpoint{0.984507in}{1.518462in}}%
\pgfpathlineto{\pgfqpoint{0.972010in}{1.563538in}}%
\pgfpathlineto{\pgfqpoint{0.960944in}{1.611679in}}%
\pgfpathlineto{\pgfqpoint{0.951531in}{1.662825in}}%
\pgfpathlineto{\pgfqpoint{0.944287in}{1.714432in}}%
\pgfpathlineto{\pgfqpoint{0.938950in}{1.768848in}}%
\pgfpathlineto{\pgfqpoint{0.935871in}{1.823492in}}%
\pgfpathlineto{\pgfqpoint{0.935035in}{1.878241in}}%
\pgfpathlineto{\pgfqpoint{0.936467in}{1.932974in}}%
\pgfpathlineto{\pgfqpoint{0.940006in}{1.985085in}}%
\pgfpathlineto{\pgfqpoint{0.945760in}{2.036936in}}%
\pgfpathlineto{\pgfqpoint{0.953411in}{2.085939in}}%
\pgfpathlineto{\pgfqpoint{0.962766in}{2.132001in}}%
\pgfpathlineto{\pgfqpoint{0.974289in}{2.177414in}}%
\pgfpathlineto{\pgfqpoint{0.987334in}{2.219653in}}%
\pgfpathlineto{\pgfqpoint{1.001670in}{2.258655in}}%
\pgfpathlineto{\pgfqpoint{1.018053in}{2.296584in}}%
\pgfpathlineto{\pgfqpoint{1.035404in}{2.331102in}}%
\pgfpathlineto{\pgfqpoint{1.054653in}{2.364276in}}%
\pgfpathlineto{\pgfqpoint{1.074410in}{2.393984in}}%
\pgfpathlineto{\pgfqpoint{1.095775in}{2.422197in}}%
\pgfpathlineto{\pgfqpoint{1.118666in}{2.448796in}}%
\pgfpathlineto{\pgfqpoint{1.142970in}{2.473700in}}%
\pgfpathlineto{\pgfqpoint{1.168554in}{2.496866in}}%
\pgfpathlineto{\pgfqpoint{1.197089in}{2.519661in}}%
\pgfpathlineto{\pgfqpoint{1.226731in}{2.540525in}}%
\pgfpathlineto{\pgfqpoint{1.259246in}{2.560671in}}%
\pgfpathlineto{\pgfqpoint{1.294617in}{2.579879in}}%
\pgfpathlineto{\pgfqpoint{1.332797in}{2.597979in}}%
\pgfpathlineto{\pgfqpoint{1.373723in}{2.614857in}}%
\pgfpathlineto{\pgfqpoint{1.417324in}{2.630442in}}%
\pgfpathlineto{\pgfqpoint{1.465637in}{2.645309in}}%
\pgfpathlineto{\pgfqpoint{1.518645in}{2.659201in}}%
\pgfpathlineto{\pgfqpoint{1.576314in}{2.671926in}}%
\pgfpathlineto{\pgfqpoint{1.638602in}{2.683341in}}%
\pgfpathlineto{\pgfqpoint{1.705467in}{2.693339in}}%
\pgfpathlineto{\pgfqpoint{1.779032in}{2.702061in}}%
\pgfpathlineto{\pgfqpoint{1.857102in}{2.709073in}}%
\pgfpathlineto{\pgfqpoint{1.939638in}{2.714276in}}%
\pgfpathlineto{\pgfqpoint{2.026603in}{2.717510in}}%
\pgfpathlineto{\pgfqpoint{2.113610in}{2.718519in}}%
\pgfpathlineto{\pgfqpoint{2.198439in}{2.717299in}}%
\pgfpathlineto{\pgfqpoint{2.278871in}{2.713924in}}%
\pgfpathlineto{\pgfqpoint{2.352682in}{2.708593in}}%
\pgfpathlineto{\pgfqpoint{2.417661in}{2.701703in}}%
\pgfpathlineto{\pgfqpoint{2.473775in}{2.693624in}}%
\pgfpathlineto{\pgfqpoint{2.523145in}{2.684361in}}%
\pgfpathlineto{\pgfqpoint{2.565731in}{2.674195in}}%
\pgfpathlineto{\pgfqpoint{2.601514in}{2.663535in}}%
\pgfpathlineto{\pgfqpoint{2.632581in}{2.652133in}}%
\pgfpathlineto{\pgfqpoint{2.658903in}{2.640320in}}%
\pgfpathlineto{\pgfqpoint{2.682441in}{2.627424in}}%
\pgfpathlineto{\pgfqpoint{2.703065in}{2.613558in}}%
\pgfpathlineto{\pgfqpoint{2.720676in}{2.598964in}}%
\pgfpathlineto{\pgfqpoint{2.735263in}{2.584037in}}%
\pgfpathlineto{\pgfqpoint{2.748319in}{2.567360in}}%
\pgfpathlineto{\pgfqpoint{2.759552in}{2.549028in}}%
\pgfpathlineto{\pgfqpoint{2.768785in}{2.529288in}}%
\pgfpathlineto{\pgfqpoint{2.776013in}{2.508479in}}%
\pgfpathlineto{\pgfqpoint{2.781879in}{2.484520in}}%
\pgfpathlineto{\pgfqpoint{2.786096in}{2.457577in}}%
\pgfpathlineto{\pgfqpoint{2.788713in}{2.425365in}}%
\pgfpathlineto{\pgfqpoint{2.789420in}{2.388041in}}%
\pgfpathlineto{\pgfqpoint{2.787955in}{2.340781in}}%
\pgfpathlineto{\pgfqpoint{2.783664in}{2.278749in}}%
\pgfpathlineto{\pgfqpoint{2.774281in}{2.179763in}}%
\pgfpathlineto{\pgfqpoint{2.743605in}{1.868099in}}%
\pgfpathlineto{\pgfqpoint{2.730106in}{1.702040in}}%
\pgfpathlineto{\pgfqpoint{2.717282in}{1.515929in}}%
\pgfpathlineto{\pgfqpoint{2.702597in}{1.267577in}}%
\pgfpathlineto{\pgfqpoint{2.684427in}{0.964610in}}%
\pgfpathlineto{\pgfqpoint{2.675366in}{0.850580in}}%
\pgfpathlineto{\pgfqpoint{2.667020in}{0.771504in}}%
\pgfpathlineto{\pgfqpoint{2.658741in}{0.712524in}}%
\pgfpathlineto{\pgfqpoint{2.650163in}{0.666265in}}%
\pgfpathlineto{\pgfqpoint{2.640805in}{0.627913in}}%
\pgfpathlineto{\pgfqpoint{2.631127in}{0.597517in}}%
\pgfpathlineto{\pgfqpoint{2.620984in}{0.572729in}}%
\pgfpathlineto{\pgfqpoint{2.609834in}{0.551369in}}%
\pgfpathlineto{\pgfqpoint{2.598017in}{0.533521in}}%
\pgfpathlineto{\pgfqpoint{2.584469in}{0.517368in}}%
\pgfpathlineto{\pgfqpoint{2.571081in}{0.504661in}}%
\pgfpathlineto{\pgfqpoint{2.554760in}{0.492307in}}%
\pgfpathlineto{\pgfqpoint{2.537426in}{0.481910in}}%
\pgfpathlineto{\pgfqpoint{2.517343in}{0.472364in}}%
\pgfpathlineto{\pgfqpoint{2.492511in}{0.463177in}}%
\pgfpathlineto{\pgfqpoint{2.462948in}{0.454833in}}%
\pgfpathlineto{\pgfqpoint{2.428734in}{0.447543in}}%
\pgfpathlineto{\pgfqpoint{2.385639in}{0.440736in}}%
\pgfpathlineto{\pgfqpoint{2.331525in}{0.434583in}}%
\pgfpathlineto{\pgfqpoint{2.262083in}{0.429079in}}%
\pgfpathlineto{\pgfqpoint{2.170819in}{0.424238in}}%
\pgfpathlineto{\pgfqpoint{2.049054in}{0.420136in}}%
\pgfpathlineto{\pgfqpoint{1.879404in}{0.416785in}}%
\pgfpathlineto{\pgfqpoint{1.640127in}{0.414419in}}%
\pgfpathlineto{\pgfqpoint{1.322530in}{0.413571in}}%
\pgfpathlineto{\pgfqpoint{1.020162in}{0.414852in}}%
\pgfpathlineto{\pgfqpoint{0.822224in}{0.417718in}}%
\pgfpathlineto{\pgfqpoint{0.704803in}{0.421435in}}%
\pgfpathlineto{\pgfqpoint{0.630945in}{0.425836in}}%
\pgfpathlineto{\pgfqpoint{0.583284in}{0.430744in}}%
\pgfpathlineto{\pgfqpoint{0.551002in}{0.436136in}}%
\pgfpathlineto{\pgfqpoint{0.527678in}{0.442206in}}%
\pgfpathlineto{\pgfqpoint{0.511221in}{0.448647in}}%
\pgfpathlineto{\pgfqpoint{0.499522in}{0.455243in}}%
\pgfpathlineto{\pgfqpoint{0.488893in}{0.463875in}}%
\pgfpathlineto{\pgfqpoint{0.481304in}{0.472768in}}%
\pgfpathlineto{\pgfqpoint{0.474065in}{0.485169in}}%
\pgfpathlineto{\pgfqpoint{0.468744in}{0.498793in}}%
\pgfpathlineto{\pgfqpoint{0.463864in}{0.517894in}}%
\pgfpathlineto{\pgfqpoint{0.459676in}{0.544843in}}%
\pgfpathlineto{\pgfqpoint{0.456385in}{0.581985in}}%
\pgfpathlineto{\pgfqpoint{0.453731in}{0.639153in}}%
\pgfpathlineto{\pgfqpoint{0.451681in}{0.736201in}}%
\pgfpathlineto{\pgfqpoint{0.450220in}{0.927862in}}%
\pgfpathlineto{\pgfqpoint{0.449345in}{1.403299in}}%
\pgfpathlineto{\pgfqpoint{0.449543in}{2.682749in}}%
\pgfpathlineto{\pgfqpoint{0.451013in}{2.856979in}}%
\pgfpathlineto{\pgfqpoint{0.452812in}{2.879265in}}%
\pgfpathlineto{\pgfqpoint{0.455213in}{2.886145in}}%
\pgfpathlineto{\pgfqpoint{0.458666in}{2.889045in}}%
\pgfpathlineto{\pgfqpoint{0.465038in}{2.890558in}}%
\pgfpathlineto{\pgfqpoint{0.482419in}{2.891423in}}%
\pgfpathlineto{\pgfqpoint{0.567256in}{2.891731in}}%
\pgfpathlineto{\pgfqpoint{2.857879in}{2.891760in}}%
\pgfpathlineto{\pgfqpoint{4.789552in}{2.890878in}}%
\pgfpathlineto{\pgfqpoint{4.793765in}{2.889709in}}%
\pgfpathlineto{\pgfqpoint{4.795507in}{2.888267in}}%
\pgfpathlineto{\pgfqpoint{4.797109in}{2.881095in}}%
\pgfpathlineto{\pgfqpoint{4.798039in}{2.856232in}}%
\pgfpathlineto{\pgfqpoint{4.798039in}{2.856232in}}%
\pgfusepath{stroke}%
\end{pgfscope}%
\begin{pgfscope}%
\pgfpathrectangle{\pgfqpoint{0.448634in}{0.402556in}}{\pgfqpoint{4.350661in}{2.489204in}} %
\pgfusepath{clip}%
\pgfsetrectcap%
\pgfsetroundjoin%
\pgfsetlinewidth{1.003750pt}%
\definecolor{currentstroke}{rgb}{0.580392,0.403922,0.741176}%
\pgfsetstrokecolor{currentstroke}%
\pgfsetdash{}{0pt}%
\pgfpathmoveto{\pgfqpoint{3.427626in}{0.402607in}}%
\pgfpathlineto{\pgfqpoint{2.805485in}{0.403719in}}%
\pgfpathlineto{\pgfqpoint{2.768545in}{0.405523in}}%
\pgfpathlineto{\pgfqpoint{2.753492in}{0.408056in}}%
\pgfpathlineto{\pgfqpoint{2.745293in}{0.411321in}}%
\pgfpathlineto{\pgfqpoint{2.739958in}{0.415576in}}%
\pgfpathlineto{\pgfqpoint{2.735993in}{0.421473in}}%
\pgfpathlineto{\pgfqpoint{2.732715in}{0.430672in}}%
\pgfpathlineto{\pgfqpoint{2.730059in}{0.445282in}}%
\pgfpathlineto{\pgfqpoint{2.727854in}{0.472539in}}%
\pgfpathlineto{\pgfqpoint{2.726224in}{0.524776in}}%
\pgfpathlineto{\pgfqpoint{2.725625in}{0.624340in}}%
\pgfpathlineto{\pgfqpoint{2.726994in}{0.783640in}}%
\pgfpathlineto{\pgfqpoint{2.730969in}{0.980233in}}%
\pgfpathlineto{\pgfqpoint{2.737258in}{1.176745in}}%
\pgfpathlineto{\pgfqpoint{2.744982in}{1.345778in}}%
\pgfpathlineto{\pgfqpoint{2.753863in}{1.494783in}}%
\pgfpathlineto{\pgfqpoint{2.764083in}{1.631187in}}%
\pgfpathlineto{\pgfqpoint{2.777155in}{1.774766in}}%
\pgfpathlineto{\pgfqpoint{2.790158in}{1.888296in}}%
\pgfpathlineto{\pgfqpoint{2.807625in}{2.018663in}}%
\pgfpathlineto{\pgfqpoint{2.823342in}{2.114058in}}%
\pgfpathlineto{\pgfqpoint{2.840800in}{2.206512in}}%
\pgfpathlineto{\pgfqpoint{2.861816in}{2.305693in}}%
\pgfpathlineto{\pgfqpoint{2.884211in}{2.412095in}}%
\pgfpathlineto{\pgfqpoint{2.895378in}{2.470448in}}%
\pgfpathlineto{\pgfqpoint{2.900738in}{2.509791in}}%
\pgfpathlineto{\pgfqpoint{2.902598in}{2.539575in}}%
\pgfpathlineto{\pgfqpoint{2.901890in}{2.564440in}}%
\pgfpathlineto{\pgfqpoint{2.899255in}{2.584112in}}%
\pgfpathlineto{\pgfqpoint{2.895074in}{2.600854in}}%
\pgfpathlineto{\pgfqpoint{2.888952in}{2.616792in}}%
\pgfpathlineto{\pgfqpoint{2.880901in}{2.631565in}}%
\pgfpathlineto{\pgfqpoint{2.871139in}{2.644921in}}%
\pgfpathlineto{\pgfqpoint{2.858319in}{2.658367in}}%
\pgfpathlineto{\pgfqpoint{2.844202in}{2.669994in}}%
\pgfpathlineto{\pgfqpoint{2.825324in}{2.682342in}}%
\pgfpathlineto{\pgfqpoint{2.803551in}{2.693681in}}%
\pgfpathlineto{\pgfqpoint{2.779050in}{2.703972in}}%
\pgfpathlineto{\pgfqpoint{2.747766in}{2.714563in}}%
\pgfpathlineto{\pgfqpoint{2.711783in}{2.724304in}}%
\pgfpathlineto{\pgfqpoint{2.669031in}{2.733517in}}%
\pgfpathlineto{\pgfqpoint{2.617387in}{2.742252in}}%
\pgfpathlineto{\pgfqpoint{2.556867in}{2.750101in}}%
\pgfpathlineto{\pgfqpoint{2.487507in}{2.756814in}}%
\pgfpathlineto{\pgfqpoint{2.407165in}{2.762324in}}%
\pgfpathlineto{\pgfqpoint{2.315870in}{2.766321in}}%
\pgfpathlineto{\pgfqpoint{2.137511in}{2.768628in}}%
\pgfpathlineto{\pgfqpoint{2.028754in}{2.767063in}}%
\pgfpathlineto{\pgfqpoint{1.915692in}{2.763144in}}%
\pgfpathlineto{\pgfqpoint{1.815757in}{2.757346in}}%
\pgfpathlineto{\pgfqpoint{1.711606in}{2.748969in}}%
\pgfpathlineto{\pgfqpoint{1.629328in}{2.739872in}}%
\pgfpathlineto{\pgfqpoint{1.553753in}{2.729339in}}%
\pgfpathlineto{\pgfqpoint{1.454779in}{2.712885in}}%
\pgfpathlineto{\pgfqpoint{1.395024in}{2.699403in}}%
\pgfpathlineto{\pgfqpoint{1.342079in}{2.685200in}}%
\pgfpathlineto{\pgfqpoint{1.293848in}{2.669989in}}%
\pgfpathlineto{\pgfqpoint{1.250351in}{2.654028in}}%
\pgfpathlineto{\pgfqpoint{1.207529in}{2.635842in}}%
\pgfpathlineto{\pgfqpoint{1.171571in}{2.618120in}}%
\pgfpathlineto{\pgfqpoint{1.138391in}{2.599443in}}%
\pgfpathlineto{\pgfqpoint{1.106152in}{2.578726in}}%
\pgfpathlineto{\pgfqpoint{1.088056in}{2.564963in}}%
\pgfpathlineto{\pgfqpoint{1.059488in}{2.542224in}}%
\pgfpathlineto{\pgfqpoint{1.033881in}{2.519092in}}%
\pgfpathlineto{\pgfqpoint{1.009553in}{2.494217in}}%
\pgfpathlineto{\pgfqpoint{0.986623in}{2.467661in}}%
\pgfpathlineto{\pgfqpoint{0.965180in}{2.439526in}}%
\pgfpathlineto{\pgfqpoint{0.944005in}{2.407926in}}%
\pgfpathlineto{\pgfqpoint{0.927041in}{2.378997in}}%
\pgfpathlineto{\pgfqpoint{0.883835in}{2.286967in}}%
\pgfpathlineto{\pgfqpoint{0.867783in}{2.243391in}}%
\pgfpathlineto{\pgfqpoint{0.853017in}{2.196568in}}%
\pgfpathlineto{\pgfqpoint{0.843727in}{2.160806in}}%
\pgfpathlineto{\pgfqpoint{0.824433in}{2.073976in}}%
\pgfpathlineto{\pgfqpoint{0.814311in}{2.015376in}}%
\pgfpathlineto{\pgfqpoint{0.810519in}{1.988345in}}%
\pgfpathlineto{\pgfqpoint{0.794628in}{1.835123in}}%
\pgfpathlineto{\pgfqpoint{0.789429in}{1.753197in}}%
\pgfpathlineto{\pgfqpoint{0.785628in}{1.658710in}}%
\pgfpathlineto{\pgfqpoint{0.778013in}{1.464769in}}%
\pgfpathlineto{\pgfqpoint{0.773308in}{1.417793in}}%
\pgfpathlineto{\pgfqpoint{0.768283in}{1.388491in}}%
\pgfpathlineto{\pgfqpoint{0.762681in}{1.367036in}}%
\pgfpathlineto{\pgfqpoint{0.756585in}{1.351084in}}%
\pgfpathlineto{\pgfqpoint{0.749532in}{1.338542in}}%
\pgfpathlineto{\pgfqpoint{0.741894in}{1.329721in}}%
\pgfpathlineto{\pgfqpoint{0.734479in}{1.324560in}}%
\pgfpathlineto{\pgfqpoint{0.726142in}{1.321825in}}%
\pgfpathlineto{\pgfqpoint{0.717466in}{1.321782in}}%
\pgfpathlineto{\pgfqpoint{0.709018in}{1.324088in}}%
\pgfpathlineto{\pgfqpoint{0.699193in}{1.329380in}}%
\pgfpathlineto{\pgfqpoint{0.688582in}{1.338049in}}%
\pgfpathlineto{\pgfqpoint{0.677635in}{1.350142in}}%
\pgfpathlineto{\pgfqpoint{0.666648in}{1.365572in}}%
\pgfpathlineto{\pgfqpoint{0.654705in}{1.386366in}}%
\pgfpathlineto{\pgfqpoint{0.642394in}{1.412696in}}%
\pgfpathlineto{\pgfqpoint{0.630171in}{1.444606in}}%
\pgfpathlineto{\pgfqpoint{0.618367in}{1.482066in}}%
\pgfpathlineto{\pgfqpoint{0.608484in}{1.520245in}}%
\pgfpathlineto{\pgfqpoint{0.590093in}{1.612439in}}%
\pgfpathlineto{\pgfqpoint{0.581746in}{1.668880in}}%
\pgfpathlineto{\pgfqpoint{0.573046in}{1.740374in}}%
\pgfpathlineto{\pgfqpoint{0.566968in}{1.807211in}}%
\pgfpathlineto{\pgfqpoint{0.560446in}{1.896508in}}%
\pgfpathlineto{\pgfqpoint{0.555446in}{1.995909in}}%
\pgfpathlineto{\pgfqpoint{0.552481in}{2.097907in}}%
\pgfpathlineto{\pgfqpoint{0.551445in}{2.204934in}}%
\pgfpathlineto{\pgfqpoint{0.552648in}{2.309469in}}%
\pgfpathlineto{\pgfqpoint{0.555929in}{2.403980in}}%
\pgfpathlineto{\pgfqpoint{0.560747in}{2.480945in}}%
\pgfpathlineto{\pgfqpoint{0.567009in}{2.547766in}}%
\pgfpathlineto{\pgfqpoint{0.574530in}{2.604361in}}%
\pgfpathlineto{\pgfqpoint{0.582465in}{2.648230in}}%
\pgfpathlineto{\pgfqpoint{0.592087in}{2.689074in}}%
\pgfpathlineto{\pgfqpoint{0.601811in}{2.719451in}}%
\pgfpathlineto{\pgfqpoint{0.611962in}{2.744235in}}%
\pgfpathlineto{\pgfqpoint{0.623074in}{2.765621in}}%
\pgfpathlineto{\pgfqpoint{0.634821in}{2.783529in}}%
\pgfpathlineto{\pgfqpoint{0.648293in}{2.799766in}}%
\pgfpathlineto{\pgfqpoint{0.661637in}{2.812534in}}%
\pgfpathlineto{\pgfqpoint{0.677958in}{2.824886in}}%
\pgfpathlineto{\pgfqpoint{0.695343in}{2.835167in}}%
\pgfpathlineto{\pgfqpoint{0.713524in}{2.843448in}}%
\pgfpathlineto{\pgfqpoint{0.736339in}{2.851680in}}%
\pgfpathlineto{\pgfqpoint{0.763850in}{2.859147in}}%
\pgfpathlineto{\pgfqpoint{0.795994in}{2.865541in}}%
\pgfpathlineto{\pgfqpoint{0.837010in}{2.871332in}}%
\pgfpathlineto{\pgfqpoint{0.889034in}{2.876308in}}%
\pgfpathlineto{\pgfqpoint{0.958544in}{2.880565in}}%
\pgfpathlineto{\pgfqpoint{1.054209in}{2.884046in}}%
\pgfpathlineto{\pgfqpoint{1.193409in}{2.886752in}}%
\pgfpathlineto{\pgfqpoint{2.117914in}{2.890561in}}%
\pgfpathlineto{\pgfqpoint{3.407885in}{2.890611in}}%
\pgfpathlineto{\pgfqpoint{4.075709in}{2.888960in}}%
\pgfpathlineto{\pgfqpoint{4.312809in}{2.886381in}}%
\pgfpathlineto{\pgfqpoint{4.430239in}{2.883069in}}%
\pgfpathlineto{\pgfqpoint{4.501935in}{2.879022in}}%
\pgfpathlineto{\pgfqpoint{4.551778in}{2.874107in}}%
\pgfpathlineto{\pgfqpoint{4.586248in}{2.868631in}}%
\pgfpathlineto{\pgfqpoint{4.613920in}{2.862001in}}%
\pgfpathlineto{\pgfqpoint{4.634734in}{2.854797in}}%
\pgfpathlineto{\pgfqpoint{4.650800in}{2.847167in}}%
\pgfpathlineto{\pgfqpoint{4.665966in}{2.837429in}}%
\pgfpathlineto{\pgfqpoint{4.678130in}{2.826969in}}%
\pgfpathlineto{\pgfqpoint{4.688958in}{2.814737in}}%
\pgfpathlineto{\pgfqpoint{4.698293in}{2.800984in}}%
\pgfpathlineto{\pgfqpoint{4.707212in}{2.783899in}}%
\pgfpathlineto{\pgfqpoint{4.715320in}{2.763518in}}%
\pgfpathlineto{\pgfqpoint{4.723087in}{2.737629in}}%
\pgfpathlineto{\pgfqpoint{4.730052in}{2.706274in}}%
\pgfpathlineto{\pgfqpoint{4.736512in}{2.667144in}}%
\pgfpathlineto{\pgfqpoint{4.742669in}{2.615353in}}%
\pgfpathlineto{\pgfqpoint{4.748203in}{2.548447in}}%
\pgfpathlineto{\pgfqpoint{4.752888in}{2.463986in}}%
\pgfpathlineto{\pgfqpoint{4.756813in}{2.352064in}}%
\pgfpathlineto{\pgfqpoint{4.759543in}{2.207726in}}%
\pgfpathlineto{\pgfqpoint{4.760603in}{2.030998in}}%
\pgfpathlineto{\pgfqpoint{4.759524in}{1.839334in}}%
\pgfpathlineto{\pgfqpoint{4.756287in}{1.655172in}}%
\pgfpathlineto{\pgfqpoint{4.751200in}{1.493481in}}%
\pgfpathlineto{\pgfqpoint{4.744975in}{1.366734in}}%
\pgfpathlineto{\pgfqpoint{4.737712in}{1.262521in}}%
\pgfpathlineto{\pgfqpoint{4.729349in}{1.173427in}}%
\pgfpathlineto{\pgfqpoint{4.720412in}{1.101972in}}%
\pgfpathlineto{\pgfqpoint{4.710604in}{1.040767in}}%
\pgfpathlineto{\pgfqpoint{4.699819in}{0.987419in}}%
\pgfpathlineto{\pgfqpoint{4.688393in}{0.941973in}}%
\pgfpathlineto{\pgfqpoint{4.676092in}{0.902074in}}%
\pgfpathlineto{\pgfqpoint{4.676092in}{0.902074in}}%
\pgfusepath{stroke}%
\end{pgfscope}%
\begin{pgfscope}%
\pgfpathrectangle{\pgfqpoint{0.448634in}{0.402556in}}{\pgfqpoint{4.350661in}{2.489204in}} %
\pgfusepath{clip}%
\pgfsetrectcap%
\pgfsetroundjoin%
\pgfsetlinewidth{1.003750pt}%
\definecolor{currentstroke}{rgb}{0.580392,0.403922,0.741176}%
\pgfsetstrokecolor{currentstroke}%
\pgfsetdash{}{0pt}%
\pgfpathmoveto{\pgfqpoint{2.795521in}{1.982745in}}%
\pgfpathlineto{\pgfqpoint{2.781780in}{1.874357in}}%
\pgfpathlineto{\pgfqpoint{2.769352in}{1.758234in}}%
\pgfpathlineto{\pgfqpoint{2.758095in}{1.631942in}}%
\pgfpathlineto{\pgfqpoint{2.747786in}{1.490551in}}%
\pgfpathlineto{\pgfqpoint{2.738644in}{1.334082in}}%
\pgfpathlineto{\pgfqpoint{2.730580in}{1.157591in}}%
\pgfpathlineto{\pgfqpoint{2.723334in}{0.948663in}}%
\pgfpathlineto{\pgfqpoint{2.709609in}{0.528306in}}%
\pgfpathlineto{\pgfqpoint{2.705546in}{0.486255in}}%
\pgfpathlineto{\pgfqpoint{2.701218in}{0.461873in}}%
\pgfpathlineto{\pgfqpoint{2.696134in}{0.445472in}}%
\pgfpathlineto{\pgfqpoint{2.690586in}{0.434794in}}%
\pgfpathlineto{\pgfqpoint{2.684630in}{0.427561in}}%
\pgfpathlineto{\pgfqpoint{2.677469in}{0.421936in}}%
\pgfpathlineto{\pgfqpoint{2.667488in}{0.417033in}}%
\pgfpathlineto{\pgfqpoint{2.654858in}{0.413305in}}%
\pgfpathlineto{\pgfqpoint{2.635492in}{0.410064in}}%
\pgfpathlineto{\pgfqpoint{2.605125in}{0.407486in}}%
\pgfpathlineto{\pgfqpoint{2.550772in}{0.405469in}}%
\pgfpathlineto{\pgfqpoint{2.442013in}{0.404079in}}%
\pgfpathlineto{\pgfqpoint{2.148345in}{0.403239in}}%
\pgfpathlineto{\pgfqpoint{1.021524in}{0.402980in}}%
\pgfpathlineto{\pgfqpoint{0.512500in}{0.404247in}}%
\pgfpathlineto{\pgfqpoint{0.464682in}{0.406193in}}%
\pgfpathlineto{\pgfqpoint{0.456130in}{0.407928in}}%
\pgfpathlineto{\pgfqpoint{0.452340in}{0.410300in}}%
\pgfpathlineto{\pgfqpoint{0.450346in}{0.414659in}}%
\pgfpathlineto{\pgfqpoint{0.449266in}{0.424520in}}%
\pgfpathlineto{\pgfqpoint{0.448771in}{0.464341in}}%
\pgfpathlineto{\pgfqpoint{0.448640in}{0.850168in}}%
\pgfpathlineto{\pgfqpoint{0.448679in}{2.891315in}}%
\pgfpathlineto{\pgfqpoint{0.448679in}{2.891315in}}%
\pgfusepath{stroke}%
\end{pgfscope}%
\begin{pgfscope}%
\pgfpathrectangle{\pgfqpoint{0.448634in}{0.402556in}}{\pgfqpoint{4.350661in}{2.489204in}} %
\pgfusepath{clip}%
\pgfsetrectcap%
\pgfsetroundjoin%
\pgfsetlinewidth{1.003750pt}%
\definecolor{currentstroke}{rgb}{0.580392,0.403922,0.741176}%
\pgfsetstrokecolor{currentstroke}%
\pgfsetdash{}{0pt}%
\pgfpathmoveto{\pgfqpoint{3.427603in}{0.402582in}}%
\pgfpathlineto{\pgfqpoint{2.777184in}{0.403702in}}%
\pgfpathlineto{\pgfqpoint{2.751151in}{0.405685in}}%
\pgfpathlineto{\pgfqpoint{2.742696in}{0.407954in}}%
\pgfpathlineto{\pgfqpoint{2.737083in}{0.411689in}}%
\pgfpathlineto{\pgfqpoint{2.733114in}{0.417564in}}%
\pgfpathlineto{\pgfqpoint{2.730222in}{0.426931in}}%
\pgfpathlineto{\pgfqpoint{2.727854in}{0.444128in}}%
\pgfpathlineto{\pgfqpoint{2.726081in}{0.478913in}}%
\pgfpathlineto{\pgfqpoint{2.724952in}{0.551086in}}%
\pgfpathlineto{\pgfqpoint{2.725324in}{0.687990in}}%
\pgfpathlineto{\pgfqpoint{2.728060in}{0.872163in}}%
\pgfpathlineto{\pgfqpoint{2.733125in}{1.071213in}}%
\pgfpathlineto{\pgfqpoint{2.739986in}{1.250262in}}%
\pgfpathlineto{\pgfqpoint{2.748886in}{1.421713in}}%
\pgfpathlineto{\pgfqpoint{2.758913in}{1.570622in}}%
\pgfpathlineto{\pgfqpoint{2.771658in}{1.724246in}}%
\pgfpathlineto{\pgfqpoint{2.783932in}{1.840390in}}%
\pgfpathlineto{\pgfqpoint{2.800716in}{1.975907in}}%
\pgfpathlineto{\pgfqpoint{2.815799in}{2.073965in}}%
\pgfpathlineto{\pgfqpoint{2.832722in}{2.169090in}}%
\pgfpathlineto{\pgfqpoint{2.852301in}{2.266102in}}%
\pgfpathlineto{\pgfqpoint{2.892442in}{2.459809in}}%
\pgfpathlineto{\pgfqpoint{2.898568in}{2.501533in}}%
\pgfpathlineto{\pgfqpoint{2.900922in}{2.531273in}}%
\pgfpathlineto{\pgfqpoint{2.900853in}{2.556153in}}%
\pgfpathlineto{\pgfqpoint{2.898544in}{2.578385in}}%
\pgfpathlineto{\pgfqpoint{2.894184in}{2.597646in}}%
\pgfpathlineto{\pgfqpoint{2.888326in}{2.613714in}}%
\pgfpathlineto{\pgfqpoint{2.880536in}{2.628669in}}%
\pgfpathlineto{\pgfqpoint{2.871005in}{2.642241in}}%
\pgfpathlineto{\pgfqpoint{2.858390in}{2.655938in}}%
\pgfpathlineto{\pgfqpoint{2.844425in}{2.667804in}}%
\pgfpathlineto{\pgfqpoint{2.827601in}{2.679244in}}%
\pgfpathlineto{\pgfqpoint{2.807999in}{2.690022in}}%
\pgfpathlineto{\pgfqpoint{2.783672in}{2.700837in}}%
\pgfpathlineto{\pgfqpoint{2.754613in}{2.711246in}}%
\pgfpathlineto{\pgfqpoint{2.720867in}{2.720978in}}%
\pgfpathlineto{\pgfqpoint{2.680355in}{2.730327in}}%
\pgfpathlineto{\pgfqpoint{2.633099in}{2.738961in}}%
\pgfpathlineto{\pgfqpoint{2.576975in}{2.746949in}}%
\pgfpathlineto{\pgfqpoint{2.509837in}{2.754163in}}%
\pgfpathlineto{\pgfqpoint{2.433877in}{2.760064in}}%
\pgfpathlineto{\pgfqpoint{2.346955in}{2.764581in}}%
\pgfpathlineto{\pgfqpoint{2.253446in}{2.767226in}}%
\pgfpathlineto{\pgfqpoint{2.149035in}{2.768108in}}%
\pgfpathlineto{\pgfqpoint{2.040276in}{2.766784in}}%
\pgfpathlineto{\pgfqpoint{1.929381in}{2.763194in}}%
\pgfpathlineto{\pgfqpoint{1.827262in}{2.757535in}}%
\pgfpathlineto{\pgfqpoint{1.725263in}{2.749605in}}%
\pgfpathlineto{\pgfqpoint{1.640795in}{2.740570in}}%
\pgfpathlineto{\pgfqpoint{1.565188in}{2.730329in}}%
\pgfpathlineto{\pgfqpoint{1.466153in}{2.714254in}}%
\pgfpathlineto{\pgfqpoint{1.406336in}{2.701133in}}%
\pgfpathlineto{\pgfqpoint{1.353313in}{2.687318in}}%
\pgfpathlineto{\pgfqpoint{1.304982in}{2.672527in}}%
\pgfpathlineto{\pgfqpoint{1.259300in}{2.656218in}}%
\pgfpathlineto{\pgfqpoint{1.216346in}{2.638441in}}%
\pgfpathlineto{\pgfqpoint{1.178263in}{2.620078in}}%
\pgfpathlineto{\pgfqpoint{1.143032in}{2.600537in}}%
\pgfpathlineto{\pgfqpoint{1.110705in}{2.579999in}}%
\pgfpathlineto{\pgfqpoint{1.092518in}{2.566418in}}%
\pgfpathlineto{\pgfqpoint{1.076358in}{2.553779in}}%
\pgfpathlineto{\pgfqpoint{1.048385in}{2.530091in}}%
\pgfpathlineto{\pgfqpoint{1.023412in}{2.506068in}}%
\pgfpathlineto{\pgfqpoint{0.999784in}{2.480326in}}%
\pgfpathlineto{\pgfqpoint{0.977604in}{2.452949in}}%
\pgfpathlineto{\pgfqpoint{0.956945in}{2.424055in}}%
\pgfpathlineto{\pgfqpoint{0.936621in}{2.391731in}}%
\pgfpathlineto{\pgfqpoint{0.921689in}{2.364262in}}%
\pgfpathlineto{\pgfqpoint{0.903519in}{2.327413in}}%
\pgfpathlineto{\pgfqpoint{0.886308in}{2.287175in}}%
\pgfpathlineto{\pgfqpoint{0.871046in}{2.245918in}}%
\pgfpathlineto{\pgfqpoint{0.856187in}{2.199134in}}%
\pgfpathlineto{\pgfqpoint{0.846086in}{2.161042in}}%
\pgfpathlineto{\pgfqpoint{0.824147in}{2.059582in}}%
\pgfpathlineto{\pgfqpoint{0.815068in}{2.003289in}}%
\pgfpathlineto{\pgfqpoint{0.807442in}{1.944193in}}%
\pgfpathlineto{\pgfqpoint{0.800113in}{1.872499in}}%
\pgfpathlineto{\pgfqpoint{0.794199in}{1.790638in}}%
\pgfpathlineto{\pgfqpoint{0.790185in}{1.703640in}}%
\pgfpathlineto{\pgfqpoint{0.786945in}{1.581727in}}%
\pgfpathlineto{\pgfqpoint{0.785487in}{1.522009in}}%
\pgfpathlineto{\pgfqpoint{0.785487in}{1.522009in}}%
\pgfusepath{stroke}%
\end{pgfscope}%
\begin{pgfscope}%
\pgfpathrectangle{\pgfqpoint{0.448634in}{0.402556in}}{\pgfqpoint{4.350661in}{2.489204in}} %
\pgfusepath{clip}%
\pgfsetrectcap%
\pgfsetroundjoin%
\pgfsetlinewidth{1.003750pt}%
\definecolor{currentstroke}{rgb}{0.549020,0.337255,0.294118}%
\pgfsetstrokecolor{currentstroke}%
\pgfsetdash{}{0pt}%
\pgfpathmoveto{\pgfqpoint{0.448634in}{2.896245in}}%
\pgfpathlineto{\pgfqpoint{0.448593in}{0.407043in}}%
\pgfpathlineto{\pgfqpoint{0.448593in}{0.407043in}}%
\pgfusepath{stroke}%
\end{pgfscope}%
\begin{pgfscope}%
\pgfpathrectangle{\pgfqpoint{0.448634in}{0.402556in}}{\pgfqpoint{4.350661in}{2.489204in}} %
\pgfusepath{clip}%
\pgfsetrectcap%
\pgfsetroundjoin%
\pgfsetlinewidth{1.003750pt}%
\definecolor{currentstroke}{rgb}{0.549020,0.337255,0.294118}%
\pgfsetstrokecolor{currentstroke}%
\pgfsetdash{}{0pt}%
\pgfpathmoveto{\pgfqpoint{0.572611in}{1.810068in}}%
\pgfpathlineto{\pgfqpoint{0.565799in}{1.896838in}}%
\pgfpathlineto{\pgfqpoint{0.560511in}{1.993725in}}%
\pgfpathlineto{\pgfqpoint{0.557175in}{2.095709in}}%
\pgfpathlineto{\pgfqpoint{0.555948in}{2.200244in}}%
\pgfpathlineto{\pgfqpoint{0.556954in}{2.302292in}}%
\pgfpathlineto{\pgfqpoint{0.560041in}{2.394321in}}%
\pgfpathlineto{\pgfqpoint{0.564865in}{2.473780in}}%
\pgfpathlineto{\pgfqpoint{0.571081in}{2.540607in}}%
\pgfpathlineto{\pgfqpoint{0.578203in}{2.594755in}}%
\pgfpathlineto{\pgfqpoint{0.586478in}{2.641085in}}%
\pgfpathlineto{\pgfqpoint{0.595597in}{2.679513in}}%
\pgfpathlineto{\pgfqpoint{0.604986in}{2.710029in}}%
\pgfpathlineto{\pgfqpoint{0.615722in}{2.737244in}}%
\pgfpathlineto{\pgfqpoint{0.626542in}{2.758826in}}%
\pgfpathlineto{\pgfqpoint{0.637958in}{2.777013in}}%
\pgfpathlineto{\pgfqpoint{0.651058in}{2.793646in}}%
\pgfpathlineto{\pgfqpoint{0.665778in}{2.808396in}}%
\pgfpathlineto{\pgfqpoint{0.681903in}{2.821081in}}%
\pgfpathlineto{\pgfqpoint{0.699121in}{2.831725in}}%
\pgfpathlineto{\pgfqpoint{0.719164in}{2.841375in}}%
\pgfpathlineto{\pgfqpoint{0.741921in}{2.849815in}}%
\pgfpathlineto{\pgfqpoint{0.769386in}{2.857500in}}%
\pgfpathlineto{\pgfqpoint{0.801494in}{2.864121in}}%
\pgfpathlineto{\pgfqpoint{0.842485in}{2.870148in}}%
\pgfpathlineto{\pgfqpoint{0.894492in}{2.875355in}}%
\pgfpathlineto{\pgfqpoint{0.961817in}{2.879720in}}%
\pgfpathlineto{\pgfqpoint{1.055300in}{2.883382in}}%
\pgfpathlineto{\pgfqpoint{1.190146in}{2.886280in}}%
\pgfpathlineto{\pgfqpoint{1.636078in}{2.889550in}}%
\pgfpathlineto{\pgfqpoint{2.290851in}{2.890593in}}%
\pgfpathlineto{\pgfqpoint{3.585173in}{2.890306in}}%
\pgfpathlineto{\pgfqpoint{4.113775in}{2.888436in}}%
\pgfpathlineto{\pgfqpoint{4.322592in}{2.885702in}}%
\pgfpathlineto{\pgfqpoint{4.433488in}{2.882162in}}%
\pgfpathlineto{\pgfqpoint{4.502993in}{2.877825in}}%
\pgfpathlineto{\pgfqpoint{4.550639in}{2.872741in}}%
\pgfpathlineto{\pgfqpoint{4.585061in}{2.866890in}}%
\pgfpathlineto{\pgfqpoint{4.610551in}{2.860482in}}%
\pgfpathlineto{\pgfqpoint{4.631317in}{2.853102in}}%
\pgfpathlineto{\pgfqpoint{4.649285in}{2.844237in}}%
\pgfpathlineto{\pgfqpoint{4.664267in}{2.834134in}}%
\pgfpathlineto{\pgfqpoint{4.676247in}{2.823398in}}%
\pgfpathlineto{\pgfqpoint{4.686897in}{2.810962in}}%
\pgfpathlineto{\pgfqpoint{4.696037in}{2.797036in}}%
\pgfpathlineto{\pgfqpoint{4.704857in}{2.779882in}}%
\pgfpathlineto{\pgfqpoint{4.712922in}{2.759478in}}%
\pgfpathlineto{\pgfqpoint{4.720699in}{2.733593in}}%
\pgfpathlineto{\pgfqpoint{4.727707in}{2.702250in}}%
\pgfpathlineto{\pgfqpoint{4.734258in}{2.663140in}}%
\pgfpathlineto{\pgfqpoint{4.740534in}{2.611367in}}%
\pgfpathlineto{\pgfqpoint{4.746020in}{2.546956in}}%
\pgfpathlineto{\pgfqpoint{4.750752in}{2.464993in}}%
\pgfpathlineto{\pgfqpoint{4.754812in}{2.355569in}}%
\pgfpathlineto{\pgfqpoint{4.757659in}{2.216213in}}%
\pgfpathlineto{\pgfqpoint{4.758880in}{2.044465in}}%
\pgfpathlineto{\pgfqpoint{4.757981in}{1.857779in}}%
\pgfpathlineto{\pgfqpoint{4.754918in}{1.676103in}}%
\pgfpathlineto{\pgfqpoint{4.749899in}{1.511919in}}%
\pgfpathlineto{\pgfqpoint{4.743613in}{1.382683in}}%
\pgfpathlineto{\pgfqpoint{4.736212in}{1.275985in}}%
\pgfpathlineto{\pgfqpoint{4.727655in}{1.184411in}}%
\pgfpathlineto{\pgfqpoint{4.718699in}{1.112959in}}%
\pgfpathlineto{\pgfqpoint{4.708898in}{1.051753in}}%
\pgfpathlineto{\pgfqpoint{4.698166in}{0.998390in}}%
\pgfpathlineto{\pgfqpoint{4.686824in}{0.952917in}}%
\pgfpathlineto{\pgfqpoint{4.674640in}{0.912970in}}%
\pgfpathlineto{\pgfqpoint{4.661967in}{0.878570in}}%
\pgfpathlineto{\pgfqpoint{4.648255in}{0.847462in}}%
\pgfpathlineto{\pgfqpoint{4.633712in}{0.819718in}}%
\pgfpathlineto{\pgfqpoint{4.617337in}{0.793347in}}%
\pgfpathlineto{\pgfqpoint{4.600526in}{0.770506in}}%
\pgfpathlineto{\pgfqpoint{4.582145in}{0.749308in}}%
\pgfpathlineto{\pgfqpoint{4.562325in}{0.729881in}}%
\pgfpathlineto{\pgfqpoint{4.541242in}{0.712282in}}%
\pgfpathlineto{\pgfqpoint{4.517200in}{0.695258in}}%
\pgfpathlineto{\pgfqpoint{4.492163in}{0.680226in}}%
\pgfpathlineto{\pgfqpoint{4.466273in}{0.667228in}}%
\pgfpathlineto{\pgfqpoint{4.435759in}{0.654016in}}%
\pgfpathlineto{\pgfqpoint{4.402581in}{0.641998in}}%
\pgfpathlineto{\pgfqpoint{4.364704in}{0.630658in}}%
\pgfpathlineto{\pgfqpoint{4.322150in}{0.620315in}}%
\pgfpathlineto{\pgfqpoint{4.274963in}{0.611204in}}%
\pgfpathlineto{\pgfqpoint{4.223195in}{0.603491in}}%
\pgfpathlineto{\pgfqpoint{4.171218in}{0.597900in}}%
\pgfpathlineto{\pgfqpoint{4.108275in}{0.593103in}}%
\pgfpathlineto{\pgfqpoint{4.043060in}{0.590404in}}%
\pgfpathlineto{\pgfqpoint{3.973455in}{0.589639in}}%
\pgfpathlineto{\pgfqpoint{3.901683in}{0.591111in}}%
\pgfpathlineto{\pgfqpoint{3.829973in}{0.594847in}}%
\pgfpathlineto{\pgfqpoint{3.762729in}{0.600615in}}%
\pgfpathlineto{\pgfqpoint{3.693500in}{0.608905in}}%
\pgfpathlineto{\pgfqpoint{3.635351in}{0.618341in}}%
\pgfpathlineto{\pgfqpoint{3.583969in}{0.628898in}}%
\pgfpathlineto{\pgfqpoint{3.526559in}{0.642997in}}%
\pgfpathlineto{\pgfqpoint{3.484545in}{0.655913in}}%
\pgfpathlineto{\pgfqpoint{3.445120in}{0.670083in}}%
\pgfpathlineto{\pgfqpoint{3.402262in}{0.688122in}}%
\pgfpathlineto{\pgfqpoint{3.368464in}{0.705285in}}%
\pgfpathlineto{\pgfqpoint{3.337516in}{0.723497in}}%
\pgfpathlineto{\pgfqpoint{3.309509in}{0.742645in}}%
\pgfpathlineto{\pgfqpoint{3.284406in}{0.762365in}}%
\pgfpathlineto{\pgfqpoint{3.260471in}{0.783902in}}%
\pgfpathlineto{\pgfqpoint{3.237888in}{0.807272in}}%
\pgfpathlineto{\pgfqpoint{3.216807in}{0.832413in}}%
\pgfpathlineto{\pgfqpoint{3.197351in}{0.859213in}}%
\pgfpathlineto{\pgfqpoint{3.179599in}{0.887520in}}%
\pgfpathlineto{\pgfqpoint{3.163583in}{0.917154in}}%
\pgfpathlineto{\pgfqpoint{3.148298in}{0.950135in}}%
\pgfpathlineto{\pgfqpoint{3.134889in}{0.984168in}}%
\pgfpathlineto{\pgfqpoint{3.122555in}{1.021403in}}%
\pgfpathlineto{\pgfqpoint{3.110968in}{1.064196in}}%
\pgfpathlineto{\pgfqpoint{3.101545in}{1.107679in}}%
\pgfpathlineto{\pgfqpoint{3.093467in}{1.156592in}}%
\pgfpathlineto{\pgfqpoint{3.087589in}{1.205914in}}%
\pgfpathlineto{\pgfqpoint{3.083612in}{1.257984in}}%
\pgfpathlineto{\pgfqpoint{3.081646in}{1.317677in}}%
\pgfpathlineto{\pgfqpoint{3.082138in}{1.374921in}}%
\pgfpathlineto{\pgfqpoint{3.084852in}{1.432084in}}%
\pgfpathlineto{\pgfqpoint{3.089757in}{1.489055in}}%
\pgfpathlineto{\pgfqpoint{3.096835in}{1.545726in}}%
\pgfpathlineto{\pgfqpoint{3.105698in}{1.599537in}}%
\pgfpathlineto{\pgfqpoint{3.116164in}{1.650416in}}%
\pgfpathlineto{\pgfqpoint{3.129119in}{1.703128in}}%
\pgfpathlineto{\pgfqpoint{3.143607in}{1.752697in}}%
\pgfpathlineto{\pgfqpoint{3.159339in}{1.799108in}}%
\pgfpathlineto{\pgfqpoint{3.177098in}{1.844549in}}%
\pgfpathlineto{\pgfqpoint{3.196841in}{1.888905in}}%
\pgfpathlineto{\pgfqpoint{3.217497in}{1.929864in}}%
\pgfpathlineto{\pgfqpoint{3.239980in}{1.969543in}}%
\pgfpathlineto{\pgfqpoint{3.264223in}{2.007841in}}%
\pgfpathlineto{\pgfqpoint{3.291524in}{2.046596in}}%
\pgfpathlineto{\pgfqpoint{3.320432in}{2.083796in}}%
\pgfpathlineto{\pgfqpoint{3.356748in}{2.126710in}}%
\pgfpathlineto{\pgfqpoint{3.413064in}{2.192485in}}%
\pgfpathlineto{\pgfqpoint{3.425012in}{2.210210in}}%
\pgfpathlineto{\pgfqpoint{3.430146in}{2.221160in}}%
\pgfpathlineto{\pgfqpoint{3.432097in}{2.230810in}}%
\pgfpathlineto{\pgfqpoint{3.430784in}{2.238050in}}%
\pgfpathlineto{\pgfqpoint{3.426592in}{2.243682in}}%
\pgfpathlineto{\pgfqpoint{3.420865in}{2.247210in}}%
\pgfpathlineto{\pgfqpoint{3.412451in}{2.249686in}}%
\pgfpathlineto{\pgfqpoint{3.399448in}{2.250773in}}%
\pgfpathlineto{\pgfqpoint{3.384255in}{2.249743in}}%
\pgfpathlineto{\pgfqpoint{3.364937in}{2.246160in}}%
\pgfpathlineto{\pgfqpoint{3.341757in}{2.239396in}}%
\pgfpathlineto{\pgfqpoint{3.317065in}{2.229729in}}%
\pgfpathlineto{\pgfqpoint{3.291062in}{2.217027in}}%
\pgfpathlineto{\pgfqpoint{3.265888in}{2.202297in}}%
\pgfpathlineto{\pgfqpoint{3.239767in}{2.184392in}}%
\pgfpathlineto{\pgfqpoint{3.214741in}{2.164545in}}%
\pgfpathlineto{\pgfqpoint{3.190868in}{2.142915in}}%
\pgfpathlineto{\pgfqpoint{3.166628in}{2.117930in}}%
\pgfpathlineto{\pgfqpoint{3.143809in}{2.091248in}}%
\pgfpathlineto{\pgfqpoint{3.121056in}{2.061119in}}%
\pgfpathlineto{\pgfqpoint{3.099932in}{2.029473in}}%
\pgfpathlineto{\pgfqpoint{3.079233in}{1.994413in}}%
\pgfpathlineto{\pgfqpoint{3.059202in}{1.955921in}}%
\pgfpathlineto{\pgfqpoint{3.040046in}{1.914019in}}%
\pgfpathlineto{\pgfqpoint{3.022799in}{1.871044in}}%
\pgfpathlineto{\pgfqpoint{3.005782in}{1.822538in}}%
\pgfpathlineto{\pgfqpoint{2.990060in}{1.770820in}}%
\pgfpathlineto{\pgfqpoint{2.975703in}{1.715981in}}%
\pgfpathlineto{\pgfqpoint{2.962280in}{1.655680in}}%
\pgfpathlineto{\pgfqpoint{2.950493in}{1.592386in}}%
\pgfpathlineto{\pgfqpoint{2.940381in}{1.526185in}}%
\pgfpathlineto{\pgfqpoint{2.931745in}{1.454681in}}%
\pgfpathlineto{\pgfqpoint{2.925082in}{1.380399in}}%
\pgfpathlineto{\pgfqpoint{2.920648in}{1.305899in}}%
\pgfpathlineto{\pgfqpoint{2.918445in}{1.231270in}}%
\pgfpathlineto{\pgfqpoint{2.918547in}{1.159087in}}%
\pgfpathlineto{\pgfqpoint{2.920789in}{1.091931in}}%
\pgfpathlineto{\pgfqpoint{2.925179in}{1.027412in}}%
\pgfpathlineto{\pgfqpoint{2.931194in}{0.970580in}}%
\pgfpathlineto{\pgfqpoint{2.938763in}{0.919034in}}%
\pgfpathlineto{\pgfqpoint{2.947653in}{0.872852in}}%
\pgfpathlineto{\pgfqpoint{2.958216in}{0.829715in}}%
\pgfpathlineto{\pgfqpoint{2.969672in}{0.792114in}}%
\pgfpathlineto{\pgfqpoint{2.982465in}{0.757774in}}%
\pgfpathlineto{\pgfqpoint{2.996428in}{0.726812in}}%
\pgfpathlineto{\pgfqpoint{3.011302in}{0.699300in}}%
\pgfpathlineto{\pgfqpoint{3.026742in}{0.675225in}}%
\pgfpathlineto{\pgfqpoint{3.043831in}{0.652656in}}%
\pgfpathlineto{\pgfqpoint{3.062498in}{0.631789in}}%
\pgfpathlineto{\pgfqpoint{3.082606in}{0.612754in}}%
\pgfpathlineto{\pgfqpoint{3.103965in}{0.595593in}}%
\pgfpathlineto{\pgfqpoint{3.128272in}{0.579071in}}%
\pgfpathlineto{\pgfqpoint{3.153541in}{0.564556in}}%
\pgfpathlineto{\pgfqpoint{3.181575in}{0.550955in}}%
\pgfpathlineto{\pgfqpoint{3.214376in}{0.537650in}}%
\pgfpathlineto{\pgfqpoint{3.249851in}{0.525716in}}%
\pgfpathlineto{\pgfqpoint{3.290015in}{0.514575in}}%
\pgfpathlineto{\pgfqpoint{3.334825in}{0.504428in}}%
\pgfpathlineto{\pgfqpoint{3.386377in}{0.495005in}}%
\pgfpathlineto{\pgfqpoint{3.446803in}{0.486264in}}%
\pgfpathlineto{\pgfqpoint{3.518248in}{0.478290in}}%
\pgfpathlineto{\pgfqpoint{3.600690in}{0.471418in}}%
\pgfpathlineto{\pgfqpoint{3.696274in}{0.465723in}}%
\pgfpathlineto{\pgfqpoint{3.807149in}{0.461379in}}%
\pgfpathlineto{\pgfqpoint{3.933296in}{0.458731in}}%
\pgfpathlineto{\pgfqpoint{4.063814in}{0.458224in}}%
\pgfpathlineto{\pgfqpoint{4.187797in}{0.459928in}}%
\pgfpathlineto{\pgfqpoint{4.294340in}{0.463537in}}%
\pgfpathlineto{\pgfqpoint{4.381239in}{0.468593in}}%
\pgfpathlineto{\pgfqpoint{4.450641in}{0.474722in}}%
\pgfpathlineto{\pgfqpoint{4.506855in}{0.481823in}}%
\pgfpathlineto{\pgfqpoint{4.552013in}{0.489684in}}%
\pgfpathlineto{\pgfqpoint{4.588243in}{0.498144in}}%
\pgfpathlineto{\pgfqpoint{4.617659in}{0.507141in}}%
\pgfpathlineto{\pgfqpoint{4.642331in}{0.516876in}}%
\pgfpathlineto{\pgfqpoint{4.664196in}{0.527976in}}%
\pgfpathlineto{\pgfqpoint{4.681239in}{0.538983in}}%
\pgfpathlineto{\pgfqpoint{4.697163in}{0.551994in}}%
\pgfpathlineto{\pgfqpoint{4.710074in}{0.565331in}}%
\pgfpathlineto{\pgfqpoint{4.721576in}{0.580261in}}%
\pgfpathlineto{\pgfqpoint{4.731554in}{0.596565in}}%
\pgfpathlineto{\pgfqpoint{4.740995in}{0.616178in}}%
\pgfpathlineto{\pgfqpoint{4.749517in}{0.639071in}}%
\pgfpathlineto{\pgfqpoint{4.757517in}{0.667494in}}%
\pgfpathlineto{\pgfqpoint{4.764566in}{0.701389in}}%
\pgfpathlineto{\pgfqpoint{4.770835in}{0.743088in}}%
\pgfpathlineto{\pgfqpoint{4.776322in}{0.794978in}}%
\pgfpathlineto{\pgfqpoint{4.781273in}{0.864442in}}%
\pgfpathlineto{\pgfqpoint{4.785465in}{0.956415in}}%
\pgfpathlineto{\pgfqpoint{4.788997in}{1.085789in}}%
\pgfpathlineto{\pgfqpoint{4.791850in}{1.277429in}}%
\pgfpathlineto{\pgfqpoint{4.793957in}{1.581102in}}%
\pgfpathlineto{\pgfqpoint{4.794961in}{2.071473in}}%
\pgfpathlineto{\pgfqpoint{4.793965in}{2.559355in}}%
\pgfpathlineto{\pgfqpoint{4.791729in}{2.746025in}}%
\pgfpathlineto{\pgfqpoint{4.788948in}{2.818135in}}%
\pgfpathlineto{\pgfqpoint{4.785716in}{2.850269in}}%
\pgfpathlineto{\pgfqpoint{4.781855in}{2.867097in}}%
\pgfpathlineto{\pgfqpoint{4.777710in}{2.875813in}}%
\pgfpathlineto{\pgfqpoint{4.773054in}{2.881005in}}%
\pgfpathlineto{\pgfqpoint{4.767315in}{2.884517in}}%
\pgfpathlineto{\pgfqpoint{4.756803in}{2.887627in}}%
\pgfpathlineto{\pgfqpoint{4.739498in}{2.889640in}}%
\pgfpathlineto{\pgfqpoint{4.704712in}{2.890882in}}%
\pgfpathlineto{\pgfqpoint{4.602474in}{2.891538in}}%
\pgfpathlineto{\pgfqpoint{3.952050in}{2.891742in}}%
\pgfpathlineto{\pgfqpoint{0.617271in}{2.890751in}}%
\pgfpathlineto{\pgfqpoint{0.549860in}{2.888853in}}%
\pgfpathlineto{\pgfqpoint{0.521686in}{2.886167in}}%
\pgfpathlineto{\pgfqpoint{0.504618in}{2.882366in}}%
\pgfpathlineto{\pgfqpoint{0.494458in}{2.877975in}}%
\pgfpathlineto{\pgfqpoint{0.487144in}{2.872617in}}%
\pgfpathlineto{\pgfqpoint{0.481126in}{2.865458in}}%
\pgfpathlineto{\pgfqpoint{0.475650in}{2.854735in}}%
\pgfpathlineto{\pgfqpoint{0.471312in}{2.840665in}}%
\pgfpathlineto{\pgfqpoint{0.467300in}{2.818749in}}%
\pgfpathlineto{\pgfqpoint{0.463929in}{2.786627in}}%
\pgfpathlineto{\pgfqpoint{0.460921in}{2.734471in}}%
\pgfpathlineto{\pgfqpoint{0.458366in}{2.647400in}}%
\pgfpathlineto{\pgfqpoint{0.456578in}{2.522957in}}%
\pgfpathlineto{\pgfqpoint{0.456578in}{2.522957in}}%
\pgfusepath{stroke}%
\end{pgfscope}%
\begin{pgfscope}%
\pgfpathrectangle{\pgfqpoint{0.448634in}{0.402556in}}{\pgfqpoint{4.350661in}{2.489204in}} %
\pgfusepath{clip}%
\pgfsetrectcap%
\pgfsetroundjoin%
\pgfsetlinewidth{1.003750pt}%
\definecolor{currentstroke}{rgb}{0.549020,0.337255,0.294118}%
\pgfsetstrokecolor{currentstroke}%
\pgfsetdash{}{0pt}%
\pgfpathmoveto{\pgfqpoint{4.798853in}{2.849880in}}%
\pgfpathlineto{\pgfqpoint{4.797564in}{2.889610in}}%
\pgfpathlineto{\pgfqpoint{4.796215in}{2.891483in}}%
\pgfpathlineto{\pgfqpoint{4.787551in}{2.891760in}}%
\pgfpathlineto{\pgfqpoint{0.452128in}{2.891659in}}%
\pgfpathlineto{\pgfqpoint{0.450530in}{2.890082in}}%
\pgfpathlineto{\pgfqpoint{0.449454in}{2.882763in}}%
\pgfpathlineto{\pgfqpoint{0.448970in}{2.845432in}}%
\pgfpathlineto{\pgfqpoint{0.448743in}{2.494454in}}%
\pgfpathlineto{\pgfqpoint{0.449624in}{0.615107in}}%
\pgfpathlineto{\pgfqpoint{0.451434in}{0.510586in}}%
\pgfpathlineto{\pgfqpoint{0.453995in}{0.473374in}}%
\pgfpathlineto{\pgfqpoint{0.457408in}{0.453868in}}%
\pgfpathlineto{\pgfqpoint{0.461543in}{0.442385in}}%
\pgfpathlineto{\pgfqpoint{0.466743in}{0.434438in}}%
\pgfpathlineto{\pgfqpoint{0.473599in}{0.428352in}}%
\pgfpathlineto{\pgfqpoint{0.483497in}{0.423246in}}%
\pgfpathlineto{\pgfqpoint{0.491858in}{0.420503in}}%
\pgfpathlineto{\pgfqpoint{0.491858in}{0.420503in}}%
\pgfusepath{stroke}%
\end{pgfscope}%
\begin{pgfscope}%
\pgfpathrectangle{\pgfqpoint{0.448634in}{0.402556in}}{\pgfqpoint{4.350661in}{2.489204in}} %
\pgfusepath{clip}%
\pgfsetrectcap%
\pgfsetroundjoin%
\pgfsetlinewidth{1.003750pt}%
\definecolor{currentstroke}{rgb}{0.549020,0.337255,0.294118}%
\pgfsetstrokecolor{currentstroke}%
\pgfsetdash{}{0pt}%
\pgfpathmoveto{\pgfqpoint{0.456439in}{1.367580in}}%
\pgfpathlineto{\pgfqpoint{0.459650in}{1.116199in}}%
\pgfpathlineto{\pgfqpoint{0.463686in}{0.961940in}}%
\pgfpathlineto{\pgfqpoint{0.468508in}{0.857542in}}%
\pgfpathlineto{\pgfqpoint{0.474070in}{0.783142in}}%
\pgfpathlineto{\pgfqpoint{0.480213in}{0.728838in}}%
\pgfpathlineto{\pgfqpoint{0.486957in}{0.687238in}}%
\pgfpathlineto{\pgfqpoint{0.494523in}{0.653490in}}%
\pgfpathlineto{\pgfqpoint{0.503096in}{0.625288in}}%
\pgfpathlineto{\pgfqpoint{0.512183in}{0.602683in}}%
\pgfpathlineto{\pgfqpoint{0.522193in}{0.583443in}}%
\pgfpathlineto{\pgfqpoint{0.534104in}{0.565682in}}%
\pgfpathlineto{\pgfqpoint{0.546263in}{0.551450in}}%
\pgfpathlineto{\pgfqpoint{0.559731in}{0.538854in}}%
\pgfpathlineto{\pgfqpoint{0.576136in}{0.526646in}}%
\pgfpathlineto{\pgfqpoint{0.595492in}{0.515309in}}%
\pgfpathlineto{\pgfqpoint{0.617691in}{0.505111in}}%
\pgfpathlineto{\pgfqpoint{0.642580in}{0.496123in}}%
\pgfpathlineto{\pgfqpoint{0.672139in}{0.487752in}}%
\pgfpathlineto{\pgfqpoint{0.708457in}{0.479803in}}%
\pgfpathlineto{\pgfqpoint{0.753664in}{0.472309in}}%
\pgfpathlineto{\pgfqpoint{0.807733in}{0.465648in}}%
\pgfpathlineto{\pgfqpoint{0.877131in}{0.459467in}}%
\pgfpathlineto{\pgfqpoint{0.961844in}{0.454225in}}%
\pgfpathlineto{\pgfqpoint{1.068367in}{0.449914in}}%
\pgfpathlineto{\pgfqpoint{1.201034in}{0.446840in}}%
\pgfpathlineto{\pgfqpoint{1.357652in}{0.445483in}}%
\pgfpathlineto{\pgfqpoint{1.525151in}{0.446235in}}%
\pgfpathlineto{\pgfqpoint{1.686104in}{0.449145in}}%
\pgfpathlineto{\pgfqpoint{1.823089in}{0.453751in}}%
\pgfpathlineto{\pgfqpoint{1.938260in}{0.459768in}}%
\pgfpathlineto{\pgfqpoint{2.031598in}{0.466764in}}%
\pgfpathlineto{\pgfqpoint{2.109596in}{0.474750in}}%
\pgfpathlineto{\pgfqpoint{2.174399in}{0.483541in}}%
\pgfpathlineto{\pgfqpoint{2.228155in}{0.492948in}}%
\pgfpathlineto{\pgfqpoint{2.275134in}{0.503365in}}%
\pgfpathlineto{\pgfqpoint{2.315297in}{0.514512in}}%
\pgfpathlineto{\pgfqpoint{2.350712in}{0.526672in}}%
\pgfpathlineto{\pgfqpoint{2.381334in}{0.539550in}}%
\pgfpathlineto{\pgfqpoint{2.407177in}{0.552675in}}%
\pgfpathlineto{\pgfqpoint{2.430238in}{0.566658in}}%
\pgfpathlineto{\pgfqpoint{2.452292in}{0.582623in}}%
\pgfpathlineto{\pgfqpoint{2.471400in}{0.599092in}}%
\pgfpathlineto{\pgfqpoint{2.489248in}{0.617318in}}%
\pgfpathlineto{\pgfqpoint{2.505684in}{0.637207in}}%
\pgfpathlineto{\pgfqpoint{2.520624in}{0.658586in}}%
\pgfpathlineto{\pgfqpoint{2.535216in}{0.683344in}}%
\pgfpathlineto{\pgfqpoint{2.549116in}{0.711515in}}%
\pgfpathlineto{\pgfqpoint{2.562090in}{0.743036in}}%
\pgfpathlineto{\pgfqpoint{2.574018in}{0.777783in}}%
\pgfpathlineto{\pgfqpoint{2.585499in}{0.818003in}}%
\pgfpathlineto{\pgfqpoint{2.596805in}{0.866071in}}%
\pgfpathlineto{\pgfqpoint{2.607558in}{0.921981in}}%
\pgfpathlineto{\pgfqpoint{2.617920in}{0.988131in}}%
\pgfpathlineto{\pgfqpoint{2.627953in}{1.066951in}}%
\pgfpathlineto{\pgfqpoint{2.637937in}{1.163353in}}%
\pgfpathlineto{\pgfqpoint{2.648420in}{1.287232in}}%
\pgfpathlineto{\pgfqpoint{2.660099in}{1.453471in}}%
\pgfpathlineto{\pgfqpoint{2.674628in}{1.694350in}}%
\pgfpathlineto{\pgfqpoint{2.687598in}{1.942826in}}%
\pgfpathlineto{\pgfqpoint{2.692598in}{2.077119in}}%
\pgfpathlineto{\pgfqpoint{2.693819in}{2.166715in}}%
\pgfpathlineto{\pgfqpoint{2.692553in}{2.233903in}}%
\pgfpathlineto{\pgfqpoint{2.689422in}{2.286048in}}%
\pgfpathlineto{\pgfqpoint{2.684842in}{2.328032in}}%
\pgfpathlineto{\pgfqpoint{2.678705in}{2.364696in}}%
\pgfpathlineto{\pgfqpoint{2.671334in}{2.395928in}}%
\pgfpathlineto{\pgfqpoint{2.662464in}{2.424011in}}%
\pgfpathlineto{\pgfqpoint{2.652333in}{2.448807in}}%
\pgfpathlineto{\pgfqpoint{2.641333in}{2.470272in}}%
\pgfpathlineto{\pgfqpoint{2.628608in}{2.490449in}}%
\pgfpathlineto{\pgfqpoint{2.614241in}{2.509127in}}%
\pgfpathlineto{\pgfqpoint{2.598404in}{2.526178in}}%
\pgfpathlineto{\pgfqpoint{2.579548in}{2.543021in}}%
\pgfpathlineto{\pgfqpoint{2.559489in}{2.557936in}}%
\pgfpathlineto{\pgfqpoint{2.536558in}{2.572193in}}%
\pgfpathlineto{\pgfqpoint{2.510804in}{2.585546in}}%
\pgfpathlineto{\pgfqpoint{2.482314in}{2.597843in}}%
\pgfpathlineto{\pgfqpoint{2.449088in}{2.609686in}}%
\pgfpathlineto{\pgfqpoint{2.411137in}{2.620698in}}%
\pgfpathlineto{\pgfqpoint{2.368505in}{2.630606in}}%
\pgfpathlineto{\pgfqpoint{2.321246in}{2.639219in}}%
\pgfpathlineto{\pgfqpoint{2.269419in}{2.646396in}}%
\pgfpathlineto{\pgfqpoint{2.210906in}{2.652189in}}%
\pgfpathlineto{\pgfqpoint{2.147919in}{2.656148in}}%
\pgfpathlineto{\pgfqpoint{2.080508in}{2.658129in}}%
\pgfpathlineto{\pgfqpoint{2.010900in}{2.657965in}}%
\pgfpathlineto{\pgfqpoint{1.939147in}{2.655566in}}%
\pgfpathlineto{\pgfqpoint{1.867479in}{2.650906in}}%
\pgfpathlineto{\pgfqpoint{1.798123in}{2.644132in}}%
\pgfpathlineto{\pgfqpoint{1.733293in}{2.635597in}}%
\pgfpathlineto{\pgfqpoint{1.673028in}{2.625511in}}%
\pgfpathlineto{\pgfqpoint{1.615227in}{2.613599in}}%
\pgfpathlineto{\pgfqpoint{1.562086in}{2.600390in}}%
\pgfpathlineto{\pgfqpoint{1.513634in}{2.586126in}}%
\pgfpathlineto{\pgfqpoint{1.467815in}{2.570329in}}%
\pgfpathlineto{\pgfqpoint{1.426748in}{2.553906in}}%
\pgfpathlineto{\pgfqpoint{1.388401in}{2.536271in}}%
\pgfpathlineto{\pgfqpoint{1.352833in}{2.517546in}}%
\pgfpathlineto{\pgfqpoint{1.320084in}{2.497900in}}%
\pgfpathlineto{\pgfqpoint{1.288336in}{2.476212in}}%
\pgfpathlineto{\pgfqpoint{1.259550in}{2.453835in}}%
\pgfpathlineto{\pgfqpoint{1.232010in}{2.429493in}}%
\pgfpathlineto{\pgfqpoint{1.207488in}{2.404870in}}%
\pgfpathlineto{\pgfqpoint{1.184371in}{2.378527in}}%
\pgfpathlineto{\pgfqpoint{1.162791in}{2.350529in}}%
\pgfpathlineto{\pgfqpoint{1.142856in}{2.320979in}}%
\pgfpathlineto{\pgfqpoint{1.124641in}{2.290007in}}%
\pgfpathlineto{\pgfqpoint{1.108193in}{2.257767in}}%
\pgfpathlineto{\pgfqpoint{1.092608in}{2.222163in}}%
\pgfpathlineto{\pgfqpoint{1.079030in}{2.185499in}}%
\pgfpathlineto{\pgfqpoint{1.067415in}{2.147961in}}%
\pgfpathlineto{\pgfqpoint{1.057161in}{2.107310in}}%
\pgfpathlineto{\pgfqpoint{1.048978in}{2.066048in}}%
\pgfpathlineto{\pgfqpoint{1.042488in}{2.021868in}}%
\pgfpathlineto{\pgfqpoint{1.038153in}{1.977344in}}%
\pgfpathlineto{\pgfqpoint{1.035842in}{1.930129in}}%
\pgfpathlineto{\pgfqpoint{1.035803in}{1.882840in}}%
\pgfpathlineto{\pgfqpoint{1.038008in}{1.835618in}}%
\pgfpathlineto{\pgfqpoint{1.042452in}{1.788603in}}%
\pgfpathlineto{\pgfqpoint{1.049154in}{1.741941in}}%
\pgfpathlineto{\pgfqpoint{1.057622in}{1.698201in}}%
\pgfpathlineto{\pgfqpoint{1.068200in}{1.655067in}}%
\pgfpathlineto{\pgfqpoint{1.080941in}{1.612707in}}%
\pgfpathlineto{\pgfqpoint{1.095009in}{1.573579in}}%
\pgfpathlineto{\pgfqpoint{1.111093in}{1.535482in}}%
\pgfpathlineto{\pgfqpoint{1.128096in}{1.500737in}}%
\pgfpathlineto{\pgfqpoint{1.146908in}{1.467235in}}%
\pgfpathlineto{\pgfqpoint{1.167508in}{1.435142in}}%
\pgfpathlineto{\pgfqpoint{1.189851in}{1.404613in}}%
\pgfpathlineto{\pgfqpoint{1.213860in}{1.375788in}}%
\pgfpathlineto{\pgfqpoint{1.237793in}{1.350417in}}%
\pgfpathlineto{\pgfqpoint{1.264723in}{1.325196in}}%
\pgfpathlineto{\pgfqpoint{1.292965in}{1.301929in}}%
\pgfpathlineto{\pgfqpoint{1.322371in}{1.280634in}}%
\pgfpathlineto{\pgfqpoint{1.352793in}{1.261295in}}%
\pgfpathlineto{\pgfqpoint{1.386067in}{1.242841in}}%
\pgfpathlineto{\pgfqpoint{1.420162in}{1.226466in}}%
\pgfpathlineto{\pgfqpoint{1.456994in}{1.211275in}}%
\pgfpathlineto{\pgfqpoint{1.496523in}{1.197478in}}%
\pgfpathlineto{\pgfqpoint{1.538687in}{1.185224in}}%
\pgfpathlineto{\pgfqpoint{1.583408in}{1.174575in}}%
\pgfpathlineto{\pgfqpoint{1.634895in}{1.164704in}}%
\pgfpathlineto{\pgfqpoint{1.706029in}{1.153668in}}%
\pgfpathlineto{\pgfqpoint{1.768458in}{1.143337in}}%
\pgfpathlineto{\pgfqpoint{1.796088in}{1.136487in}}%
\pgfpathlineto{\pgfqpoint{1.812649in}{1.130402in}}%
\pgfpathlineto{\pgfqpoint{1.824437in}{1.124025in}}%
\pgfpathlineto{\pgfqpoint{1.833178in}{1.116667in}}%
\pgfpathlineto{\pgfqpoint{1.838471in}{1.108819in}}%
\pgfpathlineto{\pgfqpoint{1.840566in}{1.101780in}}%
\pgfpathlineto{\pgfqpoint{1.840601in}{1.094343in}}%
\pgfpathlineto{\pgfqpoint{1.837916in}{1.084916in}}%
\pgfpathlineto{\pgfqpoint{1.833233in}{1.076544in}}%
\pgfpathlineto{\pgfqpoint{1.825806in}{1.067470in}}%
\pgfpathlineto{\pgfqpoint{1.813801in}{1.056778in}}%
\pgfpathlineto{\pgfqpoint{1.798807in}{1.046691in}}%
\pgfpathlineto{\pgfqpoint{1.781003in}{1.037392in}}%
\pgfpathlineto{\pgfqpoint{1.758434in}{1.028321in}}%
\pgfpathlineto{\pgfqpoint{1.733189in}{1.020747in}}%
\pgfpathlineto{\pgfqpoint{1.705396in}{1.014806in}}%
\pgfpathlineto{\pgfqpoint{1.675164in}{1.010649in}}%
\pgfpathlineto{\pgfqpoint{1.642596in}{1.008444in}}%
\pgfpathlineto{\pgfqpoint{1.607795in}{1.008372in}}%
\pgfpathlineto{\pgfqpoint{1.570872in}{1.010632in}}%
\pgfpathlineto{\pgfqpoint{1.534104in}{1.015124in}}%
\pgfpathlineto{\pgfqpoint{1.497582in}{1.021735in}}%
\pgfpathlineto{\pgfqpoint{1.459268in}{1.030953in}}%
\pgfpathlineto{\pgfqpoint{1.421417in}{1.042408in}}%
\pgfpathlineto{\pgfqpoint{1.384136in}{1.056090in}}%
\pgfpathlineto{\pgfqpoint{1.349555in}{1.071069in}}%
\pgfpathlineto{\pgfqpoint{1.315695in}{1.088071in}}%
\pgfpathlineto{\pgfqpoint{1.282678in}{1.107120in}}%
\pgfpathlineto{\pgfqpoint{1.250638in}{1.128240in}}%
\pgfpathlineto{\pgfqpoint{1.221500in}{1.150014in}}%
\pgfpathlineto{\pgfqpoint{1.193473in}{1.173619in}}%
\pgfpathlineto{\pgfqpoint{1.166672in}{1.199021in}}%
\pgfpathlineto{\pgfqpoint{1.141209in}{1.226165in}}%
\pgfpathlineto{\pgfqpoint{1.117178in}{1.254968in}}%
\pgfpathlineto{\pgfqpoint{1.094657in}{1.285326in}}%
\pgfpathlineto{\pgfqpoint{1.072442in}{1.319148in}}%
\pgfpathlineto{\pgfqpoint{1.052027in}{1.354425in}}%
\pgfpathlineto{\pgfqpoint{1.033418in}{1.390987in}}%
\pgfpathlineto{\pgfqpoint{1.015661in}{1.430914in}}%
\pgfpathlineto{\pgfqpoint{0.999860in}{1.471904in}}%
\pgfpathlineto{\pgfqpoint{0.985234in}{1.516132in}}%
\pgfpathlineto{\pgfqpoint{0.972633in}{1.561170in}}%
\pgfpathlineto{\pgfqpoint{0.961463in}{1.609280in}}%
\pgfpathlineto{\pgfqpoint{0.951944in}{1.660401in}}%
\pgfpathlineto{\pgfqpoint{0.944600in}{1.711989in}}%
\pgfpathlineto{\pgfqpoint{0.939161in}{1.766391in}}%
\pgfpathlineto{\pgfqpoint{0.935980in}{1.821028in}}%
\pgfpathlineto{\pgfqpoint{0.935043in}{1.875774in}}%
\pgfpathlineto{\pgfqpoint{0.936372in}{1.930511in}}%
\pgfpathlineto{\pgfqpoint{0.939810in}{1.982631in}}%
\pgfpathlineto{\pgfqpoint{0.945457in}{2.034498in}}%
\pgfpathlineto{\pgfqpoint{0.953002in}{2.083522in}}%
\pgfpathlineto{\pgfqpoint{0.962248in}{2.129613in}}%
\pgfpathlineto{\pgfqpoint{0.973655in}{2.175064in}}%
\pgfpathlineto{\pgfqpoint{0.986585in}{2.217350in}}%
\pgfpathlineto{\pgfqpoint{1.000805in}{2.256406in}}%
\pgfpathlineto{\pgfqpoint{1.017069in}{2.294403in}}%
\pgfpathlineto{\pgfqpoint{1.034304in}{2.328997in}}%
\pgfpathlineto{\pgfqpoint{1.053437in}{2.362259in}}%
\pgfpathlineto{\pgfqpoint{1.073087in}{2.392059in}}%
\pgfpathlineto{\pgfqpoint{1.094349in}{2.420374in}}%
\pgfpathlineto{\pgfqpoint{1.117143in}{2.447082in}}%
\pgfpathlineto{\pgfqpoint{1.141360in}{2.472098in}}%
\pgfpathlineto{\pgfqpoint{1.166865in}{2.495377in}}%
\pgfpathlineto{\pgfqpoint{1.195326in}{2.518292in}}%
\pgfpathlineto{\pgfqpoint{1.224905in}{2.539272in}}%
\pgfpathlineto{\pgfqpoint{1.257365in}{2.559535in}}%
\pgfpathlineto{\pgfqpoint{1.292688in}{2.578857in}}%
\pgfpathlineto{\pgfqpoint{1.330829in}{2.597067in}}%
\pgfpathlineto{\pgfqpoint{1.371722in}{2.614047in}}%
\pgfpathlineto{\pgfqpoint{1.415297in}{2.629729in}}%
\pgfpathlineto{\pgfqpoint{1.463588in}{2.644687in}}%
\pgfpathlineto{\pgfqpoint{1.516579in}{2.658665in}}%
\pgfpathlineto{\pgfqpoint{1.574234in}{2.671470in}}%
\pgfpathlineto{\pgfqpoint{1.636512in}{2.682959in}}%
\pgfpathlineto{\pgfqpoint{1.703369in}{2.693024in}}%
\pgfpathlineto{\pgfqpoint{1.776929in}{2.701808in}}%
\pgfpathlineto{\pgfqpoint{1.854994in}{2.708879in}}%
\pgfpathlineto{\pgfqpoint{1.937527in}{2.714137in}}%
\pgfpathlineto{\pgfqpoint{2.024491in}{2.717425in}}%
\pgfpathlineto{\pgfqpoint{2.111498in}{2.718487in}}%
\pgfpathlineto{\pgfqpoint{2.196328in}{2.717321in}}%
\pgfpathlineto{\pgfqpoint{2.276761in}{2.714005in}}%
\pgfpathlineto{\pgfqpoint{2.350576in}{2.708739in}}%
\pgfpathlineto{\pgfqpoint{2.415561in}{2.701921in}}%
\pgfpathlineto{\pgfqpoint{2.471683in}{2.693921in}}%
\pgfpathlineto{\pgfqpoint{2.521066in}{2.684750in}}%
\pgfpathlineto{\pgfqpoint{2.563671in}{2.674688in}}%
\pgfpathlineto{\pgfqpoint{2.601572in}{2.663456in}}%
\pgfpathlineto{\pgfqpoint{2.632635in}{2.652044in}}%
\pgfpathlineto{\pgfqpoint{2.658953in}{2.640220in}}%
\pgfpathlineto{\pgfqpoint{2.682487in}{2.627312in}}%
\pgfpathlineto{\pgfqpoint{2.703104in}{2.613433in}}%
\pgfpathlineto{\pgfqpoint{2.720707in}{2.598826in}}%
\pgfpathlineto{\pgfqpoint{2.735284in}{2.583887in}}%
\pgfpathlineto{\pgfqpoint{2.748328in}{2.567198in}}%
\pgfpathlineto{\pgfqpoint{2.759548in}{2.548855in}}%
\pgfpathlineto{\pgfqpoint{2.768768in}{2.529107in}}%
\pgfpathlineto{\pgfqpoint{2.775983in}{2.508292in}}%
\pgfpathlineto{\pgfqpoint{2.781838in}{2.484330in}}%
\pgfpathlineto{\pgfqpoint{2.786045in}{2.457385in}}%
\pgfpathlineto{\pgfqpoint{2.788655in}{2.425171in}}%
\pgfpathlineto{\pgfqpoint{2.789356in}{2.387848in}}%
\pgfpathlineto{\pgfqpoint{2.787888in}{2.340588in}}%
\pgfpathlineto{\pgfqpoint{2.783597in}{2.278555in}}%
\pgfpathlineto{\pgfqpoint{2.774217in}{2.179569in}}%
\pgfpathlineto{\pgfqpoint{2.743556in}{1.867903in}}%
\pgfpathlineto{\pgfqpoint{2.730063in}{1.701844in}}%
\pgfpathlineto{\pgfqpoint{2.717242in}{1.515733in}}%
\pgfpathlineto{\pgfqpoint{2.702559in}{1.267381in}}%
\pgfpathlineto{\pgfqpoint{2.684381in}{0.964414in}}%
\pgfpathlineto{\pgfqpoint{2.675312in}{0.850385in}}%
\pgfpathlineto{\pgfqpoint{2.666956in}{0.771310in}}%
\pgfpathlineto{\pgfqpoint{2.658665in}{0.712332in}}%
\pgfpathlineto{\pgfqpoint{2.650074in}{0.666077in}}%
\pgfpathlineto{\pgfqpoint{2.640699in}{0.627730in}}%
\pgfpathlineto{\pgfqpoint{2.631004in}{0.597341in}}%
\pgfpathlineto{\pgfqpoint{2.620842in}{0.572563in}}%
\pgfpathlineto{\pgfqpoint{2.609672in}{0.551217in}}%
\pgfpathlineto{\pgfqpoint{2.597837in}{0.533386in}}%
\pgfpathlineto{\pgfqpoint{2.584271in}{0.517252in}}%
\pgfpathlineto{\pgfqpoint{2.569115in}{0.503091in}}%
\pgfpathlineto{\pgfqpoint{2.552654in}{0.490983in}}%
\pgfpathlineto{\pgfqpoint{2.535220in}{0.480807in}}%
\pgfpathlineto{\pgfqpoint{2.515062in}{0.471468in}}%
\pgfpathlineto{\pgfqpoint{2.490175in}{0.462475in}}%
\pgfpathlineto{\pgfqpoint{2.460577in}{0.454297in}}%
\pgfpathlineto{\pgfqpoint{2.424193in}{0.446753in}}%
\pgfpathlineto{\pgfqpoint{2.381072in}{0.440158in}}%
\pgfpathlineto{\pgfqpoint{2.326944in}{0.434174in}}%
\pgfpathlineto{\pgfqpoint{2.257493in}{0.428801in}}%
\pgfpathlineto{\pgfqpoint{2.166225in}{0.424059in}}%
\pgfpathlineto{\pgfqpoint{2.042284in}{0.419973in}}%
\pgfpathlineto{\pgfqpoint{1.870458in}{0.416668in}}%
\pgfpathlineto{\pgfqpoint{1.626830in}{0.414349in}}%
\pgfpathlineto{\pgfqpoint{1.307057in}{0.413590in}}%
\pgfpathlineto{\pgfqpoint{1.009040in}{0.414960in}}%
\pgfpathlineto{\pgfqpoint{0.815453in}{0.417880in}}%
\pgfpathlineto{\pgfqpoint{0.700210in}{0.421651in}}%
\pgfpathlineto{\pgfqpoint{0.628529in}{0.426040in}}%
\pgfpathlineto{\pgfqpoint{0.580880in}{0.431080in}}%
\pgfpathlineto{\pgfqpoint{0.548623in}{0.436659in}}%
\pgfpathlineto{\pgfqpoint{0.527442in}{0.442296in}}%
\pgfpathlineto{\pgfqpoint{0.510995in}{0.448771in}}%
\pgfpathlineto{\pgfqpoint{0.499312in}{0.455402in}}%
\pgfpathlineto{\pgfqpoint{0.488709in}{0.464076in}}%
\pgfpathlineto{\pgfqpoint{0.481151in}{0.473004in}}%
\pgfpathlineto{\pgfqpoint{0.473951in}{0.485434in}}%
\pgfpathlineto{\pgfqpoint{0.468663in}{0.499075in}}%
\pgfpathlineto{\pgfqpoint{0.463812in}{0.518186in}}%
\pgfpathlineto{\pgfqpoint{0.459645in}{0.545139in}}%
\pgfpathlineto{\pgfqpoint{0.456368in}{0.582283in}}%
\pgfpathlineto{\pgfqpoint{0.453723in}{0.639451in}}%
\pgfpathlineto{\pgfqpoint{0.451645in}{0.738989in}}%
\pgfpathlineto{\pgfqpoint{0.450199in}{0.933139in}}%
\pgfpathlineto{\pgfqpoint{0.449337in}{1.416044in}}%
\pgfpathlineto{\pgfqpoint{0.449567in}{2.693005in}}%
\pgfpathlineto{\pgfqpoint{0.451025in}{2.857277in}}%
\pgfpathlineto{\pgfqpoint{0.452870in}{2.879557in}}%
\pgfpathlineto{\pgfqpoint{0.455373in}{2.886383in}}%
\pgfpathlineto{\pgfqpoint{0.458915in}{2.889148in}}%
\pgfpathlineto{\pgfqpoint{0.467469in}{2.890809in}}%
\pgfpathlineto{\pgfqpoint{0.491386in}{2.891542in}}%
\pgfpathlineto{\pgfqpoint{0.639308in}{2.891749in}}%
\pgfpathlineto{\pgfqpoint{4.785483in}{2.891274in}}%
\pgfpathlineto{\pgfqpoint{4.794005in}{2.889575in}}%
\pgfpathlineto{\pgfqpoint{4.795666in}{2.888019in}}%
\pgfpathlineto{\pgfqpoint{4.797132in}{2.880791in}}%
\pgfpathlineto{\pgfqpoint{4.798041in}{2.855926in}}%
\pgfpathlineto{\pgfqpoint{4.798041in}{2.855926in}}%
\pgfusepath{stroke}%
\end{pgfscope}%
\begin{pgfscope}%
\pgfpathrectangle{\pgfqpoint{0.448634in}{0.402556in}}{\pgfqpoint{4.350661in}{2.489204in}} %
\pgfusepath{clip}%
\pgfsetrectcap%
\pgfsetroundjoin%
\pgfsetlinewidth{1.003750pt}%
\definecolor{currentstroke}{rgb}{0.549020,0.337255,0.294118}%
\pgfsetstrokecolor{currentstroke}%
\pgfsetdash{}{0pt}%
\pgfpathmoveto{\pgfqpoint{3.428831in}{0.402614in}}%
\pgfpathlineto{\pgfqpoint{2.811041in}{0.403739in}}%
\pgfpathlineto{\pgfqpoint{2.771923in}{0.405553in}}%
\pgfpathlineto{\pgfqpoint{2.754717in}{0.408406in}}%
\pgfpathlineto{\pgfqpoint{2.746523in}{0.411693in}}%
\pgfpathlineto{\pgfqpoint{2.741144in}{0.415879in}}%
\pgfpathlineto{\pgfqpoint{2.737052in}{0.421664in}}%
\pgfpathlineto{\pgfqpoint{2.733572in}{0.430765in}}%
\pgfpathlineto{\pgfqpoint{2.730719in}{0.445325in}}%
\pgfpathlineto{\pgfqpoint{2.728449in}{0.470074in}}%
\pgfpathlineto{\pgfqpoint{2.726680in}{0.517321in}}%
\pgfpathlineto{\pgfqpoint{2.725830in}{0.606926in}}%
\pgfpathlineto{\pgfqpoint{2.726770in}{0.756272in}}%
\pgfpathlineto{\pgfqpoint{2.730269in}{0.947898in}}%
\pgfpathlineto{\pgfqpoint{2.736046in}{1.141941in}}%
\pgfpathlineto{\pgfqpoint{2.743428in}{1.313486in}}%
\pgfpathlineto{\pgfqpoint{2.752329in}{1.469973in}}%
\pgfpathlineto{\pgfqpoint{2.762563in}{1.611372in}}%
\pgfpathlineto{\pgfqpoint{2.775246in}{1.754999in}}%
\pgfpathlineto{\pgfqpoint{2.788167in}{1.871051in}}%
\pgfpathlineto{\pgfqpoint{2.804847in}{1.999042in}}%
\pgfpathlineto{\pgfqpoint{2.820555in}{2.096971in}}%
\pgfpathlineto{\pgfqpoint{2.837658in}{2.189512in}}%
\pgfpathlineto{\pgfqpoint{2.857846in}{2.286360in}}%
\pgfpathlineto{\pgfqpoint{2.889487in}{2.436291in}}%
\pgfpathlineto{\pgfqpoint{2.898066in}{2.485091in}}%
\pgfpathlineto{\pgfqpoint{2.902072in}{2.519629in}}%
\pgfpathlineto{\pgfqpoint{2.903152in}{2.546972in}}%
\pgfpathlineto{\pgfqpoint{2.901991in}{2.569323in}}%
\pgfpathlineto{\pgfqpoint{2.898873in}{2.588901in}}%
\pgfpathlineto{\pgfqpoint{2.894198in}{2.605470in}}%
\pgfpathlineto{\pgfqpoint{2.887558in}{2.621134in}}%
\pgfpathlineto{\pgfqpoint{2.879026in}{2.635549in}}%
\pgfpathlineto{\pgfqpoint{2.868864in}{2.648509in}}%
\pgfpathlineto{\pgfqpoint{2.855698in}{2.661511in}}%
\pgfpathlineto{\pgfqpoint{2.839470in}{2.674022in}}%
\pgfpathlineto{\pgfqpoint{2.820295in}{2.685759in}}%
\pgfpathlineto{\pgfqpoint{2.798316in}{2.696567in}}%
\pgfpathlineto{\pgfqpoint{2.771599in}{2.707155in}}%
\pgfpathlineto{\pgfqpoint{2.740175in}{2.717194in}}%
\pgfpathlineto{\pgfqpoint{2.701965in}{2.726960in}}%
\pgfpathlineto{\pgfqpoint{2.656985in}{2.736068in}}%
\pgfpathlineto{\pgfqpoint{2.605275in}{2.744279in}}%
\pgfpathlineto{\pgfqpoint{2.542547in}{2.751917in}}%
\pgfpathlineto{\pgfqpoint{2.470983in}{2.758353in}}%
\pgfpathlineto{\pgfqpoint{2.386273in}{2.763643in}}%
\pgfpathlineto{\pgfqpoint{2.297135in}{2.766973in}}%
\pgfpathlineto{\pgfqpoint{2.194908in}{2.768736in}}%
\pgfpathlineto{\pgfqpoint{2.088318in}{2.768384in}}%
\pgfpathlineto{\pgfqpoint{1.975226in}{2.765772in}}%
\pgfpathlineto{\pgfqpoint{1.870895in}{2.761017in}}%
\pgfpathlineto{\pgfqpoint{1.766662in}{2.753998in}}%
\pgfpathlineto{\pgfqpoint{1.677777in}{2.745605in}}%
\pgfpathlineto{\pgfqpoint{1.597744in}{2.735865in}}%
\pgfpathlineto{\pgfqpoint{1.507164in}{2.722438in}}%
\pgfpathlineto{\pgfqpoint{1.425662in}{2.706874in}}%
\pgfpathlineto{\pgfqpoint{1.370355in}{2.693352in}}%
\pgfpathlineto{\pgfqpoint{1.319734in}{2.678750in}}%
\pgfpathlineto{\pgfqpoint{1.271760in}{2.662511in}}%
\pgfpathlineto{\pgfqpoint{1.228572in}{2.645490in}}%
\pgfpathlineto{\pgfqpoint{1.190195in}{2.627941in}}%
\pgfpathlineto{\pgfqpoint{1.154611in}{2.609256in}}%
\pgfpathlineto{\pgfqpoint{1.121873in}{2.589588in}}%
\pgfpathlineto{\pgfqpoint{1.093888in}{2.570404in}}%
\pgfpathlineto{\pgfqpoint{1.081433in}{2.560405in}}%
\pgfpathlineto{\pgfqpoint{1.053127in}{2.537241in}}%
\pgfpathlineto{\pgfqpoint{1.027799in}{2.513710in}}%
\pgfpathlineto{\pgfqpoint{1.003780in}{2.488446in}}%
\pgfpathlineto{\pgfqpoint{0.981181in}{2.461521in}}%
\pgfpathlineto{\pgfqpoint{0.960084in}{2.433045in}}%
\pgfpathlineto{\pgfqpoint{0.939284in}{2.401120in}}%
\pgfpathlineto{\pgfqpoint{0.922758in}{2.371864in}}%
\pgfpathlineto{\pgfqpoint{0.904202in}{2.335267in}}%
\pgfpathlineto{\pgfqpoint{0.887519in}{2.297507in}}%
\pgfpathlineto{\pgfqpoint{0.871772in}{2.256490in}}%
\pgfpathlineto{\pgfqpoint{0.857122in}{2.212272in}}%
\pgfpathlineto{\pgfqpoint{0.844322in}{2.167307in}}%
\pgfpathlineto{\pgfqpoint{0.833075in}{2.119225in}}%
\pgfpathlineto{\pgfqpoint{0.821715in}{2.063473in}}%
\pgfpathlineto{\pgfqpoint{0.812858in}{2.009669in}}%
\pgfpathlineto{\pgfqpoint{0.805321in}{1.953082in}}%
\pgfpathlineto{\pgfqpoint{0.797489in}{1.881458in}}%
\pgfpathlineto{\pgfqpoint{0.791334in}{1.804617in}}%
\pgfpathlineto{\pgfqpoint{0.786665in}{1.720158in}}%
\pgfpathlineto{\pgfqpoint{0.782197in}{1.600789in}}%
\pgfpathlineto{\pgfqpoint{0.777303in}{1.491413in}}%
\pgfpathlineto{\pgfqpoint{0.772758in}{1.439405in}}%
\pgfpathlineto{\pgfqpoint{0.767529in}{1.405084in}}%
\pgfpathlineto{\pgfqpoint{0.761798in}{1.381083in}}%
\pgfpathlineto{\pgfqpoint{0.755193in}{1.362678in}}%
\pgfpathlineto{\pgfqpoint{0.748391in}{1.349954in}}%
\pgfpathlineto{\pgfqpoint{0.741006in}{1.340852in}}%
\pgfpathlineto{\pgfqpoint{0.733776in}{1.335356in}}%
\pgfpathlineto{\pgfqpoint{0.725541in}{1.332248in}}%
\pgfpathlineto{\pgfqpoint{0.716872in}{1.331900in}}%
\pgfpathlineto{\pgfqpoint{0.708392in}{1.334035in}}%
\pgfpathlineto{\pgfqpoint{0.698536in}{1.339246in}}%
\pgfpathlineto{\pgfqpoint{0.687929in}{1.347918in}}%
\pgfpathlineto{\pgfqpoint{0.677024in}{1.360061in}}%
\pgfpathlineto{\pgfqpoint{0.666115in}{1.375563in}}%
\pgfpathlineto{\pgfqpoint{0.654281in}{1.396438in}}%
\pgfpathlineto{\pgfqpoint{0.642101in}{1.422848in}}%
\pgfpathlineto{\pgfqpoint{0.630023in}{1.454831in}}%
\pgfpathlineto{\pgfqpoint{0.618362in}{1.492350in}}%
\pgfpathlineto{\pgfqpoint{0.608102in}{1.532995in}}%
\pgfpathlineto{\pgfqpoint{0.597336in}{1.583791in}}%
\pgfpathlineto{\pgfqpoint{0.587699in}{1.639966in}}%
\pgfpathlineto{\pgfqpoint{0.579616in}{1.698982in}}%
\pgfpathlineto{\pgfqpoint{0.571900in}{1.768114in}}%
\pgfpathlineto{\pgfqpoint{0.569036in}{1.800304in}}%
\pgfpathlineto{\pgfqpoint{0.562380in}{1.887090in}}%
\pgfpathlineto{\pgfqpoint{0.557178in}{1.983984in}}%
\pgfpathlineto{\pgfqpoint{0.553501in}{2.095915in}}%
\pgfpathlineto{\pgfqpoint{0.552396in}{2.202941in}}%
\pgfpathlineto{\pgfqpoint{0.553493in}{2.304988in}}%
\pgfpathlineto{\pgfqpoint{0.556669in}{2.399504in}}%
\pgfpathlineto{\pgfqpoint{0.560933in}{2.466526in}}%
\pgfpathlineto{\pgfqpoint{0.567022in}{2.535870in}}%
\pgfpathlineto{\pgfqpoint{0.574130in}{2.592536in}}%
\pgfpathlineto{\pgfqpoint{0.582078in}{2.638942in}}%
\pgfpathlineto{\pgfqpoint{0.591354in}{2.679898in}}%
\pgfpathlineto{\pgfqpoint{0.600544in}{2.710493in}}%
\pgfpathlineto{\pgfqpoint{0.611076in}{2.737813in}}%
\pgfpathlineto{\pgfqpoint{0.621726in}{2.759506in}}%
\pgfpathlineto{\pgfqpoint{0.633003in}{2.777806in}}%
\pgfpathlineto{\pgfqpoint{0.645990in}{2.794554in}}%
\pgfpathlineto{\pgfqpoint{0.658930in}{2.807854in}}%
\pgfpathlineto{\pgfqpoint{0.674870in}{2.820840in}}%
\pgfpathlineto{\pgfqpoint{0.691967in}{2.831734in}}%
\pgfpathlineto{\pgfqpoint{0.709912in}{2.840669in}}%
\pgfpathlineto{\pgfqpoint{0.736770in}{2.850767in}}%
\pgfpathlineto{\pgfqpoint{0.764248in}{2.858390in}}%
\pgfpathlineto{\pgfqpoint{0.796370in}{2.864925in}}%
\pgfpathlineto{\pgfqpoint{0.837372in}{2.870848in}}%
\pgfpathlineto{\pgfqpoint{0.889387in}{2.875939in}}%
\pgfpathlineto{\pgfqpoint{0.956719in}{2.880187in}}%
\pgfpathlineto{\pgfqpoint{1.050205in}{2.883735in}}%
\pgfpathlineto{\pgfqpoint{1.189403in}{2.886592in}}%
\pgfpathlineto{\pgfqpoint{1.557024in}{2.889402in}}%
\pgfpathlineto{\pgfqpoint{2.135661in}{2.890546in}}%
\pgfpathlineto{\pgfqpoint{3.421281in}{2.890570in}}%
\pgfpathlineto{\pgfqpoint{4.076054in}{2.888886in}}%
\pgfpathlineto{\pgfqpoint{4.310977in}{2.886283in}}%
\pgfpathlineto{\pgfqpoint{4.428407in}{2.882931in}}%
\pgfpathlineto{\pgfqpoint{4.500101in}{2.878851in}}%
\pgfpathlineto{\pgfqpoint{4.549943in}{2.873919in}}%
\pgfpathlineto{\pgfqpoint{4.584413in}{2.868437in}}%
\pgfpathlineto{\pgfqpoint{4.612088in}{2.861823in}}%
\pgfpathlineto{\pgfqpoint{4.632913in}{2.854661in}}%
\pgfpathlineto{\pgfqpoint{4.650959in}{2.846006in}}%
\pgfpathlineto{\pgfqpoint{4.666029in}{2.836077in}}%
\pgfpathlineto{\pgfqpoint{4.678088in}{2.825459in}}%
\pgfpathlineto{\pgfqpoint{4.688803in}{2.813096in}}%
\pgfpathlineto{\pgfqpoint{4.698007in}{2.799228in}}%
\pgfpathlineto{\pgfqpoint{4.706832in}{2.782077in}}%
\pgfpathlineto{\pgfqpoint{4.714867in}{2.761659in}}%
\pgfpathlineto{\pgfqpoint{4.722583in}{2.735749in}}%
\pgfpathlineto{\pgfqpoint{4.729522in}{2.704386in}}%
\pgfpathlineto{\pgfqpoint{4.736316in}{2.662796in}}%
\pgfpathlineto{\pgfqpoint{4.742379in}{2.610990in}}%
\pgfpathlineto{\pgfqpoint{4.747849in}{2.544077in}}%
\pgfpathlineto{\pgfqpoint{4.752591in}{2.457126in}}%
\pgfpathlineto{\pgfqpoint{4.756576in}{2.340225in}}%
\pgfpathlineto{\pgfqpoint{4.759230in}{2.190905in}}%
\pgfpathlineto{\pgfqpoint{4.760103in}{2.014176in}}%
\pgfpathlineto{\pgfqpoint{4.758845in}{1.825003in}}%
\pgfpathlineto{\pgfqpoint{4.755441in}{1.643334in}}%
\pgfpathlineto{\pgfqpoint{4.750221in}{1.484141in}}%
\pgfpathlineto{\pgfqpoint{4.743741in}{1.357411in}}%
\pgfpathlineto{\pgfqpoint{4.736388in}{1.255704in}}%
\pgfpathlineto{\pgfqpoint{4.727733in}{1.166646in}}%
\pgfpathlineto{\pgfqpoint{4.718470in}{1.095246in}}%
\pgfpathlineto{\pgfqpoint{4.708291in}{1.034121in}}%
\pgfpathlineto{\pgfqpoint{4.697652in}{0.983290in}}%
\pgfpathlineto{\pgfqpoint{4.685894in}{0.937956in}}%
\pgfpathlineto{\pgfqpoint{4.673235in}{0.898204in}}%
\pgfpathlineto{\pgfqpoint{4.660050in}{0.864058in}}%
\pgfpathlineto{\pgfqpoint{4.645778in}{0.833282in}}%
\pgfpathlineto{\pgfqpoint{4.630656in}{0.805947in}}%
\pgfpathlineto{\pgfqpoint{4.615010in}{0.782046in}}%
\pgfpathlineto{\pgfqpoint{4.597743in}{0.759655in}}%
\pgfpathlineto{\pgfqpoint{4.578930in}{0.738959in}}%
\pgfpathlineto{\pgfqpoint{4.558714in}{0.720074in}}%
\pgfpathlineto{\pgfqpoint{4.537283in}{0.703032in}}%
\pgfpathlineto{\pgfqpoint{4.512933in}{0.686592in}}%
\pgfpathlineto{\pgfqpoint{4.487645in}{0.672119in}}%
\pgfpathlineto{\pgfqpoint{4.459604in}{0.658537in}}%
\pgfpathlineto{\pgfqpoint{4.426807in}{0.645220in}}%
\pgfpathlineto{\pgfqpoint{4.391330in}{0.633293in}}%
\pgfpathlineto{\pgfqpoint{4.351149in}{0.622231in}}%
\pgfpathlineto{\pgfqpoint{4.306302in}{0.612308in}}%
\pgfpathlineto{\pgfqpoint{4.256839in}{0.603714in}}%
\pgfpathlineto{\pgfqpoint{4.202812in}{0.596630in}}%
\pgfpathlineto{\pgfqpoint{4.142110in}{0.590921in}}%
\pgfpathlineto{\pgfqpoint{4.076939in}{0.587082in}}%
\pgfpathlineto{\pgfqpoint{4.007350in}{0.585219in}}%
\pgfpathlineto{\pgfqpoint{3.935566in}{0.585543in}}%
\pgfpathlineto{\pgfqpoint{3.863817in}{0.588081in}}%
\pgfpathlineto{\pgfqpoint{3.798667in}{0.592373in}}%
\pgfpathlineto{\pgfqpoint{3.731480in}{0.598961in}}%
\pgfpathlineto{\pgfqpoint{3.668826in}{0.607352in}}%
\pgfpathlineto{\pgfqpoint{3.612905in}{0.617019in}}%
\pgfpathlineto{\pgfqpoint{3.561594in}{0.628027in}}%
\pgfpathlineto{\pgfqpoint{3.514935in}{0.640188in}}%
\pgfpathlineto{\pgfqpoint{3.504395in}{0.643261in}}%
\pgfpathlineto{\pgfqpoint{3.504395in}{0.643261in}}%
\pgfusepath{stroke}%
\end{pgfscope}%
\begin{pgfscope}%
\pgfpathrectangle{\pgfqpoint{0.448634in}{0.402556in}}{\pgfqpoint{4.350661in}{2.489204in}} %
\pgfusepath{clip}%
\pgfsetrectcap%
\pgfsetroundjoin%
\pgfsetlinewidth{1.003750pt}%
\definecolor{currentstroke}{rgb}{0.549020,0.337255,0.294118}%
\pgfsetstrokecolor{currentstroke}%
\pgfsetdash{}{0pt}%
\pgfpathmoveto{\pgfqpoint{2.795522in}{1.982745in}}%
\pgfpathlineto{\pgfqpoint{2.781781in}{1.874357in}}%
\pgfpathlineto{\pgfqpoint{2.769352in}{1.758234in}}%
\pgfpathlineto{\pgfqpoint{2.758095in}{1.631942in}}%
\pgfpathlineto{\pgfqpoint{2.747786in}{1.490551in}}%
\pgfpathlineto{\pgfqpoint{2.738644in}{1.334082in}}%
\pgfpathlineto{\pgfqpoint{2.730580in}{1.157591in}}%
\pgfpathlineto{\pgfqpoint{2.723334in}{0.948663in}}%
\pgfpathlineto{\pgfqpoint{2.709611in}{0.528306in}}%
\pgfpathlineto{\pgfqpoint{2.705549in}{0.486255in}}%
\pgfpathlineto{\pgfqpoint{2.701222in}{0.461873in}}%
\pgfpathlineto{\pgfqpoint{2.696139in}{0.445470in}}%
\pgfpathlineto{\pgfqpoint{2.690593in}{0.434792in}}%
\pgfpathlineto{\pgfqpoint{2.684638in}{0.427559in}}%
\pgfpathlineto{\pgfqpoint{2.677476in}{0.421933in}}%
\pgfpathlineto{\pgfqpoint{2.667496in}{0.417030in}}%
\pgfpathlineto{\pgfqpoint{2.654866in}{0.413302in}}%
\pgfpathlineto{\pgfqpoint{2.635500in}{0.410062in}}%
\pgfpathlineto{\pgfqpoint{2.605133in}{0.407484in}}%
\pgfpathlineto{\pgfqpoint{2.550780in}{0.405468in}}%
\pgfpathlineto{\pgfqpoint{2.442021in}{0.404078in}}%
\pgfpathlineto{\pgfqpoint{2.146177in}{0.403236in}}%
\pgfpathlineto{\pgfqpoint{1.012830in}{0.402982in}}%
\pgfpathlineto{\pgfqpoint{0.512508in}{0.404245in}}%
\pgfpathlineto{\pgfqpoint{0.464690in}{0.406187in}}%
\pgfpathlineto{\pgfqpoint{0.456137in}{0.407918in}}%
\pgfpathlineto{\pgfqpoint{0.452344in}{0.410285in}}%
\pgfpathlineto{\pgfqpoint{0.450347in}{0.414641in}}%
\pgfpathlineto{\pgfqpoint{0.449265in}{0.424502in}}%
\pgfpathlineto{\pgfqpoint{0.448771in}{0.464323in}}%
\pgfpathlineto{\pgfqpoint{0.448640in}{0.850149in}}%
\pgfpathlineto{\pgfqpoint{0.448679in}{2.891296in}}%
\pgfpathlineto{\pgfqpoint{0.448679in}{2.891296in}}%
\pgfusepath{stroke}%
\end{pgfscope}%
\begin{pgfscope}%
\pgfpathrectangle{\pgfqpoint{0.448634in}{0.402556in}}{\pgfqpoint{4.350661in}{2.489204in}} %
\pgfusepath{clip}%
\pgfsetrectcap%
\pgfsetroundjoin%
\pgfsetlinewidth{1.003750pt}%
\definecolor{currentstroke}{rgb}{0.549020,0.337255,0.294118}%
\pgfsetstrokecolor{currentstroke}%
\pgfsetdash{}{0pt}%
\pgfpathmoveto{\pgfqpoint{3.427593in}{0.402582in}}%
\pgfpathlineto{\pgfqpoint{2.777174in}{0.403707in}}%
\pgfpathlineto{\pgfqpoint{2.751141in}{0.405699in}}%
\pgfpathlineto{\pgfqpoint{2.742689in}{0.407978in}}%
\pgfpathlineto{\pgfqpoint{2.737081in}{0.411723in}}%
\pgfpathlineto{\pgfqpoint{2.733117in}{0.417603in}}%
\pgfpathlineto{\pgfqpoint{2.730227in}{0.426971in}}%
\pgfpathlineto{\pgfqpoint{2.727859in}{0.444168in}}%
\pgfpathlineto{\pgfqpoint{2.726085in}{0.478953in}}%
\pgfpathlineto{\pgfqpoint{2.724955in}{0.551126in}}%
\pgfpathlineto{\pgfqpoint{2.725326in}{0.688030in}}%
\pgfpathlineto{\pgfqpoint{2.728061in}{0.872203in}}%
\pgfpathlineto{\pgfqpoint{2.733127in}{1.071253in}}%
\pgfpathlineto{\pgfqpoint{2.739989in}{1.250302in}}%
\pgfpathlineto{\pgfqpoint{2.748890in}{1.421752in}}%
\pgfpathlineto{\pgfqpoint{2.758918in}{1.570661in}}%
\pgfpathlineto{\pgfqpoint{2.771664in}{1.724286in}}%
\pgfpathlineto{\pgfqpoint{2.783939in}{1.840429in}}%
\pgfpathlineto{\pgfqpoint{2.800726in}{1.975946in}}%
\pgfpathlineto{\pgfqpoint{2.815811in}{2.074003in}}%
\pgfpathlineto{\pgfqpoint{2.832735in}{2.169128in}}%
\pgfpathlineto{\pgfqpoint{2.852317in}{2.266139in}}%
\pgfpathlineto{\pgfqpoint{2.892467in}{2.459844in}}%
\pgfpathlineto{\pgfqpoint{2.898594in}{2.501568in}}%
\pgfpathlineto{\pgfqpoint{2.900948in}{2.531307in}}%
\pgfpathlineto{\pgfqpoint{2.900878in}{2.556188in}}%
\pgfpathlineto{\pgfqpoint{2.898567in}{2.578420in}}%
\pgfpathlineto{\pgfqpoint{2.894205in}{2.597680in}}%
\pgfpathlineto{\pgfqpoint{2.888344in}{2.613747in}}%
\pgfpathlineto{\pgfqpoint{2.880552in}{2.628700in}}%
\pgfpathlineto{\pgfqpoint{2.871017in}{2.642269in}}%
\pgfpathlineto{\pgfqpoint{2.858400in}{2.655963in}}%
\pgfpathlineto{\pgfqpoint{2.844433in}{2.667826in}}%
\pgfpathlineto{\pgfqpoint{2.827607in}{2.679263in}}%
\pgfpathlineto{\pgfqpoint{2.808004in}{2.690037in}}%
\pgfpathlineto{\pgfqpoint{2.783677in}{2.700850in}}%
\pgfpathlineto{\pgfqpoint{2.754617in}{2.711258in}}%
\pgfpathlineto{\pgfqpoint{2.720870in}{2.720988in}}%
\pgfpathlineto{\pgfqpoint{2.680358in}{2.730334in}}%
\pgfpathlineto{\pgfqpoint{2.633101in}{2.738967in}}%
\pgfpathlineto{\pgfqpoint{2.576978in}{2.746954in}}%
\pgfpathlineto{\pgfqpoint{2.509840in}{2.754167in}}%
\pgfpathlineto{\pgfqpoint{2.433880in}{2.760067in}}%
\pgfpathlineto{\pgfqpoint{2.346958in}{2.764584in}}%
\pgfpathlineto{\pgfqpoint{2.253449in}{2.767228in}}%
\pgfpathlineto{\pgfqpoint{2.149037in}{2.768110in}}%
\pgfpathlineto{\pgfqpoint{2.040279in}{2.766787in}}%
\pgfpathlineto{\pgfqpoint{1.929384in}{2.763196in}}%
\pgfpathlineto{\pgfqpoint{1.827264in}{2.757538in}}%
\pgfpathlineto{\pgfqpoint{1.727437in}{2.749713in}}%
\pgfpathlineto{\pgfqpoint{1.642965in}{2.740732in}}%
\pgfpathlineto{\pgfqpoint{1.565195in}{2.730223in}}%
\pgfpathlineto{\pgfqpoint{1.464011in}{2.713822in}}%
\pgfpathlineto{\pgfqpoint{1.404207in}{2.700620in}}%
\pgfpathlineto{\pgfqpoint{1.351202in}{2.686718in}}%
\pgfpathlineto{\pgfqpoint{1.302893in}{2.671834in}}%
\pgfpathlineto{\pgfqpoint{1.257240in}{2.655418in}}%
\pgfpathlineto{\pgfqpoint{1.216357in}{2.638405in}}%
\pgfpathlineto{\pgfqpoint{1.178275in}{2.620039in}}%
\pgfpathlineto{\pgfqpoint{1.143045in}{2.600496in}}%
\pgfpathlineto{\pgfqpoint{1.110719in}{2.579955in}}%
\pgfpathlineto{\pgfqpoint{1.092573in}{2.566318in}}%
\pgfpathlineto{\pgfqpoint{1.078219in}{2.555069in}}%
\pgfpathlineto{\pgfqpoint{1.050170in}{2.531500in}}%
\pgfpathlineto{\pgfqpoint{1.025116in}{2.507587in}}%
\pgfpathlineto{\pgfqpoint{1.001399in}{2.481952in}}%
\pgfpathlineto{\pgfqpoint{0.979125in}{2.454675in}}%
\pgfpathlineto{\pgfqpoint{0.958368in}{2.425873in}}%
\pgfpathlineto{\pgfqpoint{0.937941in}{2.393635in}}%
\pgfpathlineto{\pgfqpoint{0.921769in}{2.364118in}}%
\pgfpathlineto{\pgfqpoint{0.903605in}{2.327265in}}%
\pgfpathlineto{\pgfqpoint{0.887304in}{2.289287in}}%
\pgfpathlineto{\pgfqpoint{0.871944in}{2.248078in}}%
\pgfpathlineto{\pgfqpoint{0.856988in}{2.201334in}}%
\pgfpathlineto{\pgfqpoint{0.846194in}{2.160881in}}%
\pgfpathlineto{\pgfqpoint{0.825603in}{2.066728in}}%
\pgfpathlineto{\pgfqpoint{0.816286in}{2.010487in}}%
\pgfpathlineto{\pgfqpoint{0.807258in}{1.941565in}}%
\pgfpathlineto{\pgfqpoint{0.800021in}{1.869859in}}%
\pgfpathlineto{\pgfqpoint{0.793905in}{1.783022in}}%
\pgfpathlineto{\pgfqpoint{0.789988in}{1.693526in}}%
\pgfpathlineto{\pgfqpoint{0.786696in}{1.561653in}}%
\pgfpathlineto{\pgfqpoint{0.785705in}{1.521842in}}%
\pgfpathlineto{\pgfqpoint{0.785705in}{1.521842in}}%
\pgfusepath{stroke}%
\end{pgfscope}%
\begin{pgfscope}%
\pgfpathrectangle{\pgfqpoint{0.448634in}{0.402556in}}{\pgfqpoint{4.350661in}{2.489204in}} %
\pgfusepath{clip}%
\pgfsetrectcap%
\pgfsetroundjoin%
\pgfsetlinewidth{1.003750pt}%
\definecolor{currentstroke}{rgb}{0.890196,0.466667,0.760784}%
\pgfsetstrokecolor{currentstroke}%
\pgfsetdash{}{0pt}%
\pgfpathmoveto{\pgfqpoint{0.448634in}{2.896245in}}%
\pgfpathlineto{\pgfqpoint{0.448593in}{0.407043in}}%
\pgfpathlineto{\pgfqpoint{0.448593in}{0.407043in}}%
\pgfusepath{stroke}%
\end{pgfscope}%
\begin{pgfscope}%
\pgfpathrectangle{\pgfqpoint{0.448634in}{0.402556in}}{\pgfqpoint{4.350661in}{2.489204in}} %
\pgfusepath{clip}%
\pgfsetrectcap%
\pgfsetroundjoin%
\pgfsetlinewidth{1.003750pt}%
\definecolor{currentstroke}{rgb}{0.890196,0.466667,0.760784}%
\pgfsetstrokecolor{currentstroke}%
\pgfsetdash{}{0pt}%
\pgfpathmoveto{\pgfqpoint{0.576832in}{1.762889in}}%
\pgfpathlineto{\pgfqpoint{0.569221in}{1.844567in}}%
\pgfpathlineto{\pgfqpoint{0.562998in}{1.936389in}}%
\pgfpathlineto{\pgfqpoint{0.558536in}{2.035823in}}%
\pgfpathlineto{\pgfqpoint{0.556078in}{2.140330in}}%
\pgfpathlineto{\pgfqpoint{0.555817in}{2.244873in}}%
\pgfpathlineto{\pgfqpoint{0.557722in}{2.341925in}}%
\pgfpathlineto{\pgfqpoint{0.561579in}{2.428932in}}%
\pgfpathlineto{\pgfqpoint{0.567043in}{2.503342in}}%
\pgfpathlineto{\pgfqpoint{0.573732in}{2.565094in}}%
\pgfpathlineto{\pgfqpoint{0.581500in}{2.616600in}}%
\pgfpathlineto{\pgfqpoint{0.589830in}{2.657822in}}%
\pgfpathlineto{\pgfqpoint{0.599297in}{2.693546in}}%
\pgfpathlineto{\pgfqpoint{0.608785in}{2.721364in}}%
\pgfpathlineto{\pgfqpoint{0.619397in}{2.745894in}}%
\pgfpathlineto{\pgfqpoint{0.630954in}{2.766969in}}%
\pgfpathlineto{\pgfqpoint{0.643093in}{2.784532in}}%
\pgfpathlineto{\pgfqpoint{0.656910in}{2.800385in}}%
\pgfpathlineto{\pgfqpoint{0.672273in}{2.814250in}}%
\pgfpathlineto{\pgfqpoint{0.688912in}{2.826034in}}%
\pgfpathlineto{\pgfqpoint{0.706508in}{2.835840in}}%
\pgfpathlineto{\pgfqpoint{0.726832in}{2.844689in}}%
\pgfpathlineto{\pgfqpoint{0.751888in}{2.853044in}}%
\pgfpathlineto{\pgfqpoint{0.781648in}{2.860414in}}%
\pgfpathlineto{\pgfqpoint{0.818182in}{2.866944in}}%
\pgfpathlineto{\pgfqpoint{0.863592in}{2.872596in}}%
\pgfpathlineto{\pgfqpoint{0.922171in}{2.877442in}}%
\pgfpathlineto{\pgfqpoint{1.000401in}{2.881508in}}%
\pgfpathlineto{\pgfqpoint{1.111303in}{2.884838in}}%
\pgfpathlineto{\pgfqpoint{1.274438in}{2.887329in}}%
\pgfpathlineto{\pgfqpoint{1.552875in}{2.889234in}}%
\pgfpathlineto{\pgfqpoint{2.103232in}{2.890437in}}%
\pgfpathlineto{\pgfqpoint{3.327943in}{2.890570in}}%
\pgfpathlineto{\pgfqpoint{4.034923in}{2.888956in}}%
\pgfpathlineto{\pgfqpoint{4.287251in}{2.886415in}}%
\pgfpathlineto{\pgfqpoint{4.411210in}{2.883135in}}%
\pgfpathlineto{\pgfqpoint{4.487262in}{2.879090in}}%
\pgfpathlineto{\pgfqpoint{4.541463in}{2.874081in}}%
\pgfpathlineto{\pgfqpoint{4.578112in}{2.868466in}}%
\pgfpathlineto{\pgfqpoint{4.605830in}{2.862090in}}%
\pgfpathlineto{\pgfqpoint{4.626736in}{2.855237in}}%
\pgfpathlineto{\pgfqpoint{4.644934in}{2.847005in}}%
\pgfpathlineto{\pgfqpoint{4.660247in}{2.837570in}}%
\pgfpathlineto{\pgfqpoint{4.672626in}{2.827444in}}%
\pgfpathlineto{\pgfqpoint{4.683749in}{2.815563in}}%
\pgfpathlineto{\pgfqpoint{4.693364in}{2.802069in}}%
\pgfpathlineto{\pgfqpoint{4.702692in}{2.785273in}}%
\pgfpathlineto{\pgfqpoint{4.711225in}{2.765122in}}%
\pgfpathlineto{\pgfqpoint{4.719427in}{2.739410in}}%
\pgfpathlineto{\pgfqpoint{4.726838in}{2.708189in}}%
\pgfpathlineto{\pgfqpoint{4.733661in}{2.669141in}}%
\pgfpathlineto{\pgfqpoint{4.740402in}{2.614930in}}%
\pgfpathlineto{\pgfqpoint{4.745712in}{2.553001in}}%
\pgfpathlineto{\pgfqpoint{4.750024in}{2.478492in}}%
\pgfpathlineto{\pgfqpoint{4.754169in}{2.374055in}}%
\pgfpathlineto{\pgfqpoint{4.757170in}{2.242174in}}%
\pgfpathlineto{\pgfqpoint{4.758716in}{2.075408in}}%
\pgfpathlineto{\pgfqpoint{4.758187in}{1.888720in}}%
\pgfpathlineto{\pgfqpoint{4.755483in}{1.707036in}}%
\pgfpathlineto{\pgfqpoint{4.750833in}{1.540347in}}%
\pgfpathlineto{\pgfqpoint{4.744893in}{1.408597in}}%
\pgfpathlineto{\pgfqpoint{4.737305in}{1.289436in}}%
\pgfpathlineto{\pgfqpoint{4.729228in}{1.200306in}}%
\pgfpathlineto{\pgfqpoint{4.720405in}{1.126319in}}%
\pgfpathlineto{\pgfqpoint{4.710649in}{1.062575in}}%
\pgfpathlineto{\pgfqpoint{4.699820in}{1.006686in}}%
\pgfpathlineto{\pgfqpoint{4.688236in}{0.958705in}}%
\pgfpathlineto{\pgfqpoint{4.676344in}{0.918642in}}%
\pgfpathlineto{\pgfqpoint{4.663077in}{0.881830in}}%
\pgfpathlineto{\pgfqpoint{4.649546in}{0.850618in}}%
\pgfpathlineto{\pgfqpoint{4.635185in}{0.822750in}}%
\pgfpathlineto{\pgfqpoint{4.618974in}{0.796246in}}%
\pgfpathlineto{\pgfqpoint{4.602462in}{0.773129in}}%
\pgfpathlineto{\pgfqpoint{4.584251in}{0.751739in}}%
\pgfpathlineto{\pgfqpoint{4.564586in}{0.732108in}}%
\pgfpathlineto{\pgfqpoint{4.543633in}{0.714305in}}%
\pgfpathlineto{\pgfqpoint{4.519714in}{0.697055in}}%
\pgfpathlineto{\pgfqpoint{4.494764in}{0.681835in}}%
\pgfpathlineto{\pgfqpoint{4.467000in}{0.667531in}}%
\pgfpathlineto{\pgfqpoint{4.436494in}{0.654293in}}%
\pgfpathlineto{\pgfqpoint{4.403327in}{0.642234in}}%
\pgfpathlineto{\pgfqpoint{4.363342in}{0.630281in}}%
\pgfpathlineto{\pgfqpoint{4.320776in}{0.620004in}}%
\pgfpathlineto{\pgfqpoint{4.273578in}{0.610970in}}%
\pgfpathlineto{\pgfqpoint{4.221802in}{0.603325in}}%
\pgfpathlineto{\pgfqpoint{4.165495in}{0.597268in}}%
\pgfpathlineto{\pgfqpoint{4.096021in}{0.592330in}}%
\pgfpathlineto{\pgfqpoint{4.026447in}{0.589841in}}%
\pgfpathlineto{\pgfqpoint{3.956839in}{0.589586in}}%
\pgfpathlineto{\pgfqpoint{3.885076in}{0.591570in}}%
\pgfpathlineto{\pgfqpoint{3.813389in}{0.595817in}}%
\pgfpathlineto{\pgfqpoint{3.746181in}{0.602104in}}%
\pgfpathlineto{\pgfqpoint{3.683499in}{0.610213in}}%
\pgfpathlineto{\pgfqpoint{3.625398in}{0.620030in}}%
\pgfpathlineto{\pgfqpoint{3.571920in}{0.631321in}}%
\pgfpathlineto{\pgfqpoint{3.525237in}{0.643357in}}%
\pgfpathlineto{\pgfqpoint{3.481150in}{0.657031in}}%
\pgfpathlineto{\pgfqpoint{3.445909in}{0.669846in}}%
\pgfpathlineto{\pgfqpoint{3.403055in}{0.687909in}}%
\pgfpathlineto{\pgfqpoint{3.367224in}{0.705964in}}%
\pgfpathlineto{\pgfqpoint{3.336326in}{0.724287in}}%
\pgfpathlineto{\pgfqpoint{3.308362in}{0.743516in}}%
\pgfpathlineto{\pgfqpoint{3.283306in}{0.763314in}}%
\pgfpathlineto{\pgfqpoint{3.259428in}{0.784934in}}%
\pgfpathlineto{\pgfqpoint{3.236906in}{0.808381in}}%
\pgfpathlineto{\pgfqpoint{3.215894in}{0.833596in}}%
\pgfpathlineto{\pgfqpoint{3.196512in}{0.860468in}}%
\pgfpathlineto{\pgfqpoint{3.178837in}{0.888838in}}%
\pgfpathlineto{\pgfqpoint{3.162901in}{0.918528in}}%
\pgfpathlineto{\pgfqpoint{3.146744in}{0.953796in}}%
\pgfpathlineto{\pgfqpoint{3.133529in}{0.987928in}}%
\pgfpathlineto{\pgfqpoint{3.121413in}{1.025257in}}%
\pgfpathlineto{\pgfqpoint{3.110607in}{1.065721in}}%
\pgfpathlineto{\pgfqpoint{3.101257in}{1.109225in}}%
\pgfpathlineto{\pgfqpoint{3.092614in}{1.163079in}}%
\pgfpathlineto{\pgfqpoint{3.087003in}{1.212442in}}%
\pgfpathlineto{\pgfqpoint{3.083304in}{1.264539in}}%
\pgfpathlineto{\pgfqpoint{3.081643in}{1.319263in}}%
\pgfpathlineto{\pgfqpoint{3.082191in}{1.376507in}}%
\pgfpathlineto{\pgfqpoint{3.084962in}{1.433666in}}%
\pgfpathlineto{\pgfqpoint{3.089910in}{1.490632in}}%
\pgfpathlineto{\pgfqpoint{3.097046in}{1.547295in}}%
\pgfpathlineto{\pgfqpoint{3.105960in}{1.601094in}}%
\pgfpathlineto{\pgfqpoint{3.116931in}{1.654393in}}%
\pgfpathlineto{\pgfqpoint{3.130062in}{1.707049in}}%
\pgfpathlineto{\pgfqpoint{3.144696in}{1.756562in}}%
\pgfpathlineto{\pgfqpoint{3.160605in}{1.802892in}}%
\pgfpathlineto{\pgfqpoint{3.178464in}{1.848282in}}%
\pgfpathlineto{\pgfqpoint{3.198355in}{1.892552in}}%
\pgfpathlineto{\pgfqpoint{3.219148in}{1.933421in}}%
\pgfpathlineto{\pgfqpoint{3.241770in}{1.972996in}}%
\pgfpathlineto{\pgfqpoint{3.266148in}{2.011182in}}%
\pgfpathlineto{\pgfqpoint{3.292159in}{2.047930in}}%
\pgfpathlineto{\pgfqpoint{3.321095in}{2.085102in}}%
\pgfpathlineto{\pgfqpoint{3.357429in}{2.127997in}}%
\pgfpathlineto{\pgfqpoint{3.412203in}{2.192017in}}%
\pgfpathlineto{\pgfqpoint{3.424205in}{2.209696in}}%
\pgfpathlineto{\pgfqpoint{3.429429in}{2.220593in}}%
\pgfpathlineto{\pgfqpoint{3.431557in}{2.230199in}}%
\pgfpathlineto{\pgfqpoint{3.430491in}{2.237496in}}%
\pgfpathlineto{\pgfqpoint{3.426493in}{2.243305in}}%
\pgfpathlineto{\pgfqpoint{3.420837in}{2.246976in}}%
\pgfpathlineto{\pgfqpoint{3.412450in}{2.249561in}}%
\pgfpathlineto{\pgfqpoint{3.399453in}{2.250727in}}%
\pgfpathlineto{\pgfqpoint{3.384258in}{2.249740in}}%
\pgfpathlineto{\pgfqpoint{3.364936in}{2.246183in}}%
\pgfpathlineto{\pgfqpoint{3.341753in}{2.239432in}}%
\pgfpathlineto{\pgfqpoint{3.317060in}{2.229767in}}%
\pgfpathlineto{\pgfqpoint{3.291059in}{2.217062in}}%
\pgfpathlineto{\pgfqpoint{3.265887in}{2.202327in}}%
\pgfpathlineto{\pgfqpoint{3.239770in}{2.184415in}}%
\pgfpathlineto{\pgfqpoint{3.214747in}{2.164561in}}%
\pgfpathlineto{\pgfqpoint{3.190879in}{2.142926in}}%
\pgfpathlineto{\pgfqpoint{3.166643in}{2.117936in}}%
\pgfpathlineto{\pgfqpoint{3.143829in}{2.091249in}}%
\pgfpathlineto{\pgfqpoint{3.121078in}{2.061117in}}%
\pgfpathlineto{\pgfqpoint{3.099957in}{2.029468in}}%
\pgfpathlineto{\pgfqpoint{3.079260in}{1.994407in}}%
\pgfpathlineto{\pgfqpoint{3.059230in}{1.955914in}}%
\pgfpathlineto{\pgfqpoint{3.040074in}{1.914012in}}%
\pgfpathlineto{\pgfqpoint{3.022826in}{1.871037in}}%
\pgfpathlineto{\pgfqpoint{3.005808in}{1.822532in}}%
\pgfpathlineto{\pgfqpoint{2.990084in}{1.770815in}}%
\pgfpathlineto{\pgfqpoint{2.975725in}{1.715976in}}%
\pgfpathlineto{\pgfqpoint{2.962299in}{1.655677in}}%
\pgfpathlineto{\pgfqpoint{2.950507in}{1.592384in}}%
\pgfpathlineto{\pgfqpoint{2.940390in}{1.526184in}}%
\pgfpathlineto{\pgfqpoint{2.931748in}{1.454681in}}%
\pgfpathlineto{\pgfqpoint{2.925080in}{1.380399in}}%
\pgfpathlineto{\pgfqpoint{2.920642in}{1.305900in}}%
\pgfpathlineto{\pgfqpoint{2.918434in}{1.231270in}}%
\pgfpathlineto{\pgfqpoint{2.918530in}{1.159087in}}%
\pgfpathlineto{\pgfqpoint{2.920768in}{1.091932in}}%
\pgfpathlineto{\pgfqpoint{2.925155in}{1.027412in}}%
\pgfpathlineto{\pgfqpoint{2.931165in}{0.970579in}}%
\pgfpathlineto{\pgfqpoint{2.938731in}{0.919033in}}%
\pgfpathlineto{\pgfqpoint{2.947618in}{0.872850in}}%
\pgfpathlineto{\pgfqpoint{2.958179in}{0.829712in}}%
\pgfpathlineto{\pgfqpoint{2.969633in}{0.792111in}}%
\pgfpathlineto{\pgfqpoint{2.982425in}{0.757769in}}%
\pgfpathlineto{\pgfqpoint{2.996386in}{0.726808in}}%
\pgfpathlineto{\pgfqpoint{3.011259in}{0.699295in}}%
\pgfpathlineto{\pgfqpoint{3.026699in}{0.675220in}}%
\pgfpathlineto{\pgfqpoint{3.043788in}{0.652651in}}%
\pgfpathlineto{\pgfqpoint{3.062456in}{0.631785in}}%
\pgfpathlineto{\pgfqpoint{3.082565in}{0.612751in}}%
\pgfpathlineto{\pgfqpoint{3.103925in}{0.595592in}}%
\pgfpathlineto{\pgfqpoint{3.128233in}{0.579072in}}%
\pgfpathlineto{\pgfqpoint{3.153503in}{0.564559in}}%
\pgfpathlineto{\pgfqpoint{3.181539in}{0.550961in}}%
\pgfpathlineto{\pgfqpoint{3.214341in}{0.537661in}}%
\pgfpathlineto{\pgfqpoint{3.249818in}{0.525734in}}%
\pgfpathlineto{\pgfqpoint{3.289983in}{0.514596in}}%
\pgfpathlineto{\pgfqpoint{3.334794in}{0.504454in}}%
\pgfpathlineto{\pgfqpoint{3.386346in}{0.495038in}}%
\pgfpathlineto{\pgfqpoint{3.446773in}{0.486301in}}%
\pgfpathlineto{\pgfqpoint{3.518218in}{0.478330in}}%
\pgfpathlineto{\pgfqpoint{3.600660in}{0.471462in}}%
\pgfpathlineto{\pgfqpoint{3.696244in}{0.465774in}}%
\pgfpathlineto{\pgfqpoint{3.807120in}{0.461435in}}%
\pgfpathlineto{\pgfqpoint{3.931092in}{0.458821in}}%
\pgfpathlineto{\pgfqpoint{4.061609in}{0.458284in}}%
\pgfpathlineto{\pgfqpoint{4.185593in}{0.459958in}}%
\pgfpathlineto{\pgfqpoint{4.292137in}{0.463530in}}%
\pgfpathlineto{\pgfqpoint{4.379038in}{0.468537in}}%
\pgfpathlineto{\pgfqpoint{4.448444in}{0.474607in}}%
\pgfpathlineto{\pgfqpoint{4.504665in}{0.481633in}}%
\pgfpathlineto{\pgfqpoint{4.549837in}{0.489395in}}%
\pgfpathlineto{\pgfqpoint{4.586088in}{0.497733in}}%
\pgfpathlineto{\pgfqpoint{4.615539in}{0.506583in}}%
\pgfpathlineto{\pgfqpoint{4.640261in}{0.516150in}}%
\pgfpathlineto{\pgfqpoint{4.662203in}{0.527051in}}%
\pgfpathlineto{\pgfqpoint{4.679344in}{0.537858in}}%
\pgfpathlineto{\pgfqpoint{4.695407in}{0.550643in}}%
\pgfpathlineto{\pgfqpoint{4.708478in}{0.563775in}}%
\pgfpathlineto{\pgfqpoint{4.720166in}{0.578515in}}%
\pgfpathlineto{\pgfqpoint{4.730333in}{0.594664in}}%
\pgfpathlineto{\pgfqpoint{4.739975in}{0.614149in}}%
\pgfpathlineto{\pgfqpoint{4.748692in}{0.636945in}}%
\pgfpathlineto{\pgfqpoint{4.756272in}{0.662908in}}%
\pgfpathlineto{\pgfqpoint{4.763632in}{0.696715in}}%
\pgfpathlineto{\pgfqpoint{4.769826in}{0.735901in}}%
\pgfpathlineto{\pgfqpoint{4.775613in}{0.787749in}}%
\pgfpathlineto{\pgfqpoint{4.780650in}{0.854706in}}%
\pgfpathlineto{\pgfqpoint{4.784980in}{0.944178in}}%
\pgfpathlineto{\pgfqpoint{4.788604in}{1.068567in}}%
\pgfpathlineto{\pgfqpoint{4.791530in}{1.250247in}}%
\pgfpathlineto{\pgfqpoint{4.793737in}{1.536494in}}%
\pgfpathlineto{\pgfqpoint{4.794913in}{2.001973in}}%
\pgfpathlineto{\pgfqpoint{4.794271in}{2.497323in}}%
\pgfpathlineto{\pgfqpoint{4.792171in}{2.723825in}}%
\pgfpathlineto{\pgfqpoint{4.789501in}{2.808398in}}%
\pgfpathlineto{\pgfqpoint{4.786353in}{2.845550in}}%
\pgfpathlineto{\pgfqpoint{4.782466in}{2.864935in}}%
\pgfpathlineto{\pgfqpoint{4.778777in}{2.873925in}}%
\pgfpathlineto{\pgfqpoint{4.774535in}{2.879566in}}%
\pgfpathlineto{\pgfqpoint{4.769063in}{2.883592in}}%
\pgfpathlineto{\pgfqpoint{4.760811in}{2.886693in}}%
\pgfpathlineto{\pgfqpoint{4.747910in}{2.888889in}}%
\pgfpathlineto{\pgfqpoint{4.721847in}{2.890462in}}%
\pgfpathlineto{\pgfqpoint{4.658768in}{2.891342in}}%
\pgfpathlineto{\pgfqpoint{4.393378in}{2.891699in}}%
\pgfpathlineto{\pgfqpoint{0.786681in}{2.891532in}}%
\pgfpathlineto{\pgfqpoint{0.584381in}{2.890163in}}%
\pgfpathlineto{\pgfqpoint{0.536571in}{2.887878in}}%
\pgfpathlineto{\pgfqpoint{0.514974in}{2.884974in}}%
\pgfpathlineto{\pgfqpoint{0.502288in}{2.881514in}}%
\pgfpathlineto{\pgfqpoint{0.492320in}{2.876594in}}%
\pgfpathlineto{\pgfqpoint{0.485330in}{2.870702in}}%
\pgfpathlineto{\pgfqpoint{0.479736in}{2.863104in}}%
\pgfpathlineto{\pgfqpoint{0.474720in}{2.852084in}}%
\pgfpathlineto{\pgfqpoint{0.470198in}{2.835464in}}%
\pgfpathlineto{\pgfqpoint{0.466342in}{2.810976in}}%
\pgfpathlineto{\pgfqpoint{0.463039in}{2.773835in}}%
\pgfpathlineto{\pgfqpoint{0.460182in}{2.714187in}}%
\pgfpathlineto{\pgfqpoint{0.457755in}{2.612169in}}%
\pgfpathlineto{\pgfqpoint{0.456574in}{2.520079in}}%
\pgfpathlineto{\pgfqpoint{0.456574in}{2.520079in}}%
\pgfusepath{stroke}%
\end{pgfscope}%
\begin{pgfscope}%
\pgfpathrectangle{\pgfqpoint{0.448634in}{0.402556in}}{\pgfqpoint{4.350661in}{2.489204in}} %
\pgfusepath{clip}%
\pgfsetrectcap%
\pgfsetroundjoin%
\pgfsetlinewidth{1.003750pt}%
\definecolor{currentstroke}{rgb}{0.890196,0.466667,0.760784}%
\pgfsetstrokecolor{currentstroke}%
\pgfsetdash{}{0pt}%
\pgfpathmoveto{\pgfqpoint{0.444523in}{0.402719in}}%
\pgfpathlineto{\pgfqpoint{0.457451in}{0.403513in}}%
\pgfpathlineto{\pgfqpoint{0.492251in}{0.402976in}}%
\pgfpathlineto{\pgfqpoint{0.748940in}{0.402675in}}%
\pgfpathlineto{\pgfqpoint{2.582743in}{0.403333in}}%
\pgfpathlineto{\pgfqpoint{2.661035in}{0.405033in}}%
\pgfpathlineto{\pgfqpoint{2.682695in}{0.407226in}}%
\pgfpathlineto{\pgfqpoint{2.693294in}{0.409957in}}%
\pgfpathlineto{\pgfqpoint{2.701056in}{0.414365in}}%
\pgfpathlineto{\pgfqpoint{2.705639in}{0.419639in}}%
\pgfpathlineto{\pgfqpoint{2.709585in}{0.428475in}}%
\pgfpathlineto{\pgfqpoint{2.712731in}{0.442951in}}%
\pgfpathlineto{\pgfqpoint{2.715254in}{0.467667in}}%
\pgfpathlineto{\pgfqpoint{2.717442in}{0.514892in}}%
\pgfpathlineto{\pgfqpoint{2.720022in}{0.631846in}}%
\pgfpathlineto{\pgfqpoint{2.726885in}{0.952855in}}%
\pgfpathlineto{\pgfqpoint{2.733597in}{1.156824in}}%
\pgfpathlineto{\pgfqpoint{2.741553in}{1.333321in}}%
\pgfpathlineto{\pgfqpoint{2.750724in}{1.489788in}}%
\pgfpathlineto{\pgfqpoint{2.761777in}{1.638601in}}%
\pgfpathlineto{\pgfqpoint{2.773777in}{1.769808in}}%
\pgfpathlineto{\pgfqpoint{2.786732in}{1.885856in}}%
\pgfpathlineto{\pgfqpoint{2.793548in}{1.937525in}}%
\pgfpathlineto{\pgfqpoint{2.795983in}{1.944405in}}%
\pgfpathlineto{\pgfqpoint{2.810726in}{2.045056in}}%
\pgfpathlineto{\pgfqpoint{2.826929in}{2.140344in}}%
\pgfpathlineto{\pgfqpoint{2.845299in}{2.235118in}}%
\pgfpathlineto{\pgfqpoint{2.870252in}{2.351136in}}%
\pgfpathlineto{\pgfqpoint{2.873666in}{2.362860in}}%
\pgfpathlineto{\pgfqpoint{2.875882in}{2.367088in}}%
\pgfpathlineto{\pgfqpoint{2.890000in}{2.432323in}}%
\pgfpathlineto{\pgfqpoint{2.899358in}{2.483481in}}%
\pgfpathlineto{\pgfqpoint{2.903663in}{2.517972in}}%
\pgfpathlineto{\pgfqpoint{2.905076in}{2.545296in}}%
\pgfpathlineto{\pgfqpoint{2.904262in}{2.567668in}}%
\pgfpathlineto{\pgfqpoint{2.901507in}{2.587318in}}%
\pgfpathlineto{\pgfqpoint{2.897174in}{2.604009in}}%
\pgfpathlineto{\pgfqpoint{2.890867in}{2.619852in}}%
\pgfpathlineto{\pgfqpoint{2.882625in}{2.634485in}}%
\pgfpathlineto{\pgfqpoint{2.872690in}{2.647673in}}%
\pgfpathlineto{\pgfqpoint{2.859709in}{2.660915in}}%
\pgfpathlineto{\pgfqpoint{2.845451in}{2.672314in}}%
\pgfpathlineto{\pgfqpoint{2.820541in}{2.687600in}}%
\pgfpathlineto{\pgfqpoint{2.798490in}{2.698218in}}%
\pgfpathlineto{\pgfqpoint{2.771721in}{2.708631in}}%
\pgfpathlineto{\pgfqpoint{2.740258in}{2.718512in}}%
\pgfpathlineto{\pgfqpoint{2.702020in}{2.728135in}}%
\pgfpathlineto{\pgfqpoint{2.657021in}{2.737118in}}%
\pgfpathlineto{\pgfqpoint{2.603138in}{2.745521in}}%
\pgfpathlineto{\pgfqpoint{2.540395in}{2.752994in}}%
\pgfpathlineto{\pgfqpoint{2.466651in}{2.759454in}}%
\pgfpathlineto{\pgfqpoint{2.381933in}{2.764575in}}%
\pgfpathlineto{\pgfqpoint{2.286269in}{2.768061in}}%
\pgfpathlineto{\pgfqpoint{2.181863in}{2.769575in}}%
\pgfpathlineto{\pgfqpoint{2.073099in}{2.768912in}}%
\pgfpathlineto{\pgfqpoint{1.964363in}{2.766041in}}%
\pgfpathlineto{\pgfqpoint{1.860039in}{2.761070in}}%
\pgfpathlineto{\pgfqpoint{1.764505in}{2.754377in}}%
\pgfpathlineto{\pgfqpoint{1.660442in}{2.744699in}}%
\pgfpathlineto{\pgfqpoint{1.580449in}{2.734532in}}%
\pgfpathlineto{\pgfqpoint{1.494246in}{2.721146in}}%
\pgfpathlineto{\pgfqpoint{1.432109in}{2.708701in}}%
\pgfpathlineto{\pgfqpoint{1.376748in}{2.695464in}}%
\pgfpathlineto{\pgfqpoint{1.326059in}{2.681174in}}%
\pgfpathlineto{\pgfqpoint{1.319975in}{2.678591in}}%
\pgfpathlineto{\pgfqpoint{1.316304in}{2.675966in}}%
\pgfpathlineto{\pgfqpoint{1.253986in}{2.653840in}}%
\pgfpathlineto{\pgfqpoint{1.213184in}{2.636572in}}%
\pgfpathlineto{\pgfqpoint{1.175179in}{2.617997in}}%
\pgfpathlineto{\pgfqpoint{1.141969in}{2.599393in}}%
\pgfpathlineto{\pgfqpoint{1.109691in}{2.578754in}}%
\pgfpathlineto{\pgfqpoint{1.092582in}{2.568189in}}%
\pgfpathlineto{\pgfqpoint{1.088570in}{2.566373in}}%
\pgfpathlineto{\pgfqpoint{1.059939in}{2.543737in}}%
\pgfpathlineto{\pgfqpoint{1.034264in}{2.520702in}}%
\pgfpathlineto{\pgfqpoint{1.009862in}{2.495924in}}%
\pgfpathlineto{\pgfqpoint{0.986851in}{2.469460in}}%
\pgfpathlineto{\pgfqpoint{0.965322in}{2.441410in}}%
\pgfpathlineto{\pgfqpoint{0.944053in}{2.409893in}}%
\pgfpathlineto{\pgfqpoint{0.925784in}{2.378966in}}%
\pgfpathlineto{\pgfqpoint{0.898668in}{2.325043in}}%
\pgfpathlineto{\pgfqpoint{0.881584in}{2.284734in}}%
\pgfpathlineto{\pgfqpoint{0.865635in}{2.241108in}}%
\pgfpathlineto{\pgfqpoint{0.850957in}{2.194248in}}%
\pgfpathlineto{\pgfqpoint{0.841745in}{2.158458in}}%
\pgfpathlineto{\pgfqpoint{0.829359in}{2.105567in}}%
\pgfpathlineto{\pgfqpoint{0.817556in}{2.044825in}}%
\pgfpathlineto{\pgfqpoint{0.810103in}{1.995795in}}%
\pgfpathlineto{\pgfqpoint{0.801833in}{1.931774in}}%
\pgfpathlineto{\pgfqpoint{0.794589in}{1.860069in}}%
\pgfpathlineto{\pgfqpoint{0.788511in}{1.778223in}}%
\pgfpathlineto{\pgfqpoint{0.783549in}{1.681312in}}%
\pgfpathlineto{\pgfqpoint{0.773999in}{1.482496in}}%
\pgfpathlineto{\pgfqpoint{0.769117in}{1.438048in}}%
\pgfpathlineto{\pgfqpoint{0.763400in}{1.406368in}}%
\pgfpathlineto{\pgfqpoint{0.757500in}{1.385019in}}%
\pgfpathlineto{\pgfqpoint{0.751086in}{1.369233in}}%
\pgfpathlineto{\pgfqpoint{0.743640in}{1.356998in}}%
\pgfpathlineto{\pgfqpoint{0.737321in}{1.350178in}}%
\pgfpathlineto{\pgfqpoint{0.729805in}{1.345219in}}%
\pgfpathlineto{\pgfqpoint{0.721382in}{1.342868in}}%
\pgfpathlineto{\pgfqpoint{0.712717in}{1.343347in}}%
\pgfpathlineto{\pgfqpoint{0.704397in}{1.346195in}}%
\pgfpathlineto{\pgfqpoint{0.694833in}{1.352082in}}%
\pgfpathlineto{\pgfqpoint{0.684591in}{1.361316in}}%
\pgfpathlineto{\pgfqpoint{0.674078in}{1.373905in}}%
\pgfpathlineto{\pgfqpoint{0.662335in}{1.391815in}}%
\pgfpathlineto{\pgfqpoint{0.651077in}{1.413105in}}%
\pgfpathlineto{\pgfqpoint{0.639460in}{1.439846in}}%
\pgfpathlineto{\pgfqpoint{0.627905in}{1.472083in}}%
\pgfpathlineto{\pgfqpoint{0.616727in}{1.509794in}}%
\pgfpathlineto{\pgfqpoint{0.605630in}{1.555347in}}%
\pgfpathlineto{\pgfqpoint{0.595479in}{1.606309in}}%
\pgfpathlineto{\pgfqpoint{0.584470in}{1.674854in}}%
\pgfpathlineto{\pgfqpoint{0.575604in}{1.746320in}}%
\pgfpathlineto{\pgfqpoint{0.568595in}{1.820561in}}%
\pgfpathlineto{\pgfqpoint{0.562013in}{1.912350in}}%
\pgfpathlineto{\pgfqpoint{0.557193in}{2.011763in}}%
\pgfpathlineto{\pgfqpoint{0.554353in}{2.116257in}}%
\pgfpathlineto{\pgfqpoint{0.553648in}{2.223287in}}%
\pgfpathlineto{\pgfqpoint{0.555198in}{2.325326in}}%
\pgfpathlineto{\pgfqpoint{0.558322in}{2.407389in}}%
\pgfpathlineto{\pgfqpoint{0.563461in}{2.486822in}}%
\pgfpathlineto{\pgfqpoint{0.569743in}{2.551136in}}%
\pgfpathlineto{\pgfqpoint{0.577188in}{2.605226in}}%
\pgfpathlineto{\pgfqpoint{0.585344in}{2.649043in}}%
\pgfpathlineto{\pgfqpoint{0.594114in}{2.684999in}}%
\pgfpathlineto{\pgfqpoint{0.603715in}{2.715427in}}%
\pgfpathlineto{\pgfqpoint{0.613710in}{2.740296in}}%
\pgfpathlineto{\pgfqpoint{0.624637in}{2.761807in}}%
\pgfpathlineto{\pgfqpoint{0.636183in}{2.779886in}}%
\pgfpathlineto{\pgfqpoint{0.649432in}{2.796362in}}%
\pgfpathlineto{\pgfqpoint{0.662581in}{2.809392in}}%
\pgfpathlineto{\pgfqpoint{0.678707in}{2.822073in}}%
\pgfpathlineto{\pgfqpoint{0.695938in}{2.832690in}}%
\pgfpathlineto{\pgfqpoint{0.715996in}{2.842300in}}%
\pgfpathlineto{\pgfqpoint{0.738783in}{2.850636in}}%
\pgfpathlineto{\pgfqpoint{0.766265in}{2.858238in}}%
\pgfpathlineto{\pgfqpoint{0.798388in}{2.864766in}}%
\pgfpathlineto{\pgfqpoint{0.839390in}{2.870692in}}%
\pgfpathlineto{\pgfqpoint{0.891404in}{2.875798in}}%
\pgfpathlineto{\pgfqpoint{0.958735in}{2.880060in}}%
\pgfpathlineto{\pgfqpoint{1.052221in}{2.883630in}}%
\pgfpathlineto{\pgfqpoint{1.189243in}{2.886484in}}%
\pgfpathlineto{\pgfqpoint{1.541637in}{2.889281in}}%
\pgfpathlineto{\pgfqpoint{2.091994in}{2.890474in}}%
\pgfpathlineto{\pgfqpoint{3.331933in}{2.890611in}}%
\pgfpathlineto{\pgfqpoint{4.045439in}{2.889000in}}%
\pgfpathlineto{\pgfqpoint{4.295591in}{2.886468in}}%
\pgfpathlineto{\pgfqpoint{4.419549in}{2.883131in}}%
\pgfpathlineto{\pgfqpoint{4.493424in}{2.879116in}}%
\pgfpathlineto{\pgfqpoint{4.545453in}{2.874241in}}%
\pgfpathlineto{\pgfqpoint{4.582094in}{2.868564in}}%
\pgfpathlineto{\pgfqpoint{4.609791in}{2.862071in}}%
\pgfpathlineto{\pgfqpoint{4.630655in}{2.855055in}}%
\pgfpathlineto{\pgfqpoint{4.648770in}{2.846590in}}%
\pgfpathlineto{\pgfqpoint{4.663947in}{2.836875in}}%
\pgfpathlineto{\pgfqpoint{4.676143in}{2.826462in}}%
\pgfpathlineto{\pgfqpoint{4.687026in}{2.814294in}}%
\pgfpathlineto{\pgfqpoint{4.696390in}{2.800568in}}%
\pgfpathlineto{\pgfqpoint{4.705411in}{2.783552in}}%
\pgfpathlineto{\pgfqpoint{4.713635in}{2.763233in}}%
\pgfpathlineto{\pgfqpoint{4.721532in}{2.737395in}}%
\pgfpathlineto{\pgfqpoint{4.728669in}{2.706090in}}%
\pgfpathlineto{\pgfqpoint{4.735243in}{2.666986in}}%
\pgfpathlineto{\pgfqpoint{4.741766in}{2.612740in}}%
\pgfpathlineto{\pgfqpoint{4.747094in}{2.548311in}}%
\pgfpathlineto{\pgfqpoint{4.751348in}{2.471303in}}%
\pgfpathlineto{\pgfqpoint{4.755345in}{2.364367in}}%
\pgfpathlineto{\pgfqpoint{4.758214in}{2.227502in}}%
\pgfpathlineto{\pgfqpoint{4.759549in}{2.058244in}}%
\pgfpathlineto{\pgfqpoint{4.758801in}{1.869068in}}%
\pgfpathlineto{\pgfqpoint{4.755862in}{1.682410in}}%
\pgfpathlineto{\pgfqpoint{4.751034in}{1.520708in}}%
\pgfpathlineto{\pgfqpoint{4.744996in}{1.391456in}}%
\pgfpathlineto{\pgfqpoint{4.737315in}{1.274799in}}%
\pgfpathlineto{\pgfqpoint{4.729270in}{1.188168in}}%
\pgfpathlineto{\pgfqpoint{4.720241in}{1.114214in}}%
\pgfpathlineto{\pgfqpoint{4.710226in}{1.050523in}}%
\pgfpathlineto{\pgfqpoint{4.699616in}{0.997128in}}%
\pgfpathlineto{\pgfqpoint{4.688422in}{0.951607in}}%
\pgfpathlineto{\pgfqpoint{4.676374in}{0.911606in}}%
\pgfpathlineto{\pgfqpoint{4.663828in}{0.877145in}}%
\pgfpathlineto{\pgfqpoint{4.650238in}{0.845967in}}%
\pgfpathlineto{\pgfqpoint{4.635806in}{0.818147in}}%
\pgfpathlineto{\pgfqpoint{4.619516in}{0.791706in}}%
\pgfpathlineto{\pgfqpoint{4.605731in}{0.772458in}}%
\pgfpathlineto{\pgfqpoint{4.587681in}{0.750892in}}%
\pgfpathlineto{\pgfqpoint{4.568153in}{0.731082in}}%
\pgfpathlineto{\pgfqpoint{4.547315in}{0.713104in}}%
\pgfpathlineto{\pgfqpoint{4.523492in}{0.695681in}}%
\pgfpathlineto{\pgfqpoint{4.498619in}{0.680298in}}%
\pgfpathlineto{\pgfqpoint{4.470919in}{0.665831in}}%
\pgfpathlineto{\pgfqpoint{4.440460in}{0.652455in}}%
\pgfpathlineto{\pgfqpoint{4.407323in}{0.640285in}}%
\pgfpathlineto{\pgfqpoint{4.369478in}{0.628806in}}%
\pgfpathlineto{\pgfqpoint{4.326950in}{0.618330in}}%
\pgfpathlineto{\pgfqpoint{4.279780in}{0.609100in}}%
\pgfpathlineto{\pgfqpoint{4.228025in}{0.601274in}}%
\pgfpathlineto{\pgfqpoint{4.171737in}{0.594982in}}%
\pgfpathlineto{\pgfqpoint{4.104446in}{0.589973in}}%
\pgfpathlineto{\pgfqpoint{4.034880in}{0.587210in}}%
\pgfpathlineto{\pgfqpoint{3.965273in}{0.586657in}}%
\pgfpathlineto{\pgfqpoint{3.893504in}{0.588299in}}%
\pgfpathlineto{\pgfqpoint{3.821801in}{0.592178in}}%
\pgfpathlineto{\pgfqpoint{3.754563in}{0.598042in}}%
\pgfpathlineto{\pgfqpoint{3.691835in}{0.605682in}}%
\pgfpathlineto{\pgfqpoint{3.633667in}{0.614962in}}%
\pgfpathlineto{\pgfqpoint{3.580100in}{0.625688in}}%
\pgfpathlineto{\pgfqpoint{3.531186in}{0.637706in}}%
\pgfpathlineto{\pgfqpoint{3.486963in}{0.650797in}}%
\pgfpathlineto{\pgfqpoint{3.445384in}{0.665433in}}%
\pgfpathlineto{\pgfqpoint{3.406537in}{0.681566in}}%
\pgfpathlineto{\pgfqpoint{3.372524in}{0.698162in}}%
\pgfpathlineto{\pgfqpoint{3.343259in}{0.714666in}}%
\pgfpathlineto{\pgfqpoint{3.316734in}{0.731779in}}%
\pgfpathlineto{\pgfqpoint{3.316734in}{0.731779in}}%
\pgfusepath{stroke}%
\end{pgfscope}%
\begin{pgfscope}%
\pgfpathrectangle{\pgfqpoint{0.448634in}{0.402556in}}{\pgfqpoint{4.350661in}{2.489204in}} %
\pgfusepath{clip}%
\pgfsetrectcap%
\pgfsetroundjoin%
\pgfsetlinewidth{1.003750pt}%
\definecolor{currentstroke}{rgb}{0.890196,0.466667,0.760784}%
\pgfsetstrokecolor{currentstroke}%
\pgfsetdash{}{0pt}%
\pgfpathmoveto{\pgfqpoint{3.431449in}{0.402556in}}%
\pgfpathlineto{\pgfqpoint{0.449071in}{0.402556in}}%
\pgfpathlineto{\pgfqpoint{0.449071in}{0.402556in}}%
\pgfusepath{stroke}%
\end{pgfscope}%
\begin{pgfscope}%
\pgfpathrectangle{\pgfqpoint{0.448634in}{0.402556in}}{\pgfqpoint{4.350661in}{2.489204in}} %
\pgfusepath{clip}%
\pgfsetrectcap%
\pgfsetroundjoin%
\pgfsetlinewidth{1.003750pt}%
\definecolor{currentstroke}{rgb}{0.890196,0.466667,0.760784}%
\pgfsetstrokecolor{currentstroke}%
\pgfsetdash{}{0pt}%
\pgfpathmoveto{\pgfqpoint{4.798111in}{2.852854in}}%
\pgfpathlineto{\pgfqpoint{4.796598in}{2.885133in}}%
\pgfpathlineto{\pgfqpoint{4.794535in}{2.889334in}}%
\pgfpathlineto{\pgfqpoint{4.788238in}{2.891068in}}%
\pgfpathlineto{\pgfqpoint{4.768671in}{2.891655in}}%
\pgfpathlineto{\pgfqpoint{4.568541in}{2.891758in}}%
\pgfpathlineto{\pgfqpoint{0.468053in}{2.890889in}}%
\pgfpathlineto{\pgfqpoint{0.459461in}{2.889446in}}%
\pgfpathlineto{\pgfqpoint{0.455705in}{2.887048in}}%
\pgfpathlineto{\pgfqpoint{0.453539in}{2.882768in}}%
\pgfpathlineto{\pgfqpoint{0.451896in}{2.873010in}}%
\pgfpathlineto{\pgfqpoint{0.450631in}{2.845674in}}%
\pgfpathlineto{\pgfqpoint{0.449761in}{2.758559in}}%
\pgfpathlineto{\pgfqpoint{0.449221in}{2.387668in}}%
\pgfpathlineto{\pgfqpoint{0.449752in}{1.073369in}}%
\pgfpathlineto{\pgfqpoint{0.451580in}{0.739824in}}%
\pgfpathlineto{\pgfqpoint{0.454212in}{0.622874in}}%
\pgfpathlineto{\pgfqpoint{0.457438in}{0.565747in}}%
\pgfpathlineto{\pgfqpoint{0.461306in}{0.531189in}}%
\pgfpathlineto{\pgfqpoint{0.466101in}{0.506923in}}%
\pgfpathlineto{\pgfqpoint{0.471336in}{0.490576in}}%
\pgfpathlineto{\pgfqpoint{0.477690in}{0.477552in}}%
\pgfpathlineto{\pgfqpoint{0.484559in}{0.467921in}}%
\pgfpathlineto{\pgfqpoint{0.492765in}{0.459775in}}%
\pgfpathlineto{\pgfqpoint{0.503908in}{0.452031in}}%
\pgfpathlineto{\pgfqpoint{0.517948in}{0.445318in}}%
\pgfpathlineto{\pgfqpoint{0.536773in}{0.439202in}}%
\pgfpathlineto{\pgfqpoint{0.560271in}{0.434062in}}%
\pgfpathlineto{\pgfqpoint{0.592637in}{0.429359in}}%
\pgfpathlineto{\pgfqpoint{0.638165in}{0.425118in}}%
\pgfpathlineto{\pgfqpoint{0.705516in}{0.421289in}}%
\pgfpathlineto{\pgfqpoint{0.805539in}{0.418032in}}%
\pgfpathlineto{\pgfqpoint{0.959970in}{0.415427in}}%
\pgfpathlineto{\pgfqpoint{1.199252in}{0.413794in}}%
\pgfpathlineto{\pgfqpoint{1.521200in}{0.413878in}}%
\pgfpathlineto{\pgfqpoint{1.821389in}{0.416050in}}%
\pgfpathlineto{\pgfqpoint{2.025846in}{0.419585in}}%
\pgfpathlineto{\pgfqpoint{2.167185in}{0.424126in}}%
\pgfpathlineto{\pgfqpoint{2.267139in}{0.429468in}}%
\pgfpathlineto{\pgfqpoint{2.340905in}{0.435574in}}%
\pgfpathlineto{\pgfqpoint{2.394984in}{0.442122in}}%
\pgfpathlineto{\pgfqpoint{2.438016in}{0.449424in}}%
\pgfpathlineto{\pgfqpoint{2.472128in}{0.457304in}}%
\pgfpathlineto{\pgfqpoint{2.501524in}{0.466384in}}%
\pgfpathlineto{\pgfqpoint{2.524083in}{0.475492in}}%
\pgfpathlineto{\pgfqpoint{2.543905in}{0.485725in}}%
\pgfpathlineto{\pgfqpoint{2.560892in}{0.496845in}}%
\pgfpathlineto{\pgfqpoint{2.576741in}{0.509977in}}%
\pgfpathlineto{\pgfqpoint{2.589612in}{0.523365in}}%
\pgfpathlineto{\pgfqpoint{2.602513in}{0.540198in}}%
\pgfpathlineto{\pgfqpoint{2.613679in}{0.558588in}}%
\pgfpathlineto{\pgfqpoint{2.624173in}{0.580380in}}%
\pgfpathlineto{\pgfqpoint{2.633727in}{0.605475in}}%
\pgfpathlineto{\pgfqpoint{2.642872in}{0.636087in}}%
\pgfpathlineto{\pgfqpoint{2.651273in}{0.672161in}}%
\pgfpathlineto{\pgfqpoint{2.659600in}{0.718480in}}%
\pgfpathlineto{\pgfqpoint{2.667389in}{0.775030in}}%
\pgfpathlineto{\pgfqpoint{2.675210in}{0.849164in}}%
\pgfpathlineto{\pgfqpoint{2.683279in}{0.948301in}}%
\pgfpathlineto{\pgfqpoint{2.692928in}{1.097244in}}%
\pgfpathlineto{\pgfqpoint{2.728553in}{1.678280in}}%
\pgfpathlineto{\pgfqpoint{2.742236in}{1.849318in}}%
\pgfpathlineto{\pgfqpoint{2.757765in}{2.015144in}}%
\pgfpathlineto{\pgfqpoint{2.787695in}{2.324382in}}%
\pgfpathlineto{\pgfqpoint{2.790091in}{2.379070in}}%
\pgfpathlineto{\pgfqpoint{2.789774in}{2.421377in}}%
\pgfpathlineto{\pgfqpoint{2.787251in}{2.456097in}}%
\pgfpathlineto{\pgfqpoint{2.783238in}{2.483081in}}%
\pgfpathlineto{\pgfqpoint{2.777584in}{2.507107in}}%
\pgfpathlineto{\pgfqpoint{2.770563in}{2.528008in}}%
\pgfpathlineto{\pgfqpoint{2.761537in}{2.547873in}}%
\pgfpathlineto{\pgfqpoint{2.750494in}{2.566356in}}%
\pgfpathlineto{\pgfqpoint{2.737599in}{2.583195in}}%
\pgfpathlineto{\pgfqpoint{2.723137in}{2.598281in}}%
\pgfpathlineto{\pgfqpoint{2.705629in}{2.613035in}}%
\pgfpathlineto{\pgfqpoint{2.686995in}{2.625864in}}%
\pgfpathlineto{\pgfqpoint{2.665575in}{2.638054in}}%
\pgfpathlineto{\pgfqpoint{2.639405in}{2.650300in}}%
\pgfpathlineto{\pgfqpoint{2.610533in}{2.661372in}}%
\pgfpathlineto{\pgfqpoint{2.576971in}{2.671905in}}%
\pgfpathlineto{\pgfqpoint{2.536621in}{2.682131in}}%
\pgfpathlineto{\pgfqpoint{2.491636in}{2.691208in}}%
\pgfpathlineto{\pgfqpoint{2.439922in}{2.699377in}}%
\pgfpathlineto{\pgfqpoint{2.379343in}{2.706606in}}%
\pgfpathlineto{\pgfqpoint{2.312097in}{2.712338in}}%
\pgfpathlineto{\pgfqpoint{2.238223in}{2.716412in}}%
\pgfpathlineto{\pgfqpoint{2.157761in}{2.718640in}}%
\pgfpathlineto{\pgfqpoint{2.072925in}{2.718788in}}%
\pgfpathlineto{\pgfqpoint{1.985932in}{2.716739in}}%
\pgfpathlineto{\pgfqpoint{1.899002in}{2.712454in}}%
\pgfpathlineto{\pgfqpoint{1.816523in}{2.706183in}}%
\pgfpathlineto{\pgfqpoint{1.738537in}{2.698041in}}%
\pgfpathlineto{\pgfqpoint{1.667247in}{2.688422in}}%
\pgfpathlineto{\pgfqpoint{1.600532in}{2.677196in}}%
\pgfpathlineto{\pgfqpoint{1.540578in}{2.664916in}}%
\pgfpathlineto{\pgfqpoint{1.485272in}{2.651385in}}%
\pgfpathlineto{\pgfqpoint{1.434656in}{2.636764in}}%
\pgfpathlineto{\pgfqpoint{1.388756in}{2.621274in}}%
\pgfpathlineto{\pgfqpoint{1.345538in}{2.604353in}}%
\pgfpathlineto{\pgfqpoint{1.307090in}{2.587009in}}%
\pgfpathlineto{\pgfqpoint{1.271400in}{2.568591in}}%
\pgfpathlineto{\pgfqpoint{1.238512in}{2.549250in}}%
\pgfpathlineto{\pgfqpoint{1.206603in}{2.527875in}}%
\pgfpathlineto{\pgfqpoint{1.177642in}{2.505794in}}%
\pgfpathlineto{\pgfqpoint{1.149904in}{2.481747in}}%
\pgfpathlineto{\pgfqpoint{1.125171in}{2.457402in}}%
\pgfpathlineto{\pgfqpoint{1.101810in}{2.431343in}}%
\pgfpathlineto{\pgfqpoint{1.079942in}{2.403639in}}%
\pgfpathlineto{\pgfqpoint{1.059659in}{2.374399in}}%
\pgfpathlineto{\pgfqpoint{1.041019in}{2.343760in}}%
\pgfpathlineto{\pgfqpoint{1.022974in}{2.309711in}}%
\pgfpathlineto{\pgfqpoint{1.006815in}{2.274444in}}%
\pgfpathlineto{\pgfqpoint{0.991659in}{2.235851in}}%
\pgfpathlineto{\pgfqpoint{0.978497in}{2.196312in}}%
\pgfpathlineto{\pgfqpoint{0.966624in}{2.153621in}}%
\pgfpathlineto{\pgfqpoint{0.956260in}{2.107844in}}%
\pgfpathlineto{\pgfqpoint{0.947606in}{2.059060in}}%
\pgfpathlineto{\pgfqpoint{0.941123in}{2.009837in}}%
\pgfpathlineto{\pgfqpoint{0.936520in}{1.957835in}}%
\pgfpathlineto{\pgfqpoint{0.934071in}{1.905641in}}%
\pgfpathlineto{\pgfqpoint{0.933737in}{1.850885in}}%
\pgfpathlineto{\pgfqpoint{0.935643in}{1.796171in}}%
\pgfpathlineto{\pgfqpoint{0.939782in}{1.741619in}}%
\pgfpathlineto{\pgfqpoint{0.946184in}{1.687354in}}%
\pgfpathlineto{\pgfqpoint{0.954465in}{1.635952in}}%
\pgfpathlineto{\pgfqpoint{0.964946in}{1.585079in}}%
\pgfpathlineto{\pgfqpoint{0.977064in}{1.537270in}}%
\pgfpathlineto{\pgfqpoint{0.990596in}{1.492588in}}%
\pgfpathlineto{\pgfqpoint{1.006182in}{1.448791in}}%
\pgfpathlineto{\pgfqpoint{1.022915in}{1.408289in}}%
\pgfpathlineto{\pgfqpoint{1.041618in}{1.368932in}}%
\pgfpathlineto{\pgfqpoint{1.061123in}{1.332987in}}%
\pgfpathlineto{\pgfqpoint{1.082423in}{1.298402in}}%
\pgfpathlineto{\pgfqpoint{1.105497in}{1.265342in}}%
\pgfpathlineto{\pgfqpoint{1.128788in}{1.235754in}}%
\pgfpathlineto{\pgfqpoint{1.153540in}{1.207761in}}%
\pgfpathlineto{\pgfqpoint{1.179664in}{1.181453in}}%
\pgfpathlineto{\pgfqpoint{1.207059in}{1.156895in}}%
\pgfpathlineto{\pgfqpoint{1.235611in}{1.134127in}}%
\pgfpathlineto{\pgfqpoint{1.267083in}{1.111917in}}%
\pgfpathlineto{\pgfqpoint{1.299586in}{1.091745in}}%
\pgfpathlineto{\pgfqpoint{1.332987in}{1.073594in}}%
\pgfpathlineto{\pgfqpoint{1.369193in}{1.056545in}}%
\pgfpathlineto{\pgfqpoint{1.406131in}{1.041696in}}%
\pgfpathlineto{\pgfqpoint{1.443685in}{1.029026in}}%
\pgfpathlineto{\pgfqpoint{1.481747in}{1.018532in}}%
\pgfpathlineto{\pgfqpoint{1.520222in}{1.010238in}}%
\pgfpathlineto{\pgfqpoint{1.554716in}{1.004956in}}%
\pgfpathlineto{\pgfqpoint{1.591569in}{1.001503in}}%
\pgfpathlineto{\pgfqpoint{1.628530in}{1.000325in}}%
\pgfpathlineto{\pgfqpoint{1.663316in}{1.001480in}}%
\pgfpathlineto{\pgfqpoint{1.695813in}{1.004779in}}%
\pgfpathlineto{\pgfqpoint{1.725913in}{1.010046in}}%
\pgfpathlineto{\pgfqpoint{1.753502in}{1.017119in}}%
\pgfpathlineto{\pgfqpoint{1.778458in}{1.025850in}}%
\pgfpathlineto{\pgfqpoint{1.798658in}{1.035068in}}%
\pgfpathlineto{\pgfqpoint{1.816099in}{1.045226in}}%
\pgfpathlineto{\pgfqpoint{1.830644in}{1.056136in}}%
\pgfpathlineto{\pgfqpoint{1.842096in}{1.067593in}}%
\pgfpathlineto{\pgfqpoint{1.848973in}{1.077216in}}%
\pgfpathlineto{\pgfqpoint{1.853074in}{1.085976in}}%
\pgfpathlineto{\pgfqpoint{1.855046in}{1.095631in}}%
\pgfpathlineto{\pgfqpoint{1.854499in}{1.103045in}}%
\pgfpathlineto{\pgfqpoint{1.852060in}{1.109942in}}%
\pgfpathlineto{\pgfqpoint{1.846571in}{1.117621in}}%
\pgfpathlineto{\pgfqpoint{1.839627in}{1.123596in}}%
\pgfpathlineto{\pgfqpoint{1.827966in}{1.130267in}}%
\pgfpathlineto{\pgfqpoint{1.811460in}{1.136537in}}%
\pgfpathlineto{\pgfqpoint{1.790253in}{1.142054in}}%
\pgfpathlineto{\pgfqpoint{1.758035in}{1.147943in}}%
\pgfpathlineto{\pgfqpoint{1.691097in}{1.157280in}}%
\pgfpathlineto{\pgfqpoint{1.622103in}{1.167829in}}%
\pgfpathlineto{\pgfqpoint{1.572800in}{1.177549in}}%
\pgfpathlineto{\pgfqpoint{1.528147in}{1.188566in}}%
\pgfpathlineto{\pgfqpoint{1.486086in}{1.201272in}}%
\pgfpathlineto{\pgfqpoint{1.446695in}{1.215577in}}%
\pgfpathlineto{\pgfqpoint{1.410037in}{1.231309in}}%
\pgfpathlineto{\pgfqpoint{1.376151in}{1.248242in}}%
\pgfpathlineto{\pgfqpoint{1.343134in}{1.267288in}}%
\pgfpathlineto{\pgfqpoint{1.313001in}{1.287210in}}%
\pgfpathlineto{\pgfqpoint{1.283931in}{1.309102in}}%
\pgfpathlineto{\pgfqpoint{1.256074in}{1.332968in}}%
\pgfpathlineto{\pgfqpoint{1.231188in}{1.357109in}}%
\pgfpathlineto{\pgfqpoint{1.207607in}{1.382909in}}%
\pgfpathlineto{\pgfqpoint{1.185428in}{1.410288in}}%
\pgfpathlineto{\pgfqpoint{1.163404in}{1.441118in}}%
\pgfpathlineto{\pgfqpoint{1.143135in}{1.473487in}}%
\pgfpathlineto{\pgfqpoint{1.124661in}{1.507234in}}%
\pgfpathlineto{\pgfqpoint{1.108000in}{1.542195in}}%
\pgfpathlineto{\pgfqpoint{1.092277in}{1.580489in}}%
\pgfpathlineto{\pgfqpoint{1.078564in}{1.619782in}}%
\pgfpathlineto{\pgfqpoint{1.066193in}{1.662286in}}%
\pgfpathlineto{\pgfqpoint{1.055977in}{1.705533in}}%
\pgfpathlineto{\pgfqpoint{1.047862in}{1.749361in}}%
\pgfpathlineto{\pgfqpoint{1.041527in}{1.796091in}}%
\pgfpathlineto{\pgfqpoint{1.037446in}{1.843148in}}%
\pgfpathlineto{\pgfqpoint{1.035603in}{1.890391in}}%
\pgfpathlineto{\pgfqpoint{1.036008in}{1.937677in}}%
\pgfpathlineto{\pgfqpoint{1.038692in}{1.984866in}}%
\pgfpathlineto{\pgfqpoint{1.043391in}{2.029342in}}%
\pgfpathlineto{\pgfqpoint{1.050259in}{2.073447in}}%
\pgfpathlineto{\pgfqpoint{1.058811in}{2.114611in}}%
\pgfpathlineto{\pgfqpoint{1.069450in}{2.155131in}}%
\pgfpathlineto{\pgfqpoint{1.082253in}{2.194823in}}%
\pgfpathlineto{\pgfqpoint{1.096340in}{2.231236in}}%
\pgfpathlineto{\pgfqpoint{1.112440in}{2.266538in}}%
\pgfpathlineto{\pgfqpoint{1.129371in}{2.298449in}}%
\pgfpathlineto{\pgfqpoint{1.148058in}{2.329049in}}%
\pgfpathlineto{\pgfqpoint{1.168449in}{2.358190in}}%
\pgfpathlineto{\pgfqpoint{1.190458in}{2.385747in}}%
\pgfpathlineto{\pgfqpoint{1.213970in}{2.411628in}}%
\pgfpathlineto{\pgfqpoint{1.238848in}{2.435779in}}%
\pgfpathlineto{\pgfqpoint{1.266722in}{2.459619in}}%
\pgfpathlineto{\pgfqpoint{1.295796in}{2.481504in}}%
\pgfpathlineto{\pgfqpoint{1.327801in}{2.502692in}}%
\pgfpathlineto{\pgfqpoint{1.360762in}{2.521869in}}%
\pgfpathlineto{\pgfqpoint{1.396513in}{2.540134in}}%
\pgfpathlineto{\pgfqpoint{1.435012in}{2.557327in}}%
\pgfpathlineto{\pgfqpoint{1.476206in}{2.573331in}}%
\pgfpathlineto{\pgfqpoint{1.522131in}{2.588719in}}%
\pgfpathlineto{\pgfqpoint{1.570667in}{2.602603in}}%
\pgfpathlineto{\pgfqpoint{1.623876in}{2.615449in}}%
\pgfpathlineto{\pgfqpoint{1.679582in}{2.626628in}}%
\pgfpathlineto{\pgfqpoint{1.739878in}{2.636474in}}%
\pgfpathlineto{\pgfqpoint{1.804731in}{2.644780in}}%
\pgfpathlineto{\pgfqpoint{1.871933in}{2.651158in}}%
\pgfpathlineto{\pgfqpoint{1.941435in}{2.655566in}}%
\pgfpathlineto{\pgfqpoint{2.013190in}{2.657887in}}%
\pgfpathlineto{\pgfqpoint{2.082798in}{2.657972in}}%
\pgfpathlineto{\pgfqpoint{2.150207in}{2.655908in}}%
\pgfpathlineto{\pgfqpoint{2.215359in}{2.651681in}}%
\pgfpathlineto{\pgfqpoint{2.273858in}{2.645708in}}%
\pgfpathlineto{\pgfqpoint{2.327818in}{2.637985in}}%
\pgfpathlineto{\pgfqpoint{2.375031in}{2.629055in}}%
\pgfpathlineto{\pgfqpoint{2.417599in}{2.618792in}}%
\pgfpathlineto{\pgfqpoint{2.455460in}{2.607385in}}%
\pgfpathlineto{\pgfqpoint{2.488565in}{2.595108in}}%
\pgfpathlineto{\pgfqpoint{2.516901in}{2.582354in}}%
\pgfpathlineto{\pgfqpoint{2.542451in}{2.568500in}}%
\pgfpathlineto{\pgfqpoint{2.565123in}{2.553711in}}%
\pgfpathlineto{\pgfqpoint{2.584867in}{2.538256in}}%
\pgfpathlineto{\pgfqpoint{2.603321in}{2.520841in}}%
\pgfpathlineto{\pgfqpoint{2.618717in}{2.503269in}}%
\pgfpathlineto{\pgfqpoint{2.632578in}{2.484098in}}%
\pgfpathlineto{\pgfqpoint{2.644759in}{2.463485in}}%
\pgfpathlineto{\pgfqpoint{2.655209in}{2.441663in}}%
\pgfpathlineto{\pgfqpoint{2.664765in}{2.416570in}}%
\pgfpathlineto{\pgfqpoint{2.673075in}{2.388262in}}%
\pgfpathlineto{\pgfqpoint{2.679936in}{2.356877in}}%
\pgfpathlineto{\pgfqpoint{2.685603in}{2.320114in}}%
\pgfpathlineto{\pgfqpoint{2.689784in}{2.278074in}}%
\pgfpathlineto{\pgfqpoint{2.692481in}{2.228390in}}%
\pgfpathlineto{\pgfqpoint{2.693544in}{2.166176in}}%
\pgfpathlineto{\pgfqpoint{2.692555in}{2.086532in}}%
\pgfpathlineto{\pgfqpoint{2.688813in}{1.977093in}}%
\pgfpathlineto{\pgfqpoint{2.680588in}{1.808090in}}%
\pgfpathlineto{\pgfqpoint{2.665047in}{1.537351in}}%
\pgfpathlineto{\pgfqpoint{2.650972in}{1.326384in}}%
\pgfpathlineto{\pgfqpoint{2.639271in}{1.182635in}}%
\pgfpathlineto{\pgfqpoint{2.628246in}{1.073842in}}%
\pgfpathlineto{\pgfqpoint{2.617310in}{0.987627in}}%
\pgfpathlineto{\pgfqpoint{2.606462in}{0.919048in}}%
\pgfpathlineto{\pgfqpoint{2.595032in}{0.860762in}}%
\pgfpathlineto{\pgfqpoint{2.583448in}{0.812781in}}%
\pgfpathlineto{\pgfqpoint{2.570910in}{0.770341in}}%
\pgfpathlineto{\pgfqpoint{2.558533in}{0.735801in}}%
\pgfpathlineto{\pgfqpoint{2.545067in}{0.704552in}}%
\pgfpathlineto{\pgfqpoint{2.530654in}{0.676721in}}%
\pgfpathlineto{\pgfqpoint{2.515558in}{0.652362in}}%
\pgfpathlineto{\pgfqpoint{2.500150in}{0.631423in}}%
\pgfpathlineto{\pgfqpoint{2.483268in}{0.612030in}}%
\pgfpathlineto{\pgfqpoint{2.465016in}{0.594335in}}%
\pgfpathlineto{\pgfqpoint{2.445564in}{0.578402in}}%
\pgfpathlineto{\pgfqpoint{2.423206in}{0.562999in}}%
\pgfpathlineto{\pgfqpoint{2.397937in}{0.548488in}}%
\pgfpathlineto{\pgfqpoint{2.369822in}{0.535111in}}%
\pgfpathlineto{\pgfqpoint{2.338965in}{0.522986in}}%
\pgfpathlineto{\pgfqpoint{2.303373in}{0.511516in}}%
\pgfpathlineto{\pgfqpoint{2.263086in}{0.500970in}}%
\pgfpathlineto{\pgfqpoint{2.216020in}{0.491077in}}%
\pgfpathlineto{\pgfqpoint{2.162206in}{0.482112in}}%
\pgfpathlineto{\pgfqpoint{2.099529in}{0.473948in}}%
\pgfpathlineto{\pgfqpoint{2.023680in}{0.466405in}}%
\pgfpathlineto{\pgfqpoint{1.934678in}{0.459841in}}%
\pgfpathlineto{\pgfqpoint{1.826026in}{0.454147in}}%
\pgfpathlineto{\pgfqpoint{1.695567in}{0.449643in}}%
\pgfpathlineto{\pgfqpoint{1.543317in}{0.446689in}}%
\pgfpathlineto{\pgfqpoint{1.375820in}{0.445692in}}%
\pgfpathlineto{\pgfqpoint{1.212674in}{0.446900in}}%
\pgfpathlineto{\pgfqpoint{1.071306in}{0.450062in}}%
\pgfpathlineto{\pgfqpoint{0.956088in}{0.454770in}}%
\pgfpathlineto{\pgfqpoint{0.864872in}{0.460668in}}%
\pgfpathlineto{\pgfqpoint{0.795502in}{0.467249in}}%
\pgfpathlineto{\pgfqpoint{0.739326in}{0.474733in}}%
\pgfpathlineto{\pgfqpoint{0.694218in}{0.482962in}}%
\pgfpathlineto{\pgfqpoint{0.658056in}{0.491789in}}%
\pgfpathlineto{\pgfqpoint{0.628734in}{0.501178in}}%
\pgfpathlineto{\pgfqpoint{0.604197in}{0.511347in}}%
\pgfpathlineto{\pgfqpoint{0.584456in}{0.521785in}}%
\pgfpathlineto{\pgfqpoint{0.567543in}{0.533049in}}%
\pgfpathlineto{\pgfqpoint{0.551789in}{0.546329in}}%
\pgfpathlineto{\pgfqpoint{0.539059in}{0.559893in}}%
\pgfpathlineto{\pgfqpoint{0.527750in}{0.575014in}}%
\pgfpathlineto{\pgfqpoint{0.516833in}{0.593595in}}%
\pgfpathlineto{\pgfqpoint{0.507753in}{0.613432in}}%
\pgfpathlineto{\pgfqpoint{0.499546in}{0.636476in}}%
\pgfpathlineto{\pgfqpoint{0.491800in}{0.664992in}}%
\pgfpathlineto{\pgfqpoint{0.484924in}{0.698935in}}%
\pgfpathlineto{\pgfqpoint{0.478751in}{0.740652in}}%
\pgfpathlineto{\pgfqpoint{0.473067in}{0.795023in}}%
\pgfpathlineto{\pgfqpoint{0.468149in}{0.864490in}}%
\pgfpathlineto{\pgfqpoint{0.463829in}{0.958948in}}%
\pgfpathlineto{\pgfqpoint{0.460186in}{1.090808in}}%
\pgfpathlineto{\pgfqpoint{0.457269in}{1.282447in}}%
\pgfpathlineto{\pgfqpoint{0.456462in}{1.367075in}}%
\pgfpathlineto{\pgfqpoint{0.456462in}{1.367075in}}%
\pgfusepath{stroke}%
\end{pgfscope}%
\begin{pgfscope}%
\pgfpathrectangle{\pgfqpoint{0.448634in}{0.402556in}}{\pgfqpoint{4.350661in}{2.489204in}} %
\pgfusepath{clip}%
\pgfsetrectcap%
\pgfsetroundjoin%
\pgfsetlinewidth{1.003750pt}%
\definecolor{currentstroke}{rgb}{0.890196,0.466667,0.760784}%
\pgfsetstrokecolor{currentstroke}%
\pgfsetdash{}{0pt}%
\pgfpathmoveto{\pgfqpoint{0.446327in}{0.403012in}}%
\pgfpathlineto{\pgfqpoint{0.452720in}{0.404197in}}%
\pgfpathlineto{\pgfqpoint{0.465744in}{0.403360in}}%
\pgfpathlineto{\pgfqpoint{0.539703in}{0.402861in}}%
\pgfpathlineto{\pgfqpoint{1.131393in}{0.402649in}}%
\pgfpathlineto{\pgfqpoint{2.601914in}{0.403675in}}%
\pgfpathlineto{\pgfqpoint{2.662794in}{0.405588in}}%
\pgfpathlineto{\pgfqpoint{2.682254in}{0.407952in}}%
\pgfpathlineto{\pgfqpoint{2.692780in}{0.411016in}}%
\pgfpathlineto{\pgfqpoint{2.700383in}{0.415776in}}%
\pgfpathlineto{\pgfqpoint{2.704829in}{0.421207in}}%
\pgfpathlineto{\pgfqpoint{2.708731in}{0.430075in}}%
\pgfpathlineto{\pgfqpoint{2.711958in}{0.444528in}}%
\pgfpathlineto{\pgfqpoint{2.714622in}{0.469224in}}%
\pgfpathlineto{\pgfqpoint{2.716971in}{0.516439in}}%
\pgfpathlineto{\pgfqpoint{2.719579in}{0.625922in}}%
\pgfpathlineto{\pgfqpoint{2.727354in}{0.976784in}}%
\pgfpathlineto{\pgfqpoint{2.734264in}{1.178253in}}%
\pgfpathlineto{\pgfqpoint{2.742503in}{1.354733in}}%
\pgfpathlineto{\pgfqpoint{2.751982in}{1.511176in}}%
\pgfpathlineto{\pgfqpoint{2.762759in}{1.652521in}}%
\pgfpathlineto{\pgfqpoint{2.774756in}{1.781226in}}%
\pgfpathlineto{\pgfqpoint{2.787615in}{1.894778in}}%
\pgfpathlineto{\pgfqpoint{2.793802in}{1.938964in}}%
\pgfpathlineto{\pgfqpoint{2.795733in}{1.946039in}}%
\pgfpathlineto{\pgfqpoint{2.810459in}{2.046693in}}%
\pgfpathlineto{\pgfqpoint{2.826629in}{2.141989in}}%
\pgfpathlineto{\pgfqpoint{2.845434in}{2.239201in}}%
\pgfpathlineto{\pgfqpoint{2.871363in}{2.360071in}}%
\pgfpathlineto{\pgfqpoint{2.876932in}{2.378809in}}%
\pgfpathlineto{\pgfqpoint{2.892308in}{2.453936in}}%
\pgfpathlineto{\pgfqpoint{2.899202in}{2.498034in}}%
\pgfpathlineto{\pgfqpoint{2.902049in}{2.530219in}}%
\pgfpathlineto{\pgfqpoint{2.902198in}{2.555100in}}%
\pgfpathlineto{\pgfqpoint{2.900122in}{2.577362in}}%
\pgfpathlineto{\pgfqpoint{2.895997in}{2.596691in}}%
\pgfpathlineto{\pgfqpoint{2.890347in}{2.612856in}}%
\pgfpathlineto{\pgfqpoint{2.882745in}{2.627937in}}%
\pgfpathlineto{\pgfqpoint{2.873367in}{2.641647in}}%
\pgfpathlineto{\pgfqpoint{2.862528in}{2.653871in}}%
\pgfpathlineto{\pgfqpoint{2.848797in}{2.666088in}}%
\pgfpathlineto{\pgfqpoint{2.832095in}{2.677761in}}%
\pgfpathlineto{\pgfqpoint{2.812618in}{2.688830in}}%
\pgfpathlineto{\pgfqpoint{2.788386in}{2.699917in}}%
\pgfpathlineto{\pgfqpoint{2.759390in}{2.710552in}}%
\pgfpathlineto{\pgfqpoint{2.725686in}{2.720474in}}%
\pgfpathlineto{\pgfqpoint{2.685203in}{2.729984in}}%
\pgfpathlineto{\pgfqpoint{2.637966in}{2.738756in}}%
\pgfpathlineto{\pgfqpoint{2.581856in}{2.746864in}}%
\pgfpathlineto{\pgfqpoint{2.514727in}{2.754181in}}%
\pgfpathlineto{\pgfqpoint{2.438772in}{2.760174in}}%
\pgfpathlineto{\pgfqpoint{2.351854in}{2.764782in}}%
\pgfpathlineto{\pgfqpoint{2.253999in}{2.767693in}}%
\pgfpathlineto{\pgfqpoint{2.147412in}{2.768566in}}%
\pgfpathlineto{\pgfqpoint{2.038653in}{2.767206in}}%
\pgfpathlineto{\pgfqpoint{1.929934in}{2.763600in}}%
\pgfpathlineto{\pgfqpoint{1.827814in}{2.757969in}}%
\pgfpathlineto{\pgfqpoint{1.732326in}{2.750458in}}%
\pgfpathlineto{\pgfqpoint{1.632675in}{2.740203in}}%
\pgfpathlineto{\pgfqpoint{1.557089in}{2.729767in}}%
\pgfpathlineto{\pgfqpoint{1.471014in}{2.715461in}}%
\pgfpathlineto{\pgfqpoint{1.411162in}{2.702550in}}%
\pgfpathlineto{\pgfqpoint{1.355982in}{2.688364in}}%
\pgfpathlineto{\pgfqpoint{1.322323in}{2.678264in}}%
\pgfpathlineto{\pgfqpoint{1.314675in}{2.673721in}}%
\pgfpathlineto{\pgfqpoint{1.268906in}{2.657732in}}%
\pgfpathlineto{\pgfqpoint{1.227900in}{2.641110in}}%
\pgfpathlineto{\pgfqpoint{1.189643in}{2.623225in}}%
\pgfpathlineto{\pgfqpoint{1.154202in}{2.604188in}}%
\pgfpathlineto{\pgfqpoint{1.121630in}{2.584161in}}%
\pgfpathlineto{\pgfqpoint{1.075067in}{2.552318in}}%
\pgfpathlineto{\pgfqpoint{1.048873in}{2.530062in}}%
\pgfpathlineto{\pgfqpoint{1.023893in}{2.506049in}}%
\pgfpathlineto{\pgfqpoint{1.000258in}{2.480315in}}%
\pgfpathlineto{\pgfqpoint{0.978071in}{2.452946in}}%
\pgfpathlineto{\pgfqpoint{0.957404in}{2.424059in}}%
\pgfpathlineto{\pgfqpoint{0.937074in}{2.391740in}}%
\pgfpathlineto{\pgfqpoint{0.919798in}{2.360073in}}%
\pgfpathlineto{\pgfqpoint{0.901844in}{2.323086in}}%
\pgfpathlineto{\pgfqpoint{0.884849in}{2.282727in}}%
\pgfpathlineto{\pgfqpoint{0.869007in}{2.239051in}}%
\pgfpathlineto{\pgfqpoint{0.854457in}{2.192139in}}%
\pgfpathlineto{\pgfqpoint{0.845392in}{2.156302in}}%
\pgfpathlineto{\pgfqpoint{0.833177in}{2.103359in}}%
\pgfpathlineto{\pgfqpoint{0.822027in}{2.045001in}}%
\pgfpathlineto{\pgfqpoint{0.813889in}{1.991046in}}%
\pgfpathlineto{\pgfqpoint{0.805954in}{1.926969in}}%
\pgfpathlineto{\pgfqpoint{0.799187in}{1.855203in}}%
\pgfpathlineto{\pgfqpoint{0.793921in}{1.775779in}}%
\pgfpathlineto{\pgfqpoint{0.790123in}{1.683784in}}%
\pgfpathlineto{\pgfqpoint{0.788908in}{1.639000in}}%
\pgfpathlineto{\pgfqpoint{0.788908in}{1.639000in}}%
\pgfusepath{stroke}%
\end{pgfscope}%
\begin{pgfscope}%
\pgfpathrectangle{\pgfqpoint{0.448634in}{0.402556in}}{\pgfqpoint{4.350661in}{2.489204in}} %
\pgfusepath{clip}%
\pgfsetrectcap%
\pgfsetroundjoin%
\pgfsetlinewidth{1.003750pt}%
\definecolor{currentstroke}{rgb}{0.890196,0.466667,0.760784}%
\pgfsetstrokecolor{currentstroke}%
\pgfsetdash{}{0pt}%
\pgfpathmoveto{\pgfqpoint{2.581601in}{2.736919in}}%
\pgfpathlineto{\pgfqpoint{2.635440in}{2.728155in}}%
\pgfpathlineto{\pgfqpoint{2.682505in}{2.718257in}}%
\pgfpathlineto{\pgfqpoint{2.720629in}{2.708061in}}%
\pgfpathlineto{\pgfqpoint{2.751942in}{2.697582in}}%
\pgfpathlineto{\pgfqpoint{2.778515in}{2.686530in}}%
\pgfpathlineto{\pgfqpoint{2.800327in}{2.675289in}}%
\pgfpathlineto{\pgfqpoint{2.819300in}{2.663132in}}%
\pgfpathlineto{\pgfqpoint{2.835298in}{2.650237in}}%
\pgfpathlineto{\pgfqpoint{2.848256in}{2.636963in}}%
\pgfpathlineto{\pgfqpoint{2.858265in}{2.623846in}}%
\pgfpathlineto{\pgfqpoint{2.866725in}{2.609374in}}%
\pgfpathlineto{\pgfqpoint{2.873427in}{2.593742in}}%
\pgfpathlineto{\pgfqpoint{2.878877in}{2.574847in}}%
\pgfpathlineto{\pgfqpoint{2.882159in}{2.555303in}}%
\pgfpathlineto{\pgfqpoint{2.883703in}{2.532980in}}%
\pgfpathlineto{\pgfqpoint{2.883258in}{2.505612in}}%
\pgfpathlineto{\pgfqpoint{2.880469in}{2.473417in}}%
\pgfpathlineto{\pgfqpoint{2.874569in}{2.431648in}}%
\pgfpathlineto{\pgfqpoint{2.863123in}{2.368271in}}%
\pgfpathlineto{\pgfqpoint{2.817680in}{2.127400in}}%
\pgfpathlineto{\pgfqpoint{2.801856in}{2.026965in}}%
\pgfpathlineto{\pgfqpoint{2.787495in}{1.921201in}}%
\pgfpathlineto{\pgfqpoint{2.774679in}{1.810153in}}%
\pgfpathlineto{\pgfqpoint{2.762942in}{1.688926in}}%
\pgfpathlineto{\pgfqpoint{2.752247in}{1.555069in}}%
\pgfpathlineto{\pgfqpoint{2.742609in}{1.406126in}}%
\pgfpathlineto{\pgfqpoint{2.734081in}{1.239637in}}%
\pgfpathlineto{\pgfqpoint{2.726531in}{1.048164in}}%
\pgfpathlineto{\pgfqpoint{2.719223in}{0.801875in}}%
\pgfpathlineto{\pgfqpoint{2.711905in}{0.568045in}}%
\pgfpathlineto{\pgfqpoint{2.707995in}{0.508478in}}%
\pgfpathlineto{\pgfqpoint{2.703657in}{0.473997in}}%
\pgfpathlineto{\pgfqpoint{2.699314in}{0.454727in}}%
\pgfpathlineto{\pgfqpoint{2.694130in}{0.441044in}}%
\pgfpathlineto{\pgfqpoint{2.687758in}{0.430995in}}%
\pgfpathlineto{\pgfqpoint{2.681152in}{0.424544in}}%
\pgfpathlineto{\pgfqpoint{2.671549in}{0.418753in}}%
\pgfpathlineto{\pgfqpoint{2.659089in}{0.414356in}}%
\pgfpathlineto{\pgfqpoint{2.641952in}{0.410935in}}%
\pgfpathlineto{\pgfqpoint{2.615962in}{0.408202in}}%
\pgfpathlineto{\pgfqpoint{2.570324in}{0.405995in}}%
\pgfpathlineto{\pgfqpoint{2.483323in}{0.404419in}}%
\pgfpathlineto{\pgfqpoint{2.278844in}{0.403453in}}%
\pgfpathlineto{\pgfqpoint{1.543582in}{0.402942in}}%
\pgfpathlineto{\pgfqpoint{0.532055in}{0.403929in}}%
\pgfpathlineto{\pgfqpoint{0.468995in}{0.405682in}}%
\pgfpathlineto{\pgfqpoint{0.456099in}{0.407773in}}%
\pgfpathlineto{\pgfqpoint{0.452293in}{0.410111in}}%
\pgfpathlineto{\pgfqpoint{0.450308in}{0.414474in}}%
\pgfpathlineto{\pgfqpoint{0.449244in}{0.424338in}}%
\pgfpathlineto{\pgfqpoint{0.448758in}{0.466648in}}%
\pgfpathlineto{\pgfqpoint{0.448639in}{0.904748in}}%
\pgfpathlineto{\pgfqpoint{0.448653in}{2.891133in}}%
\pgfpathlineto{\pgfqpoint{0.448653in}{2.891133in}}%
\pgfusepath{stroke}%
\end{pgfscope}%
\begin{pgfscope}%
\pgfpathrectangle{\pgfqpoint{0.448634in}{0.402556in}}{\pgfqpoint{4.350661in}{2.489204in}} %
\pgfusepath{clip}%
\pgfsetrectcap%
\pgfsetroundjoin%
\pgfsetlinewidth{1.003750pt}%
\definecolor{currentstroke}{rgb}{0.890196,0.466667,0.760784}%
\pgfsetstrokecolor{currentstroke}%
\pgfsetdash{}{0pt}%
\pgfpathmoveto{\pgfqpoint{2.028735in}{0.425781in}}%
\pgfpathlineto{\pgfqpoint{1.878677in}{0.421902in}}%
\pgfpathlineto{\pgfqpoint{1.676387in}{0.419019in}}%
\pgfpathlineto{\pgfqpoint{1.413176in}{0.417567in}}%
\pgfpathlineto{\pgfqpoint{1.134735in}{0.418186in}}%
\pgfpathlineto{\pgfqpoint{0.921565in}{0.420721in}}%
\pgfpathlineto{\pgfqpoint{0.782383in}{0.424442in}}%
\pgfpathlineto{\pgfqpoint{0.693280in}{0.428856in}}%
\pgfpathlineto{\pgfqpoint{0.632536in}{0.433929in}}%
\pgfpathlineto{\pgfqpoint{0.591482in}{0.439357in}}%
\pgfpathlineto{\pgfqpoint{0.561485in}{0.445340in}}%
\pgfpathlineto{\pgfqpoint{0.538319in}{0.452159in}}%
\pgfpathlineto{\pgfqpoint{0.521997in}{0.459039in}}%
\pgfpathlineto{\pgfqpoint{0.508474in}{0.467019in}}%
\pgfpathlineto{\pgfqpoint{0.497884in}{0.475723in}}%
\pgfpathlineto{\pgfqpoint{0.488678in}{0.486285in}}%
\pgfpathlineto{\pgfqpoint{0.481155in}{0.498470in}}%
\pgfpathlineto{\pgfqpoint{0.474454in}{0.514097in}}%
\pgfpathlineto{\pgfqpoint{0.468973in}{0.532985in}}%
\pgfpathlineto{\pgfqpoint{0.464317in}{0.557292in}}%
\pgfpathlineto{\pgfqpoint{0.460192in}{0.591813in}}%
\pgfpathlineto{\pgfqpoint{0.456772in}{0.641438in}}%
\pgfpathlineto{\pgfqpoint{0.454061in}{0.716047in}}%
\pgfpathlineto{\pgfqpoint{0.451938in}{0.842972in}}%
\pgfpathlineto{\pgfqpoint{0.450436in}{1.086907in}}%
\pgfpathlineto{\pgfqpoint{0.449580in}{1.661912in}}%
\pgfpathlineto{\pgfqpoint{0.450152in}{2.694931in}}%
\pgfpathlineto{\pgfqpoint{0.451795in}{2.841777in}}%
\pgfpathlineto{\pgfqpoint{0.453818in}{2.871542in}}%
\pgfpathlineto{\pgfqpoint{0.456043in}{2.881140in}}%
\pgfpathlineto{\pgfqpoint{0.458446in}{2.885253in}}%
\pgfpathlineto{\pgfqpoint{0.462020in}{2.888034in}}%
\pgfpathlineto{\pgfqpoint{0.468328in}{2.889865in}}%
\pgfpathlineto{\pgfqpoint{0.483511in}{2.891081in}}%
\pgfpathlineto{\pgfqpoint{0.533540in}{2.891636in}}%
\pgfpathlineto{\pgfqpoint{0.972956in}{2.891757in}}%
\pgfpathlineto{\pgfqpoint{4.784121in}{2.890876in}}%
\pgfpathlineto{\pgfqpoint{4.790501in}{2.889428in}}%
\pgfpathlineto{\pgfqpoint{4.793725in}{2.886261in}}%
\pgfpathlineto{\pgfqpoint{4.795566in}{2.879136in}}%
\pgfpathlineto{\pgfqpoint{4.796882in}{2.859289in}}%
\pgfpathlineto{\pgfqpoint{4.796962in}{2.856802in}}%
\pgfpathlineto{\pgfqpoint{4.796962in}{2.856802in}}%
\pgfusepath{stroke}%
\end{pgfscope}%
\begin{pgfscope}%
\pgfpathrectangle{\pgfqpoint{0.448634in}{0.402556in}}{\pgfqpoint{4.350661in}{2.489204in}} %
\pgfusepath{clip}%
\pgfsetrectcap%
\pgfsetroundjoin%
\pgfsetlinewidth{1.003750pt}%
\definecolor{currentstroke}{rgb}{0.498039,0.498039,0.498039}%
\pgfsetstrokecolor{currentstroke}%
\pgfsetdash{}{0pt}%
\pgfpathmoveto{\pgfqpoint{0.448634in}{2.896245in}}%
\pgfpathlineto{\pgfqpoint{0.448593in}{0.407043in}}%
\pgfpathlineto{\pgfqpoint{0.448593in}{0.407043in}}%
\pgfusepath{stroke}%
\end{pgfscope}%
\begin{pgfscope}%
\pgfpathrectangle{\pgfqpoint{0.448634in}{0.402556in}}{\pgfqpoint{4.350661in}{2.489204in}} %
\pgfusepath{clip}%
\pgfsetrectcap%
\pgfsetroundjoin%
\pgfsetlinewidth{1.003750pt}%
\definecolor{currentstroke}{rgb}{0.498039,0.498039,0.498039}%
\pgfsetstrokecolor{currentstroke}%
\pgfsetdash{}{0pt}%
\pgfpathmoveto{\pgfqpoint{4.697500in}{0.978093in}}%
\pgfpathlineto{\pgfqpoint{4.707790in}{1.026460in}}%
\pgfpathlineto{\pgfqpoint{4.716978in}{1.080199in}}%
\pgfpathlineto{\pgfqpoint{4.725746in}{1.144132in}}%
\pgfpathlineto{\pgfqpoint{4.733906in}{1.220727in}}%
\pgfpathlineto{\pgfqpoint{4.741137in}{1.309953in}}%
\pgfpathlineto{\pgfqpoint{4.746452in}{1.396861in}}%
\pgfpathlineto{\pgfqpoint{4.752049in}{1.521154in}}%
\pgfpathlineto{\pgfqpoint{4.756301in}{1.665444in}}%
\pgfpathlineto{\pgfqpoint{4.759218in}{1.837165in}}%
\pgfpathlineto{\pgfqpoint{4.760308in}{2.028829in}}%
\pgfpathlineto{\pgfqpoint{4.759162in}{2.213024in}}%
\pgfpathlineto{\pgfqpoint{4.756031in}{2.369800in}}%
\pgfpathlineto{\pgfqpoint{4.751513in}{2.486676in}}%
\pgfpathlineto{\pgfqpoint{4.746052in}{2.573570in}}%
\pgfpathlineto{\pgfqpoint{4.739493in}{2.642853in}}%
\pgfpathlineto{\pgfqpoint{4.732508in}{2.691985in}}%
\pgfpathlineto{\pgfqpoint{4.725232in}{2.728375in}}%
\pgfpathlineto{\pgfqpoint{4.717438in}{2.756873in}}%
\pgfpathlineto{\pgfqpoint{4.708988in}{2.779798in}}%
\pgfpathlineto{\pgfqpoint{4.699335in}{2.799271in}}%
\pgfpathlineto{\pgfqpoint{4.688788in}{2.815090in}}%
\pgfpathlineto{\pgfqpoint{4.677968in}{2.827331in}}%
\pgfpathlineto{\pgfqpoint{4.665803in}{2.837790in}}%
\pgfpathlineto{\pgfqpoint{4.652583in}{2.846417in}}%
\pgfpathlineto{\pgfqpoint{4.636602in}{2.854277in}}%
\pgfpathlineto{\pgfqpoint{4.617923in}{2.860953in}}%
\pgfpathlineto{\pgfqpoint{4.594602in}{2.867056in}}%
\pgfpathlineto{\pgfqpoint{4.564527in}{2.872503in}}%
\pgfpathlineto{\pgfqpoint{4.525586in}{2.877159in}}%
\pgfpathlineto{\pgfqpoint{4.473493in}{2.881069in}}%
\pgfpathlineto{\pgfqpoint{4.397412in}{2.884347in}}%
\pgfpathlineto{\pgfqpoint{4.282143in}{2.886955in}}%
\pgfpathlineto{\pgfqpoint{4.090721in}{2.888867in}}%
\pgfpathlineto{\pgfqpoint{3.681761in}{2.890253in}}%
\pgfpathlineto{\pgfqpoint{2.720266in}{2.890804in}}%
\pgfpathlineto{\pgfqpoint{1.608672in}{2.889609in}}%
\pgfpathlineto{\pgfqpoint{1.243223in}{2.887291in}}%
\pgfpathlineto{\pgfqpoint{1.058343in}{2.884070in}}%
\pgfpathlineto{\pgfqpoint{0.949637in}{2.880003in}}%
\pgfpathlineto{\pgfqpoint{0.877975in}{2.875239in}}%
\pgfpathlineto{\pgfqpoint{0.825992in}{2.869738in}}%
\pgfpathlineto{\pgfqpoint{0.785045in}{2.863340in}}%
\pgfpathlineto{\pgfqpoint{0.753033in}{2.856141in}}%
\pgfpathlineto{\pgfqpoint{0.725745in}{2.847680in}}%
\pgfpathlineto{\pgfqpoint{0.703266in}{2.838319in}}%
\pgfpathlineto{\pgfqpoint{0.687452in}{2.830032in}}%
\pgfpathlineto{\pgfqpoint{0.670572in}{2.818705in}}%
\pgfpathlineto{\pgfqpoint{0.654898in}{2.805299in}}%
\pgfpathlineto{\pgfqpoint{0.642274in}{2.791607in}}%
\pgfpathlineto{\pgfqpoint{0.629682in}{2.774468in}}%
\pgfpathlineto{\pgfqpoint{0.618813in}{2.755847in}}%
\pgfpathlineto{\pgfqpoint{0.608545in}{2.733913in}}%
\pgfpathlineto{\pgfqpoint{0.600078in}{2.710999in}}%
\pgfpathlineto{\pgfqpoint{0.589091in}{2.673220in}}%
\pgfpathlineto{\pgfqpoint{0.581733in}{2.639410in}}%
\pgfpathlineto{\pgfqpoint{0.574130in}{2.595462in}}%
\pgfpathlineto{\pgfqpoint{0.567449in}{2.543755in}}%
\pgfpathlineto{\pgfqpoint{0.561905in}{2.484358in}}%
\pgfpathlineto{\pgfqpoint{0.560131in}{2.457057in}}%
\pgfpathlineto{\pgfqpoint{0.555833in}{2.377557in}}%
\pgfpathlineto{\pgfqpoint{0.553251in}{2.287999in}}%
\pgfpathlineto{\pgfqpoint{0.552440in}{2.183459in}}%
\pgfpathlineto{\pgfqpoint{0.553855in}{2.073948in}}%
\pgfpathlineto{\pgfqpoint{0.557515in}{1.966997in}}%
\pgfpathlineto{\pgfqpoint{0.561446in}{1.892458in}}%
\pgfpathlineto{\pgfqpoint{0.568557in}{1.795726in}}%
\pgfpathlineto{\pgfqpoint{0.576861in}{1.714138in}}%
\pgfpathlineto{\pgfqpoint{0.586293in}{1.642767in}}%
\pgfpathlineto{\pgfqpoint{0.596984in}{1.579220in}}%
\pgfpathlineto{\pgfqpoint{0.608219in}{1.525994in}}%
\pgfpathlineto{\pgfqpoint{0.620554in}{1.478266in}}%
\pgfpathlineto{\pgfqpoint{0.633316in}{1.438556in}}%
\pgfpathlineto{\pgfqpoint{0.645957in}{1.406860in}}%
\pgfpathlineto{\pgfqpoint{0.658748in}{1.380834in}}%
\pgfpathlineto{\pgfqpoint{0.671237in}{1.360467in}}%
\pgfpathlineto{\pgfqpoint{0.682810in}{1.345613in}}%
\pgfpathlineto{\pgfqpoint{0.694427in}{1.334376in}}%
\pgfpathlineto{\pgfqpoint{0.703760in}{1.328013in}}%
\pgfpathlineto{\pgfqpoint{0.713996in}{1.323891in}}%
\pgfpathlineto{\pgfqpoint{0.722636in}{1.322956in}}%
\pgfpathlineto{\pgfqpoint{0.731174in}{1.324690in}}%
\pgfpathlineto{\pgfqpoint{0.738961in}{1.329067in}}%
\pgfpathlineto{\pgfqpoint{0.745601in}{1.335472in}}%
\pgfpathlineto{\pgfqpoint{0.752309in}{1.345245in}}%
\pgfpathlineto{\pgfqpoint{0.758494in}{1.358380in}}%
\pgfpathlineto{\pgfqpoint{0.764529in}{1.377044in}}%
\pgfpathlineto{\pgfqpoint{0.769792in}{1.401187in}}%
\pgfpathlineto{\pgfqpoint{0.774317in}{1.433123in}}%
\pgfpathlineto{\pgfqpoint{0.778167in}{1.477706in}}%
\pgfpathlineto{\pgfqpoint{0.783070in}{1.569555in}}%
\pgfpathlineto{\pgfqpoint{0.789923in}{1.758566in}}%
\pgfpathlineto{\pgfqpoint{0.795606in}{1.845442in}}%
\pgfpathlineto{\pgfqpoint{0.804028in}{1.937017in}}%
\pgfpathlineto{\pgfqpoint{0.813234in}{2.005909in}}%
\pgfpathlineto{\pgfqpoint{0.824001in}{2.069440in}}%
\pgfpathlineto{\pgfqpoint{0.837478in}{2.134847in}}%
\pgfpathlineto{\pgfqpoint{0.851043in}{2.187358in}}%
\pgfpathlineto{\pgfqpoint{0.865404in}{2.234346in}}%
\pgfpathlineto{\pgfqpoint{0.881038in}{2.278121in}}%
\pgfpathlineto{\pgfqpoint{0.897812in}{2.318600in}}%
\pgfpathlineto{\pgfqpoint{0.915541in}{2.355730in}}%
\pgfpathlineto{\pgfqpoint{0.928536in}{2.378653in}}%
\pgfpathlineto{\pgfqpoint{0.937116in}{2.393039in}}%
\pgfpathlineto{\pgfqpoint{0.957529in}{2.425290in}}%
\pgfpathlineto{\pgfqpoint{0.978274in}{2.454102in}}%
\pgfpathlineto{\pgfqpoint{1.000528in}{2.481400in}}%
\pgfpathlineto{\pgfqpoint{1.022708in}{2.505268in}}%
\pgfpathlineto{\pgfqpoint{1.047671in}{2.529304in}}%
\pgfpathlineto{\pgfqpoint{1.073850in}{2.551583in}}%
\pgfpathlineto{\pgfqpoint{1.102954in}{2.573414in}}%
\pgfpathlineto{\pgfqpoint{1.133106in}{2.593300in}}%
\pgfpathlineto{\pgfqpoint{1.166087in}{2.612429in}}%
\pgfpathlineto{\pgfqpoint{1.201878in}{2.630591in}}%
\pgfpathlineto{\pgfqpoint{1.238406in}{2.646718in}}%
\pgfpathlineto{\pgfqpoint{1.279617in}{2.662664in}}%
\pgfpathlineto{\pgfqpoint{1.325554in}{2.678006in}}%
\pgfpathlineto{\pgfqpoint{1.376206in}{2.692470in}}%
\pgfpathlineto{\pgfqpoint{1.431536in}{2.705868in}}%
\pgfpathlineto{\pgfqpoint{1.489366in}{2.717600in}}%
\pgfpathlineto{\pgfqpoint{1.556065in}{2.728953in}}%
\pgfpathlineto{\pgfqpoint{1.614385in}{2.736829in}}%
\pgfpathlineto{\pgfqpoint{1.694458in}{2.746147in}}%
\pgfpathlineto{\pgfqpoint{1.768194in}{2.752654in}}%
\pgfpathlineto{\pgfqpoint{1.861556in}{2.759201in}}%
\pgfpathlineto{\pgfqpoint{1.963706in}{2.764099in}}%
\pgfpathlineto{\pgfqpoint{2.065917in}{2.766816in}}%
\pgfpathlineto{\pgfqpoint{2.174680in}{2.767617in}}%
\pgfpathlineto{\pgfqpoint{2.281262in}{2.766161in}}%
\pgfpathlineto{\pgfqpoint{2.383449in}{2.762496in}}%
\pgfpathlineto{\pgfqpoint{2.470332in}{2.757086in}}%
\pgfpathlineto{\pgfqpoint{2.546226in}{2.750171in}}%
\pgfpathlineto{\pgfqpoint{2.611096in}{2.742057in}}%
\pgfpathlineto{\pgfqpoint{2.621669in}{2.739494in}}%
\pgfpathlineto{\pgfqpoint{2.628012in}{2.737869in}}%
\pgfpathlineto{\pgfqpoint{2.679566in}{2.728472in}}%
\pgfpathlineto{\pgfqpoint{2.722185in}{2.718486in}}%
\pgfpathlineto{\pgfqpoint{2.757971in}{2.707843in}}%
\pgfpathlineto{\pgfqpoint{2.786917in}{2.697030in}}%
\pgfpathlineto{\pgfqpoint{2.811077in}{2.685737in}}%
\pgfpathlineto{\pgfqpoint{2.830456in}{2.674447in}}%
\pgfpathlineto{\pgfqpoint{2.846967in}{2.662427in}}%
\pgfpathlineto{\pgfqpoint{2.860522in}{2.649957in}}%
\pgfpathlineto{\pgfqpoint{2.871158in}{2.637502in}}%
\pgfpathlineto{\pgfqpoint{2.880286in}{2.623573in}}%
\pgfpathlineto{\pgfqpoint{2.887613in}{2.608315in}}%
\pgfpathlineto{\pgfqpoint{2.892995in}{2.592030in}}%
\pgfpathlineto{\pgfqpoint{2.896861in}{2.572630in}}%
\pgfpathlineto{\pgfqpoint{2.898731in}{2.550343in}}%
\pgfpathlineto{\pgfqpoint{2.898454in}{2.525463in}}%
\pgfpathlineto{\pgfqpoint{2.895840in}{2.495750in}}%
\pgfpathlineto{\pgfqpoint{2.889977in}{2.456498in}}%
\pgfpathlineto{\pgfqpoint{2.879118in}{2.400614in}}%
\pgfpathlineto{\pgfqpoint{2.825404in}{2.138923in}}%
\pgfpathlineto{\pgfqpoint{2.810731in}{2.050903in}}%
\pgfpathlineto{\pgfqpoint{2.796303in}{1.952716in}}%
\pgfpathlineto{\pgfqpoint{2.783022in}{1.846767in}}%
\pgfpathlineto{\pgfqpoint{2.770770in}{1.730620in}}%
\pgfpathlineto{\pgfqpoint{2.759727in}{1.604303in}}%
\pgfpathlineto{\pgfqpoint{2.749681in}{1.462887in}}%
\pgfpathlineto{\pgfqpoint{2.742107in}{1.326260in}}%
\pgfpathlineto{\pgfqpoint{2.734545in}{1.157217in}}%
\pgfpathlineto{\pgfqpoint{2.728595in}{0.973145in}}%
\pgfpathlineto{\pgfqpoint{2.724150in}{0.771585in}}%
\pgfpathlineto{\pgfqpoint{2.720954in}{0.527671in}}%
\pgfpathlineto{\pgfqpoint{2.718764in}{0.430635in}}%
\pgfpathlineto{\pgfqpoint{2.716795in}{0.413382in}}%
\pgfpathlineto{\pgfqpoint{2.715073in}{0.408833in}}%
\pgfpathlineto{\pgfqpoint{2.711867in}{0.405587in}}%
\pgfpathlineto{\pgfqpoint{2.705523in}{0.403966in}}%
\pgfpathlineto{\pgfqpoint{2.688144in}{0.403047in}}%
\pgfpathlineto{\pgfqpoint{2.612009in}{0.402647in}}%
\pgfpathlineto{\pgfqpoint{1.552623in}{0.402563in}}%
\pgfpathlineto{\pgfqpoint{0.449731in}{0.402669in}}%
\pgfpathlineto{\pgfqpoint{0.449731in}{0.402669in}}%
\pgfusepath{stroke}%
\end{pgfscope}%
\begin{pgfscope}%
\pgfpathrectangle{\pgfqpoint{0.448634in}{0.402556in}}{\pgfqpoint{4.350661in}{2.489204in}} %
\pgfusepath{clip}%
\pgfsetrectcap%
\pgfsetroundjoin%
\pgfsetlinewidth{1.003750pt}%
\definecolor{currentstroke}{rgb}{0.498039,0.498039,0.498039}%
\pgfsetstrokecolor{currentstroke}%
\pgfsetdash{}{0pt}%
\pgfpathmoveto{\pgfqpoint{2.755303in}{2.082243in}}%
\pgfpathlineto{\pgfqpoint{2.734213in}{1.839499in}}%
\pgfpathlineto{\pgfqpoint{2.720266in}{1.650996in}}%
\pgfpathlineto{\pgfqpoint{2.706285in}{1.430036in}}%
\pgfpathlineto{\pgfqpoint{2.676703in}{0.948328in}}%
\pgfpathlineto{\pgfqpoint{2.667749in}{0.846789in}}%
\pgfpathlineto{\pgfqpoint{2.659135in}{0.772770in}}%
\pgfpathlineto{\pgfqpoint{2.650488in}{0.716385in}}%
\pgfpathlineto{\pgfqpoint{2.641209in}{0.670305in}}%
\pgfpathlineto{\pgfqpoint{2.631850in}{0.634543in}}%
\pgfpathlineto{\pgfqpoint{2.621683in}{0.604359in}}%
\pgfpathlineto{\pgfqpoint{2.611079in}{0.579824in}}%
\pgfpathlineto{\pgfqpoint{2.599575in}{0.558710in}}%
\pgfpathlineto{\pgfqpoint{2.587538in}{0.541055in}}%
\pgfpathlineto{\pgfqpoint{2.573885in}{0.525015in}}%
\pgfpathlineto{\pgfqpoint{2.558747in}{0.510827in}}%
\pgfpathlineto{\pgfqpoint{2.542373in}{0.498562in}}%
\pgfpathlineto{\pgfqpoint{2.523086in}{0.487073in}}%
\pgfpathlineto{\pgfqpoint{2.500948in}{0.476701in}}%
\pgfpathlineto{\pgfqpoint{2.476094in}{0.467587in}}%
\pgfpathlineto{\pgfqpoint{2.446544in}{0.459180in}}%
\pgfpathlineto{\pgfqpoint{2.410212in}{0.451316in}}%
\pgfpathlineto{\pgfqpoint{2.367136in}{0.444354in}}%
\pgfpathlineto{\pgfqpoint{2.313043in}{0.437963in}}%
\pgfpathlineto{\pgfqpoint{2.243619in}{0.432165in}}%
\pgfpathlineto{\pgfqpoint{2.154542in}{0.427104in}}%
\pgfpathlineto{\pgfqpoint{2.037136in}{0.422773in}}%
\pgfpathlineto{\pgfqpoint{1.876193in}{0.419200in}}%
\pgfpathlineto{\pgfqpoint{1.654321in}{0.416634in}}%
\pgfpathlineto{\pgfqpoint{1.365004in}{0.415584in}}%
\pgfpathlineto{\pgfqpoint{1.075687in}{0.416657in}}%
\pgfpathlineto{\pgfqpoint{0.871222in}{0.419461in}}%
\pgfpathlineto{\pgfqpoint{0.742923in}{0.423284in}}%
\pgfpathlineto{\pgfqpoint{0.662531in}{0.427709in}}%
\pgfpathlineto{\pgfqpoint{0.608330in}{0.432758in}}%
\pgfpathlineto{\pgfqpoint{0.571662in}{0.438203in}}%
\pgfpathlineto{\pgfqpoint{0.543938in}{0.444537in}}%
\pgfpathlineto{\pgfqpoint{0.525174in}{0.450899in}}%
\pgfpathlineto{\pgfqpoint{0.511144in}{0.457645in}}%
\pgfpathlineto{\pgfqpoint{0.499891in}{0.465186in}}%
\pgfpathlineto{\pgfqpoint{0.489843in}{0.474687in}}%
\pgfpathlineto{\pgfqpoint{0.481473in}{0.486120in}}%
\pgfpathlineto{\pgfqpoint{0.474947in}{0.499036in}}%
\pgfpathlineto{\pgfqpoint{0.469355in}{0.515229in}}%
\pgfpathlineto{\pgfqpoint{0.464444in}{0.536904in}}%
\pgfpathlineto{\pgfqpoint{0.460297in}{0.566387in}}%
\pgfpathlineto{\pgfqpoint{0.456857in}{0.608515in}}%
\pgfpathlineto{\pgfqpoint{0.454073in}{0.673152in}}%
\pgfpathlineto{\pgfqpoint{0.451946in}{0.780159in}}%
\pgfpathlineto{\pgfqpoint{0.450418in}{0.986755in}}%
\pgfpathlineto{\pgfqpoint{0.449503in}{1.477126in}}%
\pgfpathlineto{\pgfqpoint{0.449747in}{2.647052in}}%
\pgfpathlineto{\pgfqpoint{0.451278in}{2.843686in}}%
\pgfpathlineto{\pgfqpoint{0.453109in}{2.873468in}}%
\pgfpathlineto{\pgfqpoint{0.455370in}{2.883047in}}%
\pgfpathlineto{\pgfqpoint{0.458055in}{2.886916in}}%
\pgfpathlineto{\pgfqpoint{0.461927in}{2.889110in}}%
\pgfpathlineto{\pgfqpoint{0.470510in}{2.890640in}}%
\pgfpathlineto{\pgfqpoint{0.494424in}{2.891469in}}%
\pgfpathlineto{\pgfqpoint{0.616242in}{2.891739in}}%
\pgfpathlineto{\pgfqpoint{4.723266in}{2.891731in}}%
\pgfpathlineto{\pgfqpoint{4.786333in}{2.890873in}}%
\pgfpathlineto{\pgfqpoint{4.792614in}{2.889043in}}%
\pgfpathlineto{\pgfqpoint{4.795127in}{2.885133in}}%
\pgfpathlineto{\pgfqpoint{4.796582in}{2.875342in}}%
\pgfpathlineto{\pgfqpoint{4.797459in}{2.852965in}}%
\pgfpathlineto{\pgfqpoint{4.797459in}{2.852965in}}%
\pgfusepath{stroke}%
\end{pgfscope}%
\begin{pgfscope}%
\pgfpathrectangle{\pgfqpoint{0.448634in}{0.402556in}}{\pgfqpoint{4.350661in}{2.489204in}} %
\pgfusepath{clip}%
\pgfsetrectcap%
\pgfsetroundjoin%
\pgfsetlinewidth{1.003750pt}%
\definecolor{currentstroke}{rgb}{0.498039,0.498039,0.498039}%
\pgfsetstrokecolor{currentstroke}%
\pgfsetdash{}{0pt}%
\pgfpathmoveto{\pgfqpoint{0.752547in}{2.850332in}}%
\pgfpathlineto{\pgfqpoint{0.780073in}{2.857729in}}%
\pgfpathlineto{\pgfqpoint{0.814368in}{2.864492in}}%
\pgfpathlineto{\pgfqpoint{0.855386in}{2.870271in}}%
\pgfpathlineto{\pgfqpoint{0.907405in}{2.875314in}}%
\pgfpathlineto{\pgfqpoint{0.976909in}{2.879697in}}%
\pgfpathlineto{\pgfqpoint{1.072569in}{2.883337in}}%
\pgfpathlineto{\pgfqpoint{1.209591in}{2.886164in}}%
\pgfpathlineto{\pgfqpoint{1.427115in}{2.888375in}}%
\pgfpathlineto{\pgfqpoint{1.825198in}{2.889916in}}%
\pgfpathlineto{\pgfqpoint{2.699681in}{2.890662in}}%
\pgfpathlineto{\pgfqpoint{3.822151in}{2.889677in}}%
\pgfpathlineto{\pgfqpoint{4.207178in}{2.887353in}}%
\pgfpathlineto{\pgfqpoint{4.370303in}{2.884208in}}%
\pgfpathlineto{\pgfqpoint{4.461601in}{2.880317in}}%
\pgfpathlineto{\pgfqpoint{4.520193in}{2.875695in}}%
\pgfpathlineto{\pgfqpoint{4.561256in}{2.870368in}}%
\pgfpathlineto{\pgfqpoint{4.591264in}{2.864453in}}%
\pgfpathlineto{\pgfqpoint{4.614488in}{2.857888in}}%
\pgfpathlineto{\pgfqpoint{4.635100in}{2.849988in}}%
\pgfpathlineto{\pgfqpoint{4.650833in}{2.841501in}}%
\pgfpathlineto{\pgfqpoint{4.665568in}{2.830932in}}%
\pgfpathlineto{\pgfqpoint{4.677228in}{2.819745in}}%
\pgfpathlineto{\pgfqpoint{4.687530in}{2.806930in}}%
\pgfpathlineto{\pgfqpoint{4.697548in}{2.790663in}}%
\pgfpathlineto{\pgfqpoint{4.706765in}{2.770913in}}%
\pgfpathlineto{\pgfqpoint{4.714914in}{2.747844in}}%
\pgfpathlineto{\pgfqpoint{4.722526in}{2.719280in}}%
\pgfpathlineto{\pgfqpoint{4.729723in}{2.682869in}}%
\pgfpathlineto{\pgfqpoint{4.736115in}{2.638669in}}%
\pgfpathlineto{\pgfqpoint{4.741728in}{2.584289in}}%
\pgfpathlineto{\pgfqpoint{4.745075in}{2.539649in}}%
\pgfpathlineto{\pgfqpoint{4.750010in}{2.452713in}}%
\pgfpathlineto{\pgfqpoint{4.754011in}{2.340795in}}%
\pgfpathlineto{\pgfqpoint{4.756622in}{2.203923in}}%
\pgfpathlineto{\pgfqpoint{4.757741in}{2.032174in}}%
\pgfpathlineto{\pgfqpoint{4.756727in}{1.847978in}}%
\pgfpathlineto{\pgfqpoint{4.753542in}{1.666305in}}%
\pgfpathlineto{\pgfqpoint{4.748376in}{1.504618in}}%
\pgfpathlineto{\pgfqpoint{4.741772in}{1.372909in}}%
\pgfpathlineto{\pgfqpoint{4.736401in}{1.295991in}}%
\pgfpathlineto{\pgfqpoint{4.727930in}{1.201903in}}%
\pgfpathlineto{\pgfqpoint{4.718716in}{1.125466in}}%
\pgfpathlineto{\pgfqpoint{4.705377in}{1.037206in}}%
\pgfpathlineto{\pgfqpoint{4.694480in}{0.986447in}}%
\pgfpathlineto{\pgfqpoint{4.682468in}{0.941201in}}%
\pgfpathlineto{\pgfqpoint{4.669551in}{0.901557in}}%
\pgfpathlineto{\pgfqpoint{4.656119in}{0.867537in}}%
\pgfpathlineto{\pgfqpoint{4.643828in}{0.841192in}}%
\pgfpathlineto{\pgfqpoint{4.628756in}{0.813821in}}%
\pgfpathlineto{\pgfqpoint{4.613186in}{0.789855in}}%
\pgfpathlineto{\pgfqpoint{4.596011in}{0.767372in}}%
\pgfpathlineto{\pgfqpoint{4.577265in}{0.746598in}}%
\pgfpathlineto{\pgfqpoint{4.555392in}{0.726101in}}%
\pgfpathlineto{\pgfqpoint{4.533959in}{0.709062in}}%
\pgfpathlineto{\pgfqpoint{4.511470in}{0.693914in}}%
\pgfpathlineto{\pgfqpoint{4.486261in}{0.679263in}}%
\pgfpathlineto{\pgfqpoint{4.458291in}{0.665488in}}%
\pgfpathlineto{\pgfqpoint{4.427618in}{0.652765in}}%
\pgfpathlineto{\pgfqpoint{4.392221in}{0.640534in}}%
\pgfpathlineto{\pgfqpoint{4.364734in}{0.632970in}}%
\pgfpathlineto{\pgfqpoint{4.351955in}{0.629978in}}%
\pgfpathlineto{\pgfqpoint{4.309304in}{0.620169in}}%
\pgfpathlineto{\pgfqpoint{4.262041in}{0.611588in}}%
\pgfpathlineto{\pgfqpoint{4.208062in}{0.604036in}}%
\pgfpathlineto{\pgfqpoint{4.138748in}{0.596817in}}%
\pgfpathlineto{\pgfqpoint{4.073579in}{0.592936in}}%
\pgfpathlineto{\pgfqpoint{4.001813in}{0.591098in}}%
\pgfpathlineto{\pgfqpoint{3.932206in}{0.591515in}}%
\pgfpathlineto{\pgfqpoint{3.860460in}{0.594185in}}%
\pgfpathlineto{\pgfqpoint{3.786637in}{0.599280in}}%
\pgfpathlineto{\pgfqpoint{3.721659in}{0.606183in}}%
\pgfpathlineto{\pgfqpoint{3.656893in}{0.615314in}}%
\pgfpathlineto{\pgfqpoint{3.594620in}{0.626762in}}%
\pgfpathlineto{\pgfqpoint{3.545613in}{0.638277in}}%
\pgfpathlineto{\pgfqpoint{3.494943in}{0.652638in}}%
\pgfpathlineto{\pgfqpoint{3.453309in}{0.667071in}}%
\pgfpathlineto{\pgfqpoint{3.416433in}{0.682124in}}%
\pgfpathlineto{\pgfqpoint{3.382293in}{0.698375in}}%
\pgfpathlineto{\pgfqpoint{3.350953in}{0.715687in}}%
\pgfpathlineto{\pgfqpoint{3.322452in}{0.733855in}}%
\pgfpathlineto{\pgfqpoint{3.294988in}{0.754004in}}%
\pgfpathlineto{\pgfqpoint{3.263846in}{0.781150in}}%
\pgfpathlineto{\pgfqpoint{3.241019in}{0.804208in}}%
\pgfpathlineto{\pgfqpoint{3.219674in}{0.829055in}}%
\pgfpathlineto{\pgfqpoint{3.199932in}{0.855581in}}%
\pgfpathlineto{\pgfqpoint{3.183164in}{0.881627in}}%
\pgfpathlineto{\pgfqpoint{3.162403in}{0.919602in}}%
\pgfpathlineto{\pgfqpoint{3.144401in}{0.959381in}}%
\pgfpathlineto{\pgfqpoint{3.130622in}{0.995947in}}%
\pgfpathlineto{\pgfqpoint{3.118945in}{1.033457in}}%
\pgfpathlineto{\pgfqpoint{3.107389in}{1.078858in}}%
\pgfpathlineto{\pgfqpoint{3.098590in}{1.122513in}}%
\pgfpathlineto{\pgfqpoint{3.091414in}{1.169084in}}%
\pgfpathlineto{\pgfqpoint{3.085999in}{1.218476in}}%
\pgfpathlineto{\pgfqpoint{3.082313in}{1.273070in}}%
\pgfpathlineto{\pgfqpoint{3.080928in}{1.327805in}}%
\pgfpathlineto{\pgfqpoint{3.081752in}{1.385044in}}%
\pgfpathlineto{\pgfqpoint{3.084790in}{1.442186in}}%
\pgfpathlineto{\pgfqpoint{3.089999in}{1.499122in}}%
\pgfpathlineto{\pgfqpoint{3.097400in}{1.555739in}}%
\pgfpathlineto{\pgfqpoint{3.107021in}{1.611917in}}%
\pgfpathlineto{\pgfqpoint{3.118363in}{1.665115in}}%
\pgfpathlineto{\pgfqpoint{3.131854in}{1.717651in}}%
\pgfpathlineto{\pgfqpoint{3.146839in}{1.767026in}}%
\pgfpathlineto{\pgfqpoint{3.163907in}{1.815507in}}%
\pgfpathlineto{\pgfqpoint{3.182209in}{1.860666in}}%
\pgfpathlineto{\pgfqpoint{3.202537in}{1.904676in}}%
\pgfpathlineto{\pgfqpoint{3.223744in}{1.945264in}}%
\pgfpathlineto{\pgfqpoint{3.246756in}{1.984544in}}%
\pgfpathlineto{\pgfqpoint{3.271488in}{2.022431in}}%
\pgfpathlineto{\pgfqpoint{3.299237in}{2.060767in}}%
\pgfpathlineto{\pgfqpoint{3.329988in}{2.099418in}}%
\pgfpathlineto{\pgfqpoint{3.371179in}{2.147324in}}%
\pgfpathlineto{\pgfqpoint{3.409052in}{2.191970in}}%
\pgfpathlineto{\pgfqpoint{3.421374in}{2.209366in}}%
\pgfpathlineto{\pgfqpoint{3.428069in}{2.222162in}}%
\pgfpathlineto{\pgfqpoint{3.430669in}{2.231626in}}%
\pgfpathlineto{\pgfqpoint{3.430274in}{2.239021in}}%
\pgfpathlineto{\pgfqpoint{3.428304in}{2.243429in}}%
\pgfpathlineto{\pgfqpoint{3.423355in}{2.248217in}}%
\pgfpathlineto{\pgfqpoint{3.415228in}{2.251683in}}%
\pgfpathlineto{\pgfqpoint{3.404461in}{2.253328in}}%
\pgfpathlineto{\pgfqpoint{3.389248in}{2.252973in}}%
\pgfpathlineto{\pgfqpoint{3.369866in}{2.249891in}}%
\pgfpathlineto{\pgfqpoint{3.348700in}{2.244178in}}%
\pgfpathlineto{\pgfqpoint{3.323870in}{2.234984in}}%
\pgfpathlineto{\pgfqpoint{3.297715in}{2.222698in}}%
\pgfpathlineto{\pgfqpoint{3.272382in}{2.208332in}}%
\pgfpathlineto{\pgfqpoint{3.246079in}{2.190780in}}%
\pgfpathlineto{\pgfqpoint{3.220857in}{2.171260in}}%
\pgfpathlineto{\pgfqpoint{3.196778in}{2.149932in}}%
\pgfpathlineto{\pgfqpoint{3.172304in}{2.125247in}}%
\pgfpathlineto{\pgfqpoint{3.149247in}{2.098835in}}%
\pgfpathlineto{\pgfqpoint{3.126234in}{2.068965in}}%
\pgfpathlineto{\pgfqpoint{3.104851in}{2.037547in}}%
\pgfpathlineto{\pgfqpoint{3.083886in}{2.002696in}}%
\pgfpathlineto{\pgfqpoint{3.063583in}{1.964391in}}%
\pgfpathlineto{\pgfqpoint{3.045130in}{1.924878in}}%
\pgfpathlineto{\pgfqpoint{3.027530in}{1.882091in}}%
\pgfpathlineto{\pgfqpoint{3.010940in}{1.836074in}}%
\pgfpathlineto{\pgfqpoint{2.994781in}{1.784533in}}%
\pgfpathlineto{\pgfqpoint{2.980011in}{1.729837in}}%
\pgfpathlineto{\pgfqpoint{2.966691in}{1.672077in}}%
\pgfpathlineto{\pgfqpoint{2.954438in}{1.608899in}}%
\pgfpathlineto{\pgfqpoint{2.943888in}{1.542788in}}%
\pgfpathlineto{\pgfqpoint{2.934814in}{1.471355in}}%
\pgfpathlineto{\pgfqpoint{2.927723in}{1.397125in}}%
\pgfpathlineto{\pgfqpoint{2.922875in}{1.322659in}}%
\pgfpathlineto{\pgfqpoint{2.920264in}{1.248046in}}%
\pgfpathlineto{\pgfqpoint{2.919934in}{1.175864in}}%
\pgfpathlineto{\pgfqpoint{2.921850in}{1.106205in}}%
\pgfpathlineto{\pgfqpoint{2.925841in}{1.041652in}}%
\pgfpathlineto{\pgfqpoint{2.931727in}{0.982297in}}%
\pgfpathlineto{\pgfqpoint{2.939330in}{0.928235in}}%
\pgfpathlineto{\pgfqpoint{2.948444in}{0.879562in}}%
\pgfpathlineto{\pgfqpoint{2.958797in}{0.836358in}}%
\pgfpathlineto{\pgfqpoint{2.970024in}{0.798666in}}%
\pgfpathlineto{\pgfqpoint{2.982561in}{0.764202in}}%
\pgfpathlineto{\pgfqpoint{2.996248in}{0.733080in}}%
\pgfpathlineto{\pgfqpoint{3.010841in}{0.705371in}}%
\pgfpathlineto{\pgfqpoint{3.026008in}{0.681069in}}%
\pgfpathlineto{\pgfqpoint{3.042822in}{0.658231in}}%
\pgfpathlineto{\pgfqpoint{3.061225in}{0.637058in}}%
\pgfpathlineto{\pgfqpoint{3.081091in}{0.617695in}}%
\pgfpathlineto{\pgfqpoint{3.102240in}{0.600198in}}%
\pgfpathlineto{\pgfqpoint{3.124465in}{0.584542in}}%
\pgfpathlineto{\pgfqpoint{3.149518in}{0.569545in}}%
\pgfpathlineto{\pgfqpoint{3.177374in}{0.555473in}}%
\pgfpathlineto{\pgfqpoint{3.207964in}{0.542496in}}%
\pgfpathlineto{\pgfqpoint{3.243296in}{0.530019in}}%
\pgfpathlineto{\pgfqpoint{3.283353in}{0.518381in}}%
\pgfpathlineto{\pgfqpoint{3.328083in}{0.507787in}}%
\pgfpathlineto{\pgfqpoint{3.379578in}{0.497963in}}%
\pgfpathlineto{\pgfqpoint{3.439963in}{0.488864in}}%
\pgfpathlineto{\pgfqpoint{3.509214in}{0.480797in}}%
\pgfpathlineto{\pgfqpoint{3.589464in}{0.473762in}}%
\pgfpathlineto{\pgfqpoint{3.682859in}{0.467863in}}%
\pgfpathlineto{\pgfqpoint{3.787205in}{0.463514in}}%
\pgfpathlineto{\pgfqpoint{3.913346in}{0.460553in}}%
\pgfpathlineto{\pgfqpoint{4.039512in}{0.459867in}}%
\pgfpathlineto{\pgfqpoint{4.165673in}{0.461377in}}%
\pgfpathlineto{\pgfqpoint{4.274397in}{0.464794in}}%
\pgfpathlineto{\pgfqpoint{4.365651in}{0.469870in}}%
\pgfpathlineto{\pgfqpoint{4.439402in}{0.476208in}}%
\pgfpathlineto{\pgfqpoint{4.497792in}{0.483430in}}%
\pgfpathlineto{\pgfqpoint{4.542974in}{0.491121in}}%
\pgfpathlineto{\pgfqpoint{4.581372in}{0.499860in}}%
\pgfpathlineto{\pgfqpoint{4.612923in}{0.509359in}}%
\pgfpathlineto{\pgfqpoint{4.637634in}{0.518965in}}%
\pgfpathlineto{\pgfqpoint{4.659569in}{0.529882in}}%
\pgfpathlineto{\pgfqpoint{4.676719in}{0.540671in}}%
\pgfpathlineto{\pgfqpoint{4.692836in}{0.553369in}}%
\pgfpathlineto{\pgfqpoint{4.705982in}{0.566401in}}%
\pgfpathlineto{\pgfqpoint{4.717769in}{0.581038in}}%
\pgfpathlineto{\pgfqpoint{4.728102in}{0.597049in}}%
\pgfpathlineto{\pgfqpoint{4.737946in}{0.616401in}}%
\pgfpathlineto{\pgfqpoint{4.746872in}{0.639089in}}%
\pgfpathlineto{\pgfqpoint{4.754680in}{0.664964in}}%
\pgfpathlineto{\pgfqpoint{4.762283in}{0.698701in}}%
\pgfpathlineto{\pgfqpoint{4.768713in}{0.737836in}}%
\pgfpathlineto{\pgfqpoint{4.774718in}{0.789651in}}%
\pgfpathlineto{\pgfqpoint{4.779799in}{0.854105in}}%
\pgfpathlineto{\pgfqpoint{4.784255in}{0.941075in}}%
\pgfpathlineto{\pgfqpoint{4.788000in}{1.060478in}}%
\pgfpathlineto{\pgfqpoint{4.791055in}{1.234686in}}%
\pgfpathlineto{\pgfqpoint{4.793383in}{1.503506in}}%
\pgfpathlineto{\pgfqpoint{4.794758in}{1.941603in}}%
\pgfpathlineto{\pgfqpoint{4.794310in}{2.471803in}}%
\pgfpathlineto{\pgfqpoint{4.792264in}{2.715731in}}%
\pgfpathlineto{\pgfqpoint{4.789595in}{2.805285in}}%
\pgfpathlineto{\pgfqpoint{4.786686in}{2.842466in}}%
\pgfpathlineto{\pgfqpoint{4.783272in}{2.861974in}}%
\pgfpathlineto{\pgfqpoint{4.779044in}{2.873403in}}%
\pgfpathlineto{\pgfqpoint{4.774928in}{2.879166in}}%
\pgfpathlineto{\pgfqpoint{4.769545in}{2.883345in}}%
\pgfpathlineto{\pgfqpoint{4.761337in}{2.886590in}}%
\pgfpathlineto{\pgfqpoint{4.748451in}{2.888884in}}%
\pgfpathlineto{\pgfqpoint{4.724564in}{2.890403in}}%
\pgfpathlineto{\pgfqpoint{4.665837in}{2.891312in}}%
\pgfpathlineto{\pgfqpoint{4.430901in}{2.891694in}}%
\pgfpathlineto{\pgfqpoint{0.952548in}{2.891672in}}%
\pgfpathlineto{\pgfqpoint{0.593622in}{2.890454in}}%
\pgfpathlineto{\pgfqpoint{0.539277in}{2.888317in}}%
\pgfpathlineto{\pgfqpoint{0.515488in}{2.885451in}}%
\pgfpathlineto{\pgfqpoint{0.500675in}{2.881492in}}%
\pgfpathlineto{\pgfqpoint{0.490775in}{2.876402in}}%
\pgfpathlineto{\pgfqpoint{0.483920in}{2.870309in}}%
\pgfpathlineto{\pgfqpoint{0.478518in}{2.862530in}}%
\pgfpathlineto{\pgfqpoint{0.473732in}{2.851376in}}%
\pgfpathlineto{\pgfqpoint{0.469432in}{2.834676in}}%
\pgfpathlineto{\pgfqpoint{0.465749in}{2.810153in}}%
\pgfpathlineto{\pgfqpoint{0.462429in}{2.770513in}}%
\pgfpathlineto{\pgfqpoint{0.459599in}{2.705878in}}%
\pgfpathlineto{\pgfqpoint{0.457231in}{2.593898in}}%
\pgfpathlineto{\pgfqpoint{0.456474in}{2.531674in}}%
\pgfpathlineto{\pgfqpoint{0.456474in}{2.531674in}}%
\pgfusepath{stroke}%
\end{pgfscope}%
\begin{pgfscope}%
\pgfpathrectangle{\pgfqpoint{0.448634in}{0.402556in}}{\pgfqpoint{4.350661in}{2.489204in}} %
\pgfusepath{clip}%
\pgfsetrectcap%
\pgfsetroundjoin%
\pgfsetlinewidth{1.003750pt}%
\definecolor{currentstroke}{rgb}{0.498039,0.498039,0.498039}%
\pgfsetstrokecolor{currentstroke}%
\pgfsetdash{}{0pt}%
\pgfpathmoveto{\pgfqpoint{2.534999in}{2.741923in}}%
\pgfpathlineto{\pgfqpoint{2.597656in}{2.733566in}}%
\pgfpathlineto{\pgfqpoint{2.649259in}{2.724524in}}%
\pgfpathlineto{\pgfqpoint{2.694076in}{2.714424in}}%
\pgfpathlineto{\pgfqpoint{2.729953in}{2.704185in}}%
\pgfpathlineto{\pgfqpoint{2.761090in}{2.693042in}}%
\pgfpathlineto{\pgfqpoint{2.787405in}{2.681212in}}%
\pgfpathlineto{\pgfqpoint{2.808858in}{2.669105in}}%
\pgfpathlineto{\pgfqpoint{2.825527in}{2.657371in}}%
\pgfpathlineto{\pgfqpoint{2.839339in}{2.645273in}}%
\pgfpathlineto{\pgfqpoint{2.851808in}{2.631402in}}%
\pgfpathlineto{\pgfqpoint{2.861262in}{2.617757in}}%
\pgfpathlineto{\pgfqpoint{2.869066in}{2.602810in}}%
\pgfpathlineto{\pgfqpoint{2.875065in}{2.586808in}}%
\pgfpathlineto{\pgfqpoint{2.879742in}{2.567643in}}%
\pgfpathlineto{\pgfqpoint{2.882553in}{2.545486in}}%
\pgfpathlineto{\pgfqpoint{2.883272in}{2.520618in}}%
\pgfpathlineto{\pgfqpoint{2.881747in}{2.490807in}}%
\pgfpathlineto{\pgfqpoint{2.877481in}{2.453796in}}%
\pgfpathlineto{\pgfqpoint{2.869078in}{2.402419in}}%
\pgfpathlineto{\pgfqpoint{2.850768in}{2.307629in}}%
\pgfpathlineto{\pgfqpoint{2.824959in}{2.171401in}}%
\pgfpathlineto{\pgfqpoint{2.807884in}{2.068699in}}%
\pgfpathlineto{\pgfqpoint{2.793032in}{1.965546in}}%
\pgfpathlineto{\pgfqpoint{2.779686in}{1.857093in}}%
\pgfpathlineto{\pgfqpoint{2.767380in}{1.738446in}}%
\pgfpathlineto{\pgfqpoint{2.756285in}{1.609634in}}%
\pgfpathlineto{\pgfqpoint{2.746166in}{1.465727in}}%
\pgfpathlineto{\pgfqpoint{2.737107in}{1.304262in}}%
\pgfpathlineto{\pgfqpoint{2.729198in}{1.122777in}}%
\pgfpathlineto{\pgfqpoint{2.722164in}{0.908858in}}%
\pgfpathlineto{\pgfqpoint{2.710425in}{0.535736in}}%
\pgfpathlineto{\pgfqpoint{2.706375in}{0.488677in}}%
\pgfpathlineto{\pgfqpoint{2.701942in}{0.461784in}}%
\pgfpathlineto{\pgfqpoint{2.697072in}{0.445296in}}%
\pgfpathlineto{\pgfqpoint{2.691701in}{0.434501in}}%
\pgfpathlineto{\pgfqpoint{2.685854in}{0.427154in}}%
\pgfpathlineto{\pgfqpoint{2.678743in}{0.421448in}}%
\pgfpathlineto{\pgfqpoint{2.668772in}{0.416521in}}%
\pgfpathlineto{\pgfqpoint{2.656135in}{0.412825in}}%
\pgfpathlineto{\pgfqpoint{2.636760in}{0.409659in}}%
\pgfpathlineto{\pgfqpoint{2.606387in}{0.407178in}}%
\pgfpathlineto{\pgfqpoint{2.552031in}{0.405265in}}%
\pgfpathlineto{\pgfqpoint{2.438921in}{0.403927in}}%
\pgfpathlineto{\pgfqpoint{2.123499in}{0.403152in}}%
\pgfpathlineto{\pgfqpoint{0.870509in}{0.403024in}}%
\pgfpathlineto{\pgfqpoint{0.502882in}{0.404342in}}%
\pgfpathlineto{\pgfqpoint{0.463763in}{0.406066in}}%
\pgfpathlineto{\pgfqpoint{0.455241in}{0.407940in}}%
\pgfpathlineto{\pgfqpoint{0.451673in}{0.410672in}}%
\pgfpathlineto{\pgfqpoint{0.450062in}{0.415257in}}%
\pgfpathlineto{\pgfqpoint{0.449096in}{0.427638in}}%
\pgfpathlineto{\pgfqpoint{0.448709in}{0.487376in}}%
\pgfpathlineto{\pgfqpoint{0.448636in}{1.383489in}}%
\pgfpathlineto{\pgfqpoint{0.448655in}{2.889457in}}%
\pgfpathlineto{\pgfqpoint{0.448655in}{2.889457in}}%
\pgfusepath{stroke}%
\end{pgfscope}%
\begin{pgfscope}%
\pgfpathrectangle{\pgfqpoint{0.448634in}{0.402556in}}{\pgfqpoint{4.350661in}{2.489204in}} %
\pgfusepath{clip}%
\pgfsetrectcap%
\pgfsetroundjoin%
\pgfsetlinewidth{1.003750pt}%
\definecolor{currentstroke}{rgb}{0.498039,0.498039,0.498039}%
\pgfsetstrokecolor{currentstroke}%
\pgfsetdash{}{0pt}%
\pgfpathmoveto{\pgfqpoint{3.387283in}{2.160151in}}%
\pgfpathlineto{\pgfqpoint{3.334357in}{2.101042in}}%
\pgfpathlineto{\pgfqpoint{3.303192in}{2.062828in}}%
\pgfpathlineto{\pgfqpoint{3.275095in}{2.024825in}}%
\pgfpathlineto{\pgfqpoint{3.250059in}{1.987201in}}%
\pgfpathlineto{\pgfqpoint{3.226763in}{1.948141in}}%
\pgfpathlineto{\pgfqpoint{3.205289in}{1.907737in}}%
\pgfpathlineto{\pgfqpoint{3.185676in}{1.866112in}}%
\pgfpathlineto{\pgfqpoint{3.167028in}{1.821139in}}%
\pgfpathlineto{\pgfqpoint{3.150380in}{1.775150in}}%
\pgfpathlineto{\pgfqpoint{3.134305in}{1.723575in}}%
\pgfpathlineto{\pgfqpoint{3.121078in}{1.673546in}}%
\pgfpathlineto{\pgfqpoint{3.109349in}{1.620459in}}%
\pgfpathlineto{\pgfqpoint{3.099725in}{1.566819in}}%
\pgfpathlineto{\pgfqpoint{3.091856in}{1.510285in}}%
\pgfpathlineto{\pgfqpoint{3.086184in}{1.453407in}}%
\pgfpathlineto{\pgfqpoint{3.082689in}{1.396300in}}%
\pgfpathlineto{\pgfqpoint{3.081402in}{1.339072in}}%
\pgfpathlineto{\pgfqpoint{3.082329in}{1.284325in}}%
\pgfpathlineto{\pgfqpoint{3.085329in}{1.232169in}}%
\pgfpathlineto{\pgfqpoint{3.090465in}{1.180233in}}%
\pgfpathlineto{\pgfqpoint{3.097561in}{1.131121in}}%
\pgfpathlineto{\pgfqpoint{3.106492in}{1.084950in}}%
\pgfpathlineto{\pgfqpoint{3.117111in}{1.041830in}}%
\pgfpathlineto{\pgfqpoint{3.129894in}{0.999490in}}%
\pgfpathlineto{\pgfqpoint{3.143485in}{0.962832in}}%
\pgfpathlineto{\pgfqpoint{3.158148in}{0.929484in}}%
\pgfpathlineto{\pgfqpoint{3.174673in}{0.897298in}}%
\pgfpathlineto{\pgfqpoint{3.190521in}{0.870506in}}%
\pgfpathlineto{\pgfqpoint{3.209378in}{0.843149in}}%
\pgfpathlineto{\pgfqpoint{3.229880in}{0.817390in}}%
\pgfpathlineto{\pgfqpoint{3.251936in}{0.793371in}}%
\pgfpathlineto{\pgfqpoint{3.275400in}{0.771166in}}%
\pgfpathlineto{\pgfqpoint{3.303600in}{0.747839in}}%
\pgfpathlineto{\pgfqpoint{3.333297in}{0.727082in}}%
\pgfpathlineto{\pgfqpoint{3.362136in}{0.709625in}}%
\pgfpathlineto{\pgfqpoint{3.393763in}{0.693009in}}%
\pgfpathlineto{\pgfqpoint{3.428144in}{0.677435in}}%
\pgfpathlineto{\pgfqpoint{3.465211in}{0.663009in}}%
\pgfpathlineto{\pgfqpoint{3.507004in}{0.649191in}}%
\pgfpathlineto{\pgfqpoint{3.547190in}{0.638148in}}%
\pgfpathlineto{\pgfqpoint{3.594068in}{0.627140in}}%
\pgfpathlineto{\pgfqpoint{3.645561in}{0.617343in}}%
\pgfpathlineto{\pgfqpoint{3.686528in}{0.611111in}}%
\pgfpathlineto{\pgfqpoint{3.744896in}{0.603641in}}%
\pgfpathlineto{\pgfqpoint{3.805596in}{0.597914in}}%
\pgfpathlineto{\pgfqpoint{3.968613in}{0.591180in}}%
\pgfpathlineto{\pgfqpoint{4.038219in}{0.591891in}}%
\pgfpathlineto{\pgfqpoint{4.107779in}{0.594830in}}%
\pgfpathlineto{\pgfqpoint{4.157668in}{0.598949in}}%
\pgfpathlineto{\pgfqpoint{4.216149in}{0.605152in}}%
\pgfpathlineto{\pgfqpoint{4.270091in}{0.613035in}}%
\pgfpathlineto{\pgfqpoint{4.317302in}{0.621979in}}%
\pgfpathlineto{\pgfqpoint{4.370548in}{0.634579in}}%
\pgfpathlineto{\pgfqpoint{4.414644in}{0.648160in}}%
\pgfpathlineto{\pgfqpoint{4.447595in}{0.660971in}}%
\pgfpathlineto{\pgfqpoint{4.477815in}{0.675037in}}%
\pgfpathlineto{\pgfqpoint{4.505214in}{0.690236in}}%
\pgfpathlineto{\pgfqpoint{4.531631in}{0.707555in}}%
\pgfpathlineto{\pgfqpoint{4.553174in}{0.724410in}}%
\pgfpathlineto{\pgfqpoint{4.573542in}{0.743081in}}%
\pgfpathlineto{\pgfqpoint{4.592573in}{0.763513in}}%
\pgfpathlineto{\pgfqpoint{4.610045in}{0.785694in}}%
\pgfpathlineto{\pgfqpoint{4.625917in}{0.809399in}}%
\pgfpathlineto{\pgfqpoint{4.641309in}{0.836535in}}%
\pgfpathlineto{\pgfqpoint{4.654884in}{0.864915in}}%
\pgfpathlineto{\pgfqpoint{4.669302in}{0.901155in}}%
\pgfpathlineto{\pgfqpoint{4.681551in}{0.938428in}}%
\pgfpathlineto{\pgfqpoint{4.692542in}{0.978827in}}%
\pgfpathlineto{\pgfqpoint{4.703307in}{1.027057in}}%
\pgfpathlineto{\pgfqpoint{4.709526in}{1.063680in}}%
\pgfpathlineto{\pgfqpoint{4.716507in}{1.110281in}}%
\pgfpathlineto{\pgfqpoint{4.725111in}{1.176761in}}%
\pgfpathlineto{\pgfqpoint{4.732852in}{1.253413in}}%
\pgfpathlineto{\pgfqpoint{4.742220in}{1.382399in}}%
\pgfpathlineto{\pgfqpoint{4.748121in}{1.499194in}}%
\pgfpathlineto{\pgfqpoint{4.752580in}{1.631020in}}%
\pgfpathlineto{\pgfqpoint{4.755890in}{1.787793in}}%
\pgfpathlineto{\pgfqpoint{4.757621in}{1.969492in}}%
\pgfpathlineto{\pgfqpoint{4.757177in}{2.151202in}}%
\pgfpathlineto{\pgfqpoint{4.754578in}{2.317950in}}%
\pgfpathlineto{\pgfqpoint{4.750339in}{2.444804in}}%
\pgfpathlineto{\pgfqpoint{4.745077in}{2.539199in}}%
\pgfpathlineto{\pgfqpoint{4.738647in}{2.616003in}}%
\pgfpathlineto{\pgfqpoint{4.731730in}{2.670184in}}%
\pgfpathlineto{\pgfqpoint{4.724192in}{2.711603in}}%
\pgfpathlineto{\pgfqpoint{4.716405in}{2.742703in}}%
\pgfpathlineto{\pgfqpoint{4.707780in}{2.768232in}}%
\pgfpathlineto{\pgfqpoint{4.698818in}{2.788136in}}%
\pgfpathlineto{\pgfqpoint{4.689065in}{2.804613in}}%
\pgfpathlineto{\pgfqpoint{4.679003in}{2.817676in}}%
\pgfpathlineto{\pgfqpoint{4.667558in}{2.829151in}}%
\pgfpathlineto{\pgfqpoint{4.654896in}{2.838806in}}%
\pgfpathlineto{\pgfqpoint{4.639403in}{2.847851in}}%
\pgfpathlineto{\pgfqpoint{4.623155in}{2.854962in}}%
\pgfpathlineto{\pgfqpoint{4.595838in}{2.863296in}}%
\pgfpathlineto{\pgfqpoint{4.568037in}{2.869193in}}%
\pgfpathlineto{\pgfqpoint{4.533511in}{2.874198in}}%
\pgfpathlineto{\pgfqpoint{4.485818in}{2.878688in}}%
\pgfpathlineto{\pgfqpoint{4.422813in}{2.882264in}}%
\pgfpathlineto{\pgfqpoint{4.327138in}{2.885335in}}%
\pgfpathlineto{\pgfqpoint{4.174880in}{2.887720in}}%
\pgfpathlineto{\pgfqpoint{3.916020in}{2.889337in}}%
\pgfpathlineto{\pgfqpoint{3.346084in}{2.890469in}}%
\pgfpathlineto{\pgfqpoint{2.112672in}{2.890354in}}%
\pgfpathlineto{\pgfqpoint{1.455724in}{2.888554in}}%
\pgfpathlineto{\pgfqpoint{1.183820in}{2.885794in}}%
\pgfpathlineto{\pgfqpoint{1.040282in}{2.882308in}}%
\pgfpathlineto{\pgfqpoint{0.944647in}{2.877896in}}%
\pgfpathlineto{\pgfqpoint{0.877372in}{2.872627in}}%
\pgfpathlineto{\pgfqpoint{0.827625in}{2.866558in}}%
\pgfpathlineto{\pgfqpoint{0.788941in}{2.859662in}}%
\pgfpathlineto{\pgfqpoint{0.759185in}{2.852269in}}%
\pgfpathlineto{\pgfqpoint{0.752871in}{2.850381in}}%
\pgfpathlineto{\pgfqpoint{0.752871in}{2.850381in}}%
\pgfusepath{stroke}%
\end{pgfscope}%
\begin{pgfscope}%
\pgfpathrectangle{\pgfqpoint{0.448634in}{0.402556in}}{\pgfqpoint{4.350661in}{2.489204in}} %
\pgfusepath{clip}%
\pgfsetrectcap%
\pgfsetroundjoin%
\pgfsetlinewidth{1.003750pt}%
\definecolor{currentstroke}{rgb}{0.498039,0.498039,0.498039}%
\pgfsetstrokecolor{currentstroke}%
\pgfsetdash{}{0pt}%
\pgfpathmoveto{\pgfqpoint{0.446233in}{0.402570in}}%
\pgfpathlineto{\pgfqpoint{0.609383in}{0.402561in}}%
\pgfpathlineto{\pgfqpoint{2.710739in}{0.403382in}}%
\pgfpathlineto{\pgfqpoint{2.716978in}{0.405311in}}%
\pgfpathlineto{\pgfqpoint{2.719216in}{0.409394in}}%
\pgfpathlineto{\pgfqpoint{2.720295in}{0.421765in}}%
\pgfpathlineto{\pgfqpoint{2.721229in}{0.474025in}}%
\pgfpathlineto{\pgfqpoint{2.726037in}{0.854829in}}%
\pgfpathlineto{\pgfqpoint{2.731718in}{1.071290in}}%
\pgfpathlineto{\pgfqpoint{2.738925in}{1.255304in}}%
\pgfpathlineto{\pgfqpoint{2.748895in}{1.446627in}}%
\pgfpathlineto{\pgfqpoint{2.759086in}{1.593024in}}%
\pgfpathlineto{\pgfqpoint{2.770425in}{1.724310in}}%
\pgfpathlineto{\pgfqpoint{2.782865in}{1.842939in}}%
\pgfpathlineto{\pgfqpoint{2.796431in}{1.951356in}}%
\pgfpathlineto{\pgfqpoint{2.811254in}{2.051991in}}%
\pgfpathlineto{\pgfqpoint{2.829082in}{2.157057in}}%
\pgfpathlineto{\pgfqpoint{2.847706in}{2.251765in}}%
\pgfpathlineto{\pgfqpoint{2.878932in}{2.396779in}}%
\pgfpathlineto{\pgfqpoint{2.891716in}{2.462370in}}%
\pgfpathlineto{\pgfqpoint{2.897349in}{2.501665in}}%
\pgfpathlineto{\pgfqpoint{2.899623in}{2.531413in}}%
\pgfpathlineto{\pgfqpoint{2.899456in}{2.556293in}}%
\pgfpathlineto{\pgfqpoint{2.897027in}{2.578507in}}%
\pgfpathlineto{\pgfqpoint{2.892533in}{2.597728in}}%
\pgfpathlineto{\pgfqpoint{2.886546in}{2.613733in}}%
\pgfpathlineto{\pgfqpoint{2.878630in}{2.628601in}}%
\pgfpathlineto{\pgfqpoint{2.868986in}{2.642068in}}%
\pgfpathlineto{\pgfqpoint{2.856271in}{2.655643in}}%
\pgfpathlineto{\pgfqpoint{2.842232in}{2.667396in}}%
\pgfpathlineto{\pgfqpoint{2.825353in}{2.678729in}}%
\pgfpathlineto{\pgfqpoint{2.803705in}{2.690376in}}%
\pgfpathlineto{\pgfqpoint{2.779279in}{2.700893in}}%
\pgfpathlineto{\pgfqpoint{2.750158in}{2.711075in}}%
\pgfpathlineto{\pgfqpoint{2.714245in}{2.721146in}}%
\pgfpathlineto{\pgfqpoint{2.671541in}{2.730645in}}%
\pgfpathlineto{\pgfqpoint{2.585885in}{2.746465in}}%
\pgfpathlineto{\pgfqpoint{2.520927in}{2.753623in}}%
\pgfpathlineto{\pgfqpoint{2.444980in}{2.759721in}}%
\pgfpathlineto{\pgfqpoint{2.358065in}{2.764424in}}%
\pgfpathlineto{\pgfqpoint{2.264561in}{2.767249in}}%
\pgfpathlineto{\pgfqpoint{2.160151in}{2.768323in}}%
\pgfpathlineto{\pgfqpoint{2.051390in}{2.767215in}}%
\pgfpathlineto{\pgfqpoint{1.938315in}{2.763787in}}%
\pgfpathlineto{\pgfqpoint{1.836186in}{2.758355in}}%
\pgfpathlineto{\pgfqpoint{1.727659in}{2.750226in}}%
\pgfpathlineto{\pgfqpoint{1.643181in}{2.741311in}}%
\pgfpathlineto{\pgfqpoint{1.550267in}{2.729093in}}%
\pgfpathlineto{\pgfqpoint{1.481437in}{2.717218in}}%
\pgfpathlineto{\pgfqpoint{1.419370in}{2.704325in}}%
\pgfpathlineto{\pgfqpoint{1.364114in}{2.690534in}}%
\pgfpathlineto{\pgfqpoint{1.313558in}{2.675642in}}%
\pgfpathlineto{\pgfqpoint{1.267739in}{2.659841in}}%
\pgfpathlineto{\pgfqpoint{1.226673in}{2.643415in}}%
\pgfpathlineto{\pgfqpoint{1.186320in}{2.624825in}}%
\pgfpathlineto{\pgfqpoint{1.150837in}{2.605891in}}%
\pgfpathlineto{\pgfqpoint{1.118217in}{2.585967in}}%
\pgfpathlineto{\pgfqpoint{1.088477in}{2.565287in}}%
\pgfpathlineto{\pgfqpoint{1.059857in}{2.542633in}}%
\pgfpathlineto{\pgfqpoint{1.034198in}{2.519577in}}%
\pgfpathlineto{\pgfqpoint{1.008160in}{2.493159in}}%
\pgfpathlineto{\pgfqpoint{0.985265in}{2.466563in}}%
\pgfpathlineto{\pgfqpoint{0.962499in}{2.436446in}}%
\pgfpathlineto{\pgfqpoint{0.941482in}{2.404709in}}%
\pgfpathlineto{\pgfqpoint{0.917253in}{2.363392in}}%
\pgfpathlineto{\pgfqpoint{0.899138in}{2.326507in}}%
\pgfpathlineto{\pgfqpoint{0.881970in}{2.286244in}}%
\pgfpathlineto{\pgfqpoint{0.865942in}{2.242657in}}%
\pgfpathlineto{\pgfqpoint{0.851193in}{2.195826in}}%
\pgfpathlineto{\pgfqpoint{0.837812in}{2.145851in}}%
\pgfpathlineto{\pgfqpoint{0.826510in}{2.095208in}}%
\pgfpathlineto{\pgfqpoint{0.815494in}{2.036817in}}%
\pgfpathlineto{\pgfqpoint{0.805728in}{1.973073in}}%
\pgfpathlineto{\pgfqpoint{0.798084in}{1.908956in}}%
\pgfpathlineto{\pgfqpoint{0.790914in}{1.832231in}}%
\pgfpathlineto{\pgfqpoint{0.785033in}{1.745372in}}%
\pgfpathlineto{\pgfqpoint{0.779119in}{1.621097in}}%
\pgfpathlineto{\pgfqpoint{0.777574in}{1.586293in}}%
\pgfpathlineto{\pgfqpoint{0.777574in}{1.586293in}}%
\pgfusepath{stroke}%
\end{pgfscope}%
\begin{pgfscope}%
\pgfpathrectangle{\pgfqpoint{0.448634in}{0.402556in}}{\pgfqpoint{4.350661in}{2.489204in}} %
\pgfusepath{clip}%
\pgfsetrectcap%
\pgfsetroundjoin%
\pgfsetlinewidth{1.003750pt}%
\definecolor{currentstroke}{rgb}{0.498039,0.498039,0.498039}%
\pgfsetstrokecolor{currentstroke}%
\pgfsetdash{}{0pt}%
\pgfpathmoveto{\pgfqpoint{1.286331in}{2.670050in}}%
\pgfpathlineto{\pgfqpoint{1.332434in}{2.684729in}}%
\pgfpathlineto{\pgfqpoint{1.383217in}{2.698574in}}%
\pgfpathlineto{\pgfqpoint{1.440790in}{2.711856in}}%
\pgfpathlineto{\pgfqpoint{1.502999in}{2.723828in}}%
\pgfpathlineto{\pgfqpoint{1.569786in}{2.734482in}}%
\pgfpathlineto{\pgfqpoint{1.608683in}{2.739606in}}%
\pgfpathlineto{\pgfqpoint{1.688760in}{2.748867in}}%
\pgfpathlineto{\pgfqpoint{1.766835in}{2.755771in}}%
\pgfpathlineto{\pgfqpoint{1.862375in}{2.762342in}}%
\pgfpathlineto{\pgfqpoint{1.966702in}{2.767218in}}%
\pgfpathlineto{\pgfqpoint{2.071090in}{2.769902in}}%
\pgfpathlineto{\pgfqpoint{2.182029in}{2.770630in}}%
\pgfpathlineto{\pgfqpoint{2.288611in}{2.769130in}}%
\pgfpathlineto{\pgfqpoint{2.390798in}{2.765471in}}%
\pgfpathlineto{\pgfqpoint{2.477685in}{2.760135in}}%
\pgfpathlineto{\pgfqpoint{2.553587in}{2.753339in}}%
\pgfpathlineto{\pgfqpoint{2.618470in}{2.745361in}}%
\pgfpathlineto{\pgfqpoint{2.641692in}{2.739672in}}%
\pgfpathlineto{\pgfqpoint{2.691083in}{2.730554in}}%
\pgfpathlineto{\pgfqpoint{2.733684in}{2.720477in}}%
\pgfpathlineto{\pgfqpoint{2.769442in}{2.709707in}}%
\pgfpathlineto{\pgfqpoint{2.800446in}{2.698102in}}%
\pgfpathlineto{\pgfqpoint{2.824464in}{2.686425in}}%
\pgfpathlineto{\pgfqpoint{2.843638in}{2.674686in}}%
\pgfpathlineto{\pgfqpoint{2.859837in}{2.662127in}}%
\pgfpathlineto{\pgfqpoint{2.872958in}{2.649066in}}%
\pgfpathlineto{\pgfqpoint{2.883042in}{2.636026in}}%
\pgfpathlineto{\pgfqpoint{2.891446in}{2.621514in}}%
\pgfpathlineto{\pgfqpoint{2.897907in}{2.605754in}}%
\pgfpathlineto{\pgfqpoint{2.902367in}{2.589107in}}%
\pgfpathlineto{\pgfqpoint{2.905218in}{2.569477in}}%
\pgfpathlineto{\pgfqpoint{2.906092in}{2.547107in}}%
\pgfpathlineto{\pgfqpoint{2.904691in}{2.519783in}}%
\pgfpathlineto{\pgfqpoint{2.900724in}{2.487750in}}%
\pgfpathlineto{\pgfqpoint{2.892858in}{2.443863in}}%
\pgfpathlineto{\pgfqpoint{2.877685in}{2.373798in}}%
\pgfpathlineto{\pgfqpoint{2.842232in}{2.212036in}}%
\pgfpathlineto{\pgfqpoint{2.824575in}{2.119632in}}%
\pgfpathlineto{\pgfqpoint{2.811278in}{2.038917in}}%
\pgfpathlineto{\pgfqpoint{2.796862in}{1.940728in}}%
\pgfpathlineto{\pgfqpoint{2.784197in}{1.839707in}}%
\pgfpathlineto{\pgfqpoint{2.771922in}{1.723563in}}%
\pgfpathlineto{\pgfqpoint{2.760891in}{1.597245in}}%
\pgfpathlineto{\pgfqpoint{2.751044in}{1.458307in}}%
\pgfpathlineto{\pgfqpoint{2.743652in}{1.324162in}}%
\pgfpathlineto{\pgfqpoint{2.736271in}{1.157601in}}%
\pgfpathlineto{\pgfqpoint{2.730620in}{0.978497in}}%
\pgfpathlineto{\pgfqpoint{2.726845in}{0.794349in}}%
\pgfpathlineto{\pgfqpoint{2.725216in}{0.620116in}}%
\pgfpathlineto{\pgfqpoint{2.726091in}{0.503132in}}%
\pgfpathlineto{\pgfqpoint{2.728441in}{0.450937in}}%
\pgfpathlineto{\pgfqpoint{2.731122in}{0.431278in}}%
\pgfpathlineto{\pgfqpoint{2.734871in}{0.419638in}}%
\pgfpathlineto{\pgfqpoint{2.738926in}{0.413827in}}%
\pgfpathlineto{\pgfqpoint{2.744423in}{0.409857in}}%
\pgfpathlineto{\pgfqpoint{2.752742in}{0.407010in}}%
\pgfpathlineto{\pgfqpoint{2.767851in}{0.404928in}}%
\pgfpathlineto{\pgfqpoint{2.798281in}{0.403605in}}%
\pgfpathlineto{\pgfqpoint{2.883115in}{0.402869in}}%
\pgfpathlineto{\pgfqpoint{3.331233in}{0.402602in}}%
\pgfpathlineto{\pgfqpoint{3.470454in}{0.402593in}}%
\pgfpathlineto{\pgfqpoint{3.470454in}{0.402593in}}%
\pgfusepath{stroke}%
\end{pgfscope}%
\begin{pgfscope}%
\pgfpathrectangle{\pgfqpoint{0.448634in}{0.402556in}}{\pgfqpoint{4.350661in}{2.489204in}} %
\pgfusepath{clip}%
\pgfsetrectcap%
\pgfsetroundjoin%
\pgfsetlinewidth{1.003750pt}%
\definecolor{currentstroke}{rgb}{0.737255,0.741176,0.133333}%
\pgfsetstrokecolor{currentstroke}%
\pgfsetdash{}{0pt}%
\pgfpathmoveto{\pgfqpoint{2.760564in}{1.998204in}}%
\pgfpathlineto{\pgfqpoint{2.748852in}{1.856954in}}%
\pgfpathlineto{\pgfqpoint{2.732829in}{1.656166in}}%
\pgfpathlineto{\pgfqpoint{2.720098in}{1.470046in}}%
\pgfpathlineto{\pgfqpoint{2.706984in}{1.246522in}}%
\pgfpathlineto{\pgfqpoint{2.682613in}{0.829275in}}%
\pgfpathlineto{\pgfqpoint{2.674860in}{0.740108in}}%
\pgfpathlineto{\pgfqpoint{2.667247in}{0.675983in}}%
\pgfpathlineto{\pgfqpoint{2.659252in}{0.627053in}}%
\pgfpathlineto{\pgfqpoint{2.651220in}{0.590872in}}%
\pgfpathlineto{\pgfqpoint{2.642877in}{0.562577in}}%
\pgfpathlineto{\pgfqpoint{2.634102in}{0.539811in}}%
\pgfpathlineto{\pgfqpoint{2.610597in}{0.489386in}}%
\pgfpathlineto{\pgfqpoint{2.600934in}{0.475941in}}%
\pgfpathlineto{\pgfqpoint{2.589705in}{0.464190in}}%
\pgfpathlineto{\pgfqpoint{2.575449in}{0.452794in}}%
\pgfpathlineto{\pgfqpoint{2.560145in}{0.443330in}}%
\pgfpathlineto{\pgfqpoint{2.542105in}{0.434649in}}%
\pgfpathlineto{\pgfqpoint{2.521374in}{0.427135in}}%
\pgfpathlineto{\pgfqpoint{2.498040in}{0.421100in}}%
\pgfpathlineto{\pgfqpoint{2.451043in}{0.411872in}}%
\pgfpathlineto{\pgfqpoint{2.411896in}{0.411995in}}%
\pgfpathlineto{\pgfqpoint{2.408162in}{0.409695in}}%
\pgfpathlineto{\pgfqpoint{2.407754in}{0.407302in}}%
\pgfpathlineto{\pgfqpoint{2.408679in}{0.405108in}}%
\pgfpathlineto{\pgfqpoint{2.412659in}{0.403295in}}%
\pgfpathlineto{\pgfqpoint{2.423510in}{0.402612in}}%
\pgfpathlineto{\pgfqpoint{2.538802in}{0.402556in}}%
\pgfpathlineto{\pgfqpoint{4.798970in}{0.402556in}}%
\pgfpathlineto{\pgfqpoint{4.798970in}{0.402556in}}%
\pgfusepath{stroke}%
\end{pgfscope}%
\begin{pgfscope}%
\pgfpathrectangle{\pgfqpoint{0.448634in}{0.402556in}}{\pgfqpoint{4.350661in}{2.489204in}} %
\pgfusepath{clip}%
\pgfsetrectcap%
\pgfsetroundjoin%
\pgfsetlinewidth{1.003750pt}%
\definecolor{currentstroke}{rgb}{0.737255,0.741176,0.133333}%
\pgfsetstrokecolor{currentstroke}%
\pgfsetdash{}{0pt}%
\pgfpathmoveto{\pgfqpoint{0.448634in}{2.896245in}}%
\pgfpathlineto{\pgfqpoint{0.448593in}{0.407043in}}%
\pgfpathlineto{\pgfqpoint{0.448593in}{0.407043in}}%
\pgfusepath{stroke}%
\end{pgfscope}%
\begin{pgfscope}%
\pgfpathrectangle{\pgfqpoint{0.448634in}{0.402556in}}{\pgfqpoint{4.350661in}{2.489204in}} %
\pgfusepath{clip}%
\pgfsetrectcap%
\pgfsetroundjoin%
\pgfsetlinewidth{1.003750pt}%
\definecolor{currentstroke}{rgb}{0.737255,0.741176,0.133333}%
\pgfsetstrokecolor{currentstroke}%
\pgfsetdash{}{0pt}%
\pgfpathmoveto{\pgfqpoint{2.767732in}{1.524058in}}%
\pgfpathlineto{\pgfqpoint{2.754332in}{1.367996in}}%
\pgfpathlineto{\pgfqpoint{2.747067in}{1.258788in}}%
\pgfpathlineto{\pgfqpoint{2.739027in}{1.104734in}}%
\pgfpathlineto{\pgfqpoint{2.732794in}{0.940603in}}%
\pgfpathlineto{\pgfqpoint{2.727528in}{0.721638in}}%
\pgfpathlineto{\pgfqpoint{2.726584in}{0.574781in}}%
\pgfpathlineto{\pgfqpoint{2.727765in}{0.470249in}}%
\pgfpathlineto{\pgfqpoint{2.730141in}{0.433021in}}%
\pgfpathlineto{\pgfqpoint{2.732987in}{0.418472in}}%
\pgfpathlineto{\pgfqpoint{2.736040in}{0.411910in}}%
\pgfpathlineto{\pgfqpoint{2.741163in}{0.407388in}}%
\pgfpathlineto{\pgfqpoint{2.747384in}{0.405203in}}%
\pgfpathlineto{\pgfqpoint{2.760365in}{0.403754in}}%
\pgfpathlineto{\pgfqpoint{2.797336in}{0.402938in}}%
\pgfpathlineto{\pgfqpoint{2.975713in}{0.402618in}}%
\pgfpathlineto{\pgfqpoint{4.800802in}{0.403090in}}%
\pgfpathlineto{\pgfqpoint{4.803619in}{0.405268in}}%
\pgfpathlineto{\pgfqpoint{4.800112in}{0.419545in}}%
\pgfpathlineto{\pgfqpoint{4.799469in}{0.441929in}}%
\pgfpathlineto{\pgfqpoint{4.799306in}{0.626130in}}%
\pgfpathlineto{\pgfqpoint{4.799296in}{1.357956in}}%
\pgfpathlineto{\pgfqpoint{4.799296in}{1.357956in}}%
\pgfusepath{stroke}%
\end{pgfscope}%
\begin{pgfscope}%
\pgfpathrectangle{\pgfqpoint{0.448634in}{0.402556in}}{\pgfqpoint{4.350661in}{2.489204in}} %
\pgfusepath{clip}%
\pgfsetrectcap%
\pgfsetroundjoin%
\pgfsetlinewidth{1.003750pt}%
\definecolor{currentstroke}{rgb}{0.737255,0.741176,0.133333}%
\pgfsetstrokecolor{currentstroke}%
\pgfsetdash{}{0pt}%
\pgfpathmoveto{\pgfqpoint{1.850605in}{2.888434in}}%
\pgfpathlineto{\pgfqpoint{1.687464in}{2.889386in}}%
\pgfpathlineto{\pgfqpoint{1.567823in}{2.888684in}}%
\pgfpathlineto{\pgfqpoint{1.376397in}{2.888021in}}%
\pgfpathlineto{\pgfqpoint{1.137127in}{2.884968in}}%
\pgfpathlineto{\pgfqpoint{1.008825in}{2.881259in}}%
\pgfpathlineto{\pgfqpoint{0.921909in}{2.876612in}}%
\pgfpathlineto{\pgfqpoint{0.861184in}{2.871242in}}%
\pgfpathlineto{\pgfqpoint{0.815818in}{2.865139in}}%
\pgfpathlineto{\pgfqpoint{0.779362in}{2.858070in}}%
\pgfpathlineto{\pgfqpoint{0.749728in}{2.850070in}}%
\pgfpathlineto{\pgfqpoint{0.724866in}{2.840992in}}%
\pgfpathlineto{\pgfqpoint{0.704818in}{2.831356in}}%
\pgfpathlineto{\pgfqpoint{0.687587in}{2.820740in}}%
\pgfpathlineto{\pgfqpoint{0.671466in}{2.808051in}}%
\pgfpathlineto{\pgfqpoint{0.656752in}{2.793292in}}%
\pgfpathlineto{\pgfqpoint{0.643675in}{2.776637in}}%
\pgfpathlineto{\pgfqpoint{0.632285in}{2.758429in}}%
\pgfpathlineto{\pgfqpoint{0.621474in}{2.736839in}}%
\pgfpathlineto{\pgfqpoint{0.611530in}{2.711943in}}%
\pgfpathlineto{\pgfqpoint{0.602695in}{2.683843in}}%
\pgfpathlineto{\pgfqpoint{0.594288in}{2.650354in}}%
\pgfpathlineto{\pgfqpoint{0.586109in}{2.609091in}}%
\pgfpathlineto{\pgfqpoint{0.578701in}{2.560039in}}%
\pgfpathlineto{\pgfqpoint{0.572038in}{2.500790in}}%
\pgfpathlineto{\pgfqpoint{0.566507in}{2.431384in}}%
\pgfpathlineto{\pgfqpoint{0.562172in}{2.346900in}}%
\pgfpathlineto{\pgfqpoint{0.559624in}{2.252358in}}%
\pgfpathlineto{\pgfqpoint{0.558959in}{2.145328in}}%
\pgfpathlineto{\pgfqpoint{0.560508in}{2.040799in}}%
\pgfpathlineto{\pgfqpoint{0.564293in}{1.936344in}}%
\pgfpathlineto{\pgfqpoint{0.569971in}{1.839486in}}%
\pgfpathlineto{\pgfqpoint{0.577428in}{1.750285in}}%
\pgfpathlineto{\pgfqpoint{0.585985in}{1.673747in}}%
\pgfpathlineto{\pgfqpoint{0.596111in}{1.602500in}}%
\pgfpathlineto{\pgfqpoint{0.606842in}{1.544040in}}%
\pgfpathlineto{\pgfqpoint{0.617774in}{1.495859in}}%
\pgfpathlineto{\pgfqpoint{0.630295in}{1.450795in}}%
\pgfpathlineto{\pgfqpoint{0.642982in}{1.413721in}}%
\pgfpathlineto{\pgfqpoint{0.656458in}{1.382482in}}%
\pgfpathlineto{\pgfqpoint{0.667847in}{1.361285in}}%
\pgfpathlineto{\pgfqpoint{0.679867in}{1.343621in}}%
\pgfpathlineto{\pgfqpoint{0.690832in}{1.331558in}}%
\pgfpathlineto{\pgfqpoint{0.699831in}{1.324598in}}%
\pgfpathlineto{\pgfqpoint{0.707845in}{1.320770in}}%
\pgfpathlineto{\pgfqpoint{0.716433in}{1.319414in}}%
\pgfpathlineto{\pgfqpoint{0.722870in}{1.320510in}}%
\pgfpathlineto{\pgfqpoint{0.728865in}{1.323413in}}%
\pgfpathlineto{\pgfqpoint{0.735780in}{1.329415in}}%
\pgfpathlineto{\pgfqpoint{0.742744in}{1.338948in}}%
\pgfpathlineto{\pgfqpoint{0.749272in}{1.351865in}}%
\pgfpathlineto{\pgfqpoint{0.755935in}{1.370249in}}%
\pgfpathlineto{\pgfqpoint{0.762167in}{1.394089in}}%
\pgfpathlineto{\pgfqpoint{0.767999in}{1.425745in}}%
\pgfpathlineto{\pgfqpoint{0.772792in}{1.465186in}}%
\pgfpathlineto{\pgfqpoint{0.777040in}{1.527218in}}%
\pgfpathlineto{\pgfqpoint{0.779619in}{1.606816in}}%
\pgfpathlineto{\pgfqpoint{0.784699in}{1.773484in}}%
\pgfpathlineto{\pgfqpoint{0.789724in}{1.852926in}}%
\pgfpathlineto{\pgfqpoint{0.796456in}{1.922192in}}%
\pgfpathlineto{\pgfqpoint{0.805477in}{1.991117in}}%
\pgfpathlineto{\pgfqpoint{0.816051in}{2.052155in}}%
\pgfpathlineto{\pgfqpoint{0.828430in}{2.110186in}}%
\pgfpathlineto{\pgfqpoint{0.844196in}{2.172331in}}%
\pgfpathlineto{\pgfqpoint{0.862001in}{2.231081in}}%
\pgfpathlineto{\pgfqpoint{0.878947in}{2.276927in}}%
\pgfpathlineto{\pgfqpoint{0.897043in}{2.319441in}}%
\pgfpathlineto{\pgfqpoint{0.916186in}{2.358519in}}%
\pgfpathlineto{\pgfqpoint{0.936163in}{2.394122in}}%
\pgfpathlineto{\pgfqpoint{0.956701in}{2.426269in}}%
\pgfpathlineto{\pgfqpoint{0.979001in}{2.456839in}}%
\pgfpathlineto{\pgfqpoint{1.001473in}{2.483902in}}%
\pgfpathlineto{\pgfqpoint{1.025394in}{2.509287in}}%
\pgfpathlineto{\pgfqpoint{1.050711in}{2.532834in}}%
\pgfpathlineto{\pgfqpoint{1.077185in}{2.554651in}}%
\pgfpathlineto{\pgfqpoint{1.106587in}{2.575953in}}%
\pgfpathlineto{\pgfqpoint{1.137007in}{2.595296in}}%
\pgfpathlineto{\pgfqpoint{1.170244in}{2.613839in}}%
\pgfpathlineto{\pgfqpoint{1.206265in}{2.631394in}}%
\pgfpathlineto{\pgfqpoint{1.251250in}{2.650066in}}%
\pgfpathlineto{\pgfqpoint{1.294813in}{2.665791in}}%
\pgfpathlineto{\pgfqpoint{1.340987in}{2.680172in}}%
\pgfpathlineto{\pgfqpoint{1.387584in}{2.692636in}}%
\pgfpathlineto{\pgfqpoint{1.406782in}{2.696999in}}%
\pgfpathlineto{\pgfqpoint{1.466618in}{2.710011in}}%
\pgfpathlineto{\pgfqpoint{1.531066in}{2.721737in}}%
\pgfpathlineto{\pgfqpoint{1.585014in}{2.729522in}}%
\pgfpathlineto{\pgfqpoint{1.632571in}{2.735604in}}%
\pgfpathlineto{\pgfqpoint{1.684510in}{2.741418in}}%
\pgfpathlineto{\pgfqpoint{1.779883in}{2.750206in}}%
\pgfpathlineto{\pgfqpoint{1.823324in}{2.752817in}}%
\pgfpathlineto{\pgfqpoint{1.858018in}{2.755209in}}%
\pgfpathlineto{\pgfqpoint{1.890615in}{2.756836in}}%
\pgfpathlineto{\pgfqpoint{1.923142in}{2.759532in}}%
\pgfpathlineto{\pgfqpoint{1.936166in}{2.760303in}}%
\pgfpathlineto{\pgfqpoint{1.986189in}{2.760444in}}%
\pgfpathlineto{\pgfqpoint{2.131799in}{2.765271in}}%
\pgfpathlineto{\pgfqpoint{2.203582in}{2.764761in}}%
\pgfpathlineto{\pgfqpoint{2.347139in}{2.763952in}}%
\pgfpathlineto{\pgfqpoint{2.373177in}{2.762232in}}%
\pgfpathlineto{\pgfqpoint{2.388267in}{2.759977in}}%
\pgfpathlineto{\pgfqpoint{2.420868in}{2.758530in}}%
\pgfpathlineto{\pgfqpoint{2.503342in}{2.752362in}}%
\pgfpathlineto{\pgfqpoint{2.553158in}{2.747551in}}%
\pgfpathlineto{\pgfqpoint{2.667724in}{2.733931in}}%
\pgfpathlineto{\pgfqpoint{2.725663in}{2.723142in}}%
\pgfpathlineto{\pgfqpoint{2.763846in}{2.713239in}}%
\pgfpathlineto{\pgfqpoint{2.795197in}{2.702912in}}%
\pgfpathlineto{\pgfqpoint{2.821756in}{2.691823in}}%
\pgfpathlineto{\pgfqpoint{2.843456in}{2.680306in}}%
\pgfpathlineto{\pgfqpoint{2.860323in}{2.668952in}}%
\pgfpathlineto{\pgfqpoint{2.874245in}{2.657023in}}%
\pgfpathlineto{\pgfqpoint{2.885192in}{2.644928in}}%
\pgfpathlineto{\pgfqpoint{2.894533in}{2.631189in}}%
\pgfpathlineto{\pgfqpoint{2.901855in}{2.615933in}}%
\pgfpathlineto{\pgfqpoint{2.906919in}{2.599520in}}%
\pgfpathlineto{\pgfqpoint{2.909786in}{2.582421in}}%
\pgfpathlineto{\pgfqpoint{2.910759in}{2.562552in}}%
\pgfpathlineto{\pgfqpoint{2.909497in}{2.537717in}}%
\pgfpathlineto{\pgfqpoint{2.905338in}{2.508237in}}%
\pgfpathlineto{\pgfqpoint{2.897461in}{2.469449in}}%
\pgfpathlineto{\pgfqpoint{2.877191in}{2.385475in}}%
\pgfpathlineto{\pgfqpoint{2.866413in}{2.337311in}}%
\pgfpathlineto{\pgfqpoint{2.847474in}{2.252932in}}%
\pgfpathlineto{\pgfqpoint{2.830144in}{2.162995in}}%
\pgfpathlineto{\pgfqpoint{2.818683in}{2.092041in}}%
\pgfpathlineto{\pgfqpoint{2.815315in}{2.069977in}}%
\pgfpathlineto{\pgfqpoint{2.799017in}{1.957007in}}%
\pgfpathlineto{\pgfqpoint{2.780549in}{1.801625in}}%
\pgfpathlineto{\pgfqpoint{2.769555in}{1.687829in}}%
\pgfpathlineto{\pgfqpoint{2.762558in}{1.606080in}}%
\pgfpathlineto{\pgfqpoint{2.760666in}{1.581287in}}%
\pgfpathlineto{\pgfqpoint{2.751438in}{1.454779in}}%
\pgfpathlineto{\pgfqpoint{2.741244in}{1.275941in}}%
\pgfpathlineto{\pgfqpoint{2.733465in}{1.099435in}}%
\pgfpathlineto{\pgfqpoint{2.725633in}{0.815813in}}%
\pgfpathlineto{\pgfqpoint{2.724260in}{0.718749in}}%
\pgfpathlineto{\pgfqpoint{2.722967in}{0.634174in}}%
\pgfpathlineto{\pgfqpoint{2.720338in}{0.627500in}}%
\pgfpathlineto{\pgfqpoint{2.719019in}{0.520480in}}%
\pgfpathlineto{\pgfqpoint{2.719789in}{0.408484in}}%
\pgfpathlineto{\pgfqpoint{2.715236in}{0.404114in}}%
\pgfpathlineto{\pgfqpoint{2.708802in}{0.402989in}}%
\pgfpathlineto{\pgfqpoint{2.684878in}{0.402617in}}%
\pgfpathlineto{\pgfqpoint{2.467351in}{0.403009in}}%
\pgfpathlineto{\pgfqpoint{2.458786in}{0.404583in}}%
\pgfpathlineto{\pgfqpoint{2.455184in}{0.407266in}}%
\pgfpathlineto{\pgfqpoint{2.451757in}{0.410225in}}%
\pgfpathlineto{\pgfqpoint{2.445375in}{0.411652in}}%
\pgfpathlineto{\pgfqpoint{2.425805in}{0.412083in}}%
\pgfpathlineto{\pgfqpoint{2.410643in}{0.411546in}}%
\pgfpathlineto{\pgfqpoint{2.408765in}{0.410329in}}%
\pgfpathlineto{\pgfqpoint{2.407873in}{0.408128in}}%
\pgfpathlineto{\pgfqpoint{2.408359in}{0.405747in}}%
\pgfpathlineto{\pgfqpoint{2.412085in}{0.403450in}}%
\pgfpathlineto{\pgfqpoint{2.422922in}{0.402623in}}%
\pgfpathlineto{\pgfqpoint{2.520812in}{0.402556in}}%
\pgfpathlineto{\pgfqpoint{4.798383in}{0.402556in}}%
\pgfpathlineto{\pgfqpoint{4.798383in}{0.402556in}}%
\pgfusepath{stroke}%
\end{pgfscope}%
\begin{pgfscope}%
\pgfpathrectangle{\pgfqpoint{0.448634in}{0.402556in}}{\pgfqpoint{4.350661in}{2.489204in}} %
\pgfusepath{clip}%
\pgfsetrectcap%
\pgfsetroundjoin%
\pgfsetlinewidth{1.003750pt}%
\definecolor{currentstroke}{rgb}{0.737255,0.741176,0.133333}%
\pgfsetstrokecolor{currentstroke}%
\pgfsetdash{}{0pt}%
\pgfpathmoveto{\pgfqpoint{1.221118in}{2.807301in}}%
\pgfpathlineto{\pgfqpoint{1.307597in}{2.818295in}}%
\pgfpathlineto{\pgfqpoint{1.402923in}{2.828132in}}%
\pgfpathlineto{\pgfqpoint{1.511411in}{2.837023in}}%
\pgfpathlineto{\pgfqpoint{1.546179in}{2.838520in}}%
\pgfpathlineto{\pgfqpoint{1.574426in}{2.839659in}}%
\pgfpathlineto{\pgfqpoint{1.600350in}{2.842822in}}%
\pgfpathlineto{\pgfqpoint{1.680713in}{2.846652in}}%
\pgfpathlineto{\pgfqpoint{1.706794in}{2.847788in}}%
\pgfpathlineto{\pgfqpoint{1.785057in}{2.850358in}}%
\pgfpathlineto{\pgfqpoint{1.817636in}{2.852275in}}%
\pgfpathlineto{\pgfqpoint{2.263264in}{2.863959in}}%
\pgfpathlineto{\pgfqpoint{2.265844in}{2.867762in}}%
\pgfpathlineto{\pgfqpoint{2.269184in}{2.879310in}}%
\pgfpathlineto{\pgfqpoint{2.273002in}{2.881626in}}%
\pgfpathlineto{\pgfqpoint{2.283610in}{2.884319in}}%
\pgfpathlineto{\pgfqpoint{2.300934in}{2.886117in}}%
\pgfpathlineto{\pgfqpoint{2.337905in}{2.887015in}}%
\pgfpathlineto{\pgfqpoint{2.646782in}{2.890780in}}%
\pgfpathlineto{\pgfqpoint{3.719219in}{2.889994in}}%
\pgfpathlineto{\pgfqpoint{4.154281in}{2.887950in}}%
\pgfpathlineto{\pgfqpoint{4.345691in}{2.885009in}}%
\pgfpathlineto{\pgfqpoint{4.447876in}{2.881286in}}%
\pgfpathlineto{\pgfqpoint{4.508660in}{2.876876in}}%
\pgfpathlineto{\pgfqpoint{4.536741in}{2.873343in}}%
\pgfpathlineto{\pgfqpoint{4.542821in}{2.870788in}}%
\pgfpathlineto{\pgfqpoint{4.575000in}{2.864642in}}%
\pgfpathlineto{\pgfqpoint{4.600359in}{2.857594in}}%
\pgfpathlineto{\pgfqpoint{4.620944in}{2.849576in}}%
\pgfpathlineto{\pgfqpoint{4.636742in}{2.841247in}}%
\pgfpathlineto{\pgfqpoint{4.651594in}{2.830891in}}%
\pgfpathlineto{\pgfqpoint{4.663456in}{2.819986in}}%
\pgfpathlineto{\pgfqpoint{4.674002in}{2.807432in}}%
\pgfpathlineto{\pgfqpoint{4.684308in}{2.791405in}}%
\pgfpathlineto{\pgfqpoint{4.693879in}{2.771876in}}%
\pgfpathlineto{\pgfqpoint{4.702392in}{2.748981in}}%
\pgfpathlineto{\pgfqpoint{4.710369in}{2.720548in}}%
\pgfpathlineto{\pgfqpoint{4.717566in}{2.686692in}}%
\pgfpathlineto{\pgfqpoint{4.724551in}{2.642611in}}%
\pgfpathlineto{\pgfqpoint{4.731092in}{2.585855in}}%
\pgfpathlineto{\pgfqpoint{4.736963in}{2.513984in}}%
\pgfpathlineto{\pgfqpoint{4.741987in}{2.424560in}}%
\pgfpathlineto{\pgfqpoint{4.745911in}{2.312638in}}%
\pgfpathlineto{\pgfqpoint{4.748604in}{2.178258in}}%
\pgfpathlineto{\pgfqpoint{4.749771in}{2.016467in}}%
\pgfpathlineto{\pgfqpoint{4.748810in}{1.844717in}}%
\pgfpathlineto{\pgfqpoint{4.745670in}{1.677981in}}%
\pgfpathlineto{\pgfqpoint{4.740773in}{1.533718in}}%
\pgfpathlineto{\pgfqpoint{4.734285in}{1.406988in}}%
\pgfpathlineto{\pgfqpoint{4.726413in}{1.297838in}}%
\pgfpathlineto{\pgfqpoint{4.717665in}{1.208791in}}%
\pgfpathlineto{\pgfqpoint{4.708213in}{1.134906in}}%
\pgfpathlineto{\pgfqpoint{4.697828in}{1.071292in}}%
\pgfpathlineto{\pgfqpoint{4.686919in}{1.017977in}}%
\pgfpathlineto{\pgfqpoint{4.673745in}{0.965366in}}%
\pgfpathlineto{\pgfqpoint{4.661754in}{0.927984in}}%
\pgfpathlineto{\pgfqpoint{4.648513in}{0.893867in}}%
\pgfpathlineto{\pgfqpoint{4.634150in}{0.863146in}}%
\pgfpathlineto{\pgfqpoint{4.620161in}{0.837935in}}%
\pgfpathlineto{\pgfqpoint{4.604554in}{0.814001in}}%
\pgfpathlineto{\pgfqpoint{4.587335in}{0.791562in}}%
\pgfpathlineto{\pgfqpoint{4.568596in}{0.770778in}}%
\pgfpathlineto{\pgfqpoint{4.548478in}{0.751756in}}%
\pgfpathlineto{\pgfqpoint{4.525392in}{0.733079in}}%
\pgfpathlineto{\pgfqpoint{4.501186in}{0.716361in}}%
\pgfpathlineto{\pgfqpoint{4.481755in}{0.705214in}}%
\pgfpathlineto{\pgfqpoint{4.455917in}{0.692083in}}%
\pgfpathlineto{\pgfqpoint{4.425604in}{0.678279in}}%
\pgfpathlineto{\pgfqpoint{4.382195in}{0.662030in}}%
\pgfpathlineto{\pgfqpoint{4.342290in}{0.649731in}}%
\pgfpathlineto{\pgfqpoint{4.299809in}{0.639001in}}%
\pgfpathlineto{\pgfqpoint{4.252694in}{0.629410in}}%
\pgfpathlineto{\pgfqpoint{4.200985in}{0.621199in}}%
\pgfpathlineto{\pgfqpoint{4.144728in}{0.614559in}}%
\pgfpathlineto{\pgfqpoint{4.083970in}{0.609688in}}%
\pgfpathlineto{\pgfqpoint{4.023110in}{0.606969in}}%
\pgfpathlineto{\pgfqpoint{3.955682in}{0.606044in}}%
\pgfpathlineto{\pgfqpoint{3.888258in}{0.607306in}}%
\pgfpathlineto{\pgfqpoint{3.818718in}{0.610821in}}%
\pgfpathlineto{\pgfqpoint{3.751470in}{0.616528in}}%
\pgfpathlineto{\pgfqpoint{3.682244in}{0.624766in}}%
\pgfpathlineto{\pgfqpoint{3.624137in}{0.634536in}}%
\pgfpathlineto{\pgfqpoint{3.568488in}{0.646084in}}%
\pgfpathlineto{\pgfqpoint{3.517507in}{0.658937in}}%
\pgfpathlineto{\pgfqpoint{3.471239in}{0.672918in}}%
\pgfpathlineto{\pgfqpoint{3.429710in}{0.687741in}}%
\pgfpathlineto{\pgfqpoint{3.390898in}{0.703986in}}%
\pgfpathlineto{\pgfqpoint{3.356901in}{0.720626in}}%
\pgfpathlineto{\pgfqpoint{3.329517in}{0.735858in}}%
\pgfpathlineto{\pgfqpoint{3.310394in}{0.747709in}}%
\pgfpathlineto{\pgfqpoint{3.282679in}{0.767407in}}%
\pgfpathlineto{\pgfqpoint{3.263216in}{0.783328in}}%
\pgfpathlineto{\pgfqpoint{3.239636in}{0.805367in}}%
\pgfpathlineto{\pgfqpoint{3.217692in}{0.829520in}}%
\pgfpathlineto{\pgfqpoint{3.198893in}{0.853686in}}%
\pgfpathlineto{\pgfqpoint{3.181567in}{0.879249in}}%
\pgfpathlineto{\pgfqpoint{3.167046in}{0.904063in}}%
\pgfpathlineto{\pgfqpoint{3.149790in}{0.938641in}}%
\pgfpathlineto{\pgfqpoint{3.135674in}{0.972297in}}%
\pgfpathlineto{\pgfqpoint{3.122662in}{1.009228in}}%
\pgfpathlineto{\pgfqpoint{3.109162in}{1.056545in}}%
\pgfpathlineto{\pgfqpoint{3.097807in}{1.107168in}}%
\pgfpathlineto{\pgfqpoint{3.090387in}{1.148613in}}%
\pgfpathlineto{\pgfqpoint{3.081757in}{1.215077in}}%
\pgfpathlineto{\pgfqpoint{3.075903in}{1.294429in}}%
\pgfpathlineto{\pgfqpoint{3.073986in}{1.344155in}}%
\pgfpathlineto{\pgfqpoint{3.074419in}{1.403890in}}%
\pgfpathlineto{\pgfqpoint{3.077198in}{1.475990in}}%
\pgfpathlineto{\pgfqpoint{3.084667in}{1.557675in}}%
\pgfpathlineto{\pgfqpoint{3.092272in}{1.611736in}}%
\pgfpathlineto{\pgfqpoint{3.099376in}{1.653259in}}%
\pgfpathlineto{\pgfqpoint{3.112301in}{1.713696in}}%
\pgfpathlineto{\pgfqpoint{3.127411in}{1.770871in}}%
\pgfpathlineto{\pgfqpoint{3.141309in}{1.815408in}}%
\pgfpathlineto{\pgfqpoint{3.157877in}{1.861435in}}%
\pgfpathlineto{\pgfqpoint{3.173474in}{1.899798in}}%
\pgfpathlineto{\pgfqpoint{3.193784in}{1.943820in}}%
\pgfpathlineto{\pgfqpoint{3.212595in}{1.980249in}}%
\pgfpathlineto{\pgfqpoint{3.236582in}{2.021775in}}%
\pgfpathlineto{\pgfqpoint{3.258458in}{2.055889in}}%
\pgfpathlineto{\pgfqpoint{3.282967in}{2.090827in}}%
\pgfpathlineto{\pgfqpoint{3.321957in}{2.141079in}}%
\pgfpathlineto{\pgfqpoint{3.367046in}{2.198435in}}%
\pgfpathlineto{\pgfqpoint{3.375753in}{2.212712in}}%
\pgfpathlineto{\pgfqpoint{3.379448in}{2.221708in}}%
\pgfpathlineto{\pgfqpoint{3.380496in}{2.229042in}}%
\pgfpathlineto{\pgfqpoint{3.379570in}{2.233874in}}%
\pgfpathlineto{\pgfqpoint{3.377088in}{2.237924in}}%
\pgfpathlineto{\pgfqpoint{3.371550in}{2.241791in}}%
\pgfpathlineto{\pgfqpoint{3.363104in}{2.244075in}}%
\pgfpathlineto{\pgfqpoint{3.352246in}{2.244541in}}%
\pgfpathlineto{\pgfqpoint{3.337113in}{2.242722in}}%
\pgfpathlineto{\pgfqpoint{3.318022in}{2.237807in}}%
\pgfpathlineto{\pgfqpoint{3.297377in}{2.229989in}}%
\pgfpathlineto{\pgfqpoint{3.275389in}{2.219205in}}%
\pgfpathlineto{\pgfqpoint{3.252301in}{2.205285in}}%
\pgfpathlineto{\pgfqpoint{3.230185in}{2.189428in}}%
\pgfpathlineto{\pgfqpoint{3.207349in}{2.170353in}}%
\pgfpathlineto{\pgfqpoint{3.184081in}{2.147880in}}%
\pgfpathlineto{\pgfqpoint{3.162264in}{2.123574in}}%
\pgfpathlineto{\pgfqpoint{3.140414in}{2.095850in}}%
\pgfpathlineto{\pgfqpoint{3.120053in}{2.066682in}}%
\pgfpathlineto{\pgfqpoint{3.101212in}{2.036204in}}%
\pgfpathlineto{\pgfqpoint{3.081569in}{2.000359in}}%
\pgfpathlineto{\pgfqpoint{3.061643in}{1.958932in}}%
\pgfpathlineto{\pgfqpoint{3.042839in}{1.914046in}}%
\pgfpathlineto{\pgfqpoint{3.024272in}{1.863581in}}%
\pgfpathlineto{\pgfqpoint{3.009267in}{1.816856in}}%
\pgfpathlineto{\pgfqpoint{2.992168in}{1.755172in}}%
\pgfpathlineto{\pgfqpoint{2.976549in}{1.687813in}}%
\pgfpathlineto{\pgfqpoint{2.965958in}{1.634412in}}%
\pgfpathlineto{\pgfqpoint{2.953938in}{1.561021in}}%
\pgfpathlineto{\pgfqpoint{2.947306in}{1.511822in}}%
\pgfpathlineto{\pgfqpoint{2.936725in}{1.407994in}}%
\pgfpathlineto{\pgfqpoint{2.933692in}{1.363328in}}%
\pgfpathlineto{\pgfqpoint{2.929889in}{1.276332in}}%
\pgfpathlineto{\pgfqpoint{2.929722in}{1.194192in}}%
\pgfpathlineto{\pgfqpoint{2.932259in}{1.107119in}}%
\pgfpathlineto{\pgfqpoint{2.935055in}{1.064930in}}%
\pgfpathlineto{\pgfqpoint{2.941718in}{0.998165in}}%
\pgfpathlineto{\pgfqpoint{2.950614in}{0.939305in}}%
\pgfpathlineto{\pgfqpoint{2.960461in}{0.890820in}}%
\pgfpathlineto{\pgfqpoint{2.971526in}{0.847848in}}%
\pgfpathlineto{\pgfqpoint{2.984251in}{0.808124in}}%
\pgfpathlineto{\pgfqpoint{2.997558in}{0.774039in}}%
\pgfpathlineto{\pgfqpoint{3.010840in}{0.745479in}}%
\pgfpathlineto{\pgfqpoint{3.027137in}{0.716050in}}%
\pgfpathlineto{\pgfqpoint{3.042805in}{0.692168in}}%
\pgfpathlineto{\pgfqpoint{3.060044in}{0.669748in}}%
\pgfpathlineto{\pgfqpoint{3.078805in}{0.648990in}}%
\pgfpathlineto{\pgfqpoint{3.098953in}{0.630011in}}%
\pgfpathlineto{\pgfqpoint{3.120318in}{0.612862in}}%
\pgfpathlineto{\pgfqpoint{3.144609in}{0.596306in}}%
\pgfpathlineto{\pgfqpoint{3.171818in}{0.580668in}}%
\pgfpathlineto{\pgfqpoint{3.201885in}{0.566179in}}%
\pgfpathlineto{\pgfqpoint{3.234707in}{0.552940in}}%
\pgfpathlineto{\pgfqpoint{3.270167in}{0.540947in}}%
\pgfpathlineto{\pgfqpoint{3.312412in}{0.529060in}}%
\pgfpathlineto{\pgfqpoint{3.361448in}{0.517711in}}%
\pgfpathlineto{\pgfqpoint{3.419399in}{0.506790in}}%
\pgfpathlineto{\pgfqpoint{3.490555in}{0.495953in}}%
\pgfpathlineto{\pgfqpoint{3.587873in}{0.483888in}}%
\pgfpathlineto{\pgfqpoint{3.728653in}{0.469010in}}%
\pgfpathlineto{\pgfqpoint{3.843900in}{0.465347in}}%
\pgfpathlineto{\pgfqpoint{3.967883in}{0.463657in}}%
\pgfpathlineto{\pgfqpoint{4.091875in}{0.464171in}}%
\pgfpathlineto{\pgfqpoint{4.207143in}{0.466833in}}%
\pgfpathlineto{\pgfqpoint{4.298438in}{0.470851in}}%
\pgfpathlineto{\pgfqpoint{4.298438in}{0.470851in}}%
\pgfusepath{stroke}%
\end{pgfscope}%
\begin{pgfscope}%
\pgfpathrectangle{\pgfqpoint{0.448634in}{0.402556in}}{\pgfqpoint{4.350661in}{2.489204in}} %
\pgfusepath{clip}%
\pgfsetrectcap%
\pgfsetroundjoin%
\pgfsetlinewidth{1.003750pt}%
\definecolor{currentstroke}{rgb}{0.737255,0.741176,0.133333}%
\pgfsetstrokecolor{currentstroke}%
\pgfsetdash{}{0pt}%
\pgfpathmoveto{\pgfqpoint{4.802747in}{0.402556in}}%
\pgfpathlineto{\pgfqpoint{2.412098in}{0.403450in}}%
\pgfpathlineto{\pgfqpoint{2.408373in}{0.405747in}}%
\pgfpathlineto{\pgfqpoint{2.407886in}{0.408128in}}%
\pgfpathlineto{\pgfqpoint{2.408778in}{0.410329in}}%
\pgfpathlineto{\pgfqpoint{2.412766in}{0.412078in}}%
\pgfpathlineto{\pgfqpoint{2.447551in}{0.411385in}}%
\pgfpathlineto{\pgfqpoint{2.453669in}{0.409026in}}%
\pgfpathlineto{\pgfqpoint{2.458799in}{0.404581in}}%
\pgfpathlineto{\pgfqpoint{2.465195in}{0.403196in}}%
\pgfpathlineto{\pgfqpoint{2.484761in}{0.402605in}}%
\pgfpathlineto{\pgfqpoint{2.702294in}{0.402733in}}%
\pgfpathlineto{\pgfqpoint{2.715249in}{0.404121in}}%
\pgfpathlineto{\pgfqpoint{2.718564in}{0.407154in}}%
\pgfpathlineto{\pgfqpoint{2.719790in}{0.408502in}}%
\pgfpathlineto{\pgfqpoint{2.720128in}{0.435876in}}%
\pgfpathlineto{\pgfqpoint{2.720339in}{0.627518in}}%
\pgfpathlineto{\pgfqpoint{2.721855in}{0.632077in}}%
\pgfpathlineto{\pgfqpoint{2.723365in}{0.636631in}}%
\pgfpathlineto{\pgfqpoint{2.729147in}{0.967619in}}%
\pgfpathlineto{\pgfqpoint{2.734450in}{1.124320in}}%
\pgfpathlineto{\pgfqpoint{2.746657in}{1.377827in}}%
\pgfpathlineto{\pgfqpoint{2.753369in}{1.482090in}}%
\pgfpathlineto{\pgfqpoint{2.767376in}{1.663082in}}%
\pgfpathlineto{\pgfqpoint{2.783492in}{1.828817in}}%
\pgfpathlineto{\pgfqpoint{2.799675in}{1.961947in}}%
\pgfpathlineto{\pgfqpoint{2.821772in}{2.114161in}}%
\pgfpathlineto{\pgfqpoint{2.839670in}{2.214138in}}%
\pgfpathlineto{\pgfqpoint{2.856902in}{2.296438in}}%
\pgfpathlineto{\pgfqpoint{2.886093in}{2.421416in}}%
\pgfpathlineto{\pgfqpoint{2.901635in}{2.488799in}}%
\pgfpathlineto{\pgfqpoint{2.908033in}{2.525404in}}%
\pgfpathlineto{\pgfqpoint{2.910560in}{2.552617in}}%
\pgfpathlineto{\pgfqpoint{2.910239in}{2.577489in}}%
\pgfpathlineto{\pgfqpoint{2.907948in}{2.594702in}}%
\pgfpathlineto{\pgfqpoint{2.903529in}{2.611358in}}%
\pgfpathlineto{\pgfqpoint{2.896840in}{2.626990in}}%
\pgfpathlineto{\pgfqpoint{2.888035in}{2.641184in}}%
\pgfpathlineto{\pgfqpoint{2.877497in}{2.653744in}}%
\pgfpathlineto{\pgfqpoint{2.863904in}{2.666156in}}%
\pgfpathlineto{\pgfqpoint{2.847275in}{2.677958in}}%
\pgfpathlineto{\pgfqpoint{2.827748in}{2.688910in}}%
\pgfpathlineto{\pgfqpoint{2.803431in}{2.699751in}}%
\pgfpathlineto{\pgfqpoint{2.774341in}{2.710046in}}%
\pgfpathlineto{\pgfqpoint{2.740549in}{2.719564in}}%
\pgfpathlineto{\pgfqpoint{2.699987in}{2.728625in}}%
\pgfpathlineto{\pgfqpoint{2.665556in}{2.734294in}}%
\pgfpathlineto{\pgfqpoint{2.626695in}{2.739601in}}%
\pgfpathlineto{\pgfqpoint{2.600691in}{2.742007in}}%
\pgfpathlineto{\pgfqpoint{2.514190in}{2.751833in}}%
\pgfpathlineto{\pgfqpoint{2.405635in}{2.759117in}}%
\pgfpathlineto{\pgfqpoint{2.381754in}{2.760684in}}%
\pgfpathlineto{\pgfqpoint{2.353647in}{2.763781in}}%
\pgfpathlineto{\pgfqpoint{2.255772in}{2.765275in}}%
\pgfpathlineto{\pgfqpoint{2.201397in}{2.765341in}}%
\pgfpathlineto{\pgfqpoint{2.190640in}{2.767050in}}%
\pgfpathlineto{\pgfqpoint{2.177792in}{2.769611in}}%
\pgfpathlineto{\pgfqpoint{2.158256in}{2.770973in}}%
\pgfpathlineto{\pgfqpoint{2.114752in}{2.771244in}}%
\pgfpathlineto{\pgfqpoint{2.032106in}{2.769429in}}%
\pgfpathlineto{\pgfqpoint{1.964722in}{2.766513in}}%
\pgfpathlineto{\pgfqpoint{1.899497in}{2.764422in}}%
\pgfpathlineto{\pgfqpoint{1.899497in}{2.764422in}}%
\pgfusepath{stroke}%
\end{pgfscope}%
\begin{pgfscope}%
\pgfpathrectangle{\pgfqpoint{0.448634in}{0.402556in}}{\pgfqpoint{4.350661in}{2.489204in}} %
\pgfusepath{clip}%
\pgfsetrectcap%
\pgfsetroundjoin%
\pgfsetlinewidth{1.003750pt}%
\definecolor{currentstroke}{rgb}{0.737255,0.741176,0.133333}%
\pgfsetstrokecolor{currentstroke}%
\pgfsetdash{}{0pt}%
\pgfpathmoveto{\pgfqpoint{4.803400in}{0.402556in}}%
\pgfpathlineto{\pgfqpoint{2.412751in}{0.403451in}}%
\pgfpathlineto{\pgfqpoint{2.409021in}{0.405743in}}%
\pgfpathlineto{\pgfqpoint{2.408523in}{0.408121in}}%
\pgfpathlineto{\pgfqpoint{2.409406in}{0.410327in}}%
\pgfpathlineto{\pgfqpoint{2.413392in}{0.412078in}}%
\pgfpathlineto{\pgfqpoint{2.448164in}{0.411208in}}%
\pgfpathlineto{\pgfqpoint{2.452292in}{0.409707in}}%
\pgfpathlineto{\pgfqpoint{2.459346in}{0.404189in}}%
\pgfpathlineto{\pgfqpoint{2.467961in}{0.402931in}}%
\pgfpathlineto{\pgfqpoint{2.498412in}{0.402572in}}%
\pgfpathlineto{\pgfqpoint{2.711581in}{0.403202in}}%
\pgfpathlineto{\pgfqpoint{2.715835in}{0.404184in}}%
\pgfpathlineto{\pgfqpoint{2.719371in}{0.406978in}}%
\pgfpathlineto{\pgfqpoint{2.720040in}{0.414036in}}%
\pgfpathlineto{\pgfqpoint{2.719916in}{0.461328in}}%
\pgfpathlineto{\pgfqpoint{2.719427in}{0.575819in}}%
\pgfpathlineto{\pgfqpoint{2.720939in}{0.630477in}}%
\pgfpathlineto{\pgfqpoint{2.723121in}{0.634735in}}%
\pgfpathlineto{\pgfqpoint{2.723474in}{0.647156in}}%
\pgfpathlineto{\pgfqpoint{2.726515in}{0.851235in}}%
\pgfpathlineto{\pgfqpoint{2.733655in}{1.104997in}}%
\pgfpathlineto{\pgfqpoint{2.740647in}{1.264103in}}%
\pgfpathlineto{\pgfqpoint{2.751318in}{1.452883in}}%
\pgfpathlineto{\pgfqpoint{2.760129in}{1.574435in}}%
\pgfpathlineto{\pgfqpoint{2.770336in}{1.695836in}}%
\pgfpathlineto{\pgfqpoint{2.788361in}{1.871338in}}%
\pgfpathlineto{\pgfqpoint{2.804468in}{1.996937in}}%
\pgfpathlineto{\pgfqpoint{2.826824in}{2.144047in}}%
\pgfpathlineto{\pgfqpoint{2.843624in}{2.234115in}}%
\pgfpathlineto{\pgfqpoint{2.861438in}{2.316252in}}%
\pgfpathlineto{\pgfqpoint{2.907046in}{2.518600in}}%
\pgfpathlineto{\pgfqpoint{2.910201in}{2.545734in}}%
\pgfpathlineto{\pgfqpoint{2.910702in}{2.568120in}}%
\pgfpathlineto{\pgfqpoint{2.909090in}{2.587932in}}%
\pgfpathlineto{\pgfqpoint{2.905549in}{2.604863in}}%
\pgfpathlineto{\pgfqpoint{2.899753in}{2.620955in}}%
\pgfpathlineto{\pgfqpoint{2.891748in}{2.635755in}}%
\pgfpathlineto{\pgfqpoint{2.881844in}{2.648970in}}%
\pgfpathlineto{\pgfqpoint{2.870481in}{2.660553in}}%
\pgfpathlineto{\pgfqpoint{2.856229in}{2.671963in}}%
\pgfpathlineto{\pgfqpoint{2.839121in}{2.682837in}}%
\pgfpathlineto{\pgfqpoint{2.817242in}{2.693903in}}%
\pgfpathlineto{\pgfqpoint{2.792634in}{2.703853in}}%
\pgfpathlineto{\pgfqpoint{2.761237in}{2.713997in}}%
\pgfpathlineto{\pgfqpoint{2.725156in}{2.723258in}}%
\pgfpathlineto{\pgfqpoint{2.682323in}{2.731965in}}%
\pgfpathlineto{\pgfqpoint{2.647824in}{2.737125in}}%
\pgfpathlineto{\pgfqpoint{2.502826in}{2.752403in}}%
\pgfpathlineto{\pgfqpoint{2.398610in}{2.759344in}}%
\pgfpathlineto{\pgfqpoint{2.379101in}{2.761089in}}%
\pgfpathlineto{\pgfqpoint{2.359669in}{2.763590in}}%
\pgfpathlineto{\pgfqpoint{2.283548in}{2.765322in}}%
\pgfpathlineto{\pgfqpoint{2.237869in}{2.764997in}}%
\pgfpathlineto{\pgfqpoint{2.150857in}{2.764950in}}%
\pgfpathlineto{\pgfqpoint{2.079104in}{2.764766in}}%
\pgfpathlineto{\pgfqpoint{2.042207in}{2.762208in}}%
\pgfpathlineto{\pgfqpoint{1.909654in}{2.757861in}}%
\pgfpathlineto{\pgfqpoint{1.846624in}{2.755088in}}%
\pgfpathlineto{\pgfqpoint{1.831449in}{2.753908in}}%
\pgfpathlineto{\pgfqpoint{1.814124in}{2.752141in}}%
\pgfpathlineto{\pgfqpoint{1.701327in}{2.743153in}}%
\pgfpathlineto{\pgfqpoint{1.597468in}{2.731128in}}%
\pgfpathlineto{\pgfqpoint{1.494113in}{2.714227in}}%
\pgfpathlineto{\pgfqpoint{1.429871in}{2.701112in}}%
\pgfpathlineto{\pgfqpoint{1.363897in}{2.685202in}}%
\pgfpathlineto{\pgfqpoint{1.311254in}{2.669603in}}%
\pgfpathlineto{\pgfqpoint{1.265465in}{2.653690in}}%
\pgfpathlineto{\pgfqpoint{1.224432in}{2.637157in}}%
\pgfpathlineto{\pgfqpoint{1.182082in}{2.617591in}}%
\pgfpathlineto{\pgfqpoint{1.148645in}{2.599525in}}%
\pgfpathlineto{\pgfqpoint{1.116106in}{2.579429in}}%
\pgfpathlineto{\pgfqpoint{1.086486in}{2.558526in}}%
\pgfpathlineto{\pgfqpoint{1.059770in}{2.537097in}}%
\pgfpathlineto{\pgfqpoint{1.032505in}{2.512351in}}%
\pgfpathlineto{\pgfqpoint{1.008378in}{2.487222in}}%
\pgfpathlineto{\pgfqpoint{0.985684in}{2.460403in}}%
\pgfpathlineto{\pgfqpoint{0.964496in}{2.432015in}}%
\pgfpathlineto{\pgfqpoint{0.944881in}{2.402182in}}%
\pgfpathlineto{\pgfqpoint{0.925591in}{2.369039in}}%
\pgfpathlineto{\pgfqpoint{0.906921in}{2.332518in}}%
\pgfpathlineto{\pgfqpoint{0.889089in}{2.292635in}}%
\pgfpathlineto{\pgfqpoint{0.873144in}{2.251716in}}%
\pgfpathlineto{\pgfqpoint{0.857359in}{2.205330in}}%
\pgfpathlineto{\pgfqpoint{0.849511in}{2.176884in}}%
\pgfpathlineto{\pgfqpoint{0.837557in}{2.131614in}}%
\pgfpathlineto{\pgfqpoint{0.825462in}{2.078633in}}%
\pgfpathlineto{\pgfqpoint{0.814403in}{2.020253in}}%
\pgfpathlineto{\pgfqpoint{0.805880in}{1.963842in}}%
\pgfpathlineto{\pgfqpoint{0.798604in}{1.902176in}}%
\pgfpathlineto{\pgfqpoint{0.792932in}{1.835285in}}%
\pgfpathlineto{\pgfqpoint{0.789066in}{1.763237in}}%
\pgfpathlineto{\pgfqpoint{0.786736in}{1.676159in}}%
\pgfpathlineto{\pgfqpoint{0.785603in}{1.561663in}}%
\pgfpathlineto{\pgfqpoint{0.785603in}{1.561663in}}%
\pgfusepath{stroke}%
\end{pgfscope}%
\begin{pgfscope}%
\pgfpathrectangle{\pgfqpoint{0.448634in}{0.402556in}}{\pgfqpoint{4.350661in}{2.489204in}} %
\pgfusepath{clip}%
\pgfsetrectcap%
\pgfsetroundjoin%
\pgfsetlinewidth{1.003750pt}%
\definecolor{currentstroke}{rgb}{0.737255,0.741176,0.133333}%
\pgfsetstrokecolor{currentstroke}%
\pgfsetdash{}{0pt}%
\pgfpathmoveto{\pgfqpoint{3.714496in}{2.778934in}}%
\pgfpathlineto{\pgfqpoint{3.724777in}{2.774913in}}%
\pgfpathlineto{\pgfqpoint{3.760883in}{2.757585in}}%
\pgfpathlineto{\pgfqpoint{3.816967in}{2.730417in}}%
\pgfpathlineto{\pgfqpoint{3.872054in}{2.707106in}}%
\pgfpathlineto{\pgfqpoint{3.916656in}{2.687268in}}%
\pgfpathlineto{\pgfqpoint{3.950371in}{2.669894in}}%
\pgfpathlineto{\pgfqpoint{3.991090in}{2.646205in}}%
\pgfpathlineto{\pgfqpoint{4.084942in}{2.588397in}}%
\pgfpathlineto{\pgfqpoint{4.114521in}{2.567419in}}%
\pgfpathlineto{\pgfqpoint{4.135669in}{2.549921in}}%
\pgfpathlineto{\pgfqpoint{4.153808in}{2.532078in}}%
\pgfpathlineto{\pgfqpoint{4.168801in}{2.514059in}}%
\pgfpathlineto{\pgfqpoint{4.180650in}{2.496240in}}%
\pgfpathlineto{\pgfqpoint{4.189560in}{2.479146in}}%
\pgfpathlineto{\pgfqpoint{4.196754in}{2.461025in}}%
\pgfpathlineto{\pgfqpoint{4.202799in}{2.439729in}}%
\pgfpathlineto{\pgfqpoint{4.207319in}{2.415390in}}%
\pgfpathlineto{\pgfqpoint{4.210181in}{2.388214in}}%
\pgfpathlineto{\pgfqpoint{4.210836in}{2.360856in}}%
\pgfpathlineto{\pgfqpoint{4.209342in}{2.341028in}}%
\pgfpathlineto{\pgfqpoint{4.205976in}{2.324052in}}%
\pgfpathlineto{\pgfqpoint{4.201036in}{2.310248in}}%
\pgfpathlineto{\pgfqpoint{4.195237in}{2.299736in}}%
\pgfpathlineto{\pgfqpoint{4.187933in}{2.290536in}}%
\pgfpathlineto{\pgfqpoint{4.179371in}{2.282882in}}%
\pgfpathlineto{\pgfqpoint{4.167934in}{2.275720in}}%
\pgfpathlineto{\pgfqpoint{4.153651in}{2.269716in}}%
\pgfpathlineto{\pgfqpoint{4.136725in}{2.265125in}}%
\pgfpathlineto{\pgfqpoint{4.115197in}{2.261599in}}%
\pgfpathlineto{\pgfqpoint{4.084833in}{2.258971in}}%
\pgfpathlineto{\pgfqpoint{4.030482in}{2.256894in}}%
\pgfpathlineto{\pgfqpoint{3.926128in}{2.252884in}}%
\pgfpathlineto{\pgfqpoint{3.861006in}{2.248081in}}%
\pgfpathlineto{\pgfqpoint{3.802546in}{2.241625in}}%
\pgfpathlineto{\pgfqpoint{3.748635in}{2.233468in}}%
\pgfpathlineto{\pgfqpoint{3.709889in}{2.227769in}}%
\pgfpathlineto{\pgfqpoint{3.581609in}{2.231081in}}%
\pgfpathlineto{\pgfqpoint{3.557770in}{2.228810in}}%
\pgfpathlineto{\pgfqpoint{3.536343in}{2.224560in}}%
\pgfpathlineto{\pgfqpoint{3.513183in}{2.217699in}}%
\pgfpathlineto{\pgfqpoint{3.484285in}{2.206721in}}%
\pgfpathlineto{\pgfqpoint{3.454024in}{2.192771in}}%
\pgfpathlineto{\pgfqpoint{3.426543in}{2.177763in}}%
\pgfpathlineto{\pgfqpoint{3.398056in}{2.159568in}}%
\pgfpathlineto{\pgfqpoint{3.370648in}{2.139318in}}%
\pgfpathlineto{\pgfqpoint{3.344390in}{2.117162in}}%
\pgfpathlineto{\pgfqpoint{3.319360in}{2.093218in}}%
\pgfpathlineto{\pgfqpoint{3.295631in}{2.067597in}}%
\pgfpathlineto{\pgfqpoint{3.271819in}{2.038557in}}%
\pgfpathlineto{\pgfqpoint{3.249576in}{2.007932in}}%
\pgfpathlineto{\pgfqpoint{3.228896in}{1.975904in}}%
\pgfpathlineto{\pgfqpoint{3.208601in}{1.940535in}}%
\pgfpathlineto{\pgfqpoint{3.188959in}{1.901782in}}%
\pgfpathlineto{\pgfqpoint{3.171150in}{1.861884in}}%
\pgfpathlineto{\pgfqpoint{3.154279in}{1.818714in}}%
\pgfpathlineto{\pgfqpoint{3.138580in}{1.772290in}}%
\pgfpathlineto{\pgfqpoint{3.124921in}{1.725028in}}%
\pgfpathlineto{\pgfqpoint{3.112144in}{1.672259in}}%
\pgfpathlineto{\pgfqpoint{3.101641in}{1.618837in}}%
\pgfpathlineto{\pgfqpoint{3.092609in}{1.560004in}}%
\pgfpathlineto{\pgfqpoint{3.085688in}{1.498286in}}%
\pgfpathlineto{\pgfqpoint{3.081301in}{1.436262in}}%
\pgfpathlineto{\pgfqpoint{3.079848in}{1.384019in}}%
\pgfpathlineto{\pgfqpoint{3.080281in}{1.324285in}}%
\pgfpathlineto{\pgfqpoint{3.083224in}{1.254674in}}%
\pgfpathlineto{\pgfqpoint{3.088694in}{1.195268in}}%
\pgfpathlineto{\pgfqpoint{3.097276in}{1.133830in}}%
\pgfpathlineto{\pgfqpoint{3.108214in}{1.077975in}}%
\pgfpathlineto{\pgfqpoint{3.118479in}{1.037327in}}%
\pgfpathlineto{\pgfqpoint{3.133188in}{0.990488in}}%
\pgfpathlineto{\pgfqpoint{3.146321in}{0.956314in}}%
\pgfpathlineto{\pgfqpoint{3.160320in}{0.925372in}}%
\pgfpathlineto{\pgfqpoint{3.176193in}{0.895639in}}%
\pgfpathlineto{\pgfqpoint{3.193908in}{0.867303in}}%
\pgfpathlineto{\pgfqpoint{3.210610in}{0.844354in}}%
\pgfpathlineto{\pgfqpoint{3.230081in}{0.820897in}}%
\pgfpathlineto{\pgfqpoint{3.251022in}{0.799162in}}%
\pgfpathlineto{\pgfqpoint{3.275054in}{0.777768in}}%
\pgfpathlineto{\pgfqpoint{3.300342in}{0.758360in}}%
\pgfpathlineto{\pgfqpoint{3.328575in}{0.739658in}}%
\pgfpathlineto{\pgfqpoint{3.363701in}{0.719877in}}%
\pgfpathlineto{\pgfqpoint{3.397761in}{0.703409in}}%
\pgfpathlineto{\pgfqpoint{3.434564in}{0.688123in}}%
\pgfpathlineto{\pgfqpoint{3.476109in}{0.673358in}}%
\pgfpathlineto{\pgfqpoint{3.520274in}{0.660014in}}%
\pgfpathlineto{\pgfqpoint{3.569114in}{0.647603in}}%
\pgfpathlineto{\pgfqpoint{3.620451in}{0.636752in}}%
\pgfpathlineto{\pgfqpoint{3.676372in}{0.627075in}}%
\pgfpathlineto{\pgfqpoint{3.728222in}{0.620135in}}%
\pgfpathlineto{\pgfqpoint{3.786740in}{0.614393in}}%
\pgfpathlineto{\pgfqpoint{3.851882in}{0.609951in}}%
\pgfpathlineto{\pgfqpoint{3.919282in}{0.607549in}}%
\pgfpathlineto{\pgfqpoint{3.986715in}{0.607284in}}%
\pgfpathlineto{\pgfqpoint{4.056301in}{0.609293in}}%
\pgfpathlineto{\pgfqpoint{4.119280in}{0.613418in}}%
\pgfpathlineto{\pgfqpoint{4.177778in}{0.619399in}}%
\pgfpathlineto{\pgfqpoint{4.231743in}{0.627079in}}%
\pgfpathlineto{\pgfqpoint{4.281124in}{0.636264in}}%
\pgfpathlineto{\pgfqpoint{4.327999in}{0.647279in}}%
\pgfpathlineto{\pgfqpoint{4.370158in}{0.659554in}}%
\pgfpathlineto{\pgfqpoint{4.407584in}{0.672708in}}%
\pgfpathlineto{\pgfqpoint{4.442281in}{0.687329in}}%
\pgfpathlineto{\pgfqpoint{4.472160in}{0.702320in}}%
\pgfpathlineto{\pgfqpoint{4.501343in}{0.718974in}}%
\pgfpathlineto{\pgfqpoint{4.525487in}{0.735810in}}%
\pgfpathlineto{\pgfqpoint{4.548497in}{0.754607in}}%
\pgfpathlineto{\pgfqpoint{4.568529in}{0.773750in}}%
\pgfpathlineto{\pgfqpoint{4.587185in}{0.794631in}}%
\pgfpathlineto{\pgfqpoint{4.604323in}{0.817152in}}%
\pgfpathlineto{\pgfqpoint{4.619855in}{0.841149in}}%
\pgfpathlineto{\pgfqpoint{4.634864in}{0.868565in}}%
\pgfpathlineto{\pgfqpoint{4.649013in}{0.899416in}}%
\pgfpathlineto{\pgfqpoint{4.662060in}{0.933631in}}%
\pgfpathlineto{\pgfqpoint{4.673889in}{0.971081in}}%
\pgfpathlineto{\pgfqpoint{4.684358in}{1.011654in}}%
\pgfpathlineto{\pgfqpoint{4.707038in}{1.133333in}}%
\pgfpathlineto{\pgfqpoint{4.716073in}{1.202256in}}%
\pgfpathlineto{\pgfqpoint{4.724436in}{1.283837in}}%
\pgfpathlineto{\pgfqpoint{4.731593in}{1.375570in}}%
\pgfpathlineto{\pgfqpoint{4.737947in}{1.484851in}}%
\pgfpathlineto{\pgfqpoint{4.743247in}{1.614145in}}%
\pgfpathlineto{\pgfqpoint{4.747020in}{1.760943in}}%
\pgfpathlineto{\pgfqpoint{4.749061in}{1.925212in}}%
\pgfpathlineto{\pgfqpoint{4.748972in}{2.096966in}}%
\pgfpathlineto{\pgfqpoint{4.746714in}{2.256252in}}%
\pgfpathlineto{\pgfqpoint{4.742614in}{2.393074in}}%
\pgfpathlineto{\pgfqpoint{4.737087in}{2.502410in}}%
\pgfpathlineto{\pgfqpoint{4.730777in}{2.581732in}}%
\pgfpathlineto{\pgfqpoint{4.723990in}{2.640961in}}%
\pgfpathlineto{\pgfqpoint{4.716557in}{2.687479in}}%
\pgfpathlineto{\pgfqpoint{4.708637in}{2.723692in}}%
\pgfpathlineto{\pgfqpoint{4.700342in}{2.752005in}}%
\pgfpathlineto{\pgfqpoint{4.691480in}{2.774725in}}%
\pgfpathlineto{\pgfqpoint{4.681535in}{2.794006in}}%
\pgfpathlineto{\pgfqpoint{4.670856in}{2.809710in}}%
\pgfpathlineto{\pgfqpoint{4.660011in}{2.821925in}}%
\pgfpathlineto{\pgfqpoint{4.647888in}{2.832446in}}%
\pgfpathlineto{\pgfqpoint{4.632855in}{2.842453in}}%
\pgfpathlineto{\pgfqpoint{4.616929in}{2.850460in}}%
\pgfpathlineto{\pgfqpoint{4.596259in}{2.858187in}}%
\pgfpathlineto{\pgfqpoint{4.570850in}{2.865002in}}%
\pgfpathlineto{\pgfqpoint{4.519683in}{2.875449in}}%
\pgfpathlineto{\pgfqpoint{4.471967in}{2.879602in}}%
\pgfpathlineto{\pgfqpoint{4.408948in}{2.882787in}}%
\pgfpathlineto{\pgfqpoint{4.313267in}{2.885645in}}%
\pgfpathlineto{\pgfqpoint{4.154481in}{2.887878in}}%
\pgfpathlineto{\pgfqpoint{3.867341in}{2.889565in}}%
\pgfpathlineto{\pgfqpoint{3.203866in}{2.890726in}}%
\pgfpathlineto{\pgfqpoint{2.546919in}{2.890070in}}%
\pgfpathlineto{\pgfqpoint{2.314181in}{2.886429in}}%
\pgfpathlineto{\pgfqpoint{2.292491in}{2.884673in}}%
\pgfpathlineto{\pgfqpoint{2.279735in}{2.881601in}}%
\pgfpathlineto{\pgfqpoint{2.275944in}{2.879232in}}%
\pgfpathlineto{\pgfqpoint{2.273837in}{2.874958in}}%
\pgfpathlineto{\pgfqpoint{2.271725in}{2.865416in}}%
\pgfpathlineto{\pgfqpoint{2.267969in}{2.863083in}}%
\pgfpathlineto{\pgfqpoint{2.259324in}{2.862122in}}%
\pgfpathlineto{\pgfqpoint{2.083159in}{2.858216in}}%
\pgfpathlineto{\pgfqpoint{1.926585in}{2.854734in}}%
\pgfpathlineto{\pgfqpoint{1.876573in}{2.853173in}}%
\pgfpathlineto{\pgfqpoint{1.791771in}{2.850964in}}%
\pgfpathlineto{\pgfqpoint{1.752680in}{2.848536in}}%
\pgfpathlineto{\pgfqpoint{1.672267in}{2.845633in}}%
\pgfpathlineto{\pgfqpoint{1.650584in}{2.843756in}}%
\pgfpathlineto{\pgfqpoint{1.578934in}{2.840439in}}%
\pgfpathlineto{\pgfqpoint{1.565963in}{2.838889in}}%
\pgfpathlineto{\pgfqpoint{1.468208in}{2.833738in}}%
\pgfpathlineto{\pgfqpoint{1.361948in}{2.824166in}}%
\pgfpathlineto{\pgfqpoint{1.268864in}{2.813639in}}%
\pgfpathlineto{\pgfqpoint{1.223484in}{2.807632in}}%
\pgfpathlineto{\pgfqpoint{1.223484in}{2.807632in}}%
\pgfusepath{stroke}%
\end{pgfscope}%
\begin{pgfscope}%
\pgfpathrectangle{\pgfqpoint{0.448634in}{0.402556in}}{\pgfqpoint{4.350661in}{2.489204in}} %
\pgfusepath{clip}%
\pgfsetrectcap%
\pgfsetroundjoin%
\pgfsetlinewidth{1.003750pt}%
\definecolor{currentstroke}{rgb}{0.737255,0.741176,0.133333}%
\pgfsetstrokecolor{currentstroke}%
\pgfsetdash{}{0pt}%
\pgfpathmoveto{\pgfqpoint{4.803212in}{0.402556in}}%
\pgfpathlineto{\pgfqpoint{0.450376in}{0.402556in}}%
\pgfpathlineto{\pgfqpoint{0.450376in}{0.402556in}}%
\pgfusepath{stroke}%
\end{pgfscope}%
\begin{pgfscope}%
\pgfpathrectangle{\pgfqpoint{0.448634in}{0.402556in}}{\pgfqpoint{4.350661in}{2.489204in}} %
\pgfusepath{clip}%
\pgfsetrectcap%
\pgfsetroundjoin%
\pgfsetlinewidth{1.003750pt}%
\definecolor{currentstroke}{rgb}{0.737255,0.741176,0.133333}%
\pgfsetstrokecolor{currentstroke}%
\pgfsetdash{}{0pt}%
\pgfpathmoveto{\pgfqpoint{1.577292in}{2.886150in}}%
\pgfpathlineto{\pgfqpoint{1.320620in}{2.883790in}}%
\pgfpathlineto{\pgfqpoint{1.164038in}{2.879737in}}%
\pgfpathlineto{\pgfqpoint{1.059709in}{2.874901in}}%
\pgfpathlineto{\pgfqpoint{0.983739in}{2.869196in}}%
\pgfpathlineto{\pgfqpoint{0.927456in}{2.862842in}}%
\pgfpathlineto{\pgfqpoint{0.882220in}{2.855589in}}%
\pgfpathlineto{\pgfqpoint{0.845906in}{2.847623in}}%
\pgfpathlineto{\pgfqpoint{0.814295in}{2.838391in}}%
\pgfpathlineto{\pgfqpoint{0.787479in}{2.828144in}}%
\pgfpathlineto{\pgfqpoint{0.765472in}{2.817413in}}%
\pgfpathlineto{\pgfqpoint{0.746278in}{2.805716in}}%
\pgfpathlineto{\pgfqpoint{0.728152in}{2.791973in}}%
\pgfpathlineto{\pgfqpoint{0.712998in}{2.777802in}}%
\pgfpathlineto{\pgfqpoint{0.699129in}{2.762003in}}%
\pgfpathlineto{\pgfqpoint{0.685344in}{2.742761in}}%
\pgfpathlineto{\pgfqpoint{0.673267in}{2.722068in}}%
\pgfpathlineto{\pgfqpoint{0.661818in}{2.698034in}}%
\pgfpathlineto{\pgfqpoint{0.651252in}{2.670728in}}%
\pgfpathlineto{\pgfqpoint{0.641035in}{2.637907in}}%
\pgfpathlineto{\pgfqpoint{0.631580in}{2.599584in}}%
\pgfpathlineto{\pgfqpoint{0.623142in}{2.555837in}}%
\pgfpathlineto{\pgfqpoint{0.615548in}{2.504296in}}%
\pgfpathlineto{\pgfqpoint{0.609133in}{2.445012in}}%
\pgfpathlineto{\pgfqpoint{0.604155in}{2.378049in}}%
\pgfpathlineto{\pgfqpoint{0.600850in}{2.303472in}}%
\pgfpathlineto{\pgfqpoint{0.599524in}{2.223835in}}%
\pgfpathlineto{\pgfqpoint{0.600333in}{2.141699in}}%
\pgfpathlineto{\pgfqpoint{0.603363in}{2.059632in}}%
\pgfpathlineto{\pgfqpoint{0.608283in}{1.985173in}}%
\pgfpathlineto{\pgfqpoint{0.614871in}{1.918393in}}%
\pgfpathlineto{\pgfqpoint{0.622643in}{1.861841in}}%
\pgfpathlineto{\pgfqpoint{0.630720in}{1.818006in}}%
\pgfpathlineto{\pgfqpoint{0.639541in}{1.782068in}}%
\pgfpathlineto{\pgfqpoint{0.648056in}{1.756492in}}%
\pgfpathlineto{\pgfqpoint{0.655095in}{1.741055in}}%
\pgfpathlineto{\pgfqpoint{0.661534in}{1.731046in}}%
\pgfpathlineto{\pgfqpoint{0.668151in}{1.724637in}}%
\pgfpathlineto{\pgfqpoint{0.674263in}{1.722183in}}%
\pgfpathlineto{\pgfqpoint{0.678583in}{1.722487in}}%
\pgfpathlineto{\pgfqpoint{0.684410in}{1.725735in}}%
\pgfpathlineto{\pgfqpoint{0.690486in}{1.732817in}}%
\pgfpathlineto{\pgfqpoint{0.696225in}{1.743371in}}%
\pgfpathlineto{\pgfqpoint{0.702323in}{1.759326in}}%
\pgfpathlineto{\pgfqpoint{0.708916in}{1.783038in}}%
\pgfpathlineto{\pgfqpoint{0.715901in}{1.816952in}}%
\pgfpathlineto{\pgfqpoint{0.723569in}{1.865953in}}%
\pgfpathlineto{\pgfqpoint{0.732706in}{1.939890in}}%
\pgfpathlineto{\pgfqpoint{0.744057in}{2.051147in}}%
\pgfpathlineto{\pgfqpoint{0.753952in}{2.142542in}}%
\pgfpathlineto{\pgfqpoint{0.761931in}{2.194006in}}%
\pgfpathlineto{\pgfqpoint{0.771307in}{2.240062in}}%
\pgfpathlineto{\pgfqpoint{0.782832in}{2.285475in}}%
\pgfpathlineto{\pgfqpoint{0.795118in}{2.325381in}}%
\pgfpathlineto{\pgfqpoint{0.808760in}{2.362013in}}%
\pgfpathlineto{\pgfqpoint{0.823543in}{2.395292in}}%
\pgfpathlineto{\pgfqpoint{0.840214in}{2.427381in}}%
\pgfpathlineto{\pgfqpoint{0.858692in}{2.458147in}}%
\pgfpathlineto{\pgfqpoint{0.878896in}{2.487458in}}%
\pgfpathlineto{\pgfqpoint{0.900745in}{2.515181in}}%
\pgfpathlineto{\pgfqpoint{0.924144in}{2.541195in}}%
\pgfpathlineto{\pgfqpoint{0.948974in}{2.565410in}}%
\pgfpathlineto{\pgfqpoint{0.975096in}{2.587775in}}%
\pgfpathlineto{\pgfqpoint{1.002354in}{2.608290in}}%
\pgfpathlineto{\pgfqpoint{1.032499in}{2.628186in}}%
\pgfpathlineto{\pgfqpoint{1.065531in}{2.647199in}}%
\pgfpathlineto{\pgfqpoint{1.101410in}{2.665130in}}%
\pgfpathlineto{\pgfqpoint{1.140068in}{2.681852in}}%
\pgfpathlineto{\pgfqpoint{1.181423in}{2.697301in}}%
\pgfpathlineto{\pgfqpoint{1.227497in}{2.712097in}}%
\pgfpathlineto{\pgfqpoint{1.278266in}{2.726009in}}%
\pgfpathlineto{\pgfqpoint{1.333692in}{2.738885in}}%
\pgfpathlineto{\pgfqpoint{1.395875in}{2.751027in}}%
\pgfpathlineto{\pgfqpoint{1.464795in}{2.762204in}}%
\pgfpathlineto{\pgfqpoint{1.533478in}{2.773188in}}%
\pgfpathlineto{\pgfqpoint{1.542962in}{2.779249in}}%
\pgfpathlineto{\pgfqpoint{1.555332in}{2.783972in}}%
\pgfpathlineto{\pgfqpoint{1.572357in}{2.788054in}}%
\pgfpathlineto{\pgfqpoint{1.608939in}{2.794207in}}%
\pgfpathlineto{\pgfqpoint{1.667335in}{2.801390in}}%
\pgfpathlineto{\pgfqpoint{1.766927in}{2.812461in}}%
\pgfpathlineto{\pgfqpoint{1.803538in}{2.818388in}}%
\pgfpathlineto{\pgfqpoint{1.816460in}{2.820485in}}%
\pgfpathlineto{\pgfqpoint{1.816460in}{2.820485in}}%
\pgfusepath{stroke}%
\end{pgfscope}%
\begin{pgfscope}%
\pgfpathrectangle{\pgfqpoint{0.448634in}{0.402556in}}{\pgfqpoint{4.350661in}{2.489204in}} %
\pgfusepath{clip}%
\pgfsetbuttcap%
\pgfsetroundjoin%
\pgfsetlinewidth{1.003750pt}%
\definecolor{currentstroke}{rgb}{0.000000,0.000000,0.000000}%
\pgfsetstrokecolor{currentstroke}%
\pgfsetdash{{1.000000pt}{1.650000pt}}{0.000000pt}%
\pgfpathmoveto{\pgfqpoint{1.127319in}{2.572074in}}%
\pgfpathlineto{\pgfqpoint{1.159575in}{2.592758in}}%
\pgfpathlineto{\pgfqpoint{1.192763in}{2.611414in}}%
\pgfpathlineto{\pgfqpoint{1.228726in}{2.629126in}}%
\pgfpathlineto{\pgfqpoint{1.267413in}{2.645758in}}%
\pgfpathlineto{\pgfqpoint{1.310846in}{2.661945in}}%
\pgfpathlineto{\pgfqpoint{1.356920in}{2.676740in}}%
\pgfpathlineto{\pgfqpoint{1.407680in}{2.690702in}}%
\pgfpathlineto{\pgfqpoint{1.463094in}{2.703640in}}%
\pgfpathlineto{\pgfqpoint{1.525273in}{2.715813in}}%
\pgfpathlineto{\pgfqpoint{1.594199in}{2.726937in}}%
\pgfpathlineto{\pgfqpoint{1.669843in}{2.736808in}}%
\pgfpathlineto{\pgfqpoint{1.752172in}{2.745271in}}%
\pgfpathlineto{\pgfqpoint{1.843325in}{2.752344in}}%
\pgfpathlineto{\pgfqpoint{1.941103in}{2.757656in}}%
\pgfpathlineto{\pgfqpoint{2.043301in}{2.760987in}}%
\pgfpathlineto{\pgfqpoint{2.147710in}{2.762199in}}%
\pgfpathlineto{\pgfqpoint{2.249945in}{2.761215in}}%
\pgfpathlineto{\pgfqpoint{2.345620in}{2.758145in}}%
\pgfpathlineto{\pgfqpoint{2.432525in}{2.753210in}}%
\pgfpathlineto{\pgfqpoint{2.508451in}{2.746766in}}%
\pgfpathlineto{\pgfqpoint{2.573368in}{2.739156in}}%
\pgfpathlineto{\pgfqpoint{2.629410in}{2.730451in}}%
\pgfpathlineto{\pgfqpoint{2.676543in}{2.720985in}}%
\pgfpathlineto{\pgfqpoint{2.716874in}{2.710666in}}%
\pgfpathlineto{\pgfqpoint{2.750366in}{2.699848in}}%
\pgfpathlineto{\pgfqpoint{2.779059in}{2.688192in}}%
\pgfpathlineto{\pgfqpoint{2.802882in}{2.676004in}}%
\pgfpathlineto{\pgfqpoint{2.821842in}{2.663819in}}%
\pgfpathlineto{\pgfqpoint{2.837815in}{2.650886in}}%
\pgfpathlineto{\pgfqpoint{2.850736in}{2.637564in}}%
\pgfpathlineto{\pgfqpoint{2.860694in}{2.624398in}}%
\pgfpathlineto{\pgfqpoint{2.869084in}{2.609873in}}%
\pgfpathlineto{\pgfqpoint{2.875698in}{2.594192in}}%
\pgfpathlineto{\pgfqpoint{2.881035in}{2.575255in}}%
\pgfpathlineto{\pgfqpoint{2.884200in}{2.555685in}}%
\pgfpathlineto{\pgfqpoint{2.885619in}{2.533351in}}%
\pgfpathlineto{\pgfqpoint{2.885038in}{2.505987in}}%
\pgfpathlineto{\pgfqpoint{2.882112in}{2.473807in}}%
\pgfpathlineto{\pgfqpoint{2.875657in}{2.429620in}}%
\pgfpathlineto{\pgfqpoint{2.863489in}{2.363873in}}%
\pgfpathlineto{\pgfqpoint{2.821102in}{2.142619in}}%
\pgfpathlineto{\pgfqpoint{2.804859in}{2.042271in}}%
\pgfpathlineto{\pgfqpoint{2.790421in}{1.939040in}}%
\pgfpathlineto{\pgfqpoint{2.777207in}{1.828054in}}%
\pgfpathlineto{\pgfqpoint{2.765338in}{1.709349in}}%
\pgfpathlineto{\pgfqpoint{2.754471in}{1.578010in}}%
\pgfpathlineto{\pgfqpoint{2.744640in}{1.431580in}}%
\pgfpathlineto{\pgfqpoint{2.735914in}{1.267598in}}%
\pgfpathlineto{\pgfqpoint{2.728277in}{1.081114in}}%
\pgfpathlineto{\pgfqpoint{2.721437in}{0.857223in}}%
\pgfpathlineto{\pgfqpoint{2.711961in}{0.541290in}}%
\pgfpathlineto{\pgfqpoint{2.708250in}{0.491694in}}%
\pgfpathlineto{\pgfqpoint{2.703951in}{0.462246in}}%
\pgfpathlineto{\pgfqpoint{2.699504in}{0.445599in}}%
\pgfpathlineto{\pgfqpoint{2.694517in}{0.434563in}}%
\pgfpathlineto{\pgfqpoint{2.688942in}{0.426947in}}%
\pgfpathlineto{\pgfqpoint{2.681980in}{0.421009in}}%
\pgfpathlineto{\pgfqpoint{2.672064in}{0.415948in}}%
\pgfpathlineto{\pgfqpoint{2.659429in}{0.412247in}}%
\pgfpathlineto{\pgfqpoint{2.640044in}{0.409163in}}%
\pgfpathlineto{\pgfqpoint{2.607490in}{0.406692in}}%
\pgfpathlineto{\pgfqpoint{2.548779in}{0.404894in}}%
\pgfpathlineto{\pgfqpoint{2.422615in}{0.403701in}}%
\pgfpathlineto{\pgfqpoint{2.026705in}{0.403016in}}%
\pgfpathlineto{\pgfqpoint{0.623617in}{0.403253in}}%
\pgfpathlineto{\pgfqpoint{0.477880in}{0.404742in}}%
\pgfpathlineto{\pgfqpoint{0.458368in}{0.406382in}}%
\pgfpathlineto{\pgfqpoint{0.452304in}{0.408937in}}%
\pgfpathlineto{\pgfqpoint{0.450213in}{0.413215in}}%
\pgfpathlineto{\pgfqpoint{0.449165in}{0.423080in}}%
\pgfpathlineto{\pgfqpoint{0.448735in}{0.465392in}}%
\pgfpathlineto{\pgfqpoint{0.448637in}{0.983146in}}%
\pgfpathlineto{\pgfqpoint{0.448652in}{2.889876in}}%
\pgfpathlineto{\pgfqpoint{0.448652in}{2.889876in}}%
\pgfusepath{stroke}%
\end{pgfscope}%
\begin{pgfscope}%
\pgfpathrectangle{\pgfqpoint{0.448634in}{0.402556in}}{\pgfqpoint{4.350661in}{2.489204in}} %
\pgfusepath{clip}%
\pgfsetbuttcap%
\pgfsetroundjoin%
\pgfsetlinewidth{1.003750pt}%
\definecolor{currentstroke}{rgb}{0.000000,0.000000,0.000000}%
\pgfsetstrokecolor{currentstroke}%
\pgfsetdash{{1.000000pt}{1.650000pt}}{0.000000pt}%
\pgfpathmoveto{\pgfqpoint{0.448634in}{2.896245in}}%
\pgfpathlineto{\pgfqpoint{0.448593in}{0.407043in}}%
\pgfpathlineto{\pgfqpoint{0.448593in}{0.407043in}}%
\pgfusepath{stroke}%
\end{pgfscope}%
\begin{pgfscope}%
\pgfpathrectangle{\pgfqpoint{0.448634in}{0.402556in}}{\pgfqpoint{4.350661in}{2.489204in}} %
\pgfusepath{clip}%
\pgfsetbuttcap%
\pgfsetroundjoin%
\pgfsetlinewidth{1.003750pt}%
\definecolor{currentstroke}{rgb}{0.000000,0.000000,0.000000}%
\pgfsetstrokecolor{currentstroke}%
\pgfsetdash{{1.000000pt}{1.650000pt}}{0.000000pt}%
\pgfpathmoveto{\pgfqpoint{0.576853in}{1.760817in}}%
\pgfpathlineto{\pgfqpoint{0.569394in}{1.840010in}}%
\pgfpathlineto{\pgfqpoint{0.563209in}{1.929338in}}%
\pgfpathlineto{\pgfqpoint{0.558592in}{2.028764in}}%
\pgfpathlineto{\pgfqpoint{0.555985in}{2.133265in}}%
\pgfpathlineto{\pgfqpoint{0.555566in}{2.237808in}}%
\pgfpathlineto{\pgfqpoint{0.557371in}{2.337352in}}%
\pgfpathlineto{\pgfqpoint{0.561096in}{2.424366in}}%
\pgfpathlineto{\pgfqpoint{0.566403in}{2.498791in}}%
\pgfpathlineto{\pgfqpoint{0.572909in}{2.560570in}}%
\pgfpathlineto{\pgfqpoint{0.580458in}{2.612119in}}%
\pgfpathlineto{\pgfqpoint{0.589086in}{2.655816in}}%
\pgfpathlineto{\pgfqpoint{0.598406in}{2.691589in}}%
\pgfpathlineto{\pgfqpoint{0.608613in}{2.721757in}}%
\pgfpathlineto{\pgfqpoint{0.619241in}{2.746278in}}%
\pgfpathlineto{\pgfqpoint{0.630817in}{2.767339in}}%
\pgfpathlineto{\pgfqpoint{0.642975in}{2.784884in}}%
\pgfpathlineto{\pgfqpoint{0.656813in}{2.800712in}}%
\pgfpathlineto{\pgfqpoint{0.672197in}{2.814549in}}%
\pgfpathlineto{\pgfqpoint{0.688853in}{2.826301in}}%
\pgfpathlineto{\pgfqpoint{0.706461in}{2.836076in}}%
\pgfpathlineto{\pgfqpoint{0.726804in}{2.844875in}}%
\pgfpathlineto{\pgfqpoint{0.751866in}{2.853203in}}%
\pgfpathlineto{\pgfqpoint{0.781631in}{2.860547in}}%
\pgfpathlineto{\pgfqpoint{0.818168in}{2.867054in}}%
\pgfpathlineto{\pgfqpoint{0.863581in}{2.872685in}}%
\pgfpathlineto{\pgfqpoint{0.922161in}{2.877518in}}%
\pgfpathlineto{\pgfqpoint{1.000391in}{2.881567in}}%
\pgfpathlineto{\pgfqpoint{1.111294in}{2.884881in}}%
\pgfpathlineto{\pgfqpoint{1.274428in}{2.887367in}}%
\pgfpathlineto{\pgfqpoint{1.552865in}{2.889263in}}%
\pgfpathlineto{\pgfqpoint{2.107573in}{2.890457in}}%
\pgfpathlineto{\pgfqpoint{3.343161in}{2.890573in}}%
\pgfpathlineto{\pgfqpoint{4.043615in}{2.888941in}}%
\pgfpathlineto{\pgfqpoint{4.289417in}{2.886404in}}%
\pgfpathlineto{\pgfqpoint{4.413375in}{2.883093in}}%
\pgfpathlineto{\pgfqpoint{4.489424in}{2.878997in}}%
\pgfpathlineto{\pgfqpoint{4.541451in}{2.874081in}}%
\pgfpathlineto{\pgfqpoint{4.578100in}{2.868470in}}%
\pgfpathlineto{\pgfqpoint{4.605818in}{2.862092in}}%
\pgfpathlineto{\pgfqpoint{4.626725in}{2.855245in}}%
\pgfpathlineto{\pgfqpoint{4.644925in}{2.847018in}}%
\pgfpathlineto{\pgfqpoint{4.660241in}{2.837590in}}%
\pgfpathlineto{\pgfqpoint{4.672623in}{2.827468in}}%
\pgfpathlineto{\pgfqpoint{4.683751in}{2.815592in}}%
\pgfpathlineto{\pgfqpoint{4.693406in}{2.802135in}}%
\pgfpathlineto{\pgfqpoint{4.702740in}{2.785343in}}%
\pgfpathlineto{\pgfqpoint{4.711277in}{2.765194in}}%
\pgfpathlineto{\pgfqpoint{4.719482in}{2.739484in}}%
\pgfpathlineto{\pgfqpoint{4.726293in}{2.710657in}}%
\pgfpathlineto{\pgfqpoint{4.733259in}{2.671643in}}%
\pgfpathlineto{\pgfqpoint{4.739604in}{2.622396in}}%
\pgfpathlineto{\pgfqpoint{4.745236in}{2.560504in}}%
\pgfpathlineto{\pgfqpoint{4.750164in}{2.481052in}}%
\pgfpathlineto{\pgfqpoint{4.754367in}{2.376618in}}%
\pgfpathlineto{\pgfqpoint{4.757443in}{2.242249in}}%
\pgfpathlineto{\pgfqpoint{4.758977in}{2.075483in}}%
\pgfpathlineto{\pgfqpoint{4.758447in}{1.888795in}}%
\pgfpathlineto{\pgfqpoint{4.755756in}{1.707111in}}%
\pgfpathlineto{\pgfqpoint{4.750925in}{1.532957in}}%
\pgfpathlineto{\pgfqpoint{4.744785in}{1.398726in}}%
\pgfpathlineto{\pgfqpoint{4.737575in}{1.289516in}}%
\pgfpathlineto{\pgfqpoint{4.728714in}{1.190470in}}%
\pgfpathlineto{\pgfqpoint{4.719652in}{1.116521in}}%
\pgfpathlineto{\pgfqpoint{4.710036in}{1.055276in}}%
\pgfpathlineto{\pgfqpoint{4.699503in}{1.001861in}}%
\pgfpathlineto{\pgfqpoint{4.689040in}{0.958690in}}%
\pgfpathlineto{\pgfqpoint{4.677219in}{0.918600in}}%
\pgfpathlineto{\pgfqpoint{4.664034in}{0.881749in}}%
\pgfpathlineto{\pgfqpoint{4.650584in}{0.850492in}}%
\pgfpathlineto{\pgfqpoint{4.636303in}{0.822570in}}%
\pgfpathlineto{\pgfqpoint{4.620207in}{0.795974in}}%
\pgfpathlineto{\pgfqpoint{4.603640in}{0.772901in}}%
\pgfpathlineto{\pgfqpoint{4.585488in}{0.751446in}}%
\pgfpathlineto{\pgfqpoint{4.565874in}{0.731749in}}%
\pgfpathlineto{\pgfqpoint{4.544964in}{0.713879in}}%
\pgfpathlineto{\pgfqpoint{4.522958in}{0.697824in}}%
\pgfpathlineto{\pgfqpoint{4.496157in}{0.681290in}}%
\pgfpathlineto{\pgfqpoint{4.470397in}{0.667953in}}%
\pgfpathlineto{\pgfqpoint{4.439961in}{0.654509in}}%
\pgfpathlineto{\pgfqpoint{4.406841in}{0.642281in}}%
\pgfpathlineto{\pgfqpoint{4.369009in}{0.630748in}}%
\pgfpathlineto{\pgfqpoint{4.326489in}{0.620226in}}%
\pgfpathlineto{\pgfqpoint{4.279327in}{0.610949in}}%
\pgfpathlineto{\pgfqpoint{4.227576in}{0.603085in}}%
\pgfpathlineto{\pgfqpoint{4.173450in}{0.597063in}}%
\pgfpathlineto{\pgfqpoint{4.110511in}{0.592203in}}%
\pgfpathlineto{\pgfqpoint{4.047471in}{0.589537in}}%
\pgfpathlineto{\pgfqpoint{3.977867in}{0.588624in}}%
\pgfpathlineto{\pgfqpoint{3.906093in}{0.589934in}}%
\pgfpathlineto{\pgfqpoint{3.834377in}{0.593496in}}%
\pgfpathlineto{\pgfqpoint{3.767120in}{0.599067in}}%
\pgfpathlineto{\pgfqpoint{3.704364in}{0.606392in}}%
\pgfpathlineto{\pgfqpoint{3.678516in}{0.610510in}}%
\pgfpathlineto{\pgfqpoint{3.620438in}{0.620500in}}%
\pgfpathlineto{\pgfqpoint{3.586319in}{0.628207in}}%
\pgfpathlineto{\pgfqpoint{3.495240in}{0.652428in}}%
\pgfpathlineto{\pgfqpoint{3.451528in}{0.667583in}}%
\pgfpathlineto{\pgfqpoint{3.408538in}{0.685220in}}%
\pgfpathlineto{\pgfqpoint{3.374594in}{0.702001in}}%
\pgfpathlineto{\pgfqpoint{3.345407in}{0.718682in}}%
\pgfpathlineto{\pgfqpoint{3.315236in}{0.738520in}}%
\pgfpathlineto{\pgfqpoint{3.288127in}{0.759290in}}%
\pgfpathlineto{\pgfqpoint{3.264004in}{0.780551in}}%
\pgfpathlineto{\pgfqpoint{3.241208in}{0.803648in}}%
\pgfpathlineto{\pgfqpoint{3.219894in}{0.828530in}}%
\pgfpathlineto{\pgfqpoint{3.200189in}{0.855091in}}%
\pgfpathlineto{\pgfqpoint{3.182177in}{0.883182in}}%
\pgfpathlineto{\pgfqpoint{3.165906in}{0.912633in}}%
\pgfpathlineto{\pgfqpoint{3.150351in}{0.945448in}}%
\pgfpathlineto{\pgfqpoint{3.136682in}{0.979345in}}%
\pgfpathlineto{\pgfqpoint{3.124073in}{1.016460in}}%
\pgfpathlineto{\pgfqpoint{3.112834in}{1.056769in}}%
\pgfpathlineto{\pgfqpoint{3.103046in}{1.100146in}}%
\pgfpathlineto{\pgfqpoint{3.095343in}{1.144071in}}%
\pgfpathlineto{\pgfqpoint{3.089208in}{1.190837in}}%
\pgfpathlineto{\pgfqpoint{3.084595in}{1.242838in}}%
\pgfpathlineto{\pgfqpoint{3.082137in}{1.295031in}}%
\pgfpathlineto{\pgfqpoint{3.081687in}{1.349787in}}%
\pgfpathlineto{\pgfqpoint{3.083451in}{1.406998in}}%
\pgfpathlineto{\pgfqpoint{3.087181in}{1.461589in}}%
\pgfpathlineto{\pgfqpoint{3.093485in}{1.520888in}}%
\pgfpathlineto{\pgfqpoint{3.101823in}{1.577334in}}%
\pgfpathlineto{\pgfqpoint{3.111930in}{1.630856in}}%
\pgfpathlineto{\pgfqpoint{3.124690in}{1.686208in}}%
\pgfpathlineto{\pgfqpoint{3.139178in}{1.738395in}}%
\pgfpathlineto{\pgfqpoint{3.155145in}{1.787366in}}%
\pgfpathlineto{\pgfqpoint{3.172353in}{1.833085in}}%
\pgfpathlineto{\pgfqpoint{3.191618in}{1.877716in}}%
\pgfpathlineto{\pgfqpoint{3.214026in}{1.923261in}}%
\pgfpathlineto{\pgfqpoint{3.236214in}{1.963157in}}%
\pgfpathlineto{\pgfqpoint{3.260178in}{2.001684in}}%
\pgfpathlineto{\pgfqpoint{3.285814in}{2.038776in}}%
\pgfpathlineto{\pgfqpoint{3.314415in}{2.076285in}}%
\pgfpathlineto{\pgfqpoint{3.348944in}{2.117711in}}%
\pgfpathlineto{\pgfqpoint{3.417133in}{2.198022in}}%
\pgfpathlineto{\pgfqpoint{3.426053in}{2.212128in}}%
\pgfpathlineto{\pgfqpoint{3.430798in}{2.223297in}}%
\pgfpathlineto{\pgfqpoint{3.432034in}{2.230603in}}%
\pgfpathlineto{\pgfqpoint{3.430773in}{2.237856in}}%
\pgfpathlineto{\pgfqpoint{3.426621in}{2.243526in}}%
\pgfpathlineto{\pgfqpoint{3.420908in}{2.247084in}}%
\pgfpathlineto{\pgfqpoint{3.412501in}{2.249583in}}%
\pgfpathlineto{\pgfqpoint{3.399499in}{2.250689in}}%
\pgfpathlineto{\pgfqpoint{3.384305in}{2.249671in}}%
\pgfpathlineto{\pgfqpoint{3.364985in}{2.246098in}}%
\pgfpathlineto{\pgfqpoint{3.341804in}{2.239342in}}%
\pgfpathlineto{\pgfqpoint{3.317109in}{2.229682in}}%
\pgfpathlineto{\pgfqpoint{3.291104in}{2.216986in}}%
\pgfpathlineto{\pgfqpoint{3.265928in}{2.202261in}}%
\pgfpathlineto{\pgfqpoint{3.239805in}{2.184361in}}%
\pgfpathlineto{\pgfqpoint{3.214775in}{2.164519in}}%
\pgfpathlineto{\pgfqpoint{3.190900in}{2.142893in}}%
\pgfpathlineto{\pgfqpoint{3.166657in}{2.117912in}}%
\pgfpathlineto{\pgfqpoint{3.143835in}{2.091233in}}%
\pgfpathlineto{\pgfqpoint{3.121079in}{2.061107in}}%
\pgfpathlineto{\pgfqpoint{3.099952in}{2.029463in}}%
\pgfpathlineto{\pgfqpoint{3.079251in}{1.994406in}}%
\pgfpathlineto{\pgfqpoint{3.059218in}{1.955915in}}%
\pgfpathlineto{\pgfqpoint{3.040058in}{1.914015in}}%
\pgfpathlineto{\pgfqpoint{3.022809in}{1.871041in}}%
\pgfpathlineto{\pgfqpoint{3.005790in}{1.822536in}}%
\pgfpathlineto{\pgfqpoint{2.990067in}{1.770819in}}%
\pgfpathlineto{\pgfqpoint{2.975708in}{1.715979in}}%
\pgfpathlineto{\pgfqpoint{2.962284in}{1.655680in}}%
\pgfpathlineto{\pgfqpoint{2.950496in}{1.592386in}}%
\pgfpathlineto{\pgfqpoint{2.940383in}{1.526185in}}%
\pgfpathlineto{\pgfqpoint{2.931745in}{1.454681in}}%
\pgfpathlineto{\pgfqpoint{2.925082in}{1.380399in}}%
\pgfpathlineto{\pgfqpoint{2.920647in}{1.305899in}}%
\pgfpathlineto{\pgfqpoint{2.918444in}{1.231270in}}%
\pgfpathlineto{\pgfqpoint{2.918545in}{1.159087in}}%
\pgfpathlineto{\pgfqpoint{2.920787in}{1.091931in}}%
\pgfpathlineto{\pgfqpoint{2.925177in}{1.027412in}}%
\pgfpathlineto{\pgfqpoint{2.931192in}{0.970580in}}%
\pgfpathlineto{\pgfqpoint{2.938760in}{0.919034in}}%
\pgfpathlineto{\pgfqpoint{2.947651in}{0.872852in}}%
\pgfpathlineto{\pgfqpoint{2.958213in}{0.829714in}}%
\pgfpathlineto{\pgfqpoint{2.969670in}{0.792114in}}%
\pgfpathlineto{\pgfqpoint{2.982463in}{0.757773in}}%
\pgfpathlineto{\pgfqpoint{2.996425in}{0.726812in}}%
\pgfpathlineto{\pgfqpoint{3.011299in}{0.699300in}}%
\pgfpathlineto{\pgfqpoint{3.026739in}{0.675225in}}%
\pgfpathlineto{\pgfqpoint{3.043828in}{0.652656in}}%
\pgfpathlineto{\pgfqpoint{3.062495in}{0.631788in}}%
\pgfpathlineto{\pgfqpoint{3.082602in}{0.612753in}}%
\pgfpathlineto{\pgfqpoint{3.103961in}{0.595592in}}%
\pgfpathlineto{\pgfqpoint{3.128268in}{0.579069in}}%
\pgfpathlineto{\pgfqpoint{3.153537in}{0.564554in}}%
\pgfpathlineto{\pgfqpoint{3.181571in}{0.550952in}}%
\pgfpathlineto{\pgfqpoint{3.214371in}{0.537647in}}%
\pgfpathlineto{\pgfqpoint{3.249846in}{0.525712in}}%
\pgfpathlineto{\pgfqpoint{3.290011in}{0.514571in}}%
\pgfpathlineto{\pgfqpoint{3.334820in}{0.504423in}}%
\pgfpathlineto{\pgfqpoint{3.386372in}{0.494999in}}%
\pgfpathlineto{\pgfqpoint{3.446798in}{0.486257in}}%
\pgfpathlineto{\pgfqpoint{3.518243in}{0.478282in}}%
\pgfpathlineto{\pgfqpoint{3.600685in}{0.471409in}}%
\pgfpathlineto{\pgfqpoint{3.696268in}{0.465713in}}%
\pgfpathlineto{\pgfqpoint{3.807144in}{0.461369in}}%
\pgfpathlineto{\pgfqpoint{3.933291in}{0.458719in}}%
\pgfpathlineto{\pgfqpoint{4.063808in}{0.458211in}}%
\pgfpathlineto{\pgfqpoint{4.187792in}{0.459914in}}%
\pgfpathlineto{\pgfqpoint{4.294335in}{0.463521in}}%
\pgfpathlineto{\pgfqpoint{4.381234in}{0.468574in}}%
\pgfpathlineto{\pgfqpoint{4.450636in}{0.474701in}}%
\pgfpathlineto{\pgfqpoint{4.506850in}{0.481799in}}%
\pgfpathlineto{\pgfqpoint{4.552009in}{0.489658in}}%
\pgfpathlineto{\pgfqpoint{4.588239in}{0.498115in}}%
\pgfpathlineto{\pgfqpoint{4.617656in}{0.507110in}}%
\pgfpathlineto{\pgfqpoint{4.642328in}{0.516843in}}%
\pgfpathlineto{\pgfqpoint{4.664194in}{0.527940in}}%
\pgfpathlineto{\pgfqpoint{4.681238in}{0.538945in}}%
\pgfpathlineto{\pgfqpoint{4.697164in}{0.551953in}}%
\pgfpathlineto{\pgfqpoint{4.710076in}{0.565289in}}%
\pgfpathlineto{\pgfqpoint{4.721578in}{0.580218in}}%
\pgfpathlineto{\pgfqpoint{4.731557in}{0.596521in}}%
\pgfpathlineto{\pgfqpoint{4.741000in}{0.616134in}}%
\pgfpathlineto{\pgfqpoint{4.749521in}{0.639027in}}%
\pgfpathlineto{\pgfqpoint{4.757522in}{0.667450in}}%
\pgfpathlineto{\pgfqpoint{4.764572in}{0.701345in}}%
\pgfpathlineto{\pgfqpoint{4.770840in}{0.743043in}}%
\pgfpathlineto{\pgfqpoint{4.776327in}{0.794934in}}%
\pgfpathlineto{\pgfqpoint{4.781278in}{0.864398in}}%
\pgfpathlineto{\pgfqpoint{4.785468in}{0.956371in}}%
\pgfpathlineto{\pgfqpoint{4.789000in}{1.085745in}}%
\pgfpathlineto{\pgfqpoint{4.791852in}{1.277385in}}%
\pgfpathlineto{\pgfqpoint{4.793959in}{1.581057in}}%
\pgfpathlineto{\pgfqpoint{4.794962in}{2.071429in}}%
\pgfpathlineto{\pgfqpoint{4.793967in}{2.559311in}}%
\pgfpathlineto{\pgfqpoint{4.791733in}{2.745981in}}%
\pgfpathlineto{\pgfqpoint{4.788955in}{2.818091in}}%
\pgfpathlineto{\pgfqpoint{4.785731in}{2.850227in}}%
\pgfpathlineto{\pgfqpoint{4.781879in}{2.867057in}}%
\pgfpathlineto{\pgfqpoint{4.777744in}{2.875780in}}%
\pgfpathlineto{\pgfqpoint{4.773097in}{2.880982in}}%
\pgfpathlineto{\pgfqpoint{4.767363in}{2.884504in}}%
\pgfpathlineto{\pgfqpoint{4.756853in}{2.887622in}}%
\pgfpathlineto{\pgfqpoint{4.739548in}{2.889639in}}%
\pgfpathlineto{\pgfqpoint{4.704762in}{2.890882in}}%
\pgfpathlineto{\pgfqpoint{4.602524in}{2.891538in}}%
\pgfpathlineto{\pgfqpoint{3.952100in}{2.891742in}}%
\pgfpathlineto{\pgfqpoint{0.617321in}{2.890753in}}%
\pgfpathlineto{\pgfqpoint{0.549910in}{2.888858in}}%
\pgfpathlineto{\pgfqpoint{0.521735in}{2.886179in}}%
\pgfpathlineto{\pgfqpoint{0.504666in}{2.882389in}}%
\pgfpathlineto{\pgfqpoint{0.494501in}{2.878011in}}%
\pgfpathlineto{\pgfqpoint{0.487180in}{2.872667in}}%
\pgfpathlineto{\pgfqpoint{0.481152in}{2.865519in}}%
\pgfpathlineto{\pgfqpoint{0.475664in}{2.854804in}}%
\pgfpathlineto{\pgfqpoint{0.471318in}{2.840737in}}%
\pgfpathlineto{\pgfqpoint{0.467301in}{2.818823in}}%
\pgfpathlineto{\pgfqpoint{0.463927in}{2.786700in}}%
\pgfpathlineto{\pgfqpoint{0.460918in}{2.734544in}}%
\pgfpathlineto{\pgfqpoint{0.458363in}{2.647473in}}%
\pgfpathlineto{\pgfqpoint{0.456575in}{2.523031in}}%
\pgfpathlineto{\pgfqpoint{0.456575in}{2.523031in}}%
\pgfusepath{stroke}%
\end{pgfscope}%
\begin{pgfscope}%
\pgfpathrectangle{\pgfqpoint{0.448634in}{0.402556in}}{\pgfqpoint{4.350661in}{2.489204in}} %
\pgfusepath{clip}%
\pgfsetbuttcap%
\pgfsetroundjoin%
\pgfsetlinewidth{1.003750pt}%
\definecolor{currentstroke}{rgb}{0.000000,0.000000,0.000000}%
\pgfsetstrokecolor{currentstroke}%
\pgfsetdash{{1.000000pt}{1.650000pt}}{0.000000pt}%
\pgfpathmoveto{\pgfqpoint{0.456424in}{1.370137in}}%
\pgfpathlineto{\pgfqpoint{0.459610in}{1.118755in}}%
\pgfpathlineto{\pgfqpoint{0.463695in}{0.962007in}}%
\pgfpathlineto{\pgfqpoint{0.468519in}{0.857610in}}%
\pgfpathlineto{\pgfqpoint{0.474082in}{0.783210in}}%
\pgfpathlineto{\pgfqpoint{0.480226in}{0.728906in}}%
\pgfpathlineto{\pgfqpoint{0.486970in}{0.687306in}}%
\pgfpathlineto{\pgfqpoint{0.494537in}{0.653558in}}%
\pgfpathlineto{\pgfqpoint{0.503107in}{0.625355in}}%
\pgfpathlineto{\pgfqpoint{0.512193in}{0.602749in}}%
\pgfpathlineto{\pgfqpoint{0.522200in}{0.583508in}}%
\pgfpathlineto{\pgfqpoint{0.534108in}{0.565743in}}%
\pgfpathlineto{\pgfqpoint{0.546263in}{0.551507in}}%
\pgfpathlineto{\pgfqpoint{0.559728in}{0.538907in}}%
\pgfpathlineto{\pgfqpoint{0.576129in}{0.526693in}}%
\pgfpathlineto{\pgfqpoint{0.595483in}{0.515351in}}%
\pgfpathlineto{\pgfqpoint{0.617681in}{0.505147in}}%
\pgfpathlineto{\pgfqpoint{0.642568in}{0.496153in}}%
\pgfpathlineto{\pgfqpoint{0.672126in}{0.487778in}}%
\pgfpathlineto{\pgfqpoint{0.708443in}{0.479824in}}%
\pgfpathlineto{\pgfqpoint{0.753649in}{0.472325in}}%
\pgfpathlineto{\pgfqpoint{0.807717in}{0.465660in}}%
\pgfpathlineto{\pgfqpoint{0.877116in}{0.459475in}}%
\pgfpathlineto{\pgfqpoint{0.961828in}{0.454230in}}%
\pgfpathlineto{\pgfqpoint{1.068351in}{0.449916in}}%
\pgfpathlineto{\pgfqpoint{1.201018in}{0.446839in}}%
\pgfpathlineto{\pgfqpoint{1.357637in}{0.445481in}}%
\pgfpathlineto{\pgfqpoint{1.525135in}{0.446232in}}%
\pgfpathlineto{\pgfqpoint{1.686088in}{0.449142in}}%
\pgfpathlineto{\pgfqpoint{1.823074in}{0.453747in}}%
\pgfpathlineto{\pgfqpoint{1.938245in}{0.459764in}}%
\pgfpathlineto{\pgfqpoint{2.031582in}{0.466759in}}%
\pgfpathlineto{\pgfqpoint{2.109580in}{0.474745in}}%
\pgfpathlineto{\pgfqpoint{2.174384in}{0.483535in}}%
\pgfpathlineto{\pgfqpoint{2.228139in}{0.492940in}}%
\pgfpathlineto{\pgfqpoint{2.275119in}{0.503356in}}%
\pgfpathlineto{\pgfqpoint{2.315282in}{0.514501in}}%
\pgfpathlineto{\pgfqpoint{2.350698in}{0.526659in}}%
\pgfpathlineto{\pgfqpoint{2.381320in}{0.539536in}}%
\pgfpathlineto{\pgfqpoint{2.407164in}{0.552659in}}%
\pgfpathlineto{\pgfqpoint{2.430226in}{0.566639in}}%
\pgfpathlineto{\pgfqpoint{2.452282in}{0.582602in}}%
\pgfpathlineto{\pgfqpoint{2.471391in}{0.599069in}}%
\pgfpathlineto{\pgfqpoint{2.489240in}{0.617293in}}%
\pgfpathlineto{\pgfqpoint{2.505678in}{0.637180in}}%
\pgfpathlineto{\pgfqpoint{2.520620in}{0.658557in}}%
\pgfpathlineto{\pgfqpoint{2.535213in}{0.683314in}}%
\pgfpathlineto{\pgfqpoint{2.549115in}{0.711484in}}%
\pgfpathlineto{\pgfqpoint{2.562091in}{0.743004in}}%
\pgfpathlineto{\pgfqpoint{2.574020in}{0.777751in}}%
\pgfpathlineto{\pgfqpoint{2.585502in}{0.817970in}}%
\pgfpathlineto{\pgfqpoint{2.596809in}{0.866038in}}%
\pgfpathlineto{\pgfqpoint{2.607562in}{0.921948in}}%
\pgfpathlineto{\pgfqpoint{2.617925in}{0.988098in}}%
\pgfpathlineto{\pgfqpoint{2.627958in}{1.066918in}}%
\pgfpathlineto{\pgfqpoint{2.637941in}{1.163320in}}%
\pgfpathlineto{\pgfqpoint{2.648424in}{1.287199in}}%
\pgfpathlineto{\pgfqpoint{2.660103in}{1.453438in}}%
\pgfpathlineto{\pgfqpoint{2.674773in}{1.696801in}}%
\pgfpathlineto{\pgfqpoint{2.687716in}{1.945279in}}%
\pgfpathlineto{\pgfqpoint{2.692670in}{2.079573in}}%
\pgfpathlineto{\pgfqpoint{2.693829in}{2.166682in}}%
\pgfpathlineto{\pgfqpoint{2.692565in}{2.233870in}}%
\pgfpathlineto{\pgfqpoint{2.689436in}{2.286015in}}%
\pgfpathlineto{\pgfqpoint{2.684859in}{2.327999in}}%
\pgfpathlineto{\pgfqpoint{2.678725in}{2.364664in}}%
\pgfpathlineto{\pgfqpoint{2.671356in}{2.395897in}}%
\pgfpathlineto{\pgfqpoint{2.662489in}{2.423981in}}%
\pgfpathlineto{\pgfqpoint{2.652361in}{2.448778in}}%
\pgfpathlineto{\pgfqpoint{2.641365in}{2.470245in}}%
\pgfpathlineto{\pgfqpoint{2.628643in}{2.490425in}}%
\pgfpathlineto{\pgfqpoint{2.614279in}{2.509106in}}%
\pgfpathlineto{\pgfqpoint{2.598443in}{2.526159in}}%
\pgfpathlineto{\pgfqpoint{2.579590in}{2.543005in}}%
\pgfpathlineto{\pgfqpoint{2.559532in}{2.557923in}}%
\pgfpathlineto{\pgfqpoint{2.536602in}{2.572183in}}%
\pgfpathlineto{\pgfqpoint{2.510850in}{2.585538in}}%
\pgfpathlineto{\pgfqpoint{2.482360in}{2.597837in}}%
\pgfpathlineto{\pgfqpoint{2.449134in}{2.609683in}}%
\pgfpathlineto{\pgfqpoint{2.411184in}{2.620696in}}%
\pgfpathlineto{\pgfqpoint{2.368552in}{2.630606in}}%
\pgfpathlineto{\pgfqpoint{2.321294in}{2.639221in}}%
\pgfpathlineto{\pgfqpoint{2.269467in}{2.646399in}}%
\pgfpathlineto{\pgfqpoint{2.210954in}{2.652193in}}%
\pgfpathlineto{\pgfqpoint{2.147967in}{2.656153in}}%
\pgfpathlineto{\pgfqpoint{2.080556in}{2.658135in}}%
\pgfpathlineto{\pgfqpoint{2.010948in}{2.657971in}}%
\pgfpathlineto{\pgfqpoint{1.939195in}{2.655572in}}%
\pgfpathlineto{\pgfqpoint{1.867527in}{2.650913in}}%
\pgfpathlineto{\pgfqpoint{1.798171in}{2.644140in}}%
\pgfpathlineto{\pgfqpoint{1.733341in}{2.635606in}}%
\pgfpathlineto{\pgfqpoint{1.673075in}{2.625521in}}%
\pgfpathlineto{\pgfqpoint{1.615274in}{2.613610in}}%
\pgfpathlineto{\pgfqpoint{1.562133in}{2.600402in}}%
\pgfpathlineto{\pgfqpoint{1.513681in}{2.586139in}}%
\pgfpathlineto{\pgfqpoint{1.467862in}{2.570344in}}%
\pgfpathlineto{\pgfqpoint{1.426794in}{2.553923in}}%
\pgfpathlineto{\pgfqpoint{1.388447in}{2.536289in}}%
\pgfpathlineto{\pgfqpoint{1.352878in}{2.517566in}}%
\pgfpathlineto{\pgfqpoint{1.320128in}{2.497922in}}%
\pgfpathlineto{\pgfqpoint{1.288379in}{2.476236in}}%
\pgfpathlineto{\pgfqpoint{1.259592in}{2.453861in}}%
\pgfpathlineto{\pgfqpoint{1.232050in}{2.429520in}}%
\pgfpathlineto{\pgfqpoint{1.207527in}{2.404898in}}%
\pgfpathlineto{\pgfqpoint{1.184409in}{2.378557in}}%
\pgfpathlineto{\pgfqpoint{1.162828in}{2.350561in}}%
\pgfpathlineto{\pgfqpoint{1.142891in}{2.321011in}}%
\pgfpathlineto{\pgfqpoint{1.124675in}{2.290041in}}%
\pgfpathlineto{\pgfqpoint{1.108225in}{2.257802in}}%
\pgfpathlineto{\pgfqpoint{1.092639in}{2.222199in}}%
\pgfpathlineto{\pgfqpoint{1.079059in}{2.185535in}}%
\pgfpathlineto{\pgfqpoint{1.067443in}{2.147998in}}%
\pgfpathlineto{\pgfqpoint{1.057187in}{2.107348in}}%
\pgfpathlineto{\pgfqpoint{1.049004in}{2.066086in}}%
\pgfpathlineto{\pgfqpoint{1.042513in}{2.021906in}}%
\pgfpathlineto{\pgfqpoint{1.038177in}{1.977382in}}%
\pgfpathlineto{\pgfqpoint{1.035866in}{1.930167in}}%
\pgfpathlineto{\pgfqpoint{1.035826in}{1.882878in}}%
\pgfpathlineto{\pgfqpoint{1.038031in}{1.835656in}}%
\pgfpathlineto{\pgfqpoint{1.042474in}{1.788641in}}%
\pgfpathlineto{\pgfqpoint{1.049176in}{1.741979in}}%
\pgfpathlineto{\pgfqpoint{1.057644in}{1.698239in}}%
\pgfpathlineto{\pgfqpoint{1.068221in}{1.655105in}}%
\pgfpathlineto{\pgfqpoint{1.080962in}{1.612745in}}%
\pgfpathlineto{\pgfqpoint{1.095031in}{1.573617in}}%
\pgfpathlineto{\pgfqpoint{1.111115in}{1.535520in}}%
\pgfpathlineto{\pgfqpoint{1.128118in}{1.500775in}}%
\pgfpathlineto{\pgfqpoint{1.146930in}{1.467274in}}%
\pgfpathlineto{\pgfqpoint{1.167531in}{1.435181in}}%
\pgfpathlineto{\pgfqpoint{1.189874in}{1.404652in}}%
\pgfpathlineto{\pgfqpoint{1.213884in}{1.375828in}}%
\pgfpathlineto{\pgfqpoint{1.237817in}{1.350457in}}%
\pgfpathlineto{\pgfqpoint{1.264748in}{1.325237in}}%
\pgfpathlineto{\pgfqpoint{1.292991in}{1.301972in}}%
\pgfpathlineto{\pgfqpoint{1.322398in}{1.280678in}}%
\pgfpathlineto{\pgfqpoint{1.352820in}{1.261340in}}%
\pgfpathlineto{\pgfqpoint{1.386095in}{1.242889in}}%
\pgfpathlineto{\pgfqpoint{1.420190in}{1.226516in}}%
\pgfpathlineto{\pgfqpoint{1.457024in}{1.211329in}}%
\pgfpathlineto{\pgfqpoint{1.496554in}{1.197536in}}%
\pgfpathlineto{\pgfqpoint{1.538719in}{1.185287in}}%
\pgfpathlineto{\pgfqpoint{1.583441in}{1.174641in}}%
\pgfpathlineto{\pgfqpoint{1.634929in}{1.164775in}}%
\pgfpathlineto{\pgfqpoint{1.706063in}{1.153745in}}%
\pgfpathlineto{\pgfqpoint{1.768492in}{1.143417in}}%
\pgfpathlineto{\pgfqpoint{1.796122in}{1.136567in}}%
\pgfpathlineto{\pgfqpoint{1.812683in}{1.130481in}}%
\pgfpathlineto{\pgfqpoint{1.824471in}{1.124102in}}%
\pgfpathlineto{\pgfqpoint{1.833209in}{1.116741in}}%
\pgfpathlineto{\pgfqpoint{1.838498in}{1.108890in}}%
\pgfpathlineto{\pgfqpoint{1.840588in}{1.101849in}}%
\pgfpathlineto{\pgfqpoint{1.840619in}{1.094412in}}%
\pgfpathlineto{\pgfqpoint{1.837931in}{1.084986in}}%
\pgfpathlineto{\pgfqpoint{1.833246in}{1.076615in}}%
\pgfpathlineto{\pgfqpoint{1.825819in}{1.067542in}}%
\pgfpathlineto{\pgfqpoint{1.813813in}{1.056850in}}%
\pgfpathlineto{\pgfqpoint{1.798819in}{1.046763in}}%
\pgfpathlineto{\pgfqpoint{1.781016in}{1.037462in}}%
\pgfpathlineto{\pgfqpoint{1.758447in}{1.028391in}}%
\pgfpathlineto{\pgfqpoint{1.733203in}{1.020815in}}%
\pgfpathlineto{\pgfqpoint{1.705410in}{1.014872in}}%
\pgfpathlineto{\pgfqpoint{1.675178in}{1.010714in}}%
\pgfpathlineto{\pgfqpoint{1.642610in}{1.008507in}}%
\pgfpathlineto{\pgfqpoint{1.607809in}{1.008432in}}%
\pgfpathlineto{\pgfqpoint{1.570886in}{1.010691in}}%
\pgfpathlineto{\pgfqpoint{1.534118in}{1.015181in}}%
\pgfpathlineto{\pgfqpoint{1.495454in}{1.022233in}}%
\pgfpathlineto{\pgfqpoint{1.457161in}{1.031563in}}%
\pgfpathlineto{\pgfqpoint{1.419337in}{1.043132in}}%
\pgfpathlineto{\pgfqpoint{1.382089in}{1.056929in}}%
\pgfpathlineto{\pgfqpoint{1.347544in}{1.072019in}}%
\pgfpathlineto{\pgfqpoint{1.313727in}{1.089133in}}%
\pgfpathlineto{\pgfqpoint{1.280762in}{1.108299in}}%
\pgfpathlineto{\pgfqpoint{1.248782in}{1.129536in}}%
\pgfpathlineto{\pgfqpoint{1.219708in}{1.151422in}}%
\pgfpathlineto{\pgfqpoint{1.191752in}{1.175138in}}%
\pgfpathlineto{\pgfqpoint{1.165031in}{1.200649in}}%
\pgfpathlineto{\pgfqpoint{1.139653in}{1.227898in}}%
\pgfpathlineto{\pgfqpoint{1.115714in}{1.256800in}}%
\pgfpathlineto{\pgfqpoint{1.093288in}{1.287251in}}%
\pgfpathlineto{\pgfqpoint{1.071178in}{1.321163in}}%
\pgfpathlineto{\pgfqpoint{1.050868in}{1.356520in}}%
\pgfpathlineto{\pgfqpoint{1.032365in}{1.393152in}}%
\pgfpathlineto{\pgfqpoint{1.014718in}{1.433142in}}%
\pgfpathlineto{\pgfqpoint{0.999024in}{1.474185in}}%
\pgfpathlineto{\pgfqpoint{0.984506in}{1.518461in}}%
\pgfpathlineto{\pgfqpoint{0.972010in}{1.563537in}}%
\pgfpathlineto{\pgfqpoint{0.960944in}{1.611678in}}%
\pgfpathlineto{\pgfqpoint{0.951530in}{1.662824in}}%
\pgfpathlineto{\pgfqpoint{0.944286in}{1.714431in}}%
\pgfpathlineto{\pgfqpoint{0.938950in}{1.768847in}}%
\pgfpathlineto{\pgfqpoint{0.935870in}{1.823491in}}%
\pgfpathlineto{\pgfqpoint{0.935034in}{1.878240in}}%
\pgfpathlineto{\pgfqpoint{0.936466in}{1.932973in}}%
\pgfpathlineto{\pgfqpoint{0.940005in}{1.985084in}}%
\pgfpathlineto{\pgfqpoint{0.945759in}{2.036935in}}%
\pgfpathlineto{\pgfqpoint{0.953410in}{2.085938in}}%
\pgfpathlineto{\pgfqpoint{0.962764in}{2.132000in}}%
\pgfpathlineto{\pgfqpoint{0.974287in}{2.177414in}}%
\pgfpathlineto{\pgfqpoint{0.987332in}{2.219653in}}%
\pgfpathlineto{\pgfqpoint{1.001667in}{2.258654in}}%
\pgfpathlineto{\pgfqpoint{1.018051in}{2.296583in}}%
\pgfpathlineto{\pgfqpoint{1.035401in}{2.331101in}}%
\pgfpathlineto{\pgfqpoint{1.054650in}{2.364275in}}%
\pgfpathlineto{\pgfqpoint{1.074406in}{2.393984in}}%
\pgfpathlineto{\pgfqpoint{1.095771in}{2.422197in}}%
\pgfpathlineto{\pgfqpoint{1.118662in}{2.448797in}}%
\pgfpathlineto{\pgfqpoint{1.142967in}{2.473701in}}%
\pgfpathlineto{\pgfqpoint{1.168550in}{2.496867in}}%
\pgfpathlineto{\pgfqpoint{1.197085in}{2.519662in}}%
\pgfpathlineto{\pgfqpoint{1.226727in}{2.540526in}}%
\pgfpathlineto{\pgfqpoint{1.259242in}{2.560673in}}%
\pgfpathlineto{\pgfqpoint{1.294612in}{2.579881in}}%
\pgfpathlineto{\pgfqpoint{1.332792in}{2.597982in}}%
\pgfpathlineto{\pgfqpoint{1.373719in}{2.614859in}}%
\pgfpathlineto{\pgfqpoint{1.417319in}{2.630445in}}%
\pgfpathlineto{\pgfqpoint{1.465632in}{2.645312in}}%
\pgfpathlineto{\pgfqpoint{1.518640in}{2.659204in}}%
\pgfpathlineto{\pgfqpoint{1.576309in}{2.671929in}}%
\pgfpathlineto{\pgfqpoint{1.638597in}{2.683344in}}%
\pgfpathlineto{\pgfqpoint{1.705462in}{2.693343in}}%
\pgfpathlineto{\pgfqpoint{1.779027in}{2.702064in}}%
\pgfpathlineto{\pgfqpoint{1.857097in}{2.709077in}}%
\pgfpathlineto{\pgfqpoint{1.939633in}{2.714280in}}%
\pgfpathlineto{\pgfqpoint{2.026598in}{2.717513in}}%
\pgfpathlineto{\pgfqpoint{2.113605in}{2.718523in}}%
\pgfpathlineto{\pgfqpoint{2.198435in}{2.717303in}}%
\pgfpathlineto{\pgfqpoint{2.278866in}{2.713929in}}%
\pgfpathlineto{\pgfqpoint{2.352678in}{2.708598in}}%
\pgfpathlineto{\pgfqpoint{2.417657in}{2.701709in}}%
\pgfpathlineto{\pgfqpoint{2.473770in}{2.693630in}}%
\pgfpathlineto{\pgfqpoint{2.523140in}{2.684368in}}%
\pgfpathlineto{\pgfqpoint{2.565726in}{2.674202in}}%
\pgfpathlineto{\pgfqpoint{2.601510in}{2.663544in}}%
\pgfpathlineto{\pgfqpoint{2.632577in}{2.652142in}}%
\pgfpathlineto{\pgfqpoint{2.658899in}{2.640331in}}%
\pgfpathlineto{\pgfqpoint{2.682438in}{2.627436in}}%
\pgfpathlineto{\pgfqpoint{2.703062in}{2.613571in}}%
\pgfpathlineto{\pgfqpoint{2.720674in}{2.598978in}}%
\pgfpathlineto{\pgfqpoint{2.735263in}{2.584053in}}%
\pgfpathlineto{\pgfqpoint{2.748320in}{2.567377in}}%
\pgfpathlineto{\pgfqpoint{2.759553in}{2.549046in}}%
\pgfpathlineto{\pgfqpoint{2.768788in}{2.529306in}}%
\pgfpathlineto{\pgfqpoint{2.776017in}{2.508498in}}%
\pgfpathlineto{\pgfqpoint{2.781884in}{2.484540in}}%
\pgfpathlineto{\pgfqpoint{2.786102in}{2.457597in}}%
\pgfpathlineto{\pgfqpoint{2.788720in}{2.425384in}}%
\pgfpathlineto{\pgfqpoint{2.789427in}{2.388061in}}%
\pgfpathlineto{\pgfqpoint{2.787962in}{2.340801in}}%
\pgfpathlineto{\pgfqpoint{2.783672in}{2.278768in}}%
\pgfpathlineto{\pgfqpoint{2.774289in}{2.179783in}}%
\pgfpathlineto{\pgfqpoint{2.743611in}{1.868119in}}%
\pgfpathlineto{\pgfqpoint{2.730112in}{1.702060in}}%
\pgfpathlineto{\pgfqpoint{2.717287in}{1.515949in}}%
\pgfpathlineto{\pgfqpoint{2.702602in}{1.267597in}}%
\pgfpathlineto{\pgfqpoint{2.684434in}{0.964630in}}%
\pgfpathlineto{\pgfqpoint{2.675374in}{0.850600in}}%
\pgfpathlineto{\pgfqpoint{2.667030in}{0.771523in}}%
\pgfpathlineto{\pgfqpoint{2.658752in}{0.712543in}}%
\pgfpathlineto{\pgfqpoint{2.650176in}{0.666284in}}%
\pgfpathlineto{\pgfqpoint{2.640820in}{0.627931in}}%
\pgfpathlineto{\pgfqpoint{2.631145in}{0.597534in}}%
\pgfpathlineto{\pgfqpoint{2.621004in}{0.572745in}}%
\pgfpathlineto{\pgfqpoint{2.609856in}{0.551383in}}%
\pgfpathlineto{\pgfqpoint{2.598042in}{0.533534in}}%
\pgfpathlineto{\pgfqpoint{2.584496in}{0.517378in}}%
\pgfpathlineto{\pgfqpoint{2.571109in}{0.504669in}}%
\pgfpathlineto{\pgfqpoint{2.554789in}{0.492313in}}%
\pgfpathlineto{\pgfqpoint{2.537457in}{0.481914in}}%
\pgfpathlineto{\pgfqpoint{2.517374in}{0.472367in}}%
\pgfpathlineto{\pgfqpoint{2.492542in}{0.463178in}}%
\pgfpathlineto{\pgfqpoint{2.462979in}{0.454833in}}%
\pgfpathlineto{\pgfqpoint{2.428766in}{0.447542in}}%
\pgfpathlineto{\pgfqpoint{2.385671in}{0.440735in}}%
\pgfpathlineto{\pgfqpoint{2.331557in}{0.434581in}}%
\pgfpathlineto{\pgfqpoint{2.262115in}{0.429077in}}%
\pgfpathlineto{\pgfqpoint{2.170851in}{0.424236in}}%
\pgfpathlineto{\pgfqpoint{2.049086in}{0.420134in}}%
\pgfpathlineto{\pgfqpoint{1.879436in}{0.416783in}}%
\pgfpathlineto{\pgfqpoint{1.640159in}{0.414418in}}%
\pgfpathlineto{\pgfqpoint{1.322562in}{0.413569in}}%
\pgfpathlineto{\pgfqpoint{1.020194in}{0.414850in}}%
\pgfpathlineto{\pgfqpoint{0.822256in}{0.417715in}}%
\pgfpathlineto{\pgfqpoint{0.704835in}{0.421430in}}%
\pgfpathlineto{\pgfqpoint{0.630976in}{0.425829in}}%
\pgfpathlineto{\pgfqpoint{0.583316in}{0.430734in}}%
\pgfpathlineto{\pgfqpoint{0.551033in}{0.436123in}}%
\pgfpathlineto{\pgfqpoint{0.527708in}{0.442189in}}%
\pgfpathlineto{\pgfqpoint{0.511250in}{0.448625in}}%
\pgfpathlineto{\pgfqpoint{0.499549in}{0.455216in}}%
\pgfpathlineto{\pgfqpoint{0.488916in}{0.463841in}}%
\pgfpathlineto{\pgfqpoint{0.481322in}{0.472730in}}%
\pgfpathlineto{\pgfqpoint{0.474078in}{0.485127in}}%
\pgfpathlineto{\pgfqpoint{0.468753in}{0.498748in}}%
\pgfpathlineto{\pgfqpoint{0.463870in}{0.517848in}}%
\pgfpathlineto{\pgfqpoint{0.459679in}{0.544796in}}%
\pgfpathlineto{\pgfqpoint{0.456386in}{0.581938in}}%
\pgfpathlineto{\pgfqpoint{0.453731in}{0.639106in}}%
\pgfpathlineto{\pgfqpoint{0.451681in}{0.736155in}}%
\pgfpathlineto{\pgfqpoint{0.450220in}{0.927815in}}%
\pgfpathlineto{\pgfqpoint{0.449345in}{1.403252in}}%
\pgfpathlineto{\pgfqpoint{0.449543in}{2.682703in}}%
\pgfpathlineto{\pgfqpoint{0.451011in}{2.856932in}}%
\pgfpathlineto{\pgfqpoint{0.452802in}{2.879219in}}%
\pgfpathlineto{\pgfqpoint{0.455188in}{2.886108in}}%
\pgfpathlineto{\pgfqpoint{0.458626in}{2.889028in}}%
\pgfpathlineto{\pgfqpoint{0.464996in}{2.890553in}}%
\pgfpathlineto{\pgfqpoint{0.482377in}{2.891423in}}%
\pgfpathlineto{\pgfqpoint{0.565038in}{2.891729in}}%
\pgfpathlineto{\pgfqpoint{2.733842in}{2.891760in}}%
\pgfpathlineto{\pgfqpoint{4.789510in}{2.890885in}}%
\pgfpathlineto{\pgfqpoint{4.793727in}{2.889730in}}%
\pgfpathlineto{\pgfqpoint{4.795481in}{2.888307in}}%
\pgfpathlineto{\pgfqpoint{4.797106in}{2.881145in}}%
\pgfpathlineto{\pgfqpoint{4.797997in}{2.858771in}}%
\pgfpathlineto{\pgfqpoint{4.798039in}{2.856283in}}%
\pgfpathlineto{\pgfqpoint{4.798039in}{2.856283in}}%
\pgfusepath{stroke}%
\end{pgfscope}%
\begin{pgfscope}%
\pgfpathrectangle{\pgfqpoint{0.448634in}{0.402556in}}{\pgfqpoint{4.350661in}{2.489204in}} %
\pgfusepath{clip}%
\pgfsetbuttcap%
\pgfsetroundjoin%
\pgfsetlinewidth{1.003750pt}%
\definecolor{currentstroke}{rgb}{0.000000,0.000000,0.000000}%
\pgfsetstrokecolor{currentstroke}%
\pgfsetdash{{1.000000pt}{1.650000pt}}{0.000000pt}%
\pgfpathmoveto{\pgfqpoint{3.428772in}{0.402610in}}%
\pgfpathlineto{\pgfqpoint{2.806632in}{0.403760in}}%
\pgfpathlineto{\pgfqpoint{2.769692in}{0.405578in}}%
\pgfpathlineto{\pgfqpoint{2.754632in}{0.408064in}}%
\pgfpathlineto{\pgfqpoint{2.746391in}{0.411198in}}%
\pgfpathlineto{\pgfqpoint{2.740943in}{0.415265in}}%
\pgfpathlineto{\pgfqpoint{2.736784in}{0.420984in}}%
\pgfpathlineto{\pgfqpoint{2.733281in}{0.430071in}}%
\pgfpathlineto{\pgfqpoint{2.730449in}{0.444636in}}%
\pgfpathlineto{\pgfqpoint{2.728238in}{0.469392in}}%
\pgfpathlineto{\pgfqpoint{2.726470in}{0.519131in}}%
\pgfpathlineto{\pgfqpoint{2.725711in}{0.613715in}}%
\pgfpathlineto{\pgfqpoint{2.726842in}{0.768038in}}%
\pgfpathlineto{\pgfqpoint{2.730556in}{0.962148in}}%
\pgfpathlineto{\pgfqpoint{2.736611in}{1.158670in}}%
\pgfpathlineto{\pgfqpoint{2.744092in}{1.327718in}}%
\pgfpathlineto{\pgfqpoint{2.753201in}{1.484189in}}%
\pgfpathlineto{\pgfqpoint{2.763257in}{1.620609in}}%
\pgfpathlineto{\pgfqpoint{2.776118in}{1.764216in}}%
\pgfpathlineto{\pgfqpoint{2.788914in}{1.877776in}}%
\pgfpathlineto{\pgfqpoint{2.805748in}{2.005740in}}%
\pgfpathlineto{\pgfqpoint{2.821176in}{2.101198in}}%
\pgfpathlineto{\pgfqpoint{2.838359in}{2.193718in}}%
\pgfpathlineto{\pgfqpoint{2.859135in}{2.292966in}}%
\pgfpathlineto{\pgfqpoint{2.887209in}{2.425960in}}%
\pgfpathlineto{\pgfqpoint{2.896991in}{2.479559in}}%
\pgfpathlineto{\pgfqpoint{2.901543in}{2.516523in}}%
\pgfpathlineto{\pgfqpoint{2.902849in}{2.543854in}}%
\pgfpathlineto{\pgfqpoint{2.901957in}{2.566223in}}%
\pgfpathlineto{\pgfqpoint{2.899151in}{2.585863in}}%
\pgfpathlineto{\pgfqpoint{2.894794in}{2.602546in}}%
\pgfpathlineto{\pgfqpoint{2.888484in}{2.618388in}}%
\pgfpathlineto{\pgfqpoint{2.880257in}{2.633033in}}%
\pgfpathlineto{\pgfqpoint{2.870348in}{2.646246in}}%
\pgfpathlineto{\pgfqpoint{2.857400in}{2.659530in}}%
\pgfpathlineto{\pgfqpoint{2.843189in}{2.671010in}}%
\pgfpathlineto{\pgfqpoint{2.824237in}{2.683209in}}%
\pgfpathlineto{\pgfqpoint{2.802413in}{2.694418in}}%
\pgfpathlineto{\pgfqpoint{2.775809in}{2.705369in}}%
\pgfpathlineto{\pgfqpoint{2.744461in}{2.715715in}}%
\pgfpathlineto{\pgfqpoint{2.708436in}{2.725252in}}%
\pgfpathlineto{\pgfqpoint{2.665655in}{2.734289in}}%
\pgfpathlineto{\pgfqpoint{2.613991in}{2.742869in}}%
\pgfpathlineto{\pgfqpoint{2.553459in}{2.750589in}}%
\pgfpathlineto{\pgfqpoint{2.481920in}{2.757365in}}%
\pgfpathlineto{\pgfqpoint{2.399398in}{2.762839in}}%
\pgfpathlineto{\pgfqpoint{2.310269in}{2.766482in}}%
\pgfpathlineto{\pgfqpoint{2.175416in}{2.768725in}}%
\pgfpathlineto{\pgfqpoint{2.066653in}{2.767942in}}%
\pgfpathlineto{\pgfqpoint{1.953570in}{2.764859in}}%
\pgfpathlineto{\pgfqpoint{1.851429in}{2.759759in}}%
\pgfpathlineto{\pgfqpoint{1.745051in}{2.752169in}}%
\pgfpathlineto{\pgfqpoint{1.658373in}{2.743453in}}%
\pgfpathlineto{\pgfqpoint{1.580552in}{2.733461in}}%
\pgfpathlineto{\pgfqpoint{1.490057in}{2.719338in}}%
\pgfpathlineto{\pgfqpoint{1.417231in}{2.704698in}}%
\pgfpathlineto{\pgfqpoint{1.361992in}{2.690818in}}%
\pgfpathlineto{\pgfqpoint{1.311460in}{2.675819in}}%
\pgfpathlineto{\pgfqpoint{1.265667in}{2.659924in}}%
\pgfpathlineto{\pgfqpoint{1.222575in}{2.642586in}}%
\pgfpathlineto{\pgfqpoint{1.184324in}{2.624682in}}%
\pgfpathlineto{\pgfqpoint{1.148892in}{2.605623in}}%
\pgfpathlineto{\pgfqpoint{1.116331in}{2.585573in}}%
\pgfpathlineto{\pgfqpoint{1.092327in}{2.568512in}}%
\pgfpathlineto{\pgfqpoint{1.079760in}{2.558686in}}%
\pgfpathlineto{\pgfqpoint{1.051544in}{2.535379in}}%
\pgfpathlineto{\pgfqpoint{1.026312in}{2.511712in}}%
\pgfpathlineto{\pgfqpoint{1.002399in}{2.486318in}}%
\pgfpathlineto{\pgfqpoint{0.979913in}{2.459269in}}%
\pgfpathlineto{\pgfqpoint{0.958934in}{2.430678in}}%
\pgfpathlineto{\pgfqpoint{0.938264in}{2.398643in}}%
\pgfpathlineto{\pgfqpoint{0.923047in}{2.371385in}}%
\pgfpathlineto{\pgfqpoint{0.904513in}{2.334774in}}%
\pgfpathlineto{\pgfqpoint{0.887854in}{2.297001in}}%
\pgfpathlineto{\pgfqpoint{0.872131in}{2.255971in}}%
\pgfpathlineto{\pgfqpoint{0.857508in}{2.211741in}}%
\pgfpathlineto{\pgfqpoint{0.844762in}{2.166757in}}%
\pgfpathlineto{\pgfqpoint{0.838624in}{2.140306in}}%
\pgfpathlineto{\pgfqpoint{0.826982in}{2.087194in}}%
\pgfpathlineto{\pgfqpoint{0.816322in}{2.028715in}}%
\pgfpathlineto{\pgfqpoint{0.810087in}{1.984495in}}%
\pgfpathlineto{\pgfqpoint{0.808026in}{1.967238in}}%
\pgfpathlineto{\pgfqpoint{0.800076in}{1.898140in}}%
\pgfpathlineto{\pgfqpoint{0.793713in}{1.823823in}}%
\pgfpathlineto{\pgfqpoint{0.788799in}{1.741875in}}%
\pgfpathlineto{\pgfqpoint{0.786199in}{1.677225in}}%
\pgfpathlineto{\pgfqpoint{0.776951in}{1.453481in}}%
\pgfpathlineto{\pgfqpoint{0.773280in}{1.418894in}}%
\pgfpathlineto{\pgfqpoint{0.768298in}{1.389582in}}%
\pgfpathlineto{\pgfqpoint{0.762752in}{1.368108in}}%
\pgfpathlineto{\pgfqpoint{0.756722in}{1.352123in}}%
\pgfpathlineto{\pgfqpoint{0.749752in}{1.339519in}}%
\pgfpathlineto{\pgfqpoint{0.742201in}{1.330599in}}%
\pgfpathlineto{\pgfqpoint{0.734854in}{1.325312in}}%
\pgfpathlineto{\pgfqpoint{0.726558in}{1.322419in}}%
\pgfpathlineto{\pgfqpoint{0.717884in}{1.322223in}}%
\pgfpathlineto{\pgfqpoint{0.709412in}{1.324411in}}%
\pgfpathlineto{\pgfqpoint{0.699548in}{1.329604in}}%
\pgfpathlineto{\pgfqpoint{0.688894in}{1.338203in}}%
\pgfpathlineto{\pgfqpoint{0.677907in}{1.350248in}}%
\pgfpathlineto{\pgfqpoint{0.666886in}{1.365647in}}%
\pgfpathlineto{\pgfqpoint{0.654913in}{1.386417in}}%
\pgfpathlineto{\pgfqpoint{0.642574in}{1.412730in}}%
\pgfpathlineto{\pgfqpoint{0.630328in}{1.444629in}}%
\pgfpathlineto{\pgfqpoint{0.618504in}{1.482081in}}%
\pgfpathlineto{\pgfqpoint{0.608613in}{1.520256in}}%
\pgfpathlineto{\pgfqpoint{0.590203in}{1.612445in}}%
\pgfpathlineto{\pgfqpoint{0.581848in}{1.668884in}}%
\pgfpathlineto{\pgfqpoint{0.573137in}{1.740376in}}%
\pgfpathlineto{\pgfqpoint{0.567062in}{1.807213in}}%
\pgfpathlineto{\pgfqpoint{0.560532in}{1.896510in}}%
\pgfpathlineto{\pgfqpoint{0.555526in}{1.995910in}}%
\pgfpathlineto{\pgfqpoint{0.552564in}{2.097908in}}%
\pgfpathlineto{\pgfqpoint{0.551526in}{2.204935in}}%
\pgfpathlineto{\pgfqpoint{0.552728in}{2.309470in}}%
\pgfpathlineto{\pgfqpoint{0.556011in}{2.403981in}}%
\pgfpathlineto{\pgfqpoint{0.560953in}{2.483430in}}%
\pgfpathlineto{\pgfqpoint{0.567303in}{2.550240in}}%
\pgfpathlineto{\pgfqpoint{0.574928in}{2.606817in}}%
\pgfpathlineto{\pgfqpoint{0.582988in}{2.650657in}}%
\pgfpathlineto{\pgfqpoint{0.592756in}{2.691452in}}%
\pgfpathlineto{\pgfqpoint{0.602650in}{2.721756in}}%
\pgfpathlineto{\pgfqpoint{0.612983in}{2.746441in}}%
\pgfpathlineto{\pgfqpoint{0.624292in}{2.767692in}}%
\pgfpathlineto{\pgfqpoint{0.636231in}{2.785433in}}%
\pgfpathlineto{\pgfqpoint{0.649892in}{2.801461in}}%
\pgfpathlineto{\pgfqpoint{0.663386in}{2.814020in}}%
\pgfpathlineto{\pgfqpoint{0.679842in}{2.826135in}}%
\pgfpathlineto{\pgfqpoint{0.697326in}{2.836197in}}%
\pgfpathlineto{\pgfqpoint{0.715574in}{2.844285in}}%
\pgfpathlineto{\pgfqpoint{0.738439in}{2.852335in}}%
\pgfpathlineto{\pgfqpoint{0.765983in}{2.859639in}}%
\pgfpathlineto{\pgfqpoint{0.800300in}{2.866256in}}%
\pgfpathlineto{\pgfqpoint{0.841340in}{2.871832in}}%
\pgfpathlineto{\pgfqpoint{0.895547in}{2.876803in}}%
\pgfpathlineto{\pgfqpoint{0.969413in}{2.881069in}}%
\pgfpathlineto{\pgfqpoint{1.071608in}{2.884501in}}%
\pgfpathlineto{\pgfqpoint{1.219512in}{2.887074in}}%
\pgfpathlineto{\pgfqpoint{1.471844in}{2.889091in}}%
\pgfpathlineto{\pgfqpoint{1.956941in}{2.890384in}}%
\pgfpathlineto{\pgfqpoint{3.096814in}{2.890781in}}%
\pgfpathlineto{\pgfqpoint{3.995224in}{2.889388in}}%
\pgfpathlineto{\pgfqpoint{4.275833in}{2.887011in}}%
\pgfpathlineto{\pgfqpoint{4.412847in}{2.883743in}}%
\pgfpathlineto{\pgfqpoint{4.491081in}{2.879810in}}%
\pgfpathlineto{\pgfqpoint{4.543127in}{2.875163in}}%
\pgfpathlineto{\pgfqpoint{4.579810in}{2.869841in}}%
\pgfpathlineto{\pgfqpoint{4.607580in}{2.863763in}}%
\pgfpathlineto{\pgfqpoint{4.630623in}{2.856424in}}%
\pgfpathlineto{\pgfqpoint{4.648833in}{2.848228in}}%
\pgfpathlineto{\pgfqpoint{4.664136in}{2.838773in}}%
\pgfpathlineto{\pgfqpoint{4.676470in}{2.828576in}}%
\pgfpathlineto{\pgfqpoint{4.687502in}{2.816585in}}%
\pgfpathlineto{\pgfqpoint{4.697051in}{2.803027in}}%
\pgfpathlineto{\pgfqpoint{4.706194in}{2.786098in}}%
\pgfpathlineto{\pgfqpoint{4.714508in}{2.765827in}}%
\pgfpathlineto{\pgfqpoint{4.722462in}{2.740013in}}%
\pgfpathlineto{\pgfqpoint{4.729577in}{2.708703in}}%
\pgfpathlineto{\pgfqpoint{4.736162in}{2.669601in}}%
\pgfpathlineto{\pgfqpoint{4.742419in}{2.617826in}}%
\pgfpathlineto{\pgfqpoint{4.747859in}{2.553410in}}%
\pgfpathlineto{\pgfqpoint{4.752661in}{2.468958in}}%
\pgfpathlineto{\pgfqpoint{4.756610in}{2.359528in}}%
\pgfpathlineto{\pgfqpoint{4.759416in}{2.217681in}}%
\pgfpathlineto{\pgfqpoint{4.760596in}{2.043444in}}%
\pgfpathlineto{\pgfqpoint{4.759662in}{1.851779in}}%
\pgfpathlineto{\pgfqpoint{4.756587in}{1.667613in}}%
\pgfpathlineto{\pgfqpoint{4.751596in}{1.503428in}}%
\pgfpathlineto{\pgfqpoint{4.745410in}{1.374185in}}%
\pgfpathlineto{\pgfqpoint{4.738113in}{1.267479in}}%
\pgfpathlineto{\pgfqpoint{4.729621in}{1.175896in}}%
\pgfpathlineto{\pgfqpoint{4.720762in}{1.104428in}}%
\pgfpathlineto{\pgfqpoint{4.711045in}{1.043204in}}%
\pgfpathlineto{\pgfqpoint{4.700364in}{0.989829in}}%
\pgfpathlineto{\pgfqpoint{4.689055in}{0.944345in}}%
\pgfpathlineto{\pgfqpoint{4.676881in}{0.904394in}}%
\pgfpathlineto{\pgfqpoint{4.676095in}{0.902073in}}%
\pgfpathlineto{\pgfqpoint{4.676095in}{0.902073in}}%
\pgfusepath{stroke}%
\end{pgfscope}%
\begin{pgfscope}%
\pgfpathrectangle{\pgfqpoint{0.448634in}{0.402556in}}{\pgfqpoint{4.350661in}{2.489204in}} %
\pgfusepath{clip}%
\pgfsetbuttcap%
\pgfsetroundjoin%
\pgfsetlinewidth{1.003750pt}%
\definecolor{currentstroke}{rgb}{0.000000,0.000000,0.000000}%
\pgfsetstrokecolor{currentstroke}%
\pgfsetdash{{1.000000pt}{1.650000pt}}{0.000000pt}%
\pgfpathmoveto{\pgfqpoint{2.795520in}{1.982745in}}%
\pgfpathlineto{\pgfqpoint{2.781780in}{1.874357in}}%
\pgfpathlineto{\pgfqpoint{2.769351in}{1.758234in}}%
\pgfpathlineto{\pgfqpoint{2.758095in}{1.631942in}}%
\pgfpathlineto{\pgfqpoint{2.747786in}{1.490551in}}%
\pgfpathlineto{\pgfqpoint{2.738644in}{1.334082in}}%
\pgfpathlineto{\pgfqpoint{2.730580in}{1.157591in}}%
\pgfpathlineto{\pgfqpoint{2.723334in}{0.948663in}}%
\pgfpathlineto{\pgfqpoint{2.709783in}{0.530788in}}%
\pgfpathlineto{\pgfqpoint{2.705868in}{0.488716in}}%
\pgfpathlineto{\pgfqpoint{2.701769in}{0.464281in}}%
\pgfpathlineto{\pgfqpoint{2.697021in}{0.447744in}}%
\pgfpathlineto{\pgfqpoint{2.691859in}{0.436812in}}%
\pgfpathlineto{\pgfqpoint{2.686245in}{0.429229in}}%
\pgfpathlineto{\pgfqpoint{2.679348in}{0.423188in}}%
\pgfpathlineto{\pgfqpoint{2.669540in}{0.417856in}}%
\pgfpathlineto{\pgfqpoint{2.656987in}{0.413810in}}%
\pgfpathlineto{\pgfqpoint{2.637654in}{0.410337in}}%
\pgfpathlineto{\pgfqpoint{2.607297in}{0.407617in}}%
\pgfpathlineto{\pgfqpoint{2.555121in}{0.405574in}}%
\pgfpathlineto{\pgfqpoint{2.450714in}{0.404139in}}%
\pgfpathlineto{\pgfqpoint{2.176624in}{0.403275in}}%
\pgfpathlineto{\pgfqpoint{1.130290in}{0.402953in}}%
\pgfpathlineto{\pgfqpoint{0.516849in}{0.404175in}}%
\pgfpathlineto{\pgfqpoint{0.466848in}{0.405970in}}%
\pgfpathlineto{\pgfqpoint{0.456130in}{0.407931in}}%
\pgfpathlineto{\pgfqpoint{0.452340in}{0.410303in}}%
\pgfpathlineto{\pgfqpoint{0.450346in}{0.414662in}}%
\pgfpathlineto{\pgfqpoint{0.449266in}{0.424524in}}%
\pgfpathlineto{\pgfqpoint{0.448771in}{0.464344in}}%
\pgfpathlineto{\pgfqpoint{0.448640in}{0.850171in}}%
\pgfpathlineto{\pgfqpoint{0.448653in}{2.891318in}}%
\pgfpathlineto{\pgfqpoint{0.448653in}{2.891318in}}%
\pgfusepath{stroke}%
\end{pgfscope}%
\begin{pgfscope}%
\pgfpathrectangle{\pgfqpoint{0.448634in}{0.402556in}}{\pgfqpoint{4.350661in}{2.489204in}} %
\pgfusepath{clip}%
\pgfsetbuttcap%
\pgfsetroundjoin%
\pgfsetlinewidth{1.003750pt}%
\definecolor{currentstroke}{rgb}{0.000000,0.000000,0.000000}%
\pgfsetstrokecolor{currentstroke}%
\pgfsetdash{{1.000000pt}{1.650000pt}}{0.000000pt}%
\pgfpathmoveto{\pgfqpoint{3.428189in}{0.402586in}}%
\pgfpathlineto{\pgfqpoint{2.782121in}{0.403701in}}%
\pgfpathlineto{\pgfqpoint{2.753906in}{0.405674in}}%
\pgfpathlineto{\pgfqpoint{2.743328in}{0.408443in}}%
\pgfpathlineto{\pgfqpoint{2.737717in}{0.412188in}}%
\pgfpathlineto{\pgfqpoint{2.733668in}{0.417995in}}%
\pgfpathlineto{\pgfqpoint{2.730649in}{0.427307in}}%
\pgfpathlineto{\pgfqpoint{2.728388in}{0.442004in}}%
\pgfpathlineto{\pgfqpoint{2.726544in}{0.471794in}}%
\pgfpathlineto{\pgfqpoint{2.725216in}{0.534003in}}%
\pgfpathlineto{\pgfqpoint{2.725169in}{0.655973in}}%
\pgfpathlineto{\pgfqpoint{2.727377in}{0.832687in}}%
\pgfpathlineto{\pgfqpoint{2.732259in}{1.041703in}}%
\pgfpathlineto{\pgfqpoint{2.738851in}{1.223257in}}%
\pgfpathlineto{\pgfqpoint{2.747078in}{1.389766in}}%
\pgfpathlineto{\pgfqpoint{2.756608in}{1.538717in}}%
\pgfpathlineto{\pgfqpoint{2.768955in}{1.694887in}}%
\pgfpathlineto{\pgfqpoint{2.781228in}{1.816044in}}%
\pgfpathlineto{\pgfqpoint{2.794401in}{1.924524in}}%
\pgfpathlineto{\pgfqpoint{2.812737in}{2.054722in}}%
\pgfpathlineto{\pgfqpoint{2.828774in}{2.147512in}}%
\pgfpathlineto{\pgfqpoint{2.847382in}{2.242224in}}%
\pgfpathlineto{\pgfqpoint{2.895818in}{2.479699in}}%
\pgfpathlineto{\pgfqpoint{2.900204in}{2.516689in}}%
\pgfpathlineto{\pgfqpoint{2.901346in}{2.544029in}}%
\pgfpathlineto{\pgfqpoint{2.900291in}{2.566388in}}%
\pgfpathlineto{\pgfqpoint{2.897334in}{2.585999in}}%
\pgfpathlineto{\pgfqpoint{2.892836in}{2.602633in}}%
\pgfpathlineto{\pgfqpoint{2.886394in}{2.618405in}}%
\pgfpathlineto{\pgfqpoint{2.878058in}{2.632969in}}%
\pgfpathlineto{\pgfqpoint{2.868065in}{2.646100in}}%
\pgfpathlineto{\pgfqpoint{2.855050in}{2.659300in}}%
\pgfpathlineto{\pgfqpoint{2.840801in}{2.670717in}}%
\pgfpathlineto{\pgfqpoint{2.821822in}{2.682861in}}%
\pgfpathlineto{\pgfqpoint{2.799980in}{2.694026in}}%
\pgfpathlineto{\pgfqpoint{2.773366in}{2.704944in}}%
\pgfpathlineto{\pgfqpoint{2.742012in}{2.715266in}}%
\pgfpathlineto{\pgfqpoint{2.705983in}{2.724785in}}%
\pgfpathlineto{\pgfqpoint{2.663200in}{2.733810in}}%
\pgfpathlineto{\pgfqpoint{2.611535in}{2.742379in}}%
\pgfpathlineto{\pgfqpoint{2.551002in}{2.750090in}}%
\pgfpathlineto{\pgfqpoint{2.481632in}{2.756682in}}%
\pgfpathlineto{\pgfqpoint{2.399112in}{2.762200in}}%
\pgfpathlineto{\pgfqpoint{2.309985in}{2.765886in}}%
\pgfpathlineto{\pgfqpoint{2.188184in}{2.768096in}}%
\pgfpathlineto{\pgfqpoint{2.081595in}{2.767619in}}%
\pgfpathlineto{\pgfqpoint{1.968506in}{2.764840in}}%
\pgfpathlineto{\pgfqpoint{1.864180in}{2.759918in}}%
\pgfpathlineto{\pgfqpoint{1.757786in}{2.752593in}}%
\pgfpathlineto{\pgfqpoint{1.671087in}{2.744171in}}%
\pgfpathlineto{\pgfqpoint{1.591076in}{2.734193in}}%
\pgfpathlineto{\pgfqpoint{1.502689in}{2.720717in}}%
\pgfpathlineto{\pgfqpoint{1.427655in}{2.706083in}}%
\pgfpathlineto{\pgfqpoint{1.372350in}{2.692544in}}%
\pgfpathlineto{\pgfqpoint{1.321734in}{2.677921in}}%
\pgfpathlineto{\pgfqpoint{1.273765in}{2.661664in}}%
\pgfpathlineto{\pgfqpoint{1.230567in}{2.644672in}}%
\pgfpathlineto{\pgfqpoint{1.192197in}{2.627106in}}%
\pgfpathlineto{\pgfqpoint{1.156620in}{2.608403in}}%
\pgfpathlineto{\pgfqpoint{1.123890in}{2.588716in}}%
\pgfpathlineto{\pgfqpoint{1.095883in}{2.569568in}}%
\pgfpathlineto{\pgfqpoint{1.063936in}{2.543701in}}%
\pgfpathlineto{\pgfqpoint{1.038217in}{2.520732in}}%
\pgfpathlineto{\pgfqpoint{1.013766in}{2.496016in}}%
\pgfpathlineto{\pgfqpoint{0.990704in}{2.469610in}}%
\pgfpathlineto{\pgfqpoint{0.969124in}{2.441612in}}%
\pgfpathlineto{\pgfqpoint{0.949083in}{2.412154in}}%
\pgfpathlineto{\pgfqpoint{0.930604in}{2.381387in}}%
\pgfpathlineto{\pgfqpoint{0.906555in}{2.334052in}}%
\pgfpathlineto{\pgfqpoint{0.889925in}{2.296262in}}%
\pgfpathlineto{\pgfqpoint{0.874241in}{2.255213in}}%
\pgfpathlineto{\pgfqpoint{0.859667in}{2.210961in}}%
\pgfpathlineto{\pgfqpoint{0.846986in}{2.165954in}}%
\pgfpathlineto{\pgfqpoint{0.839633in}{2.134715in}}%
\pgfpathlineto{\pgfqpoint{0.828238in}{2.081532in}}%
\pgfpathlineto{\pgfqpoint{0.817866in}{2.022986in}}%
\pgfpathlineto{\pgfqpoint{0.810784in}{1.971352in}}%
\pgfpathlineto{\pgfqpoint{0.802846in}{1.902252in}}%
\pgfpathlineto{\pgfqpoint{0.796554in}{1.827927in}}%
\pgfpathlineto{\pgfqpoint{0.791696in}{1.743480in}}%
\pgfpathlineto{\pgfqpoint{0.787773in}{1.621595in}}%
\pgfpathlineto{\pgfqpoint{0.785408in}{1.522064in}}%
\pgfpathlineto{\pgfqpoint{0.785408in}{1.522064in}}%
\pgfusepath{stroke}%
\end{pgfscope}%
\begin{pgfscope}%
\pgfpathrectangle{\pgfqpoint{0.448634in}{0.402556in}}{\pgfqpoint{4.350661in}{2.489204in}} %
\pgfusepath{clip}%
\pgfsetbuttcap%
\pgfsetroundjoin%
\pgfsetlinewidth{1.003750pt}%
\definecolor{currentstroke}{rgb}{0.000000,0.000000,0.000000}%
\pgfsetstrokecolor{currentstroke}%
\pgfsetdash{{1.000000pt}{1.650000pt}}{0.000000pt}%
\pgfpathmoveto{\pgfqpoint{2.028735in}{0.425754in}}%
\pgfpathlineto{\pgfqpoint{1.878677in}{0.421879in}}%
\pgfpathlineto{\pgfqpoint{1.676387in}{0.418997in}}%
\pgfpathlineto{\pgfqpoint{1.413176in}{0.417558in}}%
\pgfpathlineto{\pgfqpoint{1.134735in}{0.418204in}}%
\pgfpathlineto{\pgfqpoint{0.921565in}{0.420769in}}%
\pgfpathlineto{\pgfqpoint{0.782384in}{0.424523in}}%
\pgfpathlineto{\pgfqpoint{0.693283in}{0.428974in}}%
\pgfpathlineto{\pgfqpoint{0.632541in}{0.434091in}}%
\pgfpathlineto{\pgfqpoint{0.591492in}{0.439564in}}%
\pgfpathlineto{\pgfqpoint{0.561503in}{0.445595in}}%
\pgfpathlineto{\pgfqpoint{0.538349in}{0.452466in}}%
\pgfpathlineto{\pgfqpoint{0.522042in}{0.459394in}}%
\pgfpathlineto{\pgfqpoint{0.508540in}{0.467420in}}%
\pgfpathlineto{\pgfqpoint{0.497973in}{0.476161in}}%
\pgfpathlineto{\pgfqpoint{0.488790in}{0.486750in}}%
\pgfpathlineto{\pgfqpoint{0.481284in}{0.498948in}}%
\pgfpathlineto{\pgfqpoint{0.474590in}{0.514580in}}%
\pgfpathlineto{\pgfqpoint{0.469106in}{0.533467in}}%
\pgfpathlineto{\pgfqpoint{0.464439in}{0.557771in}}%
\pgfpathlineto{\pgfqpoint{0.460297in}{0.592289in}}%
\pgfpathlineto{\pgfqpoint{0.456856in}{0.641912in}}%
\pgfpathlineto{\pgfqpoint{0.454122in}{0.716520in}}%
\pgfpathlineto{\pgfqpoint{0.451978in}{0.843444in}}%
\pgfpathlineto{\pgfqpoint{0.450459in}{1.087380in}}%
\pgfpathlineto{\pgfqpoint{0.449596in}{1.657406in}}%
\pgfpathlineto{\pgfqpoint{0.450150in}{2.687936in}}%
\pgfpathlineto{\pgfqpoint{0.451781in}{2.839761in}}%
\pgfpathlineto{\pgfqpoint{0.453975in}{2.872003in}}%
\pgfpathlineto{\pgfqpoint{0.456339in}{2.881553in}}%
\pgfpathlineto{\pgfqpoint{0.458888in}{2.885549in}}%
\pgfpathlineto{\pgfqpoint{0.462554in}{2.888171in}}%
\pgfpathlineto{\pgfqpoint{0.471046in}{2.890205in}}%
\pgfpathlineto{\pgfqpoint{0.490597in}{2.891263in}}%
\pgfpathlineto{\pgfqpoint{0.564556in}{2.891692in}}%
\pgfpathlineto{\pgfqpoint{1.569559in}{2.891759in}}%
\pgfpathlineto{\pgfqpoint{4.784679in}{2.890785in}}%
\pgfpathlineto{\pgfqpoint{4.791005in}{2.889098in}}%
\pgfpathlineto{\pgfqpoint{4.793910in}{2.885555in}}%
\pgfpathlineto{\pgfqpoint{4.795579in}{2.878366in}}%
\pgfpathlineto{\pgfqpoint{4.796850in}{2.858513in}}%
\pgfpathlineto{\pgfqpoint{4.796850in}{2.858513in}}%
\pgfusepath{stroke}%
\end{pgfscope}%
\begin{pgfscope}%
\pgfsetrectcap%
\pgfsetmiterjoin%
\pgfsetlinewidth{0.803000pt}%
\definecolor{currentstroke}{rgb}{0.000000,0.000000,0.000000}%
\pgfsetstrokecolor{currentstroke}%
\pgfsetdash{}{0pt}%
\pgfpathmoveto{\pgfqpoint{0.448634in}{0.402556in}}%
\pgfpathlineto{\pgfqpoint{0.448634in}{2.891760in}}%
\pgfusepath{stroke}%
\end{pgfscope}%
\begin{pgfscope}%
\pgfsetrectcap%
\pgfsetmiterjoin%
\pgfsetlinewidth{0.803000pt}%
\definecolor{currentstroke}{rgb}{0.000000,0.000000,0.000000}%
\pgfsetstrokecolor{currentstroke}%
\pgfsetdash{}{0pt}%
\pgfpathmoveto{\pgfqpoint{4.799294in}{0.402556in}}%
\pgfpathlineto{\pgfqpoint{4.799294in}{2.891760in}}%
\pgfusepath{stroke}%
\end{pgfscope}%
\begin{pgfscope}%
\pgfsetrectcap%
\pgfsetmiterjoin%
\pgfsetlinewidth{0.803000pt}%
\definecolor{currentstroke}{rgb}{0.000000,0.000000,0.000000}%
\pgfsetstrokecolor{currentstroke}%
\pgfsetdash{}{0pt}%
\pgfpathmoveto{\pgfqpoint{0.448634in}{0.402556in}}%
\pgfpathlineto{\pgfqpoint{4.799294in}{0.402556in}}%
\pgfusepath{stroke}%
\end{pgfscope}%
\begin{pgfscope}%
\pgfsetrectcap%
\pgfsetmiterjoin%
\pgfsetlinewidth{0.803000pt}%
\definecolor{currentstroke}{rgb}{0.000000,0.000000,0.000000}%
\pgfsetstrokecolor{currentstroke}%
\pgfsetdash{}{0pt}%
\pgfpathmoveto{\pgfqpoint{0.448634in}{2.891760in}}%
\pgfpathlineto{\pgfqpoint{4.799294in}{2.891760in}}%
\pgfusepath{stroke}%
\end{pgfscope}%
\begin{pgfscope}%
\pgfsetbuttcap%
\pgfsetmiterjoin%
\definecolor{currentfill}{rgb}{1.000000,1.000000,1.000000}%
\pgfsetfillcolor{currentfill}%
\pgfsetfillopacity{0.500000}%
\pgfsetlinewidth{1.003750pt}%
\definecolor{currentstroke}{rgb}{0.800000,0.800000,0.800000}%
\pgfsetstrokecolor{currentstroke}%
\pgfsetstrokeopacity{0.500000}%
\pgfsetdash{}{0pt}%
\pgfpathmoveto{\pgfqpoint{3.700085in}{0.761312in}}%
\pgfpathlineto{\pgfqpoint{4.547129in}{0.761312in}}%
\pgfpathquadraticcurveto{\pgfqpoint{4.574907in}{0.761312in}}{\pgfqpoint{4.574907in}{0.789090in}}%
\pgfpathlineto{\pgfqpoint{4.574907in}{2.769646in}}%
\pgfpathquadraticcurveto{\pgfqpoint{4.574907in}{2.797424in}}{\pgfqpoint{4.547129in}{2.797424in}}%
\pgfpathlineto{\pgfqpoint{3.700085in}{2.797424in}}%
\pgfpathquadraticcurveto{\pgfqpoint{3.672307in}{2.797424in}}{\pgfqpoint{3.672307in}{2.769646in}}%
\pgfpathlineto{\pgfqpoint{3.672307in}{0.789090in}}%
\pgfpathquadraticcurveto{\pgfqpoint{3.672307in}{0.761312in}}{\pgfqpoint{3.700085in}{0.761312in}}%
\pgfpathclose%
\pgfusepath{stroke,fill}%
\end{pgfscope}%
\begin{pgfscope}%
\pgfsetrectcap%
\pgfsetroundjoin%
\pgfsetlinewidth{1.003750pt}%
\definecolor{currentstroke}{rgb}{0.121569,0.466667,0.705882}%
\pgfsetstrokecolor{currentstroke}%
\pgfsetdash{}{0pt}%
\pgfpathmoveto{\pgfqpoint{3.727863in}{2.693257in}}%
\pgfpathlineto{\pgfqpoint{3.797307in}{2.693257in}}%
\pgfusepath{stroke}%
\end{pgfscope}%
\begin{pgfscope}%
\pgftext[x=3.908418in,y=2.644646in,left,base]{\rmfamily\fontsize{10.000000}{12.000000}\selectfont \(\displaystyle \textnormal{tol}=10^{{-}10}\)}%
\end{pgfscope}%
\begin{pgfscope}%
\pgfsetrectcap%
\pgfsetroundjoin%
\pgfsetlinewidth{1.003750pt}%
\definecolor{currentstroke}{rgb}{1.000000,0.498039,0.054902}%
\pgfsetstrokecolor{currentstroke}%
\pgfsetdash{}{0pt}%
\pgfpathmoveto{\pgfqpoint{3.727863in}{2.493812in}}%
\pgfpathlineto{\pgfqpoint{3.797307in}{2.493812in}}%
\pgfusepath{stroke}%
\end{pgfscope}%
\begin{pgfscope}%
\pgftext[x=3.908418in,y=2.445201in,left,base]{\rmfamily\fontsize{10.000000}{12.000000}\selectfont \(\displaystyle \textnormal{tol}=10^{{-}9}\)}%
\end{pgfscope}%
\begin{pgfscope}%
\pgfsetrectcap%
\pgfsetroundjoin%
\pgfsetlinewidth{1.003750pt}%
\definecolor{currentstroke}{rgb}{0.172549,0.627451,0.172549}%
\pgfsetstrokecolor{currentstroke}%
\pgfsetdash{}{0pt}%
\pgfpathmoveto{\pgfqpoint{3.727863in}{2.294368in}}%
\pgfpathlineto{\pgfqpoint{3.797307in}{2.294368in}}%
\pgfusepath{stroke}%
\end{pgfscope}%
\begin{pgfscope}%
\pgftext[x=3.908418in,y=2.245757in,left,base]{\rmfamily\fontsize{10.000000}{12.000000}\selectfont \(\displaystyle \textnormal{tol}=10^{{-}8}\)}%
\end{pgfscope}%
\begin{pgfscope}%
\pgfsetrectcap%
\pgfsetroundjoin%
\pgfsetlinewidth{1.003750pt}%
\definecolor{currentstroke}{rgb}{0.839216,0.152941,0.156863}%
\pgfsetstrokecolor{currentstroke}%
\pgfsetdash{}{0pt}%
\pgfpathmoveto{\pgfqpoint{3.727863in}{2.094924in}}%
\pgfpathlineto{\pgfqpoint{3.797307in}{2.094924in}}%
\pgfusepath{stroke}%
\end{pgfscope}%
\begin{pgfscope}%
\pgftext[x=3.908418in,y=2.046312in,left,base]{\rmfamily\fontsize{10.000000}{12.000000}\selectfont \(\displaystyle \textnormal{tol}=10^{{-}7}\)}%
\end{pgfscope}%
\begin{pgfscope}%
\pgfsetrectcap%
\pgfsetroundjoin%
\pgfsetlinewidth{1.003750pt}%
\definecolor{currentstroke}{rgb}{0.580392,0.403922,0.741176}%
\pgfsetstrokecolor{currentstroke}%
\pgfsetdash{}{0pt}%
\pgfpathmoveto{\pgfqpoint{3.727863in}{1.895479in}}%
\pgfpathlineto{\pgfqpoint{3.797307in}{1.895479in}}%
\pgfusepath{stroke}%
\end{pgfscope}%
\begin{pgfscope}%
\pgftext[x=3.908418in,y=1.846868in,left,base]{\rmfamily\fontsize{10.000000}{12.000000}\selectfont \(\displaystyle \textnormal{tol}=10^{{-}6}\)}%
\end{pgfscope}%
\begin{pgfscope}%
\pgfsetrectcap%
\pgfsetroundjoin%
\pgfsetlinewidth{1.003750pt}%
\definecolor{currentstroke}{rgb}{0.549020,0.337255,0.294118}%
\pgfsetstrokecolor{currentstroke}%
\pgfsetdash{}{0pt}%
\pgfpathmoveto{\pgfqpoint{3.727863in}{1.696035in}}%
\pgfpathlineto{\pgfqpoint{3.797307in}{1.696035in}}%
\pgfusepath{stroke}%
\end{pgfscope}%
\begin{pgfscope}%
\pgftext[x=3.908418in,y=1.647424in,left,base]{\rmfamily\fontsize{10.000000}{12.000000}\selectfont \(\displaystyle \textnormal{tol}=10^{{-}5}\)}%
\end{pgfscope}%
\begin{pgfscope}%
\pgfsetrectcap%
\pgfsetroundjoin%
\pgfsetlinewidth{1.003750pt}%
\definecolor{currentstroke}{rgb}{0.890196,0.466667,0.760784}%
\pgfsetstrokecolor{currentstroke}%
\pgfsetdash{}{0pt}%
\pgfpathmoveto{\pgfqpoint{3.727863in}{1.496590in}}%
\pgfpathlineto{\pgfqpoint{3.797307in}{1.496590in}}%
\pgfusepath{stroke}%
\end{pgfscope}%
\begin{pgfscope}%
\pgftext[x=3.908418in,y=1.447979in,left,base]{\rmfamily\fontsize{10.000000}{12.000000}\selectfont \(\displaystyle \textnormal{tol}=10^{{-}4}\)}%
\end{pgfscope}%
\begin{pgfscope}%
\pgfsetrectcap%
\pgfsetroundjoin%
\pgfsetlinewidth{1.003750pt}%
\definecolor{currentstroke}{rgb}{0.498039,0.498039,0.498039}%
\pgfsetstrokecolor{currentstroke}%
\pgfsetdash{}{0pt}%
\pgfpathmoveto{\pgfqpoint{3.727863in}{1.297146in}}%
\pgfpathlineto{\pgfqpoint{3.797307in}{1.297146in}}%
\pgfusepath{stroke}%
\end{pgfscope}%
\begin{pgfscope}%
\pgftext[x=3.908418in,y=1.248535in,left,base]{\rmfamily\fontsize{10.000000}{12.000000}\selectfont \(\displaystyle \textnormal{tol}=10^{{-}3}\)}%
\end{pgfscope}%
\begin{pgfscope}%
\pgfsetrectcap%
\pgfsetroundjoin%
\pgfsetlinewidth{1.003750pt}%
\definecolor{currentstroke}{rgb}{0.737255,0.741176,0.133333}%
\pgfsetstrokecolor{currentstroke}%
\pgfsetdash{}{0pt}%
\pgfpathmoveto{\pgfqpoint{3.727863in}{1.097701in}}%
\pgfpathlineto{\pgfqpoint{3.797307in}{1.097701in}}%
\pgfusepath{stroke}%
\end{pgfscope}%
\begin{pgfscope}%
\pgftext[x=3.908418in,y=1.049090in,left,base]{\rmfamily\fontsize{10.000000}{12.000000}\selectfont \(\displaystyle \textnormal{tol}=10^{{-}2}\)}%
\end{pgfscope}%
\begin{pgfscope}%
\pgfsetbuttcap%
\pgfsetroundjoin%
\pgfsetlinewidth{1.003750pt}%
\definecolor{currentstroke}{rgb}{0.000000,0.000000,0.000000}%
\pgfsetstrokecolor{currentstroke}%
\pgfsetdash{{1.000000pt}{1.650000pt}}{0.000000pt}%
\pgfpathmoveto{\pgfqpoint{3.727863in}{0.898257in}}%
\pgfpathlineto{\pgfqpoint{3.797307in}{0.898257in}}%
\pgfusepath{stroke}%
\end{pgfscope}%
\begin{pgfscope}%
\pgftext[x=3.908418in,y=0.849646in,left,base]{\rmfamily\fontsize{10.000000}{12.000000}\selectfont \textnormal{Reference}}%
\end{pgfscope}%
\end{pgfpicture}%
\makeatother%
\endgroup%

    \caption[LCS curves found by means of the Bogacki-Shampine 3(2) integration
    scheme]{
        LCS curves found by means of the Bogacki-Shampine 3(2) integration
        scheme. The reference LCS, as shown in figure
        \ref{fig:referencelcs}, is dashed on the top layer. Note that
        the LCS for the lowest tolerance level considered, that is,
        $\textnormal{tol}=10^{-1}$,
        is not included. This is because the corresponding $\mathcal{U}_{0}$
        domain, shown in~\cref{fig:u0_dom_err_bs32}, and the reference
        $\mathcal{U}_{0}$, shown in~\cref{fig:u0_domain} are very
        dissimilar. The second lowest tolerance level, i.e., $\textnormal{tol}=10^{-2}$,
        is the main culprit among the remaining, as far as discrepancies are
    concerned.}
    \label{fig:lcs_rkbs32}
\end{figure}


%\clearpage
\begin{figure}[htpb]
    \centering
    %% Creator: Matplotlib, PGF backend
%%
%% To include the figure in your LaTeX document, write
%%   \input{<filename>.pgf}
%%
%% Make sure the required packages are loaded in your preamble
%%   \usepackage{pgf}
%%
%% Figures using additional raster images can only be included by \input if
%% they are in the same directory as the main LaTeX file. For loading figures
%% from other directories you can use the `import` package
%%   \usepackage{import}
%% and then include the figures with
%%   \import{<path to file>}{<filename>.pgf}
%%
%% Matplotlib used the following preamble
%%   \usepackage[utf8x]{inputenc}
%%   \usepackage[T1]{fontenc}
%%   \usepackage[]{libertine}\usepackage[libertine]{newtxmath}
%%
\begingroup%
\makeatletter%
\begin{pgfpicture}%
\pgfpathrectangle{\pgfpointorigin}{\pgfqpoint{5.050000in}{3.100000in}}%
\pgfusepath{use as bounding box, clip}%
\begin{pgfscope}%
\pgfsetbuttcap%
\pgfsetmiterjoin%
\definecolor{currentfill}{rgb}{1.000000,1.000000,1.000000}%
\pgfsetfillcolor{currentfill}%
\pgfsetlinewidth{0.000000pt}%
\definecolor{currentstroke}{rgb}{1.000000,1.000000,1.000000}%
\pgfsetstrokecolor{currentstroke}%
\pgfsetdash{}{0pt}%
\pgfpathmoveto{\pgfqpoint{0.000000in}{0.000000in}}%
\pgfpathlineto{\pgfqpoint{5.050000in}{0.000000in}}%
\pgfpathlineto{\pgfqpoint{5.050000in}{3.100000in}}%
\pgfpathlineto{\pgfqpoint{0.000000in}{3.100000in}}%
\pgfpathclose%
\pgfusepath{fill}%
\end{pgfscope}%
\begin{pgfscope}%
\pgfsetbuttcap%
\pgfsetmiterjoin%
\definecolor{currentfill}{rgb}{1.000000,1.000000,1.000000}%
\pgfsetfillcolor{currentfill}%
\pgfsetlinewidth{0.000000pt}%
\definecolor{currentstroke}{rgb}{0.000000,0.000000,0.000000}%
\pgfsetstrokecolor{currentstroke}%
\pgfsetstrokeopacity{0.000000}%
\pgfsetdash{}{0pt}%
\pgfpathmoveto{\pgfqpoint{0.448634in}{0.402556in}}%
\pgfpathlineto{\pgfqpoint{4.799294in}{0.402556in}}%
\pgfpathlineto{\pgfqpoint{4.799294in}{2.891760in}}%
\pgfpathlineto{\pgfqpoint{0.448634in}{2.891760in}}%
\pgfpathclose%
\pgfusepath{fill}%
\end{pgfscope}%
\begin{pgfscope}%
\pgfsetbuttcap%
\pgfsetroundjoin%
\definecolor{currentfill}{rgb}{0.000000,0.000000,0.000000}%
\pgfsetfillcolor{currentfill}%
\pgfsetlinewidth{0.803000pt}%
\definecolor{currentstroke}{rgb}{0.000000,0.000000,0.000000}%
\pgfsetstrokecolor{currentstroke}%
\pgfsetdash{}{0pt}%
\pgfsys@defobject{currentmarker}{\pgfqpoint{0.000000in}{-0.048611in}}{\pgfqpoint{0.000000in}{0.000000in}}{%
\pgfpathmoveto{\pgfqpoint{0.000000in}{0.000000in}}%
\pgfpathlineto{\pgfqpoint{0.000000in}{-0.048611in}}%
\pgfusepath{stroke,fill}%
}%
\begin{pgfscope}%
\pgfsys@transformshift{0.448634in}{0.402556in}%
\pgfsys@useobject{currentmarker}{}%
\end{pgfscope}%
\end{pgfscope}%
\begin{pgfscope}%
\pgftext[x=0.448634in,y=0.305334in,,top]{\rmfamily\fontsize{12.000000}{14.400000}\selectfont \(\displaystyle 0.00\)}%
\end{pgfscope}%
\begin{pgfscope}%
\pgfsetbuttcap%
\pgfsetroundjoin%
\definecolor{currentfill}{rgb}{0.000000,0.000000,0.000000}%
\pgfsetfillcolor{currentfill}%
\pgfsetlinewidth{0.803000pt}%
\definecolor{currentstroke}{rgb}{0.000000,0.000000,0.000000}%
\pgfsetstrokecolor{currentstroke}%
\pgfsetdash{}{0pt}%
\pgfsys@defobject{currentmarker}{\pgfqpoint{0.000000in}{-0.048611in}}{\pgfqpoint{0.000000in}{0.000000in}}{%
\pgfpathmoveto{\pgfqpoint{0.000000in}{0.000000in}}%
\pgfpathlineto{\pgfqpoint{0.000000in}{-0.048611in}}%
\pgfusepath{stroke,fill}%
}%
\begin{pgfscope}%
\pgfsys@transformshift{0.992466in}{0.402556in}%
\pgfsys@useobject{currentmarker}{}%
\end{pgfscope}%
\end{pgfscope}%
\begin{pgfscope}%
\pgftext[x=0.992466in,y=0.305334in,,top]{\rmfamily\fontsize{12.000000}{14.400000}\selectfont \(\displaystyle 0.25\)}%
\end{pgfscope}%
\begin{pgfscope}%
\pgfsetbuttcap%
\pgfsetroundjoin%
\definecolor{currentfill}{rgb}{0.000000,0.000000,0.000000}%
\pgfsetfillcolor{currentfill}%
\pgfsetlinewidth{0.803000pt}%
\definecolor{currentstroke}{rgb}{0.000000,0.000000,0.000000}%
\pgfsetstrokecolor{currentstroke}%
\pgfsetdash{}{0pt}%
\pgfsys@defobject{currentmarker}{\pgfqpoint{0.000000in}{-0.048611in}}{\pgfqpoint{0.000000in}{0.000000in}}{%
\pgfpathmoveto{\pgfqpoint{0.000000in}{0.000000in}}%
\pgfpathlineto{\pgfqpoint{0.000000in}{-0.048611in}}%
\pgfusepath{stroke,fill}%
}%
\begin{pgfscope}%
\pgfsys@transformshift{1.536299in}{0.402556in}%
\pgfsys@useobject{currentmarker}{}%
\end{pgfscope}%
\end{pgfscope}%
\begin{pgfscope}%
\pgftext[x=1.536299in,y=0.305334in,,top]{\rmfamily\fontsize{12.000000}{14.400000}\selectfont \(\displaystyle 0.50\)}%
\end{pgfscope}%
\begin{pgfscope}%
\pgfsetbuttcap%
\pgfsetroundjoin%
\definecolor{currentfill}{rgb}{0.000000,0.000000,0.000000}%
\pgfsetfillcolor{currentfill}%
\pgfsetlinewidth{0.803000pt}%
\definecolor{currentstroke}{rgb}{0.000000,0.000000,0.000000}%
\pgfsetstrokecolor{currentstroke}%
\pgfsetdash{}{0pt}%
\pgfsys@defobject{currentmarker}{\pgfqpoint{0.000000in}{-0.048611in}}{\pgfqpoint{0.000000in}{0.000000in}}{%
\pgfpathmoveto{\pgfqpoint{0.000000in}{0.000000in}}%
\pgfpathlineto{\pgfqpoint{0.000000in}{-0.048611in}}%
\pgfusepath{stroke,fill}%
}%
\begin{pgfscope}%
\pgfsys@transformshift{2.080131in}{0.402556in}%
\pgfsys@useobject{currentmarker}{}%
\end{pgfscope}%
\end{pgfscope}%
\begin{pgfscope}%
\pgftext[x=2.080131in,y=0.305334in,,top]{\rmfamily\fontsize{12.000000}{14.400000}\selectfont \(\displaystyle 0.75\)}%
\end{pgfscope}%
\begin{pgfscope}%
\pgfsetbuttcap%
\pgfsetroundjoin%
\definecolor{currentfill}{rgb}{0.000000,0.000000,0.000000}%
\pgfsetfillcolor{currentfill}%
\pgfsetlinewidth{0.803000pt}%
\definecolor{currentstroke}{rgb}{0.000000,0.000000,0.000000}%
\pgfsetstrokecolor{currentstroke}%
\pgfsetdash{}{0pt}%
\pgfsys@defobject{currentmarker}{\pgfqpoint{0.000000in}{-0.048611in}}{\pgfqpoint{0.000000in}{0.000000in}}{%
\pgfpathmoveto{\pgfqpoint{0.000000in}{0.000000in}}%
\pgfpathlineto{\pgfqpoint{0.000000in}{-0.048611in}}%
\pgfusepath{stroke,fill}%
}%
\begin{pgfscope}%
\pgfsys@transformshift{2.623964in}{0.402556in}%
\pgfsys@useobject{currentmarker}{}%
\end{pgfscope}%
\end{pgfscope}%
\begin{pgfscope}%
\pgftext[x=2.623964in,y=0.305334in,,top]{\rmfamily\fontsize{12.000000}{14.400000}\selectfont \(\displaystyle 1.00\)}%
\end{pgfscope}%
\begin{pgfscope}%
\pgfsetbuttcap%
\pgfsetroundjoin%
\definecolor{currentfill}{rgb}{0.000000,0.000000,0.000000}%
\pgfsetfillcolor{currentfill}%
\pgfsetlinewidth{0.803000pt}%
\definecolor{currentstroke}{rgb}{0.000000,0.000000,0.000000}%
\pgfsetstrokecolor{currentstroke}%
\pgfsetdash{}{0pt}%
\pgfsys@defobject{currentmarker}{\pgfqpoint{0.000000in}{-0.048611in}}{\pgfqpoint{0.000000in}{0.000000in}}{%
\pgfpathmoveto{\pgfqpoint{0.000000in}{0.000000in}}%
\pgfpathlineto{\pgfqpoint{0.000000in}{-0.048611in}}%
\pgfusepath{stroke,fill}%
}%
\begin{pgfscope}%
\pgfsys@transformshift{3.167797in}{0.402556in}%
\pgfsys@useobject{currentmarker}{}%
\end{pgfscope}%
\end{pgfscope}%
\begin{pgfscope}%
\pgftext[x=3.167797in,y=0.305334in,,top]{\rmfamily\fontsize{12.000000}{14.400000}\selectfont \(\displaystyle 1.25\)}%
\end{pgfscope}%
\begin{pgfscope}%
\pgfsetbuttcap%
\pgfsetroundjoin%
\definecolor{currentfill}{rgb}{0.000000,0.000000,0.000000}%
\pgfsetfillcolor{currentfill}%
\pgfsetlinewidth{0.803000pt}%
\definecolor{currentstroke}{rgb}{0.000000,0.000000,0.000000}%
\pgfsetstrokecolor{currentstroke}%
\pgfsetdash{}{0pt}%
\pgfsys@defobject{currentmarker}{\pgfqpoint{0.000000in}{-0.048611in}}{\pgfqpoint{0.000000in}{0.000000in}}{%
\pgfpathmoveto{\pgfqpoint{0.000000in}{0.000000in}}%
\pgfpathlineto{\pgfqpoint{0.000000in}{-0.048611in}}%
\pgfusepath{stroke,fill}%
}%
\begin{pgfscope}%
\pgfsys@transformshift{3.711629in}{0.402556in}%
\pgfsys@useobject{currentmarker}{}%
\end{pgfscope}%
\end{pgfscope}%
\begin{pgfscope}%
\pgftext[x=3.711629in,y=0.305334in,,top]{\rmfamily\fontsize{12.000000}{14.400000}\selectfont \(\displaystyle 1.50\)}%
\end{pgfscope}%
\begin{pgfscope}%
\pgfsetbuttcap%
\pgfsetroundjoin%
\definecolor{currentfill}{rgb}{0.000000,0.000000,0.000000}%
\pgfsetfillcolor{currentfill}%
\pgfsetlinewidth{0.803000pt}%
\definecolor{currentstroke}{rgb}{0.000000,0.000000,0.000000}%
\pgfsetstrokecolor{currentstroke}%
\pgfsetdash{}{0pt}%
\pgfsys@defobject{currentmarker}{\pgfqpoint{0.000000in}{-0.048611in}}{\pgfqpoint{0.000000in}{0.000000in}}{%
\pgfpathmoveto{\pgfqpoint{0.000000in}{0.000000in}}%
\pgfpathlineto{\pgfqpoint{0.000000in}{-0.048611in}}%
\pgfusepath{stroke,fill}%
}%
\begin{pgfscope}%
\pgfsys@transformshift{4.255462in}{0.402556in}%
\pgfsys@useobject{currentmarker}{}%
\end{pgfscope}%
\end{pgfscope}%
\begin{pgfscope}%
\pgftext[x=4.255462in,y=0.305334in,,top]{\rmfamily\fontsize{12.000000}{14.400000}\selectfont \(\displaystyle 1.75\)}%
\end{pgfscope}%
\begin{pgfscope}%
\pgfsetbuttcap%
\pgfsetroundjoin%
\definecolor{currentfill}{rgb}{0.000000,0.000000,0.000000}%
\pgfsetfillcolor{currentfill}%
\pgfsetlinewidth{0.803000pt}%
\definecolor{currentstroke}{rgb}{0.000000,0.000000,0.000000}%
\pgfsetstrokecolor{currentstroke}%
\pgfsetdash{}{0pt}%
\pgfsys@defobject{currentmarker}{\pgfqpoint{0.000000in}{-0.048611in}}{\pgfqpoint{0.000000in}{0.000000in}}{%
\pgfpathmoveto{\pgfqpoint{0.000000in}{0.000000in}}%
\pgfpathlineto{\pgfqpoint{0.000000in}{-0.048611in}}%
\pgfusepath{stroke,fill}%
}%
\begin{pgfscope}%
\pgfsys@transformshift{4.799294in}{0.402556in}%
\pgfsys@useobject{currentmarker}{}%
\end{pgfscope}%
\end{pgfscope}%
\begin{pgfscope}%
\pgftext[x=4.799294in,y=0.305334in,,top]{\rmfamily\fontsize{12.000000}{14.400000}\selectfont \(\displaystyle 2.00\)}%
\end{pgfscope}%
\begin{pgfscope}%
\pgfsetbuttcap%
\pgfsetroundjoin%
\definecolor{currentfill}{rgb}{0.000000,0.000000,0.000000}%
\pgfsetfillcolor{currentfill}%
\pgfsetlinewidth{0.803000pt}%
\definecolor{currentstroke}{rgb}{0.000000,0.000000,0.000000}%
\pgfsetstrokecolor{currentstroke}%
\pgfsetdash{}{0pt}%
\pgfsys@defobject{currentmarker}{\pgfqpoint{-0.048611in}{0.000000in}}{\pgfqpoint{0.000000in}{0.000000in}}{%
\pgfpathmoveto{\pgfqpoint{0.000000in}{0.000000in}}%
\pgfpathlineto{\pgfqpoint{-0.048611in}{0.000000in}}%
\pgfusepath{stroke,fill}%
}%
\begin{pgfscope}%
\pgfsys@transformshift{0.448634in}{0.402556in}%
\pgfsys@useobject{currentmarker}{}%
\end{pgfscope}%
\end{pgfscope}%
\begin{pgfscope}%
\pgftext[x=0.149245in,y=0.345015in,left,base]{\rmfamily\fontsize{12.000000}{14.400000}\selectfont \(\displaystyle 0.0\)}%
\end{pgfscope}%
\begin{pgfscope}%
\pgfsetbuttcap%
\pgfsetroundjoin%
\definecolor{currentfill}{rgb}{0.000000,0.000000,0.000000}%
\pgfsetfillcolor{currentfill}%
\pgfsetlinewidth{0.803000pt}%
\definecolor{currentstroke}{rgb}{0.000000,0.000000,0.000000}%
\pgfsetstrokecolor{currentstroke}%
\pgfsetdash{}{0pt}%
\pgfsys@defobject{currentmarker}{\pgfqpoint{-0.048611in}{0.000000in}}{\pgfqpoint{0.000000in}{0.000000in}}{%
\pgfpathmoveto{\pgfqpoint{0.000000in}{0.000000in}}%
\pgfpathlineto{\pgfqpoint{-0.048611in}{0.000000in}}%
\pgfusepath{stroke,fill}%
}%
\begin{pgfscope}%
\pgfsys@transformshift{0.448634in}{0.900397in}%
\pgfsys@useobject{currentmarker}{}%
\end{pgfscope}%
\end{pgfscope}%
\begin{pgfscope}%
\pgftext[x=0.149245in,y=0.842855in,left,base]{\rmfamily\fontsize{12.000000}{14.400000}\selectfont \(\displaystyle 0.2\)}%
\end{pgfscope}%
\begin{pgfscope}%
\pgfsetbuttcap%
\pgfsetroundjoin%
\definecolor{currentfill}{rgb}{0.000000,0.000000,0.000000}%
\pgfsetfillcolor{currentfill}%
\pgfsetlinewidth{0.803000pt}%
\definecolor{currentstroke}{rgb}{0.000000,0.000000,0.000000}%
\pgfsetstrokecolor{currentstroke}%
\pgfsetdash{}{0pt}%
\pgfsys@defobject{currentmarker}{\pgfqpoint{-0.048611in}{0.000000in}}{\pgfqpoint{0.000000in}{0.000000in}}{%
\pgfpathmoveto{\pgfqpoint{0.000000in}{0.000000in}}%
\pgfpathlineto{\pgfqpoint{-0.048611in}{0.000000in}}%
\pgfusepath{stroke,fill}%
}%
\begin{pgfscope}%
\pgfsys@transformshift{0.448634in}{1.398238in}%
\pgfsys@useobject{currentmarker}{}%
\end{pgfscope}%
\end{pgfscope}%
\begin{pgfscope}%
\pgftext[x=0.149245in,y=1.340696in,left,base]{\rmfamily\fontsize{12.000000}{14.400000}\selectfont \(\displaystyle 0.4\)}%
\end{pgfscope}%
\begin{pgfscope}%
\pgfsetbuttcap%
\pgfsetroundjoin%
\definecolor{currentfill}{rgb}{0.000000,0.000000,0.000000}%
\pgfsetfillcolor{currentfill}%
\pgfsetlinewidth{0.803000pt}%
\definecolor{currentstroke}{rgb}{0.000000,0.000000,0.000000}%
\pgfsetstrokecolor{currentstroke}%
\pgfsetdash{}{0pt}%
\pgfsys@defobject{currentmarker}{\pgfqpoint{-0.048611in}{0.000000in}}{\pgfqpoint{0.000000in}{0.000000in}}{%
\pgfpathmoveto{\pgfqpoint{0.000000in}{0.000000in}}%
\pgfpathlineto{\pgfqpoint{-0.048611in}{0.000000in}}%
\pgfusepath{stroke,fill}%
}%
\begin{pgfscope}%
\pgfsys@transformshift{0.448634in}{1.896079in}%
\pgfsys@useobject{currentmarker}{}%
\end{pgfscope}%
\end{pgfscope}%
\begin{pgfscope}%
\pgftext[x=0.149245in,y=1.838537in,left,base]{\rmfamily\fontsize{12.000000}{14.400000}\selectfont \(\displaystyle 0.6\)}%
\end{pgfscope}%
\begin{pgfscope}%
\pgfsetbuttcap%
\pgfsetroundjoin%
\definecolor{currentfill}{rgb}{0.000000,0.000000,0.000000}%
\pgfsetfillcolor{currentfill}%
\pgfsetlinewidth{0.803000pt}%
\definecolor{currentstroke}{rgb}{0.000000,0.000000,0.000000}%
\pgfsetstrokecolor{currentstroke}%
\pgfsetdash{}{0pt}%
\pgfsys@defobject{currentmarker}{\pgfqpoint{-0.048611in}{0.000000in}}{\pgfqpoint{0.000000in}{0.000000in}}{%
\pgfpathmoveto{\pgfqpoint{0.000000in}{0.000000in}}%
\pgfpathlineto{\pgfqpoint{-0.048611in}{0.000000in}}%
\pgfusepath{stroke,fill}%
}%
\begin{pgfscope}%
\pgfsys@transformshift{0.448634in}{2.393919in}%
\pgfsys@useobject{currentmarker}{}%
\end{pgfscope}%
\end{pgfscope}%
\begin{pgfscope}%
\pgftext[x=0.149245in,y=2.336378in,left,base]{\rmfamily\fontsize{12.000000}{14.400000}\selectfont \(\displaystyle 0.8\)}%
\end{pgfscope}%
\begin{pgfscope}%
\pgfsetbuttcap%
\pgfsetroundjoin%
\definecolor{currentfill}{rgb}{0.000000,0.000000,0.000000}%
\pgfsetfillcolor{currentfill}%
\pgfsetlinewidth{0.803000pt}%
\definecolor{currentstroke}{rgb}{0.000000,0.000000,0.000000}%
\pgfsetstrokecolor{currentstroke}%
\pgfsetdash{}{0pt}%
\pgfsys@defobject{currentmarker}{\pgfqpoint{-0.048611in}{0.000000in}}{\pgfqpoint{0.000000in}{0.000000in}}{%
\pgfpathmoveto{\pgfqpoint{0.000000in}{0.000000in}}%
\pgfpathlineto{\pgfqpoint{-0.048611in}{0.000000in}}%
\pgfusepath{stroke,fill}%
}%
\begin{pgfscope}%
\pgfsys@transformshift{0.448634in}{2.891760in}%
\pgfsys@useobject{currentmarker}{}%
\end{pgfscope}%
\end{pgfscope}%
\begin{pgfscope}%
\pgftext[x=0.149245in,y=2.834219in,left,base]{\rmfamily\fontsize{12.000000}{14.400000}\selectfont \(\displaystyle 1.0\)}%
\end{pgfscope}%
\begin{pgfscope}%
\pgfpathrectangle{\pgfqpoint{0.448634in}{0.402556in}}{\pgfqpoint{4.350661in}{2.489204in}} %
\pgfusepath{clip}%
\pgfsetrectcap%
\pgfsetroundjoin%
\pgfsetlinewidth{1.003750pt}%
\definecolor{currentstroke}{rgb}{0.121569,0.466667,0.705882}%
\pgfsetstrokecolor{currentstroke}%
\pgfsetdash{}{0pt}%
\pgfpathmoveto{\pgfqpoint{0.448634in}{2.896245in}}%
\pgfpathlineto{\pgfqpoint{0.448593in}{0.407043in}}%
\pgfpathlineto{\pgfqpoint{0.448593in}{0.407043in}}%
\pgfusepath{stroke}%
\end{pgfscope}%
\begin{pgfscope}%
\pgfpathrectangle{\pgfqpoint{0.448634in}{0.402556in}}{\pgfqpoint{4.350661in}{2.489204in}} %
\pgfusepath{clip}%
\pgfsetrectcap%
\pgfsetroundjoin%
\pgfsetlinewidth{1.003750pt}%
\definecolor{currentstroke}{rgb}{0.121569,0.466667,0.705882}%
\pgfsetstrokecolor{currentstroke}%
\pgfsetdash{}{0pt}%
\pgfpathmoveto{\pgfqpoint{0.576853in}{1.760817in}}%
\pgfpathlineto{\pgfqpoint{0.569394in}{1.840010in}}%
\pgfpathlineto{\pgfqpoint{0.563209in}{1.929338in}}%
\pgfpathlineto{\pgfqpoint{0.558592in}{2.028764in}}%
\pgfpathlineto{\pgfqpoint{0.555985in}{2.133265in}}%
\pgfpathlineto{\pgfqpoint{0.555566in}{2.237808in}}%
\pgfpathlineto{\pgfqpoint{0.557371in}{2.337352in}}%
\pgfpathlineto{\pgfqpoint{0.561096in}{2.424366in}}%
\pgfpathlineto{\pgfqpoint{0.566403in}{2.498791in}}%
\pgfpathlineto{\pgfqpoint{0.572909in}{2.560570in}}%
\pgfpathlineto{\pgfqpoint{0.580458in}{2.612119in}}%
\pgfpathlineto{\pgfqpoint{0.589086in}{2.655816in}}%
\pgfpathlineto{\pgfqpoint{0.598406in}{2.691589in}}%
\pgfpathlineto{\pgfqpoint{0.608613in}{2.721757in}}%
\pgfpathlineto{\pgfqpoint{0.619241in}{2.746278in}}%
\pgfpathlineto{\pgfqpoint{0.630817in}{2.767339in}}%
\pgfpathlineto{\pgfqpoint{0.642975in}{2.784884in}}%
\pgfpathlineto{\pgfqpoint{0.656813in}{2.800712in}}%
\pgfpathlineto{\pgfqpoint{0.672197in}{2.814549in}}%
\pgfpathlineto{\pgfqpoint{0.688853in}{2.826301in}}%
\pgfpathlineto{\pgfqpoint{0.706461in}{2.836076in}}%
\pgfpathlineto{\pgfqpoint{0.726804in}{2.844875in}}%
\pgfpathlineto{\pgfqpoint{0.751866in}{2.853203in}}%
\pgfpathlineto{\pgfqpoint{0.781631in}{2.860547in}}%
\pgfpathlineto{\pgfqpoint{0.818168in}{2.867054in}}%
\pgfpathlineto{\pgfqpoint{0.863581in}{2.872685in}}%
\pgfpathlineto{\pgfqpoint{0.922161in}{2.877518in}}%
\pgfpathlineto{\pgfqpoint{1.000391in}{2.881567in}}%
\pgfpathlineto{\pgfqpoint{1.111294in}{2.884881in}}%
\pgfpathlineto{\pgfqpoint{1.274428in}{2.887367in}}%
\pgfpathlineto{\pgfqpoint{1.552865in}{2.889263in}}%
\pgfpathlineto{\pgfqpoint{2.107573in}{2.890457in}}%
\pgfpathlineto{\pgfqpoint{3.343161in}{2.890573in}}%
\pgfpathlineto{\pgfqpoint{4.043615in}{2.888941in}}%
\pgfpathlineto{\pgfqpoint{4.289417in}{2.886404in}}%
\pgfpathlineto{\pgfqpoint{4.413375in}{2.883093in}}%
\pgfpathlineto{\pgfqpoint{4.489424in}{2.878997in}}%
\pgfpathlineto{\pgfqpoint{4.541451in}{2.874081in}}%
\pgfpathlineto{\pgfqpoint{4.578100in}{2.868470in}}%
\pgfpathlineto{\pgfqpoint{4.605818in}{2.862092in}}%
\pgfpathlineto{\pgfqpoint{4.626725in}{2.855245in}}%
\pgfpathlineto{\pgfqpoint{4.644925in}{2.847018in}}%
\pgfpathlineto{\pgfqpoint{4.660241in}{2.837590in}}%
\pgfpathlineto{\pgfqpoint{4.672623in}{2.827468in}}%
\pgfpathlineto{\pgfqpoint{4.683751in}{2.815592in}}%
\pgfpathlineto{\pgfqpoint{4.693406in}{2.802135in}}%
\pgfpathlineto{\pgfqpoint{4.702740in}{2.785343in}}%
\pgfpathlineto{\pgfqpoint{4.711277in}{2.765194in}}%
\pgfpathlineto{\pgfqpoint{4.719482in}{2.739484in}}%
\pgfpathlineto{\pgfqpoint{4.726293in}{2.710657in}}%
\pgfpathlineto{\pgfqpoint{4.733259in}{2.671643in}}%
\pgfpathlineto{\pgfqpoint{4.739604in}{2.622396in}}%
\pgfpathlineto{\pgfqpoint{4.745236in}{2.560504in}}%
\pgfpathlineto{\pgfqpoint{4.750164in}{2.481052in}}%
\pgfpathlineto{\pgfqpoint{4.754367in}{2.376618in}}%
\pgfpathlineto{\pgfqpoint{4.757443in}{2.242249in}}%
\pgfpathlineto{\pgfqpoint{4.758977in}{2.075483in}}%
\pgfpathlineto{\pgfqpoint{4.758447in}{1.888795in}}%
\pgfpathlineto{\pgfqpoint{4.755756in}{1.707111in}}%
\pgfpathlineto{\pgfqpoint{4.750925in}{1.532957in}}%
\pgfpathlineto{\pgfqpoint{4.744785in}{1.398726in}}%
\pgfpathlineto{\pgfqpoint{4.737575in}{1.289516in}}%
\pgfpathlineto{\pgfqpoint{4.728714in}{1.190470in}}%
\pgfpathlineto{\pgfqpoint{4.719652in}{1.116521in}}%
\pgfpathlineto{\pgfqpoint{4.710036in}{1.055276in}}%
\pgfpathlineto{\pgfqpoint{4.699503in}{1.001861in}}%
\pgfpathlineto{\pgfqpoint{4.689040in}{0.958690in}}%
\pgfpathlineto{\pgfqpoint{4.677219in}{0.918600in}}%
\pgfpathlineto{\pgfqpoint{4.664034in}{0.881749in}}%
\pgfpathlineto{\pgfqpoint{4.650584in}{0.850492in}}%
\pgfpathlineto{\pgfqpoint{4.636303in}{0.822570in}}%
\pgfpathlineto{\pgfqpoint{4.620207in}{0.795974in}}%
\pgfpathlineto{\pgfqpoint{4.603640in}{0.772901in}}%
\pgfpathlineto{\pgfqpoint{4.585488in}{0.751446in}}%
\pgfpathlineto{\pgfqpoint{4.565874in}{0.731749in}}%
\pgfpathlineto{\pgfqpoint{4.544964in}{0.713879in}}%
\pgfpathlineto{\pgfqpoint{4.522958in}{0.697824in}}%
\pgfpathlineto{\pgfqpoint{4.496157in}{0.681290in}}%
\pgfpathlineto{\pgfqpoint{4.470397in}{0.667953in}}%
\pgfpathlineto{\pgfqpoint{4.439961in}{0.654509in}}%
\pgfpathlineto{\pgfqpoint{4.406841in}{0.642281in}}%
\pgfpathlineto{\pgfqpoint{4.369009in}{0.630748in}}%
\pgfpathlineto{\pgfqpoint{4.326489in}{0.620226in}}%
\pgfpathlineto{\pgfqpoint{4.279327in}{0.610949in}}%
\pgfpathlineto{\pgfqpoint{4.227576in}{0.603085in}}%
\pgfpathlineto{\pgfqpoint{4.173450in}{0.597063in}}%
\pgfpathlineto{\pgfqpoint{4.110511in}{0.592203in}}%
\pgfpathlineto{\pgfqpoint{4.047471in}{0.589537in}}%
\pgfpathlineto{\pgfqpoint{3.977867in}{0.588624in}}%
\pgfpathlineto{\pgfqpoint{3.906093in}{0.589934in}}%
\pgfpathlineto{\pgfqpoint{3.834377in}{0.593496in}}%
\pgfpathlineto{\pgfqpoint{3.767120in}{0.599067in}}%
\pgfpathlineto{\pgfqpoint{3.704364in}{0.606392in}}%
\pgfpathlineto{\pgfqpoint{3.678516in}{0.610510in}}%
\pgfpathlineto{\pgfqpoint{3.620438in}{0.620500in}}%
\pgfpathlineto{\pgfqpoint{3.586319in}{0.628207in}}%
\pgfpathlineto{\pgfqpoint{3.495240in}{0.652428in}}%
\pgfpathlineto{\pgfqpoint{3.451528in}{0.667583in}}%
\pgfpathlineto{\pgfqpoint{3.408538in}{0.685220in}}%
\pgfpathlineto{\pgfqpoint{3.374594in}{0.702001in}}%
\pgfpathlineto{\pgfqpoint{3.345407in}{0.718682in}}%
\pgfpathlineto{\pgfqpoint{3.315236in}{0.738520in}}%
\pgfpathlineto{\pgfqpoint{3.288127in}{0.759290in}}%
\pgfpathlineto{\pgfqpoint{3.264004in}{0.780551in}}%
\pgfpathlineto{\pgfqpoint{3.241208in}{0.803648in}}%
\pgfpathlineto{\pgfqpoint{3.219894in}{0.828530in}}%
\pgfpathlineto{\pgfqpoint{3.200189in}{0.855091in}}%
\pgfpathlineto{\pgfqpoint{3.182177in}{0.883182in}}%
\pgfpathlineto{\pgfqpoint{3.165906in}{0.912633in}}%
\pgfpathlineto{\pgfqpoint{3.150351in}{0.945448in}}%
\pgfpathlineto{\pgfqpoint{3.136682in}{0.979345in}}%
\pgfpathlineto{\pgfqpoint{3.124073in}{1.016460in}}%
\pgfpathlineto{\pgfqpoint{3.112834in}{1.056769in}}%
\pgfpathlineto{\pgfqpoint{3.103046in}{1.100146in}}%
\pgfpathlineto{\pgfqpoint{3.095343in}{1.144071in}}%
\pgfpathlineto{\pgfqpoint{3.089208in}{1.190837in}}%
\pgfpathlineto{\pgfqpoint{3.084595in}{1.242838in}}%
\pgfpathlineto{\pgfqpoint{3.082137in}{1.295031in}}%
\pgfpathlineto{\pgfqpoint{3.081687in}{1.349787in}}%
\pgfpathlineto{\pgfqpoint{3.083451in}{1.406998in}}%
\pgfpathlineto{\pgfqpoint{3.087181in}{1.461589in}}%
\pgfpathlineto{\pgfqpoint{3.093485in}{1.520888in}}%
\pgfpathlineto{\pgfqpoint{3.101823in}{1.577334in}}%
\pgfpathlineto{\pgfqpoint{3.111930in}{1.630856in}}%
\pgfpathlineto{\pgfqpoint{3.124690in}{1.686208in}}%
\pgfpathlineto{\pgfqpoint{3.139178in}{1.738395in}}%
\pgfpathlineto{\pgfqpoint{3.155145in}{1.787366in}}%
\pgfpathlineto{\pgfqpoint{3.172353in}{1.833085in}}%
\pgfpathlineto{\pgfqpoint{3.191618in}{1.877716in}}%
\pgfpathlineto{\pgfqpoint{3.214026in}{1.923261in}}%
\pgfpathlineto{\pgfqpoint{3.236214in}{1.963157in}}%
\pgfpathlineto{\pgfqpoint{3.260178in}{2.001684in}}%
\pgfpathlineto{\pgfqpoint{3.285814in}{2.038776in}}%
\pgfpathlineto{\pgfqpoint{3.314415in}{2.076285in}}%
\pgfpathlineto{\pgfqpoint{3.348944in}{2.117711in}}%
\pgfpathlineto{\pgfqpoint{3.417133in}{2.198022in}}%
\pgfpathlineto{\pgfqpoint{3.426053in}{2.212128in}}%
\pgfpathlineto{\pgfqpoint{3.430798in}{2.223297in}}%
\pgfpathlineto{\pgfqpoint{3.432034in}{2.230603in}}%
\pgfpathlineto{\pgfqpoint{3.430773in}{2.237856in}}%
\pgfpathlineto{\pgfqpoint{3.426621in}{2.243526in}}%
\pgfpathlineto{\pgfqpoint{3.420908in}{2.247084in}}%
\pgfpathlineto{\pgfqpoint{3.412501in}{2.249583in}}%
\pgfpathlineto{\pgfqpoint{3.399499in}{2.250689in}}%
\pgfpathlineto{\pgfqpoint{3.384305in}{2.249671in}}%
\pgfpathlineto{\pgfqpoint{3.364985in}{2.246098in}}%
\pgfpathlineto{\pgfqpoint{3.341804in}{2.239342in}}%
\pgfpathlineto{\pgfqpoint{3.317109in}{2.229682in}}%
\pgfpathlineto{\pgfqpoint{3.291104in}{2.216986in}}%
\pgfpathlineto{\pgfqpoint{3.265928in}{2.202261in}}%
\pgfpathlineto{\pgfqpoint{3.239805in}{2.184361in}}%
\pgfpathlineto{\pgfqpoint{3.214775in}{2.164519in}}%
\pgfpathlineto{\pgfqpoint{3.190900in}{2.142893in}}%
\pgfpathlineto{\pgfqpoint{3.166657in}{2.117912in}}%
\pgfpathlineto{\pgfqpoint{3.143835in}{2.091233in}}%
\pgfpathlineto{\pgfqpoint{3.121079in}{2.061107in}}%
\pgfpathlineto{\pgfqpoint{3.099952in}{2.029463in}}%
\pgfpathlineto{\pgfqpoint{3.079251in}{1.994406in}}%
\pgfpathlineto{\pgfqpoint{3.059218in}{1.955915in}}%
\pgfpathlineto{\pgfqpoint{3.040058in}{1.914015in}}%
\pgfpathlineto{\pgfqpoint{3.022809in}{1.871041in}}%
\pgfpathlineto{\pgfqpoint{3.005790in}{1.822536in}}%
\pgfpathlineto{\pgfqpoint{2.990067in}{1.770819in}}%
\pgfpathlineto{\pgfqpoint{2.975708in}{1.715979in}}%
\pgfpathlineto{\pgfqpoint{2.962284in}{1.655680in}}%
\pgfpathlineto{\pgfqpoint{2.950496in}{1.592386in}}%
\pgfpathlineto{\pgfqpoint{2.940383in}{1.526185in}}%
\pgfpathlineto{\pgfqpoint{2.931745in}{1.454681in}}%
\pgfpathlineto{\pgfqpoint{2.925082in}{1.380399in}}%
\pgfpathlineto{\pgfqpoint{2.920647in}{1.305899in}}%
\pgfpathlineto{\pgfqpoint{2.918444in}{1.231270in}}%
\pgfpathlineto{\pgfqpoint{2.918545in}{1.159087in}}%
\pgfpathlineto{\pgfqpoint{2.920787in}{1.091931in}}%
\pgfpathlineto{\pgfqpoint{2.925177in}{1.027412in}}%
\pgfpathlineto{\pgfqpoint{2.931192in}{0.970580in}}%
\pgfpathlineto{\pgfqpoint{2.938760in}{0.919034in}}%
\pgfpathlineto{\pgfqpoint{2.947651in}{0.872852in}}%
\pgfpathlineto{\pgfqpoint{2.958213in}{0.829714in}}%
\pgfpathlineto{\pgfqpoint{2.969670in}{0.792114in}}%
\pgfpathlineto{\pgfqpoint{2.982463in}{0.757773in}}%
\pgfpathlineto{\pgfqpoint{2.996425in}{0.726812in}}%
\pgfpathlineto{\pgfqpoint{3.011299in}{0.699300in}}%
\pgfpathlineto{\pgfqpoint{3.026739in}{0.675225in}}%
\pgfpathlineto{\pgfqpoint{3.043828in}{0.652656in}}%
\pgfpathlineto{\pgfqpoint{3.062495in}{0.631788in}}%
\pgfpathlineto{\pgfqpoint{3.082602in}{0.612753in}}%
\pgfpathlineto{\pgfqpoint{3.103961in}{0.595592in}}%
\pgfpathlineto{\pgfqpoint{3.128268in}{0.579069in}}%
\pgfpathlineto{\pgfqpoint{3.153537in}{0.564554in}}%
\pgfpathlineto{\pgfqpoint{3.181571in}{0.550952in}}%
\pgfpathlineto{\pgfqpoint{3.214371in}{0.537647in}}%
\pgfpathlineto{\pgfqpoint{3.249846in}{0.525712in}}%
\pgfpathlineto{\pgfqpoint{3.290011in}{0.514571in}}%
\pgfpathlineto{\pgfqpoint{3.334820in}{0.504423in}}%
\pgfpathlineto{\pgfqpoint{3.386372in}{0.494999in}}%
\pgfpathlineto{\pgfqpoint{3.446798in}{0.486257in}}%
\pgfpathlineto{\pgfqpoint{3.518243in}{0.478282in}}%
\pgfpathlineto{\pgfqpoint{3.600685in}{0.471409in}}%
\pgfpathlineto{\pgfqpoint{3.696268in}{0.465713in}}%
\pgfpathlineto{\pgfqpoint{3.807144in}{0.461369in}}%
\pgfpathlineto{\pgfqpoint{3.933291in}{0.458719in}}%
\pgfpathlineto{\pgfqpoint{4.063808in}{0.458211in}}%
\pgfpathlineto{\pgfqpoint{4.187792in}{0.459914in}}%
\pgfpathlineto{\pgfqpoint{4.294335in}{0.463521in}}%
\pgfpathlineto{\pgfqpoint{4.381234in}{0.468574in}}%
\pgfpathlineto{\pgfqpoint{4.450636in}{0.474701in}}%
\pgfpathlineto{\pgfqpoint{4.506850in}{0.481799in}}%
\pgfpathlineto{\pgfqpoint{4.552009in}{0.489658in}}%
\pgfpathlineto{\pgfqpoint{4.588239in}{0.498115in}}%
\pgfpathlineto{\pgfqpoint{4.617656in}{0.507110in}}%
\pgfpathlineto{\pgfqpoint{4.642328in}{0.516843in}}%
\pgfpathlineto{\pgfqpoint{4.664194in}{0.527940in}}%
\pgfpathlineto{\pgfqpoint{4.681238in}{0.538945in}}%
\pgfpathlineto{\pgfqpoint{4.697164in}{0.551953in}}%
\pgfpathlineto{\pgfqpoint{4.710076in}{0.565289in}}%
\pgfpathlineto{\pgfqpoint{4.721578in}{0.580218in}}%
\pgfpathlineto{\pgfqpoint{4.731557in}{0.596521in}}%
\pgfpathlineto{\pgfqpoint{4.741000in}{0.616134in}}%
\pgfpathlineto{\pgfqpoint{4.749521in}{0.639027in}}%
\pgfpathlineto{\pgfqpoint{4.757522in}{0.667450in}}%
\pgfpathlineto{\pgfqpoint{4.764572in}{0.701345in}}%
\pgfpathlineto{\pgfqpoint{4.770840in}{0.743043in}}%
\pgfpathlineto{\pgfqpoint{4.776327in}{0.794934in}}%
\pgfpathlineto{\pgfqpoint{4.781278in}{0.864398in}}%
\pgfpathlineto{\pgfqpoint{4.785468in}{0.956371in}}%
\pgfpathlineto{\pgfqpoint{4.789000in}{1.085745in}}%
\pgfpathlineto{\pgfqpoint{4.791852in}{1.277385in}}%
\pgfpathlineto{\pgfqpoint{4.793959in}{1.581057in}}%
\pgfpathlineto{\pgfqpoint{4.794962in}{2.071429in}}%
\pgfpathlineto{\pgfqpoint{4.793967in}{2.559311in}}%
\pgfpathlineto{\pgfqpoint{4.791733in}{2.745981in}}%
\pgfpathlineto{\pgfqpoint{4.788955in}{2.818091in}}%
\pgfpathlineto{\pgfqpoint{4.785731in}{2.850227in}}%
\pgfpathlineto{\pgfqpoint{4.781879in}{2.867057in}}%
\pgfpathlineto{\pgfqpoint{4.777744in}{2.875780in}}%
\pgfpathlineto{\pgfqpoint{4.773097in}{2.880982in}}%
\pgfpathlineto{\pgfqpoint{4.767363in}{2.884504in}}%
\pgfpathlineto{\pgfqpoint{4.756853in}{2.887622in}}%
\pgfpathlineto{\pgfqpoint{4.739548in}{2.889639in}}%
\pgfpathlineto{\pgfqpoint{4.704762in}{2.890882in}}%
\pgfpathlineto{\pgfqpoint{4.602524in}{2.891538in}}%
\pgfpathlineto{\pgfqpoint{3.952100in}{2.891742in}}%
\pgfpathlineto{\pgfqpoint{0.617321in}{2.890753in}}%
\pgfpathlineto{\pgfqpoint{0.549910in}{2.888858in}}%
\pgfpathlineto{\pgfqpoint{0.521735in}{2.886179in}}%
\pgfpathlineto{\pgfqpoint{0.504666in}{2.882389in}}%
\pgfpathlineto{\pgfqpoint{0.494501in}{2.878011in}}%
\pgfpathlineto{\pgfqpoint{0.487180in}{2.872667in}}%
\pgfpathlineto{\pgfqpoint{0.481152in}{2.865519in}}%
\pgfpathlineto{\pgfqpoint{0.475664in}{2.854804in}}%
\pgfpathlineto{\pgfqpoint{0.471318in}{2.840737in}}%
\pgfpathlineto{\pgfqpoint{0.467301in}{2.818823in}}%
\pgfpathlineto{\pgfqpoint{0.463927in}{2.786700in}}%
\pgfpathlineto{\pgfqpoint{0.460918in}{2.734544in}}%
\pgfpathlineto{\pgfqpoint{0.458363in}{2.647473in}}%
\pgfpathlineto{\pgfqpoint{0.456575in}{2.523031in}}%
\pgfpathlineto{\pgfqpoint{0.456575in}{2.523031in}}%
\pgfusepath{stroke}%
\end{pgfscope}%
\begin{pgfscope}%
\pgfpathrectangle{\pgfqpoint{0.448634in}{0.402556in}}{\pgfqpoint{4.350661in}{2.489204in}} %
\pgfusepath{clip}%
\pgfsetrectcap%
\pgfsetroundjoin%
\pgfsetlinewidth{1.003750pt}%
\definecolor{currentstroke}{rgb}{0.121569,0.466667,0.705882}%
\pgfsetstrokecolor{currentstroke}%
\pgfsetdash{}{0pt}%
\pgfpathmoveto{\pgfqpoint{4.798840in}{2.852369in}}%
\pgfpathlineto{\pgfqpoint{4.797564in}{2.889610in}}%
\pgfpathlineto{\pgfqpoint{4.796215in}{2.891483in}}%
\pgfpathlineto{\pgfqpoint{4.787551in}{2.891760in}}%
\pgfpathlineto{\pgfqpoint{0.452128in}{2.891659in}}%
\pgfpathlineto{\pgfqpoint{0.450530in}{2.890082in}}%
\pgfpathlineto{\pgfqpoint{0.449454in}{2.882763in}}%
\pgfpathlineto{\pgfqpoint{0.448970in}{2.845432in}}%
\pgfpathlineto{\pgfqpoint{0.448743in}{2.494454in}}%
\pgfpathlineto{\pgfqpoint{0.449624in}{0.615107in}}%
\pgfpathlineto{\pgfqpoint{0.451433in}{0.510586in}}%
\pgfpathlineto{\pgfqpoint{0.453993in}{0.473374in}}%
\pgfpathlineto{\pgfqpoint{0.457406in}{0.453868in}}%
\pgfpathlineto{\pgfqpoint{0.461540in}{0.442384in}}%
\pgfpathlineto{\pgfqpoint{0.466739in}{0.434437in}}%
\pgfpathlineto{\pgfqpoint{0.473595in}{0.428350in}}%
\pgfpathlineto{\pgfqpoint{0.483492in}{0.423244in}}%
\pgfpathlineto{\pgfqpoint{0.491854in}{0.420501in}}%
\pgfpathlineto{\pgfqpoint{0.491854in}{0.420501in}}%
\pgfusepath{stroke}%
\end{pgfscope}%
\begin{pgfscope}%
\pgfpathrectangle{\pgfqpoint{0.448634in}{0.402556in}}{\pgfqpoint{4.350661in}{2.489204in}} %
\pgfusepath{clip}%
\pgfsetrectcap%
\pgfsetroundjoin%
\pgfsetlinewidth{1.003750pt}%
\definecolor{currentstroke}{rgb}{0.121569,0.466667,0.705882}%
\pgfsetstrokecolor{currentstroke}%
\pgfsetdash{}{0pt}%
\pgfpathmoveto{\pgfqpoint{0.456424in}{1.370137in}}%
\pgfpathlineto{\pgfqpoint{0.459610in}{1.118755in}}%
\pgfpathlineto{\pgfqpoint{0.463695in}{0.962007in}}%
\pgfpathlineto{\pgfqpoint{0.468519in}{0.857610in}}%
\pgfpathlineto{\pgfqpoint{0.474082in}{0.783210in}}%
\pgfpathlineto{\pgfqpoint{0.480226in}{0.728906in}}%
\pgfpathlineto{\pgfqpoint{0.486970in}{0.687306in}}%
\pgfpathlineto{\pgfqpoint{0.494537in}{0.653558in}}%
\pgfpathlineto{\pgfqpoint{0.503107in}{0.625355in}}%
\pgfpathlineto{\pgfqpoint{0.512193in}{0.602750in}}%
\pgfpathlineto{\pgfqpoint{0.522200in}{0.583508in}}%
\pgfpathlineto{\pgfqpoint{0.534108in}{0.565743in}}%
\pgfpathlineto{\pgfqpoint{0.546263in}{0.551507in}}%
\pgfpathlineto{\pgfqpoint{0.559728in}{0.538907in}}%
\pgfpathlineto{\pgfqpoint{0.576129in}{0.526693in}}%
\pgfpathlineto{\pgfqpoint{0.595483in}{0.515351in}}%
\pgfpathlineto{\pgfqpoint{0.617681in}{0.505147in}}%
\pgfpathlineto{\pgfqpoint{0.642568in}{0.496153in}}%
\pgfpathlineto{\pgfqpoint{0.672126in}{0.487778in}}%
\pgfpathlineto{\pgfqpoint{0.708443in}{0.479824in}}%
\pgfpathlineto{\pgfqpoint{0.753649in}{0.472325in}}%
\pgfpathlineto{\pgfqpoint{0.807717in}{0.465660in}}%
\pgfpathlineto{\pgfqpoint{0.877116in}{0.459475in}}%
\pgfpathlineto{\pgfqpoint{0.961828in}{0.454230in}}%
\pgfpathlineto{\pgfqpoint{1.068351in}{0.449916in}}%
\pgfpathlineto{\pgfqpoint{1.201018in}{0.446839in}}%
\pgfpathlineto{\pgfqpoint{1.357637in}{0.445481in}}%
\pgfpathlineto{\pgfqpoint{1.525135in}{0.446232in}}%
\pgfpathlineto{\pgfqpoint{1.686088in}{0.449142in}}%
\pgfpathlineto{\pgfqpoint{1.823074in}{0.453747in}}%
\pgfpathlineto{\pgfqpoint{1.938245in}{0.459764in}}%
\pgfpathlineto{\pgfqpoint{2.031582in}{0.466759in}}%
\pgfpathlineto{\pgfqpoint{2.109580in}{0.474745in}}%
\pgfpathlineto{\pgfqpoint{2.174384in}{0.483535in}}%
\pgfpathlineto{\pgfqpoint{2.228139in}{0.492940in}}%
\pgfpathlineto{\pgfqpoint{2.275119in}{0.503356in}}%
\pgfpathlineto{\pgfqpoint{2.315282in}{0.514501in}}%
\pgfpathlineto{\pgfqpoint{2.350698in}{0.526659in}}%
\pgfpathlineto{\pgfqpoint{2.381320in}{0.539536in}}%
\pgfpathlineto{\pgfqpoint{2.407164in}{0.552659in}}%
\pgfpathlineto{\pgfqpoint{2.430226in}{0.566639in}}%
\pgfpathlineto{\pgfqpoint{2.452282in}{0.582602in}}%
\pgfpathlineto{\pgfqpoint{2.471391in}{0.599069in}}%
\pgfpathlineto{\pgfqpoint{2.489240in}{0.617293in}}%
\pgfpathlineto{\pgfqpoint{2.505678in}{0.637180in}}%
\pgfpathlineto{\pgfqpoint{2.520620in}{0.658557in}}%
\pgfpathlineto{\pgfqpoint{2.535213in}{0.683314in}}%
\pgfpathlineto{\pgfqpoint{2.549115in}{0.711484in}}%
\pgfpathlineto{\pgfqpoint{2.562091in}{0.743004in}}%
\pgfpathlineto{\pgfqpoint{2.574020in}{0.777751in}}%
\pgfpathlineto{\pgfqpoint{2.585502in}{0.817970in}}%
\pgfpathlineto{\pgfqpoint{2.596809in}{0.866038in}}%
\pgfpathlineto{\pgfqpoint{2.607562in}{0.921948in}}%
\pgfpathlineto{\pgfqpoint{2.617925in}{0.988098in}}%
\pgfpathlineto{\pgfqpoint{2.627958in}{1.066918in}}%
\pgfpathlineto{\pgfqpoint{2.637941in}{1.163320in}}%
\pgfpathlineto{\pgfqpoint{2.648424in}{1.287199in}}%
\pgfpathlineto{\pgfqpoint{2.660103in}{1.453438in}}%
\pgfpathlineto{\pgfqpoint{2.674773in}{1.696801in}}%
\pgfpathlineto{\pgfqpoint{2.687716in}{1.945279in}}%
\pgfpathlineto{\pgfqpoint{2.692670in}{2.079573in}}%
\pgfpathlineto{\pgfqpoint{2.693829in}{2.166682in}}%
\pgfpathlineto{\pgfqpoint{2.692565in}{2.233870in}}%
\pgfpathlineto{\pgfqpoint{2.689436in}{2.286015in}}%
\pgfpathlineto{\pgfqpoint{2.684859in}{2.327999in}}%
\pgfpathlineto{\pgfqpoint{2.678725in}{2.364664in}}%
\pgfpathlineto{\pgfqpoint{2.671356in}{2.395897in}}%
\pgfpathlineto{\pgfqpoint{2.662489in}{2.423981in}}%
\pgfpathlineto{\pgfqpoint{2.652361in}{2.448778in}}%
\pgfpathlineto{\pgfqpoint{2.641365in}{2.470245in}}%
\pgfpathlineto{\pgfqpoint{2.628643in}{2.490425in}}%
\pgfpathlineto{\pgfqpoint{2.614279in}{2.509106in}}%
\pgfpathlineto{\pgfqpoint{2.598443in}{2.526159in}}%
\pgfpathlineto{\pgfqpoint{2.579590in}{2.543005in}}%
\pgfpathlineto{\pgfqpoint{2.559532in}{2.557923in}}%
\pgfpathlineto{\pgfqpoint{2.536602in}{2.572183in}}%
\pgfpathlineto{\pgfqpoint{2.510850in}{2.585538in}}%
\pgfpathlineto{\pgfqpoint{2.482360in}{2.597837in}}%
\pgfpathlineto{\pgfqpoint{2.449134in}{2.609683in}}%
\pgfpathlineto{\pgfqpoint{2.411184in}{2.620696in}}%
\pgfpathlineto{\pgfqpoint{2.368552in}{2.630606in}}%
\pgfpathlineto{\pgfqpoint{2.321294in}{2.639221in}}%
\pgfpathlineto{\pgfqpoint{2.269467in}{2.646399in}}%
\pgfpathlineto{\pgfqpoint{2.210954in}{2.652193in}}%
\pgfpathlineto{\pgfqpoint{2.147967in}{2.656153in}}%
\pgfpathlineto{\pgfqpoint{2.080556in}{2.658135in}}%
\pgfpathlineto{\pgfqpoint{2.010948in}{2.657971in}}%
\pgfpathlineto{\pgfqpoint{1.939195in}{2.655572in}}%
\pgfpathlineto{\pgfqpoint{1.867527in}{2.650913in}}%
\pgfpathlineto{\pgfqpoint{1.798171in}{2.644140in}}%
\pgfpathlineto{\pgfqpoint{1.733341in}{2.635606in}}%
\pgfpathlineto{\pgfqpoint{1.673075in}{2.625521in}}%
\pgfpathlineto{\pgfqpoint{1.615274in}{2.613610in}}%
\pgfpathlineto{\pgfqpoint{1.562133in}{2.600402in}}%
\pgfpathlineto{\pgfqpoint{1.513681in}{2.586139in}}%
\pgfpathlineto{\pgfqpoint{1.467862in}{2.570344in}}%
\pgfpathlineto{\pgfqpoint{1.426794in}{2.553923in}}%
\pgfpathlineto{\pgfqpoint{1.388447in}{2.536289in}}%
\pgfpathlineto{\pgfqpoint{1.352878in}{2.517566in}}%
\pgfpathlineto{\pgfqpoint{1.320128in}{2.497922in}}%
\pgfpathlineto{\pgfqpoint{1.288379in}{2.476236in}}%
\pgfpathlineto{\pgfqpoint{1.259592in}{2.453861in}}%
\pgfpathlineto{\pgfqpoint{1.232050in}{2.429520in}}%
\pgfpathlineto{\pgfqpoint{1.207527in}{2.404898in}}%
\pgfpathlineto{\pgfqpoint{1.184409in}{2.378557in}}%
\pgfpathlineto{\pgfqpoint{1.162828in}{2.350561in}}%
\pgfpathlineto{\pgfqpoint{1.142891in}{2.321011in}}%
\pgfpathlineto{\pgfqpoint{1.124675in}{2.290041in}}%
\pgfpathlineto{\pgfqpoint{1.108225in}{2.257802in}}%
\pgfpathlineto{\pgfqpoint{1.092639in}{2.222199in}}%
\pgfpathlineto{\pgfqpoint{1.079059in}{2.185535in}}%
\pgfpathlineto{\pgfqpoint{1.067443in}{2.147998in}}%
\pgfpathlineto{\pgfqpoint{1.057187in}{2.107348in}}%
\pgfpathlineto{\pgfqpoint{1.049004in}{2.066086in}}%
\pgfpathlineto{\pgfqpoint{1.042513in}{2.021906in}}%
\pgfpathlineto{\pgfqpoint{1.038177in}{1.977382in}}%
\pgfpathlineto{\pgfqpoint{1.035866in}{1.930167in}}%
\pgfpathlineto{\pgfqpoint{1.035826in}{1.882878in}}%
\pgfpathlineto{\pgfqpoint{1.038031in}{1.835656in}}%
\pgfpathlineto{\pgfqpoint{1.042474in}{1.788641in}}%
\pgfpathlineto{\pgfqpoint{1.049176in}{1.741979in}}%
\pgfpathlineto{\pgfqpoint{1.057644in}{1.698239in}}%
\pgfpathlineto{\pgfqpoint{1.068221in}{1.655105in}}%
\pgfpathlineto{\pgfqpoint{1.080962in}{1.612745in}}%
\pgfpathlineto{\pgfqpoint{1.095031in}{1.573617in}}%
\pgfpathlineto{\pgfqpoint{1.111115in}{1.535520in}}%
\pgfpathlineto{\pgfqpoint{1.128118in}{1.500775in}}%
\pgfpathlineto{\pgfqpoint{1.146930in}{1.467274in}}%
\pgfpathlineto{\pgfqpoint{1.167531in}{1.435181in}}%
\pgfpathlineto{\pgfqpoint{1.189874in}{1.404652in}}%
\pgfpathlineto{\pgfqpoint{1.213884in}{1.375828in}}%
\pgfpathlineto{\pgfqpoint{1.237817in}{1.350457in}}%
\pgfpathlineto{\pgfqpoint{1.264748in}{1.325237in}}%
\pgfpathlineto{\pgfqpoint{1.292991in}{1.301972in}}%
\pgfpathlineto{\pgfqpoint{1.322398in}{1.280678in}}%
\pgfpathlineto{\pgfqpoint{1.352820in}{1.261340in}}%
\pgfpathlineto{\pgfqpoint{1.386095in}{1.242889in}}%
\pgfpathlineto{\pgfqpoint{1.420190in}{1.226516in}}%
\pgfpathlineto{\pgfqpoint{1.457024in}{1.211329in}}%
\pgfpathlineto{\pgfqpoint{1.496554in}{1.197536in}}%
\pgfpathlineto{\pgfqpoint{1.538719in}{1.185287in}}%
\pgfpathlineto{\pgfqpoint{1.583441in}{1.174641in}}%
\pgfpathlineto{\pgfqpoint{1.634929in}{1.164775in}}%
\pgfpathlineto{\pgfqpoint{1.706063in}{1.153745in}}%
\pgfpathlineto{\pgfqpoint{1.768492in}{1.143417in}}%
\pgfpathlineto{\pgfqpoint{1.796122in}{1.136567in}}%
\pgfpathlineto{\pgfqpoint{1.812683in}{1.130481in}}%
\pgfpathlineto{\pgfqpoint{1.824471in}{1.124102in}}%
\pgfpathlineto{\pgfqpoint{1.833209in}{1.116741in}}%
\pgfpathlineto{\pgfqpoint{1.838498in}{1.108890in}}%
\pgfpathlineto{\pgfqpoint{1.840588in}{1.101849in}}%
\pgfpathlineto{\pgfqpoint{1.840619in}{1.094412in}}%
\pgfpathlineto{\pgfqpoint{1.837931in}{1.084986in}}%
\pgfpathlineto{\pgfqpoint{1.833246in}{1.076615in}}%
\pgfpathlineto{\pgfqpoint{1.825819in}{1.067542in}}%
\pgfpathlineto{\pgfqpoint{1.813813in}{1.056850in}}%
\pgfpathlineto{\pgfqpoint{1.798819in}{1.046763in}}%
\pgfpathlineto{\pgfqpoint{1.781016in}{1.037462in}}%
\pgfpathlineto{\pgfqpoint{1.758447in}{1.028391in}}%
\pgfpathlineto{\pgfqpoint{1.733203in}{1.020815in}}%
\pgfpathlineto{\pgfqpoint{1.705410in}{1.014872in}}%
\pgfpathlineto{\pgfqpoint{1.675178in}{1.010714in}}%
\pgfpathlineto{\pgfqpoint{1.642610in}{1.008507in}}%
\pgfpathlineto{\pgfqpoint{1.607809in}{1.008432in}}%
\pgfpathlineto{\pgfqpoint{1.570886in}{1.010691in}}%
\pgfpathlineto{\pgfqpoint{1.534118in}{1.015181in}}%
\pgfpathlineto{\pgfqpoint{1.495454in}{1.022233in}}%
\pgfpathlineto{\pgfqpoint{1.457161in}{1.031563in}}%
\pgfpathlineto{\pgfqpoint{1.419337in}{1.043132in}}%
\pgfpathlineto{\pgfqpoint{1.382089in}{1.056929in}}%
\pgfpathlineto{\pgfqpoint{1.347544in}{1.072019in}}%
\pgfpathlineto{\pgfqpoint{1.313727in}{1.089133in}}%
\pgfpathlineto{\pgfqpoint{1.280762in}{1.108299in}}%
\pgfpathlineto{\pgfqpoint{1.248782in}{1.129536in}}%
\pgfpathlineto{\pgfqpoint{1.219708in}{1.151422in}}%
\pgfpathlineto{\pgfqpoint{1.191752in}{1.175138in}}%
\pgfpathlineto{\pgfqpoint{1.165031in}{1.200649in}}%
\pgfpathlineto{\pgfqpoint{1.139653in}{1.227898in}}%
\pgfpathlineto{\pgfqpoint{1.115714in}{1.256800in}}%
\pgfpathlineto{\pgfqpoint{1.093288in}{1.287251in}}%
\pgfpathlineto{\pgfqpoint{1.071178in}{1.321163in}}%
\pgfpathlineto{\pgfqpoint{1.050868in}{1.356520in}}%
\pgfpathlineto{\pgfqpoint{1.032365in}{1.393152in}}%
\pgfpathlineto{\pgfqpoint{1.014718in}{1.433142in}}%
\pgfpathlineto{\pgfqpoint{0.999024in}{1.474185in}}%
\pgfpathlineto{\pgfqpoint{0.984506in}{1.518461in}}%
\pgfpathlineto{\pgfqpoint{0.972010in}{1.563537in}}%
\pgfpathlineto{\pgfqpoint{0.960944in}{1.611678in}}%
\pgfpathlineto{\pgfqpoint{0.951530in}{1.662824in}}%
\pgfpathlineto{\pgfqpoint{0.944286in}{1.714431in}}%
\pgfpathlineto{\pgfqpoint{0.938950in}{1.768847in}}%
\pgfpathlineto{\pgfqpoint{0.935870in}{1.823491in}}%
\pgfpathlineto{\pgfqpoint{0.935034in}{1.878240in}}%
\pgfpathlineto{\pgfqpoint{0.936466in}{1.932973in}}%
\pgfpathlineto{\pgfqpoint{0.940005in}{1.985084in}}%
\pgfpathlineto{\pgfqpoint{0.945759in}{2.036935in}}%
\pgfpathlineto{\pgfqpoint{0.953410in}{2.085938in}}%
\pgfpathlineto{\pgfqpoint{0.962764in}{2.132000in}}%
\pgfpathlineto{\pgfqpoint{0.974287in}{2.177414in}}%
\pgfpathlineto{\pgfqpoint{0.987332in}{2.219653in}}%
\pgfpathlineto{\pgfqpoint{1.001667in}{2.258654in}}%
\pgfpathlineto{\pgfqpoint{1.018051in}{2.296583in}}%
\pgfpathlineto{\pgfqpoint{1.035401in}{2.331101in}}%
\pgfpathlineto{\pgfqpoint{1.054650in}{2.364275in}}%
\pgfpathlineto{\pgfqpoint{1.074406in}{2.393984in}}%
\pgfpathlineto{\pgfqpoint{1.095771in}{2.422197in}}%
\pgfpathlineto{\pgfqpoint{1.118662in}{2.448797in}}%
\pgfpathlineto{\pgfqpoint{1.142967in}{2.473701in}}%
\pgfpathlineto{\pgfqpoint{1.168550in}{2.496867in}}%
\pgfpathlineto{\pgfqpoint{1.197085in}{2.519662in}}%
\pgfpathlineto{\pgfqpoint{1.226727in}{2.540526in}}%
\pgfpathlineto{\pgfqpoint{1.259242in}{2.560673in}}%
\pgfpathlineto{\pgfqpoint{1.294612in}{2.579881in}}%
\pgfpathlineto{\pgfqpoint{1.332792in}{2.597982in}}%
\pgfpathlineto{\pgfqpoint{1.373719in}{2.614859in}}%
\pgfpathlineto{\pgfqpoint{1.417319in}{2.630445in}}%
\pgfpathlineto{\pgfqpoint{1.465632in}{2.645312in}}%
\pgfpathlineto{\pgfqpoint{1.518640in}{2.659204in}}%
\pgfpathlineto{\pgfqpoint{1.576309in}{2.671929in}}%
\pgfpathlineto{\pgfqpoint{1.638597in}{2.683344in}}%
\pgfpathlineto{\pgfqpoint{1.705462in}{2.693343in}}%
\pgfpathlineto{\pgfqpoint{1.779027in}{2.702064in}}%
\pgfpathlineto{\pgfqpoint{1.857097in}{2.709077in}}%
\pgfpathlineto{\pgfqpoint{1.939633in}{2.714280in}}%
\pgfpathlineto{\pgfqpoint{2.026598in}{2.717513in}}%
\pgfpathlineto{\pgfqpoint{2.113605in}{2.718523in}}%
\pgfpathlineto{\pgfqpoint{2.198435in}{2.717303in}}%
\pgfpathlineto{\pgfqpoint{2.278866in}{2.713929in}}%
\pgfpathlineto{\pgfqpoint{2.352678in}{2.708598in}}%
\pgfpathlineto{\pgfqpoint{2.417657in}{2.701709in}}%
\pgfpathlineto{\pgfqpoint{2.473770in}{2.693630in}}%
\pgfpathlineto{\pgfqpoint{2.523140in}{2.684368in}}%
\pgfpathlineto{\pgfqpoint{2.565726in}{2.674202in}}%
\pgfpathlineto{\pgfqpoint{2.601510in}{2.663544in}}%
\pgfpathlineto{\pgfqpoint{2.632577in}{2.652142in}}%
\pgfpathlineto{\pgfqpoint{2.658899in}{2.640331in}}%
\pgfpathlineto{\pgfqpoint{2.682438in}{2.627436in}}%
\pgfpathlineto{\pgfqpoint{2.703062in}{2.613571in}}%
\pgfpathlineto{\pgfqpoint{2.720674in}{2.598978in}}%
\pgfpathlineto{\pgfqpoint{2.735263in}{2.584053in}}%
\pgfpathlineto{\pgfqpoint{2.748320in}{2.567377in}}%
\pgfpathlineto{\pgfqpoint{2.759553in}{2.549046in}}%
\pgfpathlineto{\pgfqpoint{2.768788in}{2.529306in}}%
\pgfpathlineto{\pgfqpoint{2.776017in}{2.508498in}}%
\pgfpathlineto{\pgfqpoint{2.781884in}{2.484540in}}%
\pgfpathlineto{\pgfqpoint{2.786102in}{2.457597in}}%
\pgfpathlineto{\pgfqpoint{2.788720in}{2.425384in}}%
\pgfpathlineto{\pgfqpoint{2.789427in}{2.388061in}}%
\pgfpathlineto{\pgfqpoint{2.787962in}{2.340801in}}%
\pgfpathlineto{\pgfqpoint{2.783672in}{2.278768in}}%
\pgfpathlineto{\pgfqpoint{2.774289in}{2.179783in}}%
\pgfpathlineto{\pgfqpoint{2.743611in}{1.868119in}}%
\pgfpathlineto{\pgfqpoint{2.730112in}{1.702060in}}%
\pgfpathlineto{\pgfqpoint{2.717287in}{1.515949in}}%
\pgfpathlineto{\pgfqpoint{2.702602in}{1.267597in}}%
\pgfpathlineto{\pgfqpoint{2.684434in}{0.964630in}}%
\pgfpathlineto{\pgfqpoint{2.675374in}{0.850600in}}%
\pgfpathlineto{\pgfqpoint{2.667030in}{0.771523in}}%
\pgfpathlineto{\pgfqpoint{2.658752in}{0.712543in}}%
\pgfpathlineto{\pgfqpoint{2.650176in}{0.666284in}}%
\pgfpathlineto{\pgfqpoint{2.640820in}{0.627931in}}%
\pgfpathlineto{\pgfqpoint{2.631145in}{0.597534in}}%
\pgfpathlineto{\pgfqpoint{2.621004in}{0.572745in}}%
\pgfpathlineto{\pgfqpoint{2.609856in}{0.551383in}}%
\pgfpathlineto{\pgfqpoint{2.598042in}{0.533534in}}%
\pgfpathlineto{\pgfqpoint{2.584496in}{0.517378in}}%
\pgfpathlineto{\pgfqpoint{2.571109in}{0.504669in}}%
\pgfpathlineto{\pgfqpoint{2.554789in}{0.492313in}}%
\pgfpathlineto{\pgfqpoint{2.537457in}{0.481914in}}%
\pgfpathlineto{\pgfqpoint{2.517374in}{0.472367in}}%
\pgfpathlineto{\pgfqpoint{2.492542in}{0.463178in}}%
\pgfpathlineto{\pgfqpoint{2.462979in}{0.454833in}}%
\pgfpathlineto{\pgfqpoint{2.428766in}{0.447542in}}%
\pgfpathlineto{\pgfqpoint{2.385671in}{0.440735in}}%
\pgfpathlineto{\pgfqpoint{2.331557in}{0.434581in}}%
\pgfpathlineto{\pgfqpoint{2.262115in}{0.429077in}}%
\pgfpathlineto{\pgfqpoint{2.170851in}{0.424236in}}%
\pgfpathlineto{\pgfqpoint{2.049086in}{0.420134in}}%
\pgfpathlineto{\pgfqpoint{1.879436in}{0.416783in}}%
\pgfpathlineto{\pgfqpoint{1.640159in}{0.414418in}}%
\pgfpathlineto{\pgfqpoint{1.322562in}{0.413569in}}%
\pgfpathlineto{\pgfqpoint{1.020194in}{0.414850in}}%
\pgfpathlineto{\pgfqpoint{0.822256in}{0.417715in}}%
\pgfpathlineto{\pgfqpoint{0.704835in}{0.421430in}}%
\pgfpathlineto{\pgfqpoint{0.630976in}{0.425829in}}%
\pgfpathlineto{\pgfqpoint{0.583316in}{0.430734in}}%
\pgfpathlineto{\pgfqpoint{0.551033in}{0.436123in}}%
\pgfpathlineto{\pgfqpoint{0.527708in}{0.442189in}}%
\pgfpathlineto{\pgfqpoint{0.511250in}{0.448625in}}%
\pgfpathlineto{\pgfqpoint{0.499549in}{0.455216in}}%
\pgfpathlineto{\pgfqpoint{0.488916in}{0.463841in}}%
\pgfpathlineto{\pgfqpoint{0.481322in}{0.472730in}}%
\pgfpathlineto{\pgfqpoint{0.474078in}{0.485127in}}%
\pgfpathlineto{\pgfqpoint{0.468753in}{0.498748in}}%
\pgfpathlineto{\pgfqpoint{0.463870in}{0.517848in}}%
\pgfpathlineto{\pgfqpoint{0.459679in}{0.544796in}}%
\pgfpathlineto{\pgfqpoint{0.456386in}{0.581938in}}%
\pgfpathlineto{\pgfqpoint{0.453731in}{0.639106in}}%
\pgfpathlineto{\pgfqpoint{0.451681in}{0.736155in}}%
\pgfpathlineto{\pgfqpoint{0.450220in}{0.927815in}}%
\pgfpathlineto{\pgfqpoint{0.449345in}{1.403252in}}%
\pgfpathlineto{\pgfqpoint{0.449543in}{2.682703in}}%
\pgfpathlineto{\pgfqpoint{0.451011in}{2.856932in}}%
\pgfpathlineto{\pgfqpoint{0.452802in}{2.879219in}}%
\pgfpathlineto{\pgfqpoint{0.455188in}{2.886108in}}%
\pgfpathlineto{\pgfqpoint{0.458626in}{2.889028in}}%
\pgfpathlineto{\pgfqpoint{0.464996in}{2.890553in}}%
\pgfpathlineto{\pgfqpoint{0.482377in}{2.891423in}}%
\pgfpathlineto{\pgfqpoint{0.565038in}{2.891729in}}%
\pgfpathlineto{\pgfqpoint{2.733842in}{2.891760in}}%
\pgfpathlineto{\pgfqpoint{4.789510in}{2.890885in}}%
\pgfpathlineto{\pgfqpoint{4.793727in}{2.889730in}}%
\pgfpathlineto{\pgfqpoint{4.795481in}{2.888307in}}%
\pgfpathlineto{\pgfqpoint{4.797106in}{2.881145in}}%
\pgfpathlineto{\pgfqpoint{4.797997in}{2.858771in}}%
\pgfpathlineto{\pgfqpoint{4.798039in}{2.856283in}}%
\pgfpathlineto{\pgfqpoint{4.798039in}{2.856283in}}%
\pgfusepath{stroke}%
\end{pgfscope}%
\begin{pgfscope}%
\pgfpathrectangle{\pgfqpoint{0.448634in}{0.402556in}}{\pgfqpoint{4.350661in}{2.489204in}} %
\pgfusepath{clip}%
\pgfsetrectcap%
\pgfsetroundjoin%
\pgfsetlinewidth{1.003750pt}%
\definecolor{currentstroke}{rgb}{0.121569,0.466667,0.705882}%
\pgfsetstrokecolor{currentstroke}%
\pgfsetdash{}{0pt}%
\pgfpathmoveto{\pgfqpoint{3.428772in}{0.402610in}}%
\pgfpathlineto{\pgfqpoint{2.806632in}{0.403760in}}%
\pgfpathlineto{\pgfqpoint{2.769692in}{0.405578in}}%
\pgfpathlineto{\pgfqpoint{2.754632in}{0.408064in}}%
\pgfpathlineto{\pgfqpoint{2.746391in}{0.411198in}}%
\pgfpathlineto{\pgfqpoint{2.740943in}{0.415265in}}%
\pgfpathlineto{\pgfqpoint{2.736784in}{0.420984in}}%
\pgfpathlineto{\pgfqpoint{2.733281in}{0.430071in}}%
\pgfpathlineto{\pgfqpoint{2.730449in}{0.444636in}}%
\pgfpathlineto{\pgfqpoint{2.728238in}{0.469392in}}%
\pgfpathlineto{\pgfqpoint{2.726470in}{0.519131in}}%
\pgfpathlineto{\pgfqpoint{2.725711in}{0.613715in}}%
\pgfpathlineto{\pgfqpoint{2.726842in}{0.768038in}}%
\pgfpathlineto{\pgfqpoint{2.730556in}{0.962148in}}%
\pgfpathlineto{\pgfqpoint{2.736611in}{1.158670in}}%
\pgfpathlineto{\pgfqpoint{2.744092in}{1.327718in}}%
\pgfpathlineto{\pgfqpoint{2.753201in}{1.484189in}}%
\pgfpathlineto{\pgfqpoint{2.763257in}{1.620609in}}%
\pgfpathlineto{\pgfqpoint{2.776118in}{1.764216in}}%
\pgfpathlineto{\pgfqpoint{2.788914in}{1.877776in}}%
\pgfpathlineto{\pgfqpoint{2.805748in}{2.005740in}}%
\pgfpathlineto{\pgfqpoint{2.821176in}{2.101198in}}%
\pgfpathlineto{\pgfqpoint{2.838359in}{2.193718in}}%
\pgfpathlineto{\pgfqpoint{2.859135in}{2.292966in}}%
\pgfpathlineto{\pgfqpoint{2.887209in}{2.425960in}}%
\pgfpathlineto{\pgfqpoint{2.896991in}{2.479559in}}%
\pgfpathlineto{\pgfqpoint{2.901543in}{2.516523in}}%
\pgfpathlineto{\pgfqpoint{2.902849in}{2.543854in}}%
\pgfpathlineto{\pgfqpoint{2.901957in}{2.566223in}}%
\pgfpathlineto{\pgfqpoint{2.899151in}{2.585863in}}%
\pgfpathlineto{\pgfqpoint{2.894794in}{2.602546in}}%
\pgfpathlineto{\pgfqpoint{2.888484in}{2.618388in}}%
\pgfpathlineto{\pgfqpoint{2.880257in}{2.633033in}}%
\pgfpathlineto{\pgfqpoint{2.870348in}{2.646246in}}%
\pgfpathlineto{\pgfqpoint{2.857400in}{2.659530in}}%
\pgfpathlineto{\pgfqpoint{2.843189in}{2.671010in}}%
\pgfpathlineto{\pgfqpoint{2.824237in}{2.683209in}}%
\pgfpathlineto{\pgfqpoint{2.802413in}{2.694418in}}%
\pgfpathlineto{\pgfqpoint{2.775809in}{2.705369in}}%
\pgfpathlineto{\pgfqpoint{2.744461in}{2.715715in}}%
\pgfpathlineto{\pgfqpoint{2.708436in}{2.725252in}}%
\pgfpathlineto{\pgfqpoint{2.665655in}{2.734289in}}%
\pgfpathlineto{\pgfqpoint{2.613991in}{2.742869in}}%
\pgfpathlineto{\pgfqpoint{2.553459in}{2.750589in}}%
\pgfpathlineto{\pgfqpoint{2.481920in}{2.757365in}}%
\pgfpathlineto{\pgfqpoint{2.399398in}{2.762839in}}%
\pgfpathlineto{\pgfqpoint{2.310269in}{2.766482in}}%
\pgfpathlineto{\pgfqpoint{2.175416in}{2.768725in}}%
\pgfpathlineto{\pgfqpoint{2.066653in}{2.767942in}}%
\pgfpathlineto{\pgfqpoint{1.953570in}{2.764859in}}%
\pgfpathlineto{\pgfqpoint{1.851429in}{2.759759in}}%
\pgfpathlineto{\pgfqpoint{1.745051in}{2.752169in}}%
\pgfpathlineto{\pgfqpoint{1.658373in}{2.743453in}}%
\pgfpathlineto{\pgfqpoint{1.580552in}{2.733461in}}%
\pgfpathlineto{\pgfqpoint{1.490057in}{2.719338in}}%
\pgfpathlineto{\pgfqpoint{1.417231in}{2.704698in}}%
\pgfpathlineto{\pgfqpoint{1.361992in}{2.690818in}}%
\pgfpathlineto{\pgfqpoint{1.311460in}{2.675819in}}%
\pgfpathlineto{\pgfqpoint{1.265667in}{2.659924in}}%
\pgfpathlineto{\pgfqpoint{1.222575in}{2.642586in}}%
\pgfpathlineto{\pgfqpoint{1.184324in}{2.624682in}}%
\pgfpathlineto{\pgfqpoint{1.148892in}{2.605623in}}%
\pgfpathlineto{\pgfqpoint{1.116331in}{2.585573in}}%
\pgfpathlineto{\pgfqpoint{1.092327in}{2.568512in}}%
\pgfpathlineto{\pgfqpoint{1.079760in}{2.558686in}}%
\pgfpathlineto{\pgfqpoint{1.051544in}{2.535379in}}%
\pgfpathlineto{\pgfqpoint{1.026312in}{2.511712in}}%
\pgfpathlineto{\pgfqpoint{1.002399in}{2.486318in}}%
\pgfpathlineto{\pgfqpoint{0.979913in}{2.459269in}}%
\pgfpathlineto{\pgfqpoint{0.958934in}{2.430678in}}%
\pgfpathlineto{\pgfqpoint{0.938264in}{2.398643in}}%
\pgfpathlineto{\pgfqpoint{0.923047in}{2.371385in}}%
\pgfpathlineto{\pgfqpoint{0.904513in}{2.334774in}}%
\pgfpathlineto{\pgfqpoint{0.887854in}{2.297001in}}%
\pgfpathlineto{\pgfqpoint{0.872131in}{2.255971in}}%
\pgfpathlineto{\pgfqpoint{0.857508in}{2.211741in}}%
\pgfpathlineto{\pgfqpoint{0.844762in}{2.166757in}}%
\pgfpathlineto{\pgfqpoint{0.838624in}{2.140306in}}%
\pgfpathlineto{\pgfqpoint{0.826982in}{2.087194in}}%
\pgfpathlineto{\pgfqpoint{0.816322in}{2.028715in}}%
\pgfpathlineto{\pgfqpoint{0.810087in}{1.984495in}}%
\pgfpathlineto{\pgfqpoint{0.808026in}{1.967238in}}%
\pgfpathlineto{\pgfqpoint{0.800076in}{1.898140in}}%
\pgfpathlineto{\pgfqpoint{0.793713in}{1.823823in}}%
\pgfpathlineto{\pgfqpoint{0.788799in}{1.741875in}}%
\pgfpathlineto{\pgfqpoint{0.786199in}{1.677225in}}%
\pgfpathlineto{\pgfqpoint{0.776951in}{1.453481in}}%
\pgfpathlineto{\pgfqpoint{0.773280in}{1.418894in}}%
\pgfpathlineto{\pgfqpoint{0.768298in}{1.389582in}}%
\pgfpathlineto{\pgfqpoint{0.762752in}{1.368108in}}%
\pgfpathlineto{\pgfqpoint{0.756722in}{1.352123in}}%
\pgfpathlineto{\pgfqpoint{0.749752in}{1.339519in}}%
\pgfpathlineto{\pgfqpoint{0.742201in}{1.330599in}}%
\pgfpathlineto{\pgfqpoint{0.734854in}{1.325312in}}%
\pgfpathlineto{\pgfqpoint{0.726558in}{1.322419in}}%
\pgfpathlineto{\pgfqpoint{0.717884in}{1.322223in}}%
\pgfpathlineto{\pgfqpoint{0.709412in}{1.324411in}}%
\pgfpathlineto{\pgfqpoint{0.699548in}{1.329604in}}%
\pgfpathlineto{\pgfqpoint{0.688894in}{1.338203in}}%
\pgfpathlineto{\pgfqpoint{0.677907in}{1.350248in}}%
\pgfpathlineto{\pgfqpoint{0.666886in}{1.365647in}}%
\pgfpathlineto{\pgfqpoint{0.654913in}{1.386417in}}%
\pgfpathlineto{\pgfqpoint{0.642574in}{1.412730in}}%
\pgfpathlineto{\pgfqpoint{0.630328in}{1.444629in}}%
\pgfpathlineto{\pgfqpoint{0.618504in}{1.482081in}}%
\pgfpathlineto{\pgfqpoint{0.608613in}{1.520256in}}%
\pgfpathlineto{\pgfqpoint{0.590203in}{1.612445in}}%
\pgfpathlineto{\pgfqpoint{0.581848in}{1.668884in}}%
\pgfpathlineto{\pgfqpoint{0.573137in}{1.740376in}}%
\pgfpathlineto{\pgfqpoint{0.567062in}{1.807213in}}%
\pgfpathlineto{\pgfqpoint{0.560532in}{1.896510in}}%
\pgfpathlineto{\pgfqpoint{0.555526in}{1.995910in}}%
\pgfpathlineto{\pgfqpoint{0.552564in}{2.097908in}}%
\pgfpathlineto{\pgfqpoint{0.551526in}{2.204935in}}%
\pgfpathlineto{\pgfqpoint{0.552728in}{2.309470in}}%
\pgfpathlineto{\pgfqpoint{0.556011in}{2.403981in}}%
\pgfpathlineto{\pgfqpoint{0.560953in}{2.483430in}}%
\pgfpathlineto{\pgfqpoint{0.567303in}{2.550240in}}%
\pgfpathlineto{\pgfqpoint{0.574928in}{2.606817in}}%
\pgfpathlineto{\pgfqpoint{0.582988in}{2.650657in}}%
\pgfpathlineto{\pgfqpoint{0.592756in}{2.691452in}}%
\pgfpathlineto{\pgfqpoint{0.602650in}{2.721756in}}%
\pgfpathlineto{\pgfqpoint{0.612983in}{2.746441in}}%
\pgfpathlineto{\pgfqpoint{0.624292in}{2.767692in}}%
\pgfpathlineto{\pgfqpoint{0.636231in}{2.785433in}}%
\pgfpathlineto{\pgfqpoint{0.649892in}{2.801461in}}%
\pgfpathlineto{\pgfqpoint{0.663386in}{2.814020in}}%
\pgfpathlineto{\pgfqpoint{0.679842in}{2.826135in}}%
\pgfpathlineto{\pgfqpoint{0.697326in}{2.836197in}}%
\pgfpathlineto{\pgfqpoint{0.715574in}{2.844285in}}%
\pgfpathlineto{\pgfqpoint{0.738439in}{2.852335in}}%
\pgfpathlineto{\pgfqpoint{0.765983in}{2.859639in}}%
\pgfpathlineto{\pgfqpoint{0.800300in}{2.866256in}}%
\pgfpathlineto{\pgfqpoint{0.841340in}{2.871832in}}%
\pgfpathlineto{\pgfqpoint{0.895547in}{2.876803in}}%
\pgfpathlineto{\pgfqpoint{0.969413in}{2.881069in}}%
\pgfpathlineto{\pgfqpoint{1.071608in}{2.884501in}}%
\pgfpathlineto{\pgfqpoint{1.219512in}{2.887074in}}%
\pgfpathlineto{\pgfqpoint{1.471844in}{2.889091in}}%
\pgfpathlineto{\pgfqpoint{1.956941in}{2.890384in}}%
\pgfpathlineto{\pgfqpoint{3.096814in}{2.890781in}}%
\pgfpathlineto{\pgfqpoint{3.995224in}{2.889388in}}%
\pgfpathlineto{\pgfqpoint{4.275833in}{2.887011in}}%
\pgfpathlineto{\pgfqpoint{4.412847in}{2.883743in}}%
\pgfpathlineto{\pgfqpoint{4.491081in}{2.879810in}}%
\pgfpathlineto{\pgfqpoint{4.543127in}{2.875163in}}%
\pgfpathlineto{\pgfqpoint{4.579810in}{2.869841in}}%
\pgfpathlineto{\pgfqpoint{4.607580in}{2.863763in}}%
\pgfpathlineto{\pgfqpoint{4.630623in}{2.856424in}}%
\pgfpathlineto{\pgfqpoint{4.648833in}{2.848228in}}%
\pgfpathlineto{\pgfqpoint{4.664136in}{2.838773in}}%
\pgfpathlineto{\pgfqpoint{4.676470in}{2.828576in}}%
\pgfpathlineto{\pgfqpoint{4.687502in}{2.816585in}}%
\pgfpathlineto{\pgfqpoint{4.697051in}{2.803027in}}%
\pgfpathlineto{\pgfqpoint{4.706194in}{2.786098in}}%
\pgfpathlineto{\pgfqpoint{4.714508in}{2.765827in}}%
\pgfpathlineto{\pgfqpoint{4.722462in}{2.740013in}}%
\pgfpathlineto{\pgfqpoint{4.729577in}{2.708703in}}%
\pgfpathlineto{\pgfqpoint{4.736162in}{2.669601in}}%
\pgfpathlineto{\pgfqpoint{4.742419in}{2.617826in}}%
\pgfpathlineto{\pgfqpoint{4.747859in}{2.553410in}}%
\pgfpathlineto{\pgfqpoint{4.752661in}{2.468958in}}%
\pgfpathlineto{\pgfqpoint{4.756610in}{2.359528in}}%
\pgfpathlineto{\pgfqpoint{4.759416in}{2.217681in}}%
\pgfpathlineto{\pgfqpoint{4.760596in}{2.043444in}}%
\pgfpathlineto{\pgfqpoint{4.759662in}{1.851779in}}%
\pgfpathlineto{\pgfqpoint{4.756587in}{1.667613in}}%
\pgfpathlineto{\pgfqpoint{4.751596in}{1.503428in}}%
\pgfpathlineto{\pgfqpoint{4.745410in}{1.374185in}}%
\pgfpathlineto{\pgfqpoint{4.738113in}{1.267479in}}%
\pgfpathlineto{\pgfqpoint{4.729621in}{1.175896in}}%
\pgfpathlineto{\pgfqpoint{4.720762in}{1.104428in}}%
\pgfpathlineto{\pgfqpoint{4.711045in}{1.043204in}}%
\pgfpathlineto{\pgfqpoint{4.700364in}{0.989829in}}%
\pgfpathlineto{\pgfqpoint{4.689055in}{0.944345in}}%
\pgfpathlineto{\pgfqpoint{4.676881in}{0.904394in}}%
\pgfpathlineto{\pgfqpoint{4.676095in}{0.902073in}}%
\pgfpathlineto{\pgfqpoint{4.676095in}{0.902073in}}%
\pgfusepath{stroke}%
\end{pgfscope}%
\begin{pgfscope}%
\pgfpathrectangle{\pgfqpoint{0.448634in}{0.402556in}}{\pgfqpoint{4.350661in}{2.489204in}} %
\pgfusepath{clip}%
\pgfsetrectcap%
\pgfsetroundjoin%
\pgfsetlinewidth{1.003750pt}%
\definecolor{currentstroke}{rgb}{0.121569,0.466667,0.705882}%
\pgfsetstrokecolor{currentstroke}%
\pgfsetdash{}{0pt}%
\pgfpathmoveto{\pgfqpoint{2.795520in}{1.982745in}}%
\pgfpathlineto{\pgfqpoint{2.781780in}{1.874357in}}%
\pgfpathlineto{\pgfqpoint{2.769351in}{1.758234in}}%
\pgfpathlineto{\pgfqpoint{2.758095in}{1.631942in}}%
\pgfpathlineto{\pgfqpoint{2.747786in}{1.490551in}}%
\pgfpathlineto{\pgfqpoint{2.738644in}{1.334082in}}%
\pgfpathlineto{\pgfqpoint{2.730580in}{1.157591in}}%
\pgfpathlineto{\pgfqpoint{2.723334in}{0.948663in}}%
\pgfpathlineto{\pgfqpoint{2.709783in}{0.530788in}}%
\pgfpathlineto{\pgfqpoint{2.705868in}{0.488716in}}%
\pgfpathlineto{\pgfqpoint{2.701769in}{0.464281in}}%
\pgfpathlineto{\pgfqpoint{2.697021in}{0.447744in}}%
\pgfpathlineto{\pgfqpoint{2.691859in}{0.436812in}}%
\pgfpathlineto{\pgfqpoint{2.686245in}{0.429229in}}%
\pgfpathlineto{\pgfqpoint{2.679348in}{0.423188in}}%
\pgfpathlineto{\pgfqpoint{2.669540in}{0.417856in}}%
\pgfpathlineto{\pgfqpoint{2.656987in}{0.413810in}}%
\pgfpathlineto{\pgfqpoint{2.637654in}{0.410337in}}%
\pgfpathlineto{\pgfqpoint{2.607297in}{0.407617in}}%
\pgfpathlineto{\pgfqpoint{2.555121in}{0.405574in}}%
\pgfpathlineto{\pgfqpoint{2.450714in}{0.404139in}}%
\pgfpathlineto{\pgfqpoint{2.176624in}{0.403275in}}%
\pgfpathlineto{\pgfqpoint{1.130290in}{0.402953in}}%
\pgfpathlineto{\pgfqpoint{0.516849in}{0.404175in}}%
\pgfpathlineto{\pgfqpoint{0.466848in}{0.405970in}}%
\pgfpathlineto{\pgfqpoint{0.456130in}{0.407931in}}%
\pgfpathlineto{\pgfqpoint{0.452340in}{0.410303in}}%
\pgfpathlineto{\pgfqpoint{0.450346in}{0.414662in}}%
\pgfpathlineto{\pgfqpoint{0.449266in}{0.424524in}}%
\pgfpathlineto{\pgfqpoint{0.448771in}{0.464344in}}%
\pgfpathlineto{\pgfqpoint{0.448640in}{0.850171in}}%
\pgfpathlineto{\pgfqpoint{0.448679in}{2.891318in}}%
\pgfpathlineto{\pgfqpoint{0.448679in}{2.891318in}}%
\pgfusepath{stroke}%
\end{pgfscope}%
\begin{pgfscope}%
\pgfpathrectangle{\pgfqpoint{0.448634in}{0.402556in}}{\pgfqpoint{4.350661in}{2.489204in}} %
\pgfusepath{clip}%
\pgfsetrectcap%
\pgfsetroundjoin%
\pgfsetlinewidth{1.003750pt}%
\definecolor{currentstroke}{rgb}{0.121569,0.466667,0.705882}%
\pgfsetstrokecolor{currentstroke}%
\pgfsetdash{}{0pt}%
\pgfpathmoveto{\pgfqpoint{3.428189in}{0.402586in}}%
\pgfpathlineto{\pgfqpoint{2.782121in}{0.403701in}}%
\pgfpathlineto{\pgfqpoint{2.753906in}{0.405674in}}%
\pgfpathlineto{\pgfqpoint{2.743328in}{0.408443in}}%
\pgfpathlineto{\pgfqpoint{2.737717in}{0.412188in}}%
\pgfpathlineto{\pgfqpoint{2.733668in}{0.417995in}}%
\pgfpathlineto{\pgfqpoint{2.730649in}{0.427307in}}%
\pgfpathlineto{\pgfqpoint{2.728388in}{0.442004in}}%
\pgfpathlineto{\pgfqpoint{2.726544in}{0.471794in}}%
\pgfpathlineto{\pgfqpoint{2.725216in}{0.534003in}}%
\pgfpathlineto{\pgfqpoint{2.725169in}{0.655973in}}%
\pgfpathlineto{\pgfqpoint{2.727377in}{0.832687in}}%
\pgfpathlineto{\pgfqpoint{2.732259in}{1.041703in}}%
\pgfpathlineto{\pgfqpoint{2.738851in}{1.223257in}}%
\pgfpathlineto{\pgfqpoint{2.747078in}{1.389766in}}%
\pgfpathlineto{\pgfqpoint{2.756608in}{1.538717in}}%
\pgfpathlineto{\pgfqpoint{2.768955in}{1.694887in}}%
\pgfpathlineto{\pgfqpoint{2.781228in}{1.816044in}}%
\pgfpathlineto{\pgfqpoint{2.794401in}{1.924524in}}%
\pgfpathlineto{\pgfqpoint{2.812737in}{2.054722in}}%
\pgfpathlineto{\pgfqpoint{2.828774in}{2.147512in}}%
\pgfpathlineto{\pgfqpoint{2.847382in}{2.242224in}}%
\pgfpathlineto{\pgfqpoint{2.895818in}{2.479699in}}%
\pgfpathlineto{\pgfqpoint{2.900204in}{2.516689in}}%
\pgfpathlineto{\pgfqpoint{2.901346in}{2.544029in}}%
\pgfpathlineto{\pgfqpoint{2.900291in}{2.566388in}}%
\pgfpathlineto{\pgfqpoint{2.897334in}{2.585999in}}%
\pgfpathlineto{\pgfqpoint{2.892836in}{2.602633in}}%
\pgfpathlineto{\pgfqpoint{2.886394in}{2.618405in}}%
\pgfpathlineto{\pgfqpoint{2.878058in}{2.632969in}}%
\pgfpathlineto{\pgfqpoint{2.868065in}{2.646100in}}%
\pgfpathlineto{\pgfqpoint{2.855050in}{2.659300in}}%
\pgfpathlineto{\pgfqpoint{2.840801in}{2.670717in}}%
\pgfpathlineto{\pgfqpoint{2.821822in}{2.682861in}}%
\pgfpathlineto{\pgfqpoint{2.799980in}{2.694026in}}%
\pgfpathlineto{\pgfqpoint{2.773366in}{2.704944in}}%
\pgfpathlineto{\pgfqpoint{2.742012in}{2.715266in}}%
\pgfpathlineto{\pgfqpoint{2.705983in}{2.724785in}}%
\pgfpathlineto{\pgfqpoint{2.663200in}{2.733810in}}%
\pgfpathlineto{\pgfqpoint{2.611535in}{2.742379in}}%
\pgfpathlineto{\pgfqpoint{2.551002in}{2.750090in}}%
\pgfpathlineto{\pgfqpoint{2.481632in}{2.756682in}}%
\pgfpathlineto{\pgfqpoint{2.399112in}{2.762200in}}%
\pgfpathlineto{\pgfqpoint{2.309985in}{2.765886in}}%
\pgfpathlineto{\pgfqpoint{2.188184in}{2.768096in}}%
\pgfpathlineto{\pgfqpoint{2.081595in}{2.767619in}}%
\pgfpathlineto{\pgfqpoint{1.968506in}{2.764840in}}%
\pgfpathlineto{\pgfqpoint{1.864180in}{2.759918in}}%
\pgfpathlineto{\pgfqpoint{1.757786in}{2.752593in}}%
\pgfpathlineto{\pgfqpoint{1.671087in}{2.744171in}}%
\pgfpathlineto{\pgfqpoint{1.591076in}{2.734193in}}%
\pgfpathlineto{\pgfqpoint{1.502689in}{2.720717in}}%
\pgfpathlineto{\pgfqpoint{1.427655in}{2.706083in}}%
\pgfpathlineto{\pgfqpoint{1.372350in}{2.692544in}}%
\pgfpathlineto{\pgfqpoint{1.321734in}{2.677921in}}%
\pgfpathlineto{\pgfqpoint{1.273765in}{2.661664in}}%
\pgfpathlineto{\pgfqpoint{1.230567in}{2.644672in}}%
\pgfpathlineto{\pgfqpoint{1.192197in}{2.627106in}}%
\pgfpathlineto{\pgfqpoint{1.156620in}{2.608403in}}%
\pgfpathlineto{\pgfqpoint{1.123890in}{2.588716in}}%
\pgfpathlineto{\pgfqpoint{1.095883in}{2.569568in}}%
\pgfpathlineto{\pgfqpoint{1.063936in}{2.543701in}}%
\pgfpathlineto{\pgfqpoint{1.038217in}{2.520732in}}%
\pgfpathlineto{\pgfqpoint{1.013766in}{2.496016in}}%
\pgfpathlineto{\pgfqpoint{0.990704in}{2.469610in}}%
\pgfpathlineto{\pgfqpoint{0.969124in}{2.441612in}}%
\pgfpathlineto{\pgfqpoint{0.949083in}{2.412154in}}%
\pgfpathlineto{\pgfqpoint{0.930604in}{2.381387in}}%
\pgfpathlineto{\pgfqpoint{0.906555in}{2.334052in}}%
\pgfpathlineto{\pgfqpoint{0.889925in}{2.296262in}}%
\pgfpathlineto{\pgfqpoint{0.874241in}{2.255213in}}%
\pgfpathlineto{\pgfqpoint{0.859667in}{2.210961in}}%
\pgfpathlineto{\pgfqpoint{0.846986in}{2.165954in}}%
\pgfpathlineto{\pgfqpoint{0.839633in}{2.134715in}}%
\pgfpathlineto{\pgfqpoint{0.828238in}{2.081532in}}%
\pgfpathlineto{\pgfqpoint{0.817866in}{2.022986in}}%
\pgfpathlineto{\pgfqpoint{0.810784in}{1.971352in}}%
\pgfpathlineto{\pgfqpoint{0.802846in}{1.902252in}}%
\pgfpathlineto{\pgfqpoint{0.796554in}{1.827927in}}%
\pgfpathlineto{\pgfqpoint{0.791696in}{1.743480in}}%
\pgfpathlineto{\pgfqpoint{0.787773in}{1.621595in}}%
\pgfpathlineto{\pgfqpoint{0.785408in}{1.522064in}}%
\pgfpathlineto{\pgfqpoint{0.785408in}{1.522064in}}%
\pgfusepath{stroke}%
\end{pgfscope}%
\begin{pgfscope}%
\pgfpathrectangle{\pgfqpoint{0.448634in}{0.402556in}}{\pgfqpoint{4.350661in}{2.489204in}} %
\pgfusepath{clip}%
\pgfsetrectcap%
\pgfsetroundjoin%
\pgfsetlinewidth{1.003750pt}%
\definecolor{currentstroke}{rgb}{1.000000,0.498039,0.054902}%
\pgfsetstrokecolor{currentstroke}%
\pgfsetdash{}{0pt}%
\pgfpathmoveto{\pgfqpoint{1.127319in}{2.572074in}}%
\pgfpathlineto{\pgfqpoint{1.159575in}{2.592758in}}%
\pgfpathlineto{\pgfqpoint{1.192763in}{2.611414in}}%
\pgfpathlineto{\pgfqpoint{1.228726in}{2.629126in}}%
\pgfpathlineto{\pgfqpoint{1.267413in}{2.645758in}}%
\pgfpathlineto{\pgfqpoint{1.310846in}{2.661945in}}%
\pgfpathlineto{\pgfqpoint{1.356920in}{2.676740in}}%
\pgfpathlineto{\pgfqpoint{1.407680in}{2.690702in}}%
\pgfpathlineto{\pgfqpoint{1.463094in}{2.703640in}}%
\pgfpathlineto{\pgfqpoint{1.525273in}{2.715813in}}%
\pgfpathlineto{\pgfqpoint{1.594199in}{2.726937in}}%
\pgfpathlineto{\pgfqpoint{1.669843in}{2.736808in}}%
\pgfpathlineto{\pgfqpoint{1.752172in}{2.745271in}}%
\pgfpathlineto{\pgfqpoint{1.843325in}{2.752344in}}%
\pgfpathlineto{\pgfqpoint{1.941103in}{2.757656in}}%
\pgfpathlineto{\pgfqpoint{2.043301in}{2.760987in}}%
\pgfpathlineto{\pgfqpoint{2.147710in}{2.762199in}}%
\pgfpathlineto{\pgfqpoint{2.249945in}{2.761215in}}%
\pgfpathlineto{\pgfqpoint{2.345620in}{2.758145in}}%
\pgfpathlineto{\pgfqpoint{2.432525in}{2.753210in}}%
\pgfpathlineto{\pgfqpoint{2.508451in}{2.746766in}}%
\pgfpathlineto{\pgfqpoint{2.573368in}{2.739156in}}%
\pgfpathlineto{\pgfqpoint{2.629410in}{2.730451in}}%
\pgfpathlineto{\pgfqpoint{2.676543in}{2.720985in}}%
\pgfpathlineto{\pgfqpoint{2.716874in}{2.710666in}}%
\pgfpathlineto{\pgfqpoint{2.750366in}{2.699848in}}%
\pgfpathlineto{\pgfqpoint{2.779059in}{2.688192in}}%
\pgfpathlineto{\pgfqpoint{2.802882in}{2.676004in}}%
\pgfpathlineto{\pgfqpoint{2.821842in}{2.663819in}}%
\pgfpathlineto{\pgfqpoint{2.837815in}{2.650886in}}%
\pgfpathlineto{\pgfqpoint{2.850736in}{2.637564in}}%
\pgfpathlineto{\pgfqpoint{2.860694in}{2.624398in}}%
\pgfpathlineto{\pgfqpoint{2.869084in}{2.609873in}}%
\pgfpathlineto{\pgfqpoint{2.875698in}{2.594192in}}%
\pgfpathlineto{\pgfqpoint{2.881035in}{2.575255in}}%
\pgfpathlineto{\pgfqpoint{2.884200in}{2.555685in}}%
\pgfpathlineto{\pgfqpoint{2.885619in}{2.533351in}}%
\pgfpathlineto{\pgfqpoint{2.885038in}{2.505987in}}%
\pgfpathlineto{\pgfqpoint{2.882112in}{2.473807in}}%
\pgfpathlineto{\pgfqpoint{2.875657in}{2.429620in}}%
\pgfpathlineto{\pgfqpoint{2.863489in}{2.363873in}}%
\pgfpathlineto{\pgfqpoint{2.821102in}{2.142619in}}%
\pgfpathlineto{\pgfqpoint{2.804859in}{2.042271in}}%
\pgfpathlineto{\pgfqpoint{2.790421in}{1.939040in}}%
\pgfpathlineto{\pgfqpoint{2.777207in}{1.828054in}}%
\pgfpathlineto{\pgfqpoint{2.765338in}{1.709349in}}%
\pgfpathlineto{\pgfqpoint{2.754471in}{1.578010in}}%
\pgfpathlineto{\pgfqpoint{2.744640in}{1.431580in}}%
\pgfpathlineto{\pgfqpoint{2.735914in}{1.267598in}}%
\pgfpathlineto{\pgfqpoint{2.728277in}{1.081114in}}%
\pgfpathlineto{\pgfqpoint{2.721437in}{0.857223in}}%
\pgfpathlineto{\pgfqpoint{2.711961in}{0.541290in}}%
\pgfpathlineto{\pgfqpoint{2.708250in}{0.491694in}}%
\pgfpathlineto{\pgfqpoint{2.703951in}{0.462246in}}%
\pgfpathlineto{\pgfqpoint{2.699504in}{0.445599in}}%
\pgfpathlineto{\pgfqpoint{2.694517in}{0.434563in}}%
\pgfpathlineto{\pgfqpoint{2.688942in}{0.426947in}}%
\pgfpathlineto{\pgfqpoint{2.681980in}{0.421009in}}%
\pgfpathlineto{\pgfqpoint{2.672064in}{0.415948in}}%
\pgfpathlineto{\pgfqpoint{2.659429in}{0.412247in}}%
\pgfpathlineto{\pgfqpoint{2.640044in}{0.409163in}}%
\pgfpathlineto{\pgfqpoint{2.607490in}{0.406692in}}%
\pgfpathlineto{\pgfqpoint{2.548779in}{0.404894in}}%
\pgfpathlineto{\pgfqpoint{2.422615in}{0.403701in}}%
\pgfpathlineto{\pgfqpoint{2.026705in}{0.403016in}}%
\pgfpathlineto{\pgfqpoint{0.623617in}{0.403253in}}%
\pgfpathlineto{\pgfqpoint{0.477880in}{0.404742in}}%
\pgfpathlineto{\pgfqpoint{0.458368in}{0.406382in}}%
\pgfpathlineto{\pgfqpoint{0.452304in}{0.408937in}}%
\pgfpathlineto{\pgfqpoint{0.450213in}{0.413215in}}%
\pgfpathlineto{\pgfqpoint{0.449165in}{0.423080in}}%
\pgfpathlineto{\pgfqpoint{0.448735in}{0.465392in}}%
\pgfpathlineto{\pgfqpoint{0.448637in}{0.983146in}}%
\pgfpathlineto{\pgfqpoint{0.448652in}{2.889876in}}%
\pgfpathlineto{\pgfqpoint{0.448652in}{2.889876in}}%
\pgfusepath{stroke}%
\end{pgfscope}%
\begin{pgfscope}%
\pgfpathrectangle{\pgfqpoint{0.448634in}{0.402556in}}{\pgfqpoint{4.350661in}{2.489204in}} %
\pgfusepath{clip}%
\pgfsetrectcap%
\pgfsetroundjoin%
\pgfsetlinewidth{1.003750pt}%
\definecolor{currentstroke}{rgb}{1.000000,0.498039,0.054902}%
\pgfsetstrokecolor{currentstroke}%
\pgfsetdash{}{0pt}%
\pgfpathmoveto{\pgfqpoint{0.448634in}{2.896245in}}%
\pgfpathlineto{\pgfqpoint{0.448593in}{0.407043in}}%
\pgfpathlineto{\pgfqpoint{0.448593in}{0.407043in}}%
\pgfusepath{stroke}%
\end{pgfscope}%
\begin{pgfscope}%
\pgfpathrectangle{\pgfqpoint{0.448634in}{0.402556in}}{\pgfqpoint{4.350661in}{2.489204in}} %
\pgfusepath{clip}%
\pgfsetrectcap%
\pgfsetroundjoin%
\pgfsetlinewidth{1.003750pt}%
\definecolor{currentstroke}{rgb}{1.000000,0.498039,0.054902}%
\pgfsetstrokecolor{currentstroke}%
\pgfsetdash{}{0pt}%
\pgfpathmoveto{\pgfqpoint{0.576853in}{1.760817in}}%
\pgfpathlineto{\pgfqpoint{0.569394in}{1.840010in}}%
\pgfpathlineto{\pgfqpoint{0.563209in}{1.929338in}}%
\pgfpathlineto{\pgfqpoint{0.558592in}{2.028764in}}%
\pgfpathlineto{\pgfqpoint{0.555985in}{2.133265in}}%
\pgfpathlineto{\pgfqpoint{0.555566in}{2.237808in}}%
\pgfpathlineto{\pgfqpoint{0.557371in}{2.337352in}}%
\pgfpathlineto{\pgfqpoint{0.561096in}{2.424366in}}%
\pgfpathlineto{\pgfqpoint{0.566403in}{2.498791in}}%
\pgfpathlineto{\pgfqpoint{0.572909in}{2.560570in}}%
\pgfpathlineto{\pgfqpoint{0.580458in}{2.612119in}}%
\pgfpathlineto{\pgfqpoint{0.589086in}{2.655816in}}%
\pgfpathlineto{\pgfqpoint{0.598406in}{2.691590in}}%
\pgfpathlineto{\pgfqpoint{0.608613in}{2.721757in}}%
\pgfpathlineto{\pgfqpoint{0.619241in}{2.746278in}}%
\pgfpathlineto{\pgfqpoint{0.630817in}{2.767339in}}%
\pgfpathlineto{\pgfqpoint{0.642975in}{2.784884in}}%
\pgfpathlineto{\pgfqpoint{0.656813in}{2.800712in}}%
\pgfpathlineto{\pgfqpoint{0.672197in}{2.814549in}}%
\pgfpathlineto{\pgfqpoint{0.688853in}{2.826301in}}%
\pgfpathlineto{\pgfqpoint{0.706461in}{2.836076in}}%
\pgfpathlineto{\pgfqpoint{0.726804in}{2.844875in}}%
\pgfpathlineto{\pgfqpoint{0.751866in}{2.853203in}}%
\pgfpathlineto{\pgfqpoint{0.781631in}{2.860547in}}%
\pgfpathlineto{\pgfqpoint{0.818168in}{2.867054in}}%
\pgfpathlineto{\pgfqpoint{0.863581in}{2.872685in}}%
\pgfpathlineto{\pgfqpoint{0.922161in}{2.877518in}}%
\pgfpathlineto{\pgfqpoint{1.000391in}{2.881567in}}%
\pgfpathlineto{\pgfqpoint{1.111294in}{2.884881in}}%
\pgfpathlineto{\pgfqpoint{1.274428in}{2.887367in}}%
\pgfpathlineto{\pgfqpoint{1.552865in}{2.889263in}}%
\pgfpathlineto{\pgfqpoint{2.107573in}{2.890457in}}%
\pgfpathlineto{\pgfqpoint{3.343161in}{2.890573in}}%
\pgfpathlineto{\pgfqpoint{4.043615in}{2.888941in}}%
\pgfpathlineto{\pgfqpoint{4.289417in}{2.886404in}}%
\pgfpathlineto{\pgfqpoint{4.413375in}{2.883093in}}%
\pgfpathlineto{\pgfqpoint{4.489424in}{2.878997in}}%
\pgfpathlineto{\pgfqpoint{4.541451in}{2.874081in}}%
\pgfpathlineto{\pgfqpoint{4.578100in}{2.868470in}}%
\pgfpathlineto{\pgfqpoint{4.605818in}{2.862092in}}%
\pgfpathlineto{\pgfqpoint{4.626725in}{2.855245in}}%
\pgfpathlineto{\pgfqpoint{4.644925in}{2.847018in}}%
\pgfpathlineto{\pgfqpoint{4.660241in}{2.837590in}}%
\pgfpathlineto{\pgfqpoint{4.672623in}{2.827468in}}%
\pgfpathlineto{\pgfqpoint{4.683751in}{2.815592in}}%
\pgfpathlineto{\pgfqpoint{4.693406in}{2.802135in}}%
\pgfpathlineto{\pgfqpoint{4.702740in}{2.785343in}}%
\pgfpathlineto{\pgfqpoint{4.711277in}{2.765194in}}%
\pgfpathlineto{\pgfqpoint{4.719482in}{2.739484in}}%
\pgfpathlineto{\pgfqpoint{4.726293in}{2.710657in}}%
\pgfpathlineto{\pgfqpoint{4.733260in}{2.671643in}}%
\pgfpathlineto{\pgfqpoint{4.739604in}{2.622396in}}%
\pgfpathlineto{\pgfqpoint{4.745236in}{2.560504in}}%
\pgfpathlineto{\pgfqpoint{4.750164in}{2.481052in}}%
\pgfpathlineto{\pgfqpoint{4.754367in}{2.376618in}}%
\pgfpathlineto{\pgfqpoint{4.757443in}{2.242249in}}%
\pgfpathlineto{\pgfqpoint{4.758977in}{2.075483in}}%
\pgfpathlineto{\pgfqpoint{4.758447in}{1.888795in}}%
\pgfpathlineto{\pgfqpoint{4.755756in}{1.707111in}}%
\pgfpathlineto{\pgfqpoint{4.750925in}{1.532957in}}%
\pgfpathlineto{\pgfqpoint{4.744785in}{1.398726in}}%
\pgfpathlineto{\pgfqpoint{4.737575in}{1.289515in}}%
\pgfpathlineto{\pgfqpoint{4.728714in}{1.190470in}}%
\pgfpathlineto{\pgfqpoint{4.719652in}{1.116521in}}%
\pgfpathlineto{\pgfqpoint{4.710036in}{1.055276in}}%
\pgfpathlineto{\pgfqpoint{4.699503in}{1.001861in}}%
\pgfpathlineto{\pgfqpoint{4.689040in}{0.958690in}}%
\pgfpathlineto{\pgfqpoint{4.677220in}{0.918600in}}%
\pgfpathlineto{\pgfqpoint{4.664034in}{0.881749in}}%
\pgfpathlineto{\pgfqpoint{4.650584in}{0.850491in}}%
\pgfpathlineto{\pgfqpoint{4.636303in}{0.822570in}}%
\pgfpathlineto{\pgfqpoint{4.620207in}{0.795974in}}%
\pgfpathlineto{\pgfqpoint{4.603640in}{0.772901in}}%
\pgfpathlineto{\pgfqpoint{4.585488in}{0.751446in}}%
\pgfpathlineto{\pgfqpoint{4.565874in}{0.731749in}}%
\pgfpathlineto{\pgfqpoint{4.544964in}{0.713879in}}%
\pgfpathlineto{\pgfqpoint{4.522958in}{0.697824in}}%
\pgfpathlineto{\pgfqpoint{4.496157in}{0.681290in}}%
\pgfpathlineto{\pgfqpoint{4.470397in}{0.667953in}}%
\pgfpathlineto{\pgfqpoint{4.439961in}{0.654509in}}%
\pgfpathlineto{\pgfqpoint{4.406841in}{0.642281in}}%
\pgfpathlineto{\pgfqpoint{4.369009in}{0.630748in}}%
\pgfpathlineto{\pgfqpoint{4.326489in}{0.620226in}}%
\pgfpathlineto{\pgfqpoint{4.279327in}{0.610949in}}%
\pgfpathlineto{\pgfqpoint{4.227576in}{0.603085in}}%
\pgfpathlineto{\pgfqpoint{4.173450in}{0.597063in}}%
\pgfpathlineto{\pgfqpoint{4.110511in}{0.592203in}}%
\pgfpathlineto{\pgfqpoint{4.047471in}{0.589537in}}%
\pgfpathlineto{\pgfqpoint{3.977867in}{0.588624in}}%
\pgfpathlineto{\pgfqpoint{3.906093in}{0.589934in}}%
\pgfpathlineto{\pgfqpoint{3.834377in}{0.593496in}}%
\pgfpathlineto{\pgfqpoint{3.767120in}{0.599067in}}%
\pgfpathlineto{\pgfqpoint{3.704364in}{0.606392in}}%
\pgfpathlineto{\pgfqpoint{3.678516in}{0.610510in}}%
\pgfpathlineto{\pgfqpoint{3.620438in}{0.620500in}}%
\pgfpathlineto{\pgfqpoint{3.586319in}{0.628207in}}%
\pgfpathlineto{\pgfqpoint{3.495240in}{0.652428in}}%
\pgfpathlineto{\pgfqpoint{3.451528in}{0.667583in}}%
\pgfpathlineto{\pgfqpoint{3.408538in}{0.685220in}}%
\pgfpathlineto{\pgfqpoint{3.374594in}{0.702001in}}%
\pgfpathlineto{\pgfqpoint{3.345407in}{0.718682in}}%
\pgfpathlineto{\pgfqpoint{3.315236in}{0.738520in}}%
\pgfpathlineto{\pgfqpoint{3.288127in}{0.759290in}}%
\pgfpathlineto{\pgfqpoint{3.264004in}{0.780551in}}%
\pgfpathlineto{\pgfqpoint{3.241208in}{0.803648in}}%
\pgfpathlineto{\pgfqpoint{3.219894in}{0.828530in}}%
\pgfpathlineto{\pgfqpoint{3.200189in}{0.855091in}}%
\pgfpathlineto{\pgfqpoint{3.182177in}{0.883182in}}%
\pgfpathlineto{\pgfqpoint{3.165906in}{0.912633in}}%
\pgfpathlineto{\pgfqpoint{3.150351in}{0.945448in}}%
\pgfpathlineto{\pgfqpoint{3.136682in}{0.979345in}}%
\pgfpathlineto{\pgfqpoint{3.124073in}{1.016460in}}%
\pgfpathlineto{\pgfqpoint{3.112834in}{1.056769in}}%
\pgfpathlineto{\pgfqpoint{3.103046in}{1.100146in}}%
\pgfpathlineto{\pgfqpoint{3.095343in}{1.144071in}}%
\pgfpathlineto{\pgfqpoint{3.089208in}{1.190837in}}%
\pgfpathlineto{\pgfqpoint{3.084595in}{1.242838in}}%
\pgfpathlineto{\pgfqpoint{3.082137in}{1.295031in}}%
\pgfpathlineto{\pgfqpoint{3.081687in}{1.349787in}}%
\pgfpathlineto{\pgfqpoint{3.083451in}{1.406998in}}%
\pgfpathlineto{\pgfqpoint{3.087181in}{1.461589in}}%
\pgfpathlineto{\pgfqpoint{3.093485in}{1.520888in}}%
\pgfpathlineto{\pgfqpoint{3.101823in}{1.577334in}}%
\pgfpathlineto{\pgfqpoint{3.111930in}{1.630856in}}%
\pgfpathlineto{\pgfqpoint{3.124690in}{1.686208in}}%
\pgfpathlineto{\pgfqpoint{3.139178in}{1.738395in}}%
\pgfpathlineto{\pgfqpoint{3.155145in}{1.787366in}}%
\pgfpathlineto{\pgfqpoint{3.172353in}{1.833085in}}%
\pgfpathlineto{\pgfqpoint{3.191618in}{1.877716in}}%
\pgfpathlineto{\pgfqpoint{3.214026in}{1.923261in}}%
\pgfpathlineto{\pgfqpoint{3.236214in}{1.963157in}}%
\pgfpathlineto{\pgfqpoint{3.260178in}{2.001684in}}%
\pgfpathlineto{\pgfqpoint{3.285814in}{2.038776in}}%
\pgfpathlineto{\pgfqpoint{3.314415in}{2.076285in}}%
\pgfpathlineto{\pgfqpoint{3.348944in}{2.117711in}}%
\pgfpathlineto{\pgfqpoint{3.417133in}{2.198022in}}%
\pgfpathlineto{\pgfqpoint{3.426053in}{2.212128in}}%
\pgfpathlineto{\pgfqpoint{3.430798in}{2.223297in}}%
\pgfpathlineto{\pgfqpoint{3.432034in}{2.230603in}}%
\pgfpathlineto{\pgfqpoint{3.430773in}{2.237856in}}%
\pgfpathlineto{\pgfqpoint{3.426621in}{2.243526in}}%
\pgfpathlineto{\pgfqpoint{3.420908in}{2.247084in}}%
\pgfpathlineto{\pgfqpoint{3.412501in}{2.249583in}}%
\pgfpathlineto{\pgfqpoint{3.399499in}{2.250689in}}%
\pgfpathlineto{\pgfqpoint{3.384305in}{2.249671in}}%
\pgfpathlineto{\pgfqpoint{3.364985in}{2.246098in}}%
\pgfpathlineto{\pgfqpoint{3.341804in}{2.239342in}}%
\pgfpathlineto{\pgfqpoint{3.317109in}{2.229682in}}%
\pgfpathlineto{\pgfqpoint{3.291104in}{2.216986in}}%
\pgfpathlineto{\pgfqpoint{3.265928in}{2.202261in}}%
\pgfpathlineto{\pgfqpoint{3.239805in}{2.184361in}}%
\pgfpathlineto{\pgfqpoint{3.214775in}{2.164519in}}%
\pgfpathlineto{\pgfqpoint{3.190900in}{2.142893in}}%
\pgfpathlineto{\pgfqpoint{3.166657in}{2.117912in}}%
\pgfpathlineto{\pgfqpoint{3.143835in}{2.091233in}}%
\pgfpathlineto{\pgfqpoint{3.121079in}{2.061107in}}%
\pgfpathlineto{\pgfqpoint{3.099952in}{2.029463in}}%
\pgfpathlineto{\pgfqpoint{3.079251in}{1.994406in}}%
\pgfpathlineto{\pgfqpoint{3.059218in}{1.955915in}}%
\pgfpathlineto{\pgfqpoint{3.040058in}{1.914015in}}%
\pgfpathlineto{\pgfqpoint{3.022809in}{1.871041in}}%
\pgfpathlineto{\pgfqpoint{3.005790in}{1.822536in}}%
\pgfpathlineto{\pgfqpoint{2.990067in}{1.770819in}}%
\pgfpathlineto{\pgfqpoint{2.975708in}{1.715979in}}%
\pgfpathlineto{\pgfqpoint{2.962284in}{1.655680in}}%
\pgfpathlineto{\pgfqpoint{2.950496in}{1.592386in}}%
\pgfpathlineto{\pgfqpoint{2.940383in}{1.526185in}}%
\pgfpathlineto{\pgfqpoint{2.931745in}{1.454681in}}%
\pgfpathlineto{\pgfqpoint{2.925082in}{1.380399in}}%
\pgfpathlineto{\pgfqpoint{2.920647in}{1.305899in}}%
\pgfpathlineto{\pgfqpoint{2.918444in}{1.231270in}}%
\pgfpathlineto{\pgfqpoint{2.918545in}{1.159087in}}%
\pgfpathlineto{\pgfqpoint{2.920787in}{1.091931in}}%
\pgfpathlineto{\pgfqpoint{2.925177in}{1.027412in}}%
\pgfpathlineto{\pgfqpoint{2.931192in}{0.970580in}}%
\pgfpathlineto{\pgfqpoint{2.938760in}{0.919034in}}%
\pgfpathlineto{\pgfqpoint{2.947651in}{0.872852in}}%
\pgfpathlineto{\pgfqpoint{2.958213in}{0.829714in}}%
\pgfpathlineto{\pgfqpoint{2.969670in}{0.792114in}}%
\pgfpathlineto{\pgfqpoint{2.982463in}{0.757773in}}%
\pgfpathlineto{\pgfqpoint{2.996425in}{0.726812in}}%
\pgfpathlineto{\pgfqpoint{3.011299in}{0.699300in}}%
\pgfpathlineto{\pgfqpoint{3.026739in}{0.675225in}}%
\pgfpathlineto{\pgfqpoint{3.043828in}{0.652656in}}%
\pgfpathlineto{\pgfqpoint{3.062495in}{0.631788in}}%
\pgfpathlineto{\pgfqpoint{3.082602in}{0.612753in}}%
\pgfpathlineto{\pgfqpoint{3.103961in}{0.595592in}}%
\pgfpathlineto{\pgfqpoint{3.128268in}{0.579069in}}%
\pgfpathlineto{\pgfqpoint{3.153537in}{0.564554in}}%
\pgfpathlineto{\pgfqpoint{3.181571in}{0.550952in}}%
\pgfpathlineto{\pgfqpoint{3.214371in}{0.537647in}}%
\pgfpathlineto{\pgfqpoint{3.249846in}{0.525712in}}%
\pgfpathlineto{\pgfqpoint{3.290011in}{0.514571in}}%
\pgfpathlineto{\pgfqpoint{3.334820in}{0.504423in}}%
\pgfpathlineto{\pgfqpoint{3.386372in}{0.494999in}}%
\pgfpathlineto{\pgfqpoint{3.446798in}{0.486257in}}%
\pgfpathlineto{\pgfqpoint{3.518243in}{0.478282in}}%
\pgfpathlineto{\pgfqpoint{3.600685in}{0.471409in}}%
\pgfpathlineto{\pgfqpoint{3.696268in}{0.465713in}}%
\pgfpathlineto{\pgfqpoint{3.807144in}{0.461369in}}%
\pgfpathlineto{\pgfqpoint{3.933291in}{0.458719in}}%
\pgfpathlineto{\pgfqpoint{4.063808in}{0.458211in}}%
\pgfpathlineto{\pgfqpoint{4.187792in}{0.459914in}}%
\pgfpathlineto{\pgfqpoint{4.294335in}{0.463521in}}%
\pgfpathlineto{\pgfqpoint{4.381234in}{0.468574in}}%
\pgfpathlineto{\pgfqpoint{4.450636in}{0.474701in}}%
\pgfpathlineto{\pgfqpoint{4.506850in}{0.481799in}}%
\pgfpathlineto{\pgfqpoint{4.552009in}{0.489658in}}%
\pgfpathlineto{\pgfqpoint{4.588239in}{0.498115in}}%
\pgfpathlineto{\pgfqpoint{4.617656in}{0.507110in}}%
\pgfpathlineto{\pgfqpoint{4.642328in}{0.516843in}}%
\pgfpathlineto{\pgfqpoint{4.664194in}{0.527940in}}%
\pgfpathlineto{\pgfqpoint{4.681238in}{0.538945in}}%
\pgfpathlineto{\pgfqpoint{4.697164in}{0.551953in}}%
\pgfpathlineto{\pgfqpoint{4.710076in}{0.565289in}}%
\pgfpathlineto{\pgfqpoint{4.721578in}{0.580218in}}%
\pgfpathlineto{\pgfqpoint{4.731557in}{0.596521in}}%
\pgfpathlineto{\pgfqpoint{4.741000in}{0.616134in}}%
\pgfpathlineto{\pgfqpoint{4.749521in}{0.639027in}}%
\pgfpathlineto{\pgfqpoint{4.757522in}{0.667450in}}%
\pgfpathlineto{\pgfqpoint{4.764572in}{0.701345in}}%
\pgfpathlineto{\pgfqpoint{4.770840in}{0.743043in}}%
\pgfpathlineto{\pgfqpoint{4.776327in}{0.794934in}}%
\pgfpathlineto{\pgfqpoint{4.781278in}{0.864398in}}%
\pgfpathlineto{\pgfqpoint{4.785468in}{0.956371in}}%
\pgfpathlineto{\pgfqpoint{4.789000in}{1.085745in}}%
\pgfpathlineto{\pgfqpoint{4.791852in}{1.277385in}}%
\pgfpathlineto{\pgfqpoint{4.793959in}{1.581057in}}%
\pgfpathlineto{\pgfqpoint{4.794962in}{2.071429in}}%
\pgfpathlineto{\pgfqpoint{4.793967in}{2.559311in}}%
\pgfpathlineto{\pgfqpoint{4.791733in}{2.745981in}}%
\pgfpathlineto{\pgfqpoint{4.788955in}{2.818091in}}%
\pgfpathlineto{\pgfqpoint{4.785731in}{2.850227in}}%
\pgfpathlineto{\pgfqpoint{4.781879in}{2.867057in}}%
\pgfpathlineto{\pgfqpoint{4.777744in}{2.875780in}}%
\pgfpathlineto{\pgfqpoint{4.773097in}{2.880982in}}%
\pgfpathlineto{\pgfqpoint{4.767363in}{2.884504in}}%
\pgfpathlineto{\pgfqpoint{4.756853in}{2.887622in}}%
\pgfpathlineto{\pgfqpoint{4.739548in}{2.889639in}}%
\pgfpathlineto{\pgfqpoint{4.704762in}{2.890882in}}%
\pgfpathlineto{\pgfqpoint{4.602524in}{2.891538in}}%
\pgfpathlineto{\pgfqpoint{3.952100in}{2.891742in}}%
\pgfpathlineto{\pgfqpoint{0.617321in}{2.890753in}}%
\pgfpathlineto{\pgfqpoint{0.549910in}{2.888858in}}%
\pgfpathlineto{\pgfqpoint{0.521735in}{2.886179in}}%
\pgfpathlineto{\pgfqpoint{0.504666in}{2.882389in}}%
\pgfpathlineto{\pgfqpoint{0.494501in}{2.878011in}}%
\pgfpathlineto{\pgfqpoint{0.487180in}{2.872667in}}%
\pgfpathlineto{\pgfqpoint{0.481152in}{2.865519in}}%
\pgfpathlineto{\pgfqpoint{0.475664in}{2.854804in}}%
\pgfpathlineto{\pgfqpoint{0.471318in}{2.840737in}}%
\pgfpathlineto{\pgfqpoint{0.467301in}{2.818823in}}%
\pgfpathlineto{\pgfqpoint{0.463927in}{2.786700in}}%
\pgfpathlineto{\pgfqpoint{0.460918in}{2.734544in}}%
\pgfpathlineto{\pgfqpoint{0.458363in}{2.647473in}}%
\pgfpathlineto{\pgfqpoint{0.456575in}{2.523031in}}%
\pgfpathlineto{\pgfqpoint{0.456575in}{2.523031in}}%
\pgfusepath{stroke}%
\end{pgfscope}%
\begin{pgfscope}%
\pgfpathrectangle{\pgfqpoint{0.448634in}{0.402556in}}{\pgfqpoint{4.350661in}{2.489204in}} %
\pgfusepath{clip}%
\pgfsetrectcap%
\pgfsetroundjoin%
\pgfsetlinewidth{1.003750pt}%
\definecolor{currentstroke}{rgb}{1.000000,0.498039,0.054902}%
\pgfsetstrokecolor{currentstroke}%
\pgfsetdash{}{0pt}%
\pgfpathmoveto{\pgfqpoint{0.456424in}{1.370137in}}%
\pgfpathlineto{\pgfqpoint{0.459610in}{1.118755in}}%
\pgfpathlineto{\pgfqpoint{0.463695in}{0.962007in}}%
\pgfpathlineto{\pgfqpoint{0.468519in}{0.857610in}}%
\pgfpathlineto{\pgfqpoint{0.474082in}{0.783210in}}%
\pgfpathlineto{\pgfqpoint{0.480226in}{0.728906in}}%
\pgfpathlineto{\pgfqpoint{0.486970in}{0.687306in}}%
\pgfpathlineto{\pgfqpoint{0.494537in}{0.653558in}}%
\pgfpathlineto{\pgfqpoint{0.503107in}{0.625355in}}%
\pgfpathlineto{\pgfqpoint{0.512193in}{0.602749in}}%
\pgfpathlineto{\pgfqpoint{0.522200in}{0.583508in}}%
\pgfpathlineto{\pgfqpoint{0.534108in}{0.565743in}}%
\pgfpathlineto{\pgfqpoint{0.546263in}{0.551507in}}%
\pgfpathlineto{\pgfqpoint{0.559728in}{0.538907in}}%
\pgfpathlineto{\pgfqpoint{0.576129in}{0.526693in}}%
\pgfpathlineto{\pgfqpoint{0.595483in}{0.515351in}}%
\pgfpathlineto{\pgfqpoint{0.617681in}{0.505147in}}%
\pgfpathlineto{\pgfqpoint{0.642568in}{0.496153in}}%
\pgfpathlineto{\pgfqpoint{0.672126in}{0.487778in}}%
\pgfpathlineto{\pgfqpoint{0.708443in}{0.479824in}}%
\pgfpathlineto{\pgfqpoint{0.753649in}{0.472325in}}%
\pgfpathlineto{\pgfqpoint{0.807717in}{0.465660in}}%
\pgfpathlineto{\pgfqpoint{0.877116in}{0.459475in}}%
\pgfpathlineto{\pgfqpoint{0.961828in}{0.454230in}}%
\pgfpathlineto{\pgfqpoint{1.068351in}{0.449916in}}%
\pgfpathlineto{\pgfqpoint{1.201018in}{0.446839in}}%
\pgfpathlineto{\pgfqpoint{1.357637in}{0.445481in}}%
\pgfpathlineto{\pgfqpoint{1.525135in}{0.446232in}}%
\pgfpathlineto{\pgfqpoint{1.686088in}{0.449142in}}%
\pgfpathlineto{\pgfqpoint{1.823074in}{0.453747in}}%
\pgfpathlineto{\pgfqpoint{1.938245in}{0.459764in}}%
\pgfpathlineto{\pgfqpoint{2.031582in}{0.466759in}}%
\pgfpathlineto{\pgfqpoint{2.109580in}{0.474745in}}%
\pgfpathlineto{\pgfqpoint{2.174384in}{0.483535in}}%
\pgfpathlineto{\pgfqpoint{2.228139in}{0.492940in}}%
\pgfpathlineto{\pgfqpoint{2.275119in}{0.503356in}}%
\pgfpathlineto{\pgfqpoint{2.315282in}{0.514501in}}%
\pgfpathlineto{\pgfqpoint{2.350698in}{0.526659in}}%
\pgfpathlineto{\pgfqpoint{2.381320in}{0.539536in}}%
\pgfpathlineto{\pgfqpoint{2.407164in}{0.552659in}}%
\pgfpathlineto{\pgfqpoint{2.430226in}{0.566639in}}%
\pgfpathlineto{\pgfqpoint{2.452282in}{0.582602in}}%
\pgfpathlineto{\pgfqpoint{2.471391in}{0.599069in}}%
\pgfpathlineto{\pgfqpoint{2.489240in}{0.617293in}}%
\pgfpathlineto{\pgfqpoint{2.505678in}{0.637180in}}%
\pgfpathlineto{\pgfqpoint{2.520620in}{0.658557in}}%
\pgfpathlineto{\pgfqpoint{2.535213in}{0.683314in}}%
\pgfpathlineto{\pgfqpoint{2.549115in}{0.711484in}}%
\pgfpathlineto{\pgfqpoint{2.562091in}{0.743004in}}%
\pgfpathlineto{\pgfqpoint{2.574020in}{0.777751in}}%
\pgfpathlineto{\pgfqpoint{2.585502in}{0.817970in}}%
\pgfpathlineto{\pgfqpoint{2.596809in}{0.866038in}}%
\pgfpathlineto{\pgfqpoint{2.607562in}{0.921948in}}%
\pgfpathlineto{\pgfqpoint{2.617925in}{0.988098in}}%
\pgfpathlineto{\pgfqpoint{2.627958in}{1.066918in}}%
\pgfpathlineto{\pgfqpoint{2.637941in}{1.163320in}}%
\pgfpathlineto{\pgfqpoint{2.648424in}{1.287199in}}%
\pgfpathlineto{\pgfqpoint{2.660103in}{1.453438in}}%
\pgfpathlineto{\pgfqpoint{2.674773in}{1.696801in}}%
\pgfpathlineto{\pgfqpoint{2.687716in}{1.945279in}}%
\pgfpathlineto{\pgfqpoint{2.692670in}{2.079573in}}%
\pgfpathlineto{\pgfqpoint{2.693829in}{2.166682in}}%
\pgfpathlineto{\pgfqpoint{2.692565in}{2.233870in}}%
\pgfpathlineto{\pgfqpoint{2.689436in}{2.286015in}}%
\pgfpathlineto{\pgfqpoint{2.684859in}{2.327999in}}%
\pgfpathlineto{\pgfqpoint{2.678725in}{2.364664in}}%
\pgfpathlineto{\pgfqpoint{2.671356in}{2.395897in}}%
\pgfpathlineto{\pgfqpoint{2.662489in}{2.423981in}}%
\pgfpathlineto{\pgfqpoint{2.652361in}{2.448778in}}%
\pgfpathlineto{\pgfqpoint{2.641365in}{2.470245in}}%
\pgfpathlineto{\pgfqpoint{2.628643in}{2.490425in}}%
\pgfpathlineto{\pgfqpoint{2.614279in}{2.509106in}}%
\pgfpathlineto{\pgfqpoint{2.598443in}{2.526159in}}%
\pgfpathlineto{\pgfqpoint{2.579590in}{2.543005in}}%
\pgfpathlineto{\pgfqpoint{2.559532in}{2.557923in}}%
\pgfpathlineto{\pgfqpoint{2.536602in}{2.572183in}}%
\pgfpathlineto{\pgfqpoint{2.510850in}{2.585538in}}%
\pgfpathlineto{\pgfqpoint{2.482360in}{2.597837in}}%
\pgfpathlineto{\pgfqpoint{2.449134in}{2.609683in}}%
\pgfpathlineto{\pgfqpoint{2.411184in}{2.620696in}}%
\pgfpathlineto{\pgfqpoint{2.368552in}{2.630606in}}%
\pgfpathlineto{\pgfqpoint{2.321294in}{2.639221in}}%
\pgfpathlineto{\pgfqpoint{2.269467in}{2.646399in}}%
\pgfpathlineto{\pgfqpoint{2.210954in}{2.652193in}}%
\pgfpathlineto{\pgfqpoint{2.147967in}{2.656153in}}%
\pgfpathlineto{\pgfqpoint{2.080556in}{2.658135in}}%
\pgfpathlineto{\pgfqpoint{2.010948in}{2.657971in}}%
\pgfpathlineto{\pgfqpoint{1.939195in}{2.655572in}}%
\pgfpathlineto{\pgfqpoint{1.867527in}{2.650913in}}%
\pgfpathlineto{\pgfqpoint{1.798171in}{2.644140in}}%
\pgfpathlineto{\pgfqpoint{1.733341in}{2.635606in}}%
\pgfpathlineto{\pgfqpoint{1.673075in}{2.625521in}}%
\pgfpathlineto{\pgfqpoint{1.615274in}{2.613610in}}%
\pgfpathlineto{\pgfqpoint{1.562133in}{2.600402in}}%
\pgfpathlineto{\pgfqpoint{1.513681in}{2.586139in}}%
\pgfpathlineto{\pgfqpoint{1.467862in}{2.570344in}}%
\pgfpathlineto{\pgfqpoint{1.426794in}{2.553923in}}%
\pgfpathlineto{\pgfqpoint{1.388447in}{2.536289in}}%
\pgfpathlineto{\pgfqpoint{1.352878in}{2.517566in}}%
\pgfpathlineto{\pgfqpoint{1.320128in}{2.497922in}}%
\pgfpathlineto{\pgfqpoint{1.288379in}{2.476236in}}%
\pgfpathlineto{\pgfqpoint{1.259592in}{2.453861in}}%
\pgfpathlineto{\pgfqpoint{1.232050in}{2.429520in}}%
\pgfpathlineto{\pgfqpoint{1.207527in}{2.404898in}}%
\pgfpathlineto{\pgfqpoint{1.184409in}{2.378557in}}%
\pgfpathlineto{\pgfqpoint{1.162828in}{2.350561in}}%
\pgfpathlineto{\pgfqpoint{1.142891in}{2.321011in}}%
\pgfpathlineto{\pgfqpoint{1.124675in}{2.290041in}}%
\pgfpathlineto{\pgfqpoint{1.108225in}{2.257802in}}%
\pgfpathlineto{\pgfqpoint{1.092639in}{2.222199in}}%
\pgfpathlineto{\pgfqpoint{1.079059in}{2.185535in}}%
\pgfpathlineto{\pgfqpoint{1.067443in}{2.147998in}}%
\pgfpathlineto{\pgfqpoint{1.057187in}{2.107348in}}%
\pgfpathlineto{\pgfqpoint{1.049004in}{2.066086in}}%
\pgfpathlineto{\pgfqpoint{1.042513in}{2.021906in}}%
\pgfpathlineto{\pgfqpoint{1.038177in}{1.977382in}}%
\pgfpathlineto{\pgfqpoint{1.035866in}{1.930167in}}%
\pgfpathlineto{\pgfqpoint{1.035826in}{1.882878in}}%
\pgfpathlineto{\pgfqpoint{1.038031in}{1.835656in}}%
\pgfpathlineto{\pgfqpoint{1.042474in}{1.788641in}}%
\pgfpathlineto{\pgfqpoint{1.049176in}{1.741979in}}%
\pgfpathlineto{\pgfqpoint{1.057644in}{1.698239in}}%
\pgfpathlineto{\pgfqpoint{1.068221in}{1.655105in}}%
\pgfpathlineto{\pgfqpoint{1.080962in}{1.612745in}}%
\pgfpathlineto{\pgfqpoint{1.095031in}{1.573617in}}%
\pgfpathlineto{\pgfqpoint{1.111115in}{1.535520in}}%
\pgfpathlineto{\pgfqpoint{1.128118in}{1.500775in}}%
\pgfpathlineto{\pgfqpoint{1.146930in}{1.467274in}}%
\pgfpathlineto{\pgfqpoint{1.167531in}{1.435181in}}%
\pgfpathlineto{\pgfqpoint{1.189874in}{1.404652in}}%
\pgfpathlineto{\pgfqpoint{1.213884in}{1.375828in}}%
\pgfpathlineto{\pgfqpoint{1.237817in}{1.350457in}}%
\pgfpathlineto{\pgfqpoint{1.264748in}{1.325237in}}%
\pgfpathlineto{\pgfqpoint{1.292991in}{1.301972in}}%
\pgfpathlineto{\pgfqpoint{1.322398in}{1.280678in}}%
\pgfpathlineto{\pgfqpoint{1.352820in}{1.261340in}}%
\pgfpathlineto{\pgfqpoint{1.386095in}{1.242889in}}%
\pgfpathlineto{\pgfqpoint{1.420190in}{1.226516in}}%
\pgfpathlineto{\pgfqpoint{1.457024in}{1.211329in}}%
\pgfpathlineto{\pgfqpoint{1.496554in}{1.197536in}}%
\pgfpathlineto{\pgfqpoint{1.538719in}{1.185287in}}%
\pgfpathlineto{\pgfqpoint{1.583441in}{1.174641in}}%
\pgfpathlineto{\pgfqpoint{1.634929in}{1.164775in}}%
\pgfpathlineto{\pgfqpoint{1.706063in}{1.153745in}}%
\pgfpathlineto{\pgfqpoint{1.768492in}{1.143417in}}%
\pgfpathlineto{\pgfqpoint{1.796122in}{1.136567in}}%
\pgfpathlineto{\pgfqpoint{1.812683in}{1.130481in}}%
\pgfpathlineto{\pgfqpoint{1.824471in}{1.124102in}}%
\pgfpathlineto{\pgfqpoint{1.833209in}{1.116741in}}%
\pgfpathlineto{\pgfqpoint{1.838498in}{1.108890in}}%
\pgfpathlineto{\pgfqpoint{1.840588in}{1.101849in}}%
\pgfpathlineto{\pgfqpoint{1.840619in}{1.094412in}}%
\pgfpathlineto{\pgfqpoint{1.837931in}{1.084986in}}%
\pgfpathlineto{\pgfqpoint{1.833246in}{1.076615in}}%
\pgfpathlineto{\pgfqpoint{1.825819in}{1.067542in}}%
\pgfpathlineto{\pgfqpoint{1.813813in}{1.056850in}}%
\pgfpathlineto{\pgfqpoint{1.798819in}{1.046763in}}%
\pgfpathlineto{\pgfqpoint{1.781016in}{1.037462in}}%
\pgfpathlineto{\pgfqpoint{1.758447in}{1.028391in}}%
\pgfpathlineto{\pgfqpoint{1.733203in}{1.020815in}}%
\pgfpathlineto{\pgfqpoint{1.705410in}{1.014872in}}%
\pgfpathlineto{\pgfqpoint{1.675178in}{1.010714in}}%
\pgfpathlineto{\pgfqpoint{1.642610in}{1.008507in}}%
\pgfpathlineto{\pgfqpoint{1.607809in}{1.008432in}}%
\pgfpathlineto{\pgfqpoint{1.570886in}{1.010691in}}%
\pgfpathlineto{\pgfqpoint{1.534118in}{1.015181in}}%
\pgfpathlineto{\pgfqpoint{1.495454in}{1.022233in}}%
\pgfpathlineto{\pgfqpoint{1.457161in}{1.031563in}}%
\pgfpathlineto{\pgfqpoint{1.419337in}{1.043132in}}%
\pgfpathlineto{\pgfqpoint{1.382089in}{1.056929in}}%
\pgfpathlineto{\pgfqpoint{1.347544in}{1.072019in}}%
\pgfpathlineto{\pgfqpoint{1.313727in}{1.089133in}}%
\pgfpathlineto{\pgfqpoint{1.280762in}{1.108299in}}%
\pgfpathlineto{\pgfqpoint{1.248782in}{1.129536in}}%
\pgfpathlineto{\pgfqpoint{1.219708in}{1.151422in}}%
\pgfpathlineto{\pgfqpoint{1.191752in}{1.175138in}}%
\pgfpathlineto{\pgfqpoint{1.165031in}{1.200649in}}%
\pgfpathlineto{\pgfqpoint{1.139653in}{1.227898in}}%
\pgfpathlineto{\pgfqpoint{1.115714in}{1.256800in}}%
\pgfpathlineto{\pgfqpoint{1.093288in}{1.287251in}}%
\pgfpathlineto{\pgfqpoint{1.071178in}{1.321163in}}%
\pgfpathlineto{\pgfqpoint{1.050868in}{1.356520in}}%
\pgfpathlineto{\pgfqpoint{1.032365in}{1.393152in}}%
\pgfpathlineto{\pgfqpoint{1.014718in}{1.433142in}}%
\pgfpathlineto{\pgfqpoint{0.999024in}{1.474185in}}%
\pgfpathlineto{\pgfqpoint{0.984506in}{1.518461in}}%
\pgfpathlineto{\pgfqpoint{0.972010in}{1.563537in}}%
\pgfpathlineto{\pgfqpoint{0.960944in}{1.611678in}}%
\pgfpathlineto{\pgfqpoint{0.951530in}{1.662824in}}%
\pgfpathlineto{\pgfqpoint{0.944286in}{1.714431in}}%
\pgfpathlineto{\pgfqpoint{0.938950in}{1.768847in}}%
\pgfpathlineto{\pgfqpoint{0.935870in}{1.823491in}}%
\pgfpathlineto{\pgfqpoint{0.935034in}{1.878240in}}%
\pgfpathlineto{\pgfqpoint{0.936466in}{1.932973in}}%
\pgfpathlineto{\pgfqpoint{0.940005in}{1.985084in}}%
\pgfpathlineto{\pgfqpoint{0.945759in}{2.036935in}}%
\pgfpathlineto{\pgfqpoint{0.953410in}{2.085938in}}%
\pgfpathlineto{\pgfqpoint{0.962764in}{2.132000in}}%
\pgfpathlineto{\pgfqpoint{0.974287in}{2.177414in}}%
\pgfpathlineto{\pgfqpoint{0.987332in}{2.219653in}}%
\pgfpathlineto{\pgfqpoint{1.001667in}{2.258654in}}%
\pgfpathlineto{\pgfqpoint{1.018051in}{2.296583in}}%
\pgfpathlineto{\pgfqpoint{1.035401in}{2.331101in}}%
\pgfpathlineto{\pgfqpoint{1.054650in}{2.364275in}}%
\pgfpathlineto{\pgfqpoint{1.074406in}{2.393984in}}%
\pgfpathlineto{\pgfqpoint{1.095771in}{2.422197in}}%
\pgfpathlineto{\pgfqpoint{1.118662in}{2.448797in}}%
\pgfpathlineto{\pgfqpoint{1.142967in}{2.473701in}}%
\pgfpathlineto{\pgfqpoint{1.168550in}{2.496867in}}%
\pgfpathlineto{\pgfqpoint{1.197085in}{2.519662in}}%
\pgfpathlineto{\pgfqpoint{1.226727in}{2.540526in}}%
\pgfpathlineto{\pgfqpoint{1.259242in}{2.560673in}}%
\pgfpathlineto{\pgfqpoint{1.294612in}{2.579881in}}%
\pgfpathlineto{\pgfqpoint{1.332792in}{2.597982in}}%
\pgfpathlineto{\pgfqpoint{1.373719in}{2.614859in}}%
\pgfpathlineto{\pgfqpoint{1.417319in}{2.630445in}}%
\pgfpathlineto{\pgfqpoint{1.465632in}{2.645312in}}%
\pgfpathlineto{\pgfqpoint{1.518640in}{2.659204in}}%
\pgfpathlineto{\pgfqpoint{1.576309in}{2.671929in}}%
\pgfpathlineto{\pgfqpoint{1.638597in}{2.683344in}}%
\pgfpathlineto{\pgfqpoint{1.705462in}{2.693343in}}%
\pgfpathlineto{\pgfqpoint{1.779027in}{2.702064in}}%
\pgfpathlineto{\pgfqpoint{1.857097in}{2.709077in}}%
\pgfpathlineto{\pgfqpoint{1.939633in}{2.714280in}}%
\pgfpathlineto{\pgfqpoint{2.026598in}{2.717513in}}%
\pgfpathlineto{\pgfqpoint{2.113605in}{2.718523in}}%
\pgfpathlineto{\pgfqpoint{2.198435in}{2.717303in}}%
\pgfpathlineto{\pgfqpoint{2.278866in}{2.713929in}}%
\pgfpathlineto{\pgfqpoint{2.352678in}{2.708598in}}%
\pgfpathlineto{\pgfqpoint{2.417657in}{2.701709in}}%
\pgfpathlineto{\pgfqpoint{2.473770in}{2.693630in}}%
\pgfpathlineto{\pgfqpoint{2.523140in}{2.684368in}}%
\pgfpathlineto{\pgfqpoint{2.565726in}{2.674202in}}%
\pgfpathlineto{\pgfqpoint{2.601510in}{2.663544in}}%
\pgfpathlineto{\pgfqpoint{2.632577in}{2.652142in}}%
\pgfpathlineto{\pgfqpoint{2.658899in}{2.640331in}}%
\pgfpathlineto{\pgfqpoint{2.682438in}{2.627436in}}%
\pgfpathlineto{\pgfqpoint{2.703062in}{2.613571in}}%
\pgfpathlineto{\pgfqpoint{2.720674in}{2.598978in}}%
\pgfpathlineto{\pgfqpoint{2.735263in}{2.584053in}}%
\pgfpathlineto{\pgfqpoint{2.748320in}{2.567377in}}%
\pgfpathlineto{\pgfqpoint{2.759553in}{2.549046in}}%
\pgfpathlineto{\pgfqpoint{2.768788in}{2.529306in}}%
\pgfpathlineto{\pgfqpoint{2.776017in}{2.508498in}}%
\pgfpathlineto{\pgfqpoint{2.781884in}{2.484540in}}%
\pgfpathlineto{\pgfqpoint{2.786102in}{2.457597in}}%
\pgfpathlineto{\pgfqpoint{2.788720in}{2.425384in}}%
\pgfpathlineto{\pgfqpoint{2.789427in}{2.388061in}}%
\pgfpathlineto{\pgfqpoint{2.787962in}{2.340801in}}%
\pgfpathlineto{\pgfqpoint{2.783672in}{2.278768in}}%
\pgfpathlineto{\pgfqpoint{2.774289in}{2.179783in}}%
\pgfpathlineto{\pgfqpoint{2.743611in}{1.868119in}}%
\pgfpathlineto{\pgfqpoint{2.730112in}{1.702060in}}%
\pgfpathlineto{\pgfqpoint{2.717287in}{1.515949in}}%
\pgfpathlineto{\pgfqpoint{2.702602in}{1.267597in}}%
\pgfpathlineto{\pgfqpoint{2.684434in}{0.964630in}}%
\pgfpathlineto{\pgfqpoint{2.675374in}{0.850600in}}%
\pgfpathlineto{\pgfqpoint{2.667030in}{0.771523in}}%
\pgfpathlineto{\pgfqpoint{2.658752in}{0.712543in}}%
\pgfpathlineto{\pgfqpoint{2.650176in}{0.666284in}}%
\pgfpathlineto{\pgfqpoint{2.640820in}{0.627931in}}%
\pgfpathlineto{\pgfqpoint{2.631145in}{0.597534in}}%
\pgfpathlineto{\pgfqpoint{2.621004in}{0.572745in}}%
\pgfpathlineto{\pgfqpoint{2.609856in}{0.551383in}}%
\pgfpathlineto{\pgfqpoint{2.598042in}{0.533534in}}%
\pgfpathlineto{\pgfqpoint{2.584496in}{0.517378in}}%
\pgfpathlineto{\pgfqpoint{2.571109in}{0.504669in}}%
\pgfpathlineto{\pgfqpoint{2.554789in}{0.492313in}}%
\pgfpathlineto{\pgfqpoint{2.537457in}{0.481914in}}%
\pgfpathlineto{\pgfqpoint{2.517374in}{0.472367in}}%
\pgfpathlineto{\pgfqpoint{2.492542in}{0.463178in}}%
\pgfpathlineto{\pgfqpoint{2.462979in}{0.454833in}}%
\pgfpathlineto{\pgfqpoint{2.428766in}{0.447542in}}%
\pgfpathlineto{\pgfqpoint{2.385671in}{0.440735in}}%
\pgfpathlineto{\pgfqpoint{2.331557in}{0.434581in}}%
\pgfpathlineto{\pgfqpoint{2.262115in}{0.429077in}}%
\pgfpathlineto{\pgfqpoint{2.170851in}{0.424236in}}%
\pgfpathlineto{\pgfqpoint{2.049086in}{0.420134in}}%
\pgfpathlineto{\pgfqpoint{1.879436in}{0.416783in}}%
\pgfpathlineto{\pgfqpoint{1.640159in}{0.414418in}}%
\pgfpathlineto{\pgfqpoint{1.322562in}{0.413569in}}%
\pgfpathlineto{\pgfqpoint{1.020194in}{0.414850in}}%
\pgfpathlineto{\pgfqpoint{0.822256in}{0.417715in}}%
\pgfpathlineto{\pgfqpoint{0.704835in}{0.421430in}}%
\pgfpathlineto{\pgfqpoint{0.630976in}{0.425829in}}%
\pgfpathlineto{\pgfqpoint{0.583316in}{0.430734in}}%
\pgfpathlineto{\pgfqpoint{0.551033in}{0.436123in}}%
\pgfpathlineto{\pgfqpoint{0.527708in}{0.442189in}}%
\pgfpathlineto{\pgfqpoint{0.511250in}{0.448625in}}%
\pgfpathlineto{\pgfqpoint{0.499549in}{0.455216in}}%
\pgfpathlineto{\pgfqpoint{0.488916in}{0.463841in}}%
\pgfpathlineto{\pgfqpoint{0.481322in}{0.472730in}}%
\pgfpathlineto{\pgfqpoint{0.474078in}{0.485127in}}%
\pgfpathlineto{\pgfqpoint{0.468753in}{0.498748in}}%
\pgfpathlineto{\pgfqpoint{0.463870in}{0.517848in}}%
\pgfpathlineto{\pgfqpoint{0.459679in}{0.544796in}}%
\pgfpathlineto{\pgfqpoint{0.456386in}{0.581938in}}%
\pgfpathlineto{\pgfqpoint{0.453731in}{0.639106in}}%
\pgfpathlineto{\pgfqpoint{0.451681in}{0.736155in}}%
\pgfpathlineto{\pgfqpoint{0.450220in}{0.927815in}}%
\pgfpathlineto{\pgfqpoint{0.449345in}{1.403252in}}%
\pgfpathlineto{\pgfqpoint{0.449543in}{2.682703in}}%
\pgfpathlineto{\pgfqpoint{0.451011in}{2.856932in}}%
\pgfpathlineto{\pgfqpoint{0.452802in}{2.879219in}}%
\pgfpathlineto{\pgfqpoint{0.455188in}{2.886108in}}%
\pgfpathlineto{\pgfqpoint{0.458626in}{2.889028in}}%
\pgfpathlineto{\pgfqpoint{0.464996in}{2.890553in}}%
\pgfpathlineto{\pgfqpoint{0.482377in}{2.891423in}}%
\pgfpathlineto{\pgfqpoint{0.565038in}{2.891729in}}%
\pgfpathlineto{\pgfqpoint{2.733842in}{2.891760in}}%
\pgfpathlineto{\pgfqpoint{4.789510in}{2.890885in}}%
\pgfpathlineto{\pgfqpoint{4.793727in}{2.889730in}}%
\pgfpathlineto{\pgfqpoint{4.795481in}{2.888307in}}%
\pgfpathlineto{\pgfqpoint{4.797106in}{2.881145in}}%
\pgfpathlineto{\pgfqpoint{4.797997in}{2.858771in}}%
\pgfpathlineto{\pgfqpoint{4.798039in}{2.856283in}}%
\pgfpathlineto{\pgfqpoint{4.798039in}{2.856283in}}%
\pgfusepath{stroke}%
\end{pgfscope}%
\begin{pgfscope}%
\pgfpathrectangle{\pgfqpoint{0.448634in}{0.402556in}}{\pgfqpoint{4.350661in}{2.489204in}} %
\pgfusepath{clip}%
\pgfsetrectcap%
\pgfsetroundjoin%
\pgfsetlinewidth{1.003750pt}%
\definecolor{currentstroke}{rgb}{1.000000,0.498039,0.054902}%
\pgfsetstrokecolor{currentstroke}%
\pgfsetdash{}{0pt}%
\pgfpathmoveto{\pgfqpoint{3.428772in}{0.402610in}}%
\pgfpathlineto{\pgfqpoint{2.806632in}{0.403760in}}%
\pgfpathlineto{\pgfqpoint{2.769692in}{0.405578in}}%
\pgfpathlineto{\pgfqpoint{2.754632in}{0.408064in}}%
\pgfpathlineto{\pgfqpoint{2.746391in}{0.411198in}}%
\pgfpathlineto{\pgfqpoint{2.740943in}{0.415265in}}%
\pgfpathlineto{\pgfqpoint{2.736784in}{0.420984in}}%
\pgfpathlineto{\pgfqpoint{2.733281in}{0.430071in}}%
\pgfpathlineto{\pgfqpoint{2.730449in}{0.444636in}}%
\pgfpathlineto{\pgfqpoint{2.728238in}{0.469392in}}%
\pgfpathlineto{\pgfqpoint{2.726470in}{0.519131in}}%
\pgfpathlineto{\pgfqpoint{2.725711in}{0.613715in}}%
\pgfpathlineto{\pgfqpoint{2.726842in}{0.768038in}}%
\pgfpathlineto{\pgfqpoint{2.730556in}{0.962148in}}%
\pgfpathlineto{\pgfqpoint{2.736611in}{1.158670in}}%
\pgfpathlineto{\pgfqpoint{2.744092in}{1.327718in}}%
\pgfpathlineto{\pgfqpoint{2.753201in}{1.484189in}}%
\pgfpathlineto{\pgfqpoint{2.763257in}{1.620609in}}%
\pgfpathlineto{\pgfqpoint{2.776118in}{1.764216in}}%
\pgfpathlineto{\pgfqpoint{2.788914in}{1.877776in}}%
\pgfpathlineto{\pgfqpoint{2.805748in}{2.005740in}}%
\pgfpathlineto{\pgfqpoint{2.821176in}{2.101198in}}%
\pgfpathlineto{\pgfqpoint{2.838359in}{2.193718in}}%
\pgfpathlineto{\pgfqpoint{2.859135in}{2.292966in}}%
\pgfpathlineto{\pgfqpoint{2.887209in}{2.425960in}}%
\pgfpathlineto{\pgfqpoint{2.896991in}{2.479560in}}%
\pgfpathlineto{\pgfqpoint{2.901543in}{2.516523in}}%
\pgfpathlineto{\pgfqpoint{2.902849in}{2.543854in}}%
\pgfpathlineto{\pgfqpoint{2.901957in}{2.566223in}}%
\pgfpathlineto{\pgfqpoint{2.899151in}{2.585863in}}%
\pgfpathlineto{\pgfqpoint{2.894794in}{2.602546in}}%
\pgfpathlineto{\pgfqpoint{2.888484in}{2.618388in}}%
\pgfpathlineto{\pgfqpoint{2.880257in}{2.633033in}}%
\pgfpathlineto{\pgfqpoint{2.870348in}{2.646246in}}%
\pgfpathlineto{\pgfqpoint{2.857400in}{2.659530in}}%
\pgfpathlineto{\pgfqpoint{2.843189in}{2.671010in}}%
\pgfpathlineto{\pgfqpoint{2.824237in}{2.683209in}}%
\pgfpathlineto{\pgfqpoint{2.802413in}{2.694418in}}%
\pgfpathlineto{\pgfqpoint{2.775809in}{2.705369in}}%
\pgfpathlineto{\pgfqpoint{2.744461in}{2.715715in}}%
\pgfpathlineto{\pgfqpoint{2.708436in}{2.725252in}}%
\pgfpathlineto{\pgfqpoint{2.665655in}{2.734289in}}%
\pgfpathlineto{\pgfqpoint{2.613991in}{2.742869in}}%
\pgfpathlineto{\pgfqpoint{2.553459in}{2.750589in}}%
\pgfpathlineto{\pgfqpoint{2.481920in}{2.757365in}}%
\pgfpathlineto{\pgfqpoint{2.399398in}{2.762839in}}%
\pgfpathlineto{\pgfqpoint{2.310269in}{2.766482in}}%
\pgfpathlineto{\pgfqpoint{2.175416in}{2.768725in}}%
\pgfpathlineto{\pgfqpoint{2.066653in}{2.767942in}}%
\pgfpathlineto{\pgfqpoint{1.953570in}{2.764859in}}%
\pgfpathlineto{\pgfqpoint{1.851429in}{2.759759in}}%
\pgfpathlineto{\pgfqpoint{1.745051in}{2.752169in}}%
\pgfpathlineto{\pgfqpoint{1.658373in}{2.743453in}}%
\pgfpathlineto{\pgfqpoint{1.580552in}{2.733461in}}%
\pgfpathlineto{\pgfqpoint{1.490057in}{2.719338in}}%
\pgfpathlineto{\pgfqpoint{1.417231in}{2.704698in}}%
\pgfpathlineto{\pgfqpoint{1.361992in}{2.690818in}}%
\pgfpathlineto{\pgfqpoint{1.311460in}{2.675819in}}%
\pgfpathlineto{\pgfqpoint{1.265667in}{2.659924in}}%
\pgfpathlineto{\pgfqpoint{1.222575in}{2.642586in}}%
\pgfpathlineto{\pgfqpoint{1.184324in}{2.624682in}}%
\pgfpathlineto{\pgfqpoint{1.148892in}{2.605623in}}%
\pgfpathlineto{\pgfqpoint{1.116331in}{2.585573in}}%
\pgfpathlineto{\pgfqpoint{1.092327in}{2.568512in}}%
\pgfpathlineto{\pgfqpoint{1.079760in}{2.558686in}}%
\pgfpathlineto{\pgfqpoint{1.051544in}{2.535379in}}%
\pgfpathlineto{\pgfqpoint{1.026312in}{2.511712in}}%
\pgfpathlineto{\pgfqpoint{1.002399in}{2.486318in}}%
\pgfpathlineto{\pgfqpoint{0.979913in}{2.459269in}}%
\pgfpathlineto{\pgfqpoint{0.958934in}{2.430678in}}%
\pgfpathlineto{\pgfqpoint{0.938264in}{2.398643in}}%
\pgfpathlineto{\pgfqpoint{0.923047in}{2.371385in}}%
\pgfpathlineto{\pgfqpoint{0.904513in}{2.334774in}}%
\pgfpathlineto{\pgfqpoint{0.887854in}{2.297001in}}%
\pgfpathlineto{\pgfqpoint{0.872131in}{2.255971in}}%
\pgfpathlineto{\pgfqpoint{0.857508in}{2.211741in}}%
\pgfpathlineto{\pgfqpoint{0.844762in}{2.166757in}}%
\pgfpathlineto{\pgfqpoint{0.838624in}{2.140306in}}%
\pgfpathlineto{\pgfqpoint{0.826982in}{2.087194in}}%
\pgfpathlineto{\pgfqpoint{0.816322in}{2.028715in}}%
\pgfpathlineto{\pgfqpoint{0.810087in}{1.984495in}}%
\pgfpathlineto{\pgfqpoint{0.808026in}{1.967238in}}%
\pgfpathlineto{\pgfqpoint{0.800076in}{1.898140in}}%
\pgfpathlineto{\pgfqpoint{0.793713in}{1.823823in}}%
\pgfpathlineto{\pgfqpoint{0.788799in}{1.741875in}}%
\pgfpathlineto{\pgfqpoint{0.786199in}{1.677225in}}%
\pgfpathlineto{\pgfqpoint{0.776951in}{1.453481in}}%
\pgfpathlineto{\pgfqpoint{0.773280in}{1.418894in}}%
\pgfpathlineto{\pgfqpoint{0.768298in}{1.389582in}}%
\pgfpathlineto{\pgfqpoint{0.762752in}{1.368108in}}%
\pgfpathlineto{\pgfqpoint{0.756722in}{1.352123in}}%
\pgfpathlineto{\pgfqpoint{0.749752in}{1.339519in}}%
\pgfpathlineto{\pgfqpoint{0.742201in}{1.330599in}}%
\pgfpathlineto{\pgfqpoint{0.734854in}{1.325312in}}%
\pgfpathlineto{\pgfqpoint{0.726558in}{1.322419in}}%
\pgfpathlineto{\pgfqpoint{0.717884in}{1.322223in}}%
\pgfpathlineto{\pgfqpoint{0.709412in}{1.324411in}}%
\pgfpathlineto{\pgfqpoint{0.699548in}{1.329604in}}%
\pgfpathlineto{\pgfqpoint{0.688894in}{1.338203in}}%
\pgfpathlineto{\pgfqpoint{0.677907in}{1.350248in}}%
\pgfpathlineto{\pgfqpoint{0.666886in}{1.365647in}}%
\pgfpathlineto{\pgfqpoint{0.654913in}{1.386417in}}%
\pgfpathlineto{\pgfqpoint{0.642574in}{1.412730in}}%
\pgfpathlineto{\pgfqpoint{0.630328in}{1.444629in}}%
\pgfpathlineto{\pgfqpoint{0.618504in}{1.482081in}}%
\pgfpathlineto{\pgfqpoint{0.608613in}{1.520256in}}%
\pgfpathlineto{\pgfqpoint{0.590203in}{1.612445in}}%
\pgfpathlineto{\pgfqpoint{0.581848in}{1.668884in}}%
\pgfpathlineto{\pgfqpoint{0.573137in}{1.740376in}}%
\pgfpathlineto{\pgfqpoint{0.567062in}{1.807213in}}%
\pgfpathlineto{\pgfqpoint{0.560532in}{1.896510in}}%
\pgfpathlineto{\pgfqpoint{0.555526in}{1.995910in}}%
\pgfpathlineto{\pgfqpoint{0.552564in}{2.097908in}}%
\pgfpathlineto{\pgfqpoint{0.551526in}{2.204935in}}%
\pgfpathlineto{\pgfqpoint{0.552728in}{2.309470in}}%
\pgfpathlineto{\pgfqpoint{0.556011in}{2.403981in}}%
\pgfpathlineto{\pgfqpoint{0.560953in}{2.483430in}}%
\pgfpathlineto{\pgfqpoint{0.567303in}{2.550240in}}%
\pgfpathlineto{\pgfqpoint{0.574928in}{2.606817in}}%
\pgfpathlineto{\pgfqpoint{0.582988in}{2.650657in}}%
\pgfpathlineto{\pgfqpoint{0.592756in}{2.691452in}}%
\pgfpathlineto{\pgfqpoint{0.602650in}{2.721756in}}%
\pgfpathlineto{\pgfqpoint{0.612983in}{2.746441in}}%
\pgfpathlineto{\pgfqpoint{0.624292in}{2.767692in}}%
\pgfpathlineto{\pgfqpoint{0.636231in}{2.785433in}}%
\pgfpathlineto{\pgfqpoint{0.649892in}{2.801461in}}%
\pgfpathlineto{\pgfqpoint{0.663386in}{2.814020in}}%
\pgfpathlineto{\pgfqpoint{0.679842in}{2.826135in}}%
\pgfpathlineto{\pgfqpoint{0.697326in}{2.836197in}}%
\pgfpathlineto{\pgfqpoint{0.715574in}{2.844285in}}%
\pgfpathlineto{\pgfqpoint{0.738439in}{2.852335in}}%
\pgfpathlineto{\pgfqpoint{0.765983in}{2.859639in}}%
\pgfpathlineto{\pgfqpoint{0.800300in}{2.866256in}}%
\pgfpathlineto{\pgfqpoint{0.841340in}{2.871832in}}%
\pgfpathlineto{\pgfqpoint{0.895547in}{2.876803in}}%
\pgfpathlineto{\pgfqpoint{0.969413in}{2.881069in}}%
\pgfpathlineto{\pgfqpoint{1.071608in}{2.884501in}}%
\pgfpathlineto{\pgfqpoint{1.219512in}{2.887074in}}%
\pgfpathlineto{\pgfqpoint{1.471844in}{2.889091in}}%
\pgfpathlineto{\pgfqpoint{1.956941in}{2.890384in}}%
\pgfpathlineto{\pgfqpoint{3.096814in}{2.890781in}}%
\pgfpathlineto{\pgfqpoint{3.995224in}{2.889388in}}%
\pgfpathlineto{\pgfqpoint{4.275833in}{2.887011in}}%
\pgfpathlineto{\pgfqpoint{4.412847in}{2.883743in}}%
\pgfpathlineto{\pgfqpoint{4.491081in}{2.879810in}}%
\pgfpathlineto{\pgfqpoint{4.543127in}{2.875163in}}%
\pgfpathlineto{\pgfqpoint{4.579810in}{2.869841in}}%
\pgfpathlineto{\pgfqpoint{4.607580in}{2.863763in}}%
\pgfpathlineto{\pgfqpoint{4.630623in}{2.856424in}}%
\pgfpathlineto{\pgfqpoint{4.648833in}{2.848228in}}%
\pgfpathlineto{\pgfqpoint{4.664136in}{2.838773in}}%
\pgfpathlineto{\pgfqpoint{4.676470in}{2.828576in}}%
\pgfpathlineto{\pgfqpoint{4.687502in}{2.816585in}}%
\pgfpathlineto{\pgfqpoint{4.697051in}{2.803027in}}%
\pgfpathlineto{\pgfqpoint{4.706194in}{2.786098in}}%
\pgfpathlineto{\pgfqpoint{4.714508in}{2.765827in}}%
\pgfpathlineto{\pgfqpoint{4.722462in}{2.740013in}}%
\pgfpathlineto{\pgfqpoint{4.729577in}{2.708703in}}%
\pgfpathlineto{\pgfqpoint{4.736162in}{2.669601in}}%
\pgfpathlineto{\pgfqpoint{4.742419in}{2.617826in}}%
\pgfpathlineto{\pgfqpoint{4.747859in}{2.553410in}}%
\pgfpathlineto{\pgfqpoint{4.752661in}{2.468958in}}%
\pgfpathlineto{\pgfqpoint{4.756610in}{2.359528in}}%
\pgfpathlineto{\pgfqpoint{4.759416in}{2.217681in}}%
\pgfpathlineto{\pgfqpoint{4.760596in}{2.043444in}}%
\pgfpathlineto{\pgfqpoint{4.759662in}{1.851779in}}%
\pgfpathlineto{\pgfqpoint{4.756587in}{1.667613in}}%
\pgfpathlineto{\pgfqpoint{4.751596in}{1.503428in}}%
\pgfpathlineto{\pgfqpoint{4.745410in}{1.374185in}}%
\pgfpathlineto{\pgfqpoint{4.738113in}{1.267479in}}%
\pgfpathlineto{\pgfqpoint{4.729621in}{1.175896in}}%
\pgfpathlineto{\pgfqpoint{4.720762in}{1.104428in}}%
\pgfpathlineto{\pgfqpoint{4.711045in}{1.043204in}}%
\pgfpathlineto{\pgfqpoint{4.700364in}{0.989829in}}%
\pgfpathlineto{\pgfqpoint{4.689055in}{0.944345in}}%
\pgfpathlineto{\pgfqpoint{4.676881in}{0.904394in}}%
\pgfpathlineto{\pgfqpoint{4.676095in}{0.902073in}}%
\pgfpathlineto{\pgfqpoint{4.676095in}{0.902073in}}%
\pgfusepath{stroke}%
\end{pgfscope}%
\begin{pgfscope}%
\pgfpathrectangle{\pgfqpoint{0.448634in}{0.402556in}}{\pgfqpoint{4.350661in}{2.489204in}} %
\pgfusepath{clip}%
\pgfsetrectcap%
\pgfsetroundjoin%
\pgfsetlinewidth{1.003750pt}%
\definecolor{currentstroke}{rgb}{1.000000,0.498039,0.054902}%
\pgfsetstrokecolor{currentstroke}%
\pgfsetdash{}{0pt}%
\pgfpathmoveto{\pgfqpoint{2.795520in}{1.982745in}}%
\pgfpathlineto{\pgfqpoint{2.781780in}{1.874357in}}%
\pgfpathlineto{\pgfqpoint{2.769351in}{1.758234in}}%
\pgfpathlineto{\pgfqpoint{2.758095in}{1.631942in}}%
\pgfpathlineto{\pgfqpoint{2.747786in}{1.490551in}}%
\pgfpathlineto{\pgfqpoint{2.738644in}{1.334082in}}%
\pgfpathlineto{\pgfqpoint{2.730580in}{1.157591in}}%
\pgfpathlineto{\pgfqpoint{2.723334in}{0.948663in}}%
\pgfpathlineto{\pgfqpoint{2.709783in}{0.530788in}}%
\pgfpathlineto{\pgfqpoint{2.705868in}{0.488716in}}%
\pgfpathlineto{\pgfqpoint{2.701769in}{0.464281in}}%
\pgfpathlineto{\pgfqpoint{2.697021in}{0.447744in}}%
\pgfpathlineto{\pgfqpoint{2.691859in}{0.436812in}}%
\pgfpathlineto{\pgfqpoint{2.686245in}{0.429229in}}%
\pgfpathlineto{\pgfqpoint{2.679348in}{0.423188in}}%
\pgfpathlineto{\pgfqpoint{2.669540in}{0.417856in}}%
\pgfpathlineto{\pgfqpoint{2.656987in}{0.413810in}}%
\pgfpathlineto{\pgfqpoint{2.637654in}{0.410337in}}%
\pgfpathlineto{\pgfqpoint{2.607297in}{0.407617in}}%
\pgfpathlineto{\pgfqpoint{2.555121in}{0.405574in}}%
\pgfpathlineto{\pgfqpoint{2.450714in}{0.404139in}}%
\pgfpathlineto{\pgfqpoint{2.176624in}{0.403275in}}%
\pgfpathlineto{\pgfqpoint{1.130290in}{0.402953in}}%
\pgfpathlineto{\pgfqpoint{0.516849in}{0.404175in}}%
\pgfpathlineto{\pgfqpoint{0.466848in}{0.405970in}}%
\pgfpathlineto{\pgfqpoint{0.456130in}{0.407931in}}%
\pgfpathlineto{\pgfqpoint{0.452340in}{0.410303in}}%
\pgfpathlineto{\pgfqpoint{0.450346in}{0.414662in}}%
\pgfpathlineto{\pgfqpoint{0.449266in}{0.424524in}}%
\pgfpathlineto{\pgfqpoint{0.448771in}{0.464344in}}%
\pgfpathlineto{\pgfqpoint{0.448640in}{0.850171in}}%
\pgfpathlineto{\pgfqpoint{0.448653in}{2.891318in}}%
\pgfpathlineto{\pgfqpoint{0.448653in}{2.891318in}}%
\pgfusepath{stroke}%
\end{pgfscope}%
\begin{pgfscope}%
\pgfpathrectangle{\pgfqpoint{0.448634in}{0.402556in}}{\pgfqpoint{4.350661in}{2.489204in}} %
\pgfusepath{clip}%
\pgfsetrectcap%
\pgfsetroundjoin%
\pgfsetlinewidth{1.003750pt}%
\definecolor{currentstroke}{rgb}{1.000000,0.498039,0.054902}%
\pgfsetstrokecolor{currentstroke}%
\pgfsetdash{}{0pt}%
\pgfpathmoveto{\pgfqpoint{3.428189in}{0.402586in}}%
\pgfpathlineto{\pgfqpoint{2.782121in}{0.403701in}}%
\pgfpathlineto{\pgfqpoint{2.753906in}{0.405674in}}%
\pgfpathlineto{\pgfqpoint{2.743328in}{0.408444in}}%
\pgfpathlineto{\pgfqpoint{2.737717in}{0.412188in}}%
\pgfpathlineto{\pgfqpoint{2.733668in}{0.417995in}}%
\pgfpathlineto{\pgfqpoint{2.730649in}{0.427307in}}%
\pgfpathlineto{\pgfqpoint{2.728388in}{0.442004in}}%
\pgfpathlineto{\pgfqpoint{2.726544in}{0.471794in}}%
\pgfpathlineto{\pgfqpoint{2.725216in}{0.534003in}}%
\pgfpathlineto{\pgfqpoint{2.725169in}{0.655973in}}%
\pgfpathlineto{\pgfqpoint{2.727377in}{0.832687in}}%
\pgfpathlineto{\pgfqpoint{2.732259in}{1.041703in}}%
\pgfpathlineto{\pgfqpoint{2.738851in}{1.223257in}}%
\pgfpathlineto{\pgfqpoint{2.747078in}{1.389766in}}%
\pgfpathlineto{\pgfqpoint{2.756608in}{1.538717in}}%
\pgfpathlineto{\pgfqpoint{2.768955in}{1.694887in}}%
\pgfpathlineto{\pgfqpoint{2.781228in}{1.816044in}}%
\pgfpathlineto{\pgfqpoint{2.794401in}{1.924524in}}%
\pgfpathlineto{\pgfqpoint{2.812737in}{2.054722in}}%
\pgfpathlineto{\pgfqpoint{2.828774in}{2.147512in}}%
\pgfpathlineto{\pgfqpoint{2.847382in}{2.242224in}}%
\pgfpathlineto{\pgfqpoint{2.895818in}{2.479699in}}%
\pgfpathlineto{\pgfqpoint{2.900204in}{2.516689in}}%
\pgfpathlineto{\pgfqpoint{2.901346in}{2.544029in}}%
\pgfpathlineto{\pgfqpoint{2.900291in}{2.566388in}}%
\pgfpathlineto{\pgfqpoint{2.897334in}{2.585999in}}%
\pgfpathlineto{\pgfqpoint{2.892836in}{2.602633in}}%
\pgfpathlineto{\pgfqpoint{2.886394in}{2.618406in}}%
\pgfpathlineto{\pgfqpoint{2.878058in}{2.632970in}}%
\pgfpathlineto{\pgfqpoint{2.868065in}{2.646100in}}%
\pgfpathlineto{\pgfqpoint{2.855050in}{2.659300in}}%
\pgfpathlineto{\pgfqpoint{2.840801in}{2.670717in}}%
\pgfpathlineto{\pgfqpoint{2.821822in}{2.682861in}}%
\pgfpathlineto{\pgfqpoint{2.799980in}{2.694026in}}%
\pgfpathlineto{\pgfqpoint{2.773366in}{2.704944in}}%
\pgfpathlineto{\pgfqpoint{2.742012in}{2.715266in}}%
\pgfpathlineto{\pgfqpoint{2.705983in}{2.724785in}}%
\pgfpathlineto{\pgfqpoint{2.663200in}{2.733810in}}%
\pgfpathlineto{\pgfqpoint{2.611535in}{2.742379in}}%
\pgfpathlineto{\pgfqpoint{2.551002in}{2.750090in}}%
\pgfpathlineto{\pgfqpoint{2.481632in}{2.756682in}}%
\pgfpathlineto{\pgfqpoint{2.399112in}{2.762200in}}%
\pgfpathlineto{\pgfqpoint{2.309985in}{2.765886in}}%
\pgfpathlineto{\pgfqpoint{2.188184in}{2.768096in}}%
\pgfpathlineto{\pgfqpoint{2.081595in}{2.767619in}}%
\pgfpathlineto{\pgfqpoint{1.968506in}{2.764840in}}%
\pgfpathlineto{\pgfqpoint{1.864180in}{2.759918in}}%
\pgfpathlineto{\pgfqpoint{1.757786in}{2.752593in}}%
\pgfpathlineto{\pgfqpoint{1.671087in}{2.744171in}}%
\pgfpathlineto{\pgfqpoint{1.591076in}{2.734193in}}%
\pgfpathlineto{\pgfqpoint{1.502689in}{2.720717in}}%
\pgfpathlineto{\pgfqpoint{1.427655in}{2.706083in}}%
\pgfpathlineto{\pgfqpoint{1.372350in}{2.692544in}}%
\pgfpathlineto{\pgfqpoint{1.321734in}{2.677921in}}%
\pgfpathlineto{\pgfqpoint{1.273765in}{2.661664in}}%
\pgfpathlineto{\pgfqpoint{1.230567in}{2.644672in}}%
\pgfpathlineto{\pgfqpoint{1.192197in}{2.627106in}}%
\pgfpathlineto{\pgfqpoint{1.156620in}{2.608403in}}%
\pgfpathlineto{\pgfqpoint{1.123890in}{2.588716in}}%
\pgfpathlineto{\pgfqpoint{1.095883in}{2.569568in}}%
\pgfpathlineto{\pgfqpoint{1.063936in}{2.543701in}}%
\pgfpathlineto{\pgfqpoint{1.038217in}{2.520732in}}%
\pgfpathlineto{\pgfqpoint{1.013766in}{2.496016in}}%
\pgfpathlineto{\pgfqpoint{0.990704in}{2.469610in}}%
\pgfpathlineto{\pgfqpoint{0.969124in}{2.441612in}}%
\pgfpathlineto{\pgfqpoint{0.949083in}{2.412154in}}%
\pgfpathlineto{\pgfqpoint{0.930604in}{2.381387in}}%
\pgfpathlineto{\pgfqpoint{0.906555in}{2.334052in}}%
\pgfpathlineto{\pgfqpoint{0.889925in}{2.296262in}}%
\pgfpathlineto{\pgfqpoint{0.874241in}{2.255213in}}%
\pgfpathlineto{\pgfqpoint{0.859667in}{2.210961in}}%
\pgfpathlineto{\pgfqpoint{0.846985in}{2.165954in}}%
\pgfpathlineto{\pgfqpoint{0.839633in}{2.134715in}}%
\pgfpathlineto{\pgfqpoint{0.828238in}{2.081532in}}%
\pgfpathlineto{\pgfqpoint{0.817866in}{2.022986in}}%
\pgfpathlineto{\pgfqpoint{0.810784in}{1.971352in}}%
\pgfpathlineto{\pgfqpoint{0.802845in}{1.902252in}}%
\pgfpathlineto{\pgfqpoint{0.796554in}{1.827927in}}%
\pgfpathlineto{\pgfqpoint{0.791696in}{1.743480in}}%
\pgfpathlineto{\pgfqpoint{0.787773in}{1.621595in}}%
\pgfpathlineto{\pgfqpoint{0.785407in}{1.522064in}}%
\pgfpathlineto{\pgfqpoint{0.785407in}{1.522064in}}%
\pgfusepath{stroke}%
\end{pgfscope}%
\begin{pgfscope}%
\pgfpathrectangle{\pgfqpoint{0.448634in}{0.402556in}}{\pgfqpoint{4.350661in}{2.489204in}} %
\pgfusepath{clip}%
\pgfsetrectcap%
\pgfsetroundjoin%
\pgfsetlinewidth{1.003750pt}%
\definecolor{currentstroke}{rgb}{1.000000,0.498039,0.054902}%
\pgfsetstrokecolor{currentstroke}%
\pgfsetdash{}{0pt}%
\pgfpathmoveto{\pgfqpoint{2.028735in}{0.425754in}}%
\pgfpathlineto{\pgfqpoint{1.878677in}{0.421879in}}%
\pgfpathlineto{\pgfqpoint{1.676387in}{0.418997in}}%
\pgfpathlineto{\pgfqpoint{1.413176in}{0.417558in}}%
\pgfpathlineto{\pgfqpoint{1.134735in}{0.418204in}}%
\pgfpathlineto{\pgfqpoint{0.921565in}{0.420769in}}%
\pgfpathlineto{\pgfqpoint{0.782384in}{0.424523in}}%
\pgfpathlineto{\pgfqpoint{0.693283in}{0.428974in}}%
\pgfpathlineto{\pgfqpoint{0.632541in}{0.434091in}}%
\pgfpathlineto{\pgfqpoint{0.591492in}{0.439564in}}%
\pgfpathlineto{\pgfqpoint{0.561503in}{0.445595in}}%
\pgfpathlineto{\pgfqpoint{0.538349in}{0.452466in}}%
\pgfpathlineto{\pgfqpoint{0.522042in}{0.459394in}}%
\pgfpathlineto{\pgfqpoint{0.508540in}{0.467420in}}%
\pgfpathlineto{\pgfqpoint{0.497973in}{0.476161in}}%
\pgfpathlineto{\pgfqpoint{0.488790in}{0.486750in}}%
\pgfpathlineto{\pgfqpoint{0.481284in}{0.498948in}}%
\pgfpathlineto{\pgfqpoint{0.474590in}{0.514580in}}%
\pgfpathlineto{\pgfqpoint{0.469106in}{0.533467in}}%
\pgfpathlineto{\pgfqpoint{0.464439in}{0.557771in}}%
\pgfpathlineto{\pgfqpoint{0.460297in}{0.592289in}}%
\pgfpathlineto{\pgfqpoint{0.456856in}{0.641912in}}%
\pgfpathlineto{\pgfqpoint{0.454122in}{0.716520in}}%
\pgfpathlineto{\pgfqpoint{0.451978in}{0.843444in}}%
\pgfpathlineto{\pgfqpoint{0.450459in}{1.087380in}}%
\pgfpathlineto{\pgfqpoint{0.449596in}{1.657406in}}%
\pgfpathlineto{\pgfqpoint{0.450150in}{2.687936in}}%
\pgfpathlineto{\pgfqpoint{0.451781in}{2.839761in}}%
\pgfpathlineto{\pgfqpoint{0.453975in}{2.872003in}}%
\pgfpathlineto{\pgfqpoint{0.456339in}{2.881553in}}%
\pgfpathlineto{\pgfqpoint{0.458888in}{2.885549in}}%
\pgfpathlineto{\pgfqpoint{0.462554in}{2.888171in}}%
\pgfpathlineto{\pgfqpoint{0.471046in}{2.890205in}}%
\pgfpathlineto{\pgfqpoint{0.490597in}{2.891263in}}%
\pgfpathlineto{\pgfqpoint{0.564556in}{2.891692in}}%
\pgfpathlineto{\pgfqpoint{1.569559in}{2.891759in}}%
\pgfpathlineto{\pgfqpoint{4.784679in}{2.890785in}}%
\pgfpathlineto{\pgfqpoint{4.791005in}{2.889098in}}%
\pgfpathlineto{\pgfqpoint{4.793910in}{2.885555in}}%
\pgfpathlineto{\pgfqpoint{4.795579in}{2.878366in}}%
\pgfpathlineto{\pgfqpoint{4.796850in}{2.858513in}}%
\pgfpathlineto{\pgfqpoint{4.796850in}{2.858513in}}%
\pgfusepath{stroke}%
\end{pgfscope}%
\begin{pgfscope}%
\pgfpathrectangle{\pgfqpoint{0.448634in}{0.402556in}}{\pgfqpoint{4.350661in}{2.489204in}} %
\pgfusepath{clip}%
\pgfsetrectcap%
\pgfsetroundjoin%
\pgfsetlinewidth{1.003750pt}%
\definecolor{currentstroke}{rgb}{0.172549,0.627451,0.172549}%
\pgfsetstrokecolor{currentstroke}%
\pgfsetdash{}{0pt}%
\pgfpathmoveto{\pgfqpoint{1.127319in}{2.572074in}}%
\pgfpathlineto{\pgfqpoint{1.159575in}{2.592758in}}%
\pgfpathlineto{\pgfqpoint{1.192763in}{2.611414in}}%
\pgfpathlineto{\pgfqpoint{1.228726in}{2.629126in}}%
\pgfpathlineto{\pgfqpoint{1.267413in}{2.645758in}}%
\pgfpathlineto{\pgfqpoint{1.310846in}{2.661945in}}%
\pgfpathlineto{\pgfqpoint{1.356920in}{2.676740in}}%
\pgfpathlineto{\pgfqpoint{1.407680in}{2.690702in}}%
\pgfpathlineto{\pgfqpoint{1.463094in}{2.703640in}}%
\pgfpathlineto{\pgfqpoint{1.525273in}{2.715813in}}%
\pgfpathlineto{\pgfqpoint{1.594199in}{2.726937in}}%
\pgfpathlineto{\pgfqpoint{1.669843in}{2.736808in}}%
\pgfpathlineto{\pgfqpoint{1.752172in}{2.745271in}}%
\pgfpathlineto{\pgfqpoint{1.843325in}{2.752344in}}%
\pgfpathlineto{\pgfqpoint{1.941103in}{2.757656in}}%
\pgfpathlineto{\pgfqpoint{2.043301in}{2.760987in}}%
\pgfpathlineto{\pgfqpoint{2.147710in}{2.762199in}}%
\pgfpathlineto{\pgfqpoint{2.249945in}{2.761215in}}%
\pgfpathlineto{\pgfqpoint{2.345620in}{2.758145in}}%
\pgfpathlineto{\pgfqpoint{2.432525in}{2.753210in}}%
\pgfpathlineto{\pgfqpoint{2.508451in}{2.746766in}}%
\pgfpathlineto{\pgfqpoint{2.573368in}{2.739156in}}%
\pgfpathlineto{\pgfqpoint{2.629410in}{2.730451in}}%
\pgfpathlineto{\pgfqpoint{2.676543in}{2.720985in}}%
\pgfpathlineto{\pgfqpoint{2.716874in}{2.710666in}}%
\pgfpathlineto{\pgfqpoint{2.750366in}{2.699848in}}%
\pgfpathlineto{\pgfqpoint{2.779059in}{2.688192in}}%
\pgfpathlineto{\pgfqpoint{2.802882in}{2.676004in}}%
\pgfpathlineto{\pgfqpoint{2.821842in}{2.663819in}}%
\pgfpathlineto{\pgfqpoint{2.837815in}{2.650886in}}%
\pgfpathlineto{\pgfqpoint{2.850736in}{2.637564in}}%
\pgfpathlineto{\pgfqpoint{2.860694in}{2.624398in}}%
\pgfpathlineto{\pgfqpoint{2.869084in}{2.609873in}}%
\pgfpathlineto{\pgfqpoint{2.875698in}{2.594192in}}%
\pgfpathlineto{\pgfqpoint{2.881035in}{2.575255in}}%
\pgfpathlineto{\pgfqpoint{2.884200in}{2.555685in}}%
\pgfpathlineto{\pgfqpoint{2.885619in}{2.533351in}}%
\pgfpathlineto{\pgfqpoint{2.885038in}{2.505987in}}%
\pgfpathlineto{\pgfqpoint{2.882112in}{2.473807in}}%
\pgfpathlineto{\pgfqpoint{2.875657in}{2.429620in}}%
\pgfpathlineto{\pgfqpoint{2.863489in}{2.363873in}}%
\pgfpathlineto{\pgfqpoint{2.821102in}{2.142619in}}%
\pgfpathlineto{\pgfqpoint{2.804859in}{2.042271in}}%
\pgfpathlineto{\pgfqpoint{2.790421in}{1.939040in}}%
\pgfpathlineto{\pgfqpoint{2.777207in}{1.828054in}}%
\pgfpathlineto{\pgfqpoint{2.765338in}{1.709349in}}%
\pgfpathlineto{\pgfqpoint{2.754471in}{1.578010in}}%
\pgfpathlineto{\pgfqpoint{2.744640in}{1.431580in}}%
\pgfpathlineto{\pgfqpoint{2.735914in}{1.267598in}}%
\pgfpathlineto{\pgfqpoint{2.728277in}{1.081114in}}%
\pgfpathlineto{\pgfqpoint{2.721437in}{0.857223in}}%
\pgfpathlineto{\pgfqpoint{2.711961in}{0.541290in}}%
\pgfpathlineto{\pgfqpoint{2.708250in}{0.491694in}}%
\pgfpathlineto{\pgfqpoint{2.703951in}{0.462246in}}%
\pgfpathlineto{\pgfqpoint{2.699504in}{0.445599in}}%
\pgfpathlineto{\pgfqpoint{2.694517in}{0.434563in}}%
\pgfpathlineto{\pgfqpoint{2.688942in}{0.426947in}}%
\pgfpathlineto{\pgfqpoint{2.681980in}{0.421009in}}%
\pgfpathlineto{\pgfqpoint{2.672064in}{0.415948in}}%
\pgfpathlineto{\pgfqpoint{2.659429in}{0.412247in}}%
\pgfpathlineto{\pgfqpoint{2.640044in}{0.409163in}}%
\pgfpathlineto{\pgfqpoint{2.607490in}{0.406692in}}%
\pgfpathlineto{\pgfqpoint{2.548779in}{0.404894in}}%
\pgfpathlineto{\pgfqpoint{2.422615in}{0.403701in}}%
\pgfpathlineto{\pgfqpoint{2.026705in}{0.403016in}}%
\pgfpathlineto{\pgfqpoint{0.623617in}{0.403253in}}%
\pgfpathlineto{\pgfqpoint{0.477880in}{0.404742in}}%
\pgfpathlineto{\pgfqpoint{0.458368in}{0.406382in}}%
\pgfpathlineto{\pgfqpoint{0.452304in}{0.408937in}}%
\pgfpathlineto{\pgfqpoint{0.450213in}{0.413215in}}%
\pgfpathlineto{\pgfqpoint{0.449165in}{0.423080in}}%
\pgfpathlineto{\pgfqpoint{0.448735in}{0.465392in}}%
\pgfpathlineto{\pgfqpoint{0.448637in}{0.983146in}}%
\pgfpathlineto{\pgfqpoint{0.448652in}{2.889876in}}%
\pgfpathlineto{\pgfqpoint{0.448652in}{2.889876in}}%
\pgfusepath{stroke}%
\end{pgfscope}%
\begin{pgfscope}%
\pgfpathrectangle{\pgfqpoint{0.448634in}{0.402556in}}{\pgfqpoint{4.350661in}{2.489204in}} %
\pgfusepath{clip}%
\pgfsetrectcap%
\pgfsetroundjoin%
\pgfsetlinewidth{1.003750pt}%
\definecolor{currentstroke}{rgb}{0.172549,0.627451,0.172549}%
\pgfsetstrokecolor{currentstroke}%
\pgfsetdash{}{0pt}%
\pgfpathmoveto{\pgfqpoint{0.448634in}{2.896245in}}%
\pgfpathlineto{\pgfqpoint{0.448593in}{0.407043in}}%
\pgfpathlineto{\pgfqpoint{0.448593in}{0.407043in}}%
\pgfusepath{stroke}%
\end{pgfscope}%
\begin{pgfscope}%
\pgfpathrectangle{\pgfqpoint{0.448634in}{0.402556in}}{\pgfqpoint{4.350661in}{2.489204in}} %
\pgfusepath{clip}%
\pgfsetrectcap%
\pgfsetroundjoin%
\pgfsetlinewidth{1.003750pt}%
\definecolor{currentstroke}{rgb}{0.172549,0.627451,0.172549}%
\pgfsetstrokecolor{currentstroke}%
\pgfsetdash{}{0pt}%
\pgfpathmoveto{\pgfqpoint{0.576853in}{1.760817in}}%
\pgfpathlineto{\pgfqpoint{0.569394in}{1.840010in}}%
\pgfpathlineto{\pgfqpoint{0.563208in}{1.929339in}}%
\pgfpathlineto{\pgfqpoint{0.558592in}{2.028764in}}%
\pgfpathlineto{\pgfqpoint{0.555985in}{2.133265in}}%
\pgfpathlineto{\pgfqpoint{0.555565in}{2.237808in}}%
\pgfpathlineto{\pgfqpoint{0.557371in}{2.337352in}}%
\pgfpathlineto{\pgfqpoint{0.561096in}{2.424367in}}%
\pgfpathlineto{\pgfqpoint{0.566403in}{2.498791in}}%
\pgfpathlineto{\pgfqpoint{0.572909in}{2.560570in}}%
\pgfpathlineto{\pgfqpoint{0.580458in}{2.612119in}}%
\pgfpathlineto{\pgfqpoint{0.589086in}{2.655816in}}%
\pgfpathlineto{\pgfqpoint{0.598406in}{2.691590in}}%
\pgfpathlineto{\pgfqpoint{0.608613in}{2.721757in}}%
\pgfpathlineto{\pgfqpoint{0.619241in}{2.746278in}}%
\pgfpathlineto{\pgfqpoint{0.630817in}{2.767339in}}%
\pgfpathlineto{\pgfqpoint{0.642975in}{2.784884in}}%
\pgfpathlineto{\pgfqpoint{0.656813in}{2.800713in}}%
\pgfpathlineto{\pgfqpoint{0.672197in}{2.814549in}}%
\pgfpathlineto{\pgfqpoint{0.688853in}{2.826301in}}%
\pgfpathlineto{\pgfqpoint{0.706461in}{2.836076in}}%
\pgfpathlineto{\pgfqpoint{0.726804in}{2.844876in}}%
\pgfpathlineto{\pgfqpoint{0.751866in}{2.853203in}}%
\pgfpathlineto{\pgfqpoint{0.781632in}{2.860547in}}%
\pgfpathlineto{\pgfqpoint{0.818168in}{2.867054in}}%
\pgfpathlineto{\pgfqpoint{0.863581in}{2.872685in}}%
\pgfpathlineto{\pgfqpoint{0.922161in}{2.877518in}}%
\pgfpathlineto{\pgfqpoint{1.000391in}{2.881567in}}%
\pgfpathlineto{\pgfqpoint{1.111294in}{2.884881in}}%
\pgfpathlineto{\pgfqpoint{1.274428in}{2.887367in}}%
\pgfpathlineto{\pgfqpoint{1.552865in}{2.889263in}}%
\pgfpathlineto{\pgfqpoint{2.107573in}{2.890457in}}%
\pgfpathlineto{\pgfqpoint{3.343161in}{2.890573in}}%
\pgfpathlineto{\pgfqpoint{4.043615in}{2.888941in}}%
\pgfpathlineto{\pgfqpoint{4.289417in}{2.886404in}}%
\pgfpathlineto{\pgfqpoint{4.413375in}{2.883093in}}%
\pgfpathlineto{\pgfqpoint{4.489425in}{2.878997in}}%
\pgfpathlineto{\pgfqpoint{4.541451in}{2.874081in}}%
\pgfpathlineto{\pgfqpoint{4.578100in}{2.868470in}}%
\pgfpathlineto{\pgfqpoint{4.605819in}{2.862092in}}%
\pgfpathlineto{\pgfqpoint{4.626726in}{2.855245in}}%
\pgfpathlineto{\pgfqpoint{4.644925in}{2.847018in}}%
\pgfpathlineto{\pgfqpoint{4.660241in}{2.837590in}}%
\pgfpathlineto{\pgfqpoint{4.672623in}{2.827468in}}%
\pgfpathlineto{\pgfqpoint{4.683751in}{2.815592in}}%
\pgfpathlineto{\pgfqpoint{4.693406in}{2.802135in}}%
\pgfpathlineto{\pgfqpoint{4.702740in}{2.785343in}}%
\pgfpathlineto{\pgfqpoint{4.711277in}{2.765194in}}%
\pgfpathlineto{\pgfqpoint{4.719483in}{2.739483in}}%
\pgfpathlineto{\pgfqpoint{4.726294in}{2.710657in}}%
\pgfpathlineto{\pgfqpoint{4.733260in}{2.671642in}}%
\pgfpathlineto{\pgfqpoint{4.739604in}{2.622396in}}%
\pgfpathlineto{\pgfqpoint{4.745236in}{2.560504in}}%
\pgfpathlineto{\pgfqpoint{4.750164in}{2.481051in}}%
\pgfpathlineto{\pgfqpoint{4.754367in}{2.376618in}}%
\pgfpathlineto{\pgfqpoint{4.757443in}{2.242249in}}%
\pgfpathlineto{\pgfqpoint{4.758977in}{2.075483in}}%
\pgfpathlineto{\pgfqpoint{4.758447in}{1.888795in}}%
\pgfpathlineto{\pgfqpoint{4.755756in}{1.707110in}}%
\pgfpathlineto{\pgfqpoint{4.750925in}{1.532957in}}%
\pgfpathlineto{\pgfqpoint{4.744785in}{1.398726in}}%
\pgfpathlineto{\pgfqpoint{4.737575in}{1.289515in}}%
\pgfpathlineto{\pgfqpoint{4.728714in}{1.190469in}}%
\pgfpathlineto{\pgfqpoint{4.719652in}{1.116521in}}%
\pgfpathlineto{\pgfqpoint{4.710036in}{1.055276in}}%
\pgfpathlineto{\pgfqpoint{4.699503in}{1.001861in}}%
\pgfpathlineto{\pgfqpoint{4.689040in}{0.958690in}}%
\pgfpathlineto{\pgfqpoint{4.677220in}{0.918600in}}%
\pgfpathlineto{\pgfqpoint{4.664034in}{0.881749in}}%
\pgfpathlineto{\pgfqpoint{4.650584in}{0.850491in}}%
\pgfpathlineto{\pgfqpoint{4.636303in}{0.822569in}}%
\pgfpathlineto{\pgfqpoint{4.620207in}{0.795974in}}%
\pgfpathlineto{\pgfqpoint{4.603640in}{0.772901in}}%
\pgfpathlineto{\pgfqpoint{4.585488in}{0.751446in}}%
\pgfpathlineto{\pgfqpoint{4.565874in}{0.731748in}}%
\pgfpathlineto{\pgfqpoint{4.544964in}{0.713879in}}%
\pgfpathlineto{\pgfqpoint{4.522958in}{0.697824in}}%
\pgfpathlineto{\pgfqpoint{4.496157in}{0.681290in}}%
\pgfpathlineto{\pgfqpoint{4.470397in}{0.667953in}}%
\pgfpathlineto{\pgfqpoint{4.439961in}{0.654509in}}%
\pgfpathlineto{\pgfqpoint{4.406841in}{0.642281in}}%
\pgfpathlineto{\pgfqpoint{4.369009in}{0.630748in}}%
\pgfpathlineto{\pgfqpoint{4.326489in}{0.620226in}}%
\pgfpathlineto{\pgfqpoint{4.279327in}{0.610949in}}%
\pgfpathlineto{\pgfqpoint{4.227576in}{0.603085in}}%
\pgfpathlineto{\pgfqpoint{4.173450in}{0.597063in}}%
\pgfpathlineto{\pgfqpoint{4.110511in}{0.592203in}}%
\pgfpathlineto{\pgfqpoint{4.047471in}{0.589536in}}%
\pgfpathlineto{\pgfqpoint{3.977867in}{0.588624in}}%
\pgfpathlineto{\pgfqpoint{3.906093in}{0.589933in}}%
\pgfpathlineto{\pgfqpoint{3.834377in}{0.593496in}}%
\pgfpathlineto{\pgfqpoint{3.767120in}{0.599067in}}%
\pgfpathlineto{\pgfqpoint{3.704364in}{0.606392in}}%
\pgfpathlineto{\pgfqpoint{3.678516in}{0.610510in}}%
\pgfpathlineto{\pgfqpoint{3.620438in}{0.620500in}}%
\pgfpathlineto{\pgfqpoint{3.586319in}{0.628207in}}%
\pgfpathlineto{\pgfqpoint{3.495240in}{0.652428in}}%
\pgfpathlineto{\pgfqpoint{3.451528in}{0.667583in}}%
\pgfpathlineto{\pgfqpoint{3.408538in}{0.685220in}}%
\pgfpathlineto{\pgfqpoint{3.374594in}{0.702001in}}%
\pgfpathlineto{\pgfqpoint{3.345407in}{0.718682in}}%
\pgfpathlineto{\pgfqpoint{3.315236in}{0.738520in}}%
\pgfpathlineto{\pgfqpoint{3.288127in}{0.759290in}}%
\pgfpathlineto{\pgfqpoint{3.264004in}{0.780551in}}%
\pgfpathlineto{\pgfqpoint{3.241208in}{0.803648in}}%
\pgfpathlineto{\pgfqpoint{3.219894in}{0.828530in}}%
\pgfpathlineto{\pgfqpoint{3.200189in}{0.855091in}}%
\pgfpathlineto{\pgfqpoint{3.182177in}{0.883182in}}%
\pgfpathlineto{\pgfqpoint{3.165906in}{0.912633in}}%
\pgfpathlineto{\pgfqpoint{3.150351in}{0.945448in}}%
\pgfpathlineto{\pgfqpoint{3.136682in}{0.979345in}}%
\pgfpathlineto{\pgfqpoint{3.124073in}{1.016460in}}%
\pgfpathlineto{\pgfqpoint{3.112834in}{1.056769in}}%
\pgfpathlineto{\pgfqpoint{3.103046in}{1.100146in}}%
\pgfpathlineto{\pgfqpoint{3.095343in}{1.144071in}}%
\pgfpathlineto{\pgfqpoint{3.089208in}{1.190837in}}%
\pgfpathlineto{\pgfqpoint{3.084595in}{1.242838in}}%
\pgfpathlineto{\pgfqpoint{3.082137in}{1.295031in}}%
\pgfpathlineto{\pgfqpoint{3.081687in}{1.349787in}}%
\pgfpathlineto{\pgfqpoint{3.083451in}{1.406998in}}%
\pgfpathlineto{\pgfqpoint{3.087181in}{1.461589in}}%
\pgfpathlineto{\pgfqpoint{3.093485in}{1.520888in}}%
\pgfpathlineto{\pgfqpoint{3.101823in}{1.577334in}}%
\pgfpathlineto{\pgfqpoint{3.111930in}{1.630856in}}%
\pgfpathlineto{\pgfqpoint{3.124690in}{1.686208in}}%
\pgfpathlineto{\pgfqpoint{3.139178in}{1.738395in}}%
\pgfpathlineto{\pgfqpoint{3.155145in}{1.787366in}}%
\pgfpathlineto{\pgfqpoint{3.172353in}{1.833085in}}%
\pgfpathlineto{\pgfqpoint{3.191618in}{1.877716in}}%
\pgfpathlineto{\pgfqpoint{3.214026in}{1.923261in}}%
\pgfpathlineto{\pgfqpoint{3.236214in}{1.963157in}}%
\pgfpathlineto{\pgfqpoint{3.260178in}{2.001684in}}%
\pgfpathlineto{\pgfqpoint{3.285814in}{2.038776in}}%
\pgfpathlineto{\pgfqpoint{3.314415in}{2.076285in}}%
\pgfpathlineto{\pgfqpoint{3.348944in}{2.117711in}}%
\pgfpathlineto{\pgfqpoint{3.417133in}{2.198022in}}%
\pgfpathlineto{\pgfqpoint{3.426053in}{2.212128in}}%
\pgfpathlineto{\pgfqpoint{3.430798in}{2.223297in}}%
\pgfpathlineto{\pgfqpoint{3.432034in}{2.230603in}}%
\pgfpathlineto{\pgfqpoint{3.430773in}{2.237856in}}%
\pgfpathlineto{\pgfqpoint{3.426621in}{2.243526in}}%
\pgfpathlineto{\pgfqpoint{3.420908in}{2.247084in}}%
\pgfpathlineto{\pgfqpoint{3.412501in}{2.249583in}}%
\pgfpathlineto{\pgfqpoint{3.399499in}{2.250689in}}%
\pgfpathlineto{\pgfqpoint{3.384305in}{2.249671in}}%
\pgfpathlineto{\pgfqpoint{3.364985in}{2.246098in}}%
\pgfpathlineto{\pgfqpoint{3.341804in}{2.239342in}}%
\pgfpathlineto{\pgfqpoint{3.317109in}{2.229682in}}%
\pgfpathlineto{\pgfqpoint{3.291104in}{2.216986in}}%
\pgfpathlineto{\pgfqpoint{3.265928in}{2.202261in}}%
\pgfpathlineto{\pgfqpoint{3.239805in}{2.184361in}}%
\pgfpathlineto{\pgfqpoint{3.214775in}{2.164519in}}%
\pgfpathlineto{\pgfqpoint{3.190900in}{2.142893in}}%
\pgfpathlineto{\pgfqpoint{3.166657in}{2.117912in}}%
\pgfpathlineto{\pgfqpoint{3.143835in}{2.091233in}}%
\pgfpathlineto{\pgfqpoint{3.121079in}{2.061107in}}%
\pgfpathlineto{\pgfqpoint{3.099952in}{2.029463in}}%
\pgfpathlineto{\pgfqpoint{3.079251in}{1.994406in}}%
\pgfpathlineto{\pgfqpoint{3.059218in}{1.955915in}}%
\pgfpathlineto{\pgfqpoint{3.040058in}{1.914015in}}%
\pgfpathlineto{\pgfqpoint{3.022809in}{1.871041in}}%
\pgfpathlineto{\pgfqpoint{3.005790in}{1.822536in}}%
\pgfpathlineto{\pgfqpoint{2.990067in}{1.770819in}}%
\pgfpathlineto{\pgfqpoint{2.975708in}{1.715979in}}%
\pgfpathlineto{\pgfqpoint{2.962284in}{1.655680in}}%
\pgfpathlineto{\pgfqpoint{2.950496in}{1.592386in}}%
\pgfpathlineto{\pgfqpoint{2.940383in}{1.526185in}}%
\pgfpathlineto{\pgfqpoint{2.931745in}{1.454681in}}%
\pgfpathlineto{\pgfqpoint{2.925082in}{1.380399in}}%
\pgfpathlineto{\pgfqpoint{2.920647in}{1.305899in}}%
\pgfpathlineto{\pgfqpoint{2.918444in}{1.231270in}}%
\pgfpathlineto{\pgfqpoint{2.918545in}{1.159087in}}%
\pgfpathlineto{\pgfqpoint{2.920787in}{1.091931in}}%
\pgfpathlineto{\pgfqpoint{2.925177in}{1.027412in}}%
\pgfpathlineto{\pgfqpoint{2.931192in}{0.970580in}}%
\pgfpathlineto{\pgfqpoint{2.938760in}{0.919034in}}%
\pgfpathlineto{\pgfqpoint{2.947651in}{0.872852in}}%
\pgfpathlineto{\pgfqpoint{2.958213in}{0.829714in}}%
\pgfpathlineto{\pgfqpoint{2.969670in}{0.792114in}}%
\pgfpathlineto{\pgfqpoint{2.982463in}{0.757773in}}%
\pgfpathlineto{\pgfqpoint{2.996425in}{0.726812in}}%
\pgfpathlineto{\pgfqpoint{3.011299in}{0.699300in}}%
\pgfpathlineto{\pgfqpoint{3.026739in}{0.675225in}}%
\pgfpathlineto{\pgfqpoint{3.043828in}{0.652656in}}%
\pgfpathlineto{\pgfqpoint{3.062495in}{0.631788in}}%
\pgfpathlineto{\pgfqpoint{3.082602in}{0.612753in}}%
\pgfpathlineto{\pgfqpoint{3.103961in}{0.595592in}}%
\pgfpathlineto{\pgfqpoint{3.128268in}{0.579069in}}%
\pgfpathlineto{\pgfqpoint{3.153537in}{0.564554in}}%
\pgfpathlineto{\pgfqpoint{3.181571in}{0.550952in}}%
\pgfpathlineto{\pgfqpoint{3.214371in}{0.537647in}}%
\pgfpathlineto{\pgfqpoint{3.249846in}{0.525712in}}%
\pgfpathlineto{\pgfqpoint{3.290011in}{0.514571in}}%
\pgfpathlineto{\pgfqpoint{3.334820in}{0.504423in}}%
\pgfpathlineto{\pgfqpoint{3.386372in}{0.494999in}}%
\pgfpathlineto{\pgfqpoint{3.446798in}{0.486257in}}%
\pgfpathlineto{\pgfqpoint{3.518243in}{0.478282in}}%
\pgfpathlineto{\pgfqpoint{3.600685in}{0.471409in}}%
\pgfpathlineto{\pgfqpoint{3.696268in}{0.465713in}}%
\pgfpathlineto{\pgfqpoint{3.807144in}{0.461369in}}%
\pgfpathlineto{\pgfqpoint{3.933291in}{0.458719in}}%
\pgfpathlineto{\pgfqpoint{4.063808in}{0.458211in}}%
\pgfpathlineto{\pgfqpoint{4.187792in}{0.459914in}}%
\pgfpathlineto{\pgfqpoint{4.294335in}{0.463521in}}%
\pgfpathlineto{\pgfqpoint{4.381234in}{0.468574in}}%
\pgfpathlineto{\pgfqpoint{4.450636in}{0.474701in}}%
\pgfpathlineto{\pgfqpoint{4.506850in}{0.481799in}}%
\pgfpathlineto{\pgfqpoint{4.552009in}{0.489658in}}%
\pgfpathlineto{\pgfqpoint{4.588239in}{0.498115in}}%
\pgfpathlineto{\pgfqpoint{4.617656in}{0.507110in}}%
\pgfpathlineto{\pgfqpoint{4.642328in}{0.516843in}}%
\pgfpathlineto{\pgfqpoint{4.664194in}{0.527940in}}%
\pgfpathlineto{\pgfqpoint{4.681238in}{0.538945in}}%
\pgfpathlineto{\pgfqpoint{4.697164in}{0.551953in}}%
\pgfpathlineto{\pgfqpoint{4.710076in}{0.565289in}}%
\pgfpathlineto{\pgfqpoint{4.721578in}{0.580218in}}%
\pgfpathlineto{\pgfqpoint{4.731557in}{0.596521in}}%
\pgfpathlineto{\pgfqpoint{4.741000in}{0.616134in}}%
\pgfpathlineto{\pgfqpoint{4.749521in}{0.639027in}}%
\pgfpathlineto{\pgfqpoint{4.757522in}{0.667450in}}%
\pgfpathlineto{\pgfqpoint{4.764572in}{0.701345in}}%
\pgfpathlineto{\pgfqpoint{4.770840in}{0.743043in}}%
\pgfpathlineto{\pgfqpoint{4.776327in}{0.794934in}}%
\pgfpathlineto{\pgfqpoint{4.781278in}{0.864398in}}%
\pgfpathlineto{\pgfqpoint{4.785468in}{0.956371in}}%
\pgfpathlineto{\pgfqpoint{4.789000in}{1.085745in}}%
\pgfpathlineto{\pgfqpoint{4.791852in}{1.277385in}}%
\pgfpathlineto{\pgfqpoint{4.793959in}{1.581057in}}%
\pgfpathlineto{\pgfqpoint{4.794962in}{2.071429in}}%
\pgfpathlineto{\pgfqpoint{4.793967in}{2.559311in}}%
\pgfpathlineto{\pgfqpoint{4.791733in}{2.745981in}}%
\pgfpathlineto{\pgfqpoint{4.788955in}{2.818091in}}%
\pgfpathlineto{\pgfqpoint{4.785731in}{2.850227in}}%
\pgfpathlineto{\pgfqpoint{4.781879in}{2.867057in}}%
\pgfpathlineto{\pgfqpoint{4.777744in}{2.875780in}}%
\pgfpathlineto{\pgfqpoint{4.773097in}{2.880982in}}%
\pgfpathlineto{\pgfqpoint{4.767363in}{2.884504in}}%
\pgfpathlineto{\pgfqpoint{4.756853in}{2.887622in}}%
\pgfpathlineto{\pgfqpoint{4.739548in}{2.889639in}}%
\pgfpathlineto{\pgfqpoint{4.704762in}{2.890882in}}%
\pgfpathlineto{\pgfqpoint{4.602524in}{2.891538in}}%
\pgfpathlineto{\pgfqpoint{3.952100in}{2.891742in}}%
\pgfpathlineto{\pgfqpoint{0.617321in}{2.890753in}}%
\pgfpathlineto{\pgfqpoint{0.549910in}{2.888858in}}%
\pgfpathlineto{\pgfqpoint{0.521735in}{2.886179in}}%
\pgfpathlineto{\pgfqpoint{0.504666in}{2.882389in}}%
\pgfpathlineto{\pgfqpoint{0.494501in}{2.878011in}}%
\pgfpathlineto{\pgfqpoint{0.487180in}{2.872667in}}%
\pgfpathlineto{\pgfqpoint{0.481152in}{2.865519in}}%
\pgfpathlineto{\pgfqpoint{0.475664in}{2.854804in}}%
\pgfpathlineto{\pgfqpoint{0.471318in}{2.840737in}}%
\pgfpathlineto{\pgfqpoint{0.467301in}{2.818823in}}%
\pgfpathlineto{\pgfqpoint{0.463927in}{2.786700in}}%
\pgfpathlineto{\pgfqpoint{0.460918in}{2.734544in}}%
\pgfpathlineto{\pgfqpoint{0.458363in}{2.647473in}}%
\pgfpathlineto{\pgfqpoint{0.456575in}{2.523031in}}%
\pgfpathlineto{\pgfqpoint{0.456575in}{2.523031in}}%
\pgfusepath{stroke}%
\end{pgfscope}%
\begin{pgfscope}%
\pgfpathrectangle{\pgfqpoint{0.448634in}{0.402556in}}{\pgfqpoint{4.350661in}{2.489204in}} %
\pgfusepath{clip}%
\pgfsetrectcap%
\pgfsetroundjoin%
\pgfsetlinewidth{1.003750pt}%
\definecolor{currentstroke}{rgb}{0.172549,0.627451,0.172549}%
\pgfsetstrokecolor{currentstroke}%
\pgfsetdash{}{0pt}%
\pgfpathmoveto{\pgfqpoint{0.456424in}{1.370137in}}%
\pgfpathlineto{\pgfqpoint{0.459610in}{1.118755in}}%
\pgfpathlineto{\pgfqpoint{0.463695in}{0.962007in}}%
\pgfpathlineto{\pgfqpoint{0.468519in}{0.857610in}}%
\pgfpathlineto{\pgfqpoint{0.474082in}{0.783210in}}%
\pgfpathlineto{\pgfqpoint{0.480226in}{0.728906in}}%
\pgfpathlineto{\pgfqpoint{0.486970in}{0.687306in}}%
\pgfpathlineto{\pgfqpoint{0.494537in}{0.653558in}}%
\pgfpathlineto{\pgfqpoint{0.503107in}{0.625355in}}%
\pgfpathlineto{\pgfqpoint{0.512193in}{0.602749in}}%
\pgfpathlineto{\pgfqpoint{0.522200in}{0.583508in}}%
\pgfpathlineto{\pgfqpoint{0.534108in}{0.565743in}}%
\pgfpathlineto{\pgfqpoint{0.546263in}{0.551507in}}%
\pgfpathlineto{\pgfqpoint{0.559728in}{0.538907in}}%
\pgfpathlineto{\pgfqpoint{0.576130in}{0.526693in}}%
\pgfpathlineto{\pgfqpoint{0.595483in}{0.515351in}}%
\pgfpathlineto{\pgfqpoint{0.617681in}{0.505147in}}%
\pgfpathlineto{\pgfqpoint{0.642568in}{0.496153in}}%
\pgfpathlineto{\pgfqpoint{0.672126in}{0.487778in}}%
\pgfpathlineto{\pgfqpoint{0.708443in}{0.479824in}}%
\pgfpathlineto{\pgfqpoint{0.753649in}{0.472325in}}%
\pgfpathlineto{\pgfqpoint{0.807718in}{0.465660in}}%
\pgfpathlineto{\pgfqpoint{0.877116in}{0.459475in}}%
\pgfpathlineto{\pgfqpoint{0.961828in}{0.454230in}}%
\pgfpathlineto{\pgfqpoint{1.068351in}{0.449916in}}%
\pgfpathlineto{\pgfqpoint{1.201018in}{0.446839in}}%
\pgfpathlineto{\pgfqpoint{1.357637in}{0.445481in}}%
\pgfpathlineto{\pgfqpoint{1.525135in}{0.446232in}}%
\pgfpathlineto{\pgfqpoint{1.686088in}{0.449142in}}%
\pgfpathlineto{\pgfqpoint{1.823074in}{0.453747in}}%
\pgfpathlineto{\pgfqpoint{1.938245in}{0.459764in}}%
\pgfpathlineto{\pgfqpoint{2.031582in}{0.466759in}}%
\pgfpathlineto{\pgfqpoint{2.109580in}{0.474745in}}%
\pgfpathlineto{\pgfqpoint{2.174384in}{0.483535in}}%
\pgfpathlineto{\pgfqpoint{2.228139in}{0.492940in}}%
\pgfpathlineto{\pgfqpoint{2.275119in}{0.503356in}}%
\pgfpathlineto{\pgfqpoint{2.315282in}{0.514501in}}%
\pgfpathlineto{\pgfqpoint{2.350698in}{0.526659in}}%
\pgfpathlineto{\pgfqpoint{2.381320in}{0.539536in}}%
\pgfpathlineto{\pgfqpoint{2.407164in}{0.552659in}}%
\pgfpathlineto{\pgfqpoint{2.430226in}{0.566639in}}%
\pgfpathlineto{\pgfqpoint{2.452282in}{0.582602in}}%
\pgfpathlineto{\pgfqpoint{2.471391in}{0.599069in}}%
\pgfpathlineto{\pgfqpoint{2.489240in}{0.617293in}}%
\pgfpathlineto{\pgfqpoint{2.505678in}{0.637180in}}%
\pgfpathlineto{\pgfqpoint{2.520620in}{0.658557in}}%
\pgfpathlineto{\pgfqpoint{2.535213in}{0.683314in}}%
\pgfpathlineto{\pgfqpoint{2.549115in}{0.711484in}}%
\pgfpathlineto{\pgfqpoint{2.562091in}{0.743004in}}%
\pgfpathlineto{\pgfqpoint{2.574020in}{0.777751in}}%
\pgfpathlineto{\pgfqpoint{2.585502in}{0.817970in}}%
\pgfpathlineto{\pgfqpoint{2.596809in}{0.866038in}}%
\pgfpathlineto{\pgfqpoint{2.607562in}{0.921948in}}%
\pgfpathlineto{\pgfqpoint{2.617925in}{0.988098in}}%
\pgfpathlineto{\pgfqpoint{2.627958in}{1.066918in}}%
\pgfpathlineto{\pgfqpoint{2.637941in}{1.163320in}}%
\pgfpathlineto{\pgfqpoint{2.648424in}{1.287199in}}%
\pgfpathlineto{\pgfqpoint{2.660103in}{1.453438in}}%
\pgfpathlineto{\pgfqpoint{2.674773in}{1.696801in}}%
\pgfpathlineto{\pgfqpoint{2.687716in}{1.945279in}}%
\pgfpathlineto{\pgfqpoint{2.692670in}{2.079573in}}%
\pgfpathlineto{\pgfqpoint{2.693829in}{2.166682in}}%
\pgfpathlineto{\pgfqpoint{2.692565in}{2.233870in}}%
\pgfpathlineto{\pgfqpoint{2.689436in}{2.286015in}}%
\pgfpathlineto{\pgfqpoint{2.684859in}{2.327999in}}%
\pgfpathlineto{\pgfqpoint{2.678725in}{2.364664in}}%
\pgfpathlineto{\pgfqpoint{2.671356in}{2.395897in}}%
\pgfpathlineto{\pgfqpoint{2.662489in}{2.423981in}}%
\pgfpathlineto{\pgfqpoint{2.652361in}{2.448778in}}%
\pgfpathlineto{\pgfqpoint{2.641365in}{2.470245in}}%
\pgfpathlineto{\pgfqpoint{2.628643in}{2.490425in}}%
\pgfpathlineto{\pgfqpoint{2.614279in}{2.509106in}}%
\pgfpathlineto{\pgfqpoint{2.598443in}{2.526159in}}%
\pgfpathlineto{\pgfqpoint{2.579590in}{2.543005in}}%
\pgfpathlineto{\pgfqpoint{2.559532in}{2.557923in}}%
\pgfpathlineto{\pgfqpoint{2.536602in}{2.572183in}}%
\pgfpathlineto{\pgfqpoint{2.510850in}{2.585538in}}%
\pgfpathlineto{\pgfqpoint{2.482360in}{2.597837in}}%
\pgfpathlineto{\pgfqpoint{2.449134in}{2.609683in}}%
\pgfpathlineto{\pgfqpoint{2.411184in}{2.620696in}}%
\pgfpathlineto{\pgfqpoint{2.368552in}{2.630606in}}%
\pgfpathlineto{\pgfqpoint{2.321294in}{2.639221in}}%
\pgfpathlineto{\pgfqpoint{2.269467in}{2.646399in}}%
\pgfpathlineto{\pgfqpoint{2.210954in}{2.652193in}}%
\pgfpathlineto{\pgfqpoint{2.147967in}{2.656153in}}%
\pgfpathlineto{\pgfqpoint{2.080556in}{2.658135in}}%
\pgfpathlineto{\pgfqpoint{2.010948in}{2.657971in}}%
\pgfpathlineto{\pgfqpoint{1.939195in}{2.655572in}}%
\pgfpathlineto{\pgfqpoint{1.867527in}{2.650913in}}%
\pgfpathlineto{\pgfqpoint{1.798171in}{2.644140in}}%
\pgfpathlineto{\pgfqpoint{1.733341in}{2.635606in}}%
\pgfpathlineto{\pgfqpoint{1.673075in}{2.625521in}}%
\pgfpathlineto{\pgfqpoint{1.615274in}{2.613610in}}%
\pgfpathlineto{\pgfqpoint{1.562133in}{2.600402in}}%
\pgfpathlineto{\pgfqpoint{1.513681in}{2.586139in}}%
\pgfpathlineto{\pgfqpoint{1.467862in}{2.570344in}}%
\pgfpathlineto{\pgfqpoint{1.426794in}{2.553923in}}%
\pgfpathlineto{\pgfqpoint{1.388447in}{2.536289in}}%
\pgfpathlineto{\pgfqpoint{1.352878in}{2.517566in}}%
\pgfpathlineto{\pgfqpoint{1.320128in}{2.497922in}}%
\pgfpathlineto{\pgfqpoint{1.288379in}{2.476236in}}%
\pgfpathlineto{\pgfqpoint{1.259592in}{2.453861in}}%
\pgfpathlineto{\pgfqpoint{1.232050in}{2.429520in}}%
\pgfpathlineto{\pgfqpoint{1.207527in}{2.404898in}}%
\pgfpathlineto{\pgfqpoint{1.184409in}{2.378557in}}%
\pgfpathlineto{\pgfqpoint{1.162828in}{2.350561in}}%
\pgfpathlineto{\pgfqpoint{1.142891in}{2.321011in}}%
\pgfpathlineto{\pgfqpoint{1.124675in}{2.290041in}}%
\pgfpathlineto{\pgfqpoint{1.108225in}{2.257802in}}%
\pgfpathlineto{\pgfqpoint{1.092639in}{2.222199in}}%
\pgfpathlineto{\pgfqpoint{1.079059in}{2.185535in}}%
\pgfpathlineto{\pgfqpoint{1.067443in}{2.147998in}}%
\pgfpathlineto{\pgfqpoint{1.057187in}{2.107348in}}%
\pgfpathlineto{\pgfqpoint{1.049004in}{2.066086in}}%
\pgfpathlineto{\pgfqpoint{1.042513in}{2.021906in}}%
\pgfpathlineto{\pgfqpoint{1.038177in}{1.977382in}}%
\pgfpathlineto{\pgfqpoint{1.035866in}{1.930167in}}%
\pgfpathlineto{\pgfqpoint{1.035826in}{1.882878in}}%
\pgfpathlineto{\pgfqpoint{1.038031in}{1.835656in}}%
\pgfpathlineto{\pgfqpoint{1.042474in}{1.788641in}}%
\pgfpathlineto{\pgfqpoint{1.049176in}{1.741979in}}%
\pgfpathlineto{\pgfqpoint{1.057644in}{1.698239in}}%
\pgfpathlineto{\pgfqpoint{1.068221in}{1.655105in}}%
\pgfpathlineto{\pgfqpoint{1.080962in}{1.612745in}}%
\pgfpathlineto{\pgfqpoint{1.095031in}{1.573617in}}%
\pgfpathlineto{\pgfqpoint{1.111115in}{1.535520in}}%
\pgfpathlineto{\pgfqpoint{1.128118in}{1.500775in}}%
\pgfpathlineto{\pgfqpoint{1.146930in}{1.467274in}}%
\pgfpathlineto{\pgfqpoint{1.167531in}{1.435181in}}%
\pgfpathlineto{\pgfqpoint{1.189874in}{1.404652in}}%
\pgfpathlineto{\pgfqpoint{1.213884in}{1.375828in}}%
\pgfpathlineto{\pgfqpoint{1.237817in}{1.350457in}}%
\pgfpathlineto{\pgfqpoint{1.264748in}{1.325237in}}%
\pgfpathlineto{\pgfqpoint{1.292991in}{1.301972in}}%
\pgfpathlineto{\pgfqpoint{1.322398in}{1.280678in}}%
\pgfpathlineto{\pgfqpoint{1.352820in}{1.261340in}}%
\pgfpathlineto{\pgfqpoint{1.386095in}{1.242889in}}%
\pgfpathlineto{\pgfqpoint{1.420190in}{1.226516in}}%
\pgfpathlineto{\pgfqpoint{1.457024in}{1.211329in}}%
\pgfpathlineto{\pgfqpoint{1.496554in}{1.197536in}}%
\pgfpathlineto{\pgfqpoint{1.538719in}{1.185287in}}%
\pgfpathlineto{\pgfqpoint{1.583441in}{1.174641in}}%
\pgfpathlineto{\pgfqpoint{1.634929in}{1.164775in}}%
\pgfpathlineto{\pgfqpoint{1.706063in}{1.153745in}}%
\pgfpathlineto{\pgfqpoint{1.768492in}{1.143417in}}%
\pgfpathlineto{\pgfqpoint{1.796122in}{1.136567in}}%
\pgfpathlineto{\pgfqpoint{1.812683in}{1.130481in}}%
\pgfpathlineto{\pgfqpoint{1.824471in}{1.124102in}}%
\pgfpathlineto{\pgfqpoint{1.833209in}{1.116741in}}%
\pgfpathlineto{\pgfqpoint{1.838498in}{1.108890in}}%
\pgfpathlineto{\pgfqpoint{1.840588in}{1.101849in}}%
\pgfpathlineto{\pgfqpoint{1.840619in}{1.094412in}}%
\pgfpathlineto{\pgfqpoint{1.837931in}{1.084986in}}%
\pgfpathlineto{\pgfqpoint{1.833246in}{1.076615in}}%
\pgfpathlineto{\pgfqpoint{1.825819in}{1.067542in}}%
\pgfpathlineto{\pgfqpoint{1.813813in}{1.056850in}}%
\pgfpathlineto{\pgfqpoint{1.798819in}{1.046763in}}%
\pgfpathlineto{\pgfqpoint{1.781016in}{1.037462in}}%
\pgfpathlineto{\pgfqpoint{1.758447in}{1.028391in}}%
\pgfpathlineto{\pgfqpoint{1.733203in}{1.020815in}}%
\pgfpathlineto{\pgfqpoint{1.705410in}{1.014872in}}%
\pgfpathlineto{\pgfqpoint{1.675178in}{1.010714in}}%
\pgfpathlineto{\pgfqpoint{1.642610in}{1.008507in}}%
\pgfpathlineto{\pgfqpoint{1.607809in}{1.008432in}}%
\pgfpathlineto{\pgfqpoint{1.570886in}{1.010691in}}%
\pgfpathlineto{\pgfqpoint{1.534118in}{1.015181in}}%
\pgfpathlineto{\pgfqpoint{1.495454in}{1.022233in}}%
\pgfpathlineto{\pgfqpoint{1.457161in}{1.031563in}}%
\pgfpathlineto{\pgfqpoint{1.419337in}{1.043132in}}%
\pgfpathlineto{\pgfqpoint{1.382089in}{1.056929in}}%
\pgfpathlineto{\pgfqpoint{1.347544in}{1.072019in}}%
\pgfpathlineto{\pgfqpoint{1.313727in}{1.089133in}}%
\pgfpathlineto{\pgfqpoint{1.280762in}{1.108299in}}%
\pgfpathlineto{\pgfqpoint{1.248782in}{1.129536in}}%
\pgfpathlineto{\pgfqpoint{1.219708in}{1.151422in}}%
\pgfpathlineto{\pgfqpoint{1.191752in}{1.175138in}}%
\pgfpathlineto{\pgfqpoint{1.165031in}{1.200649in}}%
\pgfpathlineto{\pgfqpoint{1.139653in}{1.227898in}}%
\pgfpathlineto{\pgfqpoint{1.115714in}{1.256800in}}%
\pgfpathlineto{\pgfqpoint{1.093288in}{1.287251in}}%
\pgfpathlineto{\pgfqpoint{1.071178in}{1.321163in}}%
\pgfpathlineto{\pgfqpoint{1.050868in}{1.356520in}}%
\pgfpathlineto{\pgfqpoint{1.032365in}{1.393152in}}%
\pgfpathlineto{\pgfqpoint{1.014718in}{1.433142in}}%
\pgfpathlineto{\pgfqpoint{0.999024in}{1.474185in}}%
\pgfpathlineto{\pgfqpoint{0.984506in}{1.518461in}}%
\pgfpathlineto{\pgfqpoint{0.972010in}{1.563537in}}%
\pgfpathlineto{\pgfqpoint{0.960944in}{1.611678in}}%
\pgfpathlineto{\pgfqpoint{0.951530in}{1.662824in}}%
\pgfpathlineto{\pgfqpoint{0.944286in}{1.714431in}}%
\pgfpathlineto{\pgfqpoint{0.938950in}{1.768847in}}%
\pgfpathlineto{\pgfqpoint{0.935870in}{1.823491in}}%
\pgfpathlineto{\pgfqpoint{0.935034in}{1.878240in}}%
\pgfpathlineto{\pgfqpoint{0.936466in}{1.932973in}}%
\pgfpathlineto{\pgfqpoint{0.940005in}{1.985084in}}%
\pgfpathlineto{\pgfqpoint{0.945759in}{2.036935in}}%
\pgfpathlineto{\pgfqpoint{0.953410in}{2.085938in}}%
\pgfpathlineto{\pgfqpoint{0.962764in}{2.132000in}}%
\pgfpathlineto{\pgfqpoint{0.974287in}{2.177414in}}%
\pgfpathlineto{\pgfqpoint{0.987332in}{2.219653in}}%
\pgfpathlineto{\pgfqpoint{1.001667in}{2.258654in}}%
\pgfpathlineto{\pgfqpoint{1.018051in}{2.296583in}}%
\pgfpathlineto{\pgfqpoint{1.035401in}{2.331101in}}%
\pgfpathlineto{\pgfqpoint{1.054650in}{2.364275in}}%
\pgfpathlineto{\pgfqpoint{1.074406in}{2.393984in}}%
\pgfpathlineto{\pgfqpoint{1.095771in}{2.422197in}}%
\pgfpathlineto{\pgfqpoint{1.118662in}{2.448797in}}%
\pgfpathlineto{\pgfqpoint{1.142967in}{2.473701in}}%
\pgfpathlineto{\pgfqpoint{1.168550in}{2.496867in}}%
\pgfpathlineto{\pgfqpoint{1.197085in}{2.519662in}}%
\pgfpathlineto{\pgfqpoint{1.226727in}{2.540526in}}%
\pgfpathlineto{\pgfqpoint{1.259242in}{2.560673in}}%
\pgfpathlineto{\pgfqpoint{1.294612in}{2.579881in}}%
\pgfpathlineto{\pgfqpoint{1.332792in}{2.597982in}}%
\pgfpathlineto{\pgfqpoint{1.373719in}{2.614859in}}%
\pgfpathlineto{\pgfqpoint{1.417319in}{2.630445in}}%
\pgfpathlineto{\pgfqpoint{1.465632in}{2.645312in}}%
\pgfpathlineto{\pgfqpoint{1.518640in}{2.659204in}}%
\pgfpathlineto{\pgfqpoint{1.576309in}{2.671929in}}%
\pgfpathlineto{\pgfqpoint{1.638597in}{2.683344in}}%
\pgfpathlineto{\pgfqpoint{1.705462in}{2.693343in}}%
\pgfpathlineto{\pgfqpoint{1.779027in}{2.702064in}}%
\pgfpathlineto{\pgfqpoint{1.857097in}{2.709077in}}%
\pgfpathlineto{\pgfqpoint{1.939633in}{2.714280in}}%
\pgfpathlineto{\pgfqpoint{2.026598in}{2.717513in}}%
\pgfpathlineto{\pgfqpoint{2.113605in}{2.718523in}}%
\pgfpathlineto{\pgfqpoint{2.198435in}{2.717303in}}%
\pgfpathlineto{\pgfqpoint{2.278866in}{2.713929in}}%
\pgfpathlineto{\pgfqpoint{2.352678in}{2.708598in}}%
\pgfpathlineto{\pgfqpoint{2.417657in}{2.701709in}}%
\pgfpathlineto{\pgfqpoint{2.473770in}{2.693630in}}%
\pgfpathlineto{\pgfqpoint{2.523140in}{2.684368in}}%
\pgfpathlineto{\pgfqpoint{2.565726in}{2.674202in}}%
\pgfpathlineto{\pgfqpoint{2.601510in}{2.663544in}}%
\pgfpathlineto{\pgfqpoint{2.632577in}{2.652142in}}%
\pgfpathlineto{\pgfqpoint{2.658899in}{2.640331in}}%
\pgfpathlineto{\pgfqpoint{2.682438in}{2.627436in}}%
\pgfpathlineto{\pgfqpoint{2.703062in}{2.613571in}}%
\pgfpathlineto{\pgfqpoint{2.720674in}{2.598978in}}%
\pgfpathlineto{\pgfqpoint{2.735263in}{2.584053in}}%
\pgfpathlineto{\pgfqpoint{2.748320in}{2.567377in}}%
\pgfpathlineto{\pgfqpoint{2.759553in}{2.549046in}}%
\pgfpathlineto{\pgfqpoint{2.768788in}{2.529306in}}%
\pgfpathlineto{\pgfqpoint{2.776017in}{2.508498in}}%
\pgfpathlineto{\pgfqpoint{2.781884in}{2.484540in}}%
\pgfpathlineto{\pgfqpoint{2.786102in}{2.457597in}}%
\pgfpathlineto{\pgfqpoint{2.788720in}{2.425384in}}%
\pgfpathlineto{\pgfqpoint{2.789427in}{2.388061in}}%
\pgfpathlineto{\pgfqpoint{2.787962in}{2.340801in}}%
\pgfpathlineto{\pgfqpoint{2.783672in}{2.278768in}}%
\pgfpathlineto{\pgfqpoint{2.774289in}{2.179783in}}%
\pgfpathlineto{\pgfqpoint{2.743611in}{1.868119in}}%
\pgfpathlineto{\pgfqpoint{2.730112in}{1.702060in}}%
\pgfpathlineto{\pgfqpoint{2.717287in}{1.515949in}}%
\pgfpathlineto{\pgfqpoint{2.702602in}{1.267597in}}%
\pgfpathlineto{\pgfqpoint{2.684434in}{0.964630in}}%
\pgfpathlineto{\pgfqpoint{2.675374in}{0.850600in}}%
\pgfpathlineto{\pgfqpoint{2.667030in}{0.771523in}}%
\pgfpathlineto{\pgfqpoint{2.658752in}{0.712543in}}%
\pgfpathlineto{\pgfqpoint{2.650176in}{0.666284in}}%
\pgfpathlineto{\pgfqpoint{2.640820in}{0.627931in}}%
\pgfpathlineto{\pgfqpoint{2.631145in}{0.597534in}}%
\pgfpathlineto{\pgfqpoint{2.621004in}{0.572745in}}%
\pgfpathlineto{\pgfqpoint{2.609856in}{0.551383in}}%
\pgfpathlineto{\pgfqpoint{2.598042in}{0.533534in}}%
\pgfpathlineto{\pgfqpoint{2.584496in}{0.517378in}}%
\pgfpathlineto{\pgfqpoint{2.571109in}{0.504669in}}%
\pgfpathlineto{\pgfqpoint{2.554789in}{0.492313in}}%
\pgfpathlineto{\pgfqpoint{2.537457in}{0.481914in}}%
\pgfpathlineto{\pgfqpoint{2.517374in}{0.472367in}}%
\pgfpathlineto{\pgfqpoint{2.492542in}{0.463178in}}%
\pgfpathlineto{\pgfqpoint{2.462979in}{0.454833in}}%
\pgfpathlineto{\pgfqpoint{2.428766in}{0.447542in}}%
\pgfpathlineto{\pgfqpoint{2.385671in}{0.440735in}}%
\pgfpathlineto{\pgfqpoint{2.331557in}{0.434581in}}%
\pgfpathlineto{\pgfqpoint{2.262115in}{0.429077in}}%
\pgfpathlineto{\pgfqpoint{2.170851in}{0.424236in}}%
\pgfpathlineto{\pgfqpoint{2.049086in}{0.420134in}}%
\pgfpathlineto{\pgfqpoint{1.879436in}{0.416783in}}%
\pgfpathlineto{\pgfqpoint{1.640159in}{0.414418in}}%
\pgfpathlineto{\pgfqpoint{1.322562in}{0.413569in}}%
\pgfpathlineto{\pgfqpoint{1.020194in}{0.414850in}}%
\pgfpathlineto{\pgfqpoint{0.822256in}{0.417715in}}%
\pgfpathlineto{\pgfqpoint{0.704835in}{0.421430in}}%
\pgfpathlineto{\pgfqpoint{0.630976in}{0.425829in}}%
\pgfpathlineto{\pgfqpoint{0.583316in}{0.430734in}}%
\pgfpathlineto{\pgfqpoint{0.551033in}{0.436123in}}%
\pgfpathlineto{\pgfqpoint{0.527708in}{0.442189in}}%
\pgfpathlineto{\pgfqpoint{0.511250in}{0.448625in}}%
\pgfpathlineto{\pgfqpoint{0.499549in}{0.455216in}}%
\pgfpathlineto{\pgfqpoint{0.488916in}{0.463841in}}%
\pgfpathlineto{\pgfqpoint{0.481322in}{0.472730in}}%
\pgfpathlineto{\pgfqpoint{0.474078in}{0.485127in}}%
\pgfpathlineto{\pgfqpoint{0.468753in}{0.498748in}}%
\pgfpathlineto{\pgfqpoint{0.463870in}{0.517848in}}%
\pgfpathlineto{\pgfqpoint{0.459679in}{0.544796in}}%
\pgfpathlineto{\pgfqpoint{0.456386in}{0.581938in}}%
\pgfpathlineto{\pgfqpoint{0.453731in}{0.639106in}}%
\pgfpathlineto{\pgfqpoint{0.451681in}{0.736155in}}%
\pgfpathlineto{\pgfqpoint{0.450220in}{0.927815in}}%
\pgfpathlineto{\pgfqpoint{0.449345in}{1.403252in}}%
\pgfpathlineto{\pgfqpoint{0.449543in}{2.682703in}}%
\pgfpathlineto{\pgfqpoint{0.451011in}{2.856932in}}%
\pgfpathlineto{\pgfqpoint{0.452802in}{2.879219in}}%
\pgfpathlineto{\pgfqpoint{0.455188in}{2.886108in}}%
\pgfpathlineto{\pgfqpoint{0.458626in}{2.889028in}}%
\pgfpathlineto{\pgfqpoint{0.464996in}{2.890553in}}%
\pgfpathlineto{\pgfqpoint{0.482377in}{2.891423in}}%
\pgfpathlineto{\pgfqpoint{0.565038in}{2.891729in}}%
\pgfpathlineto{\pgfqpoint{2.733842in}{2.891760in}}%
\pgfpathlineto{\pgfqpoint{4.789510in}{2.890885in}}%
\pgfpathlineto{\pgfqpoint{4.793727in}{2.889730in}}%
\pgfpathlineto{\pgfqpoint{4.795481in}{2.888307in}}%
\pgfpathlineto{\pgfqpoint{4.797106in}{2.881145in}}%
\pgfpathlineto{\pgfqpoint{4.797997in}{2.858771in}}%
\pgfpathlineto{\pgfqpoint{4.798039in}{2.856283in}}%
\pgfpathlineto{\pgfqpoint{4.798039in}{2.856283in}}%
\pgfusepath{stroke}%
\end{pgfscope}%
\begin{pgfscope}%
\pgfpathrectangle{\pgfqpoint{0.448634in}{0.402556in}}{\pgfqpoint{4.350661in}{2.489204in}} %
\pgfusepath{clip}%
\pgfsetrectcap%
\pgfsetroundjoin%
\pgfsetlinewidth{1.003750pt}%
\definecolor{currentstroke}{rgb}{0.172549,0.627451,0.172549}%
\pgfsetstrokecolor{currentstroke}%
\pgfsetdash{}{0pt}%
\pgfpathmoveto{\pgfqpoint{3.428773in}{0.402610in}}%
\pgfpathlineto{\pgfqpoint{2.806632in}{0.403760in}}%
\pgfpathlineto{\pgfqpoint{2.769692in}{0.405578in}}%
\pgfpathlineto{\pgfqpoint{2.754633in}{0.408064in}}%
\pgfpathlineto{\pgfqpoint{2.746391in}{0.411198in}}%
\pgfpathlineto{\pgfqpoint{2.740943in}{0.415265in}}%
\pgfpathlineto{\pgfqpoint{2.736785in}{0.420984in}}%
\pgfpathlineto{\pgfqpoint{2.733281in}{0.430071in}}%
\pgfpathlineto{\pgfqpoint{2.730449in}{0.444636in}}%
\pgfpathlineto{\pgfqpoint{2.728238in}{0.469392in}}%
\pgfpathlineto{\pgfqpoint{2.726470in}{0.519131in}}%
\pgfpathlineto{\pgfqpoint{2.725711in}{0.613715in}}%
\pgfpathlineto{\pgfqpoint{2.726842in}{0.768039in}}%
\pgfpathlineto{\pgfqpoint{2.730556in}{0.962149in}}%
\pgfpathlineto{\pgfqpoint{2.736611in}{1.158671in}}%
\pgfpathlineto{\pgfqpoint{2.744092in}{1.327718in}}%
\pgfpathlineto{\pgfqpoint{2.753201in}{1.484189in}}%
\pgfpathlineto{\pgfqpoint{2.763257in}{1.620609in}}%
\pgfpathlineto{\pgfqpoint{2.776118in}{1.764216in}}%
\pgfpathlineto{\pgfqpoint{2.788914in}{1.877777in}}%
\pgfpathlineto{\pgfqpoint{2.805748in}{2.005740in}}%
\pgfpathlineto{\pgfqpoint{2.821176in}{2.101198in}}%
\pgfpathlineto{\pgfqpoint{2.838360in}{2.193719in}}%
\pgfpathlineto{\pgfqpoint{2.859135in}{2.292966in}}%
\pgfpathlineto{\pgfqpoint{2.887209in}{2.425960in}}%
\pgfpathlineto{\pgfqpoint{2.896991in}{2.479560in}}%
\pgfpathlineto{\pgfqpoint{2.901543in}{2.516523in}}%
\pgfpathlineto{\pgfqpoint{2.902849in}{2.543854in}}%
\pgfpathlineto{\pgfqpoint{2.901958in}{2.566223in}}%
\pgfpathlineto{\pgfqpoint{2.899152in}{2.585863in}}%
\pgfpathlineto{\pgfqpoint{2.894794in}{2.602546in}}%
\pgfpathlineto{\pgfqpoint{2.888484in}{2.618388in}}%
\pgfpathlineto{\pgfqpoint{2.880258in}{2.633033in}}%
\pgfpathlineto{\pgfqpoint{2.870348in}{2.646246in}}%
\pgfpathlineto{\pgfqpoint{2.857400in}{2.659531in}}%
\pgfpathlineto{\pgfqpoint{2.843189in}{2.671010in}}%
\pgfpathlineto{\pgfqpoint{2.824238in}{2.683209in}}%
\pgfpathlineto{\pgfqpoint{2.802413in}{2.694419in}}%
\pgfpathlineto{\pgfqpoint{2.775809in}{2.705369in}}%
\pgfpathlineto{\pgfqpoint{2.744462in}{2.715715in}}%
\pgfpathlineto{\pgfqpoint{2.708436in}{2.725252in}}%
\pgfpathlineto{\pgfqpoint{2.665655in}{2.734289in}}%
\pgfpathlineto{\pgfqpoint{2.613992in}{2.742869in}}%
\pgfpathlineto{\pgfqpoint{2.553459in}{2.750589in}}%
\pgfpathlineto{\pgfqpoint{2.481920in}{2.757365in}}%
\pgfpathlineto{\pgfqpoint{2.399398in}{2.762839in}}%
\pgfpathlineto{\pgfqpoint{2.310269in}{2.766482in}}%
\pgfpathlineto{\pgfqpoint{2.175416in}{2.768725in}}%
\pgfpathlineto{\pgfqpoint{2.066654in}{2.767942in}}%
\pgfpathlineto{\pgfqpoint{1.953571in}{2.764859in}}%
\pgfpathlineto{\pgfqpoint{1.851429in}{2.759759in}}%
\pgfpathlineto{\pgfqpoint{1.745051in}{2.752169in}}%
\pgfpathlineto{\pgfqpoint{1.658374in}{2.743454in}}%
\pgfpathlineto{\pgfqpoint{1.580552in}{2.733461in}}%
\pgfpathlineto{\pgfqpoint{1.490058in}{2.719338in}}%
\pgfpathlineto{\pgfqpoint{1.417232in}{2.704698in}}%
\pgfpathlineto{\pgfqpoint{1.361992in}{2.690818in}}%
\pgfpathlineto{\pgfqpoint{1.311460in}{2.675819in}}%
\pgfpathlineto{\pgfqpoint{1.265667in}{2.659924in}}%
\pgfpathlineto{\pgfqpoint{1.222575in}{2.642586in}}%
\pgfpathlineto{\pgfqpoint{1.184324in}{2.624682in}}%
\pgfpathlineto{\pgfqpoint{1.148892in}{2.605623in}}%
\pgfpathlineto{\pgfqpoint{1.116332in}{2.585573in}}%
\pgfpathlineto{\pgfqpoint{1.092327in}{2.568512in}}%
\pgfpathlineto{\pgfqpoint{1.079760in}{2.558686in}}%
\pgfpathlineto{\pgfqpoint{1.051544in}{2.535379in}}%
\pgfpathlineto{\pgfqpoint{1.026312in}{2.511712in}}%
\pgfpathlineto{\pgfqpoint{1.002399in}{2.486318in}}%
\pgfpathlineto{\pgfqpoint{0.979913in}{2.459269in}}%
\pgfpathlineto{\pgfqpoint{0.958934in}{2.430678in}}%
\pgfpathlineto{\pgfqpoint{0.938264in}{2.398644in}}%
\pgfpathlineto{\pgfqpoint{0.923047in}{2.371385in}}%
\pgfpathlineto{\pgfqpoint{0.904513in}{2.334774in}}%
\pgfpathlineto{\pgfqpoint{0.887854in}{2.297001in}}%
\pgfpathlineto{\pgfqpoint{0.872132in}{2.255972in}}%
\pgfpathlineto{\pgfqpoint{0.857508in}{2.211741in}}%
\pgfpathlineto{\pgfqpoint{0.844762in}{2.166757in}}%
\pgfpathlineto{\pgfqpoint{0.838624in}{2.140307in}}%
\pgfpathlineto{\pgfqpoint{0.826982in}{2.087194in}}%
\pgfpathlineto{\pgfqpoint{0.816322in}{2.028716in}}%
\pgfpathlineto{\pgfqpoint{0.810087in}{1.984495in}}%
\pgfpathlineto{\pgfqpoint{0.808026in}{1.967238in}}%
\pgfpathlineto{\pgfqpoint{0.800076in}{1.898141in}}%
\pgfpathlineto{\pgfqpoint{0.793713in}{1.823823in}}%
\pgfpathlineto{\pgfqpoint{0.788798in}{1.741875in}}%
\pgfpathlineto{\pgfqpoint{0.786199in}{1.677226in}}%
\pgfpathlineto{\pgfqpoint{0.776951in}{1.453482in}}%
\pgfpathlineto{\pgfqpoint{0.773280in}{1.418895in}}%
\pgfpathlineto{\pgfqpoint{0.768298in}{1.389583in}}%
\pgfpathlineto{\pgfqpoint{0.762752in}{1.368109in}}%
\pgfpathlineto{\pgfqpoint{0.756722in}{1.352123in}}%
\pgfpathlineto{\pgfqpoint{0.749752in}{1.339519in}}%
\pgfpathlineto{\pgfqpoint{0.742201in}{1.330600in}}%
\pgfpathlineto{\pgfqpoint{0.734854in}{1.325312in}}%
\pgfpathlineto{\pgfqpoint{0.726558in}{1.322419in}}%
\pgfpathlineto{\pgfqpoint{0.717884in}{1.322223in}}%
\pgfpathlineto{\pgfqpoint{0.709413in}{1.324411in}}%
\pgfpathlineto{\pgfqpoint{0.699548in}{1.329604in}}%
\pgfpathlineto{\pgfqpoint{0.688894in}{1.338203in}}%
\pgfpathlineto{\pgfqpoint{0.677907in}{1.350248in}}%
\pgfpathlineto{\pgfqpoint{0.666886in}{1.365647in}}%
\pgfpathlineto{\pgfqpoint{0.654913in}{1.386417in}}%
\pgfpathlineto{\pgfqpoint{0.642574in}{1.412730in}}%
\pgfpathlineto{\pgfqpoint{0.630328in}{1.444629in}}%
\pgfpathlineto{\pgfqpoint{0.618505in}{1.482081in}}%
\pgfpathlineto{\pgfqpoint{0.608613in}{1.520256in}}%
\pgfpathlineto{\pgfqpoint{0.590203in}{1.612445in}}%
\pgfpathlineto{\pgfqpoint{0.581848in}{1.668884in}}%
\pgfpathlineto{\pgfqpoint{0.573137in}{1.740376in}}%
\pgfpathlineto{\pgfqpoint{0.567062in}{1.807213in}}%
\pgfpathlineto{\pgfqpoint{0.560532in}{1.896510in}}%
\pgfpathlineto{\pgfqpoint{0.555526in}{1.995910in}}%
\pgfpathlineto{\pgfqpoint{0.552564in}{2.097908in}}%
\pgfpathlineto{\pgfqpoint{0.551526in}{2.204935in}}%
\pgfpathlineto{\pgfqpoint{0.552728in}{2.309470in}}%
\pgfpathlineto{\pgfqpoint{0.556011in}{2.403981in}}%
\pgfpathlineto{\pgfqpoint{0.560953in}{2.483430in}}%
\pgfpathlineto{\pgfqpoint{0.567303in}{2.550240in}}%
\pgfpathlineto{\pgfqpoint{0.574928in}{2.606817in}}%
\pgfpathlineto{\pgfqpoint{0.582988in}{2.650657in}}%
\pgfpathlineto{\pgfqpoint{0.592756in}{2.691452in}}%
\pgfpathlineto{\pgfqpoint{0.602650in}{2.721756in}}%
\pgfpathlineto{\pgfqpoint{0.612983in}{2.746441in}}%
\pgfpathlineto{\pgfqpoint{0.624292in}{2.767692in}}%
\pgfpathlineto{\pgfqpoint{0.636231in}{2.785433in}}%
\pgfpathlineto{\pgfqpoint{0.649892in}{2.801461in}}%
\pgfpathlineto{\pgfqpoint{0.663386in}{2.814020in}}%
\pgfpathlineto{\pgfqpoint{0.679842in}{2.826135in}}%
\pgfpathlineto{\pgfqpoint{0.697326in}{2.836197in}}%
\pgfpathlineto{\pgfqpoint{0.715574in}{2.844285in}}%
\pgfpathlineto{\pgfqpoint{0.738439in}{2.852335in}}%
\pgfpathlineto{\pgfqpoint{0.765983in}{2.859639in}}%
\pgfpathlineto{\pgfqpoint{0.800300in}{2.866256in}}%
\pgfpathlineto{\pgfqpoint{0.841340in}{2.871832in}}%
\pgfpathlineto{\pgfqpoint{0.895547in}{2.876803in}}%
\pgfpathlineto{\pgfqpoint{0.969413in}{2.881069in}}%
\pgfpathlineto{\pgfqpoint{1.071608in}{2.884501in}}%
\pgfpathlineto{\pgfqpoint{1.219512in}{2.887074in}}%
\pgfpathlineto{\pgfqpoint{1.471844in}{2.889091in}}%
\pgfpathlineto{\pgfqpoint{1.956941in}{2.890384in}}%
\pgfpathlineto{\pgfqpoint{3.096814in}{2.890781in}}%
\pgfpathlineto{\pgfqpoint{3.995224in}{2.889388in}}%
\pgfpathlineto{\pgfqpoint{4.275833in}{2.887011in}}%
\pgfpathlineto{\pgfqpoint{4.412847in}{2.883743in}}%
\pgfpathlineto{\pgfqpoint{4.491081in}{2.879810in}}%
\pgfpathlineto{\pgfqpoint{4.543127in}{2.875163in}}%
\pgfpathlineto{\pgfqpoint{4.579810in}{2.869841in}}%
\pgfpathlineto{\pgfqpoint{4.607580in}{2.863763in}}%
\pgfpathlineto{\pgfqpoint{4.630623in}{2.856424in}}%
\pgfpathlineto{\pgfqpoint{4.648833in}{2.848228in}}%
\pgfpathlineto{\pgfqpoint{4.664136in}{2.838773in}}%
\pgfpathlineto{\pgfqpoint{4.676470in}{2.828576in}}%
\pgfpathlineto{\pgfqpoint{4.687502in}{2.816585in}}%
\pgfpathlineto{\pgfqpoint{4.697051in}{2.803027in}}%
\pgfpathlineto{\pgfqpoint{4.706194in}{2.786098in}}%
\pgfpathlineto{\pgfqpoint{4.714508in}{2.765827in}}%
\pgfpathlineto{\pgfqpoint{4.722462in}{2.740013in}}%
\pgfpathlineto{\pgfqpoint{4.729577in}{2.708703in}}%
\pgfpathlineto{\pgfqpoint{4.736162in}{2.669601in}}%
\pgfpathlineto{\pgfqpoint{4.742419in}{2.617826in}}%
\pgfpathlineto{\pgfqpoint{4.747859in}{2.553410in}}%
\pgfpathlineto{\pgfqpoint{4.752661in}{2.468958in}}%
\pgfpathlineto{\pgfqpoint{4.756610in}{2.359528in}}%
\pgfpathlineto{\pgfqpoint{4.759416in}{2.217681in}}%
\pgfpathlineto{\pgfqpoint{4.760596in}{2.043444in}}%
\pgfpathlineto{\pgfqpoint{4.759662in}{1.851779in}}%
\pgfpathlineto{\pgfqpoint{4.756587in}{1.667613in}}%
\pgfpathlineto{\pgfqpoint{4.751596in}{1.503428in}}%
\pgfpathlineto{\pgfqpoint{4.745410in}{1.374185in}}%
\pgfpathlineto{\pgfqpoint{4.738113in}{1.267479in}}%
\pgfpathlineto{\pgfqpoint{4.729621in}{1.175896in}}%
\pgfpathlineto{\pgfqpoint{4.720762in}{1.104428in}}%
\pgfpathlineto{\pgfqpoint{4.711045in}{1.043204in}}%
\pgfpathlineto{\pgfqpoint{4.700364in}{0.989829in}}%
\pgfpathlineto{\pgfqpoint{4.689055in}{0.944345in}}%
\pgfpathlineto{\pgfqpoint{4.676881in}{0.904394in}}%
\pgfpathlineto{\pgfqpoint{4.676095in}{0.902073in}}%
\pgfpathlineto{\pgfqpoint{4.676095in}{0.902073in}}%
\pgfusepath{stroke}%
\end{pgfscope}%
\begin{pgfscope}%
\pgfpathrectangle{\pgfqpoint{0.448634in}{0.402556in}}{\pgfqpoint{4.350661in}{2.489204in}} %
\pgfusepath{clip}%
\pgfsetrectcap%
\pgfsetroundjoin%
\pgfsetlinewidth{1.003750pt}%
\definecolor{currentstroke}{rgb}{0.172549,0.627451,0.172549}%
\pgfsetstrokecolor{currentstroke}%
\pgfsetdash{}{0pt}%
\pgfpathmoveto{\pgfqpoint{2.795520in}{1.982745in}}%
\pgfpathlineto{\pgfqpoint{2.781780in}{1.874357in}}%
\pgfpathlineto{\pgfqpoint{2.769351in}{1.758234in}}%
\pgfpathlineto{\pgfqpoint{2.758095in}{1.631942in}}%
\pgfpathlineto{\pgfqpoint{2.747786in}{1.490551in}}%
\pgfpathlineto{\pgfqpoint{2.738644in}{1.334082in}}%
\pgfpathlineto{\pgfqpoint{2.730580in}{1.157591in}}%
\pgfpathlineto{\pgfqpoint{2.723334in}{0.948663in}}%
\pgfpathlineto{\pgfqpoint{2.709783in}{0.530788in}}%
\pgfpathlineto{\pgfqpoint{2.705868in}{0.488716in}}%
\pgfpathlineto{\pgfqpoint{2.701769in}{0.464281in}}%
\pgfpathlineto{\pgfqpoint{2.697021in}{0.447744in}}%
\pgfpathlineto{\pgfqpoint{2.691859in}{0.436812in}}%
\pgfpathlineto{\pgfqpoint{2.686245in}{0.429229in}}%
\pgfpathlineto{\pgfqpoint{2.679348in}{0.423188in}}%
\pgfpathlineto{\pgfqpoint{2.669540in}{0.417856in}}%
\pgfpathlineto{\pgfqpoint{2.656987in}{0.413810in}}%
\pgfpathlineto{\pgfqpoint{2.637654in}{0.410337in}}%
\pgfpathlineto{\pgfqpoint{2.607297in}{0.407617in}}%
\pgfpathlineto{\pgfqpoint{2.555121in}{0.405574in}}%
\pgfpathlineto{\pgfqpoint{2.450714in}{0.404139in}}%
\pgfpathlineto{\pgfqpoint{2.176624in}{0.403275in}}%
\pgfpathlineto{\pgfqpoint{1.130290in}{0.402953in}}%
\pgfpathlineto{\pgfqpoint{0.516849in}{0.404175in}}%
\pgfpathlineto{\pgfqpoint{0.466848in}{0.405970in}}%
\pgfpathlineto{\pgfqpoint{0.456130in}{0.407931in}}%
\pgfpathlineto{\pgfqpoint{0.452340in}{0.410303in}}%
\pgfpathlineto{\pgfqpoint{0.450346in}{0.414662in}}%
\pgfpathlineto{\pgfqpoint{0.449266in}{0.424524in}}%
\pgfpathlineto{\pgfqpoint{0.448771in}{0.464344in}}%
\pgfpathlineto{\pgfqpoint{0.448640in}{0.850171in}}%
\pgfpathlineto{\pgfqpoint{0.448653in}{2.891318in}}%
\pgfpathlineto{\pgfqpoint{0.448653in}{2.891318in}}%
\pgfusepath{stroke}%
\end{pgfscope}%
\begin{pgfscope}%
\pgfpathrectangle{\pgfqpoint{0.448634in}{0.402556in}}{\pgfqpoint{4.350661in}{2.489204in}} %
\pgfusepath{clip}%
\pgfsetrectcap%
\pgfsetroundjoin%
\pgfsetlinewidth{1.003750pt}%
\definecolor{currentstroke}{rgb}{0.172549,0.627451,0.172549}%
\pgfsetstrokecolor{currentstroke}%
\pgfsetdash{}{0pt}%
\pgfpathmoveto{\pgfqpoint{3.428190in}{0.402586in}}%
\pgfpathlineto{\pgfqpoint{2.782122in}{0.403702in}}%
\pgfpathlineto{\pgfqpoint{2.753907in}{0.405674in}}%
\pgfpathlineto{\pgfqpoint{2.743329in}{0.408444in}}%
\pgfpathlineto{\pgfqpoint{2.737718in}{0.412188in}}%
\pgfpathlineto{\pgfqpoint{2.733668in}{0.417995in}}%
\pgfpathlineto{\pgfqpoint{2.730649in}{0.427308in}}%
\pgfpathlineto{\pgfqpoint{2.728388in}{0.442005in}}%
\pgfpathlineto{\pgfqpoint{2.726544in}{0.471795in}}%
\pgfpathlineto{\pgfqpoint{2.725216in}{0.534004in}}%
\pgfpathlineto{\pgfqpoint{2.725169in}{0.655973in}}%
\pgfpathlineto{\pgfqpoint{2.727377in}{0.832687in}}%
\pgfpathlineto{\pgfqpoint{2.732259in}{1.041703in}}%
\pgfpathlineto{\pgfqpoint{2.738851in}{1.223257in}}%
\pgfpathlineto{\pgfqpoint{2.747078in}{1.389766in}}%
\pgfpathlineto{\pgfqpoint{2.756608in}{1.538718in}}%
\pgfpathlineto{\pgfqpoint{2.768955in}{1.694887in}}%
\pgfpathlineto{\pgfqpoint{2.781228in}{1.816045in}}%
\pgfpathlineto{\pgfqpoint{2.794401in}{1.924525in}}%
\pgfpathlineto{\pgfqpoint{2.812737in}{2.054723in}}%
\pgfpathlineto{\pgfqpoint{2.828774in}{2.147512in}}%
\pgfpathlineto{\pgfqpoint{2.847382in}{2.242225in}}%
\pgfpathlineto{\pgfqpoint{2.895818in}{2.479700in}}%
\pgfpathlineto{\pgfqpoint{2.900204in}{2.516689in}}%
\pgfpathlineto{\pgfqpoint{2.901346in}{2.544030in}}%
\pgfpathlineto{\pgfqpoint{2.900292in}{2.566389in}}%
\pgfpathlineto{\pgfqpoint{2.897334in}{2.586000in}}%
\pgfpathlineto{\pgfqpoint{2.892836in}{2.602634in}}%
\pgfpathlineto{\pgfqpoint{2.886394in}{2.618406in}}%
\pgfpathlineto{\pgfqpoint{2.878058in}{2.632970in}}%
\pgfpathlineto{\pgfqpoint{2.868065in}{2.646101in}}%
\pgfpathlineto{\pgfqpoint{2.855050in}{2.659301in}}%
\pgfpathlineto{\pgfqpoint{2.840801in}{2.670717in}}%
\pgfpathlineto{\pgfqpoint{2.821822in}{2.682861in}}%
\pgfpathlineto{\pgfqpoint{2.799980in}{2.694026in}}%
\pgfpathlineto{\pgfqpoint{2.773366in}{2.704944in}}%
\pgfpathlineto{\pgfqpoint{2.742012in}{2.715266in}}%
\pgfpathlineto{\pgfqpoint{2.705983in}{2.724785in}}%
\pgfpathlineto{\pgfqpoint{2.663200in}{2.733811in}}%
\pgfpathlineto{\pgfqpoint{2.611535in}{2.742379in}}%
\pgfpathlineto{\pgfqpoint{2.551002in}{2.750090in}}%
\pgfpathlineto{\pgfqpoint{2.481632in}{2.756682in}}%
\pgfpathlineto{\pgfqpoint{2.399112in}{2.762200in}}%
\pgfpathlineto{\pgfqpoint{2.309985in}{2.765886in}}%
\pgfpathlineto{\pgfqpoint{2.188184in}{2.768097in}}%
\pgfpathlineto{\pgfqpoint{2.081595in}{2.767619in}}%
\pgfpathlineto{\pgfqpoint{1.968506in}{2.764840in}}%
\pgfpathlineto{\pgfqpoint{1.864180in}{2.759918in}}%
\pgfpathlineto{\pgfqpoint{1.757786in}{2.752593in}}%
\pgfpathlineto{\pgfqpoint{1.671087in}{2.744171in}}%
\pgfpathlineto{\pgfqpoint{1.591076in}{2.734193in}}%
\pgfpathlineto{\pgfqpoint{1.502689in}{2.720717in}}%
\pgfpathlineto{\pgfqpoint{1.427655in}{2.706083in}}%
\pgfpathlineto{\pgfqpoint{1.372350in}{2.692544in}}%
\pgfpathlineto{\pgfqpoint{1.321734in}{2.677921in}}%
\pgfpathlineto{\pgfqpoint{1.273765in}{2.661664in}}%
\pgfpathlineto{\pgfqpoint{1.230567in}{2.644672in}}%
\pgfpathlineto{\pgfqpoint{1.192197in}{2.627106in}}%
\pgfpathlineto{\pgfqpoint{1.156620in}{2.608403in}}%
\pgfpathlineto{\pgfqpoint{1.123890in}{2.588716in}}%
\pgfpathlineto{\pgfqpoint{1.095883in}{2.569568in}}%
\pgfpathlineto{\pgfqpoint{1.063936in}{2.543701in}}%
\pgfpathlineto{\pgfqpoint{1.038217in}{2.520732in}}%
\pgfpathlineto{\pgfqpoint{1.013766in}{2.496016in}}%
\pgfpathlineto{\pgfqpoint{0.990704in}{2.469610in}}%
\pgfpathlineto{\pgfqpoint{0.969124in}{2.441612in}}%
\pgfpathlineto{\pgfqpoint{0.949082in}{2.412154in}}%
\pgfpathlineto{\pgfqpoint{0.930604in}{2.381387in}}%
\pgfpathlineto{\pgfqpoint{0.906555in}{2.334052in}}%
\pgfpathlineto{\pgfqpoint{0.889925in}{2.296262in}}%
\pgfpathlineto{\pgfqpoint{0.874241in}{2.255213in}}%
\pgfpathlineto{\pgfqpoint{0.859667in}{2.210961in}}%
\pgfpathlineto{\pgfqpoint{0.846985in}{2.165954in}}%
\pgfpathlineto{\pgfqpoint{0.839633in}{2.134715in}}%
\pgfpathlineto{\pgfqpoint{0.828238in}{2.081532in}}%
\pgfpathlineto{\pgfqpoint{0.817866in}{2.022986in}}%
\pgfpathlineto{\pgfqpoint{0.810784in}{1.971352in}}%
\pgfpathlineto{\pgfqpoint{0.802845in}{1.902253in}}%
\pgfpathlineto{\pgfqpoint{0.796554in}{1.827928in}}%
\pgfpathlineto{\pgfqpoint{0.791696in}{1.743480in}}%
\pgfpathlineto{\pgfqpoint{0.787773in}{1.621595in}}%
\pgfpathlineto{\pgfqpoint{0.785407in}{1.522064in}}%
\pgfpathlineto{\pgfqpoint{0.785407in}{1.522064in}}%
\pgfusepath{stroke}%
\end{pgfscope}%
\begin{pgfscope}%
\pgfpathrectangle{\pgfqpoint{0.448634in}{0.402556in}}{\pgfqpoint{4.350661in}{2.489204in}} %
\pgfusepath{clip}%
\pgfsetrectcap%
\pgfsetroundjoin%
\pgfsetlinewidth{1.003750pt}%
\definecolor{currentstroke}{rgb}{0.172549,0.627451,0.172549}%
\pgfsetstrokecolor{currentstroke}%
\pgfsetdash{}{0pt}%
\pgfpathmoveto{\pgfqpoint{2.028735in}{0.425754in}}%
\pgfpathlineto{\pgfqpoint{1.878677in}{0.421879in}}%
\pgfpathlineto{\pgfqpoint{1.676387in}{0.418997in}}%
\pgfpathlineto{\pgfqpoint{1.413176in}{0.417558in}}%
\pgfpathlineto{\pgfqpoint{1.134735in}{0.418204in}}%
\pgfpathlineto{\pgfqpoint{0.921565in}{0.420769in}}%
\pgfpathlineto{\pgfqpoint{0.782384in}{0.424523in}}%
\pgfpathlineto{\pgfqpoint{0.693283in}{0.428974in}}%
\pgfpathlineto{\pgfqpoint{0.632541in}{0.434091in}}%
\pgfpathlineto{\pgfqpoint{0.591492in}{0.439564in}}%
\pgfpathlineto{\pgfqpoint{0.561503in}{0.445595in}}%
\pgfpathlineto{\pgfqpoint{0.538349in}{0.452466in}}%
\pgfpathlineto{\pgfqpoint{0.522042in}{0.459394in}}%
\pgfpathlineto{\pgfqpoint{0.508540in}{0.467420in}}%
\pgfpathlineto{\pgfqpoint{0.497973in}{0.476161in}}%
\pgfpathlineto{\pgfqpoint{0.488790in}{0.486750in}}%
\pgfpathlineto{\pgfqpoint{0.481284in}{0.498948in}}%
\pgfpathlineto{\pgfqpoint{0.474590in}{0.514580in}}%
\pgfpathlineto{\pgfqpoint{0.469106in}{0.533467in}}%
\pgfpathlineto{\pgfqpoint{0.464439in}{0.557771in}}%
\pgfpathlineto{\pgfqpoint{0.460297in}{0.592289in}}%
\pgfpathlineto{\pgfqpoint{0.456856in}{0.641912in}}%
\pgfpathlineto{\pgfqpoint{0.454122in}{0.716520in}}%
\pgfpathlineto{\pgfqpoint{0.451978in}{0.843444in}}%
\pgfpathlineto{\pgfqpoint{0.450459in}{1.087380in}}%
\pgfpathlineto{\pgfqpoint{0.449596in}{1.657406in}}%
\pgfpathlineto{\pgfqpoint{0.450150in}{2.687936in}}%
\pgfpathlineto{\pgfqpoint{0.451781in}{2.839761in}}%
\pgfpathlineto{\pgfqpoint{0.453975in}{2.872003in}}%
\pgfpathlineto{\pgfqpoint{0.456339in}{2.881553in}}%
\pgfpathlineto{\pgfqpoint{0.458888in}{2.885549in}}%
\pgfpathlineto{\pgfqpoint{0.462554in}{2.888171in}}%
\pgfpathlineto{\pgfqpoint{0.471046in}{2.890205in}}%
\pgfpathlineto{\pgfqpoint{0.490597in}{2.891263in}}%
\pgfpathlineto{\pgfqpoint{0.564556in}{2.891692in}}%
\pgfpathlineto{\pgfqpoint{1.569559in}{2.891759in}}%
\pgfpathlineto{\pgfqpoint{4.784679in}{2.890785in}}%
\pgfpathlineto{\pgfqpoint{4.791005in}{2.889098in}}%
\pgfpathlineto{\pgfqpoint{4.793910in}{2.885555in}}%
\pgfpathlineto{\pgfqpoint{4.795579in}{2.878366in}}%
\pgfpathlineto{\pgfqpoint{4.796850in}{2.858513in}}%
\pgfpathlineto{\pgfqpoint{4.796850in}{2.858513in}}%
\pgfusepath{stroke}%
\end{pgfscope}%
\begin{pgfscope}%
\pgfpathrectangle{\pgfqpoint{0.448634in}{0.402556in}}{\pgfqpoint{4.350661in}{2.489204in}} %
\pgfusepath{clip}%
\pgfsetrectcap%
\pgfsetroundjoin%
\pgfsetlinewidth{1.003750pt}%
\definecolor{currentstroke}{rgb}{0.839216,0.152941,0.156863}%
\pgfsetstrokecolor{currentstroke}%
\pgfsetdash{}{0pt}%
\pgfpathmoveto{\pgfqpoint{0.448634in}{2.896245in}}%
\pgfpathlineto{\pgfqpoint{0.448593in}{0.407043in}}%
\pgfpathlineto{\pgfqpoint{0.448593in}{0.407043in}}%
\pgfusepath{stroke}%
\end{pgfscope}%
\begin{pgfscope}%
\pgfpathrectangle{\pgfqpoint{0.448634in}{0.402556in}}{\pgfqpoint{4.350661in}{2.489204in}} %
\pgfusepath{clip}%
\pgfsetrectcap%
\pgfsetroundjoin%
\pgfsetlinewidth{1.003750pt}%
\definecolor{currentstroke}{rgb}{0.839216,0.152941,0.156863}%
\pgfsetstrokecolor{currentstroke}%
\pgfsetdash{}{0pt}%
\pgfpathmoveto{\pgfqpoint{0.576852in}{1.760819in}}%
\pgfpathlineto{\pgfqpoint{0.569393in}{1.840012in}}%
\pgfpathlineto{\pgfqpoint{0.563208in}{1.929341in}}%
\pgfpathlineto{\pgfqpoint{0.558592in}{2.028766in}}%
\pgfpathlineto{\pgfqpoint{0.555985in}{2.133267in}}%
\pgfpathlineto{\pgfqpoint{0.555565in}{2.237810in}}%
\pgfpathlineto{\pgfqpoint{0.557371in}{2.337354in}}%
\pgfpathlineto{\pgfqpoint{0.561095in}{2.424369in}}%
\pgfpathlineto{\pgfqpoint{0.566403in}{2.498793in}}%
\pgfpathlineto{\pgfqpoint{0.572908in}{2.560572in}}%
\pgfpathlineto{\pgfqpoint{0.580458in}{2.612121in}}%
\pgfpathlineto{\pgfqpoint{0.589086in}{2.655818in}}%
\pgfpathlineto{\pgfqpoint{0.598406in}{2.691592in}}%
\pgfpathlineto{\pgfqpoint{0.608613in}{2.721759in}}%
\pgfpathlineto{\pgfqpoint{0.619241in}{2.746280in}}%
\pgfpathlineto{\pgfqpoint{0.630817in}{2.767341in}}%
\pgfpathlineto{\pgfqpoint{0.642976in}{2.784886in}}%
\pgfpathlineto{\pgfqpoint{0.656814in}{2.800714in}}%
\pgfpathlineto{\pgfqpoint{0.672197in}{2.814550in}}%
\pgfpathlineto{\pgfqpoint{0.688854in}{2.826302in}}%
\pgfpathlineto{\pgfqpoint{0.706462in}{2.836077in}}%
\pgfpathlineto{\pgfqpoint{0.726805in}{2.844876in}}%
\pgfpathlineto{\pgfqpoint{0.751867in}{2.853203in}}%
\pgfpathlineto{\pgfqpoint{0.781633in}{2.860548in}}%
\pgfpathlineto{\pgfqpoint{0.818169in}{2.867054in}}%
\pgfpathlineto{\pgfqpoint{0.863582in}{2.872685in}}%
\pgfpathlineto{\pgfqpoint{0.922162in}{2.877518in}}%
\pgfpathlineto{\pgfqpoint{1.000392in}{2.881567in}}%
\pgfpathlineto{\pgfqpoint{1.111295in}{2.884881in}}%
\pgfpathlineto{\pgfqpoint{1.274429in}{2.887367in}}%
\pgfpathlineto{\pgfqpoint{1.552866in}{2.889263in}}%
\pgfpathlineto{\pgfqpoint{2.107575in}{2.890457in}}%
\pgfpathlineto{\pgfqpoint{3.343162in}{2.890573in}}%
\pgfpathlineto{\pgfqpoint{4.043617in}{2.888941in}}%
\pgfpathlineto{\pgfqpoint{4.289418in}{2.886404in}}%
\pgfpathlineto{\pgfqpoint{4.413376in}{2.883093in}}%
\pgfpathlineto{\pgfqpoint{4.489426in}{2.878997in}}%
\pgfpathlineto{\pgfqpoint{4.541453in}{2.874081in}}%
\pgfpathlineto{\pgfqpoint{4.578102in}{2.868470in}}%
\pgfpathlineto{\pgfqpoint{4.605820in}{2.862092in}}%
\pgfpathlineto{\pgfqpoint{4.626727in}{2.855245in}}%
\pgfpathlineto{\pgfqpoint{4.644927in}{2.847018in}}%
\pgfpathlineto{\pgfqpoint{4.660242in}{2.837589in}}%
\pgfpathlineto{\pgfqpoint{4.672624in}{2.827468in}}%
\pgfpathlineto{\pgfqpoint{4.683752in}{2.815592in}}%
\pgfpathlineto{\pgfqpoint{4.693407in}{2.802134in}}%
\pgfpathlineto{\pgfqpoint{4.702741in}{2.785342in}}%
\pgfpathlineto{\pgfqpoint{4.711278in}{2.765193in}}%
\pgfpathlineto{\pgfqpoint{4.719483in}{2.739483in}}%
\pgfpathlineto{\pgfqpoint{4.726294in}{2.710657in}}%
\pgfpathlineto{\pgfqpoint{4.733260in}{2.671642in}}%
\pgfpathlineto{\pgfqpoint{4.739604in}{2.622395in}}%
\pgfpathlineto{\pgfqpoint{4.745236in}{2.560503in}}%
\pgfpathlineto{\pgfqpoint{4.750164in}{2.481051in}}%
\pgfpathlineto{\pgfqpoint{4.754367in}{2.376617in}}%
\pgfpathlineto{\pgfqpoint{4.757443in}{2.242248in}}%
\pgfpathlineto{\pgfqpoint{4.758977in}{2.075482in}}%
\pgfpathlineto{\pgfqpoint{4.758447in}{1.888794in}}%
\pgfpathlineto{\pgfqpoint{4.755756in}{1.707110in}}%
\pgfpathlineto{\pgfqpoint{4.750925in}{1.532956in}}%
\pgfpathlineto{\pgfqpoint{4.744786in}{1.398726in}}%
\pgfpathlineto{\pgfqpoint{4.737575in}{1.289515in}}%
\pgfpathlineto{\pgfqpoint{4.728714in}{1.190469in}}%
\pgfpathlineto{\pgfqpoint{4.719653in}{1.116520in}}%
\pgfpathlineto{\pgfqpoint{4.710036in}{1.055275in}}%
\pgfpathlineto{\pgfqpoint{4.699504in}{1.001860in}}%
\pgfpathlineto{\pgfqpoint{4.689040in}{0.958689in}}%
\pgfpathlineto{\pgfqpoint{4.677220in}{0.918599in}}%
\pgfpathlineto{\pgfqpoint{4.664034in}{0.881748in}}%
\pgfpathlineto{\pgfqpoint{4.650584in}{0.850491in}}%
\pgfpathlineto{\pgfqpoint{4.636303in}{0.822569in}}%
\pgfpathlineto{\pgfqpoint{4.620207in}{0.795974in}}%
\pgfpathlineto{\pgfqpoint{4.603640in}{0.772900in}}%
\pgfpathlineto{\pgfqpoint{4.585488in}{0.751445in}}%
\pgfpathlineto{\pgfqpoint{4.565874in}{0.731748in}}%
\pgfpathlineto{\pgfqpoint{4.544964in}{0.713878in}}%
\pgfpathlineto{\pgfqpoint{4.522958in}{0.697823in}}%
\pgfpathlineto{\pgfqpoint{4.496157in}{0.681289in}}%
\pgfpathlineto{\pgfqpoint{4.470397in}{0.667952in}}%
\pgfpathlineto{\pgfqpoint{4.439961in}{0.654509in}}%
\pgfpathlineto{\pgfqpoint{4.406841in}{0.642281in}}%
\pgfpathlineto{\pgfqpoint{4.369009in}{0.630748in}}%
\pgfpathlineto{\pgfqpoint{4.326489in}{0.620226in}}%
\pgfpathlineto{\pgfqpoint{4.279327in}{0.610948in}}%
\pgfpathlineto{\pgfqpoint{4.227576in}{0.603084in}}%
\pgfpathlineto{\pgfqpoint{4.173450in}{0.597062in}}%
\pgfpathlineto{\pgfqpoint{4.110511in}{0.592202in}}%
\pgfpathlineto{\pgfqpoint{4.047471in}{0.589536in}}%
\pgfpathlineto{\pgfqpoint{3.977867in}{0.588623in}}%
\pgfpathlineto{\pgfqpoint{3.906093in}{0.589933in}}%
\pgfpathlineto{\pgfqpoint{3.834377in}{0.593496in}}%
\pgfpathlineto{\pgfqpoint{3.767120in}{0.599067in}}%
\pgfpathlineto{\pgfqpoint{3.704364in}{0.606392in}}%
\pgfpathlineto{\pgfqpoint{3.678516in}{0.610510in}}%
\pgfpathlineto{\pgfqpoint{3.620438in}{0.620500in}}%
\pgfpathlineto{\pgfqpoint{3.586319in}{0.628207in}}%
\pgfpathlineto{\pgfqpoint{3.495241in}{0.652428in}}%
\pgfpathlineto{\pgfqpoint{3.451528in}{0.667583in}}%
\pgfpathlineto{\pgfqpoint{3.408538in}{0.685220in}}%
\pgfpathlineto{\pgfqpoint{3.374594in}{0.702001in}}%
\pgfpathlineto{\pgfqpoint{3.345407in}{0.718682in}}%
\pgfpathlineto{\pgfqpoint{3.315236in}{0.738520in}}%
\pgfpathlineto{\pgfqpoint{3.288127in}{0.759290in}}%
\pgfpathlineto{\pgfqpoint{3.264004in}{0.780550in}}%
\pgfpathlineto{\pgfqpoint{3.241208in}{0.803648in}}%
\pgfpathlineto{\pgfqpoint{3.219894in}{0.828529in}}%
\pgfpathlineto{\pgfqpoint{3.200189in}{0.855091in}}%
\pgfpathlineto{\pgfqpoint{3.182177in}{0.883182in}}%
\pgfpathlineto{\pgfqpoint{3.165906in}{0.912633in}}%
\pgfpathlineto{\pgfqpoint{3.150351in}{0.945448in}}%
\pgfpathlineto{\pgfqpoint{3.136682in}{0.979345in}}%
\pgfpathlineto{\pgfqpoint{3.124073in}{1.016460in}}%
\pgfpathlineto{\pgfqpoint{3.112834in}{1.056769in}}%
\pgfpathlineto{\pgfqpoint{3.103046in}{1.100146in}}%
\pgfpathlineto{\pgfqpoint{3.095343in}{1.144071in}}%
\pgfpathlineto{\pgfqpoint{3.089208in}{1.190837in}}%
\pgfpathlineto{\pgfqpoint{3.084595in}{1.242838in}}%
\pgfpathlineto{\pgfqpoint{3.082137in}{1.295031in}}%
\pgfpathlineto{\pgfqpoint{3.081687in}{1.349787in}}%
\pgfpathlineto{\pgfqpoint{3.083451in}{1.406998in}}%
\pgfpathlineto{\pgfqpoint{3.087181in}{1.461589in}}%
\pgfpathlineto{\pgfqpoint{3.093485in}{1.520887in}}%
\pgfpathlineto{\pgfqpoint{3.101823in}{1.577334in}}%
\pgfpathlineto{\pgfqpoint{3.111930in}{1.630856in}}%
\pgfpathlineto{\pgfqpoint{3.124690in}{1.686208in}}%
\pgfpathlineto{\pgfqpoint{3.139178in}{1.738395in}}%
\pgfpathlineto{\pgfqpoint{3.155145in}{1.787366in}}%
\pgfpathlineto{\pgfqpoint{3.172353in}{1.833084in}}%
\pgfpathlineto{\pgfqpoint{3.191618in}{1.877716in}}%
\pgfpathlineto{\pgfqpoint{3.214026in}{1.923261in}}%
\pgfpathlineto{\pgfqpoint{3.236214in}{1.963157in}}%
\pgfpathlineto{\pgfqpoint{3.260178in}{2.001684in}}%
\pgfpathlineto{\pgfqpoint{3.285813in}{2.038776in}}%
\pgfpathlineto{\pgfqpoint{3.314415in}{2.076285in}}%
\pgfpathlineto{\pgfqpoint{3.348944in}{2.117711in}}%
\pgfpathlineto{\pgfqpoint{3.417133in}{2.198022in}}%
\pgfpathlineto{\pgfqpoint{3.426053in}{2.212128in}}%
\pgfpathlineto{\pgfqpoint{3.430798in}{2.223297in}}%
\pgfpathlineto{\pgfqpoint{3.432034in}{2.230603in}}%
\pgfpathlineto{\pgfqpoint{3.430773in}{2.237856in}}%
\pgfpathlineto{\pgfqpoint{3.426621in}{2.243526in}}%
\pgfpathlineto{\pgfqpoint{3.420908in}{2.247084in}}%
\pgfpathlineto{\pgfqpoint{3.412501in}{2.249583in}}%
\pgfpathlineto{\pgfqpoint{3.399499in}{2.250689in}}%
\pgfpathlineto{\pgfqpoint{3.384305in}{2.249671in}}%
\pgfpathlineto{\pgfqpoint{3.364985in}{2.246098in}}%
\pgfpathlineto{\pgfqpoint{3.341804in}{2.239342in}}%
\pgfpathlineto{\pgfqpoint{3.317109in}{2.229682in}}%
\pgfpathlineto{\pgfqpoint{3.291104in}{2.216986in}}%
\pgfpathlineto{\pgfqpoint{3.265928in}{2.202261in}}%
\pgfpathlineto{\pgfqpoint{3.239805in}{2.184361in}}%
\pgfpathlineto{\pgfqpoint{3.214775in}{2.164519in}}%
\pgfpathlineto{\pgfqpoint{3.190900in}{2.142893in}}%
\pgfpathlineto{\pgfqpoint{3.166657in}{2.117912in}}%
\pgfpathlineto{\pgfqpoint{3.143835in}{2.091233in}}%
\pgfpathlineto{\pgfqpoint{3.121079in}{2.061107in}}%
\pgfpathlineto{\pgfqpoint{3.099952in}{2.029463in}}%
\pgfpathlineto{\pgfqpoint{3.079251in}{1.994406in}}%
\pgfpathlineto{\pgfqpoint{3.059218in}{1.955915in}}%
\pgfpathlineto{\pgfqpoint{3.040058in}{1.914015in}}%
\pgfpathlineto{\pgfqpoint{3.022809in}{1.871041in}}%
\pgfpathlineto{\pgfqpoint{3.005790in}{1.822536in}}%
\pgfpathlineto{\pgfqpoint{2.990067in}{1.770819in}}%
\pgfpathlineto{\pgfqpoint{2.975708in}{1.715979in}}%
\pgfpathlineto{\pgfqpoint{2.962284in}{1.655680in}}%
\pgfpathlineto{\pgfqpoint{2.950496in}{1.592386in}}%
\pgfpathlineto{\pgfqpoint{2.940383in}{1.526185in}}%
\pgfpathlineto{\pgfqpoint{2.931745in}{1.454681in}}%
\pgfpathlineto{\pgfqpoint{2.925082in}{1.380399in}}%
\pgfpathlineto{\pgfqpoint{2.920647in}{1.305899in}}%
\pgfpathlineto{\pgfqpoint{2.918444in}{1.231270in}}%
\pgfpathlineto{\pgfqpoint{2.918545in}{1.159087in}}%
\pgfpathlineto{\pgfqpoint{2.920787in}{1.091931in}}%
\pgfpathlineto{\pgfqpoint{2.925177in}{1.027412in}}%
\pgfpathlineto{\pgfqpoint{2.931192in}{0.970580in}}%
\pgfpathlineto{\pgfqpoint{2.938760in}{0.919034in}}%
\pgfpathlineto{\pgfqpoint{2.947651in}{0.872852in}}%
\pgfpathlineto{\pgfqpoint{2.958213in}{0.829714in}}%
\pgfpathlineto{\pgfqpoint{2.969670in}{0.792114in}}%
\pgfpathlineto{\pgfqpoint{2.982463in}{0.757773in}}%
\pgfpathlineto{\pgfqpoint{2.996425in}{0.726812in}}%
\pgfpathlineto{\pgfqpoint{3.011299in}{0.699300in}}%
\pgfpathlineto{\pgfqpoint{3.026739in}{0.675225in}}%
\pgfpathlineto{\pgfqpoint{3.043828in}{0.652656in}}%
\pgfpathlineto{\pgfqpoint{3.062495in}{0.631788in}}%
\pgfpathlineto{\pgfqpoint{3.082602in}{0.612753in}}%
\pgfpathlineto{\pgfqpoint{3.103961in}{0.595592in}}%
\pgfpathlineto{\pgfqpoint{3.128268in}{0.579069in}}%
\pgfpathlineto{\pgfqpoint{3.153537in}{0.564554in}}%
\pgfpathlineto{\pgfqpoint{3.181571in}{0.550952in}}%
\pgfpathlineto{\pgfqpoint{3.214371in}{0.537647in}}%
\pgfpathlineto{\pgfqpoint{3.249846in}{0.525712in}}%
\pgfpathlineto{\pgfqpoint{3.290011in}{0.514571in}}%
\pgfpathlineto{\pgfqpoint{3.334820in}{0.504423in}}%
\pgfpathlineto{\pgfqpoint{3.386372in}{0.494999in}}%
\pgfpathlineto{\pgfqpoint{3.446798in}{0.486257in}}%
\pgfpathlineto{\pgfqpoint{3.518243in}{0.478282in}}%
\pgfpathlineto{\pgfqpoint{3.600685in}{0.471409in}}%
\pgfpathlineto{\pgfqpoint{3.696268in}{0.465713in}}%
\pgfpathlineto{\pgfqpoint{3.807144in}{0.461369in}}%
\pgfpathlineto{\pgfqpoint{3.933291in}{0.458719in}}%
\pgfpathlineto{\pgfqpoint{4.063808in}{0.458211in}}%
\pgfpathlineto{\pgfqpoint{4.187792in}{0.459914in}}%
\pgfpathlineto{\pgfqpoint{4.294335in}{0.463521in}}%
\pgfpathlineto{\pgfqpoint{4.381234in}{0.468574in}}%
\pgfpathlineto{\pgfqpoint{4.450636in}{0.474701in}}%
\pgfpathlineto{\pgfqpoint{4.506850in}{0.481799in}}%
\pgfpathlineto{\pgfqpoint{4.552009in}{0.489658in}}%
\pgfpathlineto{\pgfqpoint{4.588239in}{0.498115in}}%
\pgfpathlineto{\pgfqpoint{4.617656in}{0.507110in}}%
\pgfpathlineto{\pgfqpoint{4.642328in}{0.516843in}}%
\pgfpathlineto{\pgfqpoint{4.664194in}{0.527940in}}%
\pgfpathlineto{\pgfqpoint{4.681238in}{0.538945in}}%
\pgfpathlineto{\pgfqpoint{4.697164in}{0.551953in}}%
\pgfpathlineto{\pgfqpoint{4.710076in}{0.565289in}}%
\pgfpathlineto{\pgfqpoint{4.721578in}{0.580218in}}%
\pgfpathlineto{\pgfqpoint{4.731557in}{0.596521in}}%
\pgfpathlineto{\pgfqpoint{4.741000in}{0.616134in}}%
\pgfpathlineto{\pgfqpoint{4.749521in}{0.639027in}}%
\pgfpathlineto{\pgfqpoint{4.757522in}{0.667450in}}%
\pgfpathlineto{\pgfqpoint{4.764572in}{0.701345in}}%
\pgfpathlineto{\pgfqpoint{4.770840in}{0.743043in}}%
\pgfpathlineto{\pgfqpoint{4.776327in}{0.794934in}}%
\pgfpathlineto{\pgfqpoint{4.781278in}{0.864398in}}%
\pgfpathlineto{\pgfqpoint{4.785468in}{0.956371in}}%
\pgfpathlineto{\pgfqpoint{4.789000in}{1.085745in}}%
\pgfpathlineto{\pgfqpoint{4.791852in}{1.277385in}}%
\pgfpathlineto{\pgfqpoint{4.793959in}{1.581057in}}%
\pgfpathlineto{\pgfqpoint{4.794962in}{2.071429in}}%
\pgfpathlineto{\pgfqpoint{4.793967in}{2.559311in}}%
\pgfpathlineto{\pgfqpoint{4.791733in}{2.745981in}}%
\pgfpathlineto{\pgfqpoint{4.788955in}{2.818091in}}%
\pgfpathlineto{\pgfqpoint{4.785731in}{2.850227in}}%
\pgfpathlineto{\pgfqpoint{4.781879in}{2.867057in}}%
\pgfpathlineto{\pgfqpoint{4.777744in}{2.875780in}}%
\pgfpathlineto{\pgfqpoint{4.773097in}{2.880982in}}%
\pgfpathlineto{\pgfqpoint{4.767363in}{2.884504in}}%
\pgfpathlineto{\pgfqpoint{4.756853in}{2.887622in}}%
\pgfpathlineto{\pgfqpoint{4.739548in}{2.889639in}}%
\pgfpathlineto{\pgfqpoint{4.704762in}{2.890882in}}%
\pgfpathlineto{\pgfqpoint{4.602524in}{2.891538in}}%
\pgfpathlineto{\pgfqpoint{3.952100in}{2.891742in}}%
\pgfpathlineto{\pgfqpoint{0.617321in}{2.890753in}}%
\pgfpathlineto{\pgfqpoint{0.549910in}{2.888858in}}%
\pgfpathlineto{\pgfqpoint{0.521735in}{2.886179in}}%
\pgfpathlineto{\pgfqpoint{0.504666in}{2.882389in}}%
\pgfpathlineto{\pgfqpoint{0.494501in}{2.878011in}}%
\pgfpathlineto{\pgfqpoint{0.487180in}{2.872667in}}%
\pgfpathlineto{\pgfqpoint{0.481152in}{2.865519in}}%
\pgfpathlineto{\pgfqpoint{0.475664in}{2.854804in}}%
\pgfpathlineto{\pgfqpoint{0.471318in}{2.840737in}}%
\pgfpathlineto{\pgfqpoint{0.467301in}{2.818823in}}%
\pgfpathlineto{\pgfqpoint{0.463927in}{2.786700in}}%
\pgfpathlineto{\pgfqpoint{0.460918in}{2.734544in}}%
\pgfpathlineto{\pgfqpoint{0.458363in}{2.647474in}}%
\pgfpathlineto{\pgfqpoint{0.456575in}{2.523031in}}%
\pgfpathlineto{\pgfqpoint{0.456575in}{2.523031in}}%
\pgfusepath{stroke}%
\end{pgfscope}%
\begin{pgfscope}%
\pgfpathrectangle{\pgfqpoint{0.448634in}{0.402556in}}{\pgfqpoint{4.350661in}{2.489204in}} %
\pgfusepath{clip}%
\pgfsetrectcap%
\pgfsetroundjoin%
\pgfsetlinewidth{1.003750pt}%
\definecolor{currentstroke}{rgb}{0.839216,0.152941,0.156863}%
\pgfsetstrokecolor{currentstroke}%
\pgfsetdash{}{0pt}%
\pgfpathmoveto{\pgfqpoint{4.798840in}{2.852369in}}%
\pgfpathlineto{\pgfqpoint{4.797564in}{2.889610in}}%
\pgfpathlineto{\pgfqpoint{4.796215in}{2.891483in}}%
\pgfpathlineto{\pgfqpoint{4.787551in}{2.891760in}}%
\pgfpathlineto{\pgfqpoint{0.452128in}{2.891659in}}%
\pgfpathlineto{\pgfqpoint{0.450530in}{2.890082in}}%
\pgfpathlineto{\pgfqpoint{0.449454in}{2.882763in}}%
\pgfpathlineto{\pgfqpoint{0.448970in}{2.845432in}}%
\pgfpathlineto{\pgfqpoint{0.448743in}{2.494454in}}%
\pgfpathlineto{\pgfqpoint{0.449624in}{0.615107in}}%
\pgfpathlineto{\pgfqpoint{0.451433in}{0.510586in}}%
\pgfpathlineto{\pgfqpoint{0.453993in}{0.473374in}}%
\pgfpathlineto{\pgfqpoint{0.457406in}{0.453868in}}%
\pgfpathlineto{\pgfqpoint{0.461540in}{0.442384in}}%
\pgfpathlineto{\pgfqpoint{0.466739in}{0.434437in}}%
\pgfpathlineto{\pgfqpoint{0.473595in}{0.428350in}}%
\pgfpathlineto{\pgfqpoint{0.483492in}{0.423244in}}%
\pgfpathlineto{\pgfqpoint{0.491854in}{0.420501in}}%
\pgfpathlineto{\pgfqpoint{0.491854in}{0.420501in}}%
\pgfusepath{stroke}%
\end{pgfscope}%
\begin{pgfscope}%
\pgfpathrectangle{\pgfqpoint{0.448634in}{0.402556in}}{\pgfqpoint{4.350661in}{2.489204in}} %
\pgfusepath{clip}%
\pgfsetrectcap%
\pgfsetroundjoin%
\pgfsetlinewidth{1.003750pt}%
\definecolor{currentstroke}{rgb}{0.839216,0.152941,0.156863}%
\pgfsetstrokecolor{currentstroke}%
\pgfsetdash{}{0pt}%
\pgfpathmoveto{\pgfqpoint{0.456424in}{1.370137in}}%
\pgfpathlineto{\pgfqpoint{0.459610in}{1.118755in}}%
\pgfpathlineto{\pgfqpoint{0.463695in}{0.962007in}}%
\pgfpathlineto{\pgfqpoint{0.468519in}{0.857610in}}%
\pgfpathlineto{\pgfqpoint{0.474082in}{0.783210in}}%
\pgfpathlineto{\pgfqpoint{0.480226in}{0.728906in}}%
\pgfpathlineto{\pgfqpoint{0.486970in}{0.687306in}}%
\pgfpathlineto{\pgfqpoint{0.494537in}{0.653558in}}%
\pgfpathlineto{\pgfqpoint{0.503107in}{0.625355in}}%
\pgfpathlineto{\pgfqpoint{0.512193in}{0.602749in}}%
\pgfpathlineto{\pgfqpoint{0.522200in}{0.583508in}}%
\pgfpathlineto{\pgfqpoint{0.534108in}{0.565743in}}%
\pgfpathlineto{\pgfqpoint{0.546264in}{0.551507in}}%
\pgfpathlineto{\pgfqpoint{0.559728in}{0.538907in}}%
\pgfpathlineto{\pgfqpoint{0.576130in}{0.526693in}}%
\pgfpathlineto{\pgfqpoint{0.595483in}{0.515351in}}%
\pgfpathlineto{\pgfqpoint{0.617681in}{0.505147in}}%
\pgfpathlineto{\pgfqpoint{0.642568in}{0.496153in}}%
\pgfpathlineto{\pgfqpoint{0.672126in}{0.487778in}}%
\pgfpathlineto{\pgfqpoint{0.708443in}{0.479824in}}%
\pgfpathlineto{\pgfqpoint{0.753650in}{0.472325in}}%
\pgfpathlineto{\pgfqpoint{0.807718in}{0.465660in}}%
\pgfpathlineto{\pgfqpoint{0.877116in}{0.459475in}}%
\pgfpathlineto{\pgfqpoint{0.961828in}{0.454230in}}%
\pgfpathlineto{\pgfqpoint{1.068352in}{0.449916in}}%
\pgfpathlineto{\pgfqpoint{1.201018in}{0.446839in}}%
\pgfpathlineto{\pgfqpoint{1.357637in}{0.445481in}}%
\pgfpathlineto{\pgfqpoint{1.525135in}{0.446232in}}%
\pgfpathlineto{\pgfqpoint{1.686088in}{0.449142in}}%
\pgfpathlineto{\pgfqpoint{1.823074in}{0.453747in}}%
\pgfpathlineto{\pgfqpoint{1.938245in}{0.459764in}}%
\pgfpathlineto{\pgfqpoint{2.031582in}{0.466759in}}%
\pgfpathlineto{\pgfqpoint{2.109580in}{0.474745in}}%
\pgfpathlineto{\pgfqpoint{2.174384in}{0.483535in}}%
\pgfpathlineto{\pgfqpoint{2.228139in}{0.492940in}}%
\pgfpathlineto{\pgfqpoint{2.275119in}{0.503356in}}%
\pgfpathlineto{\pgfqpoint{2.315282in}{0.514501in}}%
\pgfpathlineto{\pgfqpoint{2.350698in}{0.526659in}}%
\pgfpathlineto{\pgfqpoint{2.381321in}{0.539536in}}%
\pgfpathlineto{\pgfqpoint{2.407164in}{0.552659in}}%
\pgfpathlineto{\pgfqpoint{2.430226in}{0.566639in}}%
\pgfpathlineto{\pgfqpoint{2.452282in}{0.582602in}}%
\pgfpathlineto{\pgfqpoint{2.471391in}{0.599069in}}%
\pgfpathlineto{\pgfqpoint{2.489240in}{0.617293in}}%
\pgfpathlineto{\pgfqpoint{2.505678in}{0.637180in}}%
\pgfpathlineto{\pgfqpoint{2.520620in}{0.658557in}}%
\pgfpathlineto{\pgfqpoint{2.535214in}{0.683314in}}%
\pgfpathlineto{\pgfqpoint{2.549115in}{0.711484in}}%
\pgfpathlineto{\pgfqpoint{2.562091in}{0.743004in}}%
\pgfpathlineto{\pgfqpoint{2.574020in}{0.777751in}}%
\pgfpathlineto{\pgfqpoint{2.585502in}{0.817970in}}%
\pgfpathlineto{\pgfqpoint{2.596809in}{0.866038in}}%
\pgfpathlineto{\pgfqpoint{2.607562in}{0.921948in}}%
\pgfpathlineto{\pgfqpoint{2.617925in}{0.988098in}}%
\pgfpathlineto{\pgfqpoint{2.627958in}{1.066918in}}%
\pgfpathlineto{\pgfqpoint{2.637941in}{1.163320in}}%
\pgfpathlineto{\pgfqpoint{2.648424in}{1.287199in}}%
\pgfpathlineto{\pgfqpoint{2.660103in}{1.453438in}}%
\pgfpathlineto{\pgfqpoint{2.674773in}{1.696801in}}%
\pgfpathlineto{\pgfqpoint{2.687716in}{1.945279in}}%
\pgfpathlineto{\pgfqpoint{2.692670in}{2.079574in}}%
\pgfpathlineto{\pgfqpoint{2.693829in}{2.166682in}}%
\pgfpathlineto{\pgfqpoint{2.692565in}{2.233870in}}%
\pgfpathlineto{\pgfqpoint{2.689436in}{2.286015in}}%
\pgfpathlineto{\pgfqpoint{2.684859in}{2.327999in}}%
\pgfpathlineto{\pgfqpoint{2.678725in}{2.364664in}}%
\pgfpathlineto{\pgfqpoint{2.671356in}{2.395897in}}%
\pgfpathlineto{\pgfqpoint{2.662489in}{2.423981in}}%
\pgfpathlineto{\pgfqpoint{2.652361in}{2.448778in}}%
\pgfpathlineto{\pgfqpoint{2.641365in}{2.470246in}}%
\pgfpathlineto{\pgfqpoint{2.628643in}{2.490425in}}%
\pgfpathlineto{\pgfqpoint{2.614278in}{2.509106in}}%
\pgfpathlineto{\pgfqpoint{2.598443in}{2.526159in}}%
\pgfpathlineto{\pgfqpoint{2.579590in}{2.543005in}}%
\pgfpathlineto{\pgfqpoint{2.559532in}{2.557923in}}%
\pgfpathlineto{\pgfqpoint{2.536602in}{2.572183in}}%
\pgfpathlineto{\pgfqpoint{2.510849in}{2.585538in}}%
\pgfpathlineto{\pgfqpoint{2.482360in}{2.597838in}}%
\pgfpathlineto{\pgfqpoint{2.449134in}{2.609683in}}%
\pgfpathlineto{\pgfqpoint{2.411184in}{2.620696in}}%
\pgfpathlineto{\pgfqpoint{2.368552in}{2.630606in}}%
\pgfpathlineto{\pgfqpoint{2.321294in}{2.639221in}}%
\pgfpathlineto{\pgfqpoint{2.269467in}{2.646399in}}%
\pgfpathlineto{\pgfqpoint{2.210954in}{2.652193in}}%
\pgfpathlineto{\pgfqpoint{2.147967in}{2.656153in}}%
\pgfpathlineto{\pgfqpoint{2.080556in}{2.658135in}}%
\pgfpathlineto{\pgfqpoint{2.010948in}{2.657971in}}%
\pgfpathlineto{\pgfqpoint{1.939195in}{2.655572in}}%
\pgfpathlineto{\pgfqpoint{1.867527in}{2.650913in}}%
\pgfpathlineto{\pgfqpoint{1.798171in}{2.644140in}}%
\pgfpathlineto{\pgfqpoint{1.733341in}{2.635606in}}%
\pgfpathlineto{\pgfqpoint{1.673075in}{2.625521in}}%
\pgfpathlineto{\pgfqpoint{1.615274in}{2.613610in}}%
\pgfpathlineto{\pgfqpoint{1.562133in}{2.600402in}}%
\pgfpathlineto{\pgfqpoint{1.513681in}{2.586139in}}%
\pgfpathlineto{\pgfqpoint{1.467862in}{2.570344in}}%
\pgfpathlineto{\pgfqpoint{1.426793in}{2.553923in}}%
\pgfpathlineto{\pgfqpoint{1.388447in}{2.536289in}}%
\pgfpathlineto{\pgfqpoint{1.352878in}{2.517566in}}%
\pgfpathlineto{\pgfqpoint{1.320128in}{2.497922in}}%
\pgfpathlineto{\pgfqpoint{1.288379in}{2.476236in}}%
\pgfpathlineto{\pgfqpoint{1.259592in}{2.453861in}}%
\pgfpathlineto{\pgfqpoint{1.232050in}{2.429520in}}%
\pgfpathlineto{\pgfqpoint{1.207527in}{2.404898in}}%
\pgfpathlineto{\pgfqpoint{1.184409in}{2.378557in}}%
\pgfpathlineto{\pgfqpoint{1.162828in}{2.350561in}}%
\pgfpathlineto{\pgfqpoint{1.142891in}{2.321011in}}%
\pgfpathlineto{\pgfqpoint{1.124675in}{2.290041in}}%
\pgfpathlineto{\pgfqpoint{1.108225in}{2.257802in}}%
\pgfpathlineto{\pgfqpoint{1.092639in}{2.222199in}}%
\pgfpathlineto{\pgfqpoint{1.079059in}{2.185535in}}%
\pgfpathlineto{\pgfqpoint{1.067443in}{2.147997in}}%
\pgfpathlineto{\pgfqpoint{1.057187in}{2.107347in}}%
\pgfpathlineto{\pgfqpoint{1.049004in}{2.066086in}}%
\pgfpathlineto{\pgfqpoint{1.042513in}{2.021906in}}%
\pgfpathlineto{\pgfqpoint{1.038177in}{1.977382in}}%
\pgfpathlineto{\pgfqpoint{1.035866in}{1.930167in}}%
\pgfpathlineto{\pgfqpoint{1.035826in}{1.882878in}}%
\pgfpathlineto{\pgfqpoint{1.038031in}{1.835656in}}%
\pgfpathlineto{\pgfqpoint{1.042474in}{1.788641in}}%
\pgfpathlineto{\pgfqpoint{1.049176in}{1.741978in}}%
\pgfpathlineto{\pgfqpoint{1.057644in}{1.698239in}}%
\pgfpathlineto{\pgfqpoint{1.068221in}{1.655105in}}%
\pgfpathlineto{\pgfqpoint{1.080962in}{1.612745in}}%
\pgfpathlineto{\pgfqpoint{1.095031in}{1.573616in}}%
\pgfpathlineto{\pgfqpoint{1.111115in}{1.535520in}}%
\pgfpathlineto{\pgfqpoint{1.128118in}{1.500775in}}%
\pgfpathlineto{\pgfqpoint{1.146930in}{1.467274in}}%
\pgfpathlineto{\pgfqpoint{1.167531in}{1.435180in}}%
\pgfpathlineto{\pgfqpoint{1.189874in}{1.404652in}}%
\pgfpathlineto{\pgfqpoint{1.213884in}{1.375828in}}%
\pgfpathlineto{\pgfqpoint{1.237817in}{1.350457in}}%
\pgfpathlineto{\pgfqpoint{1.264748in}{1.325237in}}%
\pgfpathlineto{\pgfqpoint{1.292991in}{1.301972in}}%
\pgfpathlineto{\pgfqpoint{1.322398in}{1.280678in}}%
\pgfpathlineto{\pgfqpoint{1.352820in}{1.261340in}}%
\pgfpathlineto{\pgfqpoint{1.386095in}{1.242889in}}%
\pgfpathlineto{\pgfqpoint{1.420191in}{1.226516in}}%
\pgfpathlineto{\pgfqpoint{1.457024in}{1.211329in}}%
\pgfpathlineto{\pgfqpoint{1.496554in}{1.197536in}}%
\pgfpathlineto{\pgfqpoint{1.538720in}{1.185287in}}%
\pgfpathlineto{\pgfqpoint{1.583441in}{1.174641in}}%
\pgfpathlineto{\pgfqpoint{1.634929in}{1.164775in}}%
\pgfpathlineto{\pgfqpoint{1.706063in}{1.153745in}}%
\pgfpathlineto{\pgfqpoint{1.768493in}{1.143417in}}%
\pgfpathlineto{\pgfqpoint{1.796122in}{1.136567in}}%
\pgfpathlineto{\pgfqpoint{1.812683in}{1.130481in}}%
\pgfpathlineto{\pgfqpoint{1.824471in}{1.124102in}}%
\pgfpathlineto{\pgfqpoint{1.833210in}{1.116741in}}%
\pgfpathlineto{\pgfqpoint{1.838498in}{1.108890in}}%
\pgfpathlineto{\pgfqpoint{1.840589in}{1.101849in}}%
\pgfpathlineto{\pgfqpoint{1.840619in}{1.094412in}}%
\pgfpathlineto{\pgfqpoint{1.837931in}{1.084986in}}%
\pgfpathlineto{\pgfqpoint{1.833246in}{1.076615in}}%
\pgfpathlineto{\pgfqpoint{1.825819in}{1.067542in}}%
\pgfpathlineto{\pgfqpoint{1.813813in}{1.056850in}}%
\pgfpathlineto{\pgfqpoint{1.798819in}{1.046763in}}%
\pgfpathlineto{\pgfqpoint{1.781016in}{1.037462in}}%
\pgfpathlineto{\pgfqpoint{1.758447in}{1.028391in}}%
\pgfpathlineto{\pgfqpoint{1.733203in}{1.020815in}}%
\pgfpathlineto{\pgfqpoint{1.705410in}{1.014872in}}%
\pgfpathlineto{\pgfqpoint{1.675178in}{1.010714in}}%
\pgfpathlineto{\pgfqpoint{1.642610in}{1.008507in}}%
\pgfpathlineto{\pgfqpoint{1.607809in}{1.008432in}}%
\pgfpathlineto{\pgfqpoint{1.570886in}{1.010691in}}%
\pgfpathlineto{\pgfqpoint{1.534118in}{1.015181in}}%
\pgfpathlineto{\pgfqpoint{1.495454in}{1.022233in}}%
\pgfpathlineto{\pgfqpoint{1.457161in}{1.031563in}}%
\pgfpathlineto{\pgfqpoint{1.419337in}{1.043132in}}%
\pgfpathlineto{\pgfqpoint{1.382089in}{1.056929in}}%
\pgfpathlineto{\pgfqpoint{1.347544in}{1.072019in}}%
\pgfpathlineto{\pgfqpoint{1.313727in}{1.089133in}}%
\pgfpathlineto{\pgfqpoint{1.280762in}{1.108299in}}%
\pgfpathlineto{\pgfqpoint{1.248782in}{1.129536in}}%
\pgfpathlineto{\pgfqpoint{1.219708in}{1.151422in}}%
\pgfpathlineto{\pgfqpoint{1.191752in}{1.175138in}}%
\pgfpathlineto{\pgfqpoint{1.165031in}{1.200649in}}%
\pgfpathlineto{\pgfqpoint{1.139653in}{1.227898in}}%
\pgfpathlineto{\pgfqpoint{1.115714in}{1.256800in}}%
\pgfpathlineto{\pgfqpoint{1.093288in}{1.287251in}}%
\pgfpathlineto{\pgfqpoint{1.071178in}{1.321163in}}%
\pgfpathlineto{\pgfqpoint{1.050868in}{1.356520in}}%
\pgfpathlineto{\pgfqpoint{1.032365in}{1.393152in}}%
\pgfpathlineto{\pgfqpoint{1.014718in}{1.433142in}}%
\pgfpathlineto{\pgfqpoint{0.999024in}{1.474185in}}%
\pgfpathlineto{\pgfqpoint{0.984506in}{1.518461in}}%
\pgfpathlineto{\pgfqpoint{0.972009in}{1.563537in}}%
\pgfpathlineto{\pgfqpoint{0.960944in}{1.611678in}}%
\pgfpathlineto{\pgfqpoint{0.951530in}{1.662824in}}%
\pgfpathlineto{\pgfqpoint{0.944286in}{1.714431in}}%
\pgfpathlineto{\pgfqpoint{0.938950in}{1.768847in}}%
\pgfpathlineto{\pgfqpoint{0.935870in}{1.823491in}}%
\pgfpathlineto{\pgfqpoint{0.935034in}{1.878240in}}%
\pgfpathlineto{\pgfqpoint{0.936466in}{1.932973in}}%
\pgfpathlineto{\pgfqpoint{0.940005in}{1.985084in}}%
\pgfpathlineto{\pgfqpoint{0.945759in}{2.036935in}}%
\pgfpathlineto{\pgfqpoint{0.953410in}{2.085938in}}%
\pgfpathlineto{\pgfqpoint{0.962764in}{2.132000in}}%
\pgfpathlineto{\pgfqpoint{0.974287in}{2.177414in}}%
\pgfpathlineto{\pgfqpoint{0.987332in}{2.219653in}}%
\pgfpathlineto{\pgfqpoint{1.001667in}{2.258654in}}%
\pgfpathlineto{\pgfqpoint{1.018051in}{2.296583in}}%
\pgfpathlineto{\pgfqpoint{1.035401in}{2.331101in}}%
\pgfpathlineto{\pgfqpoint{1.054650in}{2.364275in}}%
\pgfpathlineto{\pgfqpoint{1.074406in}{2.393984in}}%
\pgfpathlineto{\pgfqpoint{1.095771in}{2.422197in}}%
\pgfpathlineto{\pgfqpoint{1.118662in}{2.448797in}}%
\pgfpathlineto{\pgfqpoint{1.142967in}{2.473701in}}%
\pgfpathlineto{\pgfqpoint{1.168550in}{2.496867in}}%
\pgfpathlineto{\pgfqpoint{1.197085in}{2.519662in}}%
\pgfpathlineto{\pgfqpoint{1.226727in}{2.540526in}}%
\pgfpathlineto{\pgfqpoint{1.259242in}{2.560673in}}%
\pgfpathlineto{\pgfqpoint{1.294612in}{2.579881in}}%
\pgfpathlineto{\pgfqpoint{1.332792in}{2.597982in}}%
\pgfpathlineto{\pgfqpoint{1.373719in}{2.614859in}}%
\pgfpathlineto{\pgfqpoint{1.417319in}{2.630445in}}%
\pgfpathlineto{\pgfqpoint{1.465632in}{2.645312in}}%
\pgfpathlineto{\pgfqpoint{1.518640in}{2.659204in}}%
\pgfpathlineto{\pgfqpoint{1.576309in}{2.671929in}}%
\pgfpathlineto{\pgfqpoint{1.638597in}{2.683344in}}%
\pgfpathlineto{\pgfqpoint{1.705462in}{2.693343in}}%
\pgfpathlineto{\pgfqpoint{1.779027in}{2.702064in}}%
\pgfpathlineto{\pgfqpoint{1.857097in}{2.709077in}}%
\pgfpathlineto{\pgfqpoint{1.939633in}{2.714280in}}%
\pgfpathlineto{\pgfqpoint{2.026598in}{2.717513in}}%
\pgfpathlineto{\pgfqpoint{2.113605in}{2.718523in}}%
\pgfpathlineto{\pgfqpoint{2.198435in}{2.717303in}}%
\pgfpathlineto{\pgfqpoint{2.278866in}{2.713929in}}%
\pgfpathlineto{\pgfqpoint{2.352678in}{2.708598in}}%
\pgfpathlineto{\pgfqpoint{2.417657in}{2.701709in}}%
\pgfpathlineto{\pgfqpoint{2.473770in}{2.693630in}}%
\pgfpathlineto{\pgfqpoint{2.523140in}{2.684368in}}%
\pgfpathlineto{\pgfqpoint{2.565726in}{2.674202in}}%
\pgfpathlineto{\pgfqpoint{2.601510in}{2.663544in}}%
\pgfpathlineto{\pgfqpoint{2.632577in}{2.652142in}}%
\pgfpathlineto{\pgfqpoint{2.658899in}{2.640331in}}%
\pgfpathlineto{\pgfqpoint{2.682438in}{2.627436in}}%
\pgfpathlineto{\pgfqpoint{2.703062in}{2.613571in}}%
\pgfpathlineto{\pgfqpoint{2.720674in}{2.598978in}}%
\pgfpathlineto{\pgfqpoint{2.735263in}{2.584053in}}%
\pgfpathlineto{\pgfqpoint{2.748320in}{2.567377in}}%
\pgfpathlineto{\pgfqpoint{2.759553in}{2.549046in}}%
\pgfpathlineto{\pgfqpoint{2.768788in}{2.529306in}}%
\pgfpathlineto{\pgfqpoint{2.776017in}{2.508498in}}%
\pgfpathlineto{\pgfqpoint{2.781884in}{2.484540in}}%
\pgfpathlineto{\pgfqpoint{2.786102in}{2.457597in}}%
\pgfpathlineto{\pgfqpoint{2.788720in}{2.425384in}}%
\pgfpathlineto{\pgfqpoint{2.789427in}{2.388061in}}%
\pgfpathlineto{\pgfqpoint{2.787962in}{2.340801in}}%
\pgfpathlineto{\pgfqpoint{2.783672in}{2.278768in}}%
\pgfpathlineto{\pgfqpoint{2.774289in}{2.179783in}}%
\pgfpathlineto{\pgfqpoint{2.743611in}{1.868119in}}%
\pgfpathlineto{\pgfqpoint{2.730112in}{1.702060in}}%
\pgfpathlineto{\pgfqpoint{2.717287in}{1.515949in}}%
\pgfpathlineto{\pgfqpoint{2.702602in}{1.267597in}}%
\pgfpathlineto{\pgfqpoint{2.684434in}{0.964630in}}%
\pgfpathlineto{\pgfqpoint{2.675374in}{0.850600in}}%
\pgfpathlineto{\pgfqpoint{2.667030in}{0.771523in}}%
\pgfpathlineto{\pgfqpoint{2.658752in}{0.712543in}}%
\pgfpathlineto{\pgfqpoint{2.650176in}{0.666284in}}%
\pgfpathlineto{\pgfqpoint{2.640820in}{0.627931in}}%
\pgfpathlineto{\pgfqpoint{2.631145in}{0.597534in}}%
\pgfpathlineto{\pgfqpoint{2.621004in}{0.572745in}}%
\pgfpathlineto{\pgfqpoint{2.609856in}{0.551383in}}%
\pgfpathlineto{\pgfqpoint{2.598042in}{0.533534in}}%
\pgfpathlineto{\pgfqpoint{2.584496in}{0.517378in}}%
\pgfpathlineto{\pgfqpoint{2.571109in}{0.504669in}}%
\pgfpathlineto{\pgfqpoint{2.554789in}{0.492313in}}%
\pgfpathlineto{\pgfqpoint{2.537457in}{0.481914in}}%
\pgfpathlineto{\pgfqpoint{2.517374in}{0.472367in}}%
\pgfpathlineto{\pgfqpoint{2.492542in}{0.463178in}}%
\pgfpathlineto{\pgfqpoint{2.462979in}{0.454833in}}%
\pgfpathlineto{\pgfqpoint{2.428766in}{0.447542in}}%
\pgfpathlineto{\pgfqpoint{2.385671in}{0.440735in}}%
\pgfpathlineto{\pgfqpoint{2.331557in}{0.434581in}}%
\pgfpathlineto{\pgfqpoint{2.262115in}{0.429077in}}%
\pgfpathlineto{\pgfqpoint{2.170851in}{0.424236in}}%
\pgfpathlineto{\pgfqpoint{2.049086in}{0.420134in}}%
\pgfpathlineto{\pgfqpoint{1.879436in}{0.416783in}}%
\pgfpathlineto{\pgfqpoint{1.640159in}{0.414418in}}%
\pgfpathlineto{\pgfqpoint{1.322562in}{0.413569in}}%
\pgfpathlineto{\pgfqpoint{1.020194in}{0.414850in}}%
\pgfpathlineto{\pgfqpoint{0.822256in}{0.417715in}}%
\pgfpathlineto{\pgfqpoint{0.704835in}{0.421430in}}%
\pgfpathlineto{\pgfqpoint{0.630977in}{0.425829in}}%
\pgfpathlineto{\pgfqpoint{0.583316in}{0.430734in}}%
\pgfpathlineto{\pgfqpoint{0.551033in}{0.436123in}}%
\pgfpathlineto{\pgfqpoint{0.527708in}{0.442189in}}%
\pgfpathlineto{\pgfqpoint{0.511250in}{0.448625in}}%
\pgfpathlineto{\pgfqpoint{0.499549in}{0.455216in}}%
\pgfpathlineto{\pgfqpoint{0.488916in}{0.463841in}}%
\pgfpathlineto{\pgfqpoint{0.481322in}{0.472730in}}%
\pgfpathlineto{\pgfqpoint{0.474078in}{0.485127in}}%
\pgfpathlineto{\pgfqpoint{0.468753in}{0.498748in}}%
\pgfpathlineto{\pgfqpoint{0.463870in}{0.517848in}}%
\pgfpathlineto{\pgfqpoint{0.459679in}{0.544796in}}%
\pgfpathlineto{\pgfqpoint{0.456386in}{0.581938in}}%
\pgfpathlineto{\pgfqpoint{0.453731in}{0.639106in}}%
\pgfpathlineto{\pgfqpoint{0.451681in}{0.736155in}}%
\pgfpathlineto{\pgfqpoint{0.450220in}{0.927815in}}%
\pgfpathlineto{\pgfqpoint{0.449345in}{1.403252in}}%
\pgfpathlineto{\pgfqpoint{0.449543in}{2.682703in}}%
\pgfpathlineto{\pgfqpoint{0.451011in}{2.856932in}}%
\pgfpathlineto{\pgfqpoint{0.452802in}{2.879219in}}%
\pgfpathlineto{\pgfqpoint{0.455188in}{2.886108in}}%
\pgfpathlineto{\pgfqpoint{0.458626in}{2.889028in}}%
\pgfpathlineto{\pgfqpoint{0.464996in}{2.890553in}}%
\pgfpathlineto{\pgfqpoint{0.482376in}{2.891423in}}%
\pgfpathlineto{\pgfqpoint{0.565038in}{2.891729in}}%
\pgfpathlineto{\pgfqpoint{2.733842in}{2.891760in}}%
\pgfpathlineto{\pgfqpoint{4.789510in}{2.890885in}}%
\pgfpathlineto{\pgfqpoint{4.793727in}{2.889730in}}%
\pgfpathlineto{\pgfqpoint{4.795481in}{2.888307in}}%
\pgfpathlineto{\pgfqpoint{4.797106in}{2.881145in}}%
\pgfpathlineto{\pgfqpoint{4.797997in}{2.858771in}}%
\pgfpathlineto{\pgfqpoint{4.798039in}{2.856283in}}%
\pgfpathlineto{\pgfqpoint{4.798039in}{2.856283in}}%
\pgfusepath{stroke}%
\end{pgfscope}%
\begin{pgfscope}%
\pgfpathrectangle{\pgfqpoint{0.448634in}{0.402556in}}{\pgfqpoint{4.350661in}{2.489204in}} %
\pgfusepath{clip}%
\pgfsetrectcap%
\pgfsetroundjoin%
\pgfsetlinewidth{1.003750pt}%
\definecolor{currentstroke}{rgb}{0.839216,0.152941,0.156863}%
\pgfsetstrokecolor{currentstroke}%
\pgfsetdash{}{0pt}%
\pgfpathmoveto{\pgfqpoint{3.428775in}{0.402610in}}%
\pgfpathlineto{\pgfqpoint{2.806635in}{0.403761in}}%
\pgfpathlineto{\pgfqpoint{2.769695in}{0.405579in}}%
\pgfpathlineto{\pgfqpoint{2.754635in}{0.408066in}}%
\pgfpathlineto{\pgfqpoint{2.746394in}{0.411200in}}%
\pgfpathlineto{\pgfqpoint{2.740946in}{0.415267in}}%
\pgfpathlineto{\pgfqpoint{2.736787in}{0.420986in}}%
\pgfpathlineto{\pgfqpoint{2.733284in}{0.430073in}}%
\pgfpathlineto{\pgfqpoint{2.730451in}{0.444638in}}%
\pgfpathlineto{\pgfqpoint{2.728239in}{0.469393in}}%
\pgfpathlineto{\pgfqpoint{2.726471in}{0.519132in}}%
\pgfpathlineto{\pgfqpoint{2.725712in}{0.613716in}}%
\pgfpathlineto{\pgfqpoint{2.726842in}{0.768040in}}%
\pgfpathlineto{\pgfqpoint{2.730557in}{0.962150in}}%
\pgfpathlineto{\pgfqpoint{2.736611in}{1.158672in}}%
\pgfpathlineto{\pgfqpoint{2.744092in}{1.327719in}}%
\pgfpathlineto{\pgfqpoint{2.753202in}{1.484191in}}%
\pgfpathlineto{\pgfqpoint{2.763257in}{1.620611in}}%
\pgfpathlineto{\pgfqpoint{2.776119in}{1.764217in}}%
\pgfpathlineto{\pgfqpoint{2.788915in}{1.877778in}}%
\pgfpathlineto{\pgfqpoint{2.805749in}{2.005742in}}%
\pgfpathlineto{\pgfqpoint{2.821177in}{2.101199in}}%
\pgfpathlineto{\pgfqpoint{2.838360in}{2.193720in}}%
\pgfpathlineto{\pgfqpoint{2.859136in}{2.292967in}}%
\pgfpathlineto{\pgfqpoint{2.887210in}{2.425961in}}%
\pgfpathlineto{\pgfqpoint{2.896993in}{2.479561in}}%
\pgfpathlineto{\pgfqpoint{2.901544in}{2.516524in}}%
\pgfpathlineto{\pgfqpoint{2.902851in}{2.543855in}}%
\pgfpathlineto{\pgfqpoint{2.901959in}{2.566224in}}%
\pgfpathlineto{\pgfqpoint{2.899153in}{2.585864in}}%
\pgfpathlineto{\pgfqpoint{2.894795in}{2.602547in}}%
\pgfpathlineto{\pgfqpoint{2.888486in}{2.618389in}}%
\pgfpathlineto{\pgfqpoint{2.880259in}{2.633034in}}%
\pgfpathlineto{\pgfqpoint{2.870350in}{2.646247in}}%
\pgfpathlineto{\pgfqpoint{2.857401in}{2.659532in}}%
\pgfpathlineto{\pgfqpoint{2.843191in}{2.671011in}}%
\pgfpathlineto{\pgfqpoint{2.824239in}{2.683210in}}%
\pgfpathlineto{\pgfqpoint{2.802415in}{2.694419in}}%
\pgfpathlineto{\pgfqpoint{2.775810in}{2.705370in}}%
\pgfpathlineto{\pgfqpoint{2.744463in}{2.715716in}}%
\pgfpathlineto{\pgfqpoint{2.708438in}{2.725253in}}%
\pgfpathlineto{\pgfqpoint{2.665657in}{2.734290in}}%
\pgfpathlineto{\pgfqpoint{2.613993in}{2.742870in}}%
\pgfpathlineto{\pgfqpoint{2.553461in}{2.750589in}}%
\pgfpathlineto{\pgfqpoint{2.481922in}{2.757366in}}%
\pgfpathlineto{\pgfqpoint{2.399400in}{2.762840in}}%
\pgfpathlineto{\pgfqpoint{2.310271in}{2.766483in}}%
\pgfpathlineto{\pgfqpoint{2.175418in}{2.768726in}}%
\pgfpathlineto{\pgfqpoint{2.066655in}{2.767942in}}%
\pgfpathlineto{\pgfqpoint{1.953572in}{2.764860in}}%
\pgfpathlineto{\pgfqpoint{1.851430in}{2.759759in}}%
\pgfpathlineto{\pgfqpoint{1.745052in}{2.752169in}}%
\pgfpathlineto{\pgfqpoint{1.658375in}{2.743455in}}%
\pgfpathlineto{\pgfqpoint{1.580554in}{2.733462in}}%
\pgfpathlineto{\pgfqpoint{1.490059in}{2.719339in}}%
\pgfpathlineto{\pgfqpoint{1.417233in}{2.704698in}}%
\pgfpathlineto{\pgfqpoint{1.361994in}{2.690818in}}%
\pgfpathlineto{\pgfqpoint{1.311462in}{2.675819in}}%
\pgfpathlineto{\pgfqpoint{1.265668in}{2.659924in}}%
\pgfpathlineto{\pgfqpoint{1.222576in}{2.642586in}}%
\pgfpathlineto{\pgfqpoint{1.184326in}{2.624683in}}%
\pgfpathlineto{\pgfqpoint{1.148894in}{2.605624in}}%
\pgfpathlineto{\pgfqpoint{1.116333in}{2.585574in}}%
\pgfpathlineto{\pgfqpoint{1.092329in}{2.568513in}}%
\pgfpathlineto{\pgfqpoint{1.079761in}{2.558687in}}%
\pgfpathlineto{\pgfqpoint{1.051545in}{2.535380in}}%
\pgfpathlineto{\pgfqpoint{1.026313in}{2.511714in}}%
\pgfpathlineto{\pgfqpoint{1.002400in}{2.486319in}}%
\pgfpathlineto{\pgfqpoint{0.979914in}{2.459271in}}%
\pgfpathlineto{\pgfqpoint{0.958935in}{2.430680in}}%
\pgfpathlineto{\pgfqpoint{0.938265in}{2.398645in}}%
\pgfpathlineto{\pgfqpoint{0.923048in}{2.371387in}}%
\pgfpathlineto{\pgfqpoint{0.904514in}{2.334776in}}%
\pgfpathlineto{\pgfqpoint{0.887854in}{2.297003in}}%
\pgfpathlineto{\pgfqpoint{0.872132in}{2.255973in}}%
\pgfpathlineto{\pgfqpoint{0.857508in}{2.211743in}}%
\pgfpathlineto{\pgfqpoint{0.844762in}{2.166759in}}%
\pgfpathlineto{\pgfqpoint{0.838624in}{2.140308in}}%
\pgfpathlineto{\pgfqpoint{0.826982in}{2.087195in}}%
\pgfpathlineto{\pgfqpoint{0.816322in}{2.028717in}}%
\pgfpathlineto{\pgfqpoint{0.810087in}{1.984497in}}%
\pgfpathlineto{\pgfqpoint{0.808026in}{1.967240in}}%
\pgfpathlineto{\pgfqpoint{0.800077in}{1.898142in}}%
\pgfpathlineto{\pgfqpoint{0.793713in}{1.823825in}}%
\pgfpathlineto{\pgfqpoint{0.788798in}{1.741877in}}%
\pgfpathlineto{\pgfqpoint{0.786199in}{1.677227in}}%
\pgfpathlineto{\pgfqpoint{0.776951in}{1.453483in}}%
\pgfpathlineto{\pgfqpoint{0.773280in}{1.418896in}}%
\pgfpathlineto{\pgfqpoint{0.768298in}{1.389584in}}%
\pgfpathlineto{\pgfqpoint{0.762752in}{1.368110in}}%
\pgfpathlineto{\pgfqpoint{0.756722in}{1.352124in}}%
\pgfpathlineto{\pgfqpoint{0.749753in}{1.339521in}}%
\pgfpathlineto{\pgfqpoint{0.742202in}{1.330601in}}%
\pgfpathlineto{\pgfqpoint{0.734855in}{1.325313in}}%
\pgfpathlineto{\pgfqpoint{0.726558in}{1.322420in}}%
\pgfpathlineto{\pgfqpoint{0.717884in}{1.322224in}}%
\pgfpathlineto{\pgfqpoint{0.709413in}{1.324412in}}%
\pgfpathlineto{\pgfqpoint{0.699548in}{1.329605in}}%
\pgfpathlineto{\pgfqpoint{0.688895in}{1.338203in}}%
\pgfpathlineto{\pgfqpoint{0.677908in}{1.350248in}}%
\pgfpathlineto{\pgfqpoint{0.666886in}{1.365647in}}%
\pgfpathlineto{\pgfqpoint{0.654913in}{1.386417in}}%
\pgfpathlineto{\pgfqpoint{0.642574in}{1.412730in}}%
\pgfpathlineto{\pgfqpoint{0.630329in}{1.444629in}}%
\pgfpathlineto{\pgfqpoint{0.618505in}{1.482081in}}%
\pgfpathlineto{\pgfqpoint{0.608613in}{1.520256in}}%
\pgfpathlineto{\pgfqpoint{0.590203in}{1.612445in}}%
\pgfpathlineto{\pgfqpoint{0.581848in}{1.668884in}}%
\pgfpathlineto{\pgfqpoint{0.573138in}{1.740376in}}%
\pgfpathlineto{\pgfqpoint{0.567062in}{1.807213in}}%
\pgfpathlineto{\pgfqpoint{0.560532in}{1.896510in}}%
\pgfpathlineto{\pgfqpoint{0.555526in}{1.995910in}}%
\pgfpathlineto{\pgfqpoint{0.552564in}{2.097908in}}%
\pgfpathlineto{\pgfqpoint{0.551526in}{2.204935in}}%
\pgfpathlineto{\pgfqpoint{0.552728in}{2.309470in}}%
\pgfpathlineto{\pgfqpoint{0.556011in}{2.403981in}}%
\pgfpathlineto{\pgfqpoint{0.560953in}{2.483430in}}%
\pgfpathlineto{\pgfqpoint{0.567303in}{2.550240in}}%
\pgfpathlineto{\pgfqpoint{0.574928in}{2.606817in}}%
\pgfpathlineto{\pgfqpoint{0.582988in}{2.650657in}}%
\pgfpathlineto{\pgfqpoint{0.592756in}{2.691452in}}%
\pgfpathlineto{\pgfqpoint{0.602650in}{2.721756in}}%
\pgfpathlineto{\pgfqpoint{0.612983in}{2.746442in}}%
\pgfpathlineto{\pgfqpoint{0.624292in}{2.767692in}}%
\pgfpathlineto{\pgfqpoint{0.636231in}{2.785433in}}%
\pgfpathlineto{\pgfqpoint{0.649892in}{2.801461in}}%
\pgfpathlineto{\pgfqpoint{0.663386in}{2.814020in}}%
\pgfpathlineto{\pgfqpoint{0.679842in}{2.826135in}}%
\pgfpathlineto{\pgfqpoint{0.697326in}{2.836197in}}%
\pgfpathlineto{\pgfqpoint{0.715574in}{2.844285in}}%
\pgfpathlineto{\pgfqpoint{0.738439in}{2.852335in}}%
\pgfpathlineto{\pgfqpoint{0.765983in}{2.859639in}}%
\pgfpathlineto{\pgfqpoint{0.800300in}{2.866256in}}%
\pgfpathlineto{\pgfqpoint{0.841340in}{2.871832in}}%
\pgfpathlineto{\pgfqpoint{0.895547in}{2.876803in}}%
\pgfpathlineto{\pgfqpoint{0.969413in}{2.881069in}}%
\pgfpathlineto{\pgfqpoint{1.071608in}{2.884501in}}%
\pgfpathlineto{\pgfqpoint{1.219512in}{2.887074in}}%
\pgfpathlineto{\pgfqpoint{1.471844in}{2.889091in}}%
\pgfpathlineto{\pgfqpoint{1.956941in}{2.890384in}}%
\pgfpathlineto{\pgfqpoint{3.096814in}{2.890781in}}%
\pgfpathlineto{\pgfqpoint{3.995224in}{2.889388in}}%
\pgfpathlineto{\pgfqpoint{4.275833in}{2.887011in}}%
\pgfpathlineto{\pgfqpoint{4.412847in}{2.883743in}}%
\pgfpathlineto{\pgfqpoint{4.491081in}{2.879810in}}%
\pgfpathlineto{\pgfqpoint{4.543127in}{2.875163in}}%
\pgfpathlineto{\pgfqpoint{4.579810in}{2.869841in}}%
\pgfpathlineto{\pgfqpoint{4.607580in}{2.863763in}}%
\pgfpathlineto{\pgfqpoint{4.630623in}{2.856424in}}%
\pgfpathlineto{\pgfqpoint{4.648834in}{2.848228in}}%
\pgfpathlineto{\pgfqpoint{4.664136in}{2.838773in}}%
\pgfpathlineto{\pgfqpoint{4.676470in}{2.828576in}}%
\pgfpathlineto{\pgfqpoint{4.687502in}{2.816585in}}%
\pgfpathlineto{\pgfqpoint{4.697051in}{2.803027in}}%
\pgfpathlineto{\pgfqpoint{4.706194in}{2.786098in}}%
\pgfpathlineto{\pgfqpoint{4.714508in}{2.765827in}}%
\pgfpathlineto{\pgfqpoint{4.722462in}{2.740013in}}%
\pgfpathlineto{\pgfqpoint{4.729577in}{2.708703in}}%
\pgfpathlineto{\pgfqpoint{4.736162in}{2.669601in}}%
\pgfpathlineto{\pgfqpoint{4.742419in}{2.617826in}}%
\pgfpathlineto{\pgfqpoint{4.747859in}{2.553410in}}%
\pgfpathlineto{\pgfqpoint{4.752661in}{2.468958in}}%
\pgfpathlineto{\pgfqpoint{4.756610in}{2.359528in}}%
\pgfpathlineto{\pgfqpoint{4.759416in}{2.217681in}}%
\pgfpathlineto{\pgfqpoint{4.760596in}{2.043444in}}%
\pgfpathlineto{\pgfqpoint{4.759662in}{1.851779in}}%
\pgfpathlineto{\pgfqpoint{4.756587in}{1.667613in}}%
\pgfpathlineto{\pgfqpoint{4.751596in}{1.503428in}}%
\pgfpathlineto{\pgfqpoint{4.745410in}{1.374185in}}%
\pgfpathlineto{\pgfqpoint{4.738113in}{1.267479in}}%
\pgfpathlineto{\pgfqpoint{4.729621in}{1.175896in}}%
\pgfpathlineto{\pgfqpoint{4.720762in}{1.104428in}}%
\pgfpathlineto{\pgfqpoint{4.711045in}{1.043204in}}%
\pgfpathlineto{\pgfqpoint{4.700364in}{0.989829in}}%
\pgfpathlineto{\pgfqpoint{4.689055in}{0.944345in}}%
\pgfpathlineto{\pgfqpoint{4.676881in}{0.904394in}}%
\pgfpathlineto{\pgfqpoint{4.676095in}{0.902073in}}%
\pgfpathlineto{\pgfqpoint{4.676095in}{0.902073in}}%
\pgfusepath{stroke}%
\end{pgfscope}%
\begin{pgfscope}%
\pgfpathrectangle{\pgfqpoint{0.448634in}{0.402556in}}{\pgfqpoint{4.350661in}{2.489204in}} %
\pgfusepath{clip}%
\pgfsetrectcap%
\pgfsetroundjoin%
\pgfsetlinewidth{1.003750pt}%
\definecolor{currentstroke}{rgb}{0.839216,0.152941,0.156863}%
\pgfsetstrokecolor{currentstroke}%
\pgfsetdash{}{0pt}%
\pgfpathmoveto{\pgfqpoint{2.795520in}{1.982745in}}%
\pgfpathlineto{\pgfqpoint{2.781780in}{1.874357in}}%
\pgfpathlineto{\pgfqpoint{2.769351in}{1.758234in}}%
\pgfpathlineto{\pgfqpoint{2.758095in}{1.631942in}}%
\pgfpathlineto{\pgfqpoint{2.747786in}{1.490551in}}%
\pgfpathlineto{\pgfqpoint{2.738644in}{1.334082in}}%
\pgfpathlineto{\pgfqpoint{2.730580in}{1.157591in}}%
\pgfpathlineto{\pgfqpoint{2.723334in}{0.948663in}}%
\pgfpathlineto{\pgfqpoint{2.709783in}{0.530788in}}%
\pgfpathlineto{\pgfqpoint{2.705868in}{0.488716in}}%
\pgfpathlineto{\pgfqpoint{2.701769in}{0.464281in}}%
\pgfpathlineto{\pgfqpoint{2.697021in}{0.447744in}}%
\pgfpathlineto{\pgfqpoint{2.691859in}{0.436812in}}%
\pgfpathlineto{\pgfqpoint{2.686245in}{0.429229in}}%
\pgfpathlineto{\pgfqpoint{2.679348in}{0.423188in}}%
\pgfpathlineto{\pgfqpoint{2.669540in}{0.417856in}}%
\pgfpathlineto{\pgfqpoint{2.656987in}{0.413810in}}%
\pgfpathlineto{\pgfqpoint{2.637654in}{0.410337in}}%
\pgfpathlineto{\pgfqpoint{2.607297in}{0.407617in}}%
\pgfpathlineto{\pgfqpoint{2.555121in}{0.405574in}}%
\pgfpathlineto{\pgfqpoint{2.450714in}{0.404139in}}%
\pgfpathlineto{\pgfqpoint{2.176624in}{0.403275in}}%
\pgfpathlineto{\pgfqpoint{1.130290in}{0.402953in}}%
\pgfpathlineto{\pgfqpoint{0.516849in}{0.404175in}}%
\pgfpathlineto{\pgfqpoint{0.466848in}{0.405970in}}%
\pgfpathlineto{\pgfqpoint{0.456130in}{0.407931in}}%
\pgfpathlineto{\pgfqpoint{0.452340in}{0.410303in}}%
\pgfpathlineto{\pgfqpoint{0.450346in}{0.414662in}}%
\pgfpathlineto{\pgfqpoint{0.449266in}{0.424524in}}%
\pgfpathlineto{\pgfqpoint{0.448771in}{0.464345in}}%
\pgfpathlineto{\pgfqpoint{0.448640in}{0.850171in}}%
\pgfpathlineto{\pgfqpoint{0.448679in}{2.891318in}}%
\pgfpathlineto{\pgfqpoint{0.448679in}{2.891318in}}%
\pgfusepath{stroke}%
\end{pgfscope}%
\begin{pgfscope}%
\pgfpathrectangle{\pgfqpoint{0.448634in}{0.402556in}}{\pgfqpoint{4.350661in}{2.489204in}} %
\pgfusepath{clip}%
\pgfsetrectcap%
\pgfsetroundjoin%
\pgfsetlinewidth{1.003750pt}%
\definecolor{currentstroke}{rgb}{0.839216,0.152941,0.156863}%
\pgfsetstrokecolor{currentstroke}%
\pgfsetdash{}{0pt}%
\pgfpathmoveto{\pgfqpoint{3.428194in}{0.402586in}}%
\pgfpathlineto{\pgfqpoint{2.782126in}{0.403703in}}%
\pgfpathlineto{\pgfqpoint{2.753911in}{0.405677in}}%
\pgfpathlineto{\pgfqpoint{2.743334in}{0.408448in}}%
\pgfpathlineto{\pgfqpoint{2.737723in}{0.412193in}}%
\pgfpathlineto{\pgfqpoint{2.733673in}{0.417999in}}%
\pgfpathlineto{\pgfqpoint{2.730653in}{0.427311in}}%
\pgfpathlineto{\pgfqpoint{2.728391in}{0.442008in}}%
\pgfpathlineto{\pgfqpoint{2.726546in}{0.471798in}}%
\pgfpathlineto{\pgfqpoint{2.725218in}{0.534007in}}%
\pgfpathlineto{\pgfqpoint{2.725170in}{0.655976in}}%
\pgfpathlineto{\pgfqpoint{2.727378in}{0.832690in}}%
\pgfpathlineto{\pgfqpoint{2.732259in}{1.041706in}}%
\pgfpathlineto{\pgfqpoint{2.738852in}{1.223260in}}%
\pgfpathlineto{\pgfqpoint{2.747079in}{1.389769in}}%
\pgfpathlineto{\pgfqpoint{2.756609in}{1.538721in}}%
\pgfpathlineto{\pgfqpoint{2.768955in}{1.694890in}}%
\pgfpathlineto{\pgfqpoint{2.781229in}{1.816048in}}%
\pgfpathlineto{\pgfqpoint{2.794402in}{1.924528in}}%
\pgfpathlineto{\pgfqpoint{2.812738in}{2.054726in}}%
\pgfpathlineto{\pgfqpoint{2.828775in}{2.147515in}}%
\pgfpathlineto{\pgfqpoint{2.847384in}{2.242228in}}%
\pgfpathlineto{\pgfqpoint{2.895820in}{2.479703in}}%
\pgfpathlineto{\pgfqpoint{2.900206in}{2.516692in}}%
\pgfpathlineto{\pgfqpoint{2.901347in}{2.544032in}}%
\pgfpathlineto{\pgfqpoint{2.900293in}{2.566392in}}%
\pgfpathlineto{\pgfqpoint{2.897336in}{2.586002in}}%
\pgfpathlineto{\pgfqpoint{2.892837in}{2.602636in}}%
\pgfpathlineto{\pgfqpoint{2.886395in}{2.618409in}}%
\pgfpathlineto{\pgfqpoint{2.878059in}{2.632972in}}%
\pgfpathlineto{\pgfqpoint{2.868065in}{2.646103in}}%
\pgfpathlineto{\pgfqpoint{2.855051in}{2.659303in}}%
\pgfpathlineto{\pgfqpoint{2.840801in}{2.670719in}}%
\pgfpathlineto{\pgfqpoint{2.821822in}{2.682863in}}%
\pgfpathlineto{\pgfqpoint{2.799981in}{2.694027in}}%
\pgfpathlineto{\pgfqpoint{2.773366in}{2.704946in}}%
\pgfpathlineto{\pgfqpoint{2.742012in}{2.715268in}}%
\pgfpathlineto{\pgfqpoint{2.705983in}{2.724787in}}%
\pgfpathlineto{\pgfqpoint{2.663200in}{2.733812in}}%
\pgfpathlineto{\pgfqpoint{2.611535in}{2.742380in}}%
\pgfpathlineto{\pgfqpoint{2.551002in}{2.750091in}}%
\pgfpathlineto{\pgfqpoint{2.481632in}{2.756683in}}%
\pgfpathlineto{\pgfqpoint{2.399112in}{2.762201in}}%
\pgfpathlineto{\pgfqpoint{2.309985in}{2.765886in}}%
\pgfpathlineto{\pgfqpoint{2.188184in}{2.768097in}}%
\pgfpathlineto{\pgfqpoint{2.081595in}{2.767619in}}%
\pgfpathlineto{\pgfqpoint{1.968506in}{2.764841in}}%
\pgfpathlineto{\pgfqpoint{1.864180in}{2.759919in}}%
\pgfpathlineto{\pgfqpoint{1.757786in}{2.752594in}}%
\pgfpathlineto{\pgfqpoint{1.671087in}{2.744172in}}%
\pgfpathlineto{\pgfqpoint{1.591076in}{2.734194in}}%
\pgfpathlineto{\pgfqpoint{1.502689in}{2.720718in}}%
\pgfpathlineto{\pgfqpoint{1.427655in}{2.706083in}}%
\pgfpathlineto{\pgfqpoint{1.372350in}{2.692544in}}%
\pgfpathlineto{\pgfqpoint{1.321734in}{2.677921in}}%
\pgfpathlineto{\pgfqpoint{1.273765in}{2.661664in}}%
\pgfpathlineto{\pgfqpoint{1.230567in}{2.644672in}}%
\pgfpathlineto{\pgfqpoint{1.192197in}{2.627106in}}%
\pgfpathlineto{\pgfqpoint{1.156620in}{2.608403in}}%
\pgfpathlineto{\pgfqpoint{1.123890in}{2.588717in}}%
\pgfpathlineto{\pgfqpoint{1.095883in}{2.569568in}}%
\pgfpathlineto{\pgfqpoint{1.063936in}{2.543702in}}%
\pgfpathlineto{\pgfqpoint{1.038216in}{2.520733in}}%
\pgfpathlineto{\pgfqpoint{1.013765in}{2.496017in}}%
\pgfpathlineto{\pgfqpoint{0.990703in}{2.469611in}}%
\pgfpathlineto{\pgfqpoint{0.969123in}{2.441613in}}%
\pgfpathlineto{\pgfqpoint{0.949082in}{2.412155in}}%
\pgfpathlineto{\pgfqpoint{0.930603in}{2.381388in}}%
\pgfpathlineto{\pgfqpoint{0.906555in}{2.334053in}}%
\pgfpathlineto{\pgfqpoint{0.889924in}{2.296263in}}%
\pgfpathlineto{\pgfqpoint{0.874240in}{2.255214in}}%
\pgfpathlineto{\pgfqpoint{0.859667in}{2.210962in}}%
\pgfpathlineto{\pgfqpoint{0.846985in}{2.165955in}}%
\pgfpathlineto{\pgfqpoint{0.839632in}{2.134716in}}%
\pgfpathlineto{\pgfqpoint{0.828237in}{2.081533in}}%
\pgfpathlineto{\pgfqpoint{0.817866in}{2.022987in}}%
\pgfpathlineto{\pgfqpoint{0.810783in}{1.971353in}}%
\pgfpathlineto{\pgfqpoint{0.802845in}{1.902253in}}%
\pgfpathlineto{\pgfqpoint{0.796553in}{1.827928in}}%
\pgfpathlineto{\pgfqpoint{0.791695in}{1.743481in}}%
\pgfpathlineto{\pgfqpoint{0.787772in}{1.621596in}}%
\pgfpathlineto{\pgfqpoint{0.785406in}{1.522065in}}%
\pgfpathlineto{\pgfqpoint{0.785406in}{1.522065in}}%
\pgfusepath{stroke}%
\end{pgfscope}%
\begin{pgfscope}%
\pgfpathrectangle{\pgfqpoint{0.448634in}{0.402556in}}{\pgfqpoint{4.350661in}{2.489204in}} %
\pgfusepath{clip}%
\pgfsetrectcap%
\pgfsetroundjoin%
\pgfsetlinewidth{1.003750pt}%
\definecolor{currentstroke}{rgb}{0.580392,0.403922,0.741176}%
\pgfsetstrokecolor{currentstroke}%
\pgfsetdash{}{0pt}%
\pgfpathmoveto{\pgfqpoint{0.448634in}{2.896245in}}%
\pgfpathlineto{\pgfqpoint{0.448593in}{0.407043in}}%
\pgfpathlineto{\pgfqpoint{0.448593in}{0.407043in}}%
\pgfusepath{stroke}%
\end{pgfscope}%
\begin{pgfscope}%
\pgfpathrectangle{\pgfqpoint{0.448634in}{0.402556in}}{\pgfqpoint{4.350661in}{2.489204in}} %
\pgfusepath{clip}%
\pgfsetrectcap%
\pgfsetroundjoin%
\pgfsetlinewidth{1.003750pt}%
\definecolor{currentstroke}{rgb}{0.580392,0.403922,0.741176}%
\pgfsetstrokecolor{currentstroke}%
\pgfsetdash{}{0pt}%
\pgfpathmoveto{\pgfqpoint{0.576847in}{1.760831in}}%
\pgfpathlineto{\pgfqpoint{0.569389in}{1.840024in}}%
\pgfpathlineto{\pgfqpoint{0.563205in}{1.929353in}}%
\pgfpathlineto{\pgfqpoint{0.558589in}{2.028778in}}%
\pgfpathlineto{\pgfqpoint{0.555982in}{2.133280in}}%
\pgfpathlineto{\pgfqpoint{0.555563in}{2.237822in}}%
\pgfpathlineto{\pgfqpoint{0.557369in}{2.337366in}}%
\pgfpathlineto{\pgfqpoint{0.561094in}{2.424381in}}%
\pgfpathlineto{\pgfqpoint{0.566402in}{2.498806in}}%
\pgfpathlineto{\pgfqpoint{0.572907in}{2.560584in}}%
\pgfpathlineto{\pgfqpoint{0.580457in}{2.612133in}}%
\pgfpathlineto{\pgfqpoint{0.589086in}{2.655830in}}%
\pgfpathlineto{\pgfqpoint{0.598407in}{2.691603in}}%
\pgfpathlineto{\pgfqpoint{0.608615in}{2.721770in}}%
\pgfpathlineto{\pgfqpoint{0.619243in}{2.746291in}}%
\pgfpathlineto{\pgfqpoint{0.630820in}{2.767351in}}%
\pgfpathlineto{\pgfqpoint{0.642979in}{2.784895in}}%
\pgfpathlineto{\pgfqpoint{0.656819in}{2.800723in}}%
\pgfpathlineto{\pgfqpoint{0.672203in}{2.814558in}}%
\pgfpathlineto{\pgfqpoint{0.688860in}{2.826309in}}%
\pgfpathlineto{\pgfqpoint{0.706469in}{2.836083in}}%
\pgfpathlineto{\pgfqpoint{0.726812in}{2.844881in}}%
\pgfpathlineto{\pgfqpoint{0.751874in}{2.853207in}}%
\pgfpathlineto{\pgfqpoint{0.781640in}{2.860551in}}%
\pgfpathlineto{\pgfqpoint{0.818177in}{2.867056in}}%
\pgfpathlineto{\pgfqpoint{0.863589in}{2.872686in}}%
\pgfpathlineto{\pgfqpoint{0.922169in}{2.877519in}}%
\pgfpathlineto{\pgfqpoint{1.000399in}{2.881568in}}%
\pgfpathlineto{\pgfqpoint{1.111302in}{2.884882in}}%
\pgfpathlineto{\pgfqpoint{1.274437in}{2.887368in}}%
\pgfpathlineto{\pgfqpoint{1.552874in}{2.889263in}}%
\pgfpathlineto{\pgfqpoint{2.107582in}{2.890457in}}%
\pgfpathlineto{\pgfqpoint{3.343169in}{2.890573in}}%
\pgfpathlineto{\pgfqpoint{4.043624in}{2.888941in}}%
\pgfpathlineto{\pgfqpoint{4.289425in}{2.886404in}}%
\pgfpathlineto{\pgfqpoint{4.413383in}{2.883093in}}%
\pgfpathlineto{\pgfqpoint{4.489433in}{2.878996in}}%
\pgfpathlineto{\pgfqpoint{4.541460in}{2.874080in}}%
\pgfpathlineto{\pgfqpoint{4.578109in}{2.868469in}}%
\pgfpathlineto{\pgfqpoint{4.605827in}{2.862091in}}%
\pgfpathlineto{\pgfqpoint{4.626734in}{2.855243in}}%
\pgfpathlineto{\pgfqpoint{4.644933in}{2.847016in}}%
\pgfpathlineto{\pgfqpoint{4.660248in}{2.837586in}}%
\pgfpathlineto{\pgfqpoint{4.672630in}{2.827464in}}%
\pgfpathlineto{\pgfqpoint{4.683757in}{2.815587in}}%
\pgfpathlineto{\pgfqpoint{4.693411in}{2.802129in}}%
\pgfpathlineto{\pgfqpoint{4.702745in}{2.785336in}}%
\pgfpathlineto{\pgfqpoint{4.711281in}{2.765187in}}%
\pgfpathlineto{\pgfqpoint{4.719486in}{2.739477in}}%
\pgfpathlineto{\pgfqpoint{4.726296in}{2.710650in}}%
\pgfpathlineto{\pgfqpoint{4.733262in}{2.671635in}}%
\pgfpathlineto{\pgfqpoint{4.739606in}{2.622388in}}%
\pgfpathlineto{\pgfqpoint{4.745238in}{2.560496in}}%
\pgfpathlineto{\pgfqpoint{4.750165in}{2.481044in}}%
\pgfpathlineto{\pgfqpoint{4.754368in}{2.376611in}}%
\pgfpathlineto{\pgfqpoint{4.757444in}{2.242241in}}%
\pgfpathlineto{\pgfqpoint{4.758978in}{2.075475in}}%
\pgfpathlineto{\pgfqpoint{4.758448in}{1.888787in}}%
\pgfpathlineto{\pgfqpoint{4.755757in}{1.707103in}}%
\pgfpathlineto{\pgfqpoint{4.750926in}{1.532950in}}%
\pgfpathlineto{\pgfqpoint{4.744786in}{1.398719in}}%
\pgfpathlineto{\pgfqpoint{4.737575in}{1.289508in}}%
\pgfpathlineto{\pgfqpoint{4.728714in}{1.190462in}}%
\pgfpathlineto{\pgfqpoint{4.719652in}{1.116514in}}%
\pgfpathlineto{\pgfqpoint{4.710035in}{1.055269in}}%
\pgfpathlineto{\pgfqpoint{4.699503in}{1.001854in}}%
\pgfpathlineto{\pgfqpoint{4.689040in}{0.958683in}}%
\pgfpathlineto{\pgfqpoint{4.677218in}{0.918593in}}%
\pgfpathlineto{\pgfqpoint{4.664033in}{0.881742in}}%
\pgfpathlineto{\pgfqpoint{4.650582in}{0.850485in}}%
\pgfpathlineto{\pgfqpoint{4.636301in}{0.822563in}}%
\pgfpathlineto{\pgfqpoint{4.620205in}{0.795968in}}%
\pgfpathlineto{\pgfqpoint{4.603637in}{0.772895in}}%
\pgfpathlineto{\pgfqpoint{4.585485in}{0.751440in}}%
\pgfpathlineto{\pgfqpoint{4.565870in}{0.731743in}}%
\pgfpathlineto{\pgfqpoint{4.544960in}{0.713874in}}%
\pgfpathlineto{\pgfqpoint{4.522954in}{0.697819in}}%
\pgfpathlineto{\pgfqpoint{4.496153in}{0.681286in}}%
\pgfpathlineto{\pgfqpoint{4.470394in}{0.667949in}}%
\pgfpathlineto{\pgfqpoint{4.439957in}{0.654506in}}%
\pgfpathlineto{\pgfqpoint{4.406837in}{0.642278in}}%
\pgfpathlineto{\pgfqpoint{4.369005in}{0.630746in}}%
\pgfpathlineto{\pgfqpoint{4.326485in}{0.620224in}}%
\pgfpathlineto{\pgfqpoint{4.279322in}{0.610947in}}%
\pgfpathlineto{\pgfqpoint{4.227572in}{0.603083in}}%
\pgfpathlineto{\pgfqpoint{4.173446in}{0.597061in}}%
\pgfpathlineto{\pgfqpoint{4.110506in}{0.592201in}}%
\pgfpathlineto{\pgfqpoint{4.047467in}{0.589535in}}%
\pgfpathlineto{\pgfqpoint{3.977863in}{0.588622in}}%
\pgfpathlineto{\pgfqpoint{3.906088in}{0.589932in}}%
\pgfpathlineto{\pgfqpoint{3.834372in}{0.593495in}}%
\pgfpathlineto{\pgfqpoint{3.767115in}{0.599066in}}%
\pgfpathlineto{\pgfqpoint{3.704360in}{0.606391in}}%
\pgfpathlineto{\pgfqpoint{3.678512in}{0.610511in}}%
\pgfpathlineto{\pgfqpoint{3.620434in}{0.620502in}}%
\pgfpathlineto{\pgfqpoint{3.586315in}{0.628209in}}%
\pgfpathlineto{\pgfqpoint{3.495236in}{0.652429in}}%
\pgfpathlineto{\pgfqpoint{3.451524in}{0.667585in}}%
\pgfpathlineto{\pgfqpoint{3.408534in}{0.685221in}}%
\pgfpathlineto{\pgfqpoint{3.374590in}{0.702002in}}%
\pgfpathlineto{\pgfqpoint{3.345403in}{0.718684in}}%
\pgfpathlineto{\pgfqpoint{3.315232in}{0.738522in}}%
\pgfpathlineto{\pgfqpoint{3.288124in}{0.759292in}}%
\pgfpathlineto{\pgfqpoint{3.264001in}{0.780553in}}%
\pgfpathlineto{\pgfqpoint{3.241205in}{0.803651in}}%
\pgfpathlineto{\pgfqpoint{3.219891in}{0.828532in}}%
\pgfpathlineto{\pgfqpoint{3.200186in}{0.855094in}}%
\pgfpathlineto{\pgfqpoint{3.182175in}{0.883186in}}%
\pgfpathlineto{\pgfqpoint{3.165903in}{0.912636in}}%
\pgfpathlineto{\pgfqpoint{3.150349in}{0.945451in}}%
\pgfpathlineto{\pgfqpoint{3.136680in}{0.979348in}}%
\pgfpathlineto{\pgfqpoint{3.124071in}{1.016463in}}%
\pgfpathlineto{\pgfqpoint{3.112832in}{1.056773in}}%
\pgfpathlineto{\pgfqpoint{3.103044in}{1.100150in}}%
\pgfpathlineto{\pgfqpoint{3.095342in}{1.144075in}}%
\pgfpathlineto{\pgfqpoint{3.089207in}{1.190841in}}%
\pgfpathlineto{\pgfqpoint{3.084594in}{1.242842in}}%
\pgfpathlineto{\pgfqpoint{3.082136in}{1.295035in}}%
\pgfpathlineto{\pgfqpoint{3.081686in}{1.349790in}}%
\pgfpathlineto{\pgfqpoint{3.083450in}{1.407002in}}%
\pgfpathlineto{\pgfqpoint{3.087180in}{1.461593in}}%
\pgfpathlineto{\pgfqpoint{3.093485in}{1.520891in}}%
\pgfpathlineto{\pgfqpoint{3.101823in}{1.577337in}}%
\pgfpathlineto{\pgfqpoint{3.111930in}{1.630860in}}%
\pgfpathlineto{\pgfqpoint{3.124691in}{1.686212in}}%
\pgfpathlineto{\pgfqpoint{3.139179in}{1.738399in}}%
\pgfpathlineto{\pgfqpoint{3.155146in}{1.787369in}}%
\pgfpathlineto{\pgfqpoint{3.172354in}{1.833088in}}%
\pgfpathlineto{\pgfqpoint{3.191619in}{1.877719in}}%
\pgfpathlineto{\pgfqpoint{3.214027in}{1.923264in}}%
\pgfpathlineto{\pgfqpoint{3.236216in}{1.963159in}}%
\pgfpathlineto{\pgfqpoint{3.260180in}{2.001687in}}%
\pgfpathlineto{\pgfqpoint{3.285816in}{2.038779in}}%
\pgfpathlineto{\pgfqpoint{3.314417in}{2.076287in}}%
\pgfpathlineto{\pgfqpoint{3.348947in}{2.117713in}}%
\pgfpathlineto{\pgfqpoint{3.417135in}{2.198025in}}%
\pgfpathlineto{\pgfqpoint{3.426056in}{2.212131in}}%
\pgfpathlineto{\pgfqpoint{3.430800in}{2.223300in}}%
\pgfpathlineto{\pgfqpoint{3.432035in}{2.230606in}}%
\pgfpathlineto{\pgfqpoint{3.430773in}{2.237859in}}%
\pgfpathlineto{\pgfqpoint{3.426621in}{2.243528in}}%
\pgfpathlineto{\pgfqpoint{3.420908in}{2.247085in}}%
\pgfpathlineto{\pgfqpoint{3.412500in}{2.249584in}}%
\pgfpathlineto{\pgfqpoint{3.399498in}{2.250689in}}%
\pgfpathlineto{\pgfqpoint{3.384305in}{2.249672in}}%
\pgfpathlineto{\pgfqpoint{3.364985in}{2.246099in}}%
\pgfpathlineto{\pgfqpoint{3.341803in}{2.239343in}}%
\pgfpathlineto{\pgfqpoint{3.317109in}{2.229682in}}%
\pgfpathlineto{\pgfqpoint{3.291104in}{2.216986in}}%
\pgfpathlineto{\pgfqpoint{3.265928in}{2.202262in}}%
\pgfpathlineto{\pgfqpoint{3.239804in}{2.184361in}}%
\pgfpathlineto{\pgfqpoint{3.214775in}{2.164519in}}%
\pgfpathlineto{\pgfqpoint{3.190900in}{2.142893in}}%
\pgfpathlineto{\pgfqpoint{3.166656in}{2.117912in}}%
\pgfpathlineto{\pgfqpoint{3.143835in}{2.091233in}}%
\pgfpathlineto{\pgfqpoint{3.121079in}{2.061107in}}%
\pgfpathlineto{\pgfqpoint{3.099952in}{2.029463in}}%
\pgfpathlineto{\pgfqpoint{3.079251in}{1.994405in}}%
\pgfpathlineto{\pgfqpoint{3.059218in}{1.955915in}}%
\pgfpathlineto{\pgfqpoint{3.040058in}{1.914015in}}%
\pgfpathlineto{\pgfqpoint{3.022809in}{1.871041in}}%
\pgfpathlineto{\pgfqpoint{3.005790in}{1.822536in}}%
\pgfpathlineto{\pgfqpoint{2.990067in}{1.770819in}}%
\pgfpathlineto{\pgfqpoint{2.975708in}{1.715980in}}%
\pgfpathlineto{\pgfqpoint{2.962284in}{1.655680in}}%
\pgfpathlineto{\pgfqpoint{2.950495in}{1.592386in}}%
\pgfpathlineto{\pgfqpoint{2.940383in}{1.526185in}}%
\pgfpathlineto{\pgfqpoint{2.931745in}{1.454681in}}%
\pgfpathlineto{\pgfqpoint{2.925082in}{1.380399in}}%
\pgfpathlineto{\pgfqpoint{2.920647in}{1.305899in}}%
\pgfpathlineto{\pgfqpoint{2.918444in}{1.231270in}}%
\pgfpathlineto{\pgfqpoint{2.918545in}{1.159087in}}%
\pgfpathlineto{\pgfqpoint{2.920787in}{1.091931in}}%
\pgfpathlineto{\pgfqpoint{2.925177in}{1.027412in}}%
\pgfpathlineto{\pgfqpoint{2.931192in}{0.970580in}}%
\pgfpathlineto{\pgfqpoint{2.938760in}{0.919034in}}%
\pgfpathlineto{\pgfqpoint{2.947651in}{0.872852in}}%
\pgfpathlineto{\pgfqpoint{2.958214in}{0.829714in}}%
\pgfpathlineto{\pgfqpoint{2.969670in}{0.792114in}}%
\pgfpathlineto{\pgfqpoint{2.982463in}{0.757773in}}%
\pgfpathlineto{\pgfqpoint{2.996425in}{0.726812in}}%
\pgfpathlineto{\pgfqpoint{3.011299in}{0.699300in}}%
\pgfpathlineto{\pgfqpoint{3.026739in}{0.675225in}}%
\pgfpathlineto{\pgfqpoint{3.043828in}{0.652656in}}%
\pgfpathlineto{\pgfqpoint{3.062495in}{0.631788in}}%
\pgfpathlineto{\pgfqpoint{3.082603in}{0.612753in}}%
\pgfpathlineto{\pgfqpoint{3.103961in}{0.595592in}}%
\pgfpathlineto{\pgfqpoint{3.128268in}{0.579069in}}%
\pgfpathlineto{\pgfqpoint{3.153537in}{0.564554in}}%
\pgfpathlineto{\pgfqpoint{3.181571in}{0.550952in}}%
\pgfpathlineto{\pgfqpoint{3.214371in}{0.537647in}}%
\pgfpathlineto{\pgfqpoint{3.249847in}{0.525712in}}%
\pgfpathlineto{\pgfqpoint{3.290011in}{0.514571in}}%
\pgfpathlineto{\pgfqpoint{3.334821in}{0.504423in}}%
\pgfpathlineto{\pgfqpoint{3.386373in}{0.494999in}}%
\pgfpathlineto{\pgfqpoint{3.446798in}{0.486257in}}%
\pgfpathlineto{\pgfqpoint{3.518243in}{0.478282in}}%
\pgfpathlineto{\pgfqpoint{3.600685in}{0.471409in}}%
\pgfpathlineto{\pgfqpoint{3.696269in}{0.465713in}}%
\pgfpathlineto{\pgfqpoint{3.807144in}{0.461369in}}%
\pgfpathlineto{\pgfqpoint{3.933291in}{0.458719in}}%
\pgfpathlineto{\pgfqpoint{4.063809in}{0.458211in}}%
\pgfpathlineto{\pgfqpoint{4.187792in}{0.459914in}}%
\pgfpathlineto{\pgfqpoint{4.294335in}{0.463521in}}%
\pgfpathlineto{\pgfqpoint{4.381235in}{0.468574in}}%
\pgfpathlineto{\pgfqpoint{4.450636in}{0.474702in}}%
\pgfpathlineto{\pgfqpoint{4.506851in}{0.481799in}}%
\pgfpathlineto{\pgfqpoint{4.552009in}{0.489659in}}%
\pgfpathlineto{\pgfqpoint{4.588239in}{0.498115in}}%
\pgfpathlineto{\pgfqpoint{4.617656in}{0.507111in}}%
\pgfpathlineto{\pgfqpoint{4.642329in}{0.516843in}}%
\pgfpathlineto{\pgfqpoint{4.664195in}{0.527940in}}%
\pgfpathlineto{\pgfqpoint{4.681239in}{0.538945in}}%
\pgfpathlineto{\pgfqpoint{4.697164in}{0.551954in}}%
\pgfpathlineto{\pgfqpoint{4.710076in}{0.565290in}}%
\pgfpathlineto{\pgfqpoint{4.721579in}{0.580219in}}%
\pgfpathlineto{\pgfqpoint{4.731557in}{0.596522in}}%
\pgfpathlineto{\pgfqpoint{4.741000in}{0.616135in}}%
\pgfpathlineto{\pgfqpoint{4.749522in}{0.639027in}}%
\pgfpathlineto{\pgfqpoint{4.757522in}{0.667451in}}%
\pgfpathlineto{\pgfqpoint{4.764572in}{0.701346in}}%
\pgfpathlineto{\pgfqpoint{4.770840in}{0.743044in}}%
\pgfpathlineto{\pgfqpoint{4.776327in}{0.794935in}}%
\pgfpathlineto{\pgfqpoint{4.781278in}{0.864399in}}%
\pgfpathlineto{\pgfqpoint{4.785468in}{0.956372in}}%
\pgfpathlineto{\pgfqpoint{4.789000in}{1.085746in}}%
\pgfpathlineto{\pgfqpoint{4.791852in}{1.277386in}}%
\pgfpathlineto{\pgfqpoint{4.793959in}{1.581058in}}%
\pgfpathlineto{\pgfqpoint{4.794962in}{2.071430in}}%
\pgfpathlineto{\pgfqpoint{4.793967in}{2.559312in}}%
\pgfpathlineto{\pgfqpoint{4.791733in}{2.745982in}}%
\pgfpathlineto{\pgfqpoint{4.788955in}{2.818092in}}%
\pgfpathlineto{\pgfqpoint{4.785731in}{2.850227in}}%
\pgfpathlineto{\pgfqpoint{4.781879in}{2.867058in}}%
\pgfpathlineto{\pgfqpoint{4.777744in}{2.875781in}}%
\pgfpathlineto{\pgfqpoint{4.773096in}{2.880982in}}%
\pgfpathlineto{\pgfqpoint{4.767362in}{2.884504in}}%
\pgfpathlineto{\pgfqpoint{4.756852in}{2.887622in}}%
\pgfpathlineto{\pgfqpoint{4.739547in}{2.889639in}}%
\pgfpathlineto{\pgfqpoint{4.704761in}{2.890882in}}%
\pgfpathlineto{\pgfqpoint{4.602523in}{2.891538in}}%
\pgfpathlineto{\pgfqpoint{3.952099in}{2.891742in}}%
\pgfpathlineto{\pgfqpoint{0.617320in}{2.890753in}}%
\pgfpathlineto{\pgfqpoint{0.549909in}{2.888858in}}%
\pgfpathlineto{\pgfqpoint{0.521734in}{2.886179in}}%
\pgfpathlineto{\pgfqpoint{0.504665in}{2.882389in}}%
\pgfpathlineto{\pgfqpoint{0.494500in}{2.878011in}}%
\pgfpathlineto{\pgfqpoint{0.487179in}{2.872666in}}%
\pgfpathlineto{\pgfqpoint{0.481151in}{2.865518in}}%
\pgfpathlineto{\pgfqpoint{0.475663in}{2.854803in}}%
\pgfpathlineto{\pgfqpoint{0.471318in}{2.840736in}}%
\pgfpathlineto{\pgfqpoint{0.467301in}{2.818822in}}%
\pgfpathlineto{\pgfqpoint{0.463927in}{2.786699in}}%
\pgfpathlineto{\pgfqpoint{0.460918in}{2.734544in}}%
\pgfpathlineto{\pgfqpoint{0.458363in}{2.647473in}}%
\pgfpathlineto{\pgfqpoint{0.456575in}{2.523030in}}%
\pgfpathlineto{\pgfqpoint{0.456575in}{2.523030in}}%
\pgfusepath{stroke}%
\end{pgfscope}%
\begin{pgfscope}%
\pgfpathrectangle{\pgfqpoint{0.448634in}{0.402556in}}{\pgfqpoint{4.350661in}{2.489204in}} %
\pgfusepath{clip}%
\pgfsetrectcap%
\pgfsetroundjoin%
\pgfsetlinewidth{1.003750pt}%
\definecolor{currentstroke}{rgb}{0.580392,0.403922,0.741176}%
\pgfsetstrokecolor{currentstroke}%
\pgfsetdash{}{0pt}%
\pgfpathmoveto{\pgfqpoint{4.798840in}{2.852369in}}%
\pgfpathlineto{\pgfqpoint{4.797564in}{2.889610in}}%
\pgfpathlineto{\pgfqpoint{4.796215in}{2.891483in}}%
\pgfpathlineto{\pgfqpoint{4.787551in}{2.891760in}}%
\pgfpathlineto{\pgfqpoint{0.452128in}{2.891659in}}%
\pgfpathlineto{\pgfqpoint{0.450530in}{2.890082in}}%
\pgfpathlineto{\pgfqpoint{0.449454in}{2.882763in}}%
\pgfpathlineto{\pgfqpoint{0.448970in}{2.845432in}}%
\pgfpathlineto{\pgfqpoint{0.448743in}{2.494454in}}%
\pgfpathlineto{\pgfqpoint{0.449624in}{0.615107in}}%
\pgfpathlineto{\pgfqpoint{0.451433in}{0.510586in}}%
\pgfpathlineto{\pgfqpoint{0.453993in}{0.473374in}}%
\pgfpathlineto{\pgfqpoint{0.457406in}{0.453868in}}%
\pgfpathlineto{\pgfqpoint{0.461540in}{0.442384in}}%
\pgfpathlineto{\pgfqpoint{0.466739in}{0.434437in}}%
\pgfpathlineto{\pgfqpoint{0.473595in}{0.428350in}}%
\pgfpathlineto{\pgfqpoint{0.483492in}{0.423244in}}%
\pgfpathlineto{\pgfqpoint{0.491854in}{0.420501in}}%
\pgfpathlineto{\pgfqpoint{0.491854in}{0.420501in}}%
\pgfusepath{stroke}%
\end{pgfscope}%
\begin{pgfscope}%
\pgfpathrectangle{\pgfqpoint{0.448634in}{0.402556in}}{\pgfqpoint{4.350661in}{2.489204in}} %
\pgfusepath{clip}%
\pgfsetrectcap%
\pgfsetroundjoin%
\pgfsetlinewidth{1.003750pt}%
\definecolor{currentstroke}{rgb}{0.580392,0.403922,0.741176}%
\pgfsetstrokecolor{currentstroke}%
\pgfsetdash{}{0pt}%
\pgfpathmoveto{\pgfqpoint{0.456424in}{1.370168in}}%
\pgfpathlineto{\pgfqpoint{0.459610in}{1.118786in}}%
\pgfpathlineto{\pgfqpoint{0.463693in}{0.962038in}}%
\pgfpathlineto{\pgfqpoint{0.468517in}{0.857641in}}%
\pgfpathlineto{\pgfqpoint{0.474079in}{0.783241in}}%
\pgfpathlineto{\pgfqpoint{0.480221in}{0.728937in}}%
\pgfpathlineto{\pgfqpoint{0.486964in}{0.687336in}}%
\pgfpathlineto{\pgfqpoint{0.494529in}{0.653588in}}%
\pgfpathlineto{\pgfqpoint{0.503097in}{0.625384in}}%
\pgfpathlineto{\pgfqpoint{0.512180in}{0.602777in}}%
\pgfpathlineto{\pgfqpoint{0.522185in}{0.583533in}}%
\pgfpathlineto{\pgfqpoint{0.534090in}{0.565766in}}%
\pgfpathlineto{\pgfqpoint{0.546243in}{0.551528in}}%
\pgfpathlineto{\pgfqpoint{0.559706in}{0.538925in}}%
\pgfpathlineto{\pgfqpoint{0.576105in}{0.526709in}}%
\pgfpathlineto{\pgfqpoint{0.595458in}{0.515363in}}%
\pgfpathlineto{\pgfqpoint{0.617654in}{0.505157in}}%
\pgfpathlineto{\pgfqpoint{0.642541in}{0.496161in}}%
\pgfpathlineto{\pgfqpoint{0.672098in}{0.487784in}}%
\pgfpathlineto{\pgfqpoint{0.708416in}{0.479828in}}%
\pgfpathlineto{\pgfqpoint{0.753622in}{0.472329in}}%
\pgfpathlineto{\pgfqpoint{0.807690in}{0.465663in}}%
\pgfpathlineto{\pgfqpoint{0.877088in}{0.459477in}}%
\pgfpathlineto{\pgfqpoint{0.961800in}{0.454231in}}%
\pgfpathlineto{\pgfqpoint{1.068324in}{0.449917in}}%
\pgfpathlineto{\pgfqpoint{1.200991in}{0.446840in}}%
\pgfpathlineto{\pgfqpoint{1.357609in}{0.445481in}}%
\pgfpathlineto{\pgfqpoint{1.525107in}{0.446231in}}%
\pgfpathlineto{\pgfqpoint{1.686060in}{0.449141in}}%
\pgfpathlineto{\pgfqpoint{1.823046in}{0.453746in}}%
\pgfpathlineto{\pgfqpoint{1.938217in}{0.459763in}}%
\pgfpathlineto{\pgfqpoint{2.031554in}{0.466757in}}%
\pgfpathlineto{\pgfqpoint{2.109553in}{0.474742in}}%
\pgfpathlineto{\pgfqpoint{2.174356in}{0.483530in}}%
\pgfpathlineto{\pgfqpoint{2.228112in}{0.492935in}}%
\pgfpathlineto{\pgfqpoint{2.275092in}{0.503350in}}%
\pgfpathlineto{\pgfqpoint{2.315255in}{0.514493in}}%
\pgfpathlineto{\pgfqpoint{2.350672in}{0.526649in}}%
\pgfpathlineto{\pgfqpoint{2.381295in}{0.539524in}}%
\pgfpathlineto{\pgfqpoint{2.407139in}{0.552645in}}%
\pgfpathlineto{\pgfqpoint{2.430202in}{0.566623in}}%
\pgfpathlineto{\pgfqpoint{2.452259in}{0.582583in}}%
\pgfpathlineto{\pgfqpoint{2.471369in}{0.599049in}}%
\pgfpathlineto{\pgfqpoint{2.489220in}{0.617271in}}%
\pgfpathlineto{\pgfqpoint{2.505660in}{0.637156in}}%
\pgfpathlineto{\pgfqpoint{2.520604in}{0.658531in}}%
\pgfpathlineto{\pgfqpoint{2.535199in}{0.683287in}}%
\pgfpathlineto{\pgfqpoint{2.549103in}{0.711456in}}%
\pgfpathlineto{\pgfqpoint{2.562080in}{0.742975in}}%
\pgfpathlineto{\pgfqpoint{2.574011in}{0.777721in}}%
\pgfpathlineto{\pgfqpoint{2.585494in}{0.817940in}}%
\pgfpathlineto{\pgfqpoint{2.596802in}{0.866007in}}%
\pgfpathlineto{\pgfqpoint{2.607557in}{0.921916in}}%
\pgfpathlineto{\pgfqpoint{2.617920in}{0.988067in}}%
\pgfpathlineto{\pgfqpoint{2.627954in}{1.066886in}}%
\pgfpathlineto{\pgfqpoint{2.637938in}{1.163288in}}%
\pgfpathlineto{\pgfqpoint{2.648422in}{1.287167in}}%
\pgfpathlineto{\pgfqpoint{2.660101in}{1.453406in}}%
\pgfpathlineto{\pgfqpoint{2.674771in}{1.696769in}}%
\pgfpathlineto{\pgfqpoint{2.687715in}{1.945247in}}%
\pgfpathlineto{\pgfqpoint{2.692669in}{2.079542in}}%
\pgfpathlineto{\pgfqpoint{2.693823in}{2.169139in}}%
\pgfpathlineto{\pgfqpoint{2.692566in}{2.233838in}}%
\pgfpathlineto{\pgfqpoint{2.689439in}{2.285983in}}%
\pgfpathlineto{\pgfqpoint{2.684863in}{2.327968in}}%
\pgfpathlineto{\pgfqpoint{2.678730in}{2.364633in}}%
\pgfpathlineto{\pgfqpoint{2.671364in}{2.395867in}}%
\pgfpathlineto{\pgfqpoint{2.662499in}{2.423952in}}%
\pgfpathlineto{\pgfqpoint{2.652374in}{2.448750in}}%
\pgfpathlineto{\pgfqpoint{2.641379in}{2.470219in}}%
\pgfpathlineto{\pgfqpoint{2.628660in}{2.490400in}}%
\pgfpathlineto{\pgfqpoint{2.614297in}{2.509083in}}%
\pgfpathlineto{\pgfqpoint{2.598463in}{2.526138in}}%
\pgfpathlineto{\pgfqpoint{2.579612in}{2.542987in}}%
\pgfpathlineto{\pgfqpoint{2.559555in}{2.557907in}}%
\pgfpathlineto{\pgfqpoint{2.536626in}{2.572169in}}%
\pgfpathlineto{\pgfqpoint{2.510875in}{2.585526in}}%
\pgfpathlineto{\pgfqpoint{2.482385in}{2.597827in}}%
\pgfpathlineto{\pgfqpoint{2.449161in}{2.609674in}}%
\pgfpathlineto{\pgfqpoint{2.411211in}{2.620689in}}%
\pgfpathlineto{\pgfqpoint{2.368579in}{2.630600in}}%
\pgfpathlineto{\pgfqpoint{2.321321in}{2.639216in}}%
\pgfpathlineto{\pgfqpoint{2.269494in}{2.646395in}}%
\pgfpathlineto{\pgfqpoint{2.210981in}{2.652191in}}%
\pgfpathlineto{\pgfqpoint{2.147994in}{2.656152in}}%
\pgfpathlineto{\pgfqpoint{2.080583in}{2.658134in}}%
\pgfpathlineto{\pgfqpoint{2.010975in}{2.657972in}}%
\pgfpathlineto{\pgfqpoint{1.939222in}{2.655574in}}%
\pgfpathlineto{\pgfqpoint{1.867554in}{2.650915in}}%
\pgfpathlineto{\pgfqpoint{1.798198in}{2.644143in}}%
\pgfpathlineto{\pgfqpoint{1.733368in}{2.635610in}}%
\pgfpathlineto{\pgfqpoint{1.673102in}{2.625526in}}%
\pgfpathlineto{\pgfqpoint{1.615301in}{2.613616in}}%
\pgfpathlineto{\pgfqpoint{1.562159in}{2.600409in}}%
\pgfpathlineto{\pgfqpoint{1.513707in}{2.586148in}}%
\pgfpathlineto{\pgfqpoint{1.467887in}{2.570353in}}%
\pgfpathlineto{\pgfqpoint{1.426819in}{2.553934in}}%
\pgfpathlineto{\pgfqpoint{1.388472in}{2.536301in}}%
\pgfpathlineto{\pgfqpoint{1.352902in}{2.517580in}}%
\pgfpathlineto{\pgfqpoint{1.320152in}{2.497937in}}%
\pgfpathlineto{\pgfqpoint{1.288402in}{2.476253in}}%
\pgfpathlineto{\pgfqpoint{1.259614in}{2.453879in}}%
\pgfpathlineto{\pgfqpoint{1.232071in}{2.429540in}}%
\pgfpathlineto{\pgfqpoint{1.207547in}{2.404920in}}%
\pgfpathlineto{\pgfqpoint{1.184427in}{2.378580in}}%
\pgfpathlineto{\pgfqpoint{1.162845in}{2.350585in}}%
\pgfpathlineto{\pgfqpoint{1.142906in}{2.321037in}}%
\pgfpathlineto{\pgfqpoint{1.124689in}{2.290068in}}%
\pgfpathlineto{\pgfqpoint{1.108238in}{2.257830in}}%
\pgfpathlineto{\pgfqpoint{1.092650in}{2.222228in}}%
\pgfpathlineto{\pgfqpoint{1.079069in}{2.185565in}}%
\pgfpathlineto{\pgfqpoint{1.067451in}{2.148028in}}%
\pgfpathlineto{\pgfqpoint{1.057194in}{2.107378in}}%
\pgfpathlineto{\pgfqpoint{1.049009in}{2.066117in}}%
\pgfpathlineto{\pgfqpoint{1.042516in}{2.021938in}}%
\pgfpathlineto{\pgfqpoint{1.038179in}{1.977413in}}%
\pgfpathlineto{\pgfqpoint{1.035866in}{1.930199in}}%
\pgfpathlineto{\pgfqpoint{1.035825in}{1.882910in}}%
\pgfpathlineto{\pgfqpoint{1.038028in}{1.835688in}}%
\pgfpathlineto{\pgfqpoint{1.042470in}{1.788673in}}%
\pgfpathlineto{\pgfqpoint{1.049170in}{1.742010in}}%
\pgfpathlineto{\pgfqpoint{1.057636in}{1.698270in}}%
\pgfpathlineto{\pgfqpoint{1.068213in}{1.655136in}}%
\pgfpathlineto{\pgfqpoint{1.080952in}{1.612774in}}%
\pgfpathlineto{\pgfqpoint{1.095019in}{1.573645in}}%
\pgfpathlineto{\pgfqpoint{1.111102in}{1.535548in}}%
\pgfpathlineto{\pgfqpoint{1.128102in}{1.500802in}}%
\pgfpathlineto{\pgfqpoint{1.146913in}{1.467300in}}%
\pgfpathlineto{\pgfqpoint{1.167513in}{1.435205in}}%
\pgfpathlineto{\pgfqpoint{1.189855in}{1.404675in}}%
\pgfpathlineto{\pgfqpoint{1.213863in}{1.375850in}}%
\pgfpathlineto{\pgfqpoint{1.237795in}{1.350477in}}%
\pgfpathlineto{\pgfqpoint{1.264725in}{1.325256in}}%
\pgfpathlineto{\pgfqpoint{1.292966in}{1.301989in}}%
\pgfpathlineto{\pgfqpoint{1.322373in}{1.280693in}}%
\pgfpathlineto{\pgfqpoint{1.352794in}{1.261354in}}%
\pgfpathlineto{\pgfqpoint{1.386068in}{1.242901in}}%
\pgfpathlineto{\pgfqpoint{1.420163in}{1.226526in}}%
\pgfpathlineto{\pgfqpoint{1.456996in}{1.211338in}}%
\pgfpathlineto{\pgfqpoint{1.496526in}{1.197543in}}%
\pgfpathlineto{\pgfqpoint{1.538691in}{1.185292in}}%
\pgfpathlineto{\pgfqpoint{1.583412in}{1.174645in}}%
\pgfpathlineto{\pgfqpoint{1.634900in}{1.164777in}}%
\pgfpathlineto{\pgfqpoint{1.706034in}{1.153746in}}%
\pgfpathlineto{\pgfqpoint{1.768463in}{1.143419in}}%
\pgfpathlineto{\pgfqpoint{1.796094in}{1.136571in}}%
\pgfpathlineto{\pgfqpoint{1.812655in}{1.130487in}}%
\pgfpathlineto{\pgfqpoint{1.824444in}{1.124111in}}%
\pgfpathlineto{\pgfqpoint{1.833185in}{1.116754in}}%
\pgfpathlineto{\pgfqpoint{1.838478in}{1.108906in}}%
\pgfpathlineto{\pgfqpoint{1.840572in}{1.101866in}}%
\pgfpathlineto{\pgfqpoint{1.840607in}{1.094430in}}%
\pgfpathlineto{\pgfqpoint{1.837923in}{1.085002in}}%
\pgfpathlineto{\pgfqpoint{1.833241in}{1.076630in}}%
\pgfpathlineto{\pgfqpoint{1.825815in}{1.067555in}}%
\pgfpathlineto{\pgfqpoint{1.813810in}{1.056861in}}%
\pgfpathlineto{\pgfqpoint{1.798817in}{1.046773in}}%
\pgfpathlineto{\pgfqpoint{1.781015in}{1.037470in}}%
\pgfpathlineto{\pgfqpoint{1.758446in}{1.028397in}}%
\pgfpathlineto{\pgfqpoint{1.733202in}{1.020820in}}%
\pgfpathlineto{\pgfqpoint{1.705410in}{1.014876in}}%
\pgfpathlineto{\pgfqpoint{1.675178in}{1.010717in}}%
\pgfpathlineto{\pgfqpoint{1.642610in}{1.008509in}}%
\pgfpathlineto{\pgfqpoint{1.607809in}{1.008433in}}%
\pgfpathlineto{\pgfqpoint{1.570886in}{1.010692in}}%
\pgfpathlineto{\pgfqpoint{1.534118in}{1.015181in}}%
\pgfpathlineto{\pgfqpoint{1.495454in}{1.022233in}}%
\pgfpathlineto{\pgfqpoint{1.457161in}{1.031563in}}%
\pgfpathlineto{\pgfqpoint{1.419337in}{1.043131in}}%
\pgfpathlineto{\pgfqpoint{1.382089in}{1.056928in}}%
\pgfpathlineto{\pgfqpoint{1.347544in}{1.072019in}}%
\pgfpathlineto{\pgfqpoint{1.313727in}{1.089133in}}%
\pgfpathlineto{\pgfqpoint{1.280762in}{1.108299in}}%
\pgfpathlineto{\pgfqpoint{1.248782in}{1.129536in}}%
\pgfpathlineto{\pgfqpoint{1.219708in}{1.151423in}}%
\pgfpathlineto{\pgfqpoint{1.191752in}{1.175138in}}%
\pgfpathlineto{\pgfqpoint{1.165031in}{1.200649in}}%
\pgfpathlineto{\pgfqpoint{1.139653in}{1.227898in}}%
\pgfpathlineto{\pgfqpoint{1.115714in}{1.256800in}}%
\pgfpathlineto{\pgfqpoint{1.093288in}{1.287251in}}%
\pgfpathlineto{\pgfqpoint{1.071178in}{1.321163in}}%
\pgfpathlineto{\pgfqpoint{1.050868in}{1.356520in}}%
\pgfpathlineto{\pgfqpoint{1.032365in}{1.393152in}}%
\pgfpathlineto{\pgfqpoint{1.014718in}{1.433142in}}%
\pgfpathlineto{\pgfqpoint{0.999024in}{1.474185in}}%
\pgfpathlineto{\pgfqpoint{0.984506in}{1.518461in}}%
\pgfpathlineto{\pgfqpoint{0.972009in}{1.563537in}}%
\pgfpathlineto{\pgfqpoint{0.960943in}{1.611678in}}%
\pgfpathlineto{\pgfqpoint{0.951530in}{1.662824in}}%
\pgfpathlineto{\pgfqpoint{0.944286in}{1.714431in}}%
\pgfpathlineto{\pgfqpoint{0.938950in}{1.768847in}}%
\pgfpathlineto{\pgfqpoint{0.935870in}{1.823491in}}%
\pgfpathlineto{\pgfqpoint{0.935034in}{1.878240in}}%
\pgfpathlineto{\pgfqpoint{0.936466in}{1.932973in}}%
\pgfpathlineto{\pgfqpoint{0.940005in}{1.985084in}}%
\pgfpathlineto{\pgfqpoint{0.945759in}{2.036935in}}%
\pgfpathlineto{\pgfqpoint{0.953410in}{2.085938in}}%
\pgfpathlineto{\pgfqpoint{0.962764in}{2.132000in}}%
\pgfpathlineto{\pgfqpoint{0.974287in}{2.177414in}}%
\pgfpathlineto{\pgfqpoint{0.987332in}{2.219653in}}%
\pgfpathlineto{\pgfqpoint{1.001667in}{2.258654in}}%
\pgfpathlineto{\pgfqpoint{1.018051in}{2.296583in}}%
\pgfpathlineto{\pgfqpoint{1.035401in}{2.331101in}}%
\pgfpathlineto{\pgfqpoint{1.054650in}{2.364275in}}%
\pgfpathlineto{\pgfqpoint{1.074406in}{2.393984in}}%
\pgfpathlineto{\pgfqpoint{1.095771in}{2.422197in}}%
\pgfpathlineto{\pgfqpoint{1.118662in}{2.448797in}}%
\pgfpathlineto{\pgfqpoint{1.142966in}{2.473701in}}%
\pgfpathlineto{\pgfqpoint{1.168550in}{2.496867in}}%
\pgfpathlineto{\pgfqpoint{1.197085in}{2.519662in}}%
\pgfpathlineto{\pgfqpoint{1.226726in}{2.540526in}}%
\pgfpathlineto{\pgfqpoint{1.259242in}{2.560673in}}%
\pgfpathlineto{\pgfqpoint{1.294612in}{2.579881in}}%
\pgfpathlineto{\pgfqpoint{1.332792in}{2.597982in}}%
\pgfpathlineto{\pgfqpoint{1.373719in}{2.614859in}}%
\pgfpathlineto{\pgfqpoint{1.417319in}{2.630445in}}%
\pgfpathlineto{\pgfqpoint{1.465632in}{2.645312in}}%
\pgfpathlineto{\pgfqpoint{1.518640in}{2.659204in}}%
\pgfpathlineto{\pgfqpoint{1.576309in}{2.671929in}}%
\pgfpathlineto{\pgfqpoint{1.638597in}{2.683344in}}%
\pgfpathlineto{\pgfqpoint{1.705462in}{2.693343in}}%
\pgfpathlineto{\pgfqpoint{1.779027in}{2.702064in}}%
\pgfpathlineto{\pgfqpoint{1.857097in}{2.709076in}}%
\pgfpathlineto{\pgfqpoint{1.939633in}{2.714279in}}%
\pgfpathlineto{\pgfqpoint{2.026598in}{2.717513in}}%
\pgfpathlineto{\pgfqpoint{2.113605in}{2.718523in}}%
\pgfpathlineto{\pgfqpoint{2.198435in}{2.717302in}}%
\pgfpathlineto{\pgfqpoint{2.278866in}{2.713928in}}%
\pgfpathlineto{\pgfqpoint{2.352678in}{2.708597in}}%
\pgfpathlineto{\pgfqpoint{2.417657in}{2.701708in}}%
\pgfpathlineto{\pgfqpoint{2.473770in}{2.693629in}}%
\pgfpathlineto{\pgfqpoint{2.523140in}{2.684367in}}%
\pgfpathlineto{\pgfqpoint{2.565726in}{2.674201in}}%
\pgfpathlineto{\pgfqpoint{2.601510in}{2.663542in}}%
\pgfpathlineto{\pgfqpoint{2.632576in}{2.652141in}}%
\pgfpathlineto{\pgfqpoint{2.658899in}{2.640329in}}%
\pgfpathlineto{\pgfqpoint{2.682438in}{2.627434in}}%
\pgfpathlineto{\pgfqpoint{2.703062in}{2.613569in}}%
\pgfpathlineto{\pgfqpoint{2.720674in}{2.598976in}}%
\pgfpathlineto{\pgfqpoint{2.735262in}{2.584051in}}%
\pgfpathlineto{\pgfqpoint{2.748319in}{2.567374in}}%
\pgfpathlineto{\pgfqpoint{2.759552in}{2.549043in}}%
\pgfpathlineto{\pgfqpoint{2.768787in}{2.529303in}}%
\pgfpathlineto{\pgfqpoint{2.776015in}{2.508495in}}%
\pgfpathlineto{\pgfqpoint{2.781883in}{2.484537in}}%
\pgfpathlineto{\pgfqpoint{2.786101in}{2.457594in}}%
\pgfpathlineto{\pgfqpoint{2.788718in}{2.425381in}}%
\pgfpathlineto{\pgfqpoint{2.789426in}{2.388058in}}%
\pgfpathlineto{\pgfqpoint{2.787961in}{2.340798in}}%
\pgfpathlineto{\pgfqpoint{2.783671in}{2.278765in}}%
\pgfpathlineto{\pgfqpoint{2.774287in}{2.179780in}}%
\pgfpathlineto{\pgfqpoint{2.743610in}{1.868116in}}%
\pgfpathlineto{\pgfqpoint{2.730111in}{1.702057in}}%
\pgfpathlineto{\pgfqpoint{2.717286in}{1.515946in}}%
\pgfpathlineto{\pgfqpoint{2.702601in}{1.267594in}}%
\pgfpathlineto{\pgfqpoint{2.684433in}{0.964627in}}%
\pgfpathlineto{\pgfqpoint{2.675373in}{0.850597in}}%
\pgfpathlineto{\pgfqpoint{2.667028in}{0.771520in}}%
\pgfpathlineto{\pgfqpoint{2.658751in}{0.712540in}}%
\pgfpathlineto{\pgfqpoint{2.650174in}{0.666281in}}%
\pgfpathlineto{\pgfqpoint{2.640818in}{0.627928in}}%
\pgfpathlineto{\pgfqpoint{2.631142in}{0.597532in}}%
\pgfpathlineto{\pgfqpoint{2.621001in}{0.572742in}}%
\pgfpathlineto{\pgfqpoint{2.609853in}{0.551381in}}%
\pgfpathlineto{\pgfqpoint{2.598038in}{0.533532in}}%
\pgfpathlineto{\pgfqpoint{2.584491in}{0.517377in}}%
\pgfpathlineto{\pgfqpoint{2.571104in}{0.504668in}}%
\pgfpathlineto{\pgfqpoint{2.554784in}{0.492312in}}%
\pgfpathlineto{\pgfqpoint{2.537451in}{0.481913in}}%
\pgfpathlineto{\pgfqpoint{2.517369in}{0.472367in}}%
\pgfpathlineto{\pgfqpoint{2.492537in}{0.463178in}}%
\pgfpathlineto{\pgfqpoint{2.462974in}{0.454833in}}%
\pgfpathlineto{\pgfqpoint{2.428761in}{0.447543in}}%
\pgfpathlineto{\pgfqpoint{2.385666in}{0.440735in}}%
\pgfpathlineto{\pgfqpoint{2.331552in}{0.434582in}}%
\pgfpathlineto{\pgfqpoint{2.262110in}{0.429077in}}%
\pgfpathlineto{\pgfqpoint{2.170845in}{0.424236in}}%
\pgfpathlineto{\pgfqpoint{2.049081in}{0.420134in}}%
\pgfpathlineto{\pgfqpoint{1.879431in}{0.416784in}}%
\pgfpathlineto{\pgfqpoint{1.640154in}{0.414418in}}%
\pgfpathlineto{\pgfqpoint{1.322557in}{0.413570in}}%
\pgfpathlineto{\pgfqpoint{1.020189in}{0.414850in}}%
\pgfpathlineto{\pgfqpoint{0.822251in}{0.417715in}}%
\pgfpathlineto{\pgfqpoint{0.704830in}{0.421431in}}%
\pgfpathlineto{\pgfqpoint{0.630971in}{0.425830in}}%
\pgfpathlineto{\pgfqpoint{0.583311in}{0.430736in}}%
\pgfpathlineto{\pgfqpoint{0.551028in}{0.436125in}}%
\pgfpathlineto{\pgfqpoint{0.527703in}{0.442192in}}%
\pgfpathlineto{\pgfqpoint{0.511245in}{0.448629in}}%
\pgfpathlineto{\pgfqpoint{0.499544in}{0.455220in}}%
\pgfpathlineto{\pgfqpoint{0.488912in}{0.463847in}}%
\pgfpathlineto{\pgfqpoint{0.481319in}{0.472737in}}%
\pgfpathlineto{\pgfqpoint{0.474076in}{0.485134in}}%
\pgfpathlineto{\pgfqpoint{0.468752in}{0.498755in}}%
\pgfpathlineto{\pgfqpoint{0.463869in}{0.517856in}}%
\pgfpathlineto{\pgfqpoint{0.459678in}{0.544804in}}%
\pgfpathlineto{\pgfqpoint{0.456386in}{0.581946in}}%
\pgfpathlineto{\pgfqpoint{0.453731in}{0.639114in}}%
\pgfpathlineto{\pgfqpoint{0.451681in}{0.736162in}}%
\pgfpathlineto{\pgfqpoint{0.450220in}{0.927823in}}%
\pgfpathlineto{\pgfqpoint{0.449345in}{1.403260in}}%
\pgfpathlineto{\pgfqpoint{0.449543in}{2.682710in}}%
\pgfpathlineto{\pgfqpoint{0.451011in}{2.856940in}}%
\pgfpathlineto{\pgfqpoint{0.452804in}{2.879227in}}%
\pgfpathlineto{\pgfqpoint{0.455192in}{2.886114in}}%
\pgfpathlineto{\pgfqpoint{0.458632in}{2.889031in}}%
\pgfpathlineto{\pgfqpoint{0.465003in}{2.890554in}}%
\pgfpathlineto{\pgfqpoint{0.482383in}{2.891423in}}%
\pgfpathlineto{\pgfqpoint{0.565045in}{2.891729in}}%
\pgfpathlineto{\pgfqpoint{2.733849in}{2.891760in}}%
\pgfpathlineto{\pgfqpoint{4.789517in}{2.890884in}}%
\pgfpathlineto{\pgfqpoint{4.793733in}{2.889727in}}%
\pgfpathlineto{\pgfqpoint{4.795485in}{2.888300in}}%
\pgfpathlineto{\pgfqpoint{4.797106in}{2.881137in}}%
\pgfpathlineto{\pgfqpoint{4.797997in}{2.858763in}}%
\pgfpathlineto{\pgfqpoint{4.798039in}{2.856275in}}%
\pgfpathlineto{\pgfqpoint{4.798039in}{2.856275in}}%
\pgfusepath{stroke}%
\end{pgfscope}%
\begin{pgfscope}%
\pgfpathrectangle{\pgfqpoint{0.448634in}{0.402556in}}{\pgfqpoint{4.350661in}{2.489204in}} %
\pgfusepath{clip}%
\pgfsetrectcap%
\pgfsetroundjoin%
\pgfsetlinewidth{1.003750pt}%
\definecolor{currentstroke}{rgb}{0.580392,0.403922,0.741176}%
\pgfsetstrokecolor{currentstroke}%
\pgfsetdash{}{0pt}%
\pgfpathmoveto{\pgfqpoint{3.428778in}{0.402610in}}%
\pgfpathlineto{\pgfqpoint{2.806637in}{0.403762in}}%
\pgfpathlineto{\pgfqpoint{2.769698in}{0.405583in}}%
\pgfpathlineto{\pgfqpoint{2.754639in}{0.408073in}}%
\pgfpathlineto{\pgfqpoint{2.746398in}{0.411210in}}%
\pgfpathlineto{\pgfqpoint{2.740952in}{0.415280in}}%
\pgfpathlineto{\pgfqpoint{2.736794in}{0.421000in}}%
\pgfpathlineto{\pgfqpoint{2.733291in}{0.430087in}}%
\pgfpathlineto{\pgfqpoint{2.730457in}{0.444652in}}%
\pgfpathlineto{\pgfqpoint{2.728244in}{0.469408in}}%
\pgfpathlineto{\pgfqpoint{2.726475in}{0.519147in}}%
\pgfpathlineto{\pgfqpoint{2.725714in}{0.613730in}}%
\pgfpathlineto{\pgfqpoint{2.726844in}{0.768054in}}%
\pgfpathlineto{\pgfqpoint{2.730559in}{0.962164in}}%
\pgfpathlineto{\pgfqpoint{2.736613in}{1.158686in}}%
\pgfpathlineto{\pgfqpoint{2.744094in}{1.327734in}}%
\pgfpathlineto{\pgfqpoint{2.753204in}{1.484205in}}%
\pgfpathlineto{\pgfqpoint{2.763259in}{1.620625in}}%
\pgfpathlineto{\pgfqpoint{2.776122in}{1.764231in}}%
\pgfpathlineto{\pgfqpoint{2.788918in}{1.877792in}}%
\pgfpathlineto{\pgfqpoint{2.805753in}{2.005756in}}%
\pgfpathlineto{\pgfqpoint{2.821182in}{2.101213in}}%
\pgfpathlineto{\pgfqpoint{2.838366in}{2.193733in}}%
\pgfpathlineto{\pgfqpoint{2.859143in}{2.292980in}}%
\pgfpathlineto{\pgfqpoint{2.887218in}{2.425974in}}%
\pgfpathlineto{\pgfqpoint{2.897000in}{2.479574in}}%
\pgfpathlineto{\pgfqpoint{2.901551in}{2.516538in}}%
\pgfpathlineto{\pgfqpoint{2.902857in}{2.543868in}}%
\pgfpathlineto{\pgfqpoint{2.901965in}{2.566237in}}%
\pgfpathlineto{\pgfqpoint{2.899158in}{2.585877in}}%
\pgfpathlineto{\pgfqpoint{2.894800in}{2.602560in}}%
\pgfpathlineto{\pgfqpoint{2.888489in}{2.618401in}}%
\pgfpathlineto{\pgfqpoint{2.880261in}{2.633045in}}%
\pgfpathlineto{\pgfqpoint{2.870351in}{2.646258in}}%
\pgfpathlineto{\pgfqpoint{2.857402in}{2.659541in}}%
\pgfpathlineto{\pgfqpoint{2.843191in}{2.671020in}}%
\pgfpathlineto{\pgfqpoint{2.824239in}{2.683218in}}%
\pgfpathlineto{\pgfqpoint{2.802414in}{2.694426in}}%
\pgfpathlineto{\pgfqpoint{2.775809in}{2.705376in}}%
\pgfpathlineto{\pgfqpoint{2.744462in}{2.715721in}}%
\pgfpathlineto{\pgfqpoint{2.708436in}{2.725258in}}%
\pgfpathlineto{\pgfqpoint{2.665655in}{2.734294in}}%
\pgfpathlineto{\pgfqpoint{2.613992in}{2.742874in}}%
\pgfpathlineto{\pgfqpoint{2.553459in}{2.750593in}}%
\pgfpathlineto{\pgfqpoint{2.481920in}{2.757369in}}%
\pgfpathlineto{\pgfqpoint{2.399398in}{2.762843in}}%
\pgfpathlineto{\pgfqpoint{2.310269in}{2.766485in}}%
\pgfpathlineto{\pgfqpoint{2.175416in}{2.768728in}}%
\pgfpathlineto{\pgfqpoint{2.066653in}{2.767945in}}%
\pgfpathlineto{\pgfqpoint{1.953571in}{2.764863in}}%
\pgfpathlineto{\pgfqpoint{1.851429in}{2.759762in}}%
\pgfpathlineto{\pgfqpoint{1.745051in}{2.752172in}}%
\pgfpathlineto{\pgfqpoint{1.658374in}{2.743457in}}%
\pgfpathlineto{\pgfqpoint{1.580552in}{2.733465in}}%
\pgfpathlineto{\pgfqpoint{1.490058in}{2.719342in}}%
\pgfpathlineto{\pgfqpoint{1.417232in}{2.704698in}}%
\pgfpathlineto{\pgfqpoint{1.361992in}{2.690818in}}%
\pgfpathlineto{\pgfqpoint{1.311460in}{2.675818in}}%
\pgfpathlineto{\pgfqpoint{1.265667in}{2.659923in}}%
\pgfpathlineto{\pgfqpoint{1.222575in}{2.642586in}}%
\pgfpathlineto{\pgfqpoint{1.184324in}{2.624682in}}%
\pgfpathlineto{\pgfqpoint{1.148893in}{2.605623in}}%
\pgfpathlineto{\pgfqpoint{1.116332in}{2.585573in}}%
\pgfpathlineto{\pgfqpoint{1.092328in}{2.568511in}}%
\pgfpathlineto{\pgfqpoint{1.079761in}{2.558685in}}%
\pgfpathlineto{\pgfqpoint{1.051544in}{2.535378in}}%
\pgfpathlineto{\pgfqpoint{1.026313in}{2.511712in}}%
\pgfpathlineto{\pgfqpoint{1.002399in}{2.486317in}}%
\pgfpathlineto{\pgfqpoint{0.979913in}{2.459269in}}%
\pgfpathlineto{\pgfqpoint{0.958935in}{2.430678in}}%
\pgfpathlineto{\pgfqpoint{0.938265in}{2.398643in}}%
\pgfpathlineto{\pgfqpoint{0.923048in}{2.371384in}}%
\pgfpathlineto{\pgfqpoint{0.904514in}{2.334773in}}%
\pgfpathlineto{\pgfqpoint{0.887855in}{2.297000in}}%
\pgfpathlineto{\pgfqpoint{0.872133in}{2.255971in}}%
\pgfpathlineto{\pgfqpoint{0.857509in}{2.211740in}}%
\pgfpathlineto{\pgfqpoint{0.844763in}{2.166756in}}%
\pgfpathlineto{\pgfqpoint{0.838623in}{2.140306in}}%
\pgfpathlineto{\pgfqpoint{0.826981in}{2.087193in}}%
\pgfpathlineto{\pgfqpoint{0.816321in}{2.028715in}}%
\pgfpathlineto{\pgfqpoint{0.810086in}{1.984494in}}%
\pgfpathlineto{\pgfqpoint{0.808026in}{1.967238in}}%
\pgfpathlineto{\pgfqpoint{0.800076in}{1.898140in}}%
\pgfpathlineto{\pgfqpoint{0.793713in}{1.823823in}}%
\pgfpathlineto{\pgfqpoint{0.788798in}{1.741874in}}%
\pgfpathlineto{\pgfqpoint{0.786200in}{1.677225in}}%
\pgfpathlineto{\pgfqpoint{0.776951in}{1.453481in}}%
\pgfpathlineto{\pgfqpoint{0.773280in}{1.418894in}}%
\pgfpathlineto{\pgfqpoint{0.768298in}{1.389582in}}%
\pgfpathlineto{\pgfqpoint{0.762752in}{1.368108in}}%
\pgfpathlineto{\pgfqpoint{0.756722in}{1.352122in}}%
\pgfpathlineto{\pgfqpoint{0.749752in}{1.339519in}}%
\pgfpathlineto{\pgfqpoint{0.742201in}{1.330599in}}%
\pgfpathlineto{\pgfqpoint{0.734854in}{1.325311in}}%
\pgfpathlineto{\pgfqpoint{0.726557in}{1.322419in}}%
\pgfpathlineto{\pgfqpoint{0.717883in}{1.322223in}}%
\pgfpathlineto{\pgfqpoint{0.709412in}{1.324411in}}%
\pgfpathlineto{\pgfqpoint{0.699548in}{1.329604in}}%
\pgfpathlineto{\pgfqpoint{0.688894in}{1.338203in}}%
\pgfpathlineto{\pgfqpoint{0.677907in}{1.350248in}}%
\pgfpathlineto{\pgfqpoint{0.666886in}{1.365647in}}%
\pgfpathlineto{\pgfqpoint{0.654912in}{1.386417in}}%
\pgfpathlineto{\pgfqpoint{0.642574in}{1.412730in}}%
\pgfpathlineto{\pgfqpoint{0.630328in}{1.444629in}}%
\pgfpathlineto{\pgfqpoint{0.618504in}{1.482081in}}%
\pgfpathlineto{\pgfqpoint{0.608614in}{1.520256in}}%
\pgfpathlineto{\pgfqpoint{0.589795in}{1.614890in}}%
\pgfpathlineto{\pgfqpoint{0.581511in}{1.671344in}}%
\pgfpathlineto{\pgfqpoint{0.572873in}{1.742847in}}%
\pgfpathlineto{\pgfqpoint{0.566850in}{1.809691in}}%
\pgfpathlineto{\pgfqpoint{0.560380in}{1.898993in}}%
\pgfpathlineto{\pgfqpoint{0.555428in}{1.998397in}}%
\pgfpathlineto{\pgfqpoint{0.552515in}{2.100397in}}%
\pgfpathlineto{\pgfqpoint{0.551528in}{2.207424in}}%
\pgfpathlineto{\pgfqpoint{0.552728in}{2.309470in}}%
\pgfpathlineto{\pgfqpoint{0.556011in}{2.403981in}}%
\pgfpathlineto{\pgfqpoint{0.560952in}{2.483431in}}%
\pgfpathlineto{\pgfqpoint{0.567303in}{2.550241in}}%
\pgfpathlineto{\pgfqpoint{0.574927in}{2.606817in}}%
\pgfpathlineto{\pgfqpoint{0.582987in}{2.650657in}}%
\pgfpathlineto{\pgfqpoint{0.592755in}{2.691453in}}%
\pgfpathlineto{\pgfqpoint{0.602650in}{2.721756in}}%
\pgfpathlineto{\pgfqpoint{0.612983in}{2.746442in}}%
\pgfpathlineto{\pgfqpoint{0.624292in}{2.767692in}}%
\pgfpathlineto{\pgfqpoint{0.636231in}{2.785433in}}%
\pgfpathlineto{\pgfqpoint{0.649892in}{2.801461in}}%
\pgfpathlineto{\pgfqpoint{0.663386in}{2.814020in}}%
\pgfpathlineto{\pgfqpoint{0.679842in}{2.826135in}}%
\pgfpathlineto{\pgfqpoint{0.697326in}{2.836197in}}%
\pgfpathlineto{\pgfqpoint{0.715574in}{2.844285in}}%
\pgfpathlineto{\pgfqpoint{0.738439in}{2.852335in}}%
\pgfpathlineto{\pgfqpoint{0.765983in}{2.859639in}}%
\pgfpathlineto{\pgfqpoint{0.800300in}{2.866257in}}%
\pgfpathlineto{\pgfqpoint{0.841340in}{2.871832in}}%
\pgfpathlineto{\pgfqpoint{0.895547in}{2.876803in}}%
\pgfpathlineto{\pgfqpoint{0.969412in}{2.881069in}}%
\pgfpathlineto{\pgfqpoint{1.071608in}{2.884501in}}%
\pgfpathlineto{\pgfqpoint{1.219512in}{2.887074in}}%
\pgfpathlineto{\pgfqpoint{1.471844in}{2.889091in}}%
\pgfpathlineto{\pgfqpoint{1.956941in}{2.890384in}}%
\pgfpathlineto{\pgfqpoint{3.096814in}{2.890781in}}%
\pgfpathlineto{\pgfqpoint{3.995224in}{2.889388in}}%
\pgfpathlineto{\pgfqpoint{4.275833in}{2.887011in}}%
\pgfpathlineto{\pgfqpoint{4.412847in}{2.883743in}}%
\pgfpathlineto{\pgfqpoint{4.491081in}{2.879810in}}%
\pgfpathlineto{\pgfqpoint{4.543127in}{2.875163in}}%
\pgfpathlineto{\pgfqpoint{4.579810in}{2.869841in}}%
\pgfpathlineto{\pgfqpoint{4.607579in}{2.863763in}}%
\pgfpathlineto{\pgfqpoint{4.630623in}{2.856423in}}%
\pgfpathlineto{\pgfqpoint{4.648833in}{2.848228in}}%
\pgfpathlineto{\pgfqpoint{4.664136in}{2.838773in}}%
\pgfpathlineto{\pgfqpoint{4.676470in}{2.828576in}}%
\pgfpathlineto{\pgfqpoint{4.687502in}{2.816585in}}%
\pgfpathlineto{\pgfqpoint{4.697051in}{2.803027in}}%
\pgfpathlineto{\pgfqpoint{4.706194in}{2.786098in}}%
\pgfpathlineto{\pgfqpoint{4.714508in}{2.765827in}}%
\pgfpathlineto{\pgfqpoint{4.722462in}{2.740013in}}%
\pgfpathlineto{\pgfqpoint{4.729577in}{2.708703in}}%
\pgfpathlineto{\pgfqpoint{4.736162in}{2.669601in}}%
\pgfpathlineto{\pgfqpoint{4.742419in}{2.617826in}}%
\pgfpathlineto{\pgfqpoint{4.747859in}{2.553410in}}%
\pgfpathlineto{\pgfqpoint{4.752661in}{2.468958in}}%
\pgfpathlineto{\pgfqpoint{4.756610in}{2.359528in}}%
\pgfpathlineto{\pgfqpoint{4.759416in}{2.217681in}}%
\pgfpathlineto{\pgfqpoint{4.760596in}{2.043444in}}%
\pgfpathlineto{\pgfqpoint{4.759662in}{1.851779in}}%
\pgfpathlineto{\pgfqpoint{4.756587in}{1.667613in}}%
\pgfpathlineto{\pgfqpoint{4.751596in}{1.503428in}}%
\pgfpathlineto{\pgfqpoint{4.745410in}{1.374185in}}%
\pgfpathlineto{\pgfqpoint{4.738113in}{1.267479in}}%
\pgfpathlineto{\pgfqpoint{4.729620in}{1.175896in}}%
\pgfpathlineto{\pgfqpoint{4.720762in}{1.104428in}}%
\pgfpathlineto{\pgfqpoint{4.711044in}{1.043205in}}%
\pgfpathlineto{\pgfqpoint{4.700364in}{0.989829in}}%
\pgfpathlineto{\pgfqpoint{4.689054in}{0.944345in}}%
\pgfpathlineto{\pgfqpoint{4.676880in}{0.904394in}}%
\pgfpathlineto{\pgfqpoint{4.676095in}{0.902073in}}%
\pgfpathlineto{\pgfqpoint{4.676095in}{0.902073in}}%
\pgfusepath{stroke}%
\end{pgfscope}%
\begin{pgfscope}%
\pgfpathrectangle{\pgfqpoint{0.448634in}{0.402556in}}{\pgfqpoint{4.350661in}{2.489204in}} %
\pgfusepath{clip}%
\pgfsetrectcap%
\pgfsetroundjoin%
\pgfsetlinewidth{1.003750pt}%
\definecolor{currentstroke}{rgb}{0.580392,0.403922,0.741176}%
\pgfsetstrokecolor{currentstroke}%
\pgfsetdash{}{0pt}%
\pgfpathmoveto{\pgfqpoint{2.795520in}{1.982745in}}%
\pgfpathlineto{\pgfqpoint{2.781780in}{1.874357in}}%
\pgfpathlineto{\pgfqpoint{2.769351in}{1.758234in}}%
\pgfpathlineto{\pgfqpoint{2.758095in}{1.631942in}}%
\pgfpathlineto{\pgfqpoint{2.747786in}{1.490551in}}%
\pgfpathlineto{\pgfqpoint{2.738644in}{1.334082in}}%
\pgfpathlineto{\pgfqpoint{2.730580in}{1.157591in}}%
\pgfpathlineto{\pgfqpoint{2.723334in}{0.948663in}}%
\pgfpathlineto{\pgfqpoint{2.709783in}{0.530788in}}%
\pgfpathlineto{\pgfqpoint{2.705868in}{0.488716in}}%
\pgfpathlineto{\pgfqpoint{2.701769in}{0.464281in}}%
\pgfpathlineto{\pgfqpoint{2.697021in}{0.447744in}}%
\pgfpathlineto{\pgfqpoint{2.691859in}{0.436812in}}%
\pgfpathlineto{\pgfqpoint{2.686244in}{0.429229in}}%
\pgfpathlineto{\pgfqpoint{2.679348in}{0.423188in}}%
\pgfpathlineto{\pgfqpoint{2.669540in}{0.417856in}}%
\pgfpathlineto{\pgfqpoint{2.656987in}{0.413810in}}%
\pgfpathlineto{\pgfqpoint{2.637654in}{0.410337in}}%
\pgfpathlineto{\pgfqpoint{2.607296in}{0.407617in}}%
\pgfpathlineto{\pgfqpoint{2.555121in}{0.405574in}}%
\pgfpathlineto{\pgfqpoint{2.450714in}{0.404139in}}%
\pgfpathlineto{\pgfqpoint{2.176623in}{0.403275in}}%
\pgfpathlineto{\pgfqpoint{1.130290in}{0.402953in}}%
\pgfpathlineto{\pgfqpoint{0.516849in}{0.404175in}}%
\pgfpathlineto{\pgfqpoint{0.466848in}{0.405970in}}%
\pgfpathlineto{\pgfqpoint{0.456129in}{0.407931in}}%
\pgfpathlineto{\pgfqpoint{0.452339in}{0.410304in}}%
\pgfpathlineto{\pgfqpoint{0.450346in}{0.414663in}}%
\pgfpathlineto{\pgfqpoint{0.449266in}{0.424524in}}%
\pgfpathlineto{\pgfqpoint{0.448771in}{0.464345in}}%
\pgfpathlineto{\pgfqpoint{0.448640in}{0.850171in}}%
\pgfpathlineto{\pgfqpoint{0.448679in}{2.891318in}}%
\pgfpathlineto{\pgfqpoint{0.448679in}{2.891318in}}%
\pgfusepath{stroke}%
\end{pgfscope}%
\begin{pgfscope}%
\pgfpathrectangle{\pgfqpoint{0.448634in}{0.402556in}}{\pgfqpoint{4.350661in}{2.489204in}} %
\pgfusepath{clip}%
\pgfsetrectcap%
\pgfsetroundjoin%
\pgfsetlinewidth{1.003750pt}%
\definecolor{currentstroke}{rgb}{0.580392,0.403922,0.741176}%
\pgfsetstrokecolor{currentstroke}%
\pgfsetdash{}{0pt}%
\pgfpathmoveto{\pgfqpoint{3.428210in}{0.402586in}}%
\pgfpathlineto{\pgfqpoint{2.782142in}{0.403706in}}%
\pgfpathlineto{\pgfqpoint{2.753927in}{0.405686in}}%
\pgfpathlineto{\pgfqpoint{2.743350in}{0.408462in}}%
\pgfpathlineto{\pgfqpoint{2.737740in}{0.412207in}}%
\pgfpathlineto{\pgfqpoint{2.733688in}{0.418012in}}%
\pgfpathlineto{\pgfqpoint{2.730665in}{0.427323in}}%
\pgfpathlineto{\pgfqpoint{2.728399in}{0.442019in}}%
\pgfpathlineto{\pgfqpoint{2.726552in}{0.471808in}}%
\pgfpathlineto{\pgfqpoint{2.725221in}{0.534017in}}%
\pgfpathlineto{\pgfqpoint{2.725173in}{0.655987in}}%
\pgfpathlineto{\pgfqpoint{2.727380in}{0.832701in}}%
\pgfpathlineto{\pgfqpoint{2.732261in}{1.041717in}}%
\pgfpathlineto{\pgfqpoint{2.738854in}{1.223271in}}%
\pgfpathlineto{\pgfqpoint{2.747081in}{1.389780in}}%
\pgfpathlineto{\pgfqpoint{2.756611in}{1.538731in}}%
\pgfpathlineto{\pgfqpoint{2.768958in}{1.694901in}}%
\pgfpathlineto{\pgfqpoint{2.781232in}{1.816058in}}%
\pgfpathlineto{\pgfqpoint{2.794406in}{1.924538in}}%
\pgfpathlineto{\pgfqpoint{2.812742in}{2.054736in}}%
\pgfpathlineto{\pgfqpoint{2.828780in}{2.147525in}}%
\pgfpathlineto{\pgfqpoint{2.847389in}{2.242237in}}%
\pgfpathlineto{\pgfqpoint{2.895825in}{2.479712in}}%
\pgfpathlineto{\pgfqpoint{2.900211in}{2.516702in}}%
\pgfpathlineto{\pgfqpoint{2.901353in}{2.544042in}}%
\pgfpathlineto{\pgfqpoint{2.900298in}{2.566401in}}%
\pgfpathlineto{\pgfqpoint{2.897340in}{2.586012in}}%
\pgfpathlineto{\pgfqpoint{2.892841in}{2.602646in}}%
\pgfpathlineto{\pgfqpoint{2.886399in}{2.618417in}}%
\pgfpathlineto{\pgfqpoint{2.878061in}{2.632981in}}%
\pgfpathlineto{\pgfqpoint{2.868067in}{2.646111in}}%
\pgfpathlineto{\pgfqpoint{2.855052in}{2.659310in}}%
\pgfpathlineto{\pgfqpoint{2.840802in}{2.670725in}}%
\pgfpathlineto{\pgfqpoint{2.821823in}{2.682868in}}%
\pgfpathlineto{\pgfqpoint{2.799981in}{2.694033in}}%
\pgfpathlineto{\pgfqpoint{2.773366in}{2.704950in}}%
\pgfpathlineto{\pgfqpoint{2.742012in}{2.715272in}}%
\pgfpathlineto{\pgfqpoint{2.705983in}{2.724790in}}%
\pgfpathlineto{\pgfqpoint{2.663200in}{2.733815in}}%
\pgfpathlineto{\pgfqpoint{2.611535in}{2.742383in}}%
\pgfpathlineto{\pgfqpoint{2.551002in}{2.750094in}}%
\pgfpathlineto{\pgfqpoint{2.481632in}{2.756685in}}%
\pgfpathlineto{\pgfqpoint{2.399112in}{2.762203in}}%
\pgfpathlineto{\pgfqpoint{2.309985in}{2.765888in}}%
\pgfpathlineto{\pgfqpoint{2.188184in}{2.768099in}}%
\pgfpathlineto{\pgfqpoint{2.081595in}{2.767621in}}%
\pgfpathlineto{\pgfqpoint{1.968506in}{2.764843in}}%
\pgfpathlineto{\pgfqpoint{1.864180in}{2.759921in}}%
\pgfpathlineto{\pgfqpoint{1.757786in}{2.752596in}}%
\pgfpathlineto{\pgfqpoint{1.671087in}{2.744174in}}%
\pgfpathlineto{\pgfqpoint{1.591075in}{2.734196in}}%
\pgfpathlineto{\pgfqpoint{1.502689in}{2.720721in}}%
\pgfpathlineto{\pgfqpoint{1.427655in}{2.706083in}}%
\pgfpathlineto{\pgfqpoint{1.372350in}{2.692544in}}%
\pgfpathlineto{\pgfqpoint{1.321734in}{2.677921in}}%
\pgfpathlineto{\pgfqpoint{1.273765in}{2.661664in}}%
\pgfpathlineto{\pgfqpoint{1.230567in}{2.644672in}}%
\pgfpathlineto{\pgfqpoint{1.192197in}{2.627106in}}%
\pgfpathlineto{\pgfqpoint{1.156620in}{2.608403in}}%
\pgfpathlineto{\pgfqpoint{1.123890in}{2.588716in}}%
\pgfpathlineto{\pgfqpoint{1.095883in}{2.569568in}}%
\pgfpathlineto{\pgfqpoint{1.063936in}{2.543701in}}%
\pgfpathlineto{\pgfqpoint{1.038217in}{2.520732in}}%
\pgfpathlineto{\pgfqpoint{1.013766in}{2.496016in}}%
\pgfpathlineto{\pgfqpoint{0.990704in}{2.469610in}}%
\pgfpathlineto{\pgfqpoint{0.969124in}{2.441612in}}%
\pgfpathlineto{\pgfqpoint{0.949082in}{2.412154in}}%
\pgfpathlineto{\pgfqpoint{0.930604in}{2.381387in}}%
\pgfpathlineto{\pgfqpoint{0.906555in}{2.334052in}}%
\pgfpathlineto{\pgfqpoint{0.889925in}{2.296262in}}%
\pgfpathlineto{\pgfqpoint{0.874241in}{2.255213in}}%
\pgfpathlineto{\pgfqpoint{0.859668in}{2.210961in}}%
\pgfpathlineto{\pgfqpoint{0.846986in}{2.165954in}}%
\pgfpathlineto{\pgfqpoint{0.839632in}{2.134715in}}%
\pgfpathlineto{\pgfqpoint{0.828237in}{2.081532in}}%
\pgfpathlineto{\pgfqpoint{0.817865in}{2.022986in}}%
\pgfpathlineto{\pgfqpoint{0.810783in}{1.971352in}}%
\pgfpathlineto{\pgfqpoint{0.802845in}{1.902253in}}%
\pgfpathlineto{\pgfqpoint{0.796553in}{1.827928in}}%
\pgfpathlineto{\pgfqpoint{0.791695in}{1.743480in}}%
\pgfpathlineto{\pgfqpoint{0.787772in}{1.621595in}}%
\pgfpathlineto{\pgfqpoint{0.785406in}{1.522064in}}%
\pgfpathlineto{\pgfqpoint{0.785406in}{1.522064in}}%
\pgfusepath{stroke}%
\end{pgfscope}%
\begin{pgfscope}%
\pgfpathrectangle{\pgfqpoint{0.448634in}{0.402556in}}{\pgfqpoint{4.350661in}{2.489204in}} %
\pgfusepath{clip}%
\pgfsetrectcap%
\pgfsetroundjoin%
\pgfsetlinewidth{1.003750pt}%
\definecolor{currentstroke}{rgb}{0.549020,0.337255,0.294118}%
\pgfsetstrokecolor{currentstroke}%
\pgfsetdash{}{0pt}%
\pgfpathmoveto{\pgfqpoint{1.127322in}{2.572078in}}%
\pgfpathlineto{\pgfqpoint{1.159578in}{2.592762in}}%
\pgfpathlineto{\pgfqpoint{1.192766in}{2.611418in}}%
\pgfpathlineto{\pgfqpoint{1.228729in}{2.629130in}}%
\pgfpathlineto{\pgfqpoint{1.267416in}{2.645761in}}%
\pgfpathlineto{\pgfqpoint{1.310849in}{2.661948in}}%
\pgfpathlineto{\pgfqpoint{1.356923in}{2.676743in}}%
\pgfpathlineto{\pgfqpoint{1.407683in}{2.690704in}}%
\pgfpathlineto{\pgfqpoint{1.463097in}{2.703643in}}%
\pgfpathlineto{\pgfqpoint{1.525276in}{2.715814in}}%
\pgfpathlineto{\pgfqpoint{1.594202in}{2.726938in}}%
\pgfpathlineto{\pgfqpoint{1.669846in}{2.736808in}}%
\pgfpathlineto{\pgfqpoint{1.752176in}{2.745272in}}%
\pgfpathlineto{\pgfqpoint{1.843328in}{2.752344in}}%
\pgfpathlineto{\pgfqpoint{1.941107in}{2.757656in}}%
\pgfpathlineto{\pgfqpoint{2.043304in}{2.760987in}}%
\pgfpathlineto{\pgfqpoint{2.147713in}{2.762199in}}%
\pgfpathlineto{\pgfqpoint{2.249949in}{2.761215in}}%
\pgfpathlineto{\pgfqpoint{2.345624in}{2.758145in}}%
\pgfpathlineto{\pgfqpoint{2.432528in}{2.753209in}}%
\pgfpathlineto{\pgfqpoint{2.508454in}{2.746765in}}%
\pgfpathlineto{\pgfqpoint{2.573372in}{2.739155in}}%
\pgfpathlineto{\pgfqpoint{2.629413in}{2.730451in}}%
\pgfpathlineto{\pgfqpoint{2.676547in}{2.720984in}}%
\pgfpathlineto{\pgfqpoint{2.716878in}{2.710665in}}%
\pgfpathlineto{\pgfqpoint{2.750369in}{2.699848in}}%
\pgfpathlineto{\pgfqpoint{2.779063in}{2.688192in}}%
\pgfpathlineto{\pgfqpoint{2.802886in}{2.676004in}}%
\pgfpathlineto{\pgfqpoint{2.821845in}{2.663820in}}%
\pgfpathlineto{\pgfqpoint{2.837819in}{2.650887in}}%
\pgfpathlineto{\pgfqpoint{2.850739in}{2.637565in}}%
\pgfpathlineto{\pgfqpoint{2.860698in}{2.624398in}}%
\pgfpathlineto{\pgfqpoint{2.869088in}{2.609873in}}%
\pgfpathlineto{\pgfqpoint{2.875702in}{2.594192in}}%
\pgfpathlineto{\pgfqpoint{2.881038in}{2.575255in}}%
\pgfpathlineto{\pgfqpoint{2.884203in}{2.555685in}}%
\pgfpathlineto{\pgfqpoint{2.885621in}{2.533351in}}%
\pgfpathlineto{\pgfqpoint{2.885040in}{2.505987in}}%
\pgfpathlineto{\pgfqpoint{2.882113in}{2.473807in}}%
\pgfpathlineto{\pgfqpoint{2.875657in}{2.429620in}}%
\pgfpathlineto{\pgfqpoint{2.863489in}{2.363873in}}%
\pgfpathlineto{\pgfqpoint{2.821103in}{2.142618in}}%
\pgfpathlineto{\pgfqpoint{2.804859in}{2.042271in}}%
\pgfpathlineto{\pgfqpoint{2.790421in}{1.939040in}}%
\pgfpathlineto{\pgfqpoint{2.777206in}{1.828054in}}%
\pgfpathlineto{\pgfqpoint{2.765337in}{1.709349in}}%
\pgfpathlineto{\pgfqpoint{2.754471in}{1.578010in}}%
\pgfpathlineto{\pgfqpoint{2.744640in}{1.431580in}}%
\pgfpathlineto{\pgfqpoint{2.735914in}{1.267598in}}%
\pgfpathlineto{\pgfqpoint{2.728277in}{1.081114in}}%
\pgfpathlineto{\pgfqpoint{2.721437in}{0.857223in}}%
\pgfpathlineto{\pgfqpoint{2.711960in}{0.541290in}}%
\pgfpathlineto{\pgfqpoint{2.708249in}{0.491694in}}%
\pgfpathlineto{\pgfqpoint{2.703950in}{0.462246in}}%
\pgfpathlineto{\pgfqpoint{2.699503in}{0.445599in}}%
\pgfpathlineto{\pgfqpoint{2.694516in}{0.434563in}}%
\pgfpathlineto{\pgfqpoint{2.688940in}{0.426947in}}%
\pgfpathlineto{\pgfqpoint{2.681978in}{0.421009in}}%
\pgfpathlineto{\pgfqpoint{2.672062in}{0.415949in}}%
\pgfpathlineto{\pgfqpoint{2.659427in}{0.412247in}}%
\pgfpathlineto{\pgfqpoint{2.640043in}{0.409164in}}%
\pgfpathlineto{\pgfqpoint{2.607488in}{0.406692in}}%
\pgfpathlineto{\pgfqpoint{2.548777in}{0.404895in}}%
\pgfpathlineto{\pgfqpoint{2.422613in}{0.403701in}}%
\pgfpathlineto{\pgfqpoint{2.026703in}{0.403017in}}%
\pgfpathlineto{\pgfqpoint{0.623616in}{0.403253in}}%
\pgfpathlineto{\pgfqpoint{0.477878in}{0.404742in}}%
\pgfpathlineto{\pgfqpoint{0.458366in}{0.406383in}}%
\pgfpathlineto{\pgfqpoint{0.452302in}{0.408939in}}%
\pgfpathlineto{\pgfqpoint{0.450212in}{0.413217in}}%
\pgfpathlineto{\pgfqpoint{0.449165in}{0.423082in}}%
\pgfpathlineto{\pgfqpoint{0.448735in}{0.465394in}}%
\pgfpathlineto{\pgfqpoint{0.448637in}{0.983148in}}%
\pgfpathlineto{\pgfqpoint{0.448652in}{2.889878in}}%
\pgfpathlineto{\pgfqpoint{0.448652in}{2.889878in}}%
\pgfusepath{stroke}%
\end{pgfscope}%
\begin{pgfscope}%
\pgfpathrectangle{\pgfqpoint{0.448634in}{0.402556in}}{\pgfqpoint{4.350661in}{2.489204in}} %
\pgfusepath{clip}%
\pgfsetrectcap%
\pgfsetroundjoin%
\pgfsetlinewidth{1.003750pt}%
\definecolor{currentstroke}{rgb}{0.549020,0.337255,0.294118}%
\pgfsetstrokecolor{currentstroke}%
\pgfsetdash{}{0pt}%
\pgfpathmoveto{\pgfqpoint{0.448634in}{2.896245in}}%
\pgfpathlineto{\pgfqpoint{0.448593in}{0.407043in}}%
\pgfpathlineto{\pgfqpoint{0.448593in}{0.407043in}}%
\pgfusepath{stroke}%
\end{pgfscope}%
\begin{pgfscope}%
\pgfpathrectangle{\pgfqpoint{0.448634in}{0.402556in}}{\pgfqpoint{4.350661in}{2.489204in}} %
\pgfusepath{clip}%
\pgfsetrectcap%
\pgfsetroundjoin%
\pgfsetlinewidth{1.003750pt}%
\definecolor{currentstroke}{rgb}{0.549020,0.337255,0.294118}%
\pgfsetstrokecolor{currentstroke}%
\pgfsetdash{}{0pt}%
\pgfpathmoveto{\pgfqpoint{0.576817in}{1.760844in}}%
\pgfpathlineto{\pgfqpoint{0.569362in}{1.840038in}}%
\pgfpathlineto{\pgfqpoint{0.563180in}{1.929367in}}%
\pgfpathlineto{\pgfqpoint{0.558567in}{2.028792in}}%
\pgfpathlineto{\pgfqpoint{0.555962in}{2.133294in}}%
\pgfpathlineto{\pgfqpoint{0.555543in}{2.237837in}}%
\pgfpathlineto{\pgfqpoint{0.557350in}{2.337380in}}%
\pgfpathlineto{\pgfqpoint{0.561075in}{2.424395in}}%
\pgfpathlineto{\pgfqpoint{0.566383in}{2.498820in}}%
\pgfpathlineto{\pgfqpoint{0.572888in}{2.560598in}}%
\pgfpathlineto{\pgfqpoint{0.580438in}{2.612147in}}%
\pgfpathlineto{\pgfqpoint{0.589066in}{2.655844in}}%
\pgfpathlineto{\pgfqpoint{0.598386in}{2.691618in}}%
\pgfpathlineto{\pgfqpoint{0.608594in}{2.721785in}}%
\pgfpathlineto{\pgfqpoint{0.619223in}{2.746306in}}%
\pgfpathlineto{\pgfqpoint{0.630800in}{2.767366in}}%
\pgfpathlineto{\pgfqpoint{0.642959in}{2.784909in}}%
\pgfpathlineto{\pgfqpoint{0.656799in}{2.800736in}}%
\pgfpathlineto{\pgfqpoint{0.672184in}{2.814571in}}%
\pgfpathlineto{\pgfqpoint{0.688841in}{2.826321in}}%
\pgfpathlineto{\pgfqpoint{0.706450in}{2.836094in}}%
\pgfpathlineto{\pgfqpoint{0.726794in}{2.844891in}}%
\pgfpathlineto{\pgfqpoint{0.751857in}{2.853216in}}%
\pgfpathlineto{\pgfqpoint{0.781623in}{2.860558in}}%
\pgfpathlineto{\pgfqpoint{0.818160in}{2.867062in}}%
\pgfpathlineto{\pgfqpoint{0.863573in}{2.872691in}}%
\pgfpathlineto{\pgfqpoint{0.922152in}{2.877523in}}%
\pgfpathlineto{\pgfqpoint{1.000382in}{2.881571in}}%
\pgfpathlineto{\pgfqpoint{1.111285in}{2.884884in}}%
\pgfpathlineto{\pgfqpoint{1.274420in}{2.887369in}}%
\pgfpathlineto{\pgfqpoint{1.552857in}{2.889264in}}%
\pgfpathlineto{\pgfqpoint{2.107565in}{2.890457in}}%
\pgfpathlineto{\pgfqpoint{3.343153in}{2.890573in}}%
\pgfpathlineto{\pgfqpoint{4.043607in}{2.888941in}}%
\pgfpathlineto{\pgfqpoint{4.289408in}{2.886405in}}%
\pgfpathlineto{\pgfqpoint{4.413367in}{2.883096in}}%
\pgfpathlineto{\pgfqpoint{4.489416in}{2.879000in}}%
\pgfpathlineto{\pgfqpoint{4.541443in}{2.874086in}}%
\pgfpathlineto{\pgfqpoint{4.578093in}{2.868477in}}%
\pgfpathlineto{\pgfqpoint{4.605811in}{2.862102in}}%
\pgfpathlineto{\pgfqpoint{4.626719in}{2.855257in}}%
\pgfpathlineto{\pgfqpoint{4.644919in}{2.847033in}}%
\pgfpathlineto{\pgfqpoint{4.660236in}{2.837607in}}%
\pgfpathlineto{\pgfqpoint{4.672620in}{2.827488in}}%
\pgfpathlineto{\pgfqpoint{4.683749in}{2.815613in}}%
\pgfpathlineto{\pgfqpoint{4.693404in}{2.802155in}}%
\pgfpathlineto{\pgfqpoint{4.702739in}{2.785365in}}%
\pgfpathlineto{\pgfqpoint{4.711278in}{2.765217in}}%
\pgfpathlineto{\pgfqpoint{4.719484in}{2.739507in}}%
\pgfpathlineto{\pgfqpoint{4.726300in}{2.710682in}}%
\pgfpathlineto{\pgfqpoint{4.733266in}{2.671667in}}%
\pgfpathlineto{\pgfqpoint{4.739611in}{2.622420in}}%
\pgfpathlineto{\pgfqpoint{4.745243in}{2.560528in}}%
\pgfpathlineto{\pgfqpoint{4.750169in}{2.481076in}}%
\pgfpathlineto{\pgfqpoint{4.754372in}{2.376642in}}%
\pgfpathlineto{\pgfqpoint{4.757449in}{2.242273in}}%
\pgfpathlineto{\pgfqpoint{4.758982in}{2.075507in}}%
\pgfpathlineto{\pgfqpoint{4.758453in}{1.888819in}}%
\pgfpathlineto{\pgfqpoint{4.755763in}{1.707135in}}%
\pgfpathlineto{\pgfqpoint{4.750933in}{1.532982in}}%
\pgfpathlineto{\pgfqpoint{4.744794in}{1.398751in}}%
\pgfpathlineto{\pgfqpoint{4.737584in}{1.289540in}}%
\pgfpathlineto{\pgfqpoint{4.728967in}{1.192967in}}%
\pgfpathlineto{\pgfqpoint{4.719982in}{1.119006in}}%
\pgfpathlineto{\pgfqpoint{4.710022in}{1.055303in}}%
\pgfpathlineto{\pgfqpoint{4.699489in}{1.001888in}}%
\pgfpathlineto{\pgfqpoint{4.689024in}{0.958718in}}%
\pgfpathlineto{\pgfqpoint{4.677203in}{0.918628in}}%
\pgfpathlineto{\pgfqpoint{4.664017in}{0.881778in}}%
\pgfpathlineto{\pgfqpoint{4.650566in}{0.850520in}}%
\pgfpathlineto{\pgfqpoint{4.636284in}{0.822599in}}%
\pgfpathlineto{\pgfqpoint{4.620186in}{0.796005in}}%
\pgfpathlineto{\pgfqpoint{4.603619in}{0.772932in}}%
\pgfpathlineto{\pgfqpoint{4.585467in}{0.751477in}}%
\pgfpathlineto{\pgfqpoint{4.565852in}{0.731780in}}%
\pgfpathlineto{\pgfqpoint{4.544942in}{0.713911in}}%
\pgfpathlineto{\pgfqpoint{4.522936in}{0.697855in}}%
\pgfpathlineto{\pgfqpoint{4.496135in}{0.681320in}}%
\pgfpathlineto{\pgfqpoint{4.470373in}{0.667991in}}%
\pgfpathlineto{\pgfqpoint{4.439937in}{0.654548in}}%
\pgfpathlineto{\pgfqpoint{4.406817in}{0.642319in}}%
\pgfpathlineto{\pgfqpoint{4.368985in}{0.630786in}}%
\pgfpathlineto{\pgfqpoint{4.326465in}{0.620263in}}%
\pgfpathlineto{\pgfqpoint{4.279303in}{0.610985in}}%
\pgfpathlineto{\pgfqpoint{4.227552in}{0.603121in}}%
\pgfpathlineto{\pgfqpoint{4.173426in}{0.597109in}}%
\pgfpathlineto{\pgfqpoint{4.110487in}{0.592248in}}%
\pgfpathlineto{\pgfqpoint{4.047446in}{0.589591in}}%
\pgfpathlineto{\pgfqpoint{3.977843in}{0.588679in}}%
\pgfpathlineto{\pgfqpoint{3.906068in}{0.589990in}}%
\pgfpathlineto{\pgfqpoint{3.834352in}{0.593555in}}%
\pgfpathlineto{\pgfqpoint{3.767095in}{0.599129in}}%
\pgfpathlineto{\pgfqpoint{3.704341in}{0.606458in}}%
\pgfpathlineto{\pgfqpoint{3.678490in}{0.610563in}}%
\pgfpathlineto{\pgfqpoint{3.620413in}{0.620558in}}%
\pgfpathlineto{\pgfqpoint{3.586290in}{0.628251in}}%
\pgfpathlineto{\pgfqpoint{3.497302in}{0.651772in}}%
\pgfpathlineto{\pgfqpoint{3.453561in}{0.666822in}}%
\pgfpathlineto{\pgfqpoint{3.410522in}{0.684303in}}%
\pgfpathlineto{\pgfqpoint{3.376538in}{0.700975in}}%
\pgfpathlineto{\pgfqpoint{3.349215in}{0.716358in}}%
\pgfpathlineto{\pgfqpoint{3.318907in}{0.735922in}}%
\pgfpathlineto{\pgfqpoint{3.293419in}{0.754986in}}%
\pgfpathlineto{\pgfqpoint{3.269036in}{0.775856in}}%
\pgfpathlineto{\pgfqpoint{3.245944in}{0.798565in}}%
\pgfpathlineto{\pgfqpoint{3.224302in}{0.823073in}}%
\pgfpathlineto{\pgfqpoint{3.204248in}{0.849290in}}%
\pgfpathlineto{\pgfqpoint{3.185856in}{0.877058in}}%
\pgfpathlineto{\pgfqpoint{3.169236in}{0.906251in}}%
\pgfpathlineto{\pgfqpoint{3.154299in}{0.936614in}}%
\pgfpathlineto{\pgfqpoint{3.140125in}{0.970238in}}%
\pgfpathlineto{\pgfqpoint{3.126273in}{1.009466in}}%
\pgfpathlineto{\pgfqpoint{3.115292in}{1.047253in}}%
\pgfpathlineto{\pgfqpoint{3.105026in}{1.090485in}}%
\pgfpathlineto{\pgfqpoint{3.096478in}{1.136750in}}%
\pgfpathlineto{\pgfqpoint{3.089476in}{1.188400in}}%
\pgfpathlineto{\pgfqpoint{3.084758in}{1.240388in}}%
\pgfpathlineto{\pgfqpoint{3.082200in}{1.292575in}}%
\pgfpathlineto{\pgfqpoint{3.081654in}{1.347329in}}%
\pgfpathlineto{\pgfqpoint{3.083323in}{1.404544in}}%
\pgfpathlineto{\pgfqpoint{3.087175in}{1.461621in}}%
\pgfpathlineto{\pgfqpoint{3.093480in}{1.520919in}}%
\pgfpathlineto{\pgfqpoint{3.101819in}{1.577365in}}%
\pgfpathlineto{\pgfqpoint{3.111931in}{1.630886in}}%
\pgfpathlineto{\pgfqpoint{3.124693in}{1.686238in}}%
\pgfpathlineto{\pgfqpoint{3.139183in}{1.738424in}}%
\pgfpathlineto{\pgfqpoint{3.155152in}{1.787393in}}%
\pgfpathlineto{\pgfqpoint{3.172363in}{1.833111in}}%
\pgfpathlineto{\pgfqpoint{3.191630in}{1.877741in}}%
\pgfpathlineto{\pgfqpoint{3.214043in}{1.923283in}}%
\pgfpathlineto{\pgfqpoint{3.236234in}{1.963176in}}%
\pgfpathlineto{\pgfqpoint{3.260200in}{2.001702in}}%
\pgfpathlineto{\pgfqpoint{3.285838in}{2.038793in}}%
\pgfpathlineto{\pgfqpoint{3.314440in}{2.076300in}}%
\pgfpathlineto{\pgfqpoint{3.348970in}{2.117726in}}%
\pgfpathlineto{\pgfqpoint{3.417157in}{2.198039in}}%
\pgfpathlineto{\pgfqpoint{3.426075in}{2.212147in}}%
\pgfpathlineto{\pgfqpoint{3.430817in}{2.223317in}}%
\pgfpathlineto{\pgfqpoint{3.432048in}{2.230624in}}%
\pgfpathlineto{\pgfqpoint{3.430781in}{2.237875in}}%
\pgfpathlineto{\pgfqpoint{3.426624in}{2.243541in}}%
\pgfpathlineto{\pgfqpoint{3.420909in}{2.247095in}}%
\pgfpathlineto{\pgfqpoint{3.412501in}{2.249591in}}%
\pgfpathlineto{\pgfqpoint{3.399499in}{2.250694in}}%
\pgfpathlineto{\pgfqpoint{3.384305in}{2.249675in}}%
\pgfpathlineto{\pgfqpoint{3.364986in}{2.246100in}}%
\pgfpathlineto{\pgfqpoint{3.341804in}{2.239344in}}%
\pgfpathlineto{\pgfqpoint{3.317110in}{2.229682in}}%
\pgfpathlineto{\pgfqpoint{3.291105in}{2.216985in}}%
\pgfpathlineto{\pgfqpoint{3.265930in}{2.202260in}}%
\pgfpathlineto{\pgfqpoint{3.239807in}{2.184359in}}%
\pgfpathlineto{\pgfqpoint{3.214777in}{2.164516in}}%
\pgfpathlineto{\pgfqpoint{3.190903in}{2.142890in}}%
\pgfpathlineto{\pgfqpoint{3.166659in}{2.117909in}}%
\pgfpathlineto{\pgfqpoint{3.143838in}{2.091231in}}%
\pgfpathlineto{\pgfqpoint{3.121081in}{2.061104in}}%
\pgfpathlineto{\pgfqpoint{3.099954in}{2.029461in}}%
\pgfpathlineto{\pgfqpoint{3.079252in}{1.994404in}}%
\pgfpathlineto{\pgfqpoint{3.059218in}{1.955914in}}%
\pgfpathlineto{\pgfqpoint{3.040058in}{1.914014in}}%
\pgfpathlineto{\pgfqpoint{3.022808in}{1.871041in}}%
\pgfpathlineto{\pgfqpoint{3.005788in}{1.822536in}}%
\pgfpathlineto{\pgfqpoint{2.990064in}{1.770819in}}%
\pgfpathlineto{\pgfqpoint{2.975706in}{1.715980in}}%
\pgfpathlineto{\pgfqpoint{2.962282in}{1.655680in}}%
\pgfpathlineto{\pgfqpoint{2.950494in}{1.592386in}}%
\pgfpathlineto{\pgfqpoint{2.940381in}{1.526185in}}%
\pgfpathlineto{\pgfqpoint{2.931745in}{1.454681in}}%
\pgfpathlineto{\pgfqpoint{2.925082in}{1.380399in}}%
\pgfpathlineto{\pgfqpoint{2.920648in}{1.305899in}}%
\pgfpathlineto{\pgfqpoint{2.918445in}{1.231270in}}%
\pgfpathlineto{\pgfqpoint{2.918546in}{1.159087in}}%
\pgfpathlineto{\pgfqpoint{2.920788in}{1.091931in}}%
\pgfpathlineto{\pgfqpoint{2.925179in}{1.027412in}}%
\pgfpathlineto{\pgfqpoint{2.931193in}{0.970580in}}%
\pgfpathlineto{\pgfqpoint{2.938761in}{0.919034in}}%
\pgfpathlineto{\pgfqpoint{2.947652in}{0.872852in}}%
\pgfpathlineto{\pgfqpoint{2.958214in}{0.829714in}}%
\pgfpathlineto{\pgfqpoint{2.969670in}{0.792113in}}%
\pgfpathlineto{\pgfqpoint{2.982463in}{0.757773in}}%
\pgfpathlineto{\pgfqpoint{2.996425in}{0.726812in}}%
\pgfpathlineto{\pgfqpoint{3.011299in}{0.699300in}}%
\pgfpathlineto{\pgfqpoint{3.026739in}{0.675224in}}%
\pgfpathlineto{\pgfqpoint{3.043828in}{0.652655in}}%
\pgfpathlineto{\pgfqpoint{3.062495in}{0.631788in}}%
\pgfpathlineto{\pgfqpoint{3.082602in}{0.612752in}}%
\pgfpathlineto{\pgfqpoint{3.103961in}{0.595592in}}%
\pgfpathlineto{\pgfqpoint{3.128268in}{0.579068in}}%
\pgfpathlineto{\pgfqpoint{3.153537in}{0.564553in}}%
\pgfpathlineto{\pgfqpoint{3.181571in}{0.550952in}}%
\pgfpathlineto{\pgfqpoint{3.214371in}{0.537647in}}%
\pgfpathlineto{\pgfqpoint{3.249847in}{0.525713in}}%
\pgfpathlineto{\pgfqpoint{3.290011in}{0.514571in}}%
\pgfpathlineto{\pgfqpoint{3.334821in}{0.504423in}}%
\pgfpathlineto{\pgfqpoint{3.386372in}{0.494999in}}%
\pgfpathlineto{\pgfqpoint{3.446798in}{0.486257in}}%
\pgfpathlineto{\pgfqpoint{3.518243in}{0.478281in}}%
\pgfpathlineto{\pgfqpoint{3.600685in}{0.471409in}}%
\pgfpathlineto{\pgfqpoint{3.696269in}{0.465712in}}%
\pgfpathlineto{\pgfqpoint{3.807144in}{0.461368in}}%
\pgfpathlineto{\pgfqpoint{3.933291in}{0.458719in}}%
\pgfpathlineto{\pgfqpoint{4.063809in}{0.458211in}}%
\pgfpathlineto{\pgfqpoint{4.187792in}{0.459914in}}%
\pgfpathlineto{\pgfqpoint{4.294335in}{0.463521in}}%
\pgfpathlineto{\pgfqpoint{4.381234in}{0.468574in}}%
\pgfpathlineto{\pgfqpoint{4.450636in}{0.474702in}}%
\pgfpathlineto{\pgfqpoint{4.506850in}{0.481799in}}%
\pgfpathlineto{\pgfqpoint{4.552009in}{0.489659in}}%
\pgfpathlineto{\pgfqpoint{4.588239in}{0.498116in}}%
\pgfpathlineto{\pgfqpoint{4.617656in}{0.507111in}}%
\pgfpathlineto{\pgfqpoint{4.642329in}{0.516843in}}%
\pgfpathlineto{\pgfqpoint{4.664194in}{0.527940in}}%
\pgfpathlineto{\pgfqpoint{4.681238in}{0.538946in}}%
\pgfpathlineto{\pgfqpoint{4.697164in}{0.551954in}}%
\pgfpathlineto{\pgfqpoint{4.710076in}{0.565290in}}%
\pgfpathlineto{\pgfqpoint{4.721579in}{0.580219in}}%
\pgfpathlineto{\pgfqpoint{4.731557in}{0.596522in}}%
\pgfpathlineto{\pgfqpoint{4.741000in}{0.616135in}}%
\pgfpathlineto{\pgfqpoint{4.749522in}{0.639027in}}%
\pgfpathlineto{\pgfqpoint{4.757523in}{0.667450in}}%
\pgfpathlineto{\pgfqpoint{4.764572in}{0.701345in}}%
\pgfpathlineto{\pgfqpoint{4.770840in}{0.743044in}}%
\pgfpathlineto{\pgfqpoint{4.776327in}{0.794935in}}%
\pgfpathlineto{\pgfqpoint{4.781278in}{0.864398in}}%
\pgfpathlineto{\pgfqpoint{4.785468in}{0.956372in}}%
\pgfpathlineto{\pgfqpoint{4.789000in}{1.085746in}}%
\pgfpathlineto{\pgfqpoint{4.791852in}{1.277386in}}%
\pgfpathlineto{\pgfqpoint{4.793959in}{1.581058in}}%
\pgfpathlineto{\pgfqpoint{4.794962in}{2.071430in}}%
\pgfpathlineto{\pgfqpoint{4.793967in}{2.559311in}}%
\pgfpathlineto{\pgfqpoint{4.791733in}{2.745981in}}%
\pgfpathlineto{\pgfqpoint{4.788955in}{2.818092in}}%
\pgfpathlineto{\pgfqpoint{4.785731in}{2.850227in}}%
\pgfpathlineto{\pgfqpoint{4.781879in}{2.867058in}}%
\pgfpathlineto{\pgfqpoint{4.777744in}{2.875781in}}%
\pgfpathlineto{\pgfqpoint{4.773096in}{2.880982in}}%
\pgfpathlineto{\pgfqpoint{4.767362in}{2.884504in}}%
\pgfpathlineto{\pgfqpoint{4.756853in}{2.887622in}}%
\pgfpathlineto{\pgfqpoint{4.739547in}{2.889639in}}%
\pgfpathlineto{\pgfqpoint{4.704762in}{2.890882in}}%
\pgfpathlineto{\pgfqpoint{4.602523in}{2.891538in}}%
\pgfpathlineto{\pgfqpoint{3.952100in}{2.891742in}}%
\pgfpathlineto{\pgfqpoint{0.617320in}{2.890753in}}%
\pgfpathlineto{\pgfqpoint{0.549909in}{2.888859in}}%
\pgfpathlineto{\pgfqpoint{0.521734in}{2.886181in}}%
\pgfpathlineto{\pgfqpoint{0.504665in}{2.882392in}}%
\pgfpathlineto{\pgfqpoint{0.494500in}{2.878015in}}%
\pgfpathlineto{\pgfqpoint{0.487178in}{2.872671in}}%
\pgfpathlineto{\pgfqpoint{0.481150in}{2.865523in}}%
\pgfpathlineto{\pgfqpoint{0.475661in}{2.854809in}}%
\pgfpathlineto{\pgfqpoint{0.471316in}{2.840742in}}%
\pgfpathlineto{\pgfqpoint{0.467299in}{2.818828in}}%
\pgfpathlineto{\pgfqpoint{0.463925in}{2.786705in}}%
\pgfpathlineto{\pgfqpoint{0.460916in}{2.734549in}}%
\pgfpathlineto{\pgfqpoint{0.458361in}{2.647478in}}%
\pgfpathlineto{\pgfqpoint{0.456573in}{2.523036in}}%
\pgfpathlineto{\pgfqpoint{0.456573in}{2.523036in}}%
\pgfusepath{stroke}%
\end{pgfscope}%
\begin{pgfscope}%
\pgfpathrectangle{\pgfqpoint{0.448634in}{0.402556in}}{\pgfqpoint{4.350661in}{2.489204in}} %
\pgfusepath{clip}%
\pgfsetrectcap%
\pgfsetroundjoin%
\pgfsetlinewidth{1.003750pt}%
\definecolor{currentstroke}{rgb}{0.549020,0.337255,0.294118}%
\pgfsetstrokecolor{currentstroke}%
\pgfsetdash{}{0pt}%
\pgfpathmoveto{\pgfqpoint{0.456426in}{1.369766in}}%
\pgfpathlineto{\pgfqpoint{0.459617in}{1.118385in}}%
\pgfpathlineto{\pgfqpoint{0.463619in}{0.964124in}}%
\pgfpathlineto{\pgfqpoint{0.468392in}{0.859724in}}%
\pgfpathlineto{\pgfqpoint{0.473886in}{0.785317in}}%
\pgfpathlineto{\pgfqpoint{0.479940in}{0.731000in}}%
\pgfpathlineto{\pgfqpoint{0.487038in}{0.686944in}}%
\pgfpathlineto{\pgfqpoint{0.494628in}{0.653203in}}%
\pgfpathlineto{\pgfqpoint{0.503226in}{0.625011in}}%
\pgfpathlineto{\pgfqpoint{0.512341in}{0.602421in}}%
\pgfpathlineto{\pgfqpoint{0.522378in}{0.583200in}}%
\pgfpathlineto{\pgfqpoint{0.534317in}{0.565463in}}%
\pgfpathlineto{\pgfqpoint{0.546499in}{0.551257in}}%
\pgfpathlineto{\pgfqpoint{0.559986in}{0.538687in}}%
\pgfpathlineto{\pgfqpoint{0.576406in}{0.526507in}}%
\pgfpathlineto{\pgfqpoint{0.595774in}{0.515196in}}%
\pgfpathlineto{\pgfqpoint{0.617982in}{0.505021in}}%
\pgfpathlineto{\pgfqpoint{0.642876in}{0.496052in}}%
\pgfpathlineto{\pgfqpoint{0.674564in}{0.487171in}}%
\pgfpathlineto{\pgfqpoint{0.710905in}{0.479353in}}%
\pgfpathlineto{\pgfqpoint{0.756126in}{0.471970in}}%
\pgfpathlineto{\pgfqpoint{0.812369in}{0.465173in}}%
\pgfpathlineto{\pgfqpoint{0.881777in}{0.459129in}}%
\pgfpathlineto{\pgfqpoint{0.968668in}{0.453886in}}%
\pgfpathlineto{\pgfqpoint{1.077370in}{0.449636in}}%
\pgfpathlineto{\pgfqpoint{1.212214in}{0.446669in}}%
\pgfpathlineto{\pgfqpoint{1.371009in}{0.445459in}}%
\pgfpathlineto{\pgfqpoint{1.538506in}{0.446387in}}%
\pgfpathlineto{\pgfqpoint{1.697282in}{0.449434in}}%
\pgfpathlineto{\pgfqpoint{1.834262in}{0.454231in}}%
\pgfpathlineto{\pgfqpoint{1.947252in}{0.460344in}}%
\pgfpathlineto{\pgfqpoint{2.040576in}{0.467567in}}%
\pgfpathlineto{\pgfqpoint{2.116389in}{0.475565in}}%
\pgfpathlineto{\pgfqpoint{2.181166in}{0.484602in}}%
\pgfpathlineto{\pgfqpoint{2.234882in}{0.494294in}}%
\pgfpathlineto{\pgfqpoint{2.279680in}{0.504507in}}%
\pgfpathlineto{\pgfqpoint{2.319787in}{0.515914in}}%
\pgfpathlineto{\pgfqpoint{2.353063in}{0.527570in}}%
\pgfpathlineto{\pgfqpoint{2.383630in}{0.540618in}}%
\pgfpathlineto{\pgfqpoint{2.409405in}{0.553917in}}%
\pgfpathlineto{\pgfqpoint{2.432381in}{0.568080in}}%
\pgfpathlineto{\pgfqpoint{2.454327in}{0.584240in}}%
\pgfpathlineto{\pgfqpoint{2.473313in}{0.600892in}}%
\pgfpathlineto{\pgfqpoint{2.491021in}{0.619296in}}%
\pgfpathlineto{\pgfqpoint{2.507305in}{0.639348in}}%
\pgfpathlineto{\pgfqpoint{2.522090in}{0.660868in}}%
\pgfpathlineto{\pgfqpoint{2.536516in}{0.685753in}}%
\pgfpathlineto{\pgfqpoint{2.550251in}{0.714030in}}%
\pgfpathlineto{\pgfqpoint{2.563069in}{0.745635in}}%
\pgfpathlineto{\pgfqpoint{2.575579in}{0.782793in}}%
\pgfpathlineto{\pgfqpoint{2.587423in}{0.825493in}}%
\pgfpathlineto{\pgfqpoint{2.598406in}{0.873660in}}%
\pgfpathlineto{\pgfqpoint{2.608880in}{0.929639in}}%
\pgfpathlineto{\pgfqpoint{2.619343in}{0.998297in}}%
\pgfpathlineto{\pgfqpoint{2.629396in}{1.079628in}}%
\pgfpathlineto{\pgfqpoint{2.639572in}{1.181017in}}%
\pgfpathlineto{\pgfqpoint{2.650144in}{1.309887in}}%
\pgfpathlineto{\pgfqpoint{2.662036in}{1.483598in}}%
\pgfpathlineto{\pgfqpoint{2.677296in}{1.741885in}}%
\pgfpathlineto{\pgfqpoint{2.688718in}{1.968023in}}%
\pgfpathlineto{\pgfqpoint{2.692977in}{2.092385in}}%
\pgfpathlineto{\pgfqpoint{2.693785in}{2.177009in}}%
\pgfpathlineto{\pgfqpoint{2.692235in}{2.241699in}}%
\pgfpathlineto{\pgfqpoint{2.688970in}{2.291337in}}%
\pgfpathlineto{\pgfqpoint{2.684111in}{2.333280in}}%
\pgfpathlineto{\pgfqpoint{2.677647in}{2.369870in}}%
\pgfpathlineto{\pgfqpoint{2.669920in}{2.400989in}}%
\pgfpathlineto{\pgfqpoint{2.660663in}{2.428907in}}%
\pgfpathlineto{\pgfqpoint{2.650141in}{2.453488in}}%
\pgfpathlineto{\pgfqpoint{2.638771in}{2.474699in}}%
\pgfpathlineto{\pgfqpoint{2.625686in}{2.494571in}}%
\pgfpathlineto{\pgfqpoint{2.610989in}{2.512910in}}%
\pgfpathlineto{\pgfqpoint{2.594865in}{2.529607in}}%
\pgfpathlineto{\pgfqpoint{2.575753in}{2.546067in}}%
\pgfpathlineto{\pgfqpoint{2.555494in}{2.560625in}}%
\pgfpathlineto{\pgfqpoint{2.532399in}{2.574533in}}%
\pgfpathlineto{\pgfqpoint{2.506517in}{2.587558in}}%
\pgfpathlineto{\pgfqpoint{2.475865in}{2.600342in}}%
\pgfpathlineto{\pgfqpoint{2.442529in}{2.611773in}}%
\pgfpathlineto{\pgfqpoint{2.404495in}{2.622406in}}%
\pgfpathlineto{\pgfqpoint{2.361804in}{2.631975in}}%
\pgfpathlineto{\pgfqpoint{2.314504in}{2.640285in}}%
\pgfpathlineto{\pgfqpoint{2.260483in}{2.647435in}}%
\pgfpathlineto{\pgfqpoint{2.201946in}{2.652894in}}%
\pgfpathlineto{\pgfqpoint{2.138945in}{2.656547in}}%
\pgfpathlineto{\pgfqpoint{2.071528in}{2.658236in}}%
\pgfpathlineto{\pgfqpoint{2.001920in}{2.657792in}}%
\pgfpathlineto{\pgfqpoint{1.930175in}{2.655112in}}%
\pgfpathlineto{\pgfqpoint{1.860691in}{2.650347in}}%
\pgfpathlineto{\pgfqpoint{1.793516in}{2.643601in}}%
\pgfpathlineto{\pgfqpoint{1.728703in}{2.634907in}}%
\pgfpathlineto{\pgfqpoint{1.668458in}{2.624656in}}%
\pgfpathlineto{\pgfqpoint{1.610686in}{2.612562in}}%
\pgfpathlineto{\pgfqpoint{1.557582in}{2.599160in}}%
\pgfpathlineto{\pgfqpoint{1.509177in}{2.584695in}}%
\pgfpathlineto{\pgfqpoint{1.463416in}{2.568679in}}%
\pgfpathlineto{\pgfqpoint{1.422417in}{2.552032in}}%
\pgfpathlineto{\pgfqpoint{1.384154in}{2.534162in}}%
\pgfpathlineto{\pgfqpoint{1.348685in}{2.515192in}}%
\pgfpathlineto{\pgfqpoint{1.316052in}{2.495296in}}%
\pgfpathlineto{\pgfqpoint{1.284443in}{2.473343in}}%
\pgfpathlineto{\pgfqpoint{1.255813in}{2.450706in}}%
\pgfpathlineto{\pgfqpoint{1.230122in}{2.427695in}}%
\pgfpathlineto{\pgfqpoint{1.205702in}{2.402940in}}%
\pgfpathlineto{\pgfqpoint{1.182697in}{2.376469in}}%
\pgfpathlineto{\pgfqpoint{1.161239in}{2.348350in}}%
\pgfpathlineto{\pgfqpoint{1.141431in}{2.318687in}}%
\pgfpathlineto{\pgfqpoint{1.123350in}{2.287613in}}%
\pgfpathlineto{\pgfqpoint{1.107036in}{2.255284in}}%
\pgfpathlineto{\pgfqpoint{1.091595in}{2.219598in}}%
\pgfpathlineto{\pgfqpoint{1.078158in}{2.182865in}}%
\pgfpathlineto{\pgfqpoint{1.066681in}{2.145272in}}%
\pgfpathlineto{\pgfqpoint{1.056568in}{2.104575in}}%
\pgfpathlineto{\pgfqpoint{1.048522in}{2.063278in}}%
\pgfpathlineto{\pgfqpoint{1.042171in}{2.019072in}}%
\pgfpathlineto{\pgfqpoint{1.037970in}{1.974530in}}%
\pgfpathlineto{\pgfqpoint{1.035797in}{1.927307in}}%
\pgfpathlineto{\pgfqpoint{1.035895in}{1.880018in}}%
\pgfpathlineto{\pgfqpoint{1.038235in}{1.832805in}}%
\pgfpathlineto{\pgfqpoint{1.042814in}{1.785807in}}%
\pgfpathlineto{\pgfqpoint{1.049653in}{1.739170in}}%
\pgfpathlineto{\pgfqpoint{1.058254in}{1.695465in}}%
\pgfpathlineto{\pgfqpoint{1.068968in}{1.652375in}}%
\pgfpathlineto{\pgfqpoint{1.081849in}{1.610070in}}%
\pgfpathlineto{\pgfqpoint{1.096053in}{1.571006in}}%
\pgfpathlineto{\pgfqpoint{1.112275in}{1.532986in}}%
\pgfpathlineto{\pgfqpoint{1.129408in}{1.498325in}}%
\pgfpathlineto{\pgfqpoint{1.148349in}{1.464919in}}%
\pgfpathlineto{\pgfqpoint{1.169078in}{1.432934in}}%
\pgfpathlineto{\pgfqpoint{1.191543in}{1.402523in}}%
\pgfpathlineto{\pgfqpoint{1.215669in}{1.373826in}}%
\pgfpathlineto{\pgfqpoint{1.239704in}{1.348581in}}%
\pgfpathlineto{\pgfqpoint{1.266734in}{1.323500in}}%
\pgfpathlineto{\pgfqpoint{1.295065in}{1.300376in}}%
\pgfpathlineto{\pgfqpoint{1.324550in}{1.279223in}}%
\pgfpathlineto{\pgfqpoint{1.355039in}{1.260024in}}%
\pgfpathlineto{\pgfqpoint{1.388374in}{1.241716in}}%
\pgfpathlineto{\pgfqpoint{1.422520in}{1.225480in}}%
\pgfpathlineto{\pgfqpoint{1.459396in}{1.210429in}}%
\pgfpathlineto{\pgfqpoint{1.498961in}{1.196768in}}%
\pgfpathlineto{\pgfqpoint{1.541154in}{1.184644in}}%
\pgfpathlineto{\pgfqpoint{1.585896in}{1.174114in}}%
\pgfpathlineto{\pgfqpoint{1.637400in}{1.164353in}}%
\pgfpathlineto{\pgfqpoint{1.710700in}{1.153069in}}%
\pgfpathlineto{\pgfqpoint{1.768814in}{1.143381in}}%
\pgfpathlineto{\pgfqpoint{1.796442in}{1.136520in}}%
\pgfpathlineto{\pgfqpoint{1.812998in}{1.130417in}}%
\pgfpathlineto{\pgfqpoint{1.824775in}{1.124014in}}%
\pgfpathlineto{\pgfqpoint{1.833491in}{1.116619in}}%
\pgfpathlineto{\pgfqpoint{1.838739in}{1.108732in}}%
\pgfpathlineto{\pgfqpoint{1.840780in}{1.101672in}}%
\pgfpathlineto{\pgfqpoint{1.840761in}{1.094235in}}%
\pgfpathlineto{\pgfqpoint{1.838023in}{1.084827in}}%
\pgfpathlineto{\pgfqpoint{1.833310in}{1.076477in}}%
\pgfpathlineto{\pgfqpoint{1.825861in}{1.067426in}}%
\pgfpathlineto{\pgfqpoint{1.813840in}{1.056757in}}%
\pgfpathlineto{\pgfqpoint{1.798835in}{1.046691in}}%
\pgfpathlineto{\pgfqpoint{1.781025in}{1.037407in}}%
\pgfpathlineto{\pgfqpoint{1.758451in}{1.028352in}}%
\pgfpathlineto{\pgfqpoint{1.733204in}{1.020789in}}%
\pgfpathlineto{\pgfqpoint{1.705410in}{1.014856in}}%
\pgfpathlineto{\pgfqpoint{1.675177in}{1.010705in}}%
\pgfpathlineto{\pgfqpoint{1.642608in}{1.008502in}}%
\pgfpathlineto{\pgfqpoint{1.607808in}{1.008429in}}%
\pgfpathlineto{\pgfqpoint{1.570884in}{1.010687in}}%
\pgfpathlineto{\pgfqpoint{1.534116in}{1.015174in}}%
\pgfpathlineto{\pgfqpoint{1.497592in}{1.021775in}}%
\pgfpathlineto{\pgfqpoint{1.459276in}{1.030982in}}%
\pgfpathlineto{\pgfqpoint{1.421424in}{1.042428in}}%
\pgfpathlineto{\pgfqpoint{1.384141in}{1.056102in}}%
\pgfpathlineto{\pgfqpoint{1.349557in}{1.071075in}}%
\pgfpathlineto{\pgfqpoint{1.315694in}{1.088070in}}%
\pgfpathlineto{\pgfqpoint{1.282676in}{1.107115in}}%
\pgfpathlineto{\pgfqpoint{1.250634in}{1.128230in}}%
\pgfpathlineto{\pgfqpoint{1.221494in}{1.150002in}}%
\pgfpathlineto{\pgfqpoint{1.193465in}{1.173604in}}%
\pgfpathlineto{\pgfqpoint{1.166663in}{1.199004in}}%
\pgfpathlineto{\pgfqpoint{1.141198in}{1.226146in}}%
\pgfpathlineto{\pgfqpoint{1.117167in}{1.254948in}}%
\pgfpathlineto{\pgfqpoint{1.094645in}{1.285306in}}%
\pgfpathlineto{\pgfqpoint{1.072429in}{1.319128in}}%
\pgfpathlineto{\pgfqpoint{1.052013in}{1.354404in}}%
\pgfpathlineto{\pgfqpoint{1.033404in}{1.390965in}}%
\pgfpathlineto{\pgfqpoint{1.015647in}{1.430892in}}%
\pgfpathlineto{\pgfqpoint{0.999846in}{1.471882in}}%
\pgfpathlineto{\pgfqpoint{0.985220in}{1.516110in}}%
\pgfpathlineto{\pgfqpoint{0.972619in}{1.561148in}}%
\pgfpathlineto{\pgfqpoint{0.961448in}{1.609258in}}%
\pgfpathlineto{\pgfqpoint{0.951929in}{1.660378in}}%
\pgfpathlineto{\pgfqpoint{0.944584in}{1.711967in}}%
\pgfpathlineto{\pgfqpoint{0.939144in}{1.766369in}}%
\pgfpathlineto{\pgfqpoint{0.935963in}{1.821005in}}%
\pgfpathlineto{\pgfqpoint{0.935025in}{1.875752in}}%
\pgfpathlineto{\pgfqpoint{0.936352in}{1.930488in}}%
\pgfpathlineto{\pgfqpoint{0.939789in}{1.982608in}}%
\pgfpathlineto{\pgfqpoint{0.945434in}{2.034476in}}%
\pgfpathlineto{\pgfqpoint{0.952976in}{2.083501in}}%
\pgfpathlineto{\pgfqpoint{0.962220in}{2.129592in}}%
\pgfpathlineto{\pgfqpoint{0.973625in}{2.175044in}}%
\pgfpathlineto{\pgfqpoint{0.986552in}{2.217331in}}%
\pgfpathlineto{\pgfqpoint{1.000769in}{2.256389in}}%
\pgfpathlineto{\pgfqpoint{1.017030in}{2.294387in}}%
\pgfpathlineto{\pgfqpoint{1.034262in}{2.328983in}}%
\pgfpathlineto{\pgfqpoint{1.053392in}{2.362246in}}%
\pgfpathlineto{\pgfqpoint{1.073040in}{2.392049in}}%
\pgfpathlineto{\pgfqpoint{1.094299in}{2.420366in}}%
\pgfpathlineto{\pgfqpoint{1.117092in}{2.447077in}}%
\pgfpathlineto{\pgfqpoint{1.141306in}{2.472095in}}%
\pgfpathlineto{\pgfqpoint{1.166809in}{2.495378in}}%
\pgfpathlineto{\pgfqpoint{1.195268in}{2.518295in}}%
\pgfpathlineto{\pgfqpoint{1.224846in}{2.539278in}}%
\pgfpathlineto{\pgfqpoint{1.257305in}{2.559544in}}%
\pgfpathlineto{\pgfqpoint{1.292627in}{2.578869in}}%
\pgfpathlineto{\pgfqpoint{1.330766in}{2.597082in}}%
\pgfpathlineto{\pgfqpoint{1.371659in}{2.614065in}}%
\pgfpathlineto{\pgfqpoint{1.415232in}{2.629749in}}%
\pgfpathlineto{\pgfqpoint{1.463523in}{2.644709in}}%
\pgfpathlineto{\pgfqpoint{1.516513in}{2.658690in}}%
\pgfpathlineto{\pgfqpoint{1.574169in}{2.671494in}}%
\pgfpathlineto{\pgfqpoint{1.636447in}{2.682980in}}%
\pgfpathlineto{\pgfqpoint{1.703304in}{2.693045in}}%
\pgfpathlineto{\pgfqpoint{1.776864in}{2.701828in}}%
\pgfpathlineto{\pgfqpoint{1.854930in}{2.708897in}}%
\pgfpathlineto{\pgfqpoint{1.937463in}{2.714154in}}%
\pgfpathlineto{\pgfqpoint{2.024427in}{2.717441in}}%
\pgfpathlineto{\pgfqpoint{2.111433in}{2.718503in}}%
\pgfpathlineto{\pgfqpoint{2.196263in}{2.717337in}}%
\pgfpathlineto{\pgfqpoint{2.276696in}{2.714020in}}%
\pgfpathlineto{\pgfqpoint{2.350512in}{2.708754in}}%
\pgfpathlineto{\pgfqpoint{2.415496in}{2.701936in}}%
\pgfpathlineto{\pgfqpoint{2.471618in}{2.693936in}}%
\pgfpathlineto{\pgfqpoint{2.521002in}{2.684765in}}%
\pgfpathlineto{\pgfqpoint{2.563607in}{2.674704in}}%
\pgfpathlineto{\pgfqpoint{2.601507in}{2.663473in}}%
\pgfpathlineto{\pgfqpoint{2.632571in}{2.652063in}}%
\pgfpathlineto{\pgfqpoint{2.658891in}{2.640244in}}%
\pgfpathlineto{\pgfqpoint{2.682427in}{2.627341in}}%
\pgfpathlineto{\pgfqpoint{2.703047in}{2.613468in}}%
\pgfpathlineto{\pgfqpoint{2.720653in}{2.598866in}}%
\pgfpathlineto{\pgfqpoint{2.735235in}{2.583933in}}%
\pgfpathlineto{\pgfqpoint{2.748285in}{2.567249in}}%
\pgfpathlineto{\pgfqpoint{2.759510in}{2.548911in}}%
\pgfpathlineto{\pgfqpoint{2.768737in}{2.529167in}}%
\pgfpathlineto{\pgfqpoint{2.775958in}{2.508355in}}%
\pgfpathlineto{\pgfqpoint{2.781819in}{2.484395in}}%
\pgfpathlineto{\pgfqpoint{2.786032in}{2.457451in}}%
\pgfpathlineto{\pgfqpoint{2.788647in}{2.425238in}}%
\pgfpathlineto{\pgfqpoint{2.789353in}{2.387915in}}%
\pgfpathlineto{\pgfqpoint{2.787888in}{2.340655in}}%
\pgfpathlineto{\pgfqpoint{2.783601in}{2.278622in}}%
\pgfpathlineto{\pgfqpoint{2.774224in}{2.179635in}}%
\pgfpathlineto{\pgfqpoint{2.743561in}{1.867969in}}%
\pgfpathlineto{\pgfqpoint{2.730067in}{1.701911in}}%
\pgfpathlineto{\pgfqpoint{2.717245in}{1.515799in}}%
\pgfpathlineto{\pgfqpoint{2.702562in}{1.267447in}}%
\pgfpathlineto{\pgfqpoint{2.684385in}{0.964480in}}%
\pgfpathlineto{\pgfqpoint{2.675316in}{0.850451in}}%
\pgfpathlineto{\pgfqpoint{2.666960in}{0.771376in}}%
\pgfpathlineto{\pgfqpoint{2.658670in}{0.712398in}}%
\pgfpathlineto{\pgfqpoint{2.650079in}{0.666143in}}%
\pgfpathlineto{\pgfqpoint{2.640707in}{0.627795in}}%
\pgfpathlineto{\pgfqpoint{2.631013in}{0.597406in}}%
\pgfpathlineto{\pgfqpoint{2.620855in}{0.572626in}}%
\pgfpathlineto{\pgfqpoint{2.609689in}{0.551276in}}%
\pgfpathlineto{\pgfqpoint{2.597859in}{0.533441in}}%
\pgfpathlineto{\pgfqpoint{2.584298in}{0.517301in}}%
\pgfpathlineto{\pgfqpoint{2.569147in}{0.503135in}}%
\pgfpathlineto{\pgfqpoint{2.552689in}{0.491020in}}%
\pgfpathlineto{\pgfqpoint{2.535258in}{0.480838in}}%
\pgfpathlineto{\pgfqpoint{2.515102in}{0.471494in}}%
\pgfpathlineto{\pgfqpoint{2.490217in}{0.462495in}}%
\pgfpathlineto{\pgfqpoint{2.460619in}{0.454312in}}%
\pgfpathlineto{\pgfqpoint{2.424236in}{0.446763in}}%
\pgfpathlineto{\pgfqpoint{2.381116in}{0.440165in}}%
\pgfpathlineto{\pgfqpoint{2.326988in}{0.434178in}}%
\pgfpathlineto{\pgfqpoint{2.257538in}{0.428803in}}%
\pgfpathlineto{\pgfqpoint{2.166269in}{0.424059in}}%
\pgfpathlineto{\pgfqpoint{2.042328in}{0.419971in}}%
\pgfpathlineto{\pgfqpoint{1.870502in}{0.416664in}}%
\pgfpathlineto{\pgfqpoint{1.626874in}{0.414344in}}%
\pgfpathlineto{\pgfqpoint{1.307101in}{0.413587in}}%
\pgfpathlineto{\pgfqpoint{1.009084in}{0.414958in}}%
\pgfpathlineto{\pgfqpoint{0.815498in}{0.417877in}}%
\pgfpathlineto{\pgfqpoint{0.700254in}{0.421648in}}%
\pgfpathlineto{\pgfqpoint{0.628574in}{0.426036in}}%
\pgfpathlineto{\pgfqpoint{0.580924in}{0.431074in}}%
\pgfpathlineto{\pgfqpoint{0.548666in}{0.436650in}}%
\pgfpathlineto{\pgfqpoint{0.527485in}{0.442284in}}%
\pgfpathlineto{\pgfqpoint{0.511037in}{0.448754in}}%
\pgfpathlineto{\pgfqpoint{0.499351in}{0.455380in}}%
\pgfpathlineto{\pgfqpoint{0.488744in}{0.464047in}}%
\pgfpathlineto{\pgfqpoint{0.481181in}{0.472970in}}%
\pgfpathlineto{\pgfqpoint{0.473974in}{0.485395in}}%
\pgfpathlineto{\pgfqpoint{0.468680in}{0.499033in}}%
\pgfpathlineto{\pgfqpoint{0.463824in}{0.518142in}}%
\pgfpathlineto{\pgfqpoint{0.459653in}{0.545094in}}%
\pgfpathlineto{\pgfqpoint{0.456373in}{0.582238in}}%
\pgfpathlineto{\pgfqpoint{0.453726in}{0.639406in}}%
\pgfpathlineto{\pgfqpoint{0.451647in}{0.738944in}}%
\pgfpathlineto{\pgfqpoint{0.450199in}{0.933094in}}%
\pgfpathlineto{\pgfqpoint{0.449337in}{1.415998in}}%
\pgfpathlineto{\pgfqpoint{0.449567in}{2.692959in}}%
\pgfpathlineto{\pgfqpoint{0.451024in}{2.857232in}}%
\pgfpathlineto{\pgfqpoint{0.452863in}{2.879512in}}%
\pgfpathlineto{\pgfqpoint{0.455351in}{2.886346in}}%
\pgfpathlineto{\pgfqpoint{0.458881in}{2.889131in}}%
\pgfpathlineto{\pgfqpoint{0.467432in}{2.890805in}}%
\pgfpathlineto{\pgfqpoint{0.491349in}{2.891541in}}%
\pgfpathlineto{\pgfqpoint{0.637095in}{2.891749in}}%
\pgfpathlineto{\pgfqpoint{4.785446in}{2.891276in}}%
\pgfpathlineto{\pgfqpoint{4.793971in}{2.889592in}}%
\pgfpathlineto{\pgfqpoint{4.795646in}{2.888055in}}%
\pgfpathlineto{\pgfqpoint{4.797130in}{2.880834in}}%
\pgfpathlineto{\pgfqpoint{4.798041in}{2.855970in}}%
\pgfpathlineto{\pgfqpoint{4.798041in}{2.855970in}}%
\pgfusepath{stroke}%
\end{pgfscope}%
\begin{pgfscope}%
\pgfpathrectangle{\pgfqpoint{0.448634in}{0.402556in}}{\pgfqpoint{4.350661in}{2.489204in}} %
\pgfusepath{clip}%
\pgfsetrectcap%
\pgfsetroundjoin%
\pgfsetlinewidth{1.003750pt}%
\definecolor{currentstroke}{rgb}{0.549020,0.337255,0.294118}%
\pgfsetstrokecolor{currentstroke}%
\pgfsetdash{}{0pt}%
\pgfpathmoveto{\pgfqpoint{3.428489in}{0.402610in}}%
\pgfpathlineto{\pgfqpoint{2.808524in}{0.403725in}}%
\pgfpathlineto{\pgfqpoint{2.769410in}{0.405625in}}%
\pgfpathlineto{\pgfqpoint{2.754359in}{0.408171in}}%
\pgfpathlineto{\pgfqpoint{2.746144in}{0.411387in}}%
\pgfpathlineto{\pgfqpoint{2.740747in}{0.415542in}}%
\pgfpathlineto{\pgfqpoint{2.736659in}{0.421329in}}%
\pgfpathlineto{\pgfqpoint{2.733223in}{0.430451in}}%
\pgfpathlineto{\pgfqpoint{2.730430in}{0.445027in}}%
\pgfpathlineto{\pgfqpoint{2.728238in}{0.469785in}}%
\pgfpathlineto{\pgfqpoint{2.726478in}{0.519524in}}%
\pgfpathlineto{\pgfqpoint{2.725722in}{0.614108in}}%
\pgfpathlineto{\pgfqpoint{2.726855in}{0.768432in}}%
\pgfpathlineto{\pgfqpoint{2.730573in}{0.962542in}}%
\pgfpathlineto{\pgfqpoint{2.736631in}{1.159063in}}%
\pgfpathlineto{\pgfqpoint{2.744117in}{1.328111in}}%
\pgfpathlineto{\pgfqpoint{2.753231in}{1.484581in}}%
\pgfpathlineto{\pgfqpoint{2.763293in}{1.621001in}}%
\pgfpathlineto{\pgfqpoint{2.776164in}{1.764606in}}%
\pgfpathlineto{\pgfqpoint{2.788969in}{1.878165in}}%
\pgfpathlineto{\pgfqpoint{2.805816in}{2.006126in}}%
\pgfpathlineto{\pgfqpoint{2.821254in}{2.101582in}}%
\pgfpathlineto{\pgfqpoint{2.838448in}{2.194100in}}%
\pgfpathlineto{\pgfqpoint{2.859233in}{2.293344in}}%
\pgfpathlineto{\pgfqpoint{2.887307in}{2.426340in}}%
\pgfpathlineto{\pgfqpoint{2.897075in}{2.479943in}}%
\pgfpathlineto{\pgfqpoint{2.901604in}{2.516910in}}%
\pgfpathlineto{\pgfqpoint{2.902882in}{2.544243in}}%
\pgfpathlineto{\pgfqpoint{2.901957in}{2.566610in}}%
\pgfpathlineto{\pgfqpoint{2.899113in}{2.586242in}}%
\pgfpathlineto{\pgfqpoint{2.894715in}{2.602911in}}%
\pgfpathlineto{\pgfqpoint{2.888364in}{2.618731in}}%
\pgfpathlineto{\pgfqpoint{2.880099in}{2.633348in}}%
\pgfpathlineto{\pgfqpoint{2.870157in}{2.646530in}}%
\pgfpathlineto{\pgfqpoint{2.857180in}{2.659778in}}%
\pgfpathlineto{\pgfqpoint{2.842949in}{2.671222in}}%
\pgfpathlineto{\pgfqpoint{2.823980in}{2.683387in}}%
\pgfpathlineto{\pgfqpoint{2.802145in}{2.694569in}}%
\pgfpathlineto{\pgfqpoint{2.775532in}{2.705493in}}%
\pgfpathlineto{\pgfqpoint{2.744179in}{2.715817in}}%
\pgfpathlineto{\pgfqpoint{2.708150in}{2.725337in}}%
\pgfpathlineto{\pgfqpoint{2.665367in}{2.734358in}}%
\pgfpathlineto{\pgfqpoint{2.613701in}{2.742924in}}%
\pgfpathlineto{\pgfqpoint{2.553168in}{2.750632in}}%
\pgfpathlineto{\pgfqpoint{2.481628in}{2.757399in}}%
\pgfpathlineto{\pgfqpoint{2.399105in}{2.762864in}}%
\pgfpathlineto{\pgfqpoint{2.309976in}{2.766496in}}%
\pgfpathlineto{\pgfqpoint{2.186000in}{2.768689in}}%
\pgfpathlineto{\pgfqpoint{2.077236in}{2.768124in}}%
\pgfpathlineto{\pgfqpoint{1.964148in}{2.765271in}}%
\pgfpathlineto{\pgfqpoint{1.859825in}{2.760271in}}%
\pgfpathlineto{\pgfqpoint{1.753436in}{2.752886in}}%
\pgfpathlineto{\pgfqpoint{1.666742in}{2.744390in}}%
\pgfpathlineto{\pgfqpoint{1.586738in}{2.734333in}}%
\pgfpathlineto{\pgfqpoint{1.496211in}{2.720464in}}%
\pgfpathlineto{\pgfqpoint{1.421206in}{2.705640in}}%
\pgfpathlineto{\pgfqpoint{1.365936in}{2.691922in}}%
\pgfpathlineto{\pgfqpoint{1.315363in}{2.677101in}}%
\pgfpathlineto{\pgfqpoint{1.269520in}{2.661397in}}%
\pgfpathlineto{\pgfqpoint{1.226365in}{2.644265in}}%
\pgfpathlineto{\pgfqpoint{1.188038in}{2.626574in}}%
\pgfpathlineto{\pgfqpoint{1.152515in}{2.607738in}}%
\pgfpathlineto{\pgfqpoint{1.119848in}{2.587917in}}%
\pgfpathlineto{\pgfqpoint{1.093793in}{2.569893in}}%
\pgfpathlineto{\pgfqpoint{1.083160in}{2.561259in}}%
\pgfpathlineto{\pgfqpoint{1.054798in}{2.538184in}}%
\pgfpathlineto{\pgfqpoint{1.029412in}{2.514735in}}%
\pgfpathlineto{\pgfqpoint{1.005328in}{2.489552in}}%
\pgfpathlineto{\pgfqpoint{0.982660in}{2.462703in}}%
\pgfpathlineto{\pgfqpoint{0.961491in}{2.434297in}}%
\pgfpathlineto{\pgfqpoint{0.940615in}{2.402437in}}%
\pgfpathlineto{\pgfqpoint{0.924064in}{2.373204in}}%
\pgfpathlineto{\pgfqpoint{0.905432in}{2.336658in}}%
\pgfpathlineto{\pgfqpoint{0.888680in}{2.298938in}}%
\pgfpathlineto{\pgfqpoint{0.872867in}{2.257955in}}%
\pgfpathlineto{\pgfqpoint{0.858156in}{2.213763in}}%
\pgfpathlineto{\pgfqpoint{0.844714in}{2.166421in}}%
\pgfpathlineto{\pgfqpoint{0.839106in}{2.142382in}}%
\pgfpathlineto{\pgfqpoint{0.827383in}{2.089293in}}%
\pgfpathlineto{\pgfqpoint{0.816648in}{2.030832in}}%
\pgfpathlineto{\pgfqpoint{0.810038in}{1.984164in}}%
\pgfpathlineto{\pgfqpoint{0.807996in}{1.966906in}}%
\pgfpathlineto{\pgfqpoint{0.800056in}{1.897807in}}%
\pgfpathlineto{\pgfqpoint{0.793699in}{1.823489in}}%
\pgfpathlineto{\pgfqpoint{0.788800in}{1.741539in}}%
\pgfpathlineto{\pgfqpoint{0.786781in}{1.691811in}}%
\pgfpathlineto{\pgfqpoint{0.776150in}{1.443234in}}%
\pgfpathlineto{\pgfqpoint{0.771471in}{1.406289in}}%
\pgfpathlineto{\pgfqpoint{0.766064in}{1.379627in}}%
\pgfpathlineto{\pgfqpoint{0.759502in}{1.358538in}}%
\pgfpathlineto{\pgfqpoint{0.752242in}{1.343248in}}%
\pgfpathlineto{\pgfqpoint{0.745351in}{1.333644in}}%
\pgfpathlineto{\pgfqpoint{0.738562in}{1.327448in}}%
\pgfpathlineto{\pgfqpoint{0.730670in}{1.323328in}}%
\pgfpathlineto{\pgfqpoint{0.722095in}{1.321836in}}%
\pgfpathlineto{\pgfqpoint{0.713469in}{1.322942in}}%
\pgfpathlineto{\pgfqpoint{0.705240in}{1.326132in}}%
\pgfpathlineto{\pgfqpoint{0.695751in}{1.332182in}}%
\pgfpathlineto{\pgfqpoint{0.685515in}{1.341427in}}%
\pgfpathlineto{\pgfqpoint{0.674930in}{1.353938in}}%
\pgfpathlineto{\pgfqpoint{0.663033in}{1.371716in}}%
\pgfpathlineto{\pgfqpoint{0.651579in}{1.392868in}}%
\pgfpathlineto{\pgfqpoint{0.639719in}{1.419469in}}%
\pgfpathlineto{\pgfqpoint{0.627905in}{1.451582in}}%
\pgfpathlineto{\pgfqpoint{0.616458in}{1.489187in}}%
\pgfpathlineto{\pgfqpoint{0.606962in}{1.527492in}}%
\pgfpathlineto{\pgfqpoint{0.596234in}{1.578299in}}%
\pgfpathlineto{\pgfqpoint{0.585522in}{1.641841in}}%
\pgfpathlineto{\pgfqpoint{0.577351in}{1.703361in}}%
\pgfpathlineto{\pgfqpoint{0.570306in}{1.770071in}}%
\pgfpathlineto{\pgfqpoint{0.567683in}{1.799788in}}%
\pgfpathlineto{\pgfqpoint{0.561136in}{1.886585in}}%
\pgfpathlineto{\pgfqpoint{0.556017in}{1.983484in}}%
\pgfpathlineto{\pgfqpoint{0.552806in}{2.085472in}}%
\pgfpathlineto{\pgfqpoint{0.551513in}{2.192496in}}%
\pgfpathlineto{\pgfqpoint{0.552440in}{2.297034in}}%
\pgfpathlineto{\pgfqpoint{0.555524in}{2.394045in}}%
\pgfpathlineto{\pgfqpoint{0.560374in}{2.475997in}}%
\pgfpathlineto{\pgfqpoint{0.566465in}{2.542839in}}%
\pgfpathlineto{\pgfqpoint{0.573774in}{2.599471in}}%
\pgfpathlineto{\pgfqpoint{0.581965in}{2.645821in}}%
\pgfpathlineto{\pgfqpoint{0.592080in}{2.689089in}}%
\pgfpathlineto{\pgfqpoint{0.601804in}{2.719466in}}%
\pgfpathlineto{\pgfqpoint{0.611956in}{2.744250in}}%
\pgfpathlineto{\pgfqpoint{0.623069in}{2.765636in}}%
\pgfpathlineto{\pgfqpoint{0.634817in}{2.783543in}}%
\pgfpathlineto{\pgfqpoint{0.648289in}{2.799779in}}%
\pgfpathlineto{\pgfqpoint{0.661634in}{2.812545in}}%
\pgfpathlineto{\pgfqpoint{0.677955in}{2.824896in}}%
\pgfpathlineto{\pgfqpoint{0.695341in}{2.835177in}}%
\pgfpathlineto{\pgfqpoint{0.713523in}{2.843453in}}%
\pgfpathlineto{\pgfqpoint{0.736338in}{2.851685in}}%
\pgfpathlineto{\pgfqpoint{0.763849in}{2.859151in}}%
\pgfpathlineto{\pgfqpoint{0.795993in}{2.865545in}}%
\pgfpathlineto{\pgfqpoint{0.837010in}{2.871335in}}%
\pgfpathlineto{\pgfqpoint{0.889034in}{2.876311in}}%
\pgfpathlineto{\pgfqpoint{0.958543in}{2.880566in}}%
\pgfpathlineto{\pgfqpoint{1.054208in}{2.884047in}}%
\pgfpathlineto{\pgfqpoint{1.193408in}{2.886753in}}%
\pgfpathlineto{\pgfqpoint{2.107037in}{2.890551in}}%
\pgfpathlineto{\pgfqpoint{3.390482in}{2.890625in}}%
\pgfpathlineto{\pgfqpoint{4.071358in}{2.888987in}}%
\pgfpathlineto{\pgfqpoint{4.310633in}{2.886422in}}%
\pgfpathlineto{\pgfqpoint{4.430239in}{2.883069in}}%
\pgfpathlineto{\pgfqpoint{4.501935in}{2.879022in}}%
\pgfpathlineto{\pgfqpoint{4.551778in}{2.874107in}}%
\pgfpathlineto{\pgfqpoint{4.586248in}{2.868630in}}%
\pgfpathlineto{\pgfqpoint{4.613920in}{2.862001in}}%
\pgfpathlineto{\pgfqpoint{4.634734in}{2.854797in}}%
\pgfpathlineto{\pgfqpoint{4.650800in}{2.847167in}}%
\pgfpathlineto{\pgfqpoint{4.665965in}{2.837429in}}%
\pgfpathlineto{\pgfqpoint{4.678130in}{2.826969in}}%
\pgfpathlineto{\pgfqpoint{4.688958in}{2.814737in}}%
\pgfpathlineto{\pgfqpoint{4.698292in}{2.800984in}}%
\pgfpathlineto{\pgfqpoint{4.707212in}{2.783899in}}%
\pgfpathlineto{\pgfqpoint{4.715320in}{2.763518in}}%
\pgfpathlineto{\pgfqpoint{4.723086in}{2.737629in}}%
\pgfpathlineto{\pgfqpoint{4.730052in}{2.706274in}}%
\pgfpathlineto{\pgfqpoint{4.736512in}{2.667144in}}%
\pgfpathlineto{\pgfqpoint{4.742668in}{2.615353in}}%
\pgfpathlineto{\pgfqpoint{4.748203in}{2.548447in}}%
\pgfpathlineto{\pgfqpoint{4.752888in}{2.463986in}}%
\pgfpathlineto{\pgfqpoint{4.756813in}{2.352064in}}%
\pgfpathlineto{\pgfqpoint{4.759543in}{2.207726in}}%
\pgfpathlineto{\pgfqpoint{4.760603in}{2.030998in}}%
\pgfpathlineto{\pgfqpoint{4.759524in}{1.839334in}}%
\pgfpathlineto{\pgfqpoint{4.756289in}{1.655172in}}%
\pgfpathlineto{\pgfqpoint{4.751199in}{1.493481in}}%
\pgfpathlineto{\pgfqpoint{4.744973in}{1.366734in}}%
\pgfpathlineto{\pgfqpoint{4.737709in}{1.262522in}}%
\pgfpathlineto{\pgfqpoint{4.729339in}{1.173428in}}%
\pgfpathlineto{\pgfqpoint{4.720401in}{1.101973in}}%
\pgfpathlineto{\pgfqpoint{4.710592in}{1.040768in}}%
\pgfpathlineto{\pgfqpoint{4.699805in}{0.987420in}}%
\pgfpathlineto{\pgfqpoint{4.688379in}{0.941975in}}%
\pgfpathlineto{\pgfqpoint{4.676076in}{0.902076in}}%
\pgfpathlineto{\pgfqpoint{4.676076in}{0.902076in}}%
\pgfusepath{stroke}%
\end{pgfscope}%
\begin{pgfscope}%
\pgfpathrectangle{\pgfqpoint{0.448634in}{0.402556in}}{\pgfqpoint{4.350661in}{2.489204in}} %
\pgfusepath{clip}%
\pgfsetrectcap%
\pgfsetroundjoin%
\pgfsetlinewidth{1.003750pt}%
\definecolor{currentstroke}{rgb}{0.549020,0.337255,0.294118}%
\pgfsetstrokecolor{currentstroke}%
\pgfsetdash{}{0pt}%
\pgfpathmoveto{\pgfqpoint{2.795520in}{1.982745in}}%
\pgfpathlineto{\pgfqpoint{2.781779in}{1.874357in}}%
\pgfpathlineto{\pgfqpoint{2.769350in}{1.758235in}}%
\pgfpathlineto{\pgfqpoint{2.758093in}{1.631942in}}%
\pgfpathlineto{\pgfqpoint{2.747785in}{1.490551in}}%
\pgfpathlineto{\pgfqpoint{2.738643in}{1.334082in}}%
\pgfpathlineto{\pgfqpoint{2.730580in}{1.157591in}}%
\pgfpathlineto{\pgfqpoint{2.723334in}{0.948663in}}%
\pgfpathlineto{\pgfqpoint{2.709781in}{0.530788in}}%
\pgfpathlineto{\pgfqpoint{2.705866in}{0.488717in}}%
\pgfpathlineto{\pgfqpoint{2.701767in}{0.464281in}}%
\pgfpathlineto{\pgfqpoint{2.697019in}{0.447744in}}%
\pgfpathlineto{\pgfqpoint{2.691856in}{0.436812in}}%
\pgfpathlineto{\pgfqpoint{2.686241in}{0.429230in}}%
\pgfpathlineto{\pgfqpoint{2.679344in}{0.423189in}}%
\pgfpathlineto{\pgfqpoint{2.669536in}{0.417858in}}%
\pgfpathlineto{\pgfqpoint{2.656984in}{0.413811in}}%
\pgfpathlineto{\pgfqpoint{2.637650in}{0.410338in}}%
\pgfpathlineto{\pgfqpoint{2.607293in}{0.407618in}}%
\pgfpathlineto{\pgfqpoint{2.555118in}{0.405574in}}%
\pgfpathlineto{\pgfqpoint{2.450710in}{0.404140in}}%
\pgfpathlineto{\pgfqpoint{2.176620in}{0.403276in}}%
\pgfpathlineto{\pgfqpoint{1.130286in}{0.402953in}}%
\pgfpathlineto{\pgfqpoint{0.516846in}{0.404175in}}%
\pgfpathlineto{\pgfqpoint{0.466845in}{0.405971in}}%
\pgfpathlineto{\pgfqpoint{0.456126in}{0.407933in}}%
\pgfpathlineto{\pgfqpoint{0.452337in}{0.410307in}}%
\pgfpathlineto{\pgfqpoint{0.450346in}{0.414667in}}%
\pgfpathlineto{\pgfqpoint{0.449266in}{0.424529in}}%
\pgfpathlineto{\pgfqpoint{0.448771in}{0.464349in}}%
\pgfpathlineto{\pgfqpoint{0.448640in}{0.850176in}}%
\pgfpathlineto{\pgfqpoint{0.448653in}{2.891323in}}%
\pgfpathlineto{\pgfqpoint{0.448653in}{2.891323in}}%
\pgfusepath{stroke}%
\end{pgfscope}%
\begin{pgfscope}%
\pgfpathrectangle{\pgfqpoint{0.448634in}{0.402556in}}{\pgfqpoint{4.350661in}{2.489204in}} %
\pgfusepath{clip}%
\pgfsetrectcap%
\pgfsetroundjoin%
\pgfsetlinewidth{1.003750pt}%
\definecolor{currentstroke}{rgb}{0.549020,0.337255,0.294118}%
\pgfsetstrokecolor{currentstroke}%
\pgfsetdash{}{0pt}%
\pgfpathmoveto{\pgfqpoint{3.428117in}{0.402585in}}%
\pgfpathlineto{\pgfqpoint{2.782049in}{0.403685in}}%
\pgfpathlineto{\pgfqpoint{2.753833in}{0.405634in}}%
\pgfpathlineto{\pgfqpoint{2.743250in}{0.408383in}}%
\pgfpathlineto{\pgfqpoint{2.737639in}{0.412125in}}%
\pgfpathlineto{\pgfqpoint{2.733599in}{0.417940in}}%
\pgfpathlineto{\pgfqpoint{2.730595in}{0.427259in}}%
\pgfpathlineto{\pgfqpoint{2.728349in}{0.441960in}}%
\pgfpathlineto{\pgfqpoint{2.726518in}{0.471751in}}%
\pgfpathlineto{\pgfqpoint{2.725199in}{0.533960in}}%
\pgfpathlineto{\pgfqpoint{2.725158in}{0.655929in}}%
\pgfpathlineto{\pgfqpoint{2.727369in}{0.832643in}}%
\pgfpathlineto{\pgfqpoint{2.732251in}{1.041659in}}%
\pgfpathlineto{\pgfqpoint{2.738843in}{1.223213in}}%
\pgfpathlineto{\pgfqpoint{2.747069in}{1.389722in}}%
\pgfpathlineto{\pgfqpoint{2.756598in}{1.538674in}}%
\pgfpathlineto{\pgfqpoint{2.768942in}{1.694844in}}%
\pgfpathlineto{\pgfqpoint{2.781214in}{1.816001in}}%
\pgfpathlineto{\pgfqpoint{2.794387in}{1.924482in}}%
\pgfpathlineto{\pgfqpoint{2.812721in}{2.054680in}}%
\pgfpathlineto{\pgfqpoint{2.828756in}{2.147470in}}%
\pgfpathlineto{\pgfqpoint{2.847361in}{2.242183in}}%
\pgfpathlineto{\pgfqpoint{2.895794in}{2.479659in}}%
\pgfpathlineto{\pgfqpoint{2.900181in}{2.516649in}}%
\pgfpathlineto{\pgfqpoint{2.901324in}{2.543989in}}%
\pgfpathlineto{\pgfqpoint{2.900272in}{2.566348in}}%
\pgfpathlineto{\pgfqpoint{2.897318in}{2.585960in}}%
\pgfpathlineto{\pgfqpoint{2.892822in}{2.602595in}}%
\pgfpathlineto{\pgfqpoint{2.886384in}{2.618369in}}%
\pgfpathlineto{\pgfqpoint{2.878050in}{2.632935in}}%
\pgfpathlineto{\pgfqpoint{2.868060in}{2.646068in}}%
\pgfpathlineto{\pgfqpoint{2.855047in}{2.659271in}}%
\pgfpathlineto{\pgfqpoint{2.840799in}{2.670688in}}%
\pgfpathlineto{\pgfqpoint{2.821821in}{2.682835in}}%
\pgfpathlineto{\pgfqpoint{2.799980in}{2.694002in}}%
\pgfpathlineto{\pgfqpoint{2.773366in}{2.704922in}}%
\pgfpathlineto{\pgfqpoint{2.742013in}{2.715246in}}%
\pgfpathlineto{\pgfqpoint{2.705984in}{2.724767in}}%
\pgfpathlineto{\pgfqpoint{2.663202in}{2.733794in}}%
\pgfpathlineto{\pgfqpoint{2.611537in}{2.742364in}}%
\pgfpathlineto{\pgfqpoint{2.551004in}{2.750076in}}%
\pgfpathlineto{\pgfqpoint{2.481634in}{2.756669in}}%
\pgfpathlineto{\pgfqpoint{2.399114in}{2.762188in}}%
\pgfpathlineto{\pgfqpoint{2.309987in}{2.765874in}}%
\pgfpathlineto{\pgfqpoint{2.188186in}{2.768086in}}%
\pgfpathlineto{\pgfqpoint{2.081597in}{2.767608in}}%
\pgfpathlineto{\pgfqpoint{1.968508in}{2.764829in}}%
\pgfpathlineto{\pgfqpoint{1.864182in}{2.759907in}}%
\pgfpathlineto{\pgfqpoint{1.757789in}{2.752566in}}%
\pgfpathlineto{\pgfqpoint{1.671089in}{2.744143in}}%
\pgfpathlineto{\pgfqpoint{1.591078in}{2.734163in}}%
\pgfpathlineto{\pgfqpoint{1.502693in}{2.720680in}}%
\pgfpathlineto{\pgfqpoint{1.427655in}{2.706083in}}%
\pgfpathlineto{\pgfqpoint{1.372350in}{2.692544in}}%
\pgfpathlineto{\pgfqpoint{1.321734in}{2.677921in}}%
\pgfpathlineto{\pgfqpoint{1.273765in}{2.661663in}}%
\pgfpathlineto{\pgfqpoint{1.230568in}{2.644668in}}%
\pgfpathlineto{\pgfqpoint{1.192198in}{2.627101in}}%
\pgfpathlineto{\pgfqpoint{1.156621in}{2.608398in}}%
\pgfpathlineto{\pgfqpoint{1.123892in}{2.588711in}}%
\pgfpathlineto{\pgfqpoint{1.095887in}{2.569559in}}%
\pgfpathlineto{\pgfqpoint{1.072819in}{2.550889in}}%
\pgfpathlineto{\pgfqpoint{1.046710in}{2.528503in}}%
\pgfpathlineto{\pgfqpoint{1.021824in}{2.504362in}}%
\pgfpathlineto{\pgfqpoint{0.998291in}{2.478507in}}%
\pgfpathlineto{\pgfqpoint{0.976212in}{2.451022in}}%
\pgfpathlineto{\pgfqpoint{0.955657in}{2.422031in}}%
\pgfpathlineto{\pgfqpoint{0.935447in}{2.389615in}}%
\pgfpathlineto{\pgfqpoint{0.919476in}{2.359952in}}%
\pgfpathlineto{\pgfqpoint{0.901532in}{2.322958in}}%
\pgfpathlineto{\pgfqpoint{0.884546in}{2.282594in}}%
\pgfpathlineto{\pgfqpoint{0.868712in}{2.238914in}}%
\pgfpathlineto{\pgfqpoint{0.854169in}{2.192000in}}%
\pgfpathlineto{\pgfqpoint{0.843729in}{2.151421in}}%
\pgfpathlineto{\pgfqpoint{0.831698in}{2.098422in}}%
\pgfpathlineto{\pgfqpoint{0.820721in}{2.040022in}}%
\pgfpathlineto{\pgfqpoint{0.812610in}{1.986057in}}%
\pgfpathlineto{\pgfqpoint{0.807169in}{1.941689in}}%
\pgfpathlineto{\pgfqpoint{0.799921in}{1.869983in}}%
\pgfpathlineto{\pgfqpoint{0.793788in}{1.783148in}}%
\pgfpathlineto{\pgfqpoint{0.789866in}{1.693652in}}%
\pgfpathlineto{\pgfqpoint{0.786536in}{1.561781in}}%
\pgfpathlineto{\pgfqpoint{0.785525in}{1.521970in}}%
\pgfpathlineto{\pgfqpoint{0.785525in}{1.521970in}}%
\pgfusepath{stroke}%
\end{pgfscope}%
\begin{pgfscope}%
\pgfpathrectangle{\pgfqpoint{0.448634in}{0.402556in}}{\pgfqpoint{4.350661in}{2.489204in}} %
\pgfusepath{clip}%
\pgfsetrectcap%
\pgfsetroundjoin%
\pgfsetlinewidth{1.003750pt}%
\definecolor{currentstroke}{rgb}{0.549020,0.337255,0.294118}%
\pgfsetstrokecolor{currentstroke}%
\pgfsetdash{}{0pt}%
\pgfpathmoveto{\pgfqpoint{2.028736in}{0.425752in}}%
\pgfpathlineto{\pgfqpoint{1.878677in}{0.421878in}}%
\pgfpathlineto{\pgfqpoint{1.676387in}{0.418996in}}%
\pgfpathlineto{\pgfqpoint{1.413176in}{0.417558in}}%
\pgfpathlineto{\pgfqpoint{1.134735in}{0.418204in}}%
\pgfpathlineto{\pgfqpoint{0.921565in}{0.420769in}}%
\pgfpathlineto{\pgfqpoint{0.782384in}{0.424523in}}%
\pgfpathlineto{\pgfqpoint{0.693283in}{0.428974in}}%
\pgfpathlineto{\pgfqpoint{0.632541in}{0.434091in}}%
\pgfpathlineto{\pgfqpoint{0.591492in}{0.439563in}}%
\pgfpathlineto{\pgfqpoint{0.561503in}{0.445594in}}%
\pgfpathlineto{\pgfqpoint{0.538349in}{0.452466in}}%
\pgfpathlineto{\pgfqpoint{0.522042in}{0.459393in}}%
\pgfpathlineto{\pgfqpoint{0.508540in}{0.467419in}}%
\pgfpathlineto{\pgfqpoint{0.497972in}{0.476160in}}%
\pgfpathlineto{\pgfqpoint{0.488790in}{0.486749in}}%
\pgfpathlineto{\pgfqpoint{0.481284in}{0.498947in}}%
\pgfpathlineto{\pgfqpoint{0.474590in}{0.514579in}}%
\pgfpathlineto{\pgfqpoint{0.469106in}{0.533466in}}%
\pgfpathlineto{\pgfqpoint{0.464439in}{0.557770in}}%
\pgfpathlineto{\pgfqpoint{0.460296in}{0.592288in}}%
\pgfpathlineto{\pgfqpoint{0.456855in}{0.641911in}}%
\pgfpathlineto{\pgfqpoint{0.454122in}{0.716519in}}%
\pgfpathlineto{\pgfqpoint{0.451978in}{0.843443in}}%
\pgfpathlineto{\pgfqpoint{0.450459in}{1.087379in}}%
\pgfpathlineto{\pgfqpoint{0.449596in}{1.657405in}}%
\pgfpathlineto{\pgfqpoint{0.450150in}{2.687935in}}%
\pgfpathlineto{\pgfqpoint{0.451781in}{2.839760in}}%
\pgfpathlineto{\pgfqpoint{0.453975in}{2.872002in}}%
\pgfpathlineto{\pgfqpoint{0.456339in}{2.881552in}}%
\pgfpathlineto{\pgfqpoint{0.458887in}{2.885549in}}%
\pgfpathlineto{\pgfqpoint{0.462553in}{2.888170in}}%
\pgfpathlineto{\pgfqpoint{0.471045in}{2.890205in}}%
\pgfpathlineto{\pgfqpoint{0.490596in}{2.891263in}}%
\pgfpathlineto{\pgfqpoint{0.564555in}{2.891692in}}%
\pgfpathlineto{\pgfqpoint{1.569558in}{2.891759in}}%
\pgfpathlineto{\pgfqpoint{4.784678in}{2.890785in}}%
\pgfpathlineto{\pgfqpoint{4.791003in}{2.889098in}}%
\pgfpathlineto{\pgfqpoint{4.793909in}{2.885555in}}%
\pgfpathlineto{\pgfqpoint{4.795579in}{2.878366in}}%
\pgfpathlineto{\pgfqpoint{4.796849in}{2.858514in}}%
\pgfpathlineto{\pgfqpoint{4.796849in}{2.858514in}}%
\pgfusepath{stroke}%
\end{pgfscope}%
\begin{pgfscope}%
\pgfpathrectangle{\pgfqpoint{0.448634in}{0.402556in}}{\pgfqpoint{4.350661in}{2.489204in}} %
\pgfusepath{clip}%
\pgfsetrectcap%
\pgfsetroundjoin%
\pgfsetlinewidth{1.003750pt}%
\definecolor{currentstroke}{rgb}{0.890196,0.466667,0.760784}%
\pgfsetstrokecolor{currentstroke}%
\pgfsetdash{}{0pt}%
\pgfpathmoveto{\pgfqpoint{0.448634in}{2.896245in}}%
\pgfpathlineto{\pgfqpoint{0.448593in}{0.407043in}}%
\pgfpathlineto{\pgfqpoint{0.448593in}{0.407043in}}%
\pgfusepath{stroke}%
\end{pgfscope}%
\begin{pgfscope}%
\pgfpathrectangle{\pgfqpoint{0.448634in}{0.402556in}}{\pgfqpoint{4.350661in}{2.489204in}} %
\pgfusepath{clip}%
\pgfsetrectcap%
\pgfsetroundjoin%
\pgfsetlinewidth{1.003750pt}%
\definecolor{currentstroke}{rgb}{0.890196,0.466667,0.760784}%
\pgfsetstrokecolor{currentstroke}%
\pgfsetdash{}{0pt}%
\pgfpathmoveto{\pgfqpoint{0.456548in}{2.521531in}}%
\pgfpathlineto{\pgfqpoint{0.459321in}{2.688275in}}%
\pgfpathlineto{\pgfqpoint{0.462584in}{2.767837in}}%
\pgfpathlineto{\pgfqpoint{0.466434in}{2.812417in}}%
\pgfpathlineto{\pgfqpoint{0.470401in}{2.836880in}}%
\pgfpathlineto{\pgfqpoint{0.475097in}{2.853436in}}%
\pgfpathlineto{\pgfqpoint{0.480331in}{2.864319in}}%
\pgfpathlineto{\pgfqpoint{0.486133in}{2.871708in}}%
\pgfpathlineto{\pgfqpoint{0.493289in}{2.877335in}}%
\pgfpathlineto{\pgfqpoint{0.503361in}{2.881980in}}%
\pgfpathlineto{\pgfqpoint{0.516088in}{2.885245in}}%
\pgfpathlineto{\pgfqpoint{0.535531in}{2.887817in}}%
\pgfpathlineto{\pgfqpoint{0.570290in}{2.889773in}}%
\pgfpathlineto{\pgfqpoint{0.639891in}{2.890990in}}%
\pgfpathlineto{\pgfqpoint{0.844372in}{2.891601in}}%
\pgfpathlineto{\pgfqpoint{2.277914in}{2.891751in}}%
\pgfpathlineto{\pgfqpoint{4.716455in}{2.890655in}}%
\pgfpathlineto{\pgfqpoint{4.749040in}{2.888857in}}%
\pgfpathlineto{\pgfqpoint{4.761927in}{2.886562in}}%
\pgfpathlineto{\pgfqpoint{4.770120in}{2.883274in}}%
\pgfpathlineto{\pgfqpoint{4.775461in}{2.879026in}}%
\pgfpathlineto{\pgfqpoint{4.779506in}{2.873197in}}%
\pgfpathlineto{\pgfqpoint{4.782987in}{2.864095in}}%
\pgfpathlineto{\pgfqpoint{4.785981in}{2.849570in}}%
\pgfpathlineto{\pgfqpoint{4.788602in}{2.824867in}}%
\pgfpathlineto{\pgfqpoint{4.790853in}{2.780140in}}%
\pgfpathlineto{\pgfqpoint{4.792814in}{2.690559in}}%
\pgfpathlineto{\pgfqpoint{4.794328in}{2.501388in}}%
\pgfpathlineto{\pgfqpoint{4.795014in}{2.098138in}}%
\pgfpathlineto{\pgfqpoint{4.793872in}{1.548026in}}%
\pgfpathlineto{\pgfqpoint{4.791205in}{1.214488in}}%
\pgfpathlineto{\pgfqpoint{4.787643in}{1.022864in}}%
\pgfpathlineto{\pgfqpoint{4.783233in}{0.898510in}}%
\pgfpathlineto{\pgfqpoint{4.778134in}{0.814082in}}%
\pgfpathlineto{\pgfqpoint{4.772267in}{0.752221in}}%
\pgfpathlineto{\pgfqpoint{4.765682in}{0.705536in}}%
\pgfpathlineto{\pgfqpoint{4.758394in}{0.669150in}}%
\pgfpathlineto{\pgfqpoint{4.750592in}{0.640654in}}%
\pgfpathlineto{\pgfqpoint{4.741331in}{0.615420in}}%
\pgfpathlineto{\pgfqpoint{4.731892in}{0.595805in}}%
\pgfpathlineto{\pgfqpoint{4.720555in}{0.577558in}}%
\pgfpathlineto{\pgfqpoint{4.708860in}{0.562826in}}%
\pgfpathlineto{\pgfqpoint{4.695778in}{0.549709in}}%
\pgfpathlineto{\pgfqpoint{4.679698in}{0.536949in}}%
\pgfpathlineto{\pgfqpoint{4.662543in}{0.526172in}}%
\pgfpathlineto{\pgfqpoint{4.642616in}{0.516206in}}%
\pgfpathlineto{\pgfqpoint{4.617936in}{0.506501in}}%
\pgfpathlineto{\pgfqpoint{4.588511in}{0.497538in}}%
\pgfpathlineto{\pgfqpoint{4.554417in}{0.489550in}}%
\pgfpathlineto{\pgfqpoint{4.513581in}{0.482272in}}%
\pgfpathlineto{\pgfqpoint{4.461721in}{0.475415in}}%
\pgfpathlineto{\pgfqpoint{4.398854in}{0.469456in}}%
\pgfpathlineto{\pgfqpoint{4.320669in}{0.464387in}}%
\pgfpathlineto{\pgfqpoint{4.225016in}{0.460517in}}%
\pgfpathlineto{\pgfqpoint{4.111918in}{0.458215in}}%
\pgfpathlineto{\pgfqpoint{3.983575in}{0.457870in}}%
\pgfpathlineto{\pgfqpoint{3.853067in}{0.459733in}}%
\pgfpathlineto{\pgfqpoint{3.731298in}{0.463649in}}%
\pgfpathlineto{\pgfqpoint{3.624820in}{0.469241in}}%
\pgfpathlineto{\pgfqpoint{3.533662in}{0.476225in}}%
\pgfpathlineto{\pgfqpoint{3.457847in}{0.484193in}}%
\pgfpathlineto{\pgfqpoint{3.393062in}{0.493159in}}%
\pgfpathlineto{\pgfqpoint{3.337178in}{0.503101in}}%
\pgfpathlineto{\pgfqpoint{3.288085in}{0.514125in}}%
\pgfpathlineto{\pgfqpoint{3.245824in}{0.525930in}}%
\pgfpathlineto{\pgfqpoint{3.208320in}{0.538786in}}%
\pgfpathlineto{\pgfqpoint{3.175616in}{0.552399in}}%
\pgfpathlineto{\pgfqpoint{3.147703in}{0.566320in}}%
\pgfpathlineto{\pgfqpoint{3.122584in}{0.581173in}}%
\pgfpathlineto{\pgfqpoint{3.098474in}{0.598070in}}%
\pgfpathlineto{\pgfqpoint{3.077345in}{0.615597in}}%
\pgfpathlineto{\pgfqpoint{3.057512in}{0.635005in}}%
\pgfpathlineto{\pgfqpoint{3.039158in}{0.656234in}}%
\pgfpathlineto{\pgfqpoint{3.022412in}{0.679136in}}%
\pgfpathlineto{\pgfqpoint{3.007327in}{0.703505in}}%
\pgfpathlineto{\pgfqpoint{2.992837in}{0.731284in}}%
\pgfpathlineto{\pgfqpoint{2.979270in}{0.762475in}}%
\pgfpathlineto{\pgfqpoint{2.966867in}{0.797003in}}%
\pgfpathlineto{\pgfqpoint{2.955783in}{0.834750in}}%
\pgfpathlineto{\pgfqpoint{2.946095in}{0.875582in}}%
\pgfpathlineto{\pgfqpoint{2.937426in}{0.921820in}}%
\pgfpathlineto{\pgfqpoint{2.930070in}{0.973406in}}%
\pgfpathlineto{\pgfqpoint{2.924257in}{1.030266in}}%
\pgfpathlineto{\pgfqpoint{2.920184in}{1.092317in}}%
\pgfpathlineto{\pgfqpoint{2.918045in}{1.159477in}}%
\pgfpathlineto{\pgfqpoint{2.918044in}{1.231660in}}%
\pgfpathlineto{\pgfqpoint{2.920349in}{1.306286in}}%
\pgfpathlineto{\pgfqpoint{2.924884in}{1.380778in}}%
\pgfpathlineto{\pgfqpoint{2.931649in}{1.455048in}}%
\pgfpathlineto{\pgfqpoint{2.940388in}{1.526535in}}%
\pgfpathlineto{\pgfqpoint{2.951017in}{1.595159in}}%
\pgfpathlineto{\pgfqpoint{2.962987in}{1.658408in}}%
\pgfpathlineto{\pgfqpoint{2.976602in}{1.718652in}}%
\pgfpathlineto{\pgfqpoint{2.991831in}{1.775789in}}%
\pgfpathlineto{\pgfqpoint{3.007850in}{1.827388in}}%
\pgfpathlineto{\pgfqpoint{3.025178in}{1.875749in}}%
\pgfpathlineto{\pgfqpoint{3.043704in}{1.920788in}}%
\pgfpathlineto{\pgfqpoint{3.063288in}{1.962430in}}%
\pgfpathlineto{\pgfqpoint{3.083744in}{2.000627in}}%
\pgfpathlineto{\pgfqpoint{3.104859in}{2.035361in}}%
\pgfpathlineto{\pgfqpoint{3.126378in}{2.066656in}}%
\pgfpathlineto{\pgfqpoint{3.149525in}{2.096391in}}%
\pgfpathlineto{\pgfqpoint{3.172703in}{2.122664in}}%
\pgfpathlineto{\pgfqpoint{3.197289in}{2.147203in}}%
\pgfpathlineto{\pgfqpoint{3.223238in}{2.169827in}}%
\pgfpathlineto{\pgfqpoint{3.248635in}{2.189049in}}%
\pgfpathlineto{\pgfqpoint{3.275105in}{2.206268in}}%
\pgfpathlineto{\pgfqpoint{3.300589in}{2.220281in}}%
\pgfpathlineto{\pgfqpoint{3.326893in}{2.232141in}}%
\pgfpathlineto{\pgfqpoint{3.351858in}{2.240837in}}%
\pgfpathlineto{\pgfqpoint{3.373132in}{2.245991in}}%
\pgfpathlineto{\pgfqpoint{3.390418in}{2.248220in}}%
\pgfpathlineto{\pgfqpoint{3.405630in}{2.248035in}}%
\pgfpathlineto{\pgfqpoint{3.416297in}{2.245722in}}%
\pgfpathlineto{\pgfqpoint{3.422268in}{2.242748in}}%
\pgfpathlineto{\pgfqpoint{3.427095in}{2.237801in}}%
\pgfpathlineto{\pgfqpoint{3.428951in}{2.233328in}}%
\pgfpathlineto{\pgfqpoint{3.429224in}{2.225922in}}%
\pgfpathlineto{\pgfqpoint{3.426532in}{2.216490in}}%
\pgfpathlineto{\pgfqpoint{3.421018in}{2.205778in}}%
\pgfpathlineto{\pgfqpoint{3.410249in}{2.190148in}}%
\pgfpathlineto{\pgfqpoint{3.390918in}{2.166536in}}%
\pgfpathlineto{\pgfqpoint{3.303099in}{2.062915in}}%
\pgfpathlineto{\pgfqpoint{3.275066in}{2.024850in}}%
\pgfpathlineto{\pgfqpoint{3.250054in}{1.987205in}}%
\pgfpathlineto{\pgfqpoint{3.226762in}{1.948142in}}%
\pgfpathlineto{\pgfqpoint{3.205290in}{1.907736in}}%
\pgfpathlineto{\pgfqpoint{3.185648in}{1.866129in}}%
\pgfpathlineto{\pgfqpoint{3.166955in}{1.821181in}}%
\pgfpathlineto{\pgfqpoint{3.150305in}{1.775192in}}%
\pgfpathlineto{\pgfqpoint{3.134911in}{1.725981in}}%
\pgfpathlineto{\pgfqpoint{3.121585in}{1.675987in}}%
\pgfpathlineto{\pgfqpoint{3.109769in}{1.622924in}}%
\pgfpathlineto{\pgfqpoint{3.101183in}{1.576666in}}%
\pgfpathlineto{\pgfqpoint{3.092909in}{1.520207in}}%
\pgfpathlineto{\pgfqpoint{3.086563in}{1.460913in}}%
\pgfpathlineto{\pgfqpoint{3.082717in}{1.403835in}}%
\pgfpathlineto{\pgfqpoint{3.081107in}{1.346618in}}%
\pgfpathlineto{\pgfqpoint{3.081710in}{1.291864in}}%
\pgfpathlineto{\pgfqpoint{3.084493in}{1.237200in}}%
\pgfpathlineto{\pgfqpoint{3.089380in}{1.185232in}}%
\pgfpathlineto{\pgfqpoint{3.096270in}{1.136081in}}%
\pgfpathlineto{\pgfqpoint{3.103874in}{1.094674in}}%
\pgfpathlineto{\pgfqpoint{3.113957in}{1.051386in}}%
\pgfpathlineto{\pgfqpoint{3.125541in}{1.011205in}}%
\pgfpathlineto{\pgfqpoint{3.138459in}{0.974229in}}%
\pgfpathlineto{\pgfqpoint{3.151389in}{0.942685in}}%
\pgfpathlineto{\pgfqpoint{3.167114in}{0.909977in}}%
\pgfpathlineto{\pgfqpoint{3.185917in}{0.876471in}}%
\pgfpathlineto{\pgfqpoint{3.204351in}{0.848740in}}%
\pgfpathlineto{\pgfqpoint{3.224419in}{0.822538in}}%
\pgfpathlineto{\pgfqpoint{3.246087in}{0.798059in}}%
\pgfpathlineto{\pgfqpoint{3.269213in}{0.775394in}}%
\pgfpathlineto{\pgfqpoint{3.293603in}{0.754535in}}%
\pgfpathlineto{\pgfqpoint{3.320960in}{0.734197in}}%
\pgfpathlineto{\pgfqpoint{3.349392in}{0.715888in}}%
\pgfpathlineto{\pgfqpoint{3.380696in}{0.698497in}}%
\pgfpathlineto{\pgfqpoint{3.414773in}{0.682074in}}%
\pgfpathlineto{\pgfqpoint{3.445515in}{0.669583in}}%
\pgfpathlineto{\pgfqpoint{3.480801in}{0.656968in}}%
\pgfpathlineto{\pgfqpoint{3.569493in}{0.632049in}}%
\pgfpathlineto{\pgfqpoint{3.629422in}{0.619637in}}%
\pgfpathlineto{\pgfqpoint{3.685385in}{0.610287in}}%
\pgfpathlineto{\pgfqpoint{3.756674in}{0.600859in}}%
\pgfpathlineto{\pgfqpoint{3.821736in}{0.595078in}}%
\pgfpathlineto{\pgfqpoint{3.886915in}{0.591427in}}%
\pgfpathlineto{\pgfqpoint{3.958680in}{0.589509in}}%
\pgfpathlineto{\pgfqpoint{4.030462in}{0.589870in}}%
\pgfpathlineto{\pgfqpoint{4.100033in}{0.592487in}}%
\pgfpathlineto{\pgfqpoint{4.149978in}{0.595837in}}%
\pgfpathlineto{\pgfqpoint{4.210652in}{0.601914in}}%
\pgfpathlineto{\pgfqpoint{4.262495in}{0.608899in}}%
\pgfpathlineto{\pgfqpoint{4.311905in}{0.617875in}}%
\pgfpathlineto{\pgfqpoint{4.356677in}{0.628240in}}%
\pgfpathlineto{\pgfqpoint{4.396752in}{0.639789in}}%
\pgfpathlineto{\pgfqpoint{4.432093in}{0.652233in}}%
\pgfpathlineto{\pgfqpoint{4.464709in}{0.666117in}}%
\pgfpathlineto{\pgfqpoint{4.498560in}{0.683074in}}%
\pgfpathlineto{\pgfqpoint{4.523387in}{0.698555in}}%
\pgfpathlineto{\pgfqpoint{4.547148in}{0.716087in}}%
\pgfpathlineto{\pgfqpoint{4.567919in}{0.734166in}}%
\pgfpathlineto{\pgfqpoint{4.587373in}{0.754072in}}%
\pgfpathlineto{\pgfqpoint{4.605346in}{0.775723in}}%
\pgfpathlineto{\pgfqpoint{4.621668in}{0.799021in}}%
\pgfpathlineto{\pgfqpoint{4.636437in}{0.823643in}}%
\pgfpathlineto{\pgfqpoint{4.650670in}{0.851597in}}%
\pgfpathlineto{\pgfqpoint{4.664076in}{0.882880in}}%
\pgfpathlineto{\pgfqpoint{4.676463in}{0.917416in}}%
\pgfpathlineto{\pgfqpoint{4.688372in}{0.957472in}}%
\pgfpathlineto{\pgfqpoint{4.698922in}{1.000616in}}%
\pgfpathlineto{\pgfqpoint{4.709119in}{1.051566in}}%
\pgfpathlineto{\pgfqpoint{4.718565in}{1.110317in}}%
\pgfpathlineto{\pgfqpoint{4.727322in}{1.179287in}}%
\pgfpathlineto{\pgfqpoint{4.734765in}{1.255977in}}%
\pgfpathlineto{\pgfqpoint{4.741833in}{1.350218in}}%
\pgfpathlineto{\pgfqpoint{4.747934in}{1.462012in}}%
\pgfpathlineto{\pgfqpoint{4.752664in}{1.588842in}}%
\pgfpathlineto{\pgfqpoint{4.756497in}{1.745599in}}%
\pgfpathlineto{\pgfqpoint{4.758709in}{1.927292in}}%
\pgfpathlineto{\pgfqpoint{4.758778in}{2.113981in}}%
\pgfpathlineto{\pgfqpoint{4.756685in}{2.283228in}}%
\pgfpathlineto{\pgfqpoint{4.752857in}{2.420061in}}%
\pgfpathlineto{\pgfqpoint{4.747863in}{2.521955in}}%
\pgfpathlineto{\pgfqpoint{4.741684in}{2.603785in}}%
\pgfpathlineto{\pgfqpoint{4.735071in}{2.660529in}}%
\pgfpathlineto{\pgfqpoint{4.727753in}{2.704537in}}%
\pgfpathlineto{\pgfqpoint{4.719960in}{2.738216in}}%
\pgfpathlineto{\pgfqpoint{4.711869in}{2.763974in}}%
\pgfpathlineto{\pgfqpoint{4.703454in}{2.784191in}}%
\pgfpathlineto{\pgfqpoint{4.694249in}{2.801077in}}%
\pgfpathlineto{\pgfqpoint{4.685882in}{2.812525in}}%
\pgfpathlineto{\pgfqpoint{4.675074in}{2.824782in}}%
\pgfpathlineto{\pgfqpoint{4.662960in}{2.835319in}}%
\pgfpathlineto{\pgfqpoint{4.647866in}{2.845201in}}%
\pgfpathlineto{\pgfqpoint{4.631864in}{2.853006in}}%
\pgfpathlineto{\pgfqpoint{4.611109in}{2.860429in}}%
\pgfpathlineto{\pgfqpoint{4.585624in}{2.866865in}}%
\pgfpathlineto{\pgfqpoint{4.553364in}{2.872435in}}%
\pgfpathlineto{\pgfqpoint{4.512236in}{2.877067in}}%
\pgfpathlineto{\pgfqpoint{4.455788in}{2.881071in}}%
\pgfpathlineto{\pgfqpoint{4.375353in}{2.884364in}}%
\pgfpathlineto{\pgfqpoint{4.253557in}{2.886917in}}%
\pgfpathlineto{\pgfqpoint{4.044733in}{2.888912in}}%
\pgfpathlineto{\pgfqpoint{3.635772in}{2.890229in}}%
\pgfpathlineto{\pgfqpoint{2.648173in}{2.890729in}}%
\pgfpathlineto{\pgfqpoint{1.614892in}{2.889435in}}%
\pgfpathlineto{\pgfqpoint{1.255969in}{2.887033in}}%
\pgfpathlineto{\pgfqpoint{1.073266in}{2.883738in}}%
\pgfpathlineto{\pgfqpoint{0.964560in}{2.879629in}}%
\pgfpathlineto{\pgfqpoint{0.890726in}{2.874718in}}%
\pgfpathlineto{\pgfqpoint{0.836579in}{2.868969in}}%
\pgfpathlineto{\pgfqpoint{0.795641in}{2.862496in}}%
\pgfpathlineto{\pgfqpoint{0.763618in}{2.855358in}}%
\pgfpathlineto{\pgfqpoint{0.736294in}{2.847050in}}%
\pgfpathlineto{\pgfqpoint{0.713741in}{2.837922in}}%
\pgfpathlineto{\pgfqpoint{0.693996in}{2.827502in}}%
\pgfpathlineto{\pgfqpoint{0.677156in}{2.816094in}}%
\pgfpathlineto{\pgfqpoint{0.661526in}{2.802625in}}%
\pgfpathlineto{\pgfqpoint{0.647387in}{2.787148in}}%
\pgfpathlineto{\pgfqpoint{0.634902in}{2.769907in}}%
\pgfpathlineto{\pgfqpoint{0.624075in}{2.751252in}}%
\pgfpathlineto{\pgfqpoint{0.613833in}{2.729302in}}%
\pgfpathlineto{\pgfqpoint{0.603666in}{2.701800in}}%
\pgfpathlineto{\pgfqpoint{0.594752in}{2.671098in}}%
\pgfpathlineto{\pgfqpoint{0.586546in}{2.634966in}}%
\pgfpathlineto{\pgfqpoint{0.578886in}{2.591031in}}%
\pgfpathlineto{\pgfqpoint{0.572140in}{2.539335in}}%
\pgfpathlineto{\pgfqpoint{0.566312in}{2.477467in}}%
\pgfpathlineto{\pgfqpoint{0.561594in}{2.402990in}}%
\pgfpathlineto{\pgfqpoint{0.558401in}{2.315947in}}%
\pgfpathlineto{\pgfqpoint{0.557103in}{2.218882in}}%
\pgfpathlineto{\pgfqpoint{0.557974in}{2.114343in}}%
\pgfpathlineto{\pgfqpoint{0.559557in}{2.052140in}}%
\pgfpathlineto{\pgfqpoint{0.559557in}{2.052140in}}%
\pgfusepath{stroke}%
\end{pgfscope}%
\begin{pgfscope}%
\pgfpathrectangle{\pgfqpoint{0.448634in}{0.402556in}}{\pgfqpoint{4.350661in}{2.489204in}} %
\pgfusepath{clip}%
\pgfsetrectcap%
\pgfsetroundjoin%
\pgfsetlinewidth{1.003750pt}%
\definecolor{currentstroke}{rgb}{0.890196,0.466667,0.760784}%
\pgfsetstrokecolor{currentstroke}%
\pgfsetdash{}{0pt}%
\pgfpathmoveto{\pgfqpoint{0.456424in}{1.361321in}}%
\pgfpathlineto{\pgfqpoint{0.459628in}{1.112429in}}%
\pgfpathlineto{\pgfqpoint{0.463692in}{0.958171in}}%
\pgfpathlineto{\pgfqpoint{0.468565in}{0.853776in}}%
\pgfpathlineto{\pgfqpoint{0.474207in}{0.779384in}}%
\pgfpathlineto{\pgfqpoint{0.480460in}{0.725097in}}%
\pgfpathlineto{\pgfqpoint{0.487351in}{0.683529in}}%
\pgfpathlineto{\pgfqpoint{0.495107in}{0.649838in}}%
\pgfpathlineto{\pgfqpoint{0.503910in}{0.621729in}}%
\pgfpathlineto{\pgfqpoint{0.513248in}{0.599260in}}%
\pgfpathlineto{\pgfqpoint{0.523521in}{0.580203in}}%
\pgfpathlineto{\pgfqpoint{0.534261in}{0.564547in}}%
\pgfpathlineto{\pgfqpoint{0.546478in}{0.550381in}}%
\pgfpathlineto{\pgfqpoint{0.560000in}{0.537860in}}%
\pgfpathlineto{\pgfqpoint{0.576455in}{0.525742in}}%
\pgfpathlineto{\pgfqpoint{0.595854in}{0.514501in}}%
\pgfpathlineto{\pgfqpoint{0.618087in}{0.504398in}}%
\pgfpathlineto{\pgfqpoint{0.643000in}{0.495497in}}%
\pgfpathlineto{\pgfqpoint{0.674704in}{0.486688in}}%
\pgfpathlineto{\pgfqpoint{0.713202in}{0.478537in}}%
\pgfpathlineto{\pgfqpoint{0.758443in}{0.471316in}}%
\pgfpathlineto{\pgfqpoint{0.814699in}{0.464659in}}%
\pgfpathlineto{\pgfqpoint{0.884115in}{0.458735in}}%
\pgfpathlineto{\pgfqpoint{0.971011in}{0.453595in}}%
\pgfpathlineto{\pgfqpoint{1.081890in}{0.449367in}}%
\pgfpathlineto{\pgfqpoint{1.216736in}{0.446505in}}%
\pgfpathlineto{\pgfqpoint{1.375531in}{0.445384in}}%
\pgfpathlineto{\pgfqpoint{1.545204in}{0.446415in}}%
\pgfpathlineto{\pgfqpoint{1.703978in}{0.449545in}}%
\pgfpathlineto{\pgfqpoint{1.838782in}{0.454339in}}%
\pgfpathlineto{\pgfqpoint{1.949597in}{0.460384in}}%
\pgfpathlineto{\pgfqpoint{2.042917in}{0.467663in}}%
\pgfpathlineto{\pgfqpoint{2.118725in}{0.475727in}}%
\pgfpathlineto{\pgfqpoint{2.183493in}{0.484849in}}%
\pgfpathlineto{\pgfqpoint{2.237197in}{0.494634in}}%
\pgfpathlineto{\pgfqpoint{2.281977in}{0.504948in}}%
\pgfpathlineto{\pgfqpoint{2.322059in}{0.516466in}}%
\pgfpathlineto{\pgfqpoint{2.355304in}{0.528240in}}%
\pgfpathlineto{\pgfqpoint{2.385826in}{0.541423in}}%
\pgfpathlineto{\pgfqpoint{2.411546in}{0.554861in}}%
\pgfpathlineto{\pgfqpoint{2.434452in}{0.569171in}}%
\pgfpathlineto{\pgfqpoint{2.456305in}{0.585494in}}%
\pgfpathlineto{\pgfqpoint{2.475186in}{0.602303in}}%
\pgfpathlineto{\pgfqpoint{2.492771in}{0.620860in}}%
\pgfpathlineto{\pgfqpoint{2.508919in}{0.641055in}}%
\pgfpathlineto{\pgfqpoint{2.524819in}{0.664733in}}%
\pgfpathlineto{\pgfqpoint{2.538949in}{0.689840in}}%
\pgfpathlineto{\pgfqpoint{2.552390in}{0.718302in}}%
\pgfpathlineto{\pgfqpoint{2.564931in}{0.750052in}}%
\pgfpathlineto{\pgfqpoint{2.577175in}{0.787327in}}%
\pgfpathlineto{\pgfqpoint{2.588781in}{0.830113in}}%
\pgfpathlineto{\pgfqpoint{2.600049in}{0.880766in}}%
\pgfpathlineto{\pgfqpoint{2.610664in}{0.939255in}}%
\pgfpathlineto{\pgfqpoint{2.620824in}{1.007973in}}%
\pgfpathlineto{\pgfqpoint{2.630912in}{1.091812in}}%
\pgfpathlineto{\pgfqpoint{2.641076in}{1.195707in}}%
\pgfpathlineto{\pgfqpoint{2.651801in}{1.329561in}}%
\pgfpathlineto{\pgfqpoint{2.664115in}{1.513221in}}%
\pgfpathlineto{\pgfqpoint{2.679575in}{1.778977in}}%
\pgfpathlineto{\pgfqpoint{2.689824in}{1.987739in}}%
\pgfpathlineto{\pgfqpoint{2.693518in}{2.104652in}}%
\pgfpathlineto{\pgfqpoint{2.693968in}{2.186790in}}%
\pgfpathlineto{\pgfqpoint{2.692134in}{2.248980in}}%
\pgfpathlineto{\pgfqpoint{2.688506in}{2.298584in}}%
\pgfpathlineto{\pgfqpoint{2.683220in}{2.340459in}}%
\pgfpathlineto{\pgfqpoint{2.676810in}{2.374519in}}%
\pgfpathlineto{\pgfqpoint{2.668742in}{2.405524in}}%
\pgfpathlineto{\pgfqpoint{2.659125in}{2.433282in}}%
\pgfpathlineto{\pgfqpoint{2.648246in}{2.457658in}}%
\pgfpathlineto{\pgfqpoint{2.636550in}{2.478635in}}%
\pgfpathlineto{\pgfqpoint{2.623153in}{2.498235in}}%
\pgfpathlineto{\pgfqpoint{2.608180in}{2.516278in}}%
\pgfpathlineto{\pgfqpoint{2.591819in}{2.532671in}}%
\pgfpathlineto{\pgfqpoint{2.572495in}{2.548803in}}%
\pgfpathlineto{\pgfqpoint{2.552070in}{2.563055in}}%
\pgfpathlineto{\pgfqpoint{2.528838in}{2.576663in}}%
\pgfpathlineto{\pgfqpoint{2.502848in}{2.589403in}}%
\pgfpathlineto{\pgfqpoint{2.472108in}{2.601910in}}%
\pgfpathlineto{\pgfqpoint{2.438708in}{2.613094in}}%
\pgfpathlineto{\pgfqpoint{2.400626in}{2.623499in}}%
\pgfpathlineto{\pgfqpoint{2.357899in}{2.632861in}}%
\pgfpathlineto{\pgfqpoint{2.308418in}{2.641309in}}%
\pgfpathlineto{\pgfqpoint{2.254371in}{2.648199in}}%
\pgfpathlineto{\pgfqpoint{2.195818in}{2.653427in}}%
\pgfpathlineto{\pgfqpoint{2.130633in}{2.656950in}}%
\pgfpathlineto{\pgfqpoint{2.063211in}{2.658369in}}%
\pgfpathlineto{\pgfqpoint{1.991431in}{2.657608in}}%
\pgfpathlineto{\pgfqpoint{1.919695in}{2.654603in}}%
\pgfpathlineto{\pgfqpoint{1.850229in}{2.649514in}}%
\pgfpathlineto{\pgfqpoint{1.783081in}{2.642432in}}%
\pgfpathlineto{\pgfqpoint{1.718304in}{2.633384in}}%
\pgfpathlineto{\pgfqpoint{1.658109in}{2.622762in}}%
\pgfpathlineto{\pgfqpoint{1.602533in}{2.610769in}}%
\pgfpathlineto{\pgfqpoint{1.549495in}{2.597032in}}%
\pgfpathlineto{\pgfqpoint{1.501165in}{2.582235in}}%
\pgfpathlineto{\pgfqpoint{1.457561in}{2.566664in}}%
\pgfpathlineto{\pgfqpoint{1.416642in}{2.549763in}}%
\pgfpathlineto{\pgfqpoint{1.378478in}{2.531619in}}%
\pgfpathlineto{\pgfqpoint{1.343128in}{2.512361in}}%
\pgfpathlineto{\pgfqpoint{1.310635in}{2.492167in}}%
\pgfpathlineto{\pgfqpoint{1.281015in}{2.471263in}}%
\pgfpathlineto{\pgfqpoint{1.252503in}{2.448431in}}%
\pgfpathlineto{\pgfqpoint{1.226943in}{2.425230in}}%
\pgfpathlineto{\pgfqpoint{1.202672in}{2.400283in}}%
\pgfpathlineto{\pgfqpoint{1.179834in}{2.373624in}}%
\pgfpathlineto{\pgfqpoint{1.158558in}{2.345324in}}%
\pgfpathlineto{\pgfqpoint{1.138944in}{2.315493in}}%
\pgfpathlineto{\pgfqpoint{1.121063in}{2.284268in}}%
\pgfpathlineto{\pgfqpoint{1.104953in}{2.251805in}}%
\pgfpathlineto{\pgfqpoint{1.089728in}{2.215998in}}%
\pgfpathlineto{\pgfqpoint{1.076501in}{2.179165in}}%
\pgfpathlineto{\pgfqpoint{1.064586in}{2.139113in}}%
\pgfpathlineto{\pgfqpoint{1.054802in}{2.098310in}}%
\pgfpathlineto{\pgfqpoint{1.046677in}{2.054487in}}%
\pgfpathlineto{\pgfqpoint{1.040762in}{2.010201in}}%
\pgfpathlineto{\pgfqpoint{1.036978in}{1.965610in}}%
\pgfpathlineto{\pgfqpoint{1.035228in}{1.918364in}}%
\pgfpathlineto{\pgfqpoint{1.035734in}{1.871078in}}%
\pgfpathlineto{\pgfqpoint{1.038475in}{1.823893in}}%
\pgfpathlineto{\pgfqpoint{1.043451in}{1.776949in}}%
\pgfpathlineto{\pgfqpoint{1.050689in}{1.730391in}}%
\pgfpathlineto{\pgfqpoint{1.059672in}{1.686786in}}%
\pgfpathlineto{\pgfqpoint{1.070775in}{1.643826in}}%
\pgfpathlineto{\pgfqpoint{1.083259in}{1.603999in}}%
\pgfpathlineto{\pgfqpoint{1.097731in}{1.565064in}}%
\pgfpathlineto{\pgfqpoint{1.114229in}{1.527200in}}%
\pgfpathlineto{\pgfqpoint{1.131628in}{1.492713in}}%
\pgfpathlineto{\pgfqpoint{1.150841in}{1.459511in}}%
\pgfpathlineto{\pgfqpoint{1.171841in}{1.427758in}}%
\pgfpathlineto{\pgfqpoint{1.194574in}{1.397610in}}%
\pgfpathlineto{\pgfqpoint{1.217388in}{1.370923in}}%
\pgfpathlineto{\pgfqpoint{1.241565in}{1.345856in}}%
\pgfpathlineto{\pgfqpoint{1.268736in}{1.320976in}}%
\pgfpathlineto{\pgfqpoint{1.297194in}{1.298055in}}%
\pgfpathlineto{\pgfqpoint{1.326787in}{1.277102in}}%
\pgfpathlineto{\pgfqpoint{1.357366in}{1.258091in}}%
\pgfpathlineto{\pgfqpoint{1.390775in}{1.239960in}}%
\pgfpathlineto{\pgfqpoint{1.424974in}{1.223869in}}%
\pgfpathlineto{\pgfqpoint{1.461884in}{1.208929in}}%
\pgfpathlineto{\pgfqpoint{1.501466in}{1.195329in}}%
\pgfpathlineto{\pgfqpoint{1.543661in}{1.183213in}}%
\pgfpathlineto{\pgfqpoint{1.590539in}{1.172214in}}%
\pgfpathlineto{\pgfqpoint{1.644190in}{1.162052in}}%
\pgfpathlineto{\pgfqpoint{1.777408in}{1.138269in}}%
\pgfpathlineto{\pgfqpoint{1.798283in}{1.131314in}}%
\pgfpathlineto{\pgfqpoint{1.810262in}{1.125410in}}%
\pgfpathlineto{\pgfqpoint{1.819400in}{1.118698in}}%
\pgfpathlineto{\pgfqpoint{1.825387in}{1.111517in}}%
\pgfpathlineto{\pgfqpoint{1.828323in}{1.104879in}}%
\pgfpathlineto{\pgfqpoint{1.829276in}{1.097526in}}%
\pgfpathlineto{\pgfqpoint{1.828211in}{1.090184in}}%
\pgfpathlineto{\pgfqpoint{1.824522in}{1.081197in}}%
\pgfpathlineto{\pgfqpoint{1.817779in}{1.071458in}}%
\pgfpathlineto{\pgfqpoint{1.808024in}{1.061557in}}%
\pgfpathlineto{\pgfqpoint{1.795389in}{1.051851in}}%
\pgfpathlineto{\pgfqpoint{1.778009in}{1.041561in}}%
\pgfpathlineto{\pgfqpoint{1.757805in}{1.032359in}}%
\pgfpathlineto{\pgfqpoint{1.734914in}{1.024405in}}%
\pgfpathlineto{\pgfqpoint{1.709448in}{1.017869in}}%
\pgfpathlineto{\pgfqpoint{1.681505in}{1.012934in}}%
\pgfpathlineto{\pgfqpoint{1.651180in}{1.009780in}}%
\pgfpathlineto{\pgfqpoint{1.618571in}{1.008597in}}%
\pgfpathlineto{\pgfqpoint{1.583781in}{1.009585in}}%
\pgfpathlineto{\pgfqpoint{1.546923in}{1.012964in}}%
\pgfpathlineto{\pgfqpoint{1.512439in}{1.018319in}}%
\pgfpathlineto{\pgfqpoint{1.473971in}{1.026656in}}%
\pgfpathlineto{\pgfqpoint{1.435925in}{1.037230in}}%
\pgfpathlineto{\pgfqpoint{1.398403in}{1.050020in}}%
\pgfpathlineto{\pgfqpoint{1.361516in}{1.065036in}}%
\pgfpathlineto{\pgfqpoint{1.327375in}{1.081285in}}%
\pgfpathlineto{\pgfqpoint{1.294030in}{1.099570in}}%
\pgfpathlineto{\pgfqpoint{1.261611in}{1.119917in}}%
\pgfpathlineto{\pgfqpoint{1.230256in}{1.142344in}}%
\pgfpathlineto{\pgfqpoint{1.201849in}{1.165347in}}%
\pgfpathlineto{\pgfqpoint{1.174634in}{1.190165in}}%
\pgfpathlineto{\pgfqpoint{1.148724in}{1.216750in}}%
\pgfpathlineto{\pgfqpoint{1.124223in}{1.245028in}}%
\pgfpathlineto{\pgfqpoint{1.101212in}{1.274903in}}%
\pgfpathlineto{\pgfqpoint{1.079755in}{1.306254in}}%
\pgfpathlineto{\pgfqpoint{1.058701in}{1.341035in}}%
\pgfpathlineto{\pgfqpoint{1.039457in}{1.377164in}}%
\pgfpathlineto{\pgfqpoint{1.022013in}{1.414470in}}%
\pgfpathlineto{\pgfqpoint{1.005465in}{1.455072in}}%
\pgfpathlineto{\pgfqpoint{0.990078in}{1.498961in}}%
\pgfpathlineto{\pgfqpoint{0.976747in}{1.543723in}}%
\pgfpathlineto{\pgfqpoint{0.964837in}{1.591600in}}%
\pgfpathlineto{\pgfqpoint{0.955010in}{1.640092in}}%
\pgfpathlineto{\pgfqpoint{0.946838in}{1.691517in}}%
\pgfpathlineto{\pgfqpoint{0.940787in}{1.743325in}}%
\pgfpathlineto{\pgfqpoint{0.936666in}{1.797879in}}%
\pgfpathlineto{\pgfqpoint{0.934784in}{1.852594in}}%
\pgfpathlineto{\pgfqpoint{0.935150in}{1.907350in}}%
\pgfpathlineto{\pgfqpoint{0.937811in}{1.962022in}}%
\pgfpathlineto{\pgfqpoint{0.942570in}{2.014006in}}%
\pgfpathlineto{\pgfqpoint{0.949220in}{2.063200in}}%
\pgfpathlineto{\pgfqpoint{0.958061in}{2.111939in}}%
\pgfpathlineto{\pgfqpoint{0.968622in}{2.157658in}}%
\pgfpathlineto{\pgfqpoint{0.980699in}{2.200273in}}%
\pgfpathlineto{\pgfqpoint{0.994069in}{2.239721in}}%
\pgfpathlineto{\pgfqpoint{1.009446in}{2.278199in}}%
\pgfpathlineto{\pgfqpoint{1.025822in}{2.313335in}}%
\pgfpathlineto{\pgfqpoint{1.044087in}{2.347230in}}%
\pgfpathlineto{\pgfqpoint{1.062931in}{2.377705in}}%
\pgfpathlineto{\pgfqpoint{1.083412in}{2.406763in}}%
\pgfpathlineto{\pgfqpoint{1.105468in}{2.434271in}}%
\pgfpathlineto{\pgfqpoint{1.129002in}{2.460126in}}%
\pgfpathlineto{\pgfqpoint{1.153891in}{2.484262in}}%
\pgfpathlineto{\pgfqpoint{1.181775in}{2.508088in}}%
\pgfpathlineto{\pgfqpoint{1.210860in}{2.529953in}}%
\pgfpathlineto{\pgfqpoint{1.242880in}{2.551112in}}%
\pgfpathlineto{\pgfqpoint{1.275858in}{2.570249in}}%
\pgfpathlineto{\pgfqpoint{1.311625in}{2.588471in}}%
\pgfpathlineto{\pgfqpoint{1.350137in}{2.605628in}}%
\pgfpathlineto{\pgfqpoint{1.391336in}{2.621617in}}%
\pgfpathlineto{\pgfqpoint{1.437254in}{2.637034in}}%
\pgfpathlineto{\pgfqpoint{1.487885in}{2.651589in}}%
\pgfpathlineto{\pgfqpoint{1.543200in}{2.665072in}}%
\pgfpathlineto{\pgfqpoint{1.603152in}{2.677365in}}%
\pgfpathlineto{\pgfqpoint{1.667708in}{2.688285in}}%
\pgfpathlineto{\pgfqpoint{1.738988in}{2.698003in}}%
\pgfpathlineto{\pgfqpoint{1.814797in}{2.706052in}}%
\pgfpathlineto{\pgfqpoint{1.895093in}{2.712360in}}%
\pgfpathlineto{\pgfqpoint{1.979840in}{2.716800in}}%
\pgfpathlineto{\pgfqpoint{2.066828in}{2.719147in}}%
\pgfpathlineto{\pgfqpoint{2.153839in}{2.719277in}}%
\pgfpathlineto{\pgfqpoint{2.238655in}{2.717159in}}%
\pgfpathlineto{\pgfqpoint{2.314708in}{2.713130in}}%
\pgfpathlineto{\pgfqpoint{2.384131in}{2.707332in}}%
\pgfpathlineto{\pgfqpoint{2.446878in}{2.699899in}}%
\pgfpathlineto{\pgfqpoint{2.500742in}{2.691351in}}%
\pgfpathlineto{\pgfqpoint{2.547844in}{2.681684in}}%
\pgfpathlineto{\pgfqpoint{2.588145in}{2.671209in}}%
\pgfpathlineto{\pgfqpoint{2.623709in}{2.659633in}}%
\pgfpathlineto{\pgfqpoint{2.652394in}{2.647943in}}%
\pgfpathlineto{\pgfqpoint{2.678291in}{2.634963in}}%
\pgfpathlineto{\pgfqpoint{2.699360in}{2.621999in}}%
\pgfpathlineto{\pgfqpoint{2.717530in}{2.608330in}}%
\pgfpathlineto{\pgfqpoint{2.732773in}{2.594286in}}%
\pgfpathlineto{\pgfqpoint{2.746631in}{2.578478in}}%
\pgfpathlineto{\pgfqpoint{2.758776in}{2.560924in}}%
\pgfpathlineto{\pgfqpoint{2.768957in}{2.541805in}}%
\pgfpathlineto{\pgfqpoint{2.777080in}{2.521435in}}%
\pgfpathlineto{\pgfqpoint{2.783230in}{2.500177in}}%
\pgfpathlineto{\pgfqpoint{2.788019in}{2.475904in}}%
\pgfpathlineto{\pgfqpoint{2.791442in}{2.446301in}}%
\pgfpathlineto{\pgfqpoint{2.793022in}{2.411507in}}%
\pgfpathlineto{\pgfqpoint{2.792528in}{2.369200in}}%
\pgfpathlineto{\pgfqpoint{2.789450in}{2.314556in}}%
\pgfpathlineto{\pgfqpoint{2.782674in}{2.237783in}}%
\pgfpathlineto{\pgfqpoint{2.763451in}{2.054901in}}%
\pgfpathlineto{\pgfqpoint{2.746498in}{1.881742in}}%
\pgfpathlineto{\pgfqpoint{2.733009in}{1.720683in}}%
\pgfpathlineto{\pgfqpoint{2.720343in}{1.542049in}}%
\pgfpathlineto{\pgfqpoint{2.706553in}{1.313587in}}%
\pgfpathlineto{\pgfqpoint{2.684335in}{0.941078in}}%
\pgfpathlineto{\pgfqpoint{2.675406in}{0.832033in}}%
\pgfpathlineto{\pgfqpoint{2.667013in}{0.755472in}}%
\pgfpathlineto{\pgfqpoint{2.658739in}{0.699014in}}%
\pgfpathlineto{\pgfqpoint{2.650269in}{0.655276in}}%
\pgfpathlineto{\pgfqpoint{2.641175in}{0.619425in}}%
\pgfpathlineto{\pgfqpoint{2.632006in}{0.591467in}}%
\pgfpathlineto{\pgfqpoint{2.621628in}{0.566808in}}%
\pgfpathlineto{\pgfqpoint{2.610155in}{0.545675in}}%
\pgfpathlineto{\pgfqpoint{2.597991in}{0.528136in}}%
\pgfpathlineto{\pgfqpoint{2.585705in}{0.514047in}}%
\pgfpathlineto{\pgfqpoint{2.572157in}{0.501564in}}%
\pgfpathlineto{\pgfqpoint{2.555673in}{0.489497in}}%
\pgfpathlineto{\pgfqpoint{2.538207in}{0.479394in}}%
\pgfpathlineto{\pgfqpoint{2.518013in}{0.470159in}}%
\pgfpathlineto{\pgfqpoint{2.493089in}{0.461302in}}%
\pgfpathlineto{\pgfqpoint{2.463458in}{0.453277in}}%
\pgfpathlineto{\pgfqpoint{2.427047in}{0.445904in}}%
\pgfpathlineto{\pgfqpoint{2.381746in}{0.439198in}}%
\pgfpathlineto{\pgfqpoint{2.325429in}{0.433256in}}%
\pgfpathlineto{\pgfqpoint{2.251618in}{0.427886in}}%
\pgfpathlineto{\pgfqpoint{2.153816in}{0.423188in}}%
\pgfpathlineto{\pgfqpoint{2.018992in}{0.419174in}}%
\pgfpathlineto{\pgfqpoint{1.842810in}{0.416214in}}%
\pgfpathlineto{\pgfqpoint{1.579602in}{0.414019in}}%
\pgfpathlineto{\pgfqpoint{1.253303in}{0.413604in}}%
\pgfpathlineto{\pgfqpoint{0.968340in}{0.415298in}}%
\pgfpathlineto{\pgfqpoint{0.789984in}{0.418399in}}%
\pgfpathlineto{\pgfqpoint{0.683450in}{0.422318in}}%
\pgfpathlineto{\pgfqpoint{0.618307in}{0.426727in}}%
\pgfpathlineto{\pgfqpoint{0.575021in}{0.431679in}}%
\pgfpathlineto{\pgfqpoint{0.544957in}{0.437200in}}%
\pgfpathlineto{\pgfqpoint{0.523847in}{0.443169in}}%
\pgfpathlineto{\pgfqpoint{0.507545in}{0.450096in}}%
\pgfpathlineto{\pgfqpoint{0.496090in}{0.457226in}}%
\pgfpathlineto{\pgfqpoint{0.487473in}{0.464799in}}%
\pgfpathlineto{\pgfqpoint{0.480090in}{0.473917in}}%
\pgfpathlineto{\pgfqpoint{0.473119in}{0.486518in}}%
\pgfpathlineto{\pgfqpoint{0.467330in}{0.502615in}}%
\pgfpathlineto{\pgfqpoint{0.462905in}{0.521863in}}%
\pgfpathlineto{\pgfqpoint{0.459073in}{0.548883in}}%
\pgfpathlineto{\pgfqpoint{0.455875in}{0.588537in}}%
\pgfpathlineto{\pgfqpoint{0.453313in}{0.650695in}}%
\pgfpathlineto{\pgfqpoint{0.451347in}{0.760195in}}%
\pgfpathlineto{\pgfqpoint{0.450001in}{0.981728in}}%
\pgfpathlineto{\pgfqpoint{0.449245in}{1.564201in}}%
\pgfpathlineto{\pgfqpoint{0.449832in}{2.771464in}}%
\pgfpathlineto{\pgfqpoint{0.451398in}{2.866027in}}%
\pgfpathlineto{\pgfqpoint{0.453071in}{2.880819in}}%
\pgfpathlineto{\pgfqpoint{0.454689in}{2.885418in}}%
\pgfpathlineto{\pgfqpoint{0.457851in}{2.888716in}}%
\pgfpathlineto{\pgfqpoint{0.464166in}{2.890472in}}%
\pgfpathlineto{\pgfqpoint{0.481543in}{2.891415in}}%
\pgfpathlineto{\pgfqpoint{0.562029in}{2.891728in}}%
\pgfpathlineto{\pgfqpoint{2.619892in}{2.891760in}}%
\pgfpathlineto{\pgfqpoint{4.790839in}{2.890704in}}%
\pgfpathlineto{\pgfqpoint{4.794916in}{2.889086in}}%
\pgfpathlineto{\pgfqpoint{4.796718in}{2.884686in}}%
\pgfpathlineto{\pgfqpoint{4.797779in}{2.869814in}}%
\pgfpathlineto{\pgfqpoint{4.798061in}{2.857372in}}%
\pgfpathlineto{\pgfqpoint{4.798061in}{2.857372in}}%
\pgfusepath{stroke}%
\end{pgfscope}%
\begin{pgfscope}%
\pgfpathrectangle{\pgfqpoint{0.448634in}{0.402556in}}{\pgfqpoint{4.350661in}{2.489204in}} %
\pgfusepath{clip}%
\pgfsetrectcap%
\pgfsetroundjoin%
\pgfsetlinewidth{1.003750pt}%
\definecolor{currentstroke}{rgb}{0.890196,0.466667,0.760784}%
\pgfsetstrokecolor{currentstroke}%
\pgfsetdash{}{0pt}%
\pgfpathmoveto{\pgfqpoint{3.429110in}{0.402610in}}%
\pgfpathlineto{\pgfqpoint{2.806970in}{0.403758in}}%
\pgfpathlineto{\pgfqpoint{2.767863in}{0.405791in}}%
\pgfpathlineto{\pgfqpoint{2.752852in}{0.408603in}}%
\pgfpathlineto{\pgfqpoint{2.744772in}{0.412222in}}%
\pgfpathlineto{\pgfqpoint{2.739639in}{0.416790in}}%
\pgfpathlineto{\pgfqpoint{2.735901in}{0.422883in}}%
\pgfpathlineto{\pgfqpoint{2.732781in}{0.432157in}}%
\pgfpathlineto{\pgfqpoint{2.729905in}{0.449253in}}%
\pgfpathlineto{\pgfqpoint{2.727750in}{0.479015in}}%
\pgfpathlineto{\pgfqpoint{2.726190in}{0.536236in}}%
\pgfpathlineto{\pgfqpoint{2.725765in}{0.643269in}}%
\pgfpathlineto{\pgfqpoint{2.727422in}{0.807544in}}%
\pgfpathlineto{\pgfqpoint{2.731597in}{1.001641in}}%
\pgfpathlineto{\pgfqpoint{2.738321in}{1.205606in}}%
\pgfpathlineto{\pgfqpoint{2.746291in}{1.372132in}}%
\pgfpathlineto{\pgfqpoint{2.755719in}{1.523588in}}%
\pgfpathlineto{\pgfqpoint{2.766398in}{1.659945in}}%
\pgfpathlineto{\pgfqpoint{2.778048in}{1.781183in}}%
\pgfpathlineto{\pgfqpoint{2.793544in}{1.916911in}}%
\pgfpathlineto{\pgfqpoint{2.811717in}{2.047142in}}%
\pgfpathlineto{\pgfqpoint{2.828066in}{2.142398in}}%
\pgfpathlineto{\pgfqpoint{2.846602in}{2.237129in}}%
\pgfpathlineto{\pgfqpoint{2.871783in}{2.353082in}}%
\pgfpathlineto{\pgfqpoint{2.892225in}{2.449858in}}%
\pgfpathlineto{\pgfqpoint{2.899433in}{2.493891in}}%
\pgfpathlineto{\pgfqpoint{2.902737in}{2.545990in}}%
\pgfpathlineto{\pgfqpoint{2.901652in}{2.568347in}}%
\pgfpathlineto{\pgfqpoint{2.898626in}{2.587943in}}%
\pgfpathlineto{\pgfqpoint{2.894044in}{2.604547in}}%
\pgfpathlineto{\pgfqpoint{2.887500in}{2.620263in}}%
\pgfpathlineto{\pgfqpoint{2.879054in}{2.634745in}}%
\pgfpathlineto{\pgfqpoint{2.868962in}{2.647776in}}%
\pgfpathlineto{\pgfqpoint{2.855855in}{2.660856in}}%
\pgfpathlineto{\pgfqpoint{2.836390in}{2.676712in}}%
\pgfpathlineto{\pgfqpoint{2.817032in}{2.688049in}}%
\pgfpathlineto{\pgfqpoint{2.792885in}{2.699376in}}%
\pgfpathlineto{\pgfqpoint{2.784448in}{2.701678in}}%
\pgfpathlineto{\pgfqpoint{2.732181in}{2.718789in}}%
\pgfpathlineto{\pgfqpoint{2.693898in}{2.728177in}}%
\pgfpathlineto{\pgfqpoint{2.646719in}{2.737341in}}%
\pgfpathlineto{\pgfqpoint{2.590646in}{2.745783in}}%
\pgfpathlineto{\pgfqpoint{2.525710in}{2.753185in}}%
\pgfpathlineto{\pgfqpoint{2.449775in}{2.759490in}}%
\pgfpathlineto{\pgfqpoint{2.365042in}{2.764274in}}%
\pgfpathlineto{\pgfqpoint{2.273717in}{2.767235in}}%
\pgfpathlineto{\pgfqpoint{2.169309in}{2.768540in}}%
\pgfpathlineto{\pgfqpoint{2.062722in}{2.767688in}}%
\pgfpathlineto{\pgfqpoint{1.953990in}{2.764591in}}%
\pgfpathlineto{\pgfqpoint{1.849676in}{2.759357in}}%
\pgfpathlineto{\pgfqpoint{1.754158in}{2.752379in}}%
\pgfpathlineto{\pgfqpoint{1.667464in}{2.743881in}}%
\pgfpathlineto{\pgfqpoint{1.589621in}{2.734115in}}%
\pgfpathlineto{\pgfqpoint{1.503391in}{2.720933in}}%
\pgfpathlineto{\pgfqpoint{1.419790in}{2.704763in}}%
\pgfpathlineto{\pgfqpoint{1.364539in}{2.690946in}}%
\pgfpathlineto{\pgfqpoint{1.313988in}{2.676027in}}%
\pgfpathlineto{\pgfqpoint{1.268177in}{2.660199in}}%
\pgfpathlineto{\pgfqpoint{1.225080in}{2.642878in}}%
\pgfpathlineto{\pgfqpoint{1.186803in}{2.625049in}}%
\pgfpathlineto{\pgfqpoint{1.151338in}{2.606070in}}%
\pgfpathlineto{\pgfqpoint{1.118738in}{2.586104in}}%
\pgfpathlineto{\pgfqpoint{1.074025in}{2.555463in}}%
\pgfpathlineto{\pgfqpoint{1.046011in}{2.531840in}}%
\pgfpathlineto{\pgfqpoint{1.020991in}{2.507880in}}%
\pgfpathlineto{\pgfqpoint{0.997308in}{2.482204in}}%
\pgfpathlineto{\pgfqpoint{0.975067in}{2.454891in}}%
\pgfpathlineto{\pgfqpoint{0.954342in}{2.426060in}}%
\pgfpathlineto{\pgfqpoint{0.933942in}{2.393799in}}%
\pgfpathlineto{\pgfqpoint{0.917789in}{2.364267in}}%
\pgfpathlineto{\pgfqpoint{0.899639in}{2.327405in}}%
\pgfpathlineto{\pgfqpoint{0.882437in}{2.287162in}}%
\pgfpathlineto{\pgfqpoint{0.866375in}{2.243591in}}%
\pgfpathlineto{\pgfqpoint{0.851595in}{2.196773in}}%
\pgfpathlineto{\pgfqpoint{0.838898in}{2.149163in}}%
\pgfpathlineto{\pgfqpoint{0.826838in}{2.096172in}}%
\pgfpathlineto{\pgfqpoint{0.815796in}{2.037787in}}%
\pgfpathlineto{\pgfqpoint{0.806465in}{1.976486in}}%
\pgfpathlineto{\pgfqpoint{0.798015in}{1.907465in}}%
\pgfpathlineto{\pgfqpoint{0.790855in}{1.830739in}}%
\pgfpathlineto{\pgfqpoint{0.784814in}{1.741398in}}%
\pgfpathlineto{\pgfqpoint{0.777577in}{1.589783in}}%
\pgfpathlineto{\pgfqpoint{0.777577in}{1.589783in}}%
\pgfusepath{stroke}%
\end{pgfscope}%
\begin{pgfscope}%
\pgfpathrectangle{\pgfqpoint{0.448634in}{0.402556in}}{\pgfqpoint{4.350661in}{2.489204in}} %
\pgfusepath{clip}%
\pgfsetrectcap%
\pgfsetroundjoin%
\pgfsetlinewidth{1.003750pt}%
\definecolor{currentstroke}{rgb}{0.890196,0.466667,0.760784}%
\pgfsetstrokecolor{currentstroke}%
\pgfsetdash{}{0pt}%
\pgfpathmoveto{\pgfqpoint{4.798853in}{2.849880in}}%
\pgfpathlineto{\pgfqpoint{4.797564in}{2.889610in}}%
\pgfpathlineto{\pgfqpoint{4.796215in}{2.891483in}}%
\pgfpathlineto{\pgfqpoint{4.787551in}{2.891760in}}%
\pgfpathlineto{\pgfqpoint{0.452128in}{2.891660in}}%
\pgfpathlineto{\pgfqpoint{0.450530in}{2.890082in}}%
\pgfpathlineto{\pgfqpoint{0.449454in}{2.882763in}}%
\pgfpathlineto{\pgfqpoint{0.448970in}{2.845432in}}%
\pgfpathlineto{\pgfqpoint{0.448743in}{2.494454in}}%
\pgfpathlineto{\pgfqpoint{0.449607in}{0.617596in}}%
\pgfpathlineto{\pgfqpoint{0.451436in}{0.510586in}}%
\pgfpathlineto{\pgfqpoint{0.453999in}{0.473375in}}%
\pgfpathlineto{\pgfqpoint{0.457415in}{0.453870in}}%
\pgfpathlineto{\pgfqpoint{0.461553in}{0.442388in}}%
\pgfpathlineto{\pgfqpoint{0.466755in}{0.434443in}}%
\pgfpathlineto{\pgfqpoint{0.473613in}{0.428359in}}%
\pgfpathlineto{\pgfqpoint{0.483511in}{0.423254in}}%
\pgfpathlineto{\pgfqpoint{0.491872in}{0.420511in}}%
\pgfpathlineto{\pgfqpoint{0.491872in}{0.420511in}}%
\pgfusepath{stroke}%
\end{pgfscope}%
\begin{pgfscope}%
\pgfpathrectangle{\pgfqpoint{0.448634in}{0.402556in}}{\pgfqpoint{4.350661in}{2.489204in}} %
\pgfusepath{clip}%
\pgfsetrectcap%
\pgfsetroundjoin%
\pgfsetlinewidth{1.003750pt}%
\definecolor{currentstroke}{rgb}{0.890196,0.466667,0.760784}%
\pgfsetstrokecolor{currentstroke}%
\pgfsetdash{}{0pt}%
\pgfpathmoveto{\pgfqpoint{3.428521in}{0.402589in}}%
\pgfpathlineto{\pgfqpoint{2.786804in}{0.403708in}}%
\pgfpathlineto{\pgfqpoint{2.756410in}{0.405700in}}%
\pgfpathlineto{\pgfqpoint{2.745767in}{0.408176in}}%
\pgfpathlineto{\pgfqpoint{2.739890in}{0.411363in}}%
\pgfpathlineto{\pgfqpoint{2.735272in}{0.416579in}}%
\pgfpathlineto{\pgfqpoint{2.732328in}{0.423218in}}%
\pgfpathlineto{\pgfqpoint{2.729631in}{0.435260in}}%
\pgfpathlineto{\pgfqpoint{2.727495in}{0.457519in}}%
\pgfpathlineto{\pgfqpoint{2.725802in}{0.504770in}}%
\pgfpathlineto{\pgfqpoint{2.725074in}{0.596865in}}%
\pgfpathlineto{\pgfqpoint{2.726205in}{0.753678in}}%
\pgfpathlineto{\pgfqpoint{2.729917in}{0.950278in}}%
\pgfpathlineto{\pgfqpoint{2.736164in}{1.159245in}}%
\pgfpathlineto{\pgfqpoint{2.743793in}{1.330776in}}%
\pgfpathlineto{\pgfqpoint{2.752933in}{1.487245in}}%
\pgfpathlineto{\pgfqpoint{2.763829in}{1.633575in}}%
\pgfpathlineto{\pgfqpoint{2.775491in}{1.759819in}}%
\pgfpathlineto{\pgfqpoint{2.790698in}{1.898098in}}%
\pgfpathlineto{\pgfqpoint{2.808303in}{2.028435in}}%
\pgfpathlineto{\pgfqpoint{2.824155in}{2.123802in}}%
\pgfpathlineto{\pgfqpoint{2.842191in}{2.218659in}}%
\pgfpathlineto{\pgfqpoint{2.865326in}{2.327498in}}%
\pgfpathlineto{\pgfqpoint{2.891361in}{2.450894in}}%
\pgfpathlineto{\pgfqpoint{2.898335in}{2.494974in}}%
\pgfpathlineto{\pgfqpoint{2.901462in}{2.547085in}}%
\pgfpathlineto{\pgfqpoint{2.900147in}{2.569425in}}%
\pgfpathlineto{\pgfqpoint{2.896888in}{2.588972in}}%
\pgfpathlineto{\pgfqpoint{2.892084in}{2.605493in}}%
\pgfpathlineto{\pgfqpoint{2.885326in}{2.621091in}}%
\pgfpathlineto{\pgfqpoint{2.876695in}{2.635428in}}%
\pgfpathlineto{\pgfqpoint{2.866456in}{2.648309in}}%
\pgfpathlineto{\pgfqpoint{2.853229in}{2.661231in}}%
\pgfpathlineto{\pgfqpoint{2.835321in}{2.675327in}}%
\pgfpathlineto{\pgfqpoint{2.816011in}{2.686770in}}%
\pgfpathlineto{\pgfqpoint{2.791902in}{2.698201in}}%
\pgfpathlineto{\pgfqpoint{2.766980in}{2.707022in}}%
\pgfpathlineto{\pgfqpoint{2.735526in}{2.716941in}}%
\pgfpathlineto{\pgfqpoint{2.697296in}{2.726605in}}%
\pgfpathlineto{\pgfqpoint{2.652303in}{2.735631in}}%
\pgfpathlineto{\pgfqpoint{2.598426in}{2.744074in}}%
\pgfpathlineto{\pgfqpoint{2.535685in}{2.751578in}}%
\pgfpathlineto{\pgfqpoint{2.461943in}{2.758063in}}%
\pgfpathlineto{\pgfqpoint{2.379399in}{2.763084in}}%
\pgfpathlineto{\pgfqpoint{2.290258in}{2.766342in}}%
\pgfpathlineto{\pgfqpoint{2.188029in}{2.768000in}}%
\pgfpathlineto{\pgfqpoint{2.081440in}{2.767547in}}%
\pgfpathlineto{\pgfqpoint{1.972701in}{2.764832in}}%
\pgfpathlineto{\pgfqpoint{1.868372in}{2.759997in}}%
\pgfpathlineto{\pgfqpoint{1.770662in}{2.753257in}}%
\pgfpathlineto{\pgfqpoint{1.683939in}{2.745148in}}%
\pgfpathlineto{\pgfqpoint{1.603895in}{2.735516in}}%
\pgfpathlineto{\pgfqpoint{1.517612in}{2.722747in}}%
\pgfpathlineto{\pgfqpoint{1.436060in}{2.707519in}}%
\pgfpathlineto{\pgfqpoint{1.378580in}{2.693722in}}%
\pgfpathlineto{\pgfqpoint{1.327913in}{2.679332in}}%
\pgfpathlineto{\pgfqpoint{1.281951in}{2.664087in}}%
\pgfpathlineto{\pgfqpoint{1.238669in}{2.647378in}}%
\pgfpathlineto{\pgfqpoint{1.202168in}{2.631174in}}%
\pgfpathlineto{\pgfqpoint{1.166386in}{2.612990in}}%
\pgfpathlineto{\pgfqpoint{1.133415in}{2.593836in}}%
\pgfpathlineto{\pgfqpoint{1.101429in}{2.572611in}}%
\pgfpathlineto{\pgfqpoint{1.093839in}{2.567806in}}%
\pgfpathlineto{\pgfqpoint{1.087948in}{2.564728in}}%
\pgfpathlineto{\pgfqpoint{1.059390in}{2.541973in}}%
\pgfpathlineto{\pgfqpoint{1.033792in}{2.518826in}}%
\pgfpathlineto{\pgfqpoint{1.009475in}{2.493939in}}%
\pgfpathlineto{\pgfqpoint{0.986554in}{2.467372in}}%
\pgfpathlineto{\pgfqpoint{0.965120in}{2.439227in}}%
\pgfpathlineto{\pgfqpoint{0.943955in}{2.407619in}}%
\pgfpathlineto{\pgfqpoint{0.927093in}{2.378623in}}%
\pgfpathlineto{\pgfqpoint{0.919313in}{2.363653in}}%
\pgfpathlineto{\pgfqpoint{0.901184in}{2.326777in}}%
\pgfpathlineto{\pgfqpoint{0.884009in}{2.286518in}}%
\pgfpathlineto{\pgfqpoint{0.867983in}{2.242930in}}%
\pgfpathlineto{\pgfqpoint{0.853248in}{2.196094in}}%
\pgfpathlineto{\pgfqpoint{0.841316in}{2.150824in}}%
\pgfpathlineto{\pgfqpoint{0.829255in}{2.097834in}}%
\pgfpathlineto{\pgfqpoint{0.818230in}{2.039444in}}%
\pgfpathlineto{\pgfqpoint{0.809716in}{1.983037in}}%
\pgfpathlineto{\pgfqpoint{0.793478in}{1.827354in}}%
\pgfpathlineto{\pgfqpoint{0.788218in}{1.742938in}}%
\pgfpathlineto{\pgfqpoint{0.785097in}{1.665859in}}%
\pgfpathlineto{\pgfqpoint{0.776934in}{1.471943in}}%
\pgfpathlineto{\pgfqpoint{0.772138in}{1.422482in}}%
\pgfpathlineto{\pgfqpoint{0.766547in}{1.390774in}}%
\pgfpathlineto{\pgfqpoint{0.760689in}{1.369410in}}%
\pgfpathlineto{\pgfqpoint{0.754278in}{1.353622in}}%
\pgfpathlineto{\pgfqpoint{0.746838in}{1.341382in}}%
\pgfpathlineto{\pgfqpoint{0.740537in}{1.334539in}}%
\pgfpathlineto{\pgfqpoint{0.733071in}{1.329476in}}%
\pgfpathlineto{\pgfqpoint{0.724696in}{1.326898in}}%
\pgfpathlineto{\pgfqpoint{0.716021in}{1.327032in}}%
\pgfpathlineto{\pgfqpoint{0.707611in}{1.329515in}}%
\pgfpathlineto{\pgfqpoint{0.697856in}{1.334974in}}%
\pgfpathlineto{\pgfqpoint{0.687354in}{1.343814in}}%
\pgfpathlineto{\pgfqpoint{0.676529in}{1.356050in}}%
\pgfpathlineto{\pgfqpoint{0.665677in}{1.371606in}}%
\pgfpathlineto{\pgfqpoint{0.653891in}{1.392516in}}%
\pgfpathlineto{\pgfqpoint{0.641728in}{1.418937in}}%
\pgfpathlineto{\pgfqpoint{0.629660in}{1.450925in}}%
\pgfpathlineto{\pgfqpoint{0.617994in}{1.488442in}}%
\pgfpathlineto{\pgfqpoint{0.610110in}{1.519497in}}%
\pgfpathlineto{\pgfqpoint{0.607638in}{1.531600in}}%
\pgfpathlineto{\pgfqpoint{0.596915in}{1.582407in}}%
\pgfpathlineto{\pgfqpoint{0.586926in}{1.641040in}}%
\pgfpathlineto{\pgfqpoint{0.578256in}{1.704991in}}%
\pgfpathlineto{\pgfqpoint{0.571193in}{1.771699in}}%
\pgfpathlineto{\pgfqpoint{0.558587in}{1.945325in}}%
\pgfpathlineto{\pgfqpoint{0.554413in}{2.049760in}}%
\pgfpathlineto{\pgfqpoint{0.551802in}{2.176649in}}%
\pgfpathlineto{\pgfqpoint{0.552383in}{2.281191in}}%
\pgfpathlineto{\pgfqpoint{0.555151in}{2.380705in}}%
\pgfpathlineto{\pgfqpoint{0.559598in}{2.460191in}}%
\pgfpathlineto{\pgfqpoint{0.565434in}{2.529564in}}%
\pgfpathlineto{\pgfqpoint{0.572586in}{2.588737in}}%
\pgfpathlineto{\pgfqpoint{0.580762in}{2.637628in}}%
\pgfpathlineto{\pgfqpoint{0.589835in}{2.678645in}}%
\pgfpathlineto{\pgfqpoint{0.599665in}{2.711618in}}%
\pgfpathlineto{\pgfqpoint{0.610221in}{2.738925in}}%
\pgfpathlineto{\pgfqpoint{0.620905in}{2.760596in}}%
\pgfpathlineto{\pgfqpoint{0.632225in}{2.778862in}}%
\pgfpathlineto{\pgfqpoint{0.645262in}{2.795558in}}%
\pgfpathlineto{\pgfqpoint{0.658251in}{2.808796in}}%
\pgfpathlineto{\pgfqpoint{0.674241in}{2.821701in}}%
\pgfpathlineto{\pgfqpoint{0.691380in}{2.832508in}}%
\pgfpathlineto{\pgfqpoint{0.709355in}{2.841364in}}%
\pgfpathlineto{\pgfqpoint{0.738346in}{2.852000in}}%
\pgfpathlineto{\pgfqpoint{0.765877in}{2.859369in}}%
\pgfpathlineto{\pgfqpoint{0.800185in}{2.866045in}}%
\pgfpathlineto{\pgfqpoint{0.841220in}{2.871669in}}%
\pgfpathlineto{\pgfqpoint{0.895424in}{2.876682in}}%
\pgfpathlineto{\pgfqpoint{0.967114in}{2.880885in}}%
\pgfpathlineto{\pgfqpoint{1.067132in}{2.884332in}}%
\pgfpathlineto{\pgfqpoint{1.212860in}{2.886954in}}%
\pgfpathlineto{\pgfqpoint{1.456490in}{2.888990in}}%
\pgfpathlineto{\pgfqpoint{1.926360in}{2.890332in}}%
\pgfpathlineto{\pgfqpoint{3.064058in}{2.890804in}}%
\pgfpathlineto{\pgfqpoint{3.986397in}{2.889455in}}%
\pgfpathlineto{\pgfqpoint{4.275707in}{2.887083in}}%
\pgfpathlineto{\pgfqpoint{4.412722in}{2.883864in}}%
\pgfpathlineto{\pgfqpoint{4.490959in}{2.879995in}}%
\pgfpathlineto{\pgfqpoint{4.545175in}{2.875181in}}%
\pgfpathlineto{\pgfqpoint{4.581851in}{2.869809in}}%
\pgfpathlineto{\pgfqpoint{4.609607in}{2.863649in}}%
\pgfpathlineto{\pgfqpoint{4.630557in}{2.856975in}}%
\pgfpathlineto{\pgfqpoint{4.648802in}{2.848883in}}%
\pgfpathlineto{\pgfqpoint{4.664152in}{2.839527in}}%
\pgfpathlineto{\pgfqpoint{4.676539in}{2.829415in}}%
\pgfpathlineto{\pgfqpoint{4.687629in}{2.817495in}}%
\pgfpathlineto{\pgfqpoint{4.697268in}{2.804021in}}%
\pgfpathlineto{\pgfqpoint{4.706463in}{2.787128in}}%
\pgfpathlineto{\pgfqpoint{4.714816in}{2.766879in}}%
\pgfpathlineto{\pgfqpoint{4.722796in}{2.741076in}}%
\pgfpathlineto{\pgfqpoint{4.729966in}{2.709781in}}%
\pgfpathlineto{\pgfqpoint{4.736547in}{2.670678in}}%
\pgfpathlineto{\pgfqpoint{4.742791in}{2.618901in}}%
\pgfpathlineto{\pgfqpoint{4.748211in}{2.554482in}}%
\pgfpathlineto{\pgfqpoint{4.752821in}{2.472510in}}%
\pgfpathlineto{\pgfqpoint{4.756801in}{2.363082in}}%
\pgfpathlineto{\pgfqpoint{4.759592in}{2.223725in}}%
\pgfpathlineto{\pgfqpoint{4.760835in}{2.049488in}}%
\pgfpathlineto{\pgfqpoint{4.759978in}{1.857823in}}%
\pgfpathlineto{\pgfqpoint{4.756988in}{1.673655in}}%
\pgfpathlineto{\pgfqpoint{4.752121in}{1.509464in}}%
\pgfpathlineto{\pgfqpoint{4.745944in}{1.377728in}}%
\pgfpathlineto{\pgfqpoint{4.738762in}{1.271012in}}%
\pgfpathlineto{\pgfqpoint{4.730612in}{1.181890in}}%
\pgfpathlineto{\pgfqpoint{4.721620in}{1.107931in}}%
\pgfpathlineto{\pgfqpoint{4.712052in}{1.046676in}}%
\pgfpathlineto{\pgfqpoint{4.701545in}{0.993254in}}%
\pgfpathlineto{\pgfqpoint{4.696547in}{0.971594in}}%
\pgfpathlineto{\pgfqpoint{4.696547in}{0.971594in}}%
\pgfusepath{stroke}%
\end{pgfscope}%
\begin{pgfscope}%
\pgfpathrectangle{\pgfqpoint{0.448634in}{0.402556in}}{\pgfqpoint{4.350661in}{2.489204in}} %
\pgfusepath{clip}%
\pgfsetrectcap%
\pgfsetroundjoin%
\pgfsetlinewidth{1.003750pt}%
\definecolor{currentstroke}{rgb}{0.890196,0.466667,0.760784}%
\pgfsetstrokecolor{currentstroke}%
\pgfsetdash{}{0pt}%
\pgfpathmoveto{\pgfqpoint{2.583730in}{2.736619in}}%
\pgfpathlineto{\pgfqpoint{2.637557in}{2.727767in}}%
\pgfpathlineto{\pgfqpoint{2.682477in}{2.718276in}}%
\pgfpathlineto{\pgfqpoint{2.720602in}{2.708087in}}%
\pgfpathlineto{\pgfqpoint{2.751917in}{2.697614in}}%
\pgfpathlineto{\pgfqpoint{2.778492in}{2.686569in}}%
\pgfpathlineto{\pgfqpoint{2.800306in}{2.675331in}}%
\pgfpathlineto{\pgfqpoint{2.819280in}{2.663177in}}%
\pgfpathlineto{\pgfqpoint{2.835278in}{2.650283in}}%
\pgfpathlineto{\pgfqpoint{2.848234in}{2.637006in}}%
\pgfpathlineto{\pgfqpoint{2.858239in}{2.623885in}}%
\pgfpathlineto{\pgfqpoint{2.866691in}{2.609407in}}%
\pgfpathlineto{\pgfqpoint{2.873382in}{2.593769in}}%
\pgfpathlineto{\pgfqpoint{2.878819in}{2.574869in}}%
\pgfpathlineto{\pgfqpoint{2.882088in}{2.555322in}}%
\pgfpathlineto{\pgfqpoint{2.883621in}{2.532997in}}%
\pgfpathlineto{\pgfqpoint{2.883172in}{2.505630in}}%
\pgfpathlineto{\pgfqpoint{2.880385in}{2.473435in}}%
\pgfpathlineto{\pgfqpoint{2.874100in}{2.429216in}}%
\pgfpathlineto{\pgfqpoint{2.862141in}{2.363418in}}%
\pgfpathlineto{\pgfqpoint{2.818501in}{2.132287in}}%
\pgfpathlineto{\pgfqpoint{2.802563in}{2.031875in}}%
\pgfpathlineto{\pgfqpoint{2.788404in}{1.928594in}}%
\pgfpathlineto{\pgfqpoint{2.775445in}{1.817568in}}%
\pgfpathlineto{\pgfqpoint{2.763583in}{1.696357in}}%
\pgfpathlineto{\pgfqpoint{2.752777in}{1.562512in}}%
\pgfpathlineto{\pgfqpoint{2.743042in}{1.413577in}}%
\pgfpathlineto{\pgfqpoint{2.734425in}{1.247094in}}%
\pgfpathlineto{\pgfqpoint{2.726792in}{1.055626in}}%
\pgfpathlineto{\pgfqpoint{2.719550in}{0.814316in}}%
\pgfpathlineto{\pgfqpoint{2.711874in}{0.568047in}}%
\pgfpathlineto{\pgfqpoint{2.707945in}{0.508481in}}%
\pgfpathlineto{\pgfqpoint{2.703585in}{0.474003in}}%
\pgfpathlineto{\pgfqpoint{2.699221in}{0.454740in}}%
\pgfpathlineto{\pgfqpoint{2.694016in}{0.441067in}}%
\pgfpathlineto{\pgfqpoint{2.687628in}{0.431033in}}%
\pgfpathlineto{\pgfqpoint{2.681014in}{0.424592in}}%
\pgfpathlineto{\pgfqpoint{2.671408in}{0.418807in}}%
\pgfpathlineto{\pgfqpoint{2.658948in}{0.414408in}}%
\pgfpathlineto{\pgfqpoint{2.641813in}{0.410981in}}%
\pgfpathlineto{\pgfqpoint{2.615823in}{0.408237in}}%
\pgfpathlineto{\pgfqpoint{2.570185in}{0.406015in}}%
\pgfpathlineto{\pgfqpoint{2.483184in}{0.404431in}}%
\pgfpathlineto{\pgfqpoint{2.272179in}{0.403441in}}%
\pgfpathlineto{\pgfqpoint{1.508639in}{0.402944in}}%
\pgfpathlineto{\pgfqpoint{0.536268in}{0.403937in}}%
\pgfpathlineto{\pgfqpoint{0.471030in}{0.405664in}}%
\pgfpathlineto{\pgfqpoint{0.458078in}{0.407380in}}%
\pgfpathlineto{\pgfqpoint{0.453999in}{0.409061in}}%
\pgfpathlineto{\pgfqpoint{0.451036in}{0.412561in}}%
\pgfpathlineto{\pgfqpoint{0.449582in}{0.419808in}}%
\pgfpathlineto{\pgfqpoint{0.448904in}{0.442191in}}%
\pgfpathlineto{\pgfqpoint{0.448660in}{0.586564in}}%
\pgfpathlineto{\pgfqpoint{0.448682in}{2.891567in}}%
\pgfpathlineto{\pgfqpoint{0.448682in}{2.891567in}}%
\pgfusepath{stroke}%
\end{pgfscope}%
\begin{pgfscope}%
\pgfpathrectangle{\pgfqpoint{0.448634in}{0.402556in}}{\pgfqpoint{4.350661in}{2.489204in}} %
\pgfusepath{clip}%
\pgfsetrectcap%
\pgfsetroundjoin%
\pgfsetlinewidth{1.003750pt}%
\definecolor{currentstroke}{rgb}{0.498039,0.498039,0.498039}%
\pgfsetstrokecolor{currentstroke}%
\pgfsetdash{}{0pt}%
\pgfpathmoveto{\pgfqpoint{0.448634in}{2.896245in}}%
\pgfpathlineto{\pgfqpoint{0.448593in}{0.407043in}}%
\pgfpathlineto{\pgfqpoint{0.448593in}{0.407043in}}%
\pgfusepath{stroke}%
\end{pgfscope}%
\begin{pgfscope}%
\pgfpathrectangle{\pgfqpoint{0.448634in}{0.402556in}}{\pgfqpoint{4.350661in}{2.489204in}} %
\pgfusepath{clip}%
\pgfsetrectcap%
\pgfsetroundjoin%
\pgfsetlinewidth{1.003750pt}%
\definecolor{currentstroke}{rgb}{0.498039,0.498039,0.498039}%
\pgfsetstrokecolor{currentstroke}%
\pgfsetdash{}{0pt}%
\pgfpathmoveto{\pgfqpoint{2.759561in}{2.114283in}}%
\pgfpathlineto{\pgfqpoint{2.736455in}{1.861767in}}%
\pgfpathlineto{\pgfqpoint{2.722904in}{1.685716in}}%
\pgfpathlineto{\pgfqpoint{2.709186in}{1.477214in}}%
\pgfpathlineto{\pgfqpoint{2.674992in}{0.938489in}}%
\pgfpathlineto{\pgfqpoint{2.665742in}{0.839489in}}%
\pgfpathlineto{\pgfqpoint{2.656989in}{0.768004in}}%
\pgfpathlineto{\pgfqpoint{2.647959in}{0.711698in}}%
\pgfpathlineto{\pgfqpoint{2.638866in}{0.668124in}}%
\pgfpathlineto{\pgfqpoint{2.629281in}{0.632441in}}%
\pgfpathlineto{\pgfqpoint{2.611118in}{0.576454in}}%
\pgfpathlineto{\pgfqpoint{2.601219in}{0.554303in}}%
\pgfpathlineto{\pgfqpoint{2.590391in}{0.535653in}}%
\pgfpathlineto{\pgfqpoint{2.579124in}{0.520491in}}%
\pgfpathlineto{\pgfqpoint{2.566379in}{0.506949in}}%
\pgfpathlineto{\pgfqpoint{2.552362in}{0.495164in}}%
\pgfpathlineto{\pgfqpoint{2.535418in}{0.483963in}}%
\pgfpathlineto{\pgfqpoint{2.515576in}{0.473784in}}%
\pgfpathlineto{\pgfqpoint{2.492969in}{0.464833in}}%
\pgfpathlineto{\pgfqpoint{2.465637in}{0.456552in}}%
\pgfpathlineto{\pgfqpoint{2.433648in}{0.449211in}}%
\pgfpathlineto{\pgfqpoint{2.392773in}{0.442229in}}%
\pgfpathlineto{\pgfqpoint{2.343037in}{0.436027in}}%
\pgfpathlineto{\pgfqpoint{2.280145in}{0.430428in}}%
\pgfpathlineto{\pgfqpoint{2.197601in}{0.425382in}}%
\pgfpathlineto{\pgfqpoint{2.093249in}{0.421227in}}%
\pgfpathlineto{\pgfqpoint{1.964937in}{0.417984in}}%
\pgfpathlineto{\pgfqpoint{1.695218in}{0.414118in}}%
\pgfpathlineto{\pgfqpoint{1.695218in}{0.414118in}}%
\pgfusepath{stroke}%
\end{pgfscope}%
\begin{pgfscope}%
\pgfpathrectangle{\pgfqpoint{0.448634in}{0.402556in}}{\pgfqpoint{4.350661in}{2.489204in}} %
\pgfusepath{clip}%
\pgfsetrectcap%
\pgfsetroundjoin%
\pgfsetlinewidth{1.003750pt}%
\definecolor{currentstroke}{rgb}{0.498039,0.498039,0.498039}%
\pgfsetstrokecolor{currentstroke}%
\pgfsetdash{}{0pt}%
\pgfpathmoveto{\pgfqpoint{0.807474in}{2.091822in}}%
\pgfpathlineto{\pgfqpoint{0.795033in}{2.026142in}}%
\pgfpathlineto{\pgfqpoint{0.782995in}{1.950220in}}%
\pgfpathlineto{\pgfqpoint{0.771665in}{1.864070in}}%
\pgfpathlineto{\pgfqpoint{0.759448in}{1.752934in}}%
\pgfpathlineto{\pgfqpoint{0.744096in}{1.614657in}}%
\pgfpathlineto{\pgfqpoint{0.736064in}{1.563205in}}%
\pgfpathlineto{\pgfqpoint{0.728752in}{1.529384in}}%
\pgfpathlineto{\pgfqpoint{0.720949in}{1.503513in}}%
\pgfpathlineto{\pgfqpoint{0.713232in}{1.485685in}}%
\pgfpathlineto{\pgfqpoint{0.706888in}{1.475595in}}%
\pgfpathlineto{\pgfqpoint{0.700484in}{1.468890in}}%
\pgfpathlineto{\pgfqpoint{0.694699in}{1.465489in}}%
\pgfpathlineto{\pgfqpoint{0.688282in}{1.464357in}}%
\pgfpathlineto{\pgfqpoint{0.681903in}{1.465772in}}%
\pgfpathlineto{\pgfqpoint{0.674313in}{1.470568in}}%
\pgfpathlineto{\pgfqpoint{0.666379in}{1.479049in}}%
\pgfpathlineto{\pgfqpoint{0.658497in}{1.490938in}}%
\pgfpathlineto{\pgfqpoint{0.649778in}{1.508160in}}%
\pgfpathlineto{\pgfqpoint{0.639899in}{1.533089in}}%
\pgfpathlineto{\pgfqpoint{0.629580in}{1.565869in}}%
\pgfpathlineto{\pgfqpoint{0.619296in}{1.606510in}}%
\pgfpathlineto{\pgfqpoint{0.609380in}{1.654979in}}%
\pgfpathlineto{\pgfqpoint{0.599731in}{1.713686in}}%
\pgfpathlineto{\pgfqpoint{0.590783in}{1.782624in}}%
\pgfpathlineto{\pgfqpoint{0.582889in}{1.861761in}}%
\pgfpathlineto{\pgfqpoint{0.576512in}{1.948575in}}%
\pgfpathlineto{\pgfqpoint{0.571796in}{2.043008in}}%
\pgfpathlineto{\pgfqpoint{0.569027in}{2.142523in}}%
\pgfpathlineto{\pgfqpoint{0.568430in}{2.242086in}}%
\pgfpathlineto{\pgfqpoint{0.570029in}{2.336655in}}%
\pgfpathlineto{\pgfqpoint{0.573608in}{2.421185in}}%
\pgfpathlineto{\pgfqpoint{0.578740in}{2.493129in}}%
\pgfpathlineto{\pgfqpoint{0.585270in}{2.554905in}}%
\pgfpathlineto{\pgfqpoint{0.592862in}{2.606445in}}%
\pgfpathlineto{\pgfqpoint{0.601522in}{2.650134in}}%
\pgfpathlineto{\pgfqpoint{0.610840in}{2.685909in}}%
\pgfpathlineto{\pgfqpoint{0.620987in}{2.716103in}}%
\pgfpathlineto{\pgfqpoint{0.631497in}{2.740691in}}%
\pgfpathlineto{\pgfqpoint{0.642887in}{2.761886in}}%
\pgfpathlineto{\pgfqpoint{0.654811in}{2.779641in}}%
\pgfpathlineto{\pgfqpoint{0.668365in}{2.795791in}}%
\pgfpathlineto{\pgfqpoint{0.683447in}{2.810056in}}%
\pgfpathlineto{\pgfqpoint{0.699824in}{2.822313in}}%
\pgfpathlineto{\pgfqpoint{0.717200in}{2.832617in}}%
\pgfpathlineto{\pgfqpoint{0.737336in}{2.842016in}}%
\pgfpathlineto{\pgfqpoint{0.760147in}{2.850263in}}%
\pgfpathlineto{\pgfqpoint{0.787636in}{2.857837in}}%
\pgfpathlineto{\pgfqpoint{0.821913in}{2.864714in}}%
\pgfpathlineto{\pgfqpoint{0.865086in}{2.870828in}}%
\pgfpathlineto{\pgfqpoint{0.912767in}{2.875476in}}%
\pgfpathlineto{\pgfqpoint{0.977920in}{2.879708in}}%
\pgfpathlineto{\pgfqpoint{1.069225in}{2.883388in}}%
\pgfpathlineto{\pgfqpoint{1.225823in}{2.886594in}}%
\pgfpathlineto{\pgfqpoint{1.445522in}{2.888743in}}%
\pgfpathlineto{\pgfqpoint{1.856658in}{2.890084in}}%
\pgfpathlineto{\pgfqpoint{2.979128in}{2.890740in}}%
\pgfpathlineto{\pgfqpoint{3.944974in}{2.889439in}}%
\pgfpathlineto{\pgfqpoint{4.245162in}{2.887116in}}%
\pgfpathlineto{\pgfqpoint{4.386531in}{2.884024in}}%
\pgfpathlineto{\pgfqpoint{4.473472in}{2.880036in}}%
\pgfpathlineto{\pgfqpoint{4.536393in}{2.874949in}}%
\pgfpathlineto{\pgfqpoint{4.575248in}{2.869457in}}%
\pgfpathlineto{\pgfqpoint{4.603044in}{2.863530in}}%
\pgfpathlineto{\pgfqpoint{4.626149in}{2.856449in}}%
\pgfpathlineto{\pgfqpoint{4.644446in}{2.848507in}}%
\pgfpathlineto{\pgfqpoint{4.659871in}{2.839317in}}%
\pgfpathlineto{\pgfqpoint{4.674107in}{2.827893in}}%
\pgfpathlineto{\pgfqpoint{4.685106in}{2.815862in}}%
\pgfpathlineto{\pgfqpoint{4.694610in}{2.802264in}}%
\pgfpathlineto{\pgfqpoint{4.703687in}{2.785288in}}%
\pgfpathlineto{\pgfqpoint{4.711888in}{2.764960in}}%
\pgfpathlineto{\pgfqpoint{4.726447in}{2.707636in}}%
\pgfpathlineto{\pgfqpoint{4.733051in}{2.668537in}}%
\pgfpathlineto{\pgfqpoint{4.739093in}{2.619240in}}%
\pgfpathlineto{\pgfqpoint{4.744707in}{2.554844in}}%
\pgfpathlineto{\pgfqpoint{4.749719in}{2.470408in}}%
\pgfpathlineto{\pgfqpoint{4.753631in}{2.365960in}}%
\pgfpathlineto{\pgfqpoint{4.756554in}{2.231587in}}%
\pgfpathlineto{\pgfqpoint{4.758058in}{2.064821in}}%
\pgfpathlineto{\pgfqpoint{4.757481in}{1.880622in}}%
\pgfpathlineto{\pgfqpoint{4.754724in}{1.696450in}}%
\pgfpathlineto{\pgfqpoint{4.749931in}{1.527275in}}%
\pgfpathlineto{\pgfqpoint{4.743778in}{1.390552in}}%
\pgfpathlineto{\pgfqpoint{4.736201in}{1.271391in}}%
\pgfpathlineto{\pgfqpoint{4.728339in}{1.184738in}}%
\pgfpathlineto{\pgfqpoint{4.719691in}{1.113237in}}%
\pgfpathlineto{\pgfqpoint{4.710071in}{1.051994in}}%
\pgfpathlineto{\pgfqpoint{4.699870in}{1.001046in}}%
\pgfpathlineto{\pgfqpoint{4.688508in}{0.955580in}}%
\pgfpathlineto{\pgfqpoint{4.676241in}{0.915667in}}%
\pgfpathlineto{\pgfqpoint{4.661124in}{0.874374in}}%
\pgfpathlineto{\pgfqpoint{4.646963in}{0.843531in}}%
\pgfpathlineto{\pgfqpoint{4.632031in}{0.816058in}}%
\pgfpathlineto{\pgfqpoint{4.615295in}{0.789984in}}%
\pgfpathlineto{\pgfqpoint{4.598248in}{0.767374in}}%
\pgfpathlineto{\pgfqpoint{4.579708in}{0.746357in}}%
\pgfpathlineto{\pgfqpoint{4.559770in}{0.727090in}}%
\pgfpathlineto{\pgfqpoint{4.538563in}{0.709688in}}%
\pgfpathlineto{\pgfqpoint{4.514488in}{0.692724in}}%
\pgfpathlineto{\pgfqpoint{4.487462in}{0.676675in}}%
\pgfpathlineto{\pgfqpoint{4.459552in}{0.662744in}}%
\pgfpathlineto{\pgfqpoint{4.428911in}{0.649924in}}%
\pgfpathlineto{\pgfqpoint{4.395626in}{0.638296in}}%
\pgfpathlineto{\pgfqpoint{4.343017in}{0.622721in}}%
\pgfpathlineto{\pgfqpoint{4.300249in}{0.613620in}}%
\pgfpathlineto{\pgfqpoint{4.252911in}{0.605592in}}%
\pgfpathlineto{\pgfqpoint{4.183790in}{0.596380in}}%
\pgfpathlineto{\pgfqpoint{4.123045in}{0.591306in}}%
\pgfpathlineto{\pgfqpoint{4.060021in}{0.588201in}}%
\pgfpathlineto{\pgfqpoint{3.990423in}{0.586821in}}%
\pgfpathlineto{\pgfqpoint{3.918643in}{0.587657in}}%
\pgfpathlineto{\pgfqpoint{3.846909in}{0.590741in}}%
\pgfpathlineto{\pgfqpoint{3.779621in}{0.595790in}}%
\pgfpathlineto{\pgfqpoint{3.729851in}{0.601513in}}%
\pgfpathlineto{\pgfqpoint{3.669391in}{0.609933in}}%
\pgfpathlineto{\pgfqpoint{3.607053in}{0.620935in}}%
\pgfpathlineto{\pgfqpoint{3.555832in}{0.632473in}}%
\pgfpathlineto{\pgfqpoint{3.509281in}{0.645167in}}%
\pgfpathlineto{\pgfqpoint{3.465338in}{0.659435in}}%
\pgfpathlineto{\pgfqpoint{3.426175in}{0.674534in}}%
\pgfpathlineto{\pgfqpoint{3.389771in}{0.691022in}}%
\pgfpathlineto{\pgfqpoint{3.358189in}{0.707747in}}%
\pgfpathlineto{\pgfqpoint{3.329430in}{0.725376in}}%
\pgfpathlineto{\pgfqpoint{3.301690in}{0.745024in}}%
\pgfpathlineto{\pgfqpoint{3.276917in}{0.765281in}}%
\pgfpathlineto{\pgfqpoint{3.253422in}{0.787444in}}%
\pgfpathlineto{\pgfqpoint{3.231380in}{0.811480in}}%
\pgfpathlineto{\pgfqpoint{3.210920in}{0.837282in}}%
\pgfpathlineto{\pgfqpoint{3.192112in}{0.864683in}}%
\pgfpathlineto{\pgfqpoint{3.175012in}{0.893511in}}%
\pgfpathlineto{\pgfqpoint{3.158553in}{0.925743in}}%
\pgfpathlineto{\pgfqpoint{3.143968in}{0.959136in}}%
\pgfpathlineto{\pgfqpoint{3.130428in}{0.995819in}}%
\pgfpathlineto{\pgfqpoint{3.118087in}{1.035704in}}%
\pgfpathlineto{\pgfqpoint{3.107178in}{1.078730in}}%
\pgfpathlineto{\pgfqpoint{3.097839in}{1.124795in}}%
\pgfpathlineto{\pgfqpoint{3.090221in}{1.173805in}}%
\pgfpathlineto{\pgfqpoint{3.084459in}{1.225656in}}%
\pgfpathlineto{\pgfqpoint{3.080676in}{1.280242in}}%
\pgfpathlineto{\pgfqpoint{3.079021in}{1.337457in}}%
\pgfpathlineto{\pgfqpoint{3.079595in}{1.394701in}}%
\pgfpathlineto{\pgfqpoint{3.082336in}{1.451862in}}%
\pgfpathlineto{\pgfqpoint{3.087247in}{1.508832in}}%
\pgfpathlineto{\pgfqpoint{3.094358in}{1.565498in}}%
\pgfpathlineto{\pgfqpoint{3.103727in}{1.621732in}}%
\pgfpathlineto{\pgfqpoint{3.114907in}{1.674974in}}%
\pgfpathlineto{\pgfqpoint{3.127673in}{1.725159in}}%
\pgfpathlineto{\pgfqpoint{3.142559in}{1.774573in}}%
\pgfpathlineto{\pgfqpoint{3.158767in}{1.820769in}}%
\pgfpathlineto{\pgfqpoint{3.176993in}{1.865968in}}%
\pgfpathlineto{\pgfqpoint{3.196188in}{1.907847in}}%
\pgfpathlineto{\pgfqpoint{3.217197in}{1.948570in}}%
\pgfpathlineto{\pgfqpoint{3.239957in}{1.988043in}}%
\pgfpathlineto{\pgfqpoint{3.265672in}{2.028194in}}%
\pgfpathlineto{\pgfqpoint{3.295727in}{2.070804in}}%
\pgfpathlineto{\pgfqpoint{3.363650in}{2.164709in}}%
\pgfpathlineto{\pgfqpoint{3.369805in}{2.177857in}}%
\pgfpathlineto{\pgfqpoint{3.372038in}{2.187442in}}%
\pgfpathlineto{\pgfqpoint{3.371249in}{2.194787in}}%
\pgfpathlineto{\pgfqpoint{3.368917in}{2.198952in}}%
\pgfpathlineto{\pgfqpoint{3.363513in}{2.203039in}}%
\pgfpathlineto{\pgfqpoint{3.355081in}{2.205374in}}%
\pgfpathlineto{\pgfqpoint{3.344221in}{2.205667in}}%
\pgfpathlineto{\pgfqpoint{3.329125in}{2.203476in}}%
\pgfpathlineto{\pgfqpoint{3.310135in}{2.198073in}}%
\pgfpathlineto{\pgfqpoint{3.289621in}{2.189812in}}%
\pgfpathlineto{\pgfqpoint{3.267770in}{2.178669in}}%
\pgfpathlineto{\pgfqpoint{3.244793in}{2.164508in}}%
\pgfpathlineto{\pgfqpoint{3.220937in}{2.147144in}}%
\pgfpathlineto{\pgfqpoint{3.198190in}{2.127928in}}%
\pgfpathlineto{\pgfqpoint{3.174968in}{2.105393in}}%
\pgfpathlineto{\pgfqpoint{3.153089in}{2.081159in}}%
\pgfpathlineto{\pgfqpoint{3.131143in}{2.053536in}}%
\pgfpathlineto{\pgfqpoint{3.109418in}{2.022429in}}%
\pgfpathlineto{\pgfqpoint{3.089380in}{1.989871in}}%
\pgfpathlineto{\pgfqpoint{3.069855in}{1.953940in}}%
\pgfpathlineto{\pgfqpoint{3.051047in}{1.914648in}}%
\pgfpathlineto{\pgfqpoint{3.046146in}{1.903537in}}%
\pgfpathlineto{\pgfqpoint{3.046146in}{1.903537in}}%
\pgfusepath{stroke}%
\end{pgfscope}%
\begin{pgfscope}%
\pgfpathrectangle{\pgfqpoint{0.448634in}{0.402556in}}{\pgfqpoint{4.350661in}{2.489204in}} %
\pgfusepath{clip}%
\pgfsetrectcap%
\pgfsetroundjoin%
\pgfsetlinewidth{1.003750pt}%
\definecolor{currentstroke}{rgb}{0.498039,0.498039,0.498039}%
\pgfsetstrokecolor{currentstroke}%
\pgfsetdash{}{0pt}%
\pgfpathmoveto{\pgfqpoint{3.442326in}{0.402556in}}%
\pgfpathlineto{\pgfqpoint{0.449071in}{0.402556in}}%
\pgfpathlineto{\pgfqpoint{0.449071in}{0.402556in}}%
\pgfusepath{stroke}%
\end{pgfscope}%
\begin{pgfscope}%
\pgfpathrectangle{\pgfqpoint{0.448634in}{0.402556in}}{\pgfqpoint{4.350661in}{2.489204in}} %
\pgfusepath{clip}%
\pgfsetrectcap%
\pgfsetroundjoin%
\pgfsetlinewidth{1.003750pt}%
\definecolor{currentstroke}{rgb}{0.498039,0.498039,0.498039}%
\pgfsetstrokecolor{currentstroke}%
\pgfsetdash{}{0pt}%
\pgfpathmoveto{\pgfqpoint{0.696146in}{1.471374in}}%
\pgfpathlineto{\pgfqpoint{0.689892in}{1.469370in}}%
\pgfpathlineto{\pgfqpoint{0.683418in}{1.469946in}}%
\pgfpathlineto{\pgfqpoint{0.677392in}{1.472755in}}%
\pgfpathlineto{\pgfqpoint{0.670402in}{1.478649in}}%
\pgfpathlineto{\pgfqpoint{0.663057in}{1.487810in}}%
\pgfpathlineto{\pgfqpoint{0.654540in}{1.502239in}}%
\pgfpathlineto{\pgfqpoint{0.645513in}{1.522107in}}%
\pgfpathlineto{\pgfqpoint{0.635595in}{1.549729in}}%
\pgfpathlineto{\pgfqpoint{0.625407in}{1.585193in}}%
\pgfpathlineto{\pgfqpoint{0.615366in}{1.628495in}}%
\pgfpathlineto{\pgfqpoint{0.605368in}{1.682045in}}%
\pgfpathlineto{\pgfqpoint{0.596227in}{1.743386in}}%
\pgfpathlineto{\pgfqpoint{0.587851in}{1.814930in}}%
\pgfpathlineto{\pgfqpoint{0.580582in}{1.896649in}}%
\pgfpathlineto{\pgfqpoint{0.574732in}{1.988503in}}%
\pgfpathlineto{\pgfqpoint{0.570798in}{2.085474in}}%
\pgfpathlineto{\pgfqpoint{0.568935in}{2.185017in}}%
\pgfpathlineto{\pgfqpoint{0.569288in}{2.284582in}}%
\pgfpathlineto{\pgfqpoint{0.571759in}{2.374145in}}%
\pgfpathlineto{\pgfqpoint{0.576072in}{2.453643in}}%
\pgfpathlineto{\pgfqpoint{0.582034in}{2.523001in}}%
\pgfpathlineto{\pgfqpoint{0.589072in}{2.579679in}}%
\pgfpathlineto{\pgfqpoint{0.596922in}{2.626107in}}%
\pgfpathlineto{\pgfqpoint{0.605521in}{2.664692in}}%
\pgfpathlineto{\pgfqpoint{0.615069in}{2.697775in}}%
\pgfpathlineto{\pgfqpoint{0.625219in}{2.725285in}}%
\pgfpathlineto{\pgfqpoint{0.636522in}{2.749406in}}%
\pgfpathlineto{\pgfqpoint{0.648736in}{2.769989in}}%
\pgfpathlineto{\pgfqpoint{0.661441in}{2.787020in}}%
\pgfpathlineto{\pgfqpoint{0.675745in}{2.802297in}}%
\pgfpathlineto{\pgfqpoint{0.691478in}{2.815612in}}%
\pgfpathlineto{\pgfqpoint{0.708364in}{2.826929in}}%
\pgfpathlineto{\pgfqpoint{0.728122in}{2.837321in}}%
\pgfpathlineto{\pgfqpoint{0.750665in}{2.846481in}}%
\pgfpathlineto{\pgfqpoint{0.775855in}{2.854292in}}%
\pgfpathlineto{\pgfqpoint{0.807830in}{2.861705in}}%
\pgfpathlineto{\pgfqpoint{0.844418in}{2.867832in}}%
\pgfpathlineto{\pgfqpoint{0.894194in}{2.873590in}}%
\pgfpathlineto{\pgfqpoint{0.963656in}{2.878728in}}%
\pgfpathlineto{\pgfqpoint{1.063647in}{2.883088in}}%
\pgfpathlineto{\pgfqpoint{1.215892in}{2.886361in}}%
\pgfpathlineto{\pgfqpoint{1.433415in}{2.888618in}}%
\pgfpathlineto{\pgfqpoint{1.792343in}{2.889941in}}%
\pgfpathlineto{\pgfqpoint{2.721209in}{2.890739in}}%
\pgfpathlineto{\pgfqpoint{3.854555in}{2.889725in}}%
\pgfpathlineto{\pgfqpoint{4.206953in}{2.887544in}}%
\pgfpathlineto{\pgfqpoint{4.365729in}{2.884554in}}%
\pgfpathlineto{\pgfqpoint{4.459206in}{2.880721in}}%
\pgfpathlineto{\pgfqpoint{4.526497in}{2.875759in}}%
\pgfpathlineto{\pgfqpoint{4.567559in}{2.870418in}}%
\pgfpathlineto{\pgfqpoint{4.597560in}{2.864460in}}%
\pgfpathlineto{\pgfqpoint{4.620767in}{2.857820in}}%
\pgfpathlineto{\pgfqpoint{4.639232in}{2.850402in}}%
\pgfpathlineto{\pgfqpoint{4.654921in}{2.841812in}}%
\pgfpathlineto{\pgfqpoint{4.669546in}{2.831042in}}%
\pgfpathlineto{\pgfqpoint{4.679474in}{2.821362in}}%
\pgfpathlineto{\pgfqpoint{4.689699in}{2.808467in}}%
\pgfpathlineto{\pgfqpoint{4.699557in}{2.792075in}}%
\pgfpathlineto{\pgfqpoint{4.708542in}{2.772185in}}%
\pgfpathlineto{\pgfqpoint{4.713801in}{2.755855in}}%
\pgfpathlineto{\pgfqpoint{4.721016in}{2.729760in}}%
\pgfpathlineto{\pgfqpoint{4.728533in}{2.693434in}}%
\pgfpathlineto{\pgfqpoint{4.735144in}{2.649277in}}%
\pgfpathlineto{\pgfqpoint{4.741230in}{2.592455in}}%
\pgfpathlineto{\pgfqpoint{4.746594in}{2.520533in}}%
\pgfpathlineto{\pgfqpoint{4.751088in}{2.428580in}}%
\pgfpathlineto{\pgfqpoint{4.754934in}{2.304200in}}%
\pgfpathlineto{\pgfqpoint{4.757184in}{2.157361in}}%
\pgfpathlineto{\pgfqpoint{4.757769in}{1.980630in}}%
\pgfpathlineto{\pgfqpoint{4.756149in}{1.793950in}}%
\pgfpathlineto{\pgfqpoint{4.752410in}{1.617271in}}%
\pgfpathlineto{\pgfqpoint{4.746838in}{1.460584in}}%
\pgfpathlineto{\pgfqpoint{4.740310in}{1.338844in}}%
\pgfpathlineto{\pgfqpoint{4.732199in}{1.229721in}}%
\pgfpathlineto{\pgfqpoint{4.723924in}{1.150636in}}%
\pgfpathlineto{\pgfqpoint{4.714794in}{1.084248in}}%
\pgfpathlineto{\pgfqpoint{4.704846in}{1.028145in}}%
\pgfpathlineto{\pgfqpoint{4.694064in}{0.979920in}}%
\pgfpathlineto{\pgfqpoint{4.682238in}{0.937214in}}%
\pgfpathlineto{\pgfqpoint{4.667496in}{0.893047in}}%
\pgfpathlineto{\pgfqpoint{4.653336in}{0.859417in}}%
\pgfpathlineto{\pgfqpoint{4.638133in}{0.829231in}}%
\pgfpathlineto{\pgfqpoint{4.622174in}{0.802526in}}%
\pgfpathlineto{\pgfqpoint{4.604413in}{0.777356in}}%
\pgfpathlineto{\pgfqpoint{4.586438in}{0.755707in}}%
\pgfpathlineto{\pgfqpoint{4.567040in}{0.735729in}}%
\pgfpathlineto{\pgfqpoint{4.546309in}{0.717591in}}%
\pgfpathlineto{\pgfqpoint{4.517001in}{0.696130in}}%
\pgfpathlineto{\pgfqpoint{4.492061in}{0.680890in}}%
\pgfpathlineto{\pgfqpoint{4.464309in}{0.666552in}}%
\pgfpathlineto{\pgfqpoint{4.433802in}{0.653321in}}%
\pgfpathlineto{\pgfqpoint{4.396426in}{0.639986in}}%
\pgfpathlineto{\pgfqpoint{4.343862in}{0.624230in}}%
\pgfpathlineto{\pgfqpoint{4.303247in}{0.615492in}}%
\pgfpathlineto{\pgfqpoint{4.255931in}{0.607303in}}%
\pgfpathlineto{\pgfqpoint{4.180355in}{0.597104in}}%
\pgfpathlineto{\pgfqpoint{4.117431in}{0.591995in}}%
\pgfpathlineto{\pgfqpoint{4.054399in}{0.589093in}}%
\pgfpathlineto{\pgfqpoint{3.984798in}{0.587899in}}%
\pgfpathlineto{\pgfqpoint{3.913020in}{0.588938in}}%
\pgfpathlineto{\pgfqpoint{3.841294in}{0.592221in}}%
\pgfpathlineto{\pgfqpoint{3.769683in}{0.597904in}}%
\pgfpathlineto{\pgfqpoint{3.724287in}{0.603550in}}%
\pgfpathlineto{\pgfqpoint{3.657368in}{0.613036in}}%
\pgfpathlineto{\pgfqpoint{3.601565in}{0.623559in}}%
\pgfpathlineto{\pgfqpoint{3.550403in}{0.635441in}}%
\pgfpathlineto{\pgfqpoint{3.503933in}{0.648509in}}%
\pgfpathlineto{\pgfqpoint{3.460091in}{0.663175in}}%
\pgfpathlineto{\pgfqpoint{3.421060in}{0.678716in}}%
\pgfpathlineto{\pgfqpoint{3.384818in}{0.695663in}}%
\pgfpathlineto{\pgfqpoint{3.353433in}{0.712867in}}%
\pgfpathlineto{\pgfqpoint{3.324904in}{0.730978in}}%
\pgfpathlineto{\pgfqpoint{3.297429in}{0.751108in}}%
\pgfpathlineto{\pgfqpoint{3.276378in}{0.768762in}}%
\pgfpathlineto{\pgfqpoint{3.253029in}{0.791126in}}%
\pgfpathlineto{\pgfqpoint{3.231164in}{0.815373in}}%
\pgfpathlineto{\pgfqpoint{3.210904in}{0.841382in}}%
\pgfpathlineto{\pgfqpoint{3.192296in}{0.868960in}}%
\pgfpathlineto{\pgfqpoint{3.175416in}{0.897958in}}%
\pgfpathlineto{\pgfqpoint{3.159181in}{0.930338in}}%
\pgfpathlineto{\pgfqpoint{3.144812in}{0.963854in}}%
\pgfpathlineto{\pgfqpoint{3.132337in}{0.998350in}}%
\pgfpathlineto{\pgfqpoint{3.118818in}{1.043036in}}%
\pgfpathlineto{\pgfqpoint{3.108293in}{1.086186in}}%
\pgfpathlineto{\pgfqpoint{3.099310in}{1.132345in}}%
\pgfpathlineto{\pgfqpoint{3.092050in}{1.181426in}}%
\pgfpathlineto{\pgfqpoint{3.086653in}{1.233328in}}%
\pgfpathlineto{\pgfqpoint{3.083259in}{1.287947in}}%
\pgfpathlineto{\pgfqpoint{3.081999in}{1.345175in}}%
\pgfpathlineto{\pgfqpoint{3.082987in}{1.402411in}}%
\pgfpathlineto{\pgfqpoint{3.086150in}{1.459544in}}%
\pgfpathlineto{\pgfqpoint{3.091530in}{1.516458in}}%
\pgfpathlineto{\pgfqpoint{3.099149in}{1.573037in}}%
\pgfpathlineto{\pgfqpoint{3.109032in}{1.629155in}}%
\pgfpathlineto{\pgfqpoint{3.120815in}{1.682226in}}%
\pgfpathlineto{\pgfqpoint{3.134214in}{1.732194in}}%
\pgfpathlineto{\pgfqpoint{3.149019in}{1.779001in}}%
\pgfpathlineto{\pgfqpoint{3.164976in}{1.822625in}}%
\pgfpathlineto{\pgfqpoint{3.182839in}{1.865268in}}%
\pgfpathlineto{\pgfqpoint{3.202623in}{1.906787in}}%
\pgfpathlineto{\pgfqpoint{3.224334in}{1.947024in}}%
\pgfpathlineto{\pgfqpoint{3.247906in}{1.985866in}}%
\pgfpathlineto{\pgfqpoint{3.273287in}{2.023185in}}%
\pgfpathlineto{\pgfqpoint{3.300333in}{2.058942in}}%
\pgfpathlineto{\pgfqpoint{3.330360in}{2.094962in}}%
\pgfpathlineto{\pgfqpoint{3.369522in}{2.138136in}}%
\pgfpathlineto{\pgfqpoint{3.416472in}{2.189984in}}%
\pgfpathlineto{\pgfqpoint{3.429059in}{2.207125in}}%
\pgfpathlineto{\pgfqpoint{3.435713in}{2.219941in}}%
\pgfpathlineto{\pgfqpoint{3.437928in}{2.229518in}}%
\pgfpathlineto{\pgfqpoint{3.436876in}{2.236818in}}%
\pgfpathlineto{\pgfqpoint{3.432927in}{2.242677in}}%
\pgfpathlineto{\pgfqpoint{3.427329in}{2.246468in}}%
\pgfpathlineto{\pgfqpoint{3.419000in}{2.249287in}}%
\pgfpathlineto{\pgfqpoint{3.406033in}{2.250833in}}%
\pgfpathlineto{\pgfqpoint{3.390821in}{2.250298in}}%
\pgfpathlineto{\pgfqpoint{3.371424in}{2.247317in}}%
\pgfpathlineto{\pgfqpoint{3.348096in}{2.241258in}}%
\pgfpathlineto{\pgfqpoint{3.323191in}{2.232334in}}%
\pgfpathlineto{\pgfqpoint{3.296910in}{2.220406in}}%
\pgfpathlineto{\pgfqpoint{3.271419in}{2.206409in}}%
\pgfpathlineto{\pgfqpoint{3.244928in}{2.189231in}}%
\pgfpathlineto{\pgfqpoint{3.219517in}{2.170036in}}%
\pgfpathlineto{\pgfqpoint{3.195261in}{2.148972in}}%
\pgfpathlineto{\pgfqpoint{3.170627in}{2.124497in}}%
\pgfpathlineto{\pgfqpoint{3.147446in}{2.098229in}}%
\pgfpathlineto{\pgfqpoint{3.125741in}{2.070358in}}%
\pgfpathlineto{\pgfqpoint{3.104209in}{2.039074in}}%
\pgfpathlineto{\pgfqpoint{3.083122in}{2.004319in}}%
\pgfpathlineto{\pgfqpoint{3.063804in}{1.968241in}}%
\pgfpathlineto{\pgfqpoint{3.045166in}{1.928843in}}%
\pgfpathlineto{\pgfqpoint{3.027382in}{1.886155in}}%
\pgfpathlineto{\pgfqpoint{3.010595in}{1.840232in}}%
\pgfpathlineto{\pgfqpoint{2.994215in}{1.788782in}}%
\pgfpathlineto{\pgfqpoint{2.979214in}{1.734169in}}%
\pgfpathlineto{\pgfqpoint{2.965662in}{1.676480in}}%
\pgfpathlineto{\pgfqpoint{2.953174in}{1.613362in}}%
\pgfpathlineto{\pgfqpoint{2.942409in}{1.547296in}}%
\pgfpathlineto{\pgfqpoint{2.933421in}{1.478365in}}%
\pgfpathlineto{\pgfqpoint{2.926293in}{1.406644in}}%
\pgfpathlineto{\pgfqpoint{2.921151in}{1.332204in}}%
\pgfpathlineto{\pgfqpoint{2.918238in}{1.257606in}}%
\pgfpathlineto{\pgfqpoint{2.917600in}{1.185426in}}%
\pgfpathlineto{\pgfqpoint{2.919148in}{1.115755in}}%
\pgfpathlineto{\pgfqpoint{2.922716in}{1.051169in}}%
\pgfpathlineto{\pgfqpoint{2.928113in}{0.991752in}}%
\pgfpathlineto{\pgfqpoint{2.935171in}{0.937594in}}%
\pgfpathlineto{\pgfqpoint{2.943691in}{0.888779in}}%
\pgfpathlineto{\pgfqpoint{2.953424in}{0.845386in}}%
\pgfpathlineto{\pgfqpoint{2.964764in}{0.805116in}}%
\pgfpathlineto{\pgfqpoint{2.976799in}{0.770418in}}%
\pgfpathlineto{\pgfqpoint{2.989990in}{0.739016in}}%
\pgfpathlineto{\pgfqpoint{3.004096in}{0.710979in}}%
\pgfpathlineto{\pgfqpoint{3.020096in}{0.684307in}}%
\pgfpathlineto{\pgfqpoint{3.036570in}{0.661147in}}%
\pgfpathlineto{\pgfqpoint{3.054629in}{0.639589in}}%
\pgfpathlineto{\pgfqpoint{3.074156in}{0.619778in}}%
\pgfpathlineto{\pgfqpoint{3.094985in}{0.601786in}}%
\pgfpathlineto{\pgfqpoint{3.118796in}{0.584344in}}%
\pgfpathlineto{\pgfqpoint{3.143662in}{0.568945in}}%
\pgfpathlineto{\pgfqpoint{3.171361in}{0.554476in}}%
\pgfpathlineto{\pgfqpoint{3.201837in}{0.541150in}}%
\pgfpathlineto{\pgfqpoint{3.235003in}{0.529086in}}%
\pgfpathlineto{\pgfqpoint{3.272880in}{0.517746in}}%
\pgfpathlineto{\pgfqpoint{3.315433in}{0.507400in}}%
\pgfpathlineto{\pgfqpoint{3.364755in}{0.497807in}}%
\pgfpathlineto{\pgfqpoint{3.420811in}{0.489221in}}%
\pgfpathlineto{\pgfqpoint{3.487888in}{0.481292in}}%
\pgfpathlineto{\pgfqpoint{3.568132in}{0.474174in}}%
\pgfpathlineto{\pgfqpoint{3.661523in}{0.468184in}}%
\pgfpathlineto{\pgfqpoint{3.772384in}{0.463374in}}%
\pgfpathlineto{\pgfqpoint{3.894172in}{0.460315in}}%
\pgfpathlineto{\pgfqpoint{4.022513in}{0.459321in}}%
\pgfpathlineto{\pgfqpoint{4.148676in}{0.460549in}}%
\pgfpathlineto{\pgfqpoint{4.261756in}{0.463813in}}%
\pgfpathlineto{\pgfqpoint{4.357371in}{0.468751in}}%
\pgfpathlineto{\pgfqpoint{4.433321in}{0.474804in}}%
\pgfpathlineto{\pgfqpoint{4.493924in}{0.481747in}}%
\pgfpathlineto{\pgfqpoint{4.541329in}{0.489234in}}%
\pgfpathlineto{\pgfqpoint{4.579828in}{0.497382in}}%
\pgfpathlineto{\pgfqpoint{4.611526in}{0.506217in}}%
\pgfpathlineto{\pgfqpoint{4.638468in}{0.516023in}}%
\pgfpathlineto{\pgfqpoint{4.660585in}{0.526450in}}%
\pgfpathlineto{\pgfqpoint{4.679794in}{0.538107in}}%
\pgfpathlineto{\pgfqpoint{4.695963in}{0.550717in}}%
\pgfpathlineto{\pgfqpoint{4.709105in}{0.563754in}}%
\pgfpathlineto{\pgfqpoint{4.720818in}{0.578467in}}%
\pgfpathlineto{\pgfqpoint{4.730956in}{0.594640in}}%
\pgfpathlineto{\pgfqpoint{4.740501in}{0.614186in}}%
\pgfpathlineto{\pgfqpoint{4.749065in}{0.637057in}}%
\pgfpathlineto{\pgfqpoint{4.757060in}{0.665482in}}%
\pgfpathlineto{\pgfqpoint{4.764099in}{0.699380in}}%
\pgfpathlineto{\pgfqpoint{4.770390in}{0.741074in}}%
\pgfpathlineto{\pgfqpoint{4.776182in}{0.795430in}}%
\pgfpathlineto{\pgfqpoint{4.781210in}{0.864886in}}%
\pgfpathlineto{\pgfqpoint{4.785498in}{0.956854in}}%
\pgfpathlineto{\pgfqpoint{4.788958in}{1.081249in}}%
\pgfpathlineto{\pgfqpoint{4.791705in}{1.265423in}}%
\pgfpathlineto{\pgfqpoint{4.793815in}{1.571585in}}%
\pgfpathlineto{\pgfqpoint{4.794839in}{2.064445in}}%
\pgfpathlineto{\pgfqpoint{4.793828in}{2.544859in}}%
\pgfpathlineto{\pgfqpoint{4.791561in}{2.738997in}}%
\pgfpathlineto{\pgfqpoint{4.788803in}{2.813601in}}%
\pgfpathlineto{\pgfqpoint{4.785495in}{2.848230in}}%
\pgfpathlineto{\pgfqpoint{4.781845in}{2.865124in}}%
\pgfpathlineto{\pgfqpoint{4.778038in}{2.874050in}}%
\pgfpathlineto{\pgfqpoint{4.773726in}{2.879623in}}%
\pgfpathlineto{\pgfqpoint{4.768227in}{2.883602in}}%
\pgfpathlineto{\pgfqpoint{4.759970in}{2.886683in}}%
\pgfpathlineto{\pgfqpoint{4.744904in}{2.889125in}}%
\pgfpathlineto{\pgfqpoint{4.716660in}{2.890621in}}%
\pgfpathlineto{\pgfqpoint{4.644878in}{2.891420in}}%
\pgfpathlineto{\pgfqpoint{4.299001in}{2.891719in}}%
\pgfpathlineto{\pgfqpoint{0.664024in}{2.891129in}}%
\pgfpathlineto{\pgfqpoint{0.566149in}{2.889454in}}%
\pgfpathlineto{\pgfqpoint{0.529243in}{2.886897in}}%
\pgfpathlineto{\pgfqpoint{0.509909in}{2.883455in}}%
\pgfpathlineto{\pgfqpoint{0.497498in}{2.878904in}}%
\pgfpathlineto{\pgfqpoint{0.489937in}{2.874013in}}%
\pgfpathlineto{\pgfqpoint{0.483538in}{2.867299in}}%
\pgfpathlineto{\pgfqpoint{0.478595in}{2.859126in}}%
\pgfpathlineto{\pgfqpoint{0.474160in}{2.847777in}}%
\pgfpathlineto{\pgfqpoint{0.470045in}{2.831013in}}%
\pgfpathlineto{\pgfqpoint{0.466404in}{2.806480in}}%
\pgfpathlineto{\pgfqpoint{0.463045in}{2.766844in}}%
\pgfpathlineto{\pgfqpoint{0.460136in}{2.702213in}}%
\pgfpathlineto{\pgfqpoint{0.457713in}{2.592725in}}%
\pgfpathlineto{\pgfqpoint{0.456643in}{2.503122in}}%
\pgfpathlineto{\pgfqpoint{0.456643in}{2.503122in}}%
\pgfusepath{stroke}%
\end{pgfscope}%
\begin{pgfscope}%
\pgfpathrectangle{\pgfqpoint{0.448634in}{0.402556in}}{\pgfqpoint{4.350661in}{2.489204in}} %
\pgfusepath{clip}%
\pgfsetrectcap%
\pgfsetroundjoin%
\pgfsetlinewidth{1.003750pt}%
\definecolor{currentstroke}{rgb}{0.498039,0.498039,0.498039}%
\pgfsetstrokecolor{currentstroke}%
\pgfsetdash{}{0pt}%
\pgfpathmoveto{\pgfqpoint{0.598519in}{1.486863in}}%
\pgfpathlineto{\pgfqpoint{0.610603in}{1.431311in}}%
\pgfpathlineto{\pgfqpoint{0.623757in}{1.381258in}}%
\pgfpathlineto{\pgfqpoint{0.637012in}{1.339103in}}%
\pgfpathlineto{\pgfqpoint{0.650716in}{1.302501in}}%
\pgfpathlineto{\pgfqpoint{0.664619in}{1.271504in}}%
\pgfpathlineto{\pgfqpoint{0.678416in}{1.246159in}}%
\pgfpathlineto{\pgfqpoint{0.691695in}{1.226459in}}%
\pgfpathlineto{\pgfqpoint{0.703879in}{1.212260in}}%
\pgfpathlineto{\pgfqpoint{0.714199in}{1.203137in}}%
\pgfpathlineto{\pgfqpoint{0.723733in}{1.197180in}}%
\pgfpathlineto{\pgfqpoint{0.731994in}{1.194101in}}%
\pgfpathlineto{\pgfqpoint{0.740636in}{1.193202in}}%
\pgfpathlineto{\pgfqpoint{0.749160in}{1.195022in}}%
\pgfpathlineto{\pgfqpoint{0.756908in}{1.199487in}}%
\pgfpathlineto{\pgfqpoint{0.763533in}{1.205916in}}%
\pgfpathlineto{\pgfqpoint{0.770325in}{1.215616in}}%
\pgfpathlineto{\pgfqpoint{0.777725in}{1.230825in}}%
\pgfpathlineto{\pgfqpoint{0.784139in}{1.249326in}}%
\pgfpathlineto{\pgfqpoint{0.790129in}{1.273247in}}%
\pgfpathlineto{\pgfqpoint{0.795616in}{1.304983in}}%
\pgfpathlineto{\pgfqpoint{0.799637in}{1.342030in}}%
\pgfpathlineto{\pgfqpoint{0.802134in}{1.386739in}}%
\pgfpathlineto{\pgfqpoint{0.802963in}{1.443979in}}%
\pgfpathlineto{\pgfqpoint{0.801766in}{1.553495in}}%
\pgfpathlineto{\pgfqpoint{0.801451in}{1.667994in}}%
\pgfpathlineto{\pgfqpoint{0.803471in}{1.755082in}}%
\pgfpathlineto{\pgfqpoint{0.807624in}{1.837085in}}%
\pgfpathlineto{\pgfqpoint{0.813753in}{1.913927in}}%
\pgfpathlineto{\pgfqpoint{0.821687in}{1.985537in}}%
\pgfpathlineto{\pgfqpoint{0.831269in}{2.051840in}}%
\pgfpathlineto{\pgfqpoint{0.842337in}{2.112763in}}%
\pgfpathlineto{\pgfqpoint{0.854689in}{2.168237in}}%
\pgfpathlineto{\pgfqpoint{0.868062in}{2.218215in}}%
\pgfpathlineto{\pgfqpoint{0.882913in}{2.265003in}}%
\pgfpathlineto{\pgfqpoint{0.898257in}{2.306219in}}%
\pgfpathlineto{\pgfqpoint{0.914632in}{2.344154in}}%
\pgfpathlineto{\pgfqpoint{0.934987in}{2.385302in}}%
\pgfpathlineto{\pgfqpoint{0.954354in}{2.418386in}}%
\pgfpathlineto{\pgfqpoint{0.974205in}{2.448012in}}%
\pgfpathlineto{\pgfqpoint{0.995670in}{2.476125in}}%
\pgfpathlineto{\pgfqpoint{1.018695in}{2.502572in}}%
\pgfpathlineto{\pgfqpoint{1.043193in}{2.527227in}}%
\pgfpathlineto{\pgfqpoint{1.067328in}{2.548468in}}%
\pgfpathlineto{\pgfqpoint{1.101618in}{2.574857in}}%
\pgfpathlineto{\pgfqpoint{1.131697in}{2.594884in}}%
\pgfpathlineto{\pgfqpoint{1.164660in}{2.614054in}}%
\pgfpathlineto{\pgfqpoint{1.200467in}{2.632172in}}%
\pgfpathlineto{\pgfqpoint{1.259043in}{2.658603in}}%
\pgfpathlineto{\pgfqpoint{1.300603in}{2.673314in}}%
\pgfpathlineto{\pgfqpoint{1.346840in}{2.687433in}}%
\pgfpathlineto{\pgfqpoint{1.399870in}{2.701214in}}%
\pgfpathlineto{\pgfqpoint{1.459690in}{2.714317in}}%
\pgfpathlineto{\pgfqpoint{1.519828in}{2.725362in}}%
\pgfpathlineto{\pgfqpoint{1.575906in}{2.733656in}}%
\pgfpathlineto{\pgfqpoint{1.658056in}{2.744154in}}%
\pgfpathlineto{\pgfqpoint{1.751247in}{2.753355in}}%
\pgfpathlineto{\pgfqpoint{1.842423in}{2.760028in}}%
\pgfpathlineto{\pgfqpoint{1.942385in}{2.765168in}}%
\pgfpathlineto{\pgfqpoint{2.051111in}{2.768520in}}%
\pgfpathlineto{\pgfqpoint{2.164223in}{2.769537in}}%
\pgfpathlineto{\pgfqpoint{2.268632in}{2.768332in}}%
\pgfpathlineto{\pgfqpoint{2.379527in}{2.764762in}}%
\pgfpathlineto{\pgfqpoint{2.468594in}{2.759495in}}%
\pgfpathlineto{\pgfqpoint{2.546671in}{2.752586in}}%
\pgfpathlineto{\pgfqpoint{2.611557in}{2.744640in}}%
\pgfpathlineto{\pgfqpoint{2.667543in}{2.735476in}}%
\pgfpathlineto{\pgfqpoint{2.712446in}{2.725893in}}%
\pgfpathlineto{\pgfqpoint{2.750523in}{2.715469in}}%
\pgfpathlineto{\pgfqpoint{2.781732in}{2.704596in}}%
\pgfpathlineto{\pgfqpoint{2.806108in}{2.693928in}}%
\pgfpathlineto{\pgfqpoint{2.827675in}{2.682089in}}%
\pgfpathlineto{\pgfqpoint{2.844433in}{2.670523in}}%
\pgfpathlineto{\pgfqpoint{2.858303in}{2.658513in}}%
\pgfpathlineto{\pgfqpoint{2.870783in}{2.644655in}}%
\pgfpathlineto{\pgfqpoint{2.880186in}{2.630966in}}%
\pgfpathlineto{\pgfqpoint{2.887873in}{2.615942in}}%
\pgfpathlineto{\pgfqpoint{2.893701in}{2.599858in}}%
\pgfpathlineto{\pgfqpoint{2.898145in}{2.580620in}}%
\pgfpathlineto{\pgfqpoint{2.900679in}{2.558419in}}%
\pgfpathlineto{\pgfqpoint{2.901097in}{2.533542in}}%
\pgfpathlineto{\pgfqpoint{2.899199in}{2.503759in}}%
\pgfpathlineto{\pgfqpoint{2.894427in}{2.466829in}}%
\pgfpathlineto{\pgfqpoint{2.885676in}{2.418067in}}%
\pgfpathlineto{\pgfqpoint{2.868753in}{2.338241in}}%
\pgfpathlineto{\pgfqpoint{2.838023in}{2.193091in}}%
\pgfpathlineto{\pgfqpoint{2.820844in}{2.100569in}}%
\pgfpathlineto{\pgfqpoint{2.811758in}{2.041777in}}%
\pgfpathlineto{\pgfqpoint{2.797651in}{1.946052in}}%
\pgfpathlineto{\pgfqpoint{2.784387in}{1.840101in}}%
\pgfpathlineto{\pgfqpoint{2.772196in}{1.723945in}}%
\pgfpathlineto{\pgfqpoint{2.761254in}{1.597617in}}%
\pgfpathlineto{\pgfqpoint{2.751470in}{1.458674in}}%
\pgfpathlineto{\pgfqpoint{2.742743in}{1.302174in}}%
\pgfpathlineto{\pgfqpoint{2.735461in}{1.130622in}}%
\pgfpathlineto{\pgfqpoint{2.729866in}{0.946534in}}%
\pgfpathlineto{\pgfqpoint{2.726320in}{0.757400in}}%
\pgfpathlineto{\pgfqpoint{2.725112in}{0.588141in}}%
\pgfpathlineto{\pgfqpoint{2.726263in}{0.486096in}}%
\pgfpathlineto{\pgfqpoint{2.728473in}{0.443865in}}%
\pgfpathlineto{\pgfqpoint{2.731163in}{0.426733in}}%
\pgfpathlineto{\pgfqpoint{2.734393in}{0.417521in}}%
\pgfpathlineto{\pgfqpoint{2.738636in}{0.411898in}}%
\pgfpathlineto{\pgfqpoint{2.744347in}{0.408349in}}%
\pgfpathlineto{\pgfqpoint{2.752795in}{0.406033in}}%
\pgfpathlineto{\pgfqpoint{2.770122in}{0.404272in}}%
\pgfpathlineto{\pgfqpoint{2.809265in}{0.403204in}}%
\pgfpathlineto{\pgfqpoint{2.944134in}{0.402698in}}%
\pgfpathlineto{\pgfqpoint{3.440109in}{0.402586in}}%
\pgfpathlineto{\pgfqpoint{3.440109in}{0.402586in}}%
\pgfusepath{stroke}%
\end{pgfscope}%
\begin{pgfscope}%
\pgfpathrectangle{\pgfqpoint{0.448634in}{0.402556in}}{\pgfqpoint{4.350661in}{2.489204in}} %
\pgfusepath{clip}%
\pgfsetrectcap%
\pgfsetroundjoin%
\pgfsetlinewidth{1.003750pt}%
\definecolor{currentstroke}{rgb}{0.498039,0.498039,0.498039}%
\pgfsetstrokecolor{currentstroke}%
\pgfsetdash{}{0pt}%
\pgfpathmoveto{\pgfqpoint{1.456290in}{2.570509in}}%
\pgfpathlineto{\pgfqpoint{1.500424in}{2.583986in}}%
\pgfpathlineto{\pgfqpoint{1.595731in}{2.609511in}}%
\pgfpathlineto{\pgfqpoint{1.649134in}{2.621263in}}%
\pgfpathlineto{\pgfqpoint{1.711450in}{2.632465in}}%
\pgfpathlineto{\pgfqpoint{1.771888in}{2.641103in}}%
\pgfpathlineto{\pgfqpoint{1.843352in}{2.648838in}}%
\pgfpathlineto{\pgfqpoint{1.906296in}{2.653615in}}%
\pgfpathlineto{\pgfqpoint{1.969326in}{2.656578in}}%
\pgfpathlineto{\pgfqpoint{2.043275in}{2.657955in}}%
\pgfpathlineto{\pgfqpoint{2.112879in}{2.657027in}}%
\pgfpathlineto{\pgfqpoint{2.175907in}{2.654057in}}%
\pgfpathlineto{\pgfqpoint{2.236665in}{2.649185in}}%
\pgfpathlineto{\pgfqpoint{2.292920in}{2.642529in}}%
\pgfpathlineto{\pgfqpoint{2.342460in}{2.634541in}}%
\pgfpathlineto{\pgfqpoint{2.387398in}{2.625168in}}%
\pgfpathlineto{\pgfqpoint{2.427673in}{2.614563in}}%
\pgfpathlineto{\pgfqpoint{2.463233in}{2.602967in}}%
\pgfpathlineto{\pgfqpoint{2.494051in}{2.590714in}}%
\pgfpathlineto{\pgfqpoint{2.522125in}{2.577225in}}%
\pgfpathlineto{\pgfqpoint{2.547351in}{2.562616in}}%
\pgfpathlineto{\pgfqpoint{2.569646in}{2.547094in}}%
\pgfpathlineto{\pgfqpoint{2.588983in}{2.530980in}}%
\pgfpathlineto{\pgfqpoint{2.605402in}{2.514664in}}%
\pgfpathlineto{\pgfqpoint{2.620490in}{2.496744in}}%
\pgfpathlineto{\pgfqpoint{2.634122in}{2.477357in}}%
\pgfpathlineto{\pgfqpoint{2.646138in}{2.456618in}}%
\pgfpathlineto{\pgfqpoint{2.657430in}{2.432489in}}%
\pgfpathlineto{\pgfqpoint{2.666759in}{2.407284in}}%
\pgfpathlineto{\pgfqpoint{2.674875in}{2.378903in}}%
\pgfpathlineto{\pgfqpoint{2.681549in}{2.347465in}}%
\pgfpathlineto{\pgfqpoint{2.686994in}{2.310657in}}%
\pgfpathlineto{\pgfqpoint{2.690884in}{2.268581in}}%
\pgfpathlineto{\pgfqpoint{2.693196in}{2.218872in}}%
\pgfpathlineto{\pgfqpoint{2.693755in}{2.156649in}}%
\pgfpathlineto{\pgfqpoint{2.692108in}{2.074530in}}%
\pgfpathlineto{\pgfqpoint{2.687195in}{1.952690in}}%
\pgfpathlineto{\pgfqpoint{2.675693in}{1.731543in}}%
\pgfpathlineto{\pgfqpoint{2.659329in}{1.455878in}}%
\pgfpathlineto{\pgfqpoint{2.646463in}{1.274765in}}%
\pgfpathlineto{\pgfqpoint{2.634965in}{1.143498in}}%
\pgfpathlineto{\pgfqpoint{2.623416in}{1.037284in}}%
\pgfpathlineto{\pgfqpoint{2.614300in}{0.968374in}}%
\pgfpathlineto{\pgfqpoint{2.601743in}{0.892564in}}%
\pgfpathlineto{\pgfqpoint{2.588939in}{0.832091in}}%
\pgfpathlineto{\pgfqpoint{2.576261in}{0.784475in}}%
\pgfpathlineto{\pgfqpoint{2.564061in}{0.747181in}}%
\pgfpathlineto{\pgfqpoint{2.552480in}{0.717666in}}%
\pgfpathlineto{\pgfqpoint{2.539131in}{0.689148in}}%
\pgfpathlineto{\pgfqpoint{2.525077in}{0.663987in}}%
\pgfpathlineto{\pgfqpoint{2.509186in}{0.640301in}}%
\pgfpathlineto{\pgfqpoint{2.491455in}{0.618393in}}%
\pgfpathlineto{\pgfqpoint{2.473669in}{0.600090in}}%
\pgfpathlineto{\pgfqpoint{2.454609in}{0.583550in}}%
\pgfpathlineto{\pgfqpoint{2.432590in}{0.567520in}}%
\pgfpathlineto{\pgfqpoint{2.407590in}{0.552413in}}%
\pgfpathlineto{\pgfqpoint{2.379697in}{0.538438in}}%
\pgfpathlineto{\pgfqpoint{2.349017in}{0.525743in}}%
\pgfpathlineto{\pgfqpoint{2.313573in}{0.513691in}}%
\pgfpathlineto{\pgfqpoint{2.275522in}{0.503136in}}%
\pgfpathlineto{\pgfqpoint{2.230698in}{0.493072in}}%
\pgfpathlineto{\pgfqpoint{2.179115in}{0.483876in}}%
\pgfpathlineto{\pgfqpoint{2.118652in}{0.475482in}}%
\pgfpathlineto{\pgfqpoint{2.053677in}{0.468527in}}%
\pgfpathlineto{\pgfqpoint{1.969049in}{0.461736in}}%
\pgfpathlineto{\pgfqpoint{1.862593in}{0.455624in}}%
\pgfpathlineto{\pgfqpoint{1.753892in}{0.451360in}}%
\pgfpathlineto{\pgfqpoint{1.621231in}{0.447966in}}%
\pgfpathlineto{\pgfqpoint{1.458092in}{0.445955in}}%
\pgfpathlineto{\pgfqpoint{1.351505in}{0.446817in}}%
\pgfpathlineto{\pgfqpoint{1.249303in}{0.449962in}}%
\pgfpathlineto{\pgfqpoint{1.158025in}{0.454475in}}%
\pgfpathlineto{\pgfqpoint{1.158025in}{0.454475in}}%
\pgfusepath{stroke}%
\end{pgfscope}%
\begin{pgfscope}%
\pgfpathrectangle{\pgfqpoint{0.448634in}{0.402556in}}{\pgfqpoint{4.350661in}{2.489204in}} %
\pgfusepath{clip}%
\pgfsetrectcap%
\pgfsetroundjoin%
\pgfsetlinewidth{1.003750pt}%
\definecolor{currentstroke}{rgb}{0.498039,0.498039,0.498039}%
\pgfsetstrokecolor{currentstroke}%
\pgfsetdash{}{0pt}%
\pgfpathmoveto{\pgfqpoint{1.484956in}{0.409126in}}%
\pgfpathlineto{\pgfqpoint{1.117334in}{0.406335in}}%
\pgfpathlineto{\pgfqpoint{0.754054in}{0.406061in}}%
\pgfpathlineto{\pgfqpoint{0.582211in}{0.407731in}}%
\pgfpathlineto{\pgfqpoint{0.516992in}{0.410233in}}%
\pgfpathlineto{\pgfqpoint{0.488858in}{0.413388in}}%
\pgfpathlineto{\pgfqpoint{0.474025in}{0.417238in}}%
\pgfpathlineto{\pgfqpoint{0.466140in}{0.421394in}}%
\pgfpathlineto{\pgfqpoint{0.461142in}{0.426159in}}%
\pgfpathlineto{\pgfqpoint{0.457496in}{0.432327in}}%
\pgfpathlineto{\pgfqpoint{0.454538in}{0.441669in}}%
\pgfpathlineto{\pgfqpoint{0.452058in}{0.458846in}}%
\pgfpathlineto{\pgfqpoint{0.450347in}{0.491140in}}%
\pgfpathlineto{\pgfqpoint{0.449297in}{0.565804in}}%
\pgfpathlineto{\pgfqpoint{0.448790in}{0.827170in}}%
\pgfpathlineto{\pgfqpoint{0.448819in}{2.860849in}}%
\pgfpathlineto{\pgfqpoint{0.449092in}{2.890711in}}%
\pgfpathlineto{\pgfqpoint{0.449092in}{2.890711in}}%
\pgfusepath{stroke}%
\end{pgfscope}%
\begin{pgfscope}%
\pgfpathrectangle{\pgfqpoint{0.448634in}{0.402556in}}{\pgfqpoint{4.350661in}{2.489204in}} %
\pgfusepath{clip}%
\pgfsetrectcap%
\pgfsetroundjoin%
\pgfsetlinewidth{1.003750pt}%
\definecolor{currentstroke}{rgb}{0.498039,0.498039,0.498039}%
\pgfsetstrokecolor{currentstroke}%
\pgfsetdash{}{0pt}%
\pgfpathmoveto{\pgfqpoint{1.617643in}{0.420688in}}%
\pgfpathlineto{\pgfqpoint{0.965061in}{0.416674in}}%
\pgfpathlineto{\pgfqpoint{0.767115in}{0.418748in}}%
\pgfpathlineto{\pgfqpoint{0.660559in}{0.421786in}}%
\pgfpathlineto{\pgfqpoint{0.597563in}{0.425535in}}%
\pgfpathlineto{\pgfqpoint{0.558600in}{0.429921in}}%
\pgfpathlineto{\pgfqpoint{0.532868in}{0.434897in}}%
\pgfpathlineto{\pgfqpoint{0.513975in}{0.440727in}}%
\pgfpathlineto{\pgfqpoint{0.499896in}{0.447329in}}%
\pgfpathlineto{\pgfqpoint{0.488755in}{0.455073in}}%
\pgfpathlineto{\pgfqpoint{0.480622in}{0.463312in}}%
\pgfpathlineto{\pgfqpoint{0.473943in}{0.473113in}}%
\pgfpathlineto{\pgfqpoint{0.467952in}{0.486359in}}%
\pgfpathlineto{\pgfqpoint{0.463196in}{0.502897in}}%
\pgfpathlineto{\pgfqpoint{0.459317in}{0.524847in}}%
\pgfpathlineto{\pgfqpoint{0.456058in}{0.556984in}}%
\pgfpathlineto{\pgfqpoint{0.453456in}{0.606675in}}%
\pgfpathlineto{\pgfqpoint{0.451440in}{0.693765in}}%
\pgfpathlineto{\pgfqpoint{0.450045in}{0.868001in}}%
\pgfpathlineto{\pgfqpoint{0.449215in}{1.318545in}}%
\pgfpathlineto{\pgfqpoint{0.449361in}{2.705032in}}%
\pgfpathlineto{\pgfqpoint{0.450779in}{2.866813in}}%
\pgfpathlineto{\pgfqpoint{0.452765in}{2.884040in}}%
\pgfpathlineto{\pgfqpoint{0.455085in}{2.888178in}}%
\pgfpathlineto{\pgfqpoint{0.459074in}{2.890062in}}%
\pgfpathlineto{\pgfqpoint{0.469889in}{2.891235in}}%
\pgfpathlineto{\pgfqpoint{0.509041in}{2.891685in}}%
\pgfpathlineto{\pgfqpoint{1.037647in}{2.891759in}}%
\pgfpathlineto{\pgfqpoint{4.792237in}{2.890787in}}%
\pgfpathlineto{\pgfqpoint{4.796132in}{2.888858in}}%
\pgfpathlineto{\pgfqpoint{4.797587in}{2.881650in}}%
\pgfpathlineto{\pgfqpoint{4.798359in}{2.854288in}}%
\pgfpathlineto{\pgfqpoint{4.798386in}{2.851799in}}%
\pgfpathlineto{\pgfqpoint{4.798386in}{2.851799in}}%
\pgfusepath{stroke}%
\end{pgfscope}%
\begin{pgfscope}%
\pgfpathrectangle{\pgfqpoint{0.448634in}{0.402556in}}{\pgfqpoint{4.350661in}{2.489204in}} %
\pgfusepath{clip}%
\pgfsetrectcap%
\pgfsetroundjoin%
\pgfsetlinewidth{1.003750pt}%
\definecolor{currentstroke}{rgb}{0.498039,0.498039,0.498039}%
\pgfsetstrokecolor{currentstroke}%
\pgfsetdash{}{0pt}%
\pgfpathmoveto{\pgfqpoint{3.444008in}{0.402626in}}%
\pgfpathlineto{\pgfqpoint{2.821868in}{0.403784in}}%
\pgfpathlineto{\pgfqpoint{2.778400in}{0.405702in}}%
\pgfpathlineto{\pgfqpoint{2.761146in}{0.408217in}}%
\pgfpathlineto{\pgfqpoint{2.750738in}{0.411749in}}%
\pgfpathlineto{\pgfqpoint{2.745085in}{0.415448in}}%
\pgfpathlineto{\pgfqpoint{2.740462in}{0.420685in}}%
\pgfpathlineto{\pgfqpoint{2.736251in}{0.429363in}}%
\pgfpathlineto{\pgfqpoint{2.733137in}{0.441274in}}%
\pgfpathlineto{\pgfqpoint{2.730444in}{0.460938in}}%
\pgfpathlineto{\pgfqpoint{2.728260in}{0.495691in}}%
\pgfpathlineto{\pgfqpoint{2.726755in}{0.560385in}}%
\pgfpathlineto{\pgfqpoint{2.726534in}{0.674886in}}%
\pgfpathlineto{\pgfqpoint{2.728526in}{0.846625in}}%
\pgfpathlineto{\pgfqpoint{2.732920in}{1.033246in}}%
\pgfpathlineto{\pgfqpoint{2.739274in}{1.212320in}}%
\pgfpathlineto{\pgfqpoint{2.747343in}{1.378839in}}%
\pgfpathlineto{\pgfqpoint{2.756854in}{1.530288in}}%
\pgfpathlineto{\pgfqpoint{2.767594in}{1.666639in}}%
\pgfpathlineto{\pgfqpoint{2.779291in}{1.787871in}}%
\pgfpathlineto{\pgfqpoint{2.792189in}{1.898907in}}%
\pgfpathlineto{\pgfqpoint{2.806082in}{1.999715in}}%
\pgfpathlineto{\pgfqpoint{2.824425in}{2.114784in}}%
\pgfpathlineto{\pgfqpoint{2.841592in}{2.204763in}}%
\pgfpathlineto{\pgfqpoint{2.863794in}{2.308738in}}%
\pgfpathlineto{\pgfqpoint{2.892703in}{2.444142in}}%
\pgfpathlineto{\pgfqpoint{2.900154in}{2.490654in}}%
\pgfpathlineto{\pgfqpoint{2.903475in}{2.525285in}}%
\pgfpathlineto{\pgfqpoint{2.903918in}{2.552651in}}%
\pgfpathlineto{\pgfqpoint{2.902216in}{2.574957in}}%
\pgfpathlineto{\pgfqpoint{2.898655in}{2.594436in}}%
\pgfpathlineto{\pgfqpoint{2.893652in}{2.610879in}}%
\pgfpathlineto{\pgfqpoint{2.886760in}{2.626401in}}%
\pgfpathlineto{\pgfqpoint{2.878049in}{2.640676in}}%
\pgfpathlineto{\pgfqpoint{2.867759in}{2.653505in}}%
\pgfpathlineto{\pgfqpoint{2.854491in}{2.666371in}}%
\pgfpathlineto{\pgfqpoint{2.838188in}{2.678754in}}%
\pgfpathlineto{\pgfqpoint{2.818937in}{2.690326in}}%
\pgfpathlineto{\pgfqpoint{2.796888in}{2.700947in}}%
\pgfpathlineto{\pgfqpoint{2.770107in}{2.711321in}}%
\pgfpathlineto{\pgfqpoint{2.738621in}{2.721102in}}%
\pgfpathlineto{\pgfqpoint{2.700354in}{2.730572in}}%
\pgfpathlineto{\pgfqpoint{2.655325in}{2.739357in}}%
\pgfpathlineto{\pgfqpoint{2.601414in}{2.747522in}}%
\pgfpathlineto{\pgfqpoint{2.536480in}{2.754950in}}%
\pgfpathlineto{\pgfqpoint{2.462717in}{2.761111in}}%
\pgfpathlineto{\pgfqpoint{2.377985in}{2.765930in}}%
\pgfpathlineto{\pgfqpoint{2.286662in}{2.768974in}}%
\pgfpathlineto{\pgfqpoint{2.175730in}{2.770495in}}%
\pgfpathlineto{\pgfqpoint{2.064792in}{2.769686in}}%
\pgfpathlineto{\pgfqpoint{1.964756in}{2.766970in}}%
\pgfpathlineto{\pgfqpoint{1.858258in}{2.761929in}}%
\pgfpathlineto{\pgfqpoint{1.760553in}{2.755087in}}%
\pgfpathlineto{\pgfqpoint{1.678169in}{2.747350in}}%
\pgfpathlineto{\pgfqpoint{1.595960in}{2.737479in}}%
\pgfpathlineto{\pgfqpoint{1.511844in}{2.724937in}}%
\pgfpathlineto{\pgfqpoint{1.443186in}{2.711825in}}%
\pgfpathlineto{\pgfqpoint{1.385614in}{2.698542in}}%
\pgfpathlineto{\pgfqpoint{1.334820in}{2.684747in}}%
\pgfpathlineto{\pgfqpoint{1.284552in}{2.668631in}}%
\pgfpathlineto{\pgfqpoint{1.243209in}{2.653139in}}%
\pgfpathlineto{\pgfqpoint{1.233325in}{2.648030in}}%
\pgfpathlineto{\pgfqpoint{1.186847in}{2.626861in}}%
\pgfpathlineto{\pgfqpoint{1.153302in}{2.609058in}}%
\pgfpathlineto{\pgfqpoint{1.120651in}{2.589204in}}%
\pgfpathlineto{\pgfqpoint{1.090926in}{2.568496in}}%
\pgfpathlineto{\pgfqpoint{1.062318in}{2.545820in}}%
\pgfpathlineto{\pgfqpoint{1.036757in}{2.522621in}}%
\pgfpathlineto{\pgfqpoint{1.012571in}{2.497568in}}%
\pgfpathlineto{\pgfqpoint{0.989886in}{2.470739in}}%
\pgfpathlineto{\pgfqpoint{0.968768in}{2.442284in}}%
\pgfpathlineto{\pgfqpoint{0.949283in}{2.412341in}}%
\pgfpathlineto{\pgfqpoint{0.931437in}{2.381089in}}%
\pgfpathlineto{\pgfqpoint{0.901933in}{2.317318in}}%
\pgfpathlineto{\pgfqpoint{0.900069in}{2.312819in}}%
\pgfpathlineto{\pgfqpoint{0.900069in}{2.312819in}}%
\pgfusepath{stroke}%
\end{pgfscope}%
\begin{pgfscope}%
\pgfpathrectangle{\pgfqpoint{0.448634in}{0.402556in}}{\pgfqpoint{4.350661in}{2.489204in}} %
\pgfusepath{clip}%
\pgfsetrectcap%
\pgfsetroundjoin%
\pgfsetlinewidth{1.003750pt}%
\definecolor{currentstroke}{rgb}{0.737255,0.741176,0.133333}%
\pgfsetstrokecolor{currentstroke}%
\pgfsetdash{}{0pt}%
\pgfpathmoveto{\pgfqpoint{3.537684in}{0.511407in}}%
\pgfpathlineto{\pgfqpoint{3.583212in}{0.507177in}}%
\pgfpathlineto{\pgfqpoint{3.631032in}{0.505095in}}%
\pgfpathlineto{\pgfqpoint{3.731089in}{0.503932in}}%
\pgfpathlineto{\pgfqpoint{3.776752in}{0.502517in}}%
\pgfpathlineto{\pgfqpoint{3.876666in}{0.496259in}}%
\pgfpathlineto{\pgfqpoint{4.002650in}{0.488479in}}%
\pgfpathlineto{\pgfqpoint{4.091795in}{0.485483in}}%
\pgfpathlineto{\pgfqpoint{4.143996in}{0.486184in}}%
\pgfpathlineto{\pgfqpoint{4.183085in}{0.488659in}}%
\pgfpathlineto{\pgfqpoint{4.208953in}{0.492634in}}%
\pgfpathlineto{\pgfqpoint{4.228035in}{0.497609in}}%
\pgfpathlineto{\pgfqpoint{4.248706in}{0.505343in}}%
\pgfpathlineto{\pgfqpoint{4.302190in}{0.526368in}}%
\pgfpathlineto{\pgfqpoint{4.422493in}{0.569843in}}%
\pgfpathlineto{\pgfqpoint{4.465270in}{0.588158in}}%
\pgfpathlineto{\pgfqpoint{4.488825in}{0.601000in}}%
\pgfpathlineto{\pgfqpoint{4.501782in}{0.610131in}}%
\pgfpathlineto{\pgfqpoint{4.511897in}{0.619546in}}%
\pgfpathlineto{\pgfqpoint{4.519077in}{0.628874in}}%
\pgfpathlineto{\pgfqpoint{4.524603in}{0.639570in}}%
\pgfpathlineto{\pgfqpoint{4.528334in}{0.651243in}}%
\pgfpathlineto{\pgfqpoint{4.532482in}{0.673127in}}%
\pgfpathlineto{\pgfqpoint{4.536809in}{0.692404in}}%
\pgfpathlineto{\pgfqpoint{4.541879in}{0.706154in}}%
\pgfpathlineto{\pgfqpoint{4.549927in}{0.720927in}}%
\pgfpathlineto{\pgfqpoint{4.561204in}{0.736075in}}%
\pgfpathlineto{\pgfqpoint{4.577102in}{0.753050in}}%
\pgfpathlineto{\pgfqpoint{4.599172in}{0.773274in}}%
\pgfpathlineto{\pgfqpoint{4.649993in}{0.820058in}}%
\pgfpathlineto{\pgfqpoint{4.663670in}{0.836069in}}%
\pgfpathlineto{\pgfqpoint{4.673094in}{0.849742in}}%
\pgfpathlineto{\pgfqpoint{4.679971in}{0.862423in}}%
\pgfpathlineto{\pgfqpoint{4.686963in}{0.880633in}}%
\pgfpathlineto{\pgfqpoint{4.690689in}{0.897516in}}%
\pgfpathlineto{\pgfqpoint{4.692727in}{0.914769in}}%
\pgfpathlineto{\pgfqpoint{4.697695in}{0.984141in}}%
\pgfpathlineto{\pgfqpoint{4.701881in}{1.008560in}}%
\pgfpathlineto{\pgfqpoint{4.710561in}{1.052246in}}%
\pgfpathlineto{\pgfqpoint{4.721268in}{1.108163in}}%
\pgfpathlineto{\pgfqpoint{4.734735in}{1.203948in}}%
\pgfpathlineto{\pgfqpoint{4.742278in}{1.278073in}}%
\pgfpathlineto{\pgfqpoint{4.744175in}{1.305345in}}%
\pgfpathlineto{\pgfqpoint{4.746319in}{1.335102in}}%
\pgfpathlineto{\pgfqpoint{4.747246in}{1.359960in}}%
\pgfpathlineto{\pgfqpoint{4.750989in}{1.391889in}}%
\pgfpathlineto{\pgfqpoint{4.754621in}{1.458920in}}%
\pgfpathlineto{\pgfqpoint{4.755333in}{1.488768in}}%
\pgfpathlineto{\pgfqpoint{4.755796in}{1.518623in}}%
\pgfpathlineto{\pgfqpoint{4.758209in}{1.583226in}}%
\pgfpathlineto{\pgfqpoint{4.756778in}{1.637830in}}%
\pgfpathlineto{\pgfqpoint{4.758816in}{1.665102in}}%
\pgfpathlineto{\pgfqpoint{4.759173in}{1.697431in}}%
\pgfpathlineto{\pgfqpoint{4.755686in}{1.769432in}}%
\pgfpathlineto{\pgfqpoint{4.761112in}{1.853771in}}%
\pgfpathlineto{\pgfqpoint{4.759553in}{1.935805in}}%
\pgfpathlineto{\pgfqpoint{4.765373in}{2.032638in}}%
\pgfpathlineto{\pgfqpoint{4.765028in}{2.067477in}}%
\pgfpathlineto{\pgfqpoint{4.761804in}{2.186860in}}%
\pgfpathlineto{\pgfqpoint{4.764649in}{2.373519in}}%
\pgfpathlineto{\pgfqpoint{4.767780in}{2.433147in}}%
\pgfpathlineto{\pgfqpoint{4.769536in}{2.472913in}}%
\pgfpathlineto{\pgfqpoint{4.768655in}{2.500265in}}%
\pgfpathlineto{\pgfqpoint{4.764966in}{2.532343in}}%
\pgfpathlineto{\pgfqpoint{4.762667in}{2.559584in}}%
\pgfpathlineto{\pgfqpoint{4.763183in}{2.579474in}}%
\pgfpathlineto{\pgfqpoint{4.763398in}{2.596872in}}%
\pgfpathlineto{\pgfqpoint{4.760038in}{2.628984in}}%
\pgfpathlineto{\pgfqpoint{4.758809in}{2.643832in}}%
\pgfpathlineto{\pgfqpoint{4.757879in}{2.653707in}}%
\pgfpathlineto{\pgfqpoint{4.749499in}{2.687112in}}%
\pgfpathlineto{\pgfqpoint{4.745110in}{2.698462in}}%
\pgfpathlineto{\pgfqpoint{4.739053in}{2.711666in}}%
\pgfpathlineto{\pgfqpoint{4.726811in}{2.735154in}}%
\pgfpathlineto{\pgfqpoint{4.716679in}{2.754293in}}%
\pgfpathlineto{\pgfqpoint{4.692518in}{2.795292in}}%
\pgfpathlineto{\pgfqpoint{4.689329in}{2.801754in}}%
\pgfpathlineto{\pgfqpoint{4.676839in}{2.815482in}}%
\pgfpathlineto{\pgfqpoint{4.665442in}{2.822654in}}%
\pgfpathlineto{\pgfqpoint{4.653132in}{2.827575in}}%
\pgfpathlineto{\pgfqpoint{4.631960in}{2.833267in}}%
\pgfpathlineto{\pgfqpoint{4.599985in}{2.840682in}}%
\pgfpathlineto{\pgfqpoint{4.494016in}{2.868580in}}%
\pgfpathlineto{\pgfqpoint{4.459603in}{2.874456in}}%
\pgfpathlineto{\pgfqpoint{4.424995in}{2.878499in}}%
\pgfpathlineto{\pgfqpoint{4.346821in}{2.883332in}}%
\pgfpathlineto{\pgfqpoint{4.235886in}{2.883859in}}%
\pgfpathlineto{\pgfqpoint{3.955317in}{2.888348in}}%
\pgfpathlineto{\pgfqpoint{3.877008in}{2.889008in}}%
\pgfpathlineto{\pgfqpoint{3.759541in}{2.889150in}}%
\pgfpathlineto{\pgfqpoint{2.391259in}{2.891039in}}%
\pgfpathlineto{\pgfqpoint{1.740837in}{2.889979in}}%
\pgfpathlineto{\pgfqpoint{1.223119in}{2.886871in}}%
\pgfpathlineto{\pgfqpoint{1.079572in}{2.883999in}}%
\pgfpathlineto{\pgfqpoint{0.955676in}{2.878552in}}%
\pgfpathlineto{\pgfqpoint{0.868899in}{2.871328in}}%
\pgfpathlineto{\pgfqpoint{0.823513in}{2.865398in}}%
\pgfpathlineto{\pgfqpoint{0.776256in}{2.856794in}}%
\pgfpathlineto{\pgfqpoint{0.746473in}{2.849538in}}%
\pgfpathlineto{\pgfqpoint{0.715103in}{2.839301in}}%
\pgfpathlineto{\pgfqpoint{0.692765in}{2.829511in}}%
\pgfpathlineto{\pgfqpoint{0.673331in}{2.818355in}}%
\pgfpathlineto{\pgfqpoint{0.658707in}{2.807576in}}%
\pgfpathlineto{\pgfqpoint{0.645228in}{2.794999in}}%
\pgfpathlineto{\pgfqpoint{0.633119in}{2.780713in}}%
\pgfpathlineto{\pgfqpoint{0.621263in}{2.762902in}}%
\pgfpathlineto{\pgfqpoint{0.611250in}{2.743662in}}%
\pgfpathlineto{\pgfqpoint{0.602011in}{2.721137in}}%
\pgfpathlineto{\pgfqpoint{0.593045in}{2.693094in}}%
\pgfpathlineto{\pgfqpoint{0.584805in}{2.659552in}}%
\pgfpathlineto{\pgfqpoint{0.576736in}{2.615715in}}%
\pgfpathlineto{\pgfqpoint{0.570115in}{2.566516in}}%
\pgfpathlineto{\pgfqpoint{0.563885in}{2.502195in}}%
\pgfpathlineto{\pgfqpoint{0.559452in}{2.435181in}}%
\pgfpathlineto{\pgfqpoint{0.555991in}{2.355630in}}%
\pgfpathlineto{\pgfqpoint{0.552846in}{2.218772in}}%
\pgfpathlineto{\pgfqpoint{0.552527in}{2.109250in}}%
\pgfpathlineto{\pgfqpoint{0.554236in}{2.037095in}}%
\pgfpathlineto{\pgfqpoint{0.562902in}{1.855681in}}%
\pgfpathlineto{\pgfqpoint{0.575778in}{1.719599in}}%
\pgfpathlineto{\pgfqpoint{0.590101in}{1.628999in}}%
\pgfpathlineto{\pgfqpoint{0.600189in}{1.585725in}}%
\pgfpathlineto{\pgfqpoint{0.607717in}{1.559742in}}%
\pgfpathlineto{\pgfqpoint{0.614889in}{1.541614in}}%
\pgfpathlineto{\pgfqpoint{0.621834in}{1.528993in}}%
\pgfpathlineto{\pgfqpoint{0.627847in}{1.521827in}}%
\pgfpathlineto{\pgfqpoint{0.633476in}{1.518091in}}%
\pgfpathlineto{\pgfqpoint{0.639766in}{1.516194in}}%
\pgfpathlineto{\pgfqpoint{0.650630in}{1.515944in}}%
\pgfpathlineto{\pgfqpoint{0.665825in}{1.515210in}}%
\pgfpathlineto{\pgfqpoint{0.680687in}{1.511451in}}%
\pgfpathlineto{\pgfqpoint{0.712359in}{1.502711in}}%
\pgfpathlineto{\pgfqpoint{0.723218in}{1.502778in}}%
\pgfpathlineto{\pgfqpoint{0.731676in}{1.505022in}}%
\pgfpathlineto{\pgfqpoint{0.739416in}{1.509520in}}%
\pgfpathlineto{\pgfqpoint{0.746040in}{1.515941in}}%
\pgfpathlineto{\pgfqpoint{0.752821in}{1.525654in}}%
\pgfpathlineto{\pgfqpoint{0.759064in}{1.538754in}}%
\pgfpathlineto{\pgfqpoint{0.765378in}{1.557298in}}%
\pgfpathlineto{\pgfqpoint{0.772683in}{1.588548in}}%
\pgfpathlineto{\pgfqpoint{0.778829in}{1.625212in}}%
\pgfpathlineto{\pgfqpoint{0.787956in}{1.701657in}}%
\pgfpathlineto{\pgfqpoint{0.803732in}{1.884966in}}%
\pgfpathlineto{\pgfqpoint{0.815577in}{2.006162in}}%
\pgfpathlineto{\pgfqpoint{0.824264in}{2.070109in}}%
\pgfpathlineto{\pgfqpoint{0.832833in}{2.121451in}}%
\pgfpathlineto{\pgfqpoint{0.841853in}{2.167598in}}%
\pgfpathlineto{\pgfqpoint{0.852903in}{2.213155in}}%
\pgfpathlineto{\pgfqpoint{0.855512in}{2.219944in}}%
\pgfpathlineto{\pgfqpoint{0.857468in}{2.224346in}}%
\pgfpathlineto{\pgfqpoint{0.860827in}{2.233504in}}%
\pgfpathlineto{\pgfqpoint{0.868254in}{2.251423in}}%
\pgfpathlineto{\pgfqpoint{0.870124in}{2.258538in}}%
\pgfpathlineto{\pgfqpoint{0.879268in}{2.272305in}}%
\pgfpathlineto{\pgfqpoint{0.886198in}{2.293135in}}%
\pgfpathlineto{\pgfqpoint{0.888505in}{2.300025in}}%
\pgfpathlineto{\pgfqpoint{0.887044in}{2.306998in}}%
\pgfpathlineto{\pgfqpoint{0.897056in}{2.331820in}}%
\pgfpathlineto{\pgfqpoint{0.900659in}{2.340848in}}%
\pgfpathlineto{\pgfqpoint{0.908303in}{2.355844in}}%
\pgfpathlineto{\pgfqpoint{0.915685in}{2.368145in}}%
\pgfpathlineto{\pgfqpoint{0.920492in}{2.379124in}}%
\pgfpathlineto{\pgfqpoint{0.922252in}{2.383630in}}%
\pgfpathlineto{\pgfqpoint{0.937527in}{2.407806in}}%
\pgfpathlineto{\pgfqpoint{0.943841in}{2.414487in}}%
\pgfpathlineto{\pgfqpoint{0.952201in}{2.425928in}}%
\pgfpathlineto{\pgfqpoint{0.957301in}{2.430492in}}%
\pgfpathlineto{\pgfqpoint{0.966499in}{2.444358in}}%
\pgfpathlineto{\pgfqpoint{0.978527in}{2.458732in}}%
\pgfpathlineto{\pgfqpoint{0.984455in}{2.465701in}}%
\pgfpathlineto{\pgfqpoint{0.984570in}{2.470621in}}%
\pgfpathlineto{\pgfqpoint{0.989858in}{2.478484in}}%
\pgfpathlineto{\pgfqpoint{1.003344in}{2.494675in}}%
\pgfpathlineto{\pgfqpoint{1.017023in}{2.506974in}}%
\pgfpathlineto{\pgfqpoint{1.054472in}{2.541019in}}%
\pgfpathlineto{\pgfqpoint{1.058539in}{2.542559in}}%
\pgfpathlineto{\pgfqpoint{1.062546in}{2.544302in}}%
\pgfpathlineto{\pgfqpoint{1.096612in}{2.571041in}}%
\pgfpathlineto{\pgfqpoint{1.122333in}{2.589690in}}%
\pgfpathlineto{\pgfqpoint{1.166030in}{2.617493in}}%
\pgfpathlineto{\pgfqpoint{1.200809in}{2.637626in}}%
\pgfpathlineto{\pgfqpoint{1.203734in}{2.641213in}}%
\pgfpathlineto{\pgfqpoint{1.221854in}{2.649684in}}%
\pgfpathlineto{\pgfqpoint{1.252432in}{2.662689in}}%
\pgfpathlineto{\pgfqpoint{1.295873in}{2.678839in}}%
\pgfpathlineto{\pgfqpoint{1.327284in}{2.688936in}}%
\pgfpathlineto{\pgfqpoint{1.356819in}{2.697422in}}%
\pgfpathlineto{\pgfqpoint{1.422795in}{2.713359in}}%
\pgfpathlineto{\pgfqpoint{1.450646in}{2.718945in}}%
\pgfpathlineto{\pgfqpoint{1.510872in}{2.729316in}}%
\pgfpathlineto{\pgfqpoint{1.586415in}{2.740137in}}%
\pgfpathlineto{\pgfqpoint{1.681480in}{2.752114in}}%
\pgfpathlineto{\pgfqpoint{1.841960in}{2.766406in}}%
\pgfpathlineto{\pgfqpoint{1.876668in}{2.767484in}}%
\pgfpathlineto{\pgfqpoint{2.022218in}{2.775861in}}%
\pgfpathlineto{\pgfqpoint{2.083109in}{2.777393in}}%
\pgfpathlineto{\pgfqpoint{2.183169in}{2.777502in}}%
\pgfpathlineto{\pgfqpoint{2.272334in}{2.775279in}}%
\pgfpathlineto{\pgfqpoint{2.354736in}{2.770745in}}%
\pgfpathlineto{\pgfqpoint{2.378647in}{2.769972in}}%
\pgfpathlineto{\pgfqpoint{2.391141in}{2.770016in}}%
\pgfpathlineto{\pgfqpoint{2.395204in}{2.768346in}}%
\pgfpathlineto{\pgfqpoint{2.410348in}{2.766601in}}%
\pgfpathlineto{\pgfqpoint{2.460056in}{2.761162in}}%
\pgfpathlineto{\pgfqpoint{2.488256in}{2.758866in}}%
\pgfpathlineto{\pgfqpoint{2.494724in}{2.758779in}}%
\pgfpathlineto{\pgfqpoint{2.501103in}{2.757310in}}%
\pgfpathlineto{\pgfqpoint{2.518473in}{2.756257in}}%
\pgfpathlineto{\pgfqpoint{2.524787in}{2.757755in}}%
\pgfpathlineto{\pgfqpoint{2.528898in}{2.758796in}}%
\pgfpathlineto{\pgfqpoint{2.536703in}{2.754643in}}%
\pgfpathlineto{\pgfqpoint{2.547489in}{2.753201in}}%
\pgfpathlineto{\pgfqpoint{2.553941in}{2.754027in}}%
\pgfpathlineto{\pgfqpoint{2.555942in}{2.755003in}}%
\pgfpathlineto{\pgfqpoint{2.564427in}{2.748688in}}%
\pgfpathlineto{\pgfqpoint{2.573080in}{2.747768in}}%
\pgfpathlineto{\pgfqpoint{2.581556in}{2.747699in}}%
\pgfpathlineto{\pgfqpoint{2.584806in}{2.744498in}}%
\pgfpathlineto{\pgfqpoint{2.591164in}{2.742924in}}%
\pgfpathlineto{\pgfqpoint{2.599827in}{2.743099in}}%
\pgfpathlineto{\pgfqpoint{2.601978in}{2.743030in}}%
\pgfpathlineto{\pgfqpoint{2.609311in}{2.738194in}}%
\pgfpathlineto{\pgfqpoint{2.617971in}{2.737578in}}%
\pgfpathlineto{\pgfqpoint{2.626406in}{2.739181in}}%
\pgfpathlineto{\pgfqpoint{2.656125in}{2.731593in}}%
\pgfpathlineto{\pgfqpoint{2.689698in}{2.721111in}}%
\pgfpathlineto{\pgfqpoint{2.716489in}{2.710794in}}%
\pgfpathlineto{\pgfqpoint{2.728655in}{2.705412in}}%
\pgfpathlineto{\pgfqpoint{2.761044in}{2.690842in}}%
\pgfpathlineto{\pgfqpoint{2.782841in}{2.679553in}}%
\pgfpathlineto{\pgfqpoint{2.811895in}{2.662568in}}%
\pgfpathlineto{\pgfqpoint{2.838145in}{2.644916in}}%
\pgfpathlineto{\pgfqpoint{2.861441in}{2.626585in}}%
\pgfpathlineto{\pgfqpoint{2.879876in}{2.609143in}}%
\pgfpathlineto{\pgfqpoint{2.893671in}{2.593262in}}%
\pgfpathlineto{\pgfqpoint{2.904279in}{2.577496in}}%
\pgfpathlineto{\pgfqpoint{2.912829in}{2.560172in}}%
\pgfpathlineto{\pgfqpoint{2.918211in}{2.543889in}}%
\pgfpathlineto{\pgfqpoint{2.921451in}{2.526879in}}%
\pgfpathlineto{\pgfqpoint{2.922641in}{2.509523in}}%
\pgfpathlineto{\pgfqpoint{2.921712in}{2.487157in}}%
\pgfpathlineto{\pgfqpoint{2.917998in}{2.462640in}}%
\pgfpathlineto{\pgfqpoint{2.910776in}{2.431361in}}%
\pgfpathlineto{\pgfqpoint{2.898432in}{2.388846in}}%
\pgfpathlineto{\pgfqpoint{2.863793in}{2.273729in}}%
\pgfpathlineto{\pgfqpoint{2.862831in}{2.266414in}}%
\pgfpathlineto{\pgfqpoint{2.848153in}{2.209089in}}%
\pgfpathlineto{\pgfqpoint{2.829600in}{2.127176in}}%
\pgfpathlineto{\pgfqpoint{2.817056in}{2.056435in}}%
\pgfpathlineto{\pgfqpoint{2.805952in}{1.982852in}}%
\pgfpathlineto{\pgfqpoint{2.792755in}{1.879407in}}%
\pgfpathlineto{\pgfqpoint{2.786410in}{1.820110in}}%
\pgfpathlineto{\pgfqpoint{2.780142in}{1.745787in}}%
\pgfpathlineto{\pgfqpoint{2.779085in}{1.710974in}}%
\pgfpathlineto{\pgfqpoint{2.781001in}{1.686186in}}%
\pgfpathlineto{\pgfqpoint{2.783704in}{1.648985in}}%
\pgfpathlineto{\pgfqpoint{2.784312in}{1.606679in}}%
\pgfpathlineto{\pgfqpoint{2.782785in}{1.534516in}}%
\pgfpathlineto{\pgfqpoint{2.776293in}{1.367906in}}%
\pgfpathlineto{\pgfqpoint{2.761933in}{1.012336in}}%
\pgfpathlineto{\pgfqpoint{2.757120in}{0.838180in}}%
\pgfpathlineto{\pgfqpoint{2.754282in}{0.663968in}}%
\pgfpathlineto{\pgfqpoint{2.754169in}{0.534532in}}%
\pgfpathlineto{\pgfqpoint{2.755893in}{0.474830in}}%
\pgfpathlineto{\pgfqpoint{2.758550in}{0.447634in}}%
\pgfpathlineto{\pgfqpoint{2.761359in}{0.435626in}}%
\pgfpathlineto{\pgfqpoint{2.764535in}{0.429139in}}%
\pgfpathlineto{\pgfqpoint{2.767896in}{0.426069in}}%
\pgfpathlineto{\pgfqpoint{2.772133in}{0.426007in}}%
\pgfpathlineto{\pgfqpoint{2.775184in}{0.429449in}}%
\pgfpathlineto{\pgfqpoint{2.777581in}{0.436364in}}%
\pgfpathlineto{\pgfqpoint{2.780025in}{0.453555in}}%
\pgfpathlineto{\pgfqpoint{2.784653in}{0.482933in}}%
\pgfpathlineto{\pgfqpoint{2.788398in}{0.494589in}}%
\pgfpathlineto{\pgfqpoint{2.791903in}{0.500865in}}%
\pgfpathlineto{\pgfqpoint{2.797113in}{0.505249in}}%
\pgfpathlineto{\pgfqpoint{2.801358in}{0.506212in}}%
\pgfpathlineto{\pgfqpoint{2.807746in}{0.504914in}}%
\pgfpathlineto{\pgfqpoint{2.817452in}{0.499338in}}%
\pgfpathlineto{\pgfqpoint{2.835102in}{0.484795in}}%
\pgfpathlineto{\pgfqpoint{2.854878in}{0.469400in}}%
\pgfpathlineto{\pgfqpoint{2.870287in}{0.460166in}}%
\pgfpathlineto{\pgfqpoint{2.886460in}{0.452838in}}%
\pgfpathlineto{\pgfqpoint{2.909494in}{0.445457in}}%
\pgfpathlineto{\pgfqpoint{2.935165in}{0.440082in}}%
\pgfpathlineto{\pgfqpoint{3.001984in}{0.429700in}}%
\pgfpathlineto{\pgfqpoint{3.086093in}{0.417061in}}%
\pgfpathlineto{\pgfqpoint{3.129458in}{0.413087in}}%
\pgfpathlineto{\pgfqpoint{3.185954in}{0.410106in}}%
\pgfpathlineto{\pgfqpoint{3.305564in}{0.407001in}}%
\pgfpathlineto{\pgfqpoint{3.516566in}{0.405704in}}%
\pgfpathlineto{\pgfqpoint{3.538254in}{0.407262in}}%
\pgfpathlineto{\pgfqpoint{3.542321in}{0.408949in}}%
\pgfpathlineto{\pgfqpoint{3.546007in}{0.411430in}}%
\pgfpathlineto{\pgfqpoint{3.554696in}{0.411350in}}%
\pgfpathlineto{\pgfqpoint{3.569795in}{0.413513in}}%
\pgfpathlineto{\pgfqpoint{3.595666in}{0.417418in}}%
\pgfpathlineto{\pgfqpoint{3.656466in}{0.421506in}}%
\pgfpathlineto{\pgfqpoint{3.758672in}{0.424509in}}%
\pgfpathlineto{\pgfqpoint{3.882641in}{0.426849in}}%
\pgfpathlineto{\pgfqpoint{4.707081in}{0.428622in}}%
\pgfpathlineto{\pgfqpoint{4.746192in}{0.430613in}}%
\pgfpathlineto{\pgfqpoint{4.765600in}{0.433486in}}%
\pgfpathlineto{\pgfqpoint{4.773781in}{0.436750in}}%
\pgfpathlineto{\pgfqpoint{4.777025in}{0.439979in}}%
\pgfpathlineto{\pgfqpoint{4.777427in}{0.444758in}}%
\pgfpathlineto{\pgfqpoint{4.774258in}{0.448003in}}%
\pgfpathlineto{\pgfqpoint{4.765944in}{0.450843in}}%
\pgfpathlineto{\pgfqpoint{4.734081in}{0.458601in}}%
\pgfpathlineto{\pgfqpoint{4.726282in}{0.462979in}}%
\pgfpathlineto{\pgfqpoint{4.719609in}{0.469320in}}%
\pgfpathlineto{\pgfqpoint{4.714915in}{0.477640in}}%
\pgfpathlineto{\pgfqpoint{4.713013in}{0.487305in}}%
\pgfpathlineto{\pgfqpoint{4.713741in}{0.494699in}}%
\pgfpathlineto{\pgfqpoint{4.716924in}{0.503946in}}%
\pgfpathlineto{\pgfqpoint{4.726995in}{0.523140in}}%
\pgfpathlineto{\pgfqpoint{4.740651in}{0.551432in}}%
\pgfpathlineto{\pgfqpoint{4.748349in}{0.577326in}}%
\pgfpathlineto{\pgfqpoint{4.750321in}{0.589554in}}%
\pgfpathlineto{\pgfqpoint{4.754208in}{0.629086in}}%
\pgfpathlineto{\pgfqpoint{4.755260in}{0.663905in}}%
\pgfpathlineto{\pgfqpoint{4.758594in}{0.688494in}}%
\pgfpathlineto{\pgfqpoint{4.761924in}{0.708025in}}%
\pgfpathlineto{\pgfqpoint{4.778633in}{0.787884in}}%
\pgfpathlineto{\pgfqpoint{4.782185in}{0.807361in}}%
\pgfpathlineto{\pgfqpoint{4.785773in}{0.821707in}}%
\pgfpathlineto{\pgfqpoint{4.787091in}{0.828994in}}%
\pgfpathlineto{\pgfqpoint{4.787667in}{0.851284in}}%
\pgfpathlineto{\pgfqpoint{4.793249in}{0.861579in}}%
\pgfpathlineto{\pgfqpoint{4.794726in}{0.878913in}}%
\pgfpathlineto{\pgfqpoint{4.796475in}{0.923657in}}%
\pgfpathlineto{\pgfqpoint{4.795123in}{0.938495in}}%
\pgfpathlineto{\pgfqpoint{4.792043in}{0.960605in}}%
\pgfpathlineto{\pgfqpoint{4.789610in}{1.002828in}}%
\pgfpathlineto{\pgfqpoint{4.788060in}{1.052555in}}%
\pgfpathlineto{\pgfqpoint{4.788987in}{1.144545in}}%
\pgfpathlineto{\pgfqpoint{4.788541in}{1.169423in}}%
\pgfpathlineto{\pgfqpoint{4.790134in}{1.199223in}}%
\pgfpathlineto{\pgfqpoint{4.793724in}{1.246327in}}%
\pgfpathlineto{\pgfqpoint{4.794861in}{1.301071in}}%
\pgfpathlineto{\pgfqpoint{4.795070in}{1.415548in}}%
\pgfpathlineto{\pgfqpoint{4.793987in}{1.427852in}}%
\pgfpathlineto{\pgfqpoint{4.797582in}{1.433802in}}%
\pgfpathlineto{\pgfqpoint{4.798377in}{1.453688in}}%
\pgfpathlineto{\pgfqpoint{4.798908in}{1.528359in}}%
\pgfpathlineto{\pgfqpoint{4.799166in}{1.869379in}}%
\pgfpathlineto{\pgfqpoint{4.797750in}{2.451842in}}%
\pgfpathlineto{\pgfqpoint{4.795262in}{2.509016in}}%
\pgfpathlineto{\pgfqpoint{4.788440in}{2.615716in}}%
\pgfpathlineto{\pgfqpoint{4.788808in}{2.680427in}}%
\pgfpathlineto{\pgfqpoint{4.790423in}{2.735051in}}%
\pgfpathlineto{\pgfqpoint{4.788475in}{2.777278in}}%
\pgfpathlineto{\pgfqpoint{4.788280in}{2.881492in}}%
\pgfpathlineto{\pgfqpoint{4.787733in}{2.883752in}}%
\pgfpathlineto{\pgfqpoint{4.789795in}{2.883170in}}%
\pgfpathlineto{\pgfqpoint{4.792379in}{2.879212in}}%
\pgfpathlineto{\pgfqpoint{4.794258in}{2.872087in}}%
\pgfpathlineto{\pgfqpoint{4.796540in}{2.852373in}}%
\pgfpathlineto{\pgfqpoint{4.798721in}{2.747872in}}%
\pgfpathlineto{\pgfqpoint{4.799293in}{2.349601in}}%
\pgfpathlineto{\pgfqpoint{4.798794in}{0.405638in}}%
\pgfpathlineto{\pgfqpoint{4.794981in}{0.403515in}}%
\pgfpathlineto{\pgfqpoint{4.788482in}{0.402918in}}%
\pgfpathlineto{\pgfqpoint{4.788482in}{0.402918in}}%
\pgfusepath{stroke}%
\end{pgfscope}%
\begin{pgfscope}%
\pgfpathrectangle{\pgfqpoint{0.448634in}{0.402556in}}{\pgfqpoint{4.350661in}{2.489204in}} %
\pgfusepath{clip}%
\pgfsetrectcap%
\pgfsetroundjoin%
\pgfsetlinewidth{1.003750pt}%
\definecolor{currentstroke}{rgb}{0.737255,0.741176,0.133333}%
\pgfsetstrokecolor{currentstroke}%
\pgfsetdash{}{0pt}%
\pgfpathmoveto{\pgfqpoint{0.448634in}{2.896245in}}%
\pgfpathlineto{\pgfqpoint{0.448593in}{0.407043in}}%
\pgfpathlineto{\pgfqpoint{0.448593in}{0.407043in}}%
\pgfusepath{stroke}%
\end{pgfscope}%
\begin{pgfscope}%
\pgfpathrectangle{\pgfqpoint{0.448634in}{0.402556in}}{\pgfqpoint{4.350661in}{2.489204in}} %
\pgfusepath{clip}%
\pgfsetrectcap%
\pgfsetroundjoin%
\pgfsetlinewidth{1.003750pt}%
\definecolor{currentstroke}{rgb}{0.737255,0.741176,0.133333}%
\pgfsetstrokecolor{currentstroke}%
\pgfsetdash{}{0pt}%
\pgfpathmoveto{\pgfqpoint{4.785329in}{0.402818in}}%
\pgfpathlineto{\pgfqpoint{4.796144in}{0.403830in}}%
\pgfpathlineto{\pgfqpoint{4.798112in}{0.404851in}}%
\pgfpathlineto{\pgfqpoint{4.799211in}{0.406922in}}%
\pgfpathlineto{\pgfqpoint{4.799272in}{0.456701in}}%
\pgfpathlineto{\pgfqpoint{4.799293in}{1.054110in}}%
\pgfpathlineto{\pgfqpoint{4.798064in}{2.806504in}}%
\pgfpathlineto{\pgfqpoint{4.796173in}{2.856234in}}%
\pgfpathlineto{\pgfqpoint{4.794120in}{2.870970in}}%
\pgfpathlineto{\pgfqpoint{4.791304in}{2.880322in}}%
\pgfpathlineto{\pgfqpoint{4.789969in}{2.882276in}}%
\pgfpathlineto{\pgfqpoint{4.787979in}{2.883110in}}%
\pgfpathlineto{\pgfqpoint{4.787925in}{2.882181in}}%
\pgfpathlineto{\pgfqpoint{4.788629in}{2.878119in}}%
\pgfpathlineto{\pgfqpoint{4.788794in}{2.848254in}}%
\pgfpathlineto{\pgfqpoint{4.790148in}{2.738815in}}%
\pgfpathlineto{\pgfqpoint{4.791102in}{2.713958in}}%
\pgfpathlineto{\pgfqpoint{4.789824in}{2.694109in}}%
\pgfpathlineto{\pgfqpoint{4.788490in}{2.669275in}}%
\pgfpathlineto{\pgfqpoint{4.788568in}{2.607048in}}%
\pgfpathlineto{\pgfqpoint{4.790318in}{2.567281in}}%
\pgfpathlineto{\pgfqpoint{4.796837in}{2.480492in}}%
\pgfpathlineto{\pgfqpoint{4.798386in}{2.415800in}}%
\pgfpathlineto{\pgfqpoint{4.798988in}{2.273918in}}%
\pgfpathlineto{\pgfqpoint{4.798407in}{1.454975in}}%
\pgfpathlineto{\pgfqpoint{4.797335in}{1.432635in}}%
\pgfpathlineto{\pgfqpoint{4.794061in}{1.424258in}}%
\pgfpathlineto{\pgfqpoint{4.795408in}{1.409418in}}%
\pgfpathlineto{\pgfqpoint{4.795752in}{1.374576in}}%
\pgfpathlineto{\pgfqpoint{4.793676in}{1.245164in}}%
\pgfpathlineto{\pgfqpoint{4.791019in}{1.207958in}}%
\pgfpathlineto{\pgfqpoint{4.788612in}{1.173236in}}%
\pgfpathlineto{\pgfqpoint{4.789062in}{1.143383in}}%
\pgfpathlineto{\pgfqpoint{4.790262in}{1.108582in}}%
\pgfpathlineto{\pgfqpoint{4.788271in}{1.026504in}}%
\pgfpathlineto{\pgfqpoint{4.792399in}{0.956973in}}%
\pgfpathlineto{\pgfqpoint{4.796454in}{0.912538in}}%
\pgfpathlineto{\pgfqpoint{4.794179in}{0.870305in}}%
\pgfpathlineto{\pgfqpoint{4.792467in}{0.858050in}}%
\pgfpathlineto{\pgfqpoint{4.789823in}{0.854226in}}%
\pgfpathlineto{\pgfqpoint{4.787308in}{0.850249in}}%
\pgfpathlineto{\pgfqpoint{4.787141in}{0.842817in}}%
\pgfpathlineto{\pgfqpoint{4.786682in}{0.825429in}}%
\pgfpathlineto{\pgfqpoint{4.772290in}{0.757774in}}%
\pgfpathlineto{\pgfqpoint{4.766057in}{0.728777in}}%
\pgfpathlineto{\pgfqpoint{4.760243in}{0.699665in}}%
\pgfpathlineto{\pgfqpoint{4.757150in}{0.677551in}}%
\pgfpathlineto{\pgfqpoint{4.754928in}{0.657827in}}%
\pgfpathlineto{\pgfqpoint{4.750181in}{0.588454in}}%
\pgfpathlineto{\pgfqpoint{4.746819in}{0.571494in}}%
\pgfpathlineto{\pgfqpoint{4.741855in}{0.555046in}}%
\pgfpathlineto{\pgfqpoint{4.736456in}{0.541462in}}%
\pgfpathlineto{\pgfqpoint{4.726475in}{0.522202in}}%
\pgfpathlineto{\pgfqpoint{4.716484in}{0.502955in}}%
\pgfpathlineto{\pgfqpoint{4.713515in}{0.493619in}}%
\pgfpathlineto{\pgfqpoint{4.713069in}{0.486197in}}%
\pgfpathlineto{\pgfqpoint{4.714489in}{0.478930in}}%
\pgfpathlineto{\pgfqpoint{4.717491in}{0.472333in}}%
\pgfpathlineto{\pgfqpoint{4.723508in}{0.465189in}}%
\pgfpathlineto{\pgfqpoint{4.730961in}{0.460092in}}%
\pgfpathlineto{\pgfqpoint{4.741239in}{0.456070in}}%
\pgfpathlineto{\pgfqpoint{4.758318in}{0.452299in}}%
\pgfpathlineto{\pgfqpoint{4.773159in}{0.448573in}}%
\pgfpathlineto{\pgfqpoint{4.776793in}{0.445923in}}%
\pgfpathlineto{\pgfqpoint{4.777503in}{0.441230in}}%
\pgfpathlineto{\pgfqpoint{4.774792in}{0.437465in}}%
\pgfpathlineto{\pgfqpoint{4.768864in}{0.434413in}}%
\pgfpathlineto{\pgfqpoint{4.758185in}{0.432113in}}%
\pgfpathlineto{\pgfqpoint{4.734357in}{0.429691in}}%
\pgfpathlineto{\pgfqpoint{4.693045in}{0.428335in}}%
\pgfpathlineto{\pgfqpoint{4.523375in}{0.426811in}}%
\pgfpathlineto{\pgfqpoint{4.308017in}{0.426750in}}%
\pgfpathlineto{\pgfqpoint{3.979543in}{0.427000in}}%
\pgfpathlineto{\pgfqpoint{3.777259in}{0.424906in}}%
\pgfpathlineto{\pgfqpoint{3.616416in}{0.418102in}}%
\pgfpathlineto{\pgfqpoint{3.596976in}{0.415587in}}%
\pgfpathlineto{\pgfqpoint{3.567091in}{0.409651in}}%
\pgfpathlineto{\pgfqpoint{3.549702in}{0.410091in}}%
\pgfpathlineto{\pgfqpoint{3.545440in}{0.411056in}}%
\pgfpathlineto{\pgfqpoint{3.537648in}{0.407113in}}%
\pgfpathlineto{\pgfqpoint{3.524645in}{0.405941in}}%
\pgfpathlineto{\pgfqpoint{3.483318in}{0.405522in}}%
\pgfpathlineto{\pgfqpoint{3.278844in}{0.407392in}}%
\pgfpathlineto{\pgfqpoint{3.176639in}{0.410460in}}%
\pgfpathlineto{\pgfqpoint{3.117988in}{0.413967in}}%
\pgfpathlineto{\pgfqpoint{3.074667in}{0.418506in}}%
\pgfpathlineto{\pgfqpoint{3.014287in}{0.427642in}}%
\pgfpathlineto{\pgfqpoint{2.904654in}{0.446737in}}%
\pgfpathlineto{\pgfqpoint{2.883821in}{0.453871in}}%
\pgfpathlineto{\pgfqpoint{2.863853in}{0.463696in}}%
\pgfpathlineto{\pgfqpoint{2.848829in}{0.473723in}}%
\pgfpathlineto{\pgfqpoint{2.829334in}{0.489591in}}%
\pgfpathlineto{\pgfqpoint{2.813093in}{0.502054in}}%
\pgfpathlineto{\pgfqpoint{2.805005in}{0.505644in}}%
\pgfpathlineto{\pgfqpoint{2.798538in}{0.505540in}}%
\pgfpathlineto{\pgfqpoint{2.794644in}{0.503383in}}%
\pgfpathlineto{\pgfqpoint{2.790289in}{0.497885in}}%
\pgfpathlineto{\pgfqpoint{2.786526in}{0.488927in}}%
\pgfpathlineto{\pgfqpoint{2.783298in}{0.474468in}}%
\pgfpathlineto{\pgfqpoint{2.780383in}{0.449807in}}%
\pgfpathlineto{\pgfqpoint{2.777227in}{0.420190in}}%
\pgfpathlineto{\pgfqpoint{2.775210in}{0.415825in}}%
\pgfpathlineto{\pgfqpoint{2.771421in}{0.413567in}}%
\pgfpathlineto{\pgfqpoint{2.767127in}{0.414048in}}%
\pgfpathlineto{\pgfqpoint{2.763344in}{0.416446in}}%
\pgfpathlineto{\pgfqpoint{2.759476in}{0.422389in}}%
\pgfpathlineto{\pgfqpoint{2.756811in}{0.431842in}}%
\pgfpathlineto{\pgfqpoint{2.754796in}{0.449103in}}%
\pgfpathlineto{\pgfqpoint{2.753373in}{0.483909in}}%
\pgfpathlineto{\pgfqpoint{2.752844in}{0.563559in}}%
\pgfpathlineto{\pgfqpoint{2.754275in}{0.720369in}}%
\pgfpathlineto{\pgfqpoint{2.758386in}{0.914468in}}%
\pgfpathlineto{\pgfqpoint{2.764441in}{1.098538in}}%
\pgfpathlineto{\pgfqpoint{2.779991in}{1.531269in}}%
\pgfpathlineto{\pgfqpoint{2.779160in}{1.578547in}}%
\pgfpathlineto{\pgfqpoint{2.777170in}{1.595753in}}%
\pgfpathlineto{\pgfqpoint{2.774194in}{1.599152in}}%
\pgfpathlineto{\pgfqpoint{2.772733in}{1.599004in}}%
\pgfpathlineto{\pgfqpoint{2.771412in}{1.597150in}}%
\pgfpathlineto{\pgfqpoint{2.771447in}{1.587232in}}%
\pgfpathlineto{\pgfqpoint{2.773246in}{1.527535in}}%
\pgfpathlineto{\pgfqpoint{2.772411in}{1.520154in}}%
\pgfpathlineto{\pgfqpoint{2.769593in}{1.516624in}}%
\pgfpathlineto{\pgfqpoint{2.766717in}{1.513133in}}%
\pgfpathlineto{\pgfqpoint{2.766677in}{1.505706in}}%
\pgfpathlineto{\pgfqpoint{2.768815in}{1.488470in}}%
\pgfpathlineto{\pgfqpoint{2.772588in}{1.469075in}}%
\pgfpathlineto{\pgfqpoint{2.773933in}{1.444238in}}%
\pgfpathlineto{\pgfqpoint{2.773797in}{1.396947in}}%
\pgfpathlineto{\pgfqpoint{2.771072in}{1.302411in}}%
\pgfpathlineto{\pgfqpoint{2.756792in}{0.899497in}}%
\pgfpathlineto{\pgfqpoint{2.752055in}{0.683006in}}%
\pgfpathlineto{\pgfqpoint{2.747296in}{0.473998in}}%
\pgfpathlineto{\pgfqpoint{2.744657in}{0.449305in}}%
\pgfpathlineto{\pgfqpoint{2.742959in}{0.442113in}}%
\pgfpathlineto{\pgfqpoint{2.741653in}{0.440151in}}%
\pgfpathlineto{\pgfqpoint{2.739642in}{0.439326in}}%
\pgfpathlineto{\pgfqpoint{2.737552in}{0.439885in}}%
\pgfpathlineto{\pgfqpoint{2.735109in}{0.443871in}}%
\pgfpathlineto{\pgfqpoint{2.729294in}{0.467841in}}%
\pgfpathlineto{\pgfqpoint{2.726890in}{0.490062in}}%
\pgfpathlineto{\pgfqpoint{2.725536in}{0.529855in}}%
\pgfpathlineto{\pgfqpoint{2.725030in}{0.624441in}}%
\pgfpathlineto{\pgfqpoint{2.726692in}{0.771291in}}%
\pgfpathlineto{\pgfqpoint{2.730716in}{0.952942in}}%
\pgfpathlineto{\pgfqpoint{2.731917in}{0.972783in}}%
\pgfpathlineto{\pgfqpoint{2.738564in}{1.156824in}}%
\pgfpathlineto{\pgfqpoint{2.746702in}{1.315856in}}%
\pgfpathlineto{\pgfqpoint{2.755008in}{1.432459in}}%
\pgfpathlineto{\pgfqpoint{2.761441in}{1.494246in}}%
\pgfpathlineto{\pgfqpoint{2.766338in}{1.523572in}}%
\pgfpathlineto{\pgfqpoint{2.770287in}{1.532404in}}%
\pgfpathlineto{\pgfqpoint{2.772144in}{1.539516in}}%
\pgfpathlineto{\pgfqpoint{2.772614in}{1.554428in}}%
\pgfpathlineto{\pgfqpoint{2.772370in}{1.598789in}}%
\pgfpathlineto{\pgfqpoint{2.773787in}{1.599253in}}%
\pgfpathlineto{\pgfqpoint{2.776990in}{1.596173in}}%
\pgfpathlineto{\pgfqpoint{2.778473in}{1.588939in}}%
\pgfpathlineto{\pgfqpoint{2.779812in}{1.559114in}}%
\pgfpathlineto{\pgfqpoint{2.779699in}{1.504356in}}%
\pgfpathlineto{\pgfqpoint{2.776969in}{1.409819in}}%
\pgfpathlineto{\pgfqpoint{2.768799in}{1.210904in}}%
\pgfpathlineto{\pgfqpoint{2.761036in}{1.002003in}}%
\pgfpathlineto{\pgfqpoint{2.756209in}{0.822867in}}%
\pgfpathlineto{\pgfqpoint{2.753287in}{0.641188in}}%
\pgfpathlineto{\pgfqpoint{2.753540in}{0.476909in}}%
\pgfpathlineto{\pgfqpoint{2.755666in}{0.439665in}}%
\pgfpathlineto{\pgfqpoint{2.758465in}{0.425108in}}%
\pgfpathlineto{\pgfqpoint{2.761542in}{0.418557in}}%
\pgfpathlineto{\pgfqpoint{2.764782in}{0.415275in}}%
\pgfpathlineto{\pgfqpoint{2.768844in}{0.413591in}}%
\pgfpathlineto{\pgfqpoint{2.773107in}{0.414118in}}%
\pgfpathlineto{\pgfqpoint{2.776232in}{0.417444in}}%
\pgfpathlineto{\pgfqpoint{2.778101in}{0.424562in}}%
\pgfpathlineto{\pgfqpoint{2.780801in}{0.454269in}}%
\pgfpathlineto{\pgfqpoint{2.784599in}{0.481284in}}%
\pgfpathlineto{\pgfqpoint{2.788028in}{0.493071in}}%
\pgfpathlineto{\pgfqpoint{2.792675in}{0.501436in}}%
\pgfpathlineto{\pgfqpoint{2.796118in}{0.504448in}}%
\pgfpathlineto{\pgfqpoint{2.800247in}{0.505918in}}%
\pgfpathlineto{\pgfqpoint{2.806694in}{0.505163in}}%
\pgfpathlineto{\pgfqpoint{2.814632in}{0.501123in}}%
\pgfpathlineto{\pgfqpoint{2.825468in}{0.492819in}}%
\pgfpathlineto{\pgfqpoint{2.857715in}{0.467455in}}%
\pgfpathlineto{\pgfqpoint{2.875282in}{0.457590in}}%
\pgfpathlineto{\pgfqpoint{2.893761in}{0.450220in}}%
\pgfpathlineto{\pgfqpoint{2.914838in}{0.444105in}}%
\pgfpathlineto{\pgfqpoint{2.940574in}{0.439137in}}%
\pgfpathlineto{\pgfqpoint{3.041855in}{0.423173in}}%
\pgfpathlineto{\pgfqpoint{3.089373in}{0.416678in}}%
\pgfpathlineto{\pgfqpoint{3.132748in}{0.412852in}}%
\pgfpathlineto{\pgfqpoint{3.191423in}{0.409915in}}%
\pgfpathlineto{\pgfqpoint{3.315386in}{0.406895in}}%
\pgfpathlineto{\pgfqpoint{3.526384in}{0.406018in}}%
\pgfpathlineto{\pgfqpoint{3.539343in}{0.407559in}}%
\pgfpathlineto{\pgfqpoint{3.543228in}{0.409682in}}%
\pgfpathlineto{\pgfqpoint{3.545028in}{0.411074in}}%
\pgfpathlineto{\pgfqpoint{3.562334in}{0.410931in}}%
\pgfpathlineto{\pgfqpoint{3.575237in}{0.413075in}}%
\pgfpathlineto{\pgfqpoint{3.596695in}{0.417080in}}%
\pgfpathlineto{\pgfqpoint{3.635753in}{0.420219in}}%
\pgfpathlineto{\pgfqpoint{3.685748in}{0.422364in}}%
\pgfpathlineto{\pgfqpoint{3.868426in}{0.426815in}}%
\pgfpathlineto{\pgfqpoint{4.296965in}{0.426797in}}%
\pgfpathlineto{\pgfqpoint{4.534076in}{0.426874in}}%
\pgfpathlineto{\pgfqpoint{4.723315in}{0.429145in}}%
\pgfpathlineto{\pgfqpoint{4.753694in}{0.431475in}}%
\pgfpathlineto{\pgfqpoint{4.768695in}{0.434357in}}%
\pgfpathlineto{\pgfqpoint{4.774645in}{0.437356in}}%
\pgfpathlineto{\pgfqpoint{4.777445in}{0.441043in}}%
\pgfpathlineto{\pgfqpoint{4.776891in}{0.445769in}}%
\pgfpathlineto{\pgfqpoint{4.773309in}{0.448497in}}%
\pgfpathlineto{\pgfqpoint{4.764920in}{0.451064in}}%
\pgfpathlineto{\pgfqpoint{4.737211in}{0.457379in}}%
\pgfpathlineto{\pgfqpoint{4.727253in}{0.462311in}}%
\pgfpathlineto{\pgfqpoint{4.720382in}{0.468372in}}%
\pgfpathlineto{\pgfqpoint{4.715363in}{0.476449in}}%
\pgfpathlineto{\pgfqpoint{4.713082in}{0.486011in}}%
\pgfpathlineto{\pgfqpoint{4.713481in}{0.493437in}}%
\pgfpathlineto{\pgfqpoint{4.716411in}{0.502789in}}%
\pgfpathlineto{\pgfqpoint{4.725182in}{0.519972in}}%
\pgfpathlineto{\pgfqpoint{4.736384in}{0.541295in}}%
\pgfpathlineto{\pgfqpoint{4.744971in}{0.564140in}}%
\pgfpathlineto{\pgfqpoint{4.749819in}{0.585811in}}%
\pgfpathlineto{\pgfqpoint{4.754156in}{0.625301in}}%
\pgfpathlineto{\pgfqpoint{4.755934in}{0.670025in}}%
\pgfpathlineto{\pgfqpoint{4.765445in}{0.726195in}}%
\pgfpathlineto{\pgfqpoint{4.776897in}{0.779350in}}%
\pgfpathlineto{\pgfqpoint{4.786054in}{0.822854in}}%
\pgfpathlineto{\pgfqpoint{4.787116in}{0.830194in}}%
\pgfpathlineto{\pgfqpoint{4.788219in}{0.852298in}}%
\pgfpathlineto{\pgfqpoint{4.792394in}{0.857897in}}%
\pgfpathlineto{\pgfqpoint{4.793965in}{0.867660in}}%
\pgfpathlineto{\pgfqpoint{4.796445in}{0.912371in}}%
\pgfpathlineto{\pgfqpoint{4.795967in}{0.932260in}}%
\pgfpathlineto{\pgfqpoint{4.790096in}{0.994046in}}%
\pgfpathlineto{\pgfqpoint{4.788004in}{1.048735in}}%
\pgfpathlineto{\pgfqpoint{4.789918in}{1.100958in}}%
\pgfpathlineto{\pgfqpoint{4.790097in}{1.125833in}}%
\pgfpathlineto{\pgfqpoint{4.788666in}{1.175558in}}%
\pgfpathlineto{\pgfqpoint{4.790498in}{1.202847in}}%
\pgfpathlineto{\pgfqpoint{4.793558in}{1.242512in}}%
\pgfpathlineto{\pgfqpoint{4.794796in}{1.294762in}}%
\pgfpathlineto{\pgfqpoint{4.794983in}{1.416702in}}%
\pgfpathlineto{\pgfqpoint{4.794459in}{1.428837in}}%
\pgfpathlineto{\pgfqpoint{4.797285in}{1.432455in}}%
\pgfpathlineto{\pgfqpoint{4.798078in}{1.442347in}}%
\pgfpathlineto{\pgfqpoint{4.798892in}{1.502077in}}%
\pgfpathlineto{\pgfqpoint{4.797600in}{2.457918in}}%
\pgfpathlineto{\pgfqpoint{4.795011in}{2.512593in}}%
\pgfpathlineto{\pgfqpoint{4.788531in}{2.609348in}}%
\pgfpathlineto{\pgfqpoint{4.788590in}{2.674064in}}%
\pgfpathlineto{\pgfqpoint{4.790295in}{2.698868in}}%
\pgfpathlineto{\pgfqpoint{4.791100in}{2.718747in}}%
\pgfpathlineto{\pgfqpoint{4.789744in}{2.743582in}}%
\pgfpathlineto{\pgfqpoint{4.788479in}{2.768421in}}%
\pgfpathlineto{\pgfqpoint{4.788290in}{2.883466in}}%
\pgfpathlineto{\pgfqpoint{4.786249in}{2.882747in}}%
\pgfpathlineto{\pgfqpoint{4.783853in}{2.878714in}}%
\pgfpathlineto{\pgfqpoint{4.782367in}{2.866405in}}%
\pgfpathlineto{\pgfqpoint{4.781859in}{2.829075in}}%
\pgfpathlineto{\pgfqpoint{4.783042in}{2.266606in}}%
\pgfpathlineto{\pgfqpoint{4.782620in}{2.216827in}}%
\pgfpathlineto{\pgfqpoint{4.781887in}{2.107307in}}%
\pgfpathlineto{\pgfqpoint{4.782950in}{1.607003in}}%
\pgfpathlineto{\pgfqpoint{4.790815in}{1.415584in}}%
\pgfpathlineto{\pgfqpoint{4.793018in}{1.380855in}}%
\pgfpathlineto{\pgfqpoint{4.791949in}{1.351017in}}%
\pgfpathlineto{\pgfqpoint{4.787537in}{1.274047in}}%
\pgfpathlineto{\pgfqpoint{4.784739in}{1.209418in}}%
\pgfpathlineto{\pgfqpoint{4.783557in}{1.179592in}}%
\pgfpathlineto{\pgfqpoint{4.784800in}{1.134820in}}%
\pgfpathlineto{\pgfqpoint{4.784938in}{1.107452in}}%
\pgfpathlineto{\pgfqpoint{4.782660in}{1.080204in}}%
\pgfpathlineto{\pgfqpoint{4.780418in}{1.052968in}}%
\pgfpathlineto{\pgfqpoint{4.781105in}{1.030589in}}%
\pgfpathlineto{\pgfqpoint{4.783948in}{1.008435in}}%
\pgfpathlineto{\pgfqpoint{4.785645in}{0.988627in}}%
\pgfpathlineto{\pgfqpoint{4.785691in}{0.911487in}}%
\pgfpathlineto{\pgfqpoint{4.780018in}{0.839641in}}%
\pgfpathlineto{\pgfqpoint{4.776381in}{0.817651in}}%
\pgfpathlineto{\pgfqpoint{4.768931in}{0.781351in}}%
\pgfpathlineto{\pgfqpoint{4.763280in}{0.759934in}}%
\pgfpathlineto{\pgfqpoint{4.758452in}{0.743466in}}%
\pgfpathlineto{\pgfqpoint{4.740548in}{0.664159in}}%
\pgfpathlineto{\pgfqpoint{4.739727in}{0.649266in}}%
\pgfpathlineto{\pgfqpoint{4.737677in}{0.622017in}}%
\pgfpathlineto{\pgfqpoint{4.734936in}{0.597344in}}%
\pgfpathlineto{\pgfqpoint{4.732485in}{0.585248in}}%
\pgfpathlineto{\pgfqpoint{4.724975in}{0.561942in}}%
\pgfpathlineto{\pgfqpoint{4.714851in}{0.542802in}}%
\pgfpathlineto{\pgfqpoint{4.702618in}{0.525333in}}%
\pgfpathlineto{\pgfqpoint{4.676095in}{0.492447in}}%
\pgfpathlineto{\pgfqpoint{4.670543in}{0.481760in}}%
\pgfpathlineto{\pgfqpoint{4.663398in}{0.466420in}}%
\pgfpathlineto{\pgfqpoint{4.656906in}{0.459851in}}%
\pgfpathlineto{\pgfqpoint{4.647079in}{0.454564in}}%
\pgfpathlineto{\pgfqpoint{4.628092in}{0.449131in}}%
\pgfpathlineto{\pgfqpoint{4.608640in}{0.446638in}}%
\pgfpathlineto{\pgfqpoint{4.574042in}{0.442565in}}%
\pgfpathlineto{\pgfqpoint{4.528376in}{0.441412in}}%
\pgfpathlineto{\pgfqpoint{4.454431in}{0.439747in}}%
\pgfpathlineto{\pgfqpoint{4.326089in}{0.439261in}}%
\pgfpathlineto{\pgfqpoint{4.180364in}{0.442129in}}%
\pgfpathlineto{\pgfqpoint{4.073814in}{0.444487in}}%
\pgfpathlineto{\pgfqpoint{4.019450in}{0.446072in}}%
\pgfpathlineto{\pgfqpoint{3.680541in}{0.461823in}}%
\pgfpathlineto{\pgfqpoint{3.587369in}{0.471251in}}%
\pgfpathlineto{\pgfqpoint{3.522570in}{0.480087in}}%
\pgfpathlineto{\pgfqpoint{3.470951in}{0.489014in}}%
\pgfpathlineto{\pgfqpoint{3.419687in}{0.500293in}}%
\pgfpathlineto{\pgfqpoint{3.377408in}{0.512012in}}%
\pgfpathlineto{\pgfqpoint{3.344040in}{0.523322in}}%
\pgfpathlineto{\pgfqpoint{3.313307in}{0.535852in}}%
\pgfpathlineto{\pgfqpoint{3.283317in}{0.550547in}}%
\pgfpathlineto{\pgfqpoint{3.256185in}{0.566364in}}%
\pgfpathlineto{\pgfqpoint{3.228163in}{0.585480in}}%
\pgfpathlineto{\pgfqpoint{3.199412in}{0.607917in}}%
\pgfpathlineto{\pgfqpoint{3.181946in}{0.622754in}}%
\pgfpathlineto{\pgfqpoint{3.181946in}{0.622754in}}%
\pgfusepath{stroke}%
\end{pgfscope}%
\begin{pgfscope}%
\pgfpathrectangle{\pgfqpoint{0.448634in}{0.402556in}}{\pgfqpoint{4.350661in}{2.489204in}} %
\pgfusepath{clip}%
\pgfsetrectcap%
\pgfsetroundjoin%
\pgfsetlinewidth{1.003750pt}%
\definecolor{currentstroke}{rgb}{0.737255,0.741176,0.133333}%
\pgfsetstrokecolor{currentstroke}%
\pgfsetdash{}{0pt}%
\pgfpathmoveto{\pgfqpoint{4.218567in}{0.402569in}}%
\pgfpathlineto{\pgfqpoint{3.698666in}{0.403614in}}%
\pgfpathlineto{\pgfqpoint{3.611674in}{0.405644in}}%
\pgfpathlineto{\pgfqpoint{3.553023in}{0.409106in}}%
\pgfpathlineto{\pgfqpoint{3.542813in}{0.409340in}}%
\pgfpathlineto{\pgfqpoint{3.538839in}{0.407418in}}%
\pgfpathlineto{\pgfqpoint{3.528038in}{0.406105in}}%
\pgfpathlineto{\pgfqpoint{3.495415in}{0.405506in}}%
\pgfpathlineto{\pgfqpoint{3.362723in}{0.406533in}}%
\pgfpathlineto{\pgfqpoint{3.236569in}{0.408492in}}%
\pgfpathlineto{\pgfqpoint{3.123546in}{0.413535in}}%
\pgfpathlineto{\pgfqpoint{3.075880in}{0.418349in}}%
\pgfpathlineto{\pgfqpoint{3.015496in}{0.427453in}}%
\pgfpathlineto{\pgfqpoint{2.905842in}{0.446425in}}%
\pgfpathlineto{\pgfqpoint{2.884973in}{0.453428in}}%
\pgfpathlineto{\pgfqpoint{2.866893in}{0.461989in}}%
\pgfpathlineto{\pgfqpoint{2.851692in}{0.471661in}}%
\pgfpathlineto{\pgfqpoint{2.830335in}{0.488822in}}%
\pgfpathlineto{\pgfqpoint{2.814212in}{0.501496in}}%
\pgfpathlineto{\pgfqpoint{2.806249in}{0.505466in}}%
\pgfpathlineto{\pgfqpoint{2.799791in}{0.506065in}}%
\pgfpathlineto{\pgfqpoint{2.795698in}{0.504468in}}%
\pgfpathlineto{\pgfqpoint{2.790927in}{0.499460in}}%
\pgfpathlineto{\pgfqpoint{2.786827in}{0.490701in}}%
\pgfpathlineto{\pgfqpoint{2.783279in}{0.476346in}}%
\pgfpathlineto{\pgfqpoint{2.779506in}{0.449312in}}%
\pgfpathlineto{\pgfqpoint{2.776436in}{0.432290in}}%
\pgfpathlineto{\pgfqpoint{2.774286in}{0.427982in}}%
\pgfpathlineto{\pgfqpoint{2.770631in}{0.425558in}}%
\pgfpathlineto{\pgfqpoint{2.766530in}{0.426903in}}%
\pgfpathlineto{\pgfqpoint{2.762501in}{0.432700in}}%
\pgfpathlineto{\pgfqpoint{2.759610in}{0.442069in}}%
\pgfpathlineto{\pgfqpoint{2.757072in}{0.459236in}}%
\pgfpathlineto{\pgfqpoint{2.755116in}{0.491512in}}%
\pgfpathlineto{\pgfqpoint{2.754002in}{0.553726in}}%
\pgfpathlineto{\pgfqpoint{2.754396in}{0.675695in}}%
\pgfpathlineto{\pgfqpoint{2.757217in}{0.842439in}}%
\pgfpathlineto{\pgfqpoint{2.762838in}{1.038978in}}%
\pgfpathlineto{\pgfqpoint{2.770885in}{1.245373in}}%
\pgfpathlineto{\pgfqpoint{2.783621in}{1.563652in}}%
\pgfpathlineto{\pgfqpoint{2.784199in}{1.630851in}}%
\pgfpathlineto{\pgfqpoint{2.782435in}{1.670619in}}%
\pgfpathlineto{\pgfqpoint{2.779068in}{1.722699in}}%
\pgfpathlineto{\pgfqpoint{2.781365in}{1.764930in}}%
\pgfpathlineto{\pgfqpoint{2.788916in}{1.844112in}}%
\pgfpathlineto{\pgfqpoint{2.800855in}{1.945244in}}%
\pgfpathlineto{\pgfqpoint{2.814953in}{2.043492in}}%
\pgfpathlineto{\pgfqpoint{2.828998in}{2.124041in}}%
\pgfpathlineto{\pgfqpoint{2.837443in}{2.162671in}}%
\pgfpathlineto{\pgfqpoint{2.855722in}{2.239518in}}%
\pgfpathlineto{\pgfqpoint{2.863840in}{2.273010in}}%
\pgfpathlineto{\pgfqpoint{2.864534in}{2.280391in}}%
\pgfpathlineto{\pgfqpoint{2.870307in}{2.299158in}}%
\pgfpathlineto{\pgfqpoint{2.890866in}{2.364743in}}%
\pgfpathlineto{\pgfqpoint{2.913042in}{2.440227in}}%
\pgfpathlineto{\pgfqpoint{2.919611in}{2.471690in}}%
\pgfpathlineto{\pgfqpoint{2.922185in}{2.493881in}}%
\pgfpathlineto{\pgfqpoint{2.922290in}{2.518749in}}%
\pgfpathlineto{\pgfqpoint{2.920006in}{2.535964in}}%
\pgfpathlineto{\pgfqpoint{2.915621in}{2.552634in}}%
\pgfpathlineto{\pgfqpoint{2.909148in}{2.568389in}}%
\pgfpathlineto{\pgfqpoint{2.900905in}{2.583027in}}%
\pgfpathlineto{\pgfqpoint{2.889681in}{2.598226in}}%
\pgfpathlineto{\pgfqpoint{2.875464in}{2.613616in}}%
\pgfpathlineto{\pgfqpoint{2.856691in}{2.630578in}}%
\pgfpathlineto{\pgfqpoint{2.831313in}{2.649829in}}%
\pgfpathlineto{\pgfqpoint{2.802869in}{2.668113in}}%
\pgfpathlineto{\pgfqpoint{2.771586in}{2.685559in}}%
\pgfpathlineto{\pgfqpoint{2.745509in}{2.698066in}}%
\pgfpathlineto{\pgfqpoint{2.702744in}{2.716385in}}%
\pgfpathlineto{\pgfqpoint{2.669375in}{2.727678in}}%
\pgfpathlineto{\pgfqpoint{2.635555in}{2.737062in}}%
\pgfpathlineto{\pgfqpoint{2.622718in}{2.739305in}}%
\pgfpathlineto{\pgfqpoint{2.616455in}{2.737490in}}%
\pgfpathlineto{\pgfqpoint{2.607828in}{2.738574in}}%
\pgfpathlineto{\pgfqpoint{2.604113in}{2.741000in}}%
\pgfpathlineto{\pgfqpoint{2.602573in}{2.742725in}}%
\pgfpathlineto{\pgfqpoint{2.598321in}{2.742802in}}%
\pgfpathlineto{\pgfqpoint{2.589650in}{2.743120in}}%
\pgfpathlineto{\pgfqpoint{2.583474in}{2.745307in}}%
\pgfpathlineto{\pgfqpoint{2.580172in}{2.748387in}}%
\pgfpathlineto{\pgfqpoint{2.562908in}{2.748928in}}%
\pgfpathlineto{\pgfqpoint{2.558921in}{2.750787in}}%
\pgfpathlineto{\pgfqpoint{2.556534in}{2.754903in}}%
\pgfpathlineto{\pgfqpoint{2.548185in}{2.753158in}}%
\pgfpathlineto{\pgfqpoint{2.537380in}{2.754413in}}%
\pgfpathlineto{\pgfqpoint{2.531332in}{2.757035in}}%
\pgfpathlineto{\pgfqpoint{2.527440in}{2.758878in}}%
\pgfpathlineto{\pgfqpoint{2.519175in}{2.756205in}}%
\pgfpathlineto{\pgfqpoint{2.503970in}{2.756960in}}%
\pgfpathlineto{\pgfqpoint{2.439169in}{2.763879in}}%
\pgfpathlineto{\pgfqpoint{2.432732in}{2.764847in}}%
\pgfpathlineto{\pgfqpoint{2.398028in}{2.767834in}}%
\pgfpathlineto{\pgfqpoint{2.391717in}{2.769581in}}%
\pgfpathlineto{\pgfqpoint{2.387885in}{2.771679in}}%
\pgfpathlineto{\pgfqpoint{2.381531in}{2.770143in}}%
\pgfpathlineto{\pgfqpoint{2.366311in}{2.770092in}}%
\pgfpathlineto{\pgfqpoint{2.335928in}{2.772395in}}%
\pgfpathlineto{\pgfqpoint{2.316425in}{2.774266in}}%
\pgfpathlineto{\pgfqpoint{2.249119in}{2.776135in}}%
\pgfpathlineto{\pgfqpoint{2.157769in}{2.777829in}}%
\pgfpathlineto{\pgfqpoint{2.081635in}{2.777375in}}%
\pgfpathlineto{\pgfqpoint{2.009876in}{2.775346in}}%
\pgfpathlineto{\pgfqpoint{1.881712in}{2.767812in}}%
\pgfpathlineto{\pgfqpoint{1.844832in}{2.766334in}}%
\pgfpathlineto{\pgfqpoint{1.833979in}{2.765786in}}%
\pgfpathlineto{\pgfqpoint{1.658370in}{2.749470in}}%
\pgfpathlineto{\pgfqpoint{1.615232in}{2.743651in}}%
\pgfpathlineto{\pgfqpoint{1.498631in}{2.727432in}}%
\pgfpathlineto{\pgfqpoint{1.440597in}{2.717120in}}%
\pgfpathlineto{\pgfqpoint{1.391492in}{2.706165in}}%
\pgfpathlineto{\pgfqpoint{1.340591in}{2.692912in}}%
\pgfpathlineto{\pgfqpoint{1.302783in}{2.681268in}}%
\pgfpathlineto{\pgfqpoint{1.259254in}{2.665424in}}%
\pgfpathlineto{\pgfqpoint{1.212417in}{2.645334in}}%
\pgfpathlineto{\pgfqpoint{1.202522in}{2.640266in}}%
\pgfpathlineto{\pgfqpoint{1.198137in}{2.634907in}}%
\pgfpathlineto{\pgfqpoint{1.160861in}{2.614489in}}%
\pgfpathlineto{\pgfqpoint{1.132326in}{2.596424in}}%
\pgfpathlineto{\pgfqpoint{1.104577in}{2.576805in}}%
\pgfpathlineto{\pgfqpoint{1.079113in}{2.557709in}}%
\pgfpathlineto{\pgfqpoint{1.059268in}{2.542624in}}%
\pgfpathlineto{\pgfqpoint{1.055089in}{2.541491in}}%
\pgfpathlineto{\pgfqpoint{1.032683in}{2.521793in}}%
\pgfpathlineto{\pgfqpoint{0.993266in}{2.482818in}}%
\pgfpathlineto{\pgfqpoint{0.984850in}{2.471430in}}%
\pgfpathlineto{\pgfqpoint{0.982595in}{2.462380in}}%
\pgfpathlineto{\pgfqpoint{0.972928in}{2.452384in}}%
\pgfpathlineto{\pgfqpoint{0.961581in}{2.437300in}}%
\pgfpathlineto{\pgfqpoint{0.954761in}{2.427723in}}%
\pgfpathlineto{\pgfqpoint{0.951267in}{2.424807in}}%
\pgfpathlineto{\pgfqpoint{0.930080in}{2.396733in}}%
\pgfpathlineto{\pgfqpoint{0.898035in}{2.335051in}}%
\pgfpathlineto{\pgfqpoint{0.892868in}{2.321364in}}%
\pgfpathlineto{\pgfqpoint{0.886790in}{2.305467in}}%
\pgfpathlineto{\pgfqpoint{0.888290in}{2.298451in}}%
\pgfpathlineto{\pgfqpoint{0.877460in}{2.268764in}}%
\pgfpathlineto{\pgfqpoint{0.868727in}{2.257844in}}%
\pgfpathlineto{\pgfqpoint{0.855633in}{2.221038in}}%
\pgfpathlineto{\pgfqpoint{0.853013in}{2.214236in}}%
\pgfpathlineto{\pgfqpoint{0.842885in}{2.173547in}}%
\pgfpathlineto{\pgfqpoint{0.833193in}{2.127576in}}%
\pgfpathlineto{\pgfqpoint{0.821588in}{2.061698in}}%
\pgfpathlineto{\pgfqpoint{0.811622in}{1.990422in}}%
\pgfpathlineto{\pgfqpoint{0.794427in}{1.829831in}}%
\pgfpathlineto{\pgfqpoint{0.781780in}{1.713741in}}%
\pgfpathlineto{\pgfqpoint{0.773005in}{1.652332in}}%
\pgfpathlineto{\pgfqpoint{0.764961in}{1.611038in}}%
\pgfpathlineto{\pgfqpoint{0.757885in}{1.584891in}}%
\pgfpathlineto{\pgfqpoint{0.750993in}{1.566619in}}%
\pgfpathlineto{\pgfqpoint{0.744198in}{1.553886in}}%
\pgfpathlineto{\pgfqpoint{0.736931in}{1.544656in}}%
\pgfpathlineto{\pgfqpoint{0.729829in}{1.538943in}}%
\pgfpathlineto{\pgfqpoint{0.721708in}{1.535450in}}%
\pgfpathlineto{\pgfqpoint{0.713074in}{1.534429in}}%
\pgfpathlineto{\pgfqpoint{0.704434in}{1.535496in}}%
\pgfpathlineto{\pgfqpoint{0.691914in}{1.539657in}}%
\pgfpathlineto{\pgfqpoint{0.673980in}{1.548623in}}%
\pgfpathlineto{\pgfqpoint{0.627237in}{1.574936in}}%
\pgfpathlineto{\pgfqpoint{0.619595in}{1.583759in}}%
\pgfpathlineto{\pgfqpoint{0.612421in}{1.596213in}}%
\pgfpathlineto{\pgfqpoint{0.605141in}{1.614285in}}%
\pgfpathlineto{\pgfqpoint{0.597567in}{1.640248in}}%
\pgfpathlineto{\pgfqpoint{0.590224in}{1.674062in}}%
\pgfpathlineto{\pgfqpoint{0.582967in}{1.718087in}}%
\pgfpathlineto{\pgfqpoint{0.575678in}{1.777238in}}%
\pgfpathlineto{\pgfqpoint{0.569399in}{1.846562in}}%
\pgfpathlineto{\pgfqpoint{0.563767in}{1.935938in}}%
\pgfpathlineto{\pgfqpoint{0.559804in}{2.035399in}}%
\pgfpathlineto{\pgfqpoint{0.557455in}{2.154848in}}%
\pgfpathlineto{\pgfqpoint{0.557596in}{2.264371in}}%
\pgfpathlineto{\pgfqpoint{0.559579in}{2.363911in}}%
\pgfpathlineto{\pgfqpoint{0.559579in}{2.363911in}}%
\pgfusepath{stroke}%
\end{pgfscope}%
\begin{pgfscope}%
\pgfpathrectangle{\pgfqpoint{0.448634in}{0.402556in}}{\pgfqpoint{4.350661in}{2.489204in}} %
\pgfusepath{clip}%
\pgfsetrectcap%
\pgfsetroundjoin%
\pgfsetlinewidth{1.003750pt}%
\definecolor{currentstroke}{rgb}{0.737255,0.741176,0.133333}%
\pgfsetstrokecolor{currentstroke}%
\pgfsetdash{}{0pt}%
\pgfpathmoveto{\pgfqpoint{4.220088in}{0.402564in}}%
\pgfpathlineto{\pgfqpoint{3.652329in}{0.403619in}}%
\pgfpathlineto{\pgfqpoint{3.567517in}{0.405858in}}%
\pgfpathlineto{\pgfqpoint{3.552380in}{0.407609in}}%
\pgfpathlineto{\pgfqpoint{3.542531in}{0.409105in}}%
\pgfpathlineto{\pgfqpoint{3.538502in}{0.407323in}}%
\pgfpathlineto{\pgfqpoint{3.527691in}{0.406086in}}%
\pgfpathlineto{\pgfqpoint{3.492892in}{0.405505in}}%
\pgfpathlineto{\pgfqpoint{3.338447in}{0.406706in}}%
\pgfpathlineto{\pgfqpoint{3.234047in}{0.408580in}}%
\pgfpathlineto{\pgfqpoint{3.123200in}{0.413562in}}%
\pgfpathlineto{\pgfqpoint{3.077696in}{0.418113in}}%
\pgfpathlineto{\pgfqpoint{3.021614in}{0.426473in}}%
\pgfpathlineto{\pgfqpoint{2.907619in}{0.445938in}}%
\pgfpathlineto{\pgfqpoint{2.886701in}{0.452744in}}%
\pgfpathlineto{\pgfqpoint{2.868542in}{0.461084in}}%
\pgfpathlineto{\pgfqpoint{2.853237in}{0.470540in}}%
\pgfpathlineto{\pgfqpoint{2.833549in}{0.486087in}}%
\pgfpathlineto{\pgfqpoint{2.811965in}{0.502806in}}%
\pgfpathlineto{\pgfqpoint{2.803781in}{0.506056in}}%
\pgfpathlineto{\pgfqpoint{2.799448in}{0.505998in}}%
\pgfpathlineto{\pgfqpoint{2.795399in}{0.504260in}}%
\pgfpathlineto{\pgfqpoint{2.790722in}{0.499133in}}%
\pgfpathlineto{\pgfqpoint{2.786697in}{0.490326in}}%
\pgfpathlineto{\pgfqpoint{2.783205in}{0.475951in}}%
\pgfpathlineto{\pgfqpoint{2.779149in}{0.446449in}}%
\pgfpathlineto{\pgfqpoint{2.776299in}{0.431918in}}%
\pgfpathlineto{\pgfqpoint{2.774056in}{0.427675in}}%
\pgfpathlineto{\pgfqpoint{2.770275in}{0.425544in}}%
\pgfpathlineto{\pgfqpoint{2.766243in}{0.427147in}}%
\pgfpathlineto{\pgfqpoint{2.762342in}{0.433068in}}%
\pgfpathlineto{\pgfqpoint{2.759525in}{0.442468in}}%
\pgfpathlineto{\pgfqpoint{2.757034in}{0.459645in}}%
\pgfpathlineto{\pgfqpoint{2.755102in}{0.491923in}}%
\pgfpathlineto{\pgfqpoint{2.754000in}{0.554138in}}%
\pgfpathlineto{\pgfqpoint{2.754401in}{0.676106in}}%
\pgfpathlineto{\pgfqpoint{2.757227in}{0.842850in}}%
\pgfpathlineto{\pgfqpoint{2.762935in}{1.041877in}}%
\pgfpathlineto{\pgfqpoint{2.771119in}{1.250756in}}%
\pgfpathlineto{\pgfqpoint{2.783571in}{1.561575in}}%
\pgfpathlineto{\pgfqpoint{2.784232in}{1.628774in}}%
\pgfpathlineto{\pgfqpoint{2.782601in}{1.668550in}}%
\pgfpathlineto{\pgfqpoint{2.779160in}{1.725599in}}%
\pgfpathlineto{\pgfqpoint{2.781827in}{1.770295in}}%
\pgfpathlineto{\pgfqpoint{2.790836in}{1.861813in}}%
\pgfpathlineto{\pgfqpoint{2.802898in}{1.960412in}}%
\pgfpathlineto{\pgfqpoint{2.816600in}{2.053688in}}%
\pgfpathlineto{\pgfqpoint{2.830547in}{2.131719in}}%
\pgfpathlineto{\pgfqpoint{2.840165in}{2.175146in}}%
\pgfpathlineto{\pgfqpoint{2.854542in}{2.235156in}}%
\pgfpathlineto{\pgfqpoint{2.863814in}{2.273423in}}%
\pgfpathlineto{\pgfqpoint{2.864621in}{2.280791in}}%
\pgfpathlineto{\pgfqpoint{2.871209in}{2.301870in}}%
\pgfpathlineto{\pgfqpoint{2.878974in}{2.327756in}}%
\pgfpathlineto{\pgfqpoint{2.907455in}{2.419188in}}%
\pgfpathlineto{\pgfqpoint{2.916460in}{2.455067in}}%
\pgfpathlineto{\pgfqpoint{2.921104in}{2.481917in}}%
\pgfpathlineto{\pgfqpoint{2.922551in}{2.501750in}}%
\pgfpathlineto{\pgfqpoint{2.922047in}{2.521638in}}%
\pgfpathlineto{\pgfqpoint{2.918877in}{2.541198in}}%
\pgfpathlineto{\pgfqpoint{2.913830in}{2.557621in}}%
\pgfpathlineto{\pgfqpoint{2.906757in}{2.573035in}}%
\pgfpathlineto{\pgfqpoint{2.896667in}{2.589239in}}%
\pgfpathlineto{\pgfqpoint{2.884841in}{2.603835in}}%
\pgfpathlineto{\pgfqpoint{2.868508in}{2.620263in}}%
\pgfpathlineto{\pgfqpoint{2.849269in}{2.636532in}}%
\pgfpathlineto{\pgfqpoint{2.827267in}{2.652594in}}%
\pgfpathlineto{\pgfqpoint{2.798679in}{2.670585in}}%
\pgfpathlineto{\pgfqpoint{2.767297in}{2.687797in}}%
\pgfpathlineto{\pgfqpoint{2.737064in}{2.701836in}}%
\pgfpathlineto{\pgfqpoint{2.692036in}{2.720292in}}%
\pgfpathlineto{\pgfqpoint{2.660609in}{2.730318in}}%
\pgfpathlineto{\pgfqpoint{2.626672in}{2.739129in}}%
\pgfpathlineto{\pgfqpoint{2.622362in}{2.739175in}}%
\pgfpathlineto{\pgfqpoint{2.616071in}{2.737502in}}%
\pgfpathlineto{\pgfqpoint{2.607460in}{2.738724in}}%
\pgfpathlineto{\pgfqpoint{2.603897in}{2.741407in}}%
\pgfpathlineto{\pgfqpoint{2.602249in}{2.742991in}}%
\pgfpathlineto{\pgfqpoint{2.595789in}{2.742741in}}%
\pgfpathlineto{\pgfqpoint{2.587163in}{2.743896in}}%
\pgfpathlineto{\pgfqpoint{2.583279in}{2.746000in}}%
\pgfpathlineto{\pgfqpoint{2.580033in}{2.749041in}}%
\pgfpathlineto{\pgfqpoint{2.567070in}{2.749948in}}%
\pgfpathlineto{\pgfqpoint{2.560976in}{2.752520in}}%
\pgfpathlineto{\pgfqpoint{2.557385in}{2.755325in}}%
\pgfpathlineto{\pgfqpoint{2.546889in}{2.753211in}}%
\pgfpathlineto{\pgfqpoint{2.536117in}{2.754787in}}%
\pgfpathlineto{\pgfqpoint{2.532063in}{2.756533in}}%
\pgfpathlineto{\pgfqpoint{2.528316in}{2.758915in}}%
\pgfpathlineto{\pgfqpoint{2.524236in}{2.757483in}}%
\pgfpathlineto{\pgfqpoint{2.520031in}{2.756369in}}%
\pgfpathlineto{\pgfqpoint{2.506993in}{2.756769in}}%
\pgfpathlineto{\pgfqpoint{2.496209in}{2.758250in}}%
\pgfpathlineto{\pgfqpoint{2.491980in}{2.759256in}}%
\pgfpathlineto{\pgfqpoint{2.483300in}{2.759033in}}%
\pgfpathlineto{\pgfqpoint{2.450792in}{2.762223in}}%
\pgfpathlineto{\pgfqpoint{2.390653in}{2.770425in}}%
\pgfpathlineto{\pgfqpoint{2.388756in}{2.771596in}}%
\pgfpathlineto{\pgfqpoint{2.369358in}{2.769988in}}%
\pgfpathlineto{\pgfqpoint{2.338964in}{2.772098in}}%
\pgfpathlineto{\pgfqpoint{2.310808in}{2.773738in}}%
\pgfpathlineto{\pgfqpoint{2.302136in}{2.774286in}}%
\pgfpathlineto{\pgfqpoint{2.293476in}{2.774715in}}%
\pgfpathlineto{\pgfqpoint{2.276081in}{2.775124in}}%
\pgfpathlineto{\pgfqpoint{2.197794in}{2.777295in}}%
\pgfpathlineto{\pgfqpoint{2.106434in}{2.777745in}}%
\pgfpathlineto{\pgfqpoint{2.012921in}{2.775496in}}%
\pgfpathlineto{\pgfqpoint{1.893438in}{2.768652in}}%
\pgfpathlineto{\pgfqpoint{1.847840in}{2.765770in}}%
\pgfpathlineto{\pgfqpoint{1.841365in}{2.766377in}}%
\pgfpathlineto{\pgfqpoint{1.745914in}{2.758305in}}%
\pgfpathlineto{\pgfqpoint{1.670054in}{2.750958in}}%
\pgfpathlineto{\pgfqpoint{1.628912in}{2.746453in}}%
\pgfpathlineto{\pgfqpoint{1.618249in}{2.744213in}}%
\pgfpathlineto{\pgfqpoint{1.600959in}{2.741986in}}%
\pgfpathlineto{\pgfqpoint{1.546934in}{2.734878in}}%
\pgfpathlineto{\pgfqpoint{1.441454in}{2.717395in}}%
\pgfpathlineto{\pgfqpoint{1.392343in}{2.706480in}}%
\pgfpathlineto{\pgfqpoint{1.339317in}{2.692694in}}%
\pgfpathlineto{\pgfqpoint{1.299433in}{2.680300in}}%
\pgfpathlineto{\pgfqpoint{1.245717in}{2.660082in}}%
\pgfpathlineto{\pgfqpoint{1.200099in}{2.637675in}}%
\pgfpathlineto{\pgfqpoint{1.196466in}{2.635044in}}%
\pgfpathlineto{\pgfqpoint{1.157219in}{2.613574in}}%
\pgfpathlineto{\pgfqpoint{1.132427in}{2.598038in}}%
\pgfpathlineto{\pgfqpoint{1.117663in}{2.587532in}}%
\pgfpathlineto{\pgfqpoint{1.084530in}{2.563679in}}%
\pgfpathlineto{\pgfqpoint{1.060895in}{2.545979in}}%
\pgfpathlineto{\pgfqpoint{1.049645in}{2.538668in}}%
\pgfpathlineto{\pgfqpoint{1.022243in}{2.514130in}}%
\pgfpathlineto{\pgfqpoint{0.994390in}{2.486333in}}%
\pgfpathlineto{\pgfqpoint{0.965712in}{2.452300in}}%
\pgfpathlineto{\pgfqpoint{0.944615in}{2.423824in}}%
\pgfpathlineto{\pgfqpoint{0.922744in}{2.389715in}}%
\pgfpathlineto{\pgfqpoint{0.905596in}{2.357956in}}%
\pgfpathlineto{\pgfqpoint{0.891363in}{2.327160in}}%
\pgfpathlineto{\pgfqpoint{0.871586in}{2.277304in}}%
\pgfpathlineto{\pgfqpoint{0.856843in}{2.233135in}}%
\pgfpathlineto{\pgfqpoint{0.843714in}{2.185678in}}%
\pgfpathlineto{\pgfqpoint{0.832023in}{2.135150in}}%
\pgfpathlineto{\pgfqpoint{0.821629in}{2.081698in}}%
\pgfpathlineto{\pgfqpoint{0.821629in}{2.081698in}}%
\pgfusepath{stroke}%
\end{pgfscope}%
\begin{pgfscope}%
\pgfpathrectangle{\pgfqpoint{0.448634in}{0.402556in}}{\pgfqpoint{4.350661in}{2.489204in}} %
\pgfusepath{clip}%
\pgfsetrectcap%
\pgfsetroundjoin%
\pgfsetlinewidth{1.003750pt}%
\definecolor{currentstroke}{rgb}{0.737255,0.741176,0.133333}%
\pgfsetstrokecolor{currentstroke}%
\pgfsetdash{}{0pt}%
\pgfpathmoveto{\pgfqpoint{1.247048in}{0.524988in}}%
\pgfpathlineto{\pgfqpoint{1.347720in}{0.545137in}}%
\pgfpathlineto{\pgfqpoint{1.362006in}{0.551141in}}%
\pgfpathlineto{\pgfqpoint{1.379609in}{0.560929in}}%
\pgfpathlineto{\pgfqpoint{1.396327in}{0.572571in}}%
\pgfpathlineto{\pgfqpoint{1.415545in}{0.588869in}}%
\pgfpathlineto{\pgfqpoint{1.431683in}{0.605545in}}%
\pgfpathlineto{\pgfqpoint{1.449366in}{0.627510in}}%
\pgfpathlineto{\pgfqpoint{1.497220in}{0.692028in}}%
\pgfpathlineto{\pgfqpoint{1.515115in}{0.720213in}}%
\pgfpathlineto{\pgfqpoint{1.524954in}{0.739567in}}%
\pgfpathlineto{\pgfqpoint{1.532461in}{0.754403in}}%
\pgfpathlineto{\pgfqpoint{1.548065in}{0.763176in}}%
\pgfpathlineto{\pgfqpoint{1.556904in}{0.770402in}}%
\pgfpathlineto{\pgfqpoint{1.559408in}{0.774296in}}%
\pgfpathlineto{\pgfqpoint{1.557166in}{0.778385in}}%
\pgfpathlineto{\pgfqpoint{1.549962in}{0.783922in}}%
\pgfpathlineto{\pgfqpoint{1.539957in}{0.788770in}}%
\pgfpathlineto{\pgfqpoint{1.531689in}{0.791808in}}%
\pgfpathlineto{\pgfqpoint{1.529379in}{0.795912in}}%
\pgfpathlineto{\pgfqpoint{1.526706in}{0.813057in}}%
\pgfpathlineto{\pgfqpoint{1.525432in}{0.830400in}}%
\pgfpathlineto{\pgfqpoint{1.526737in}{0.845240in}}%
\pgfpathlineto{\pgfqpoint{1.529788in}{0.857137in}}%
\pgfpathlineto{\pgfqpoint{1.533160in}{0.860079in}}%
\pgfpathlineto{\pgfqpoint{1.566945in}{0.869626in}}%
\pgfpathlineto{\pgfqpoint{1.596916in}{0.875776in}}%
\pgfpathlineto{\pgfqpoint{1.627175in}{0.879665in}}%
\pgfpathlineto{\pgfqpoint{1.664096in}{0.882008in}}%
\pgfpathlineto{\pgfqpoint{1.727169in}{0.883264in}}%
\pgfpathlineto{\pgfqpoint{1.790224in}{0.885333in}}%
\pgfpathlineto{\pgfqpoint{1.831433in}{0.888902in}}%
\pgfpathlineto{\pgfqpoint{1.865950in}{0.893985in}}%
\pgfpathlineto{\pgfqpoint{1.898012in}{0.900888in}}%
\pgfpathlineto{\pgfqpoint{1.927518in}{0.909492in}}%
\pgfpathlineto{\pgfqpoint{1.954381in}{0.919584in}}%
\pgfpathlineto{\pgfqpoint{1.978482in}{0.931037in}}%
\pgfpathlineto{\pgfqpoint{1.997669in}{0.942738in}}%
\pgfpathlineto{\pgfqpoint{2.010167in}{0.952673in}}%
\pgfpathlineto{\pgfqpoint{2.019818in}{0.962710in}}%
\pgfpathlineto{\pgfqpoint{2.028112in}{0.974225in}}%
\pgfpathlineto{\pgfqpoint{2.035912in}{0.989171in}}%
\pgfpathlineto{\pgfqpoint{2.042725in}{1.007482in}}%
\pgfpathlineto{\pgfqpoint{2.049098in}{1.031273in}}%
\pgfpathlineto{\pgfqpoint{2.055781in}{1.065267in}}%
\pgfpathlineto{\pgfqpoint{2.079679in}{1.209525in}}%
\pgfpathlineto{\pgfqpoint{2.078866in}{1.214272in}}%
\pgfpathlineto{\pgfqpoint{2.077517in}{1.214994in}}%
\pgfpathlineto{\pgfqpoint{2.077383in}{1.213566in}}%
\pgfpathlineto{\pgfqpoint{2.074092in}{1.207204in}}%
\pgfpathlineto{\pgfqpoint{2.068080in}{1.194012in}}%
\pgfpathlineto{\pgfqpoint{2.053985in}{1.171914in}}%
\pgfpathlineto{\pgfqpoint{2.039290in}{1.153574in}}%
\pgfpathlineto{\pgfqpoint{2.021534in}{1.135234in}}%
\pgfpathlineto{\pgfqpoint{2.004203in}{1.120202in}}%
\pgfpathlineto{\pgfqpoint{1.984055in}{1.105444in}}%
\pgfpathlineto{\pgfqpoint{1.961003in}{1.091447in}}%
\pgfpathlineto{\pgfqpoint{1.939108in}{1.080414in}}%
\pgfpathlineto{\pgfqpoint{1.914562in}{1.070264in}}%
\pgfpathlineto{\pgfqpoint{1.885301in}{1.060626in}}%
\pgfpathlineto{\pgfqpoint{1.855592in}{1.052981in}}%
\pgfpathlineto{\pgfqpoint{1.814861in}{1.044981in}}%
\pgfpathlineto{\pgfqpoint{1.769532in}{1.038516in}}%
\pgfpathlineto{\pgfqpoint{1.665791in}{1.025259in}}%
\pgfpathlineto{\pgfqpoint{1.631716in}{1.017159in}}%
\pgfpathlineto{\pgfqpoint{1.597569in}{1.009492in}}%
\pgfpathlineto{\pgfqpoint{1.573788in}{1.006541in}}%
\pgfpathlineto{\pgfqpoint{1.552046in}{1.006081in}}%
\pgfpathlineto{\pgfqpoint{1.528549in}{1.005895in}}%
\pgfpathlineto{\pgfqpoint{1.504423in}{0.984637in}}%
\pgfpathlineto{\pgfqpoint{1.477027in}{0.960094in}}%
\pgfpathlineto{\pgfqpoint{1.463150in}{0.944310in}}%
\pgfpathlineto{\pgfqpoint{1.451042in}{0.926721in}}%
\pgfpathlineto{\pgfqpoint{1.439379in}{0.905718in}}%
\pgfpathlineto{\pgfqpoint{1.428267in}{0.881479in}}%
\pgfpathlineto{\pgfqpoint{1.421977in}{0.862929in}}%
\pgfpathlineto{\pgfqpoint{1.419389in}{0.848316in}}%
\pgfpathlineto{\pgfqpoint{1.419664in}{0.835904in}}%
\pgfpathlineto{\pgfqpoint{1.422471in}{0.823904in}}%
\pgfpathlineto{\pgfqpoint{1.435320in}{0.784425in}}%
\pgfpathlineto{\pgfqpoint{1.435264in}{0.772007in}}%
\pgfpathlineto{\pgfqpoint{1.432934in}{0.759868in}}%
\pgfpathlineto{\pgfqpoint{1.428034in}{0.746041in}}%
\pgfpathlineto{\pgfqpoint{1.419398in}{0.728764in}}%
\pgfpathlineto{\pgfqpoint{1.402007in}{0.700162in}}%
\pgfpathlineto{\pgfqpoint{1.350141in}{0.620218in}}%
\pgfpathlineto{\pgfqpoint{1.333583in}{0.597143in}}%
\pgfpathlineto{\pgfqpoint{1.319440in}{0.581672in}}%
\pgfpathlineto{\pgfqpoint{1.305545in}{0.569700in}}%
\pgfpathlineto{\pgfqpoint{1.290512in}{0.559691in}}%
\pgfpathlineto{\pgfqpoint{1.272614in}{0.550635in}}%
\pgfpathlineto{\pgfqpoint{1.256032in}{0.544635in}}%
\pgfpathlineto{\pgfqpoint{1.236802in}{0.540464in}}%
\pgfpathlineto{\pgfqpoint{1.163756in}{0.527200in}}%
\pgfpathlineto{\pgfqpoint{1.163756in}{0.527200in}}%
\pgfusepath{stroke}%
\end{pgfscope}%
\begin{pgfscope}%
\pgfpathrectangle{\pgfqpoint{0.448634in}{0.402556in}}{\pgfqpoint{4.350661in}{2.489204in}} %
\pgfusepath{clip}%
\pgfsetrectcap%
\pgfsetroundjoin%
\pgfsetlinewidth{1.003750pt}%
\definecolor{currentstroke}{rgb}{0.737255,0.741176,0.133333}%
\pgfsetstrokecolor{currentstroke}%
\pgfsetdash{}{0pt}%
\pgfpathmoveto{\pgfqpoint{3.716417in}{0.402556in}}%
\pgfpathlineto{\pgfqpoint{0.449071in}{0.402556in}}%
\pgfpathlineto{\pgfqpoint{0.449071in}{0.402556in}}%
\pgfusepath{stroke}%
\end{pgfscope}%
\begin{pgfscope}%
\pgfpathrectangle{\pgfqpoint{0.448634in}{0.402556in}}{\pgfqpoint{4.350661in}{2.489204in}} %
\pgfusepath{clip}%
\pgfsetrectcap%
\pgfsetroundjoin%
\pgfsetlinewidth{1.003750pt}%
\definecolor{currentstroke}{rgb}{0.737255,0.741176,0.133333}%
\pgfsetstrokecolor{currentstroke}%
\pgfsetdash{}{0pt}%
\pgfpathmoveto{\pgfqpoint{4.219814in}{0.402563in}}%
\pgfpathlineto{\pgfqpoint{3.641179in}{0.403624in}}%
\pgfpathlineto{\pgfqpoint{3.560722in}{0.405872in}}%
\pgfpathlineto{\pgfqpoint{3.549977in}{0.407639in}}%
\pgfpathlineto{\pgfqpoint{3.544197in}{0.410493in}}%
\pgfpathlineto{\pgfqpoint{3.540488in}{0.407950in}}%
\pgfpathlineto{\pgfqpoint{3.531920in}{0.406367in}}%
\pgfpathlineto{\pgfqpoint{3.508005in}{0.405570in}}%
\pgfpathlineto{\pgfqpoint{3.431870in}{0.405933in}}%
\pgfpathlineto{\pgfqpoint{3.207839in}{0.409405in}}%
\pgfpathlineto{\pgfqpoint{3.136112in}{0.412625in}}%
\pgfpathlineto{\pgfqpoint{3.086232in}{0.417040in}}%
\pgfpathlineto{\pgfqpoint{3.034423in}{0.424381in}}%
\pgfpathlineto{\pgfqpoint{2.909627in}{0.445394in}}%
\pgfpathlineto{\pgfqpoint{2.890738in}{0.451260in}}%
\pgfpathlineto{\pgfqpoint{2.874380in}{0.458027in}}%
\pgfpathlineto{\pgfqpoint{2.856864in}{0.468008in}}%
\pgfpathlineto{\pgfqpoint{2.840460in}{0.480224in}}%
\pgfpathlineto{\pgfqpoint{2.811824in}{0.502766in}}%
\pgfpathlineto{\pgfqpoint{2.803610in}{0.505907in}}%
\pgfpathlineto{\pgfqpoint{2.799279in}{0.505744in}}%
\pgfpathlineto{\pgfqpoint{2.795275in}{0.503878in}}%
\pgfpathlineto{\pgfqpoint{2.790705in}{0.498623in}}%
\pgfpathlineto{\pgfqpoint{2.786789in}{0.489752in}}%
\pgfpathlineto{\pgfqpoint{2.783444in}{0.475330in}}%
\pgfpathlineto{\pgfqpoint{2.780462in}{0.450680in}}%
\pgfpathlineto{\pgfqpoint{2.776739in}{0.418679in}}%
\pgfpathlineto{\pgfqpoint{2.774135in}{0.414784in}}%
\pgfpathlineto{\pgfqpoint{2.770012in}{0.413476in}}%
\pgfpathlineto{\pgfqpoint{2.765813in}{0.414643in}}%
\pgfpathlineto{\pgfqpoint{2.762311in}{0.417552in}}%
\pgfpathlineto{\pgfqpoint{2.758896in}{0.423867in}}%
\pgfpathlineto{\pgfqpoint{2.756533in}{0.433431in}}%
\pgfpathlineto{\pgfqpoint{2.754519in}{0.453200in}}%
\pgfpathlineto{\pgfqpoint{2.753204in}{0.492994in}}%
\pgfpathlineto{\pgfqpoint{2.752905in}{0.582603in}}%
\pgfpathlineto{\pgfqpoint{2.754751in}{0.749365in}}%
\pgfpathlineto{\pgfqpoint{2.759348in}{0.948430in}}%
\pgfpathlineto{\pgfqpoint{2.766981in}{1.167302in}}%
\pgfpathlineto{\pgfqpoint{2.779760in}{1.507999in}}%
\pgfpathlineto{\pgfqpoint{2.779601in}{1.567733in}}%
\pgfpathlineto{\pgfqpoint{2.778015in}{1.592543in}}%
\pgfpathlineto{\pgfqpoint{2.776499in}{1.597173in}}%
\pgfpathlineto{\pgfqpoint{2.774913in}{1.598841in}}%
\pgfpathlineto{\pgfqpoint{2.772816in}{1.599216in}}%
\pgfpathlineto{\pgfqpoint{2.771735in}{1.597999in}}%
\pgfpathlineto{\pgfqpoint{2.771205in}{1.593134in}}%
\pgfpathlineto{\pgfqpoint{2.773005in}{1.521016in}}%
\pgfpathlineto{\pgfqpoint{2.771464in}{1.516447in}}%
\pgfpathlineto{\pgfqpoint{2.767799in}{1.513926in}}%
\pgfpathlineto{\pgfqpoint{2.767239in}{1.511564in}}%
\pgfpathlineto{\pgfqpoint{2.769706in}{1.507760in}}%
\pgfpathlineto{\pgfqpoint{2.772527in}{1.504134in}}%
\pgfpathlineto{\pgfqpoint{2.774128in}{1.494383in}}%
\pgfpathlineto{\pgfqpoint{2.775402in}{1.459571in}}%
\pgfpathlineto{\pgfqpoint{2.774898in}{1.404816in}}%
\pgfpathlineto{\pgfqpoint{2.770706in}{1.282940in}}%
\pgfpathlineto{\pgfqpoint{2.759189in}{0.972073in}}%
\pgfpathlineto{\pgfqpoint{2.754288in}{0.780488in}}%
\pgfpathlineto{\pgfqpoint{2.750721in}{0.556498in}}%
\pgfpathlineto{\pgfqpoint{2.748176in}{0.454492in}}%
\pgfpathlineto{\pgfqpoint{2.746015in}{0.434748in}}%
\pgfpathlineto{\pgfqpoint{2.743089in}{0.425424in}}%
\pgfpathlineto{\pgfqpoint{2.741449in}{0.423837in}}%
\pgfpathlineto{\pgfqpoint{2.737297in}{0.424493in}}%
\pgfpathlineto{\pgfqpoint{2.734495in}{0.428220in}}%
\pgfpathlineto{\pgfqpoint{2.728541in}{0.441462in}}%
\pgfpathlineto{\pgfqpoint{2.726487in}{0.453669in}}%
\pgfpathlineto{\pgfqpoint{2.724718in}{0.480968in}}%
\pgfpathlineto{\pgfqpoint{2.723857in}{0.535719in}}%
\pgfpathlineto{\pgfqpoint{2.724418in}{0.682578in}}%
\pgfpathlineto{\pgfqpoint{2.727851in}{0.876695in}}%
\pgfpathlineto{\pgfqpoint{2.732050in}{1.013479in}}%
\pgfpathlineto{\pgfqpoint{2.739348in}{1.207454in}}%
\pgfpathlineto{\pgfqpoint{2.749688in}{1.403737in}}%
\pgfpathlineto{\pgfqpoint{2.760651in}{1.557548in}}%
\pgfpathlineto{\pgfqpoint{2.767736in}{1.634176in}}%
\pgfpathlineto{\pgfqpoint{2.768224in}{1.638999in}}%
\pgfpathlineto{\pgfqpoint{2.767266in}{1.646340in}}%
\pgfpathlineto{\pgfqpoint{2.768840in}{1.673648in}}%
\pgfpathlineto{\pgfqpoint{2.793684in}{1.913389in}}%
\pgfpathlineto{\pgfqpoint{2.803128in}{1.984760in}}%
\pgfpathlineto{\pgfqpoint{2.818926in}{2.087725in}}%
\pgfpathlineto{\pgfqpoint{2.830092in}{2.146055in}}%
\pgfpathlineto{\pgfqpoint{2.859028in}{2.276030in}}%
\pgfpathlineto{\pgfqpoint{2.862085in}{2.287959in}}%
\pgfpathlineto{\pgfqpoint{2.874419in}{2.333096in}}%
\pgfpathlineto{\pgfqpoint{2.894228in}{2.409448in}}%
\pgfpathlineto{\pgfqpoint{2.901181in}{2.443369in}}%
\pgfpathlineto{\pgfqpoint{2.904968in}{2.472910in}}%
\pgfpathlineto{\pgfqpoint{2.905625in}{2.495288in}}%
\pgfpathlineto{\pgfqpoint{2.904051in}{2.515105in}}%
\pgfpathlineto{\pgfqpoint{2.899907in}{2.534429in}}%
\pgfpathlineto{\pgfqpoint{2.893179in}{2.552781in}}%
\pgfpathlineto{\pgfqpoint{2.885372in}{2.567710in}}%
\pgfpathlineto{\pgfqpoint{2.873252in}{2.585277in}}%
\pgfpathlineto{\pgfqpoint{2.859242in}{2.600912in}}%
\pgfpathlineto{\pgfqpoint{2.842393in}{2.616624in}}%
\pgfpathlineto{\pgfqpoint{2.804118in}{2.645033in}}%
\pgfpathlineto{\pgfqpoint{2.779044in}{2.659972in}}%
\pgfpathlineto{\pgfqpoint{2.695001in}{2.700805in}}%
\pgfpathlineto{\pgfqpoint{2.634735in}{2.722073in}}%
\pgfpathlineto{\pgfqpoint{2.598896in}{2.732488in}}%
\pgfpathlineto{\pgfqpoint{2.569120in}{2.739792in}}%
\pgfpathlineto{\pgfqpoint{2.519843in}{2.749669in}}%
\pgfpathlineto{\pgfqpoint{2.465971in}{2.758151in}}%
\pgfpathlineto{\pgfqpoint{2.440140in}{2.762034in}}%
\pgfpathlineto{\pgfqpoint{2.436108in}{2.763808in}}%
\pgfpathlineto{\pgfqpoint{2.390674in}{2.768663in}}%
\pgfpathlineto{\pgfqpoint{2.386400in}{2.769181in}}%
\pgfpathlineto{\pgfqpoint{2.375536in}{2.768891in}}%
\pgfpathlineto{\pgfqpoint{2.349483in}{2.770672in}}%
\pgfpathlineto{\pgfqpoint{2.314814in}{2.773945in}}%
\pgfpathlineto{\pgfqpoint{2.304019in}{2.773371in}}%
\pgfpathlineto{\pgfqpoint{2.293219in}{2.773723in}}%
\pgfpathlineto{\pgfqpoint{2.278001in}{2.773311in}}%
\pgfpathlineto{\pgfqpoint{2.217111in}{2.772033in}}%
\pgfpathlineto{\pgfqpoint{2.149749in}{2.768475in}}%
\pgfpathlineto{\pgfqpoint{1.908524in}{2.756438in}}%
\pgfpathlineto{\pgfqpoint{1.795735in}{2.746651in}}%
\pgfpathlineto{\pgfqpoint{1.741822in}{2.739025in}}%
\pgfpathlineto{\pgfqpoint{1.720863in}{2.733580in}}%
\pgfpathlineto{\pgfqpoint{1.719884in}{2.731375in}}%
\pgfpathlineto{\pgfqpoint{1.718104in}{2.730133in}}%
\pgfpathlineto{\pgfqpoint{1.717060in}{2.733314in}}%
\pgfpathlineto{\pgfqpoint{1.713495in}{2.735832in}}%
\pgfpathlineto{\pgfqpoint{1.711377in}{2.736003in}}%
\pgfpathlineto{\pgfqpoint{1.711574in}{2.735088in}}%
\pgfpathlineto{\pgfqpoint{1.714796in}{2.729316in}}%
\pgfpathlineto{\pgfqpoint{1.719004in}{2.728681in}}%
\pgfpathlineto{\pgfqpoint{1.722459in}{2.731569in}}%
\pgfpathlineto{\pgfqpoint{1.725903in}{2.734482in}}%
\pgfpathlineto{\pgfqpoint{1.734385in}{2.736621in}}%
\pgfpathlineto{\pgfqpoint{1.749226in}{2.740448in}}%
\pgfpathlineto{\pgfqpoint{1.757771in}{2.742151in}}%
\pgfpathlineto{\pgfqpoint{1.913855in}{2.756807in}}%
\pgfpathlineto{\pgfqpoint{2.018138in}{2.762763in}}%
\pgfpathlineto{\pgfqpoint{2.302836in}{2.773483in}}%
\pgfpathlineto{\pgfqpoint{2.311501in}{2.773280in}}%
\pgfpathlineto{\pgfqpoint{2.315791in}{2.773960in}}%
\pgfpathlineto{\pgfqpoint{2.374334in}{2.768895in}}%
\pgfpathlineto{\pgfqpoint{2.395886in}{2.767424in}}%
\pgfpathlineto{\pgfqpoint{2.432746in}{2.764026in}}%
\pgfpathlineto{\pgfqpoint{2.439081in}{2.762627in}}%
\pgfpathlineto{\pgfqpoint{2.443160in}{2.761019in}}%
\pgfpathlineto{\pgfqpoint{2.507898in}{2.751644in}}%
\pgfpathlineto{\pgfqpoint{2.563652in}{2.740808in}}%
\pgfpathlineto{\pgfqpoint{2.601901in}{2.731261in}}%
\pgfpathlineto{\pgfqpoint{2.656504in}{2.714456in}}%
\pgfpathlineto{\pgfqpoint{2.699978in}{2.698433in}}%
\pgfpathlineto{\pgfqpoint{2.738358in}{2.680917in}}%
\pgfpathlineto{\pgfqpoint{2.777884in}{2.660120in}}%
\pgfpathlineto{\pgfqpoint{2.797296in}{2.648900in}}%
\pgfpathlineto{\pgfqpoint{2.827207in}{2.628568in}}%
\pgfpathlineto{\pgfqpoint{2.854928in}{2.604530in}}%
\pgfpathlineto{\pgfqpoint{2.870773in}{2.587491in}}%
\pgfpathlineto{\pgfqpoint{2.881950in}{2.572246in}}%
\pgfpathlineto{\pgfqpoint{2.892361in}{2.553313in}}%
\pgfpathlineto{\pgfqpoint{2.898437in}{2.537347in}}%
\pgfpathlineto{\pgfqpoint{2.902129in}{2.523035in}}%
\pgfpathlineto{\pgfqpoint{2.904587in}{2.505853in}}%
\pgfpathlineto{\pgfqpoint{2.905185in}{2.485963in}}%
\pgfpathlineto{\pgfqpoint{2.903683in}{2.463636in}}%
\pgfpathlineto{\pgfqpoint{2.899175in}{2.434223in}}%
\pgfpathlineto{\pgfqpoint{2.890007in}{2.393234in}}%
\pgfpathlineto{\pgfqpoint{2.869687in}{2.317051in}}%
\pgfpathlineto{\pgfqpoint{2.858554in}{2.274152in}}%
\pgfpathlineto{\pgfqpoint{2.857622in}{2.266818in}}%
\pgfpathlineto{\pgfqpoint{2.836837in}{2.180441in}}%
\pgfpathlineto{\pgfqpoint{2.821511in}{2.105315in}}%
\pgfpathlineto{\pgfqpoint{2.810648in}{2.039268in}}%
\pgfpathlineto{\pgfqpoint{2.798814in}{1.955728in}}%
\pgfpathlineto{\pgfqpoint{2.788773in}{1.874397in}}%
\pgfpathlineto{\pgfqpoint{2.775725in}{1.750840in}}%
\pgfpathlineto{\pgfqpoint{2.767038in}{1.649310in}}%
\pgfpathlineto{\pgfqpoint{2.767566in}{1.639412in}}%
\pgfpathlineto{\pgfqpoint{2.767308in}{1.634522in}}%
\pgfpathlineto{\pgfqpoint{2.766074in}{1.622226in}}%
\pgfpathlineto{\pgfqpoint{2.765090in}{1.607369in}}%
\pgfpathlineto{\pgfqpoint{2.761649in}{1.572750in}}%
\pgfpathlineto{\pgfqpoint{2.743618in}{1.299734in}}%
\pgfpathlineto{\pgfqpoint{2.738183in}{1.180417in}}%
\pgfpathlineto{\pgfqpoint{2.731972in}{1.011303in}}%
\pgfpathlineto{\pgfqpoint{2.725962in}{0.784931in}}%
\pgfpathlineto{\pgfqpoint{2.723921in}{0.630619in}}%
\pgfpathlineto{\pgfqpoint{2.724137in}{0.503672in}}%
\pgfpathlineto{\pgfqpoint{2.725961in}{0.458926in}}%
\pgfpathlineto{\pgfqpoint{2.728408in}{0.441753in}}%
\pgfpathlineto{\pgfqpoint{2.732091in}{0.432772in}}%
\pgfpathlineto{\pgfqpoint{2.736985in}{0.424630in}}%
\pgfpathlineto{\pgfqpoint{2.741103in}{0.423603in}}%
\pgfpathlineto{\pgfqpoint{2.743965in}{0.427140in}}%
\pgfpathlineto{\pgfqpoint{2.746386in}{0.436679in}}%
\pgfpathlineto{\pgfqpoint{2.748312in}{0.456459in}}%
\pgfpathlineto{\pgfqpoint{2.749868in}{0.503717in}}%
\pgfpathlineto{\pgfqpoint{2.761934in}{1.053640in}}%
\pgfpathlineto{\pgfqpoint{2.775301in}{1.466521in}}%
\pgfpathlineto{\pgfqpoint{2.773708in}{1.498817in}}%
\pgfpathlineto{\pgfqpoint{2.771766in}{1.505882in}}%
\pgfpathlineto{\pgfqpoint{2.767282in}{1.511005in}}%
\pgfpathlineto{\pgfqpoint{2.767556in}{1.513436in}}%
\pgfpathlineto{\pgfqpoint{2.772928in}{1.520431in}}%
\pgfpathlineto{\pgfqpoint{2.773326in}{1.532853in}}%
\pgfpathlineto{\pgfqpoint{2.773024in}{1.599097in}}%
\pgfpathlineto{\pgfqpoint{2.774490in}{1.599042in}}%
\pgfpathlineto{\pgfqpoint{2.777288in}{1.595416in}}%
\pgfpathlineto{\pgfqpoint{2.778730in}{1.585633in}}%
\pgfpathlineto{\pgfqpoint{2.779907in}{1.553307in}}%
\pgfpathlineto{\pgfqpoint{2.779547in}{1.496060in}}%
\pgfpathlineto{\pgfqpoint{2.776301in}{1.391581in}}%
\pgfpathlineto{\pgfqpoint{2.757000in}{0.856874in}}%
\pgfpathlineto{\pgfqpoint{2.753620in}{0.672715in}}%
\pgfpathlineto{\pgfqpoint{2.752911in}{0.520878in}}%
\pgfpathlineto{\pgfqpoint{2.754343in}{0.456187in}}%
\pgfpathlineto{\pgfqpoint{2.756875in}{0.431486in}}%
\pgfpathlineto{\pgfqpoint{2.759612in}{0.422062in}}%
\pgfpathlineto{\pgfqpoint{2.763588in}{0.416217in}}%
\pgfpathlineto{\pgfqpoint{2.767427in}{0.413941in}}%
\pgfpathlineto{\pgfqpoint{2.771730in}{0.413629in}}%
\pgfpathlineto{\pgfqpoint{2.775416in}{0.416095in}}%
\pgfpathlineto{\pgfqpoint{2.777318in}{0.420534in}}%
\pgfpathlineto{\pgfqpoint{2.778958in}{0.432820in}}%
\pgfpathlineto{\pgfqpoint{2.783791in}{0.477260in}}%
\pgfpathlineto{\pgfqpoint{2.787435in}{0.491578in}}%
\pgfpathlineto{\pgfqpoint{2.791728in}{0.500207in}}%
\pgfpathlineto{\pgfqpoint{2.796783in}{0.504828in}}%
\pgfpathlineto{\pgfqpoint{2.800987in}{0.505995in}}%
\pgfpathlineto{\pgfqpoint{2.807402in}{0.504906in}}%
\pgfpathlineto{\pgfqpoint{2.817155in}{0.499445in}}%
\pgfpathlineto{\pgfqpoint{2.833075in}{0.486414in}}%
\pgfpathlineto{\pgfqpoint{2.856475in}{0.468273in}}%
\pgfpathlineto{\pgfqpoint{2.871980in}{0.459250in}}%
\pgfpathlineto{\pgfqpoint{2.888230in}{0.452153in}}%
\pgfpathlineto{\pgfqpoint{2.911312in}{0.444974in}}%
\pgfpathlineto{\pgfqpoint{2.934856in}{0.440116in}}%
\pgfpathlineto{\pgfqpoint{2.999522in}{0.430088in}}%
\pgfpathlineto{\pgfqpoint{3.090113in}{0.416594in}}%
\pgfpathlineto{\pgfqpoint{3.133490in}{0.412801in}}%
\pgfpathlineto{\pgfqpoint{3.192167in}{0.409891in}}%
\pgfpathlineto{\pgfqpoint{3.318305in}{0.406869in}}%
\pgfpathlineto{\pgfqpoint{3.527127in}{0.406054in}}%
\pgfpathlineto{\pgfqpoint{3.540057in}{0.407800in}}%
\pgfpathlineto{\pgfqpoint{3.547870in}{0.410868in}}%
\pgfpathlineto{\pgfqpoint{3.560911in}{0.410941in}}%
\pgfpathlineto{\pgfqpoint{3.573830in}{0.412954in}}%
\pgfpathlineto{\pgfqpoint{3.595287in}{0.416979in}}%
\pgfpathlineto{\pgfqpoint{3.630000in}{0.419854in}}%
\pgfpathlineto{\pgfqpoint{3.677813in}{0.422135in}}%
\pgfpathlineto{\pgfqpoint{3.862664in}{0.426799in}}%
\pgfpathlineto{\pgfqpoint{4.252047in}{0.427001in}}%
\pgfpathlineto{\pgfqpoint{4.532664in}{0.426865in}}%
\pgfpathlineto{\pgfqpoint{4.721904in}{0.429089in}}%
\pgfpathlineto{\pgfqpoint{4.752290in}{0.431295in}}%
\pgfpathlineto{\pgfqpoint{4.769426in}{0.434608in}}%
\pgfpathlineto{\pgfqpoint{4.775273in}{0.437852in}}%
\pgfpathlineto{\pgfqpoint{4.777643in}{0.441884in}}%
\pgfpathlineto{\pgfqpoint{4.776384in}{0.446415in}}%
\pgfpathlineto{\pgfqpoint{4.770542in}{0.449608in}}%
\pgfpathlineto{\pgfqpoint{4.757729in}{0.452400in}}%
\pgfpathlineto{\pgfqpoint{4.740663in}{0.456245in}}%
\pgfpathlineto{\pgfqpoint{4.730423in}{0.460384in}}%
\pgfpathlineto{\pgfqpoint{4.723045in}{0.465619in}}%
\pgfpathlineto{\pgfqpoint{4.717146in}{0.472892in}}%
\pgfpathlineto{\pgfqpoint{4.714297in}{0.479579in}}%
\pgfpathlineto{\pgfqpoint{4.713032in}{0.486881in}}%
\pgfpathlineto{\pgfqpoint{4.713654in}{0.494288in}}%
\pgfpathlineto{\pgfqpoint{4.716757in}{0.503569in}}%
\pgfpathlineto{\pgfqpoint{4.725604in}{0.520702in}}%
\pgfpathlineto{\pgfqpoint{4.736718in}{0.542083in}}%
\pgfpathlineto{\pgfqpoint{4.745199in}{0.564977in}}%
\pgfpathlineto{\pgfqpoint{4.749934in}{0.586680in}}%
\pgfpathlineto{\pgfqpoint{4.754164in}{0.626181in}}%
\pgfpathlineto{\pgfqpoint{4.755710in}{0.668435in}}%
\pgfpathlineto{\pgfqpoint{4.763161in}{0.714928in}}%
\pgfpathlineto{\pgfqpoint{4.768719in}{0.741552in}}%
\pgfpathlineto{\pgfqpoint{4.781732in}{0.804501in}}%
\pgfpathlineto{\pgfqpoint{4.784898in}{0.818969in}}%
\pgfpathlineto{\pgfqpoint{4.787010in}{0.828595in}}%
\pgfpathlineto{\pgfqpoint{4.788694in}{0.852999in}}%
\pgfpathlineto{\pgfqpoint{4.792626in}{0.858793in}}%
\pgfpathlineto{\pgfqpoint{4.794023in}{0.868601in}}%
\pgfpathlineto{\pgfqpoint{4.796505in}{0.915804in}}%
\pgfpathlineto{\pgfqpoint{4.795858in}{0.933198in}}%
\pgfpathlineto{\pgfqpoint{4.790604in}{0.985054in}}%
\pgfpathlineto{\pgfqpoint{4.788002in}{1.049680in}}%
\pgfpathlineto{\pgfqpoint{4.789962in}{1.101901in}}%
\pgfpathlineto{\pgfqpoint{4.790156in}{1.124290in}}%
\pgfpathlineto{\pgfqpoint{4.788762in}{1.178990in}}%
\pgfpathlineto{\pgfqpoint{4.790844in}{1.206257in}}%
\pgfpathlineto{\pgfqpoint{4.793708in}{1.245941in}}%
\pgfpathlineto{\pgfqpoint{4.794857in}{1.300684in}}%
\pgfpathlineto{\pgfqpoint{4.795097in}{1.415163in}}%
\pgfpathlineto{\pgfqpoint{4.793885in}{1.427484in}}%
\pgfpathlineto{\pgfqpoint{4.797540in}{1.433375in}}%
\pgfpathlineto{\pgfqpoint{4.798332in}{1.450765in}}%
\pgfpathlineto{\pgfqpoint{4.798897in}{1.522946in}}%
\pgfpathlineto{\pgfqpoint{4.799156in}{1.893837in}}%
\pgfpathlineto{\pgfqpoint{4.797854in}{2.448920in}}%
\pgfpathlineto{\pgfqpoint{4.795387in}{2.508587in}}%
\pgfpathlineto{\pgfqpoint{4.788385in}{2.622748in}}%
\pgfpathlineto{\pgfqpoint{4.788908in}{2.682478in}}%
\pgfpathlineto{\pgfqpoint{4.790752in}{2.729651in}}%
\pgfpathlineto{\pgfqpoint{4.788554in}{2.786800in}}%
\pgfpathlineto{\pgfqpoint{4.788634in}{2.881375in}}%
\pgfpathlineto{\pgfqpoint{4.788023in}{2.883788in}}%
\pgfpathlineto{\pgfqpoint{4.790028in}{2.882978in}}%
\pgfpathlineto{\pgfqpoint{4.793311in}{2.876617in}}%
\pgfpathlineto{\pgfqpoint{4.795843in}{2.861975in}}%
\pgfpathlineto{\pgfqpoint{4.796989in}{2.842112in}}%
\pgfpathlineto{\pgfqpoint{4.798261in}{2.797336in}}%
\pgfpathlineto{\pgfqpoint{4.799152in}{2.635542in}}%
\pgfpathlineto{\pgfqpoint{4.799293in}{2.147658in}}%
\pgfpathlineto{\pgfqpoint{4.798630in}{0.405385in}}%
\pgfpathlineto{\pgfqpoint{4.794712in}{0.403458in}}%
\pgfpathlineto{\pgfqpoint{4.788209in}{0.402907in}}%
\pgfpathlineto{\pgfqpoint{4.788209in}{0.402907in}}%
\pgfusepath{stroke}%
\end{pgfscope}%
\begin{pgfscope}%
\pgfpathrectangle{\pgfqpoint{0.448634in}{0.402556in}}{\pgfqpoint{4.350661in}{2.489204in}} %
\pgfusepath{clip}%
\pgfsetrectcap%
\pgfsetroundjoin%
\pgfsetlinewidth{1.003750pt}%
\definecolor{currentstroke}{rgb}{0.737255,0.741176,0.133333}%
\pgfsetstrokecolor{currentstroke}%
\pgfsetdash{}{0pt}%
\pgfpathmoveto{\pgfqpoint{4.799283in}{2.895751in}}%
\pgfpathlineto{\pgfqpoint{4.799294in}{2.315766in}}%
\pgfpathlineto{\pgfqpoint{4.799207in}{0.406559in}}%
\pgfpathlineto{\pgfqpoint{4.797913in}{0.404639in}}%
\pgfpathlineto{\pgfqpoint{4.793761in}{0.403284in}}%
\pgfpathlineto{\pgfqpoint{4.787247in}{0.402868in}}%
\pgfpathlineto{\pgfqpoint{4.787247in}{0.402868in}}%
\pgfusepath{stroke}%
\end{pgfscope}%
\begin{pgfscope}%
\pgfpathrectangle{\pgfqpoint{0.448634in}{0.402556in}}{\pgfqpoint{4.350661in}{2.489204in}} %
\pgfusepath{clip}%
\pgfsetbuttcap%
\pgfsetroundjoin%
\pgfsetlinewidth{1.003750pt}%
\definecolor{currentstroke}{rgb}{0.000000,0.000000,0.000000}%
\pgfsetstrokecolor{currentstroke}%
\pgfsetdash{{1.000000pt}{1.650000pt}}{0.000000pt}%
\pgfpathmoveto{\pgfqpoint{1.127319in}{2.572074in}}%
\pgfpathlineto{\pgfqpoint{1.159575in}{2.592758in}}%
\pgfpathlineto{\pgfqpoint{1.192763in}{2.611414in}}%
\pgfpathlineto{\pgfqpoint{1.228726in}{2.629126in}}%
\pgfpathlineto{\pgfqpoint{1.267413in}{2.645758in}}%
\pgfpathlineto{\pgfqpoint{1.310846in}{2.661945in}}%
\pgfpathlineto{\pgfqpoint{1.356920in}{2.676740in}}%
\pgfpathlineto{\pgfqpoint{1.407680in}{2.690702in}}%
\pgfpathlineto{\pgfqpoint{1.463094in}{2.703640in}}%
\pgfpathlineto{\pgfqpoint{1.525273in}{2.715813in}}%
\pgfpathlineto{\pgfqpoint{1.594199in}{2.726937in}}%
\pgfpathlineto{\pgfqpoint{1.669843in}{2.736808in}}%
\pgfpathlineto{\pgfqpoint{1.752172in}{2.745271in}}%
\pgfpathlineto{\pgfqpoint{1.843325in}{2.752344in}}%
\pgfpathlineto{\pgfqpoint{1.941103in}{2.757656in}}%
\pgfpathlineto{\pgfqpoint{2.043301in}{2.760987in}}%
\pgfpathlineto{\pgfqpoint{2.147710in}{2.762199in}}%
\pgfpathlineto{\pgfqpoint{2.249945in}{2.761215in}}%
\pgfpathlineto{\pgfqpoint{2.345620in}{2.758145in}}%
\pgfpathlineto{\pgfqpoint{2.432525in}{2.753210in}}%
\pgfpathlineto{\pgfqpoint{2.508451in}{2.746766in}}%
\pgfpathlineto{\pgfqpoint{2.573368in}{2.739156in}}%
\pgfpathlineto{\pgfqpoint{2.629410in}{2.730451in}}%
\pgfpathlineto{\pgfqpoint{2.676543in}{2.720985in}}%
\pgfpathlineto{\pgfqpoint{2.716874in}{2.710666in}}%
\pgfpathlineto{\pgfqpoint{2.750366in}{2.699848in}}%
\pgfpathlineto{\pgfqpoint{2.779059in}{2.688192in}}%
\pgfpathlineto{\pgfqpoint{2.802882in}{2.676004in}}%
\pgfpathlineto{\pgfqpoint{2.821842in}{2.663819in}}%
\pgfpathlineto{\pgfqpoint{2.837815in}{2.650886in}}%
\pgfpathlineto{\pgfqpoint{2.850736in}{2.637564in}}%
\pgfpathlineto{\pgfqpoint{2.860694in}{2.624398in}}%
\pgfpathlineto{\pgfqpoint{2.869084in}{2.609873in}}%
\pgfpathlineto{\pgfqpoint{2.875698in}{2.594192in}}%
\pgfpathlineto{\pgfqpoint{2.881035in}{2.575255in}}%
\pgfpathlineto{\pgfqpoint{2.884200in}{2.555685in}}%
\pgfpathlineto{\pgfqpoint{2.885619in}{2.533351in}}%
\pgfpathlineto{\pgfqpoint{2.885038in}{2.505987in}}%
\pgfpathlineto{\pgfqpoint{2.882112in}{2.473807in}}%
\pgfpathlineto{\pgfqpoint{2.875657in}{2.429620in}}%
\pgfpathlineto{\pgfqpoint{2.863489in}{2.363873in}}%
\pgfpathlineto{\pgfqpoint{2.821102in}{2.142619in}}%
\pgfpathlineto{\pgfqpoint{2.804859in}{2.042271in}}%
\pgfpathlineto{\pgfqpoint{2.790421in}{1.939040in}}%
\pgfpathlineto{\pgfqpoint{2.777207in}{1.828054in}}%
\pgfpathlineto{\pgfqpoint{2.765338in}{1.709349in}}%
\pgfpathlineto{\pgfqpoint{2.754471in}{1.578010in}}%
\pgfpathlineto{\pgfqpoint{2.744640in}{1.431580in}}%
\pgfpathlineto{\pgfqpoint{2.735914in}{1.267598in}}%
\pgfpathlineto{\pgfqpoint{2.728277in}{1.081114in}}%
\pgfpathlineto{\pgfqpoint{2.721437in}{0.857223in}}%
\pgfpathlineto{\pgfqpoint{2.711961in}{0.541290in}}%
\pgfpathlineto{\pgfqpoint{2.708250in}{0.491694in}}%
\pgfpathlineto{\pgfqpoint{2.703951in}{0.462246in}}%
\pgfpathlineto{\pgfqpoint{2.699504in}{0.445599in}}%
\pgfpathlineto{\pgfqpoint{2.694517in}{0.434563in}}%
\pgfpathlineto{\pgfqpoint{2.688942in}{0.426947in}}%
\pgfpathlineto{\pgfqpoint{2.681980in}{0.421009in}}%
\pgfpathlineto{\pgfqpoint{2.672064in}{0.415948in}}%
\pgfpathlineto{\pgfqpoint{2.659429in}{0.412247in}}%
\pgfpathlineto{\pgfqpoint{2.640044in}{0.409163in}}%
\pgfpathlineto{\pgfqpoint{2.607490in}{0.406692in}}%
\pgfpathlineto{\pgfqpoint{2.548779in}{0.404894in}}%
\pgfpathlineto{\pgfqpoint{2.422615in}{0.403701in}}%
\pgfpathlineto{\pgfqpoint{2.026705in}{0.403016in}}%
\pgfpathlineto{\pgfqpoint{0.623617in}{0.403253in}}%
\pgfpathlineto{\pgfqpoint{0.477880in}{0.404742in}}%
\pgfpathlineto{\pgfqpoint{0.458368in}{0.406382in}}%
\pgfpathlineto{\pgfqpoint{0.452304in}{0.408937in}}%
\pgfpathlineto{\pgfqpoint{0.450213in}{0.413215in}}%
\pgfpathlineto{\pgfqpoint{0.449165in}{0.423080in}}%
\pgfpathlineto{\pgfqpoint{0.448735in}{0.465392in}}%
\pgfpathlineto{\pgfqpoint{0.448637in}{0.983146in}}%
\pgfpathlineto{\pgfqpoint{0.448652in}{2.889876in}}%
\pgfpathlineto{\pgfqpoint{0.448652in}{2.889876in}}%
\pgfusepath{stroke}%
\end{pgfscope}%
\begin{pgfscope}%
\pgfpathrectangle{\pgfqpoint{0.448634in}{0.402556in}}{\pgfqpoint{4.350661in}{2.489204in}} %
\pgfusepath{clip}%
\pgfsetbuttcap%
\pgfsetroundjoin%
\pgfsetlinewidth{1.003750pt}%
\definecolor{currentstroke}{rgb}{0.000000,0.000000,0.000000}%
\pgfsetstrokecolor{currentstroke}%
\pgfsetdash{{1.000000pt}{1.650000pt}}{0.000000pt}%
\pgfpathmoveto{\pgfqpoint{0.448634in}{2.896245in}}%
\pgfpathlineto{\pgfqpoint{0.448593in}{0.407043in}}%
\pgfpathlineto{\pgfqpoint{0.448593in}{0.407043in}}%
\pgfusepath{stroke}%
\end{pgfscope}%
\begin{pgfscope}%
\pgfpathrectangle{\pgfqpoint{0.448634in}{0.402556in}}{\pgfqpoint{4.350661in}{2.489204in}} %
\pgfusepath{clip}%
\pgfsetbuttcap%
\pgfsetroundjoin%
\pgfsetlinewidth{1.003750pt}%
\definecolor{currentstroke}{rgb}{0.000000,0.000000,0.000000}%
\pgfsetstrokecolor{currentstroke}%
\pgfsetdash{{1.000000pt}{1.650000pt}}{0.000000pt}%
\pgfpathmoveto{\pgfqpoint{0.576853in}{1.760817in}}%
\pgfpathlineto{\pgfqpoint{0.569394in}{1.840010in}}%
\pgfpathlineto{\pgfqpoint{0.563209in}{1.929338in}}%
\pgfpathlineto{\pgfqpoint{0.558592in}{2.028764in}}%
\pgfpathlineto{\pgfqpoint{0.555985in}{2.133265in}}%
\pgfpathlineto{\pgfqpoint{0.555566in}{2.237808in}}%
\pgfpathlineto{\pgfqpoint{0.557371in}{2.337352in}}%
\pgfpathlineto{\pgfqpoint{0.561096in}{2.424366in}}%
\pgfpathlineto{\pgfqpoint{0.566403in}{2.498791in}}%
\pgfpathlineto{\pgfqpoint{0.572909in}{2.560570in}}%
\pgfpathlineto{\pgfqpoint{0.580458in}{2.612119in}}%
\pgfpathlineto{\pgfqpoint{0.589086in}{2.655816in}}%
\pgfpathlineto{\pgfqpoint{0.598406in}{2.691589in}}%
\pgfpathlineto{\pgfqpoint{0.608613in}{2.721757in}}%
\pgfpathlineto{\pgfqpoint{0.619241in}{2.746278in}}%
\pgfpathlineto{\pgfqpoint{0.630817in}{2.767339in}}%
\pgfpathlineto{\pgfqpoint{0.642975in}{2.784884in}}%
\pgfpathlineto{\pgfqpoint{0.656813in}{2.800712in}}%
\pgfpathlineto{\pgfqpoint{0.672197in}{2.814549in}}%
\pgfpathlineto{\pgfqpoint{0.688853in}{2.826301in}}%
\pgfpathlineto{\pgfqpoint{0.706461in}{2.836076in}}%
\pgfpathlineto{\pgfqpoint{0.726804in}{2.844875in}}%
\pgfpathlineto{\pgfqpoint{0.751866in}{2.853203in}}%
\pgfpathlineto{\pgfqpoint{0.781631in}{2.860547in}}%
\pgfpathlineto{\pgfqpoint{0.818168in}{2.867054in}}%
\pgfpathlineto{\pgfqpoint{0.863581in}{2.872685in}}%
\pgfpathlineto{\pgfqpoint{0.922161in}{2.877518in}}%
\pgfpathlineto{\pgfqpoint{1.000391in}{2.881567in}}%
\pgfpathlineto{\pgfqpoint{1.111294in}{2.884881in}}%
\pgfpathlineto{\pgfqpoint{1.274428in}{2.887367in}}%
\pgfpathlineto{\pgfqpoint{1.552865in}{2.889263in}}%
\pgfpathlineto{\pgfqpoint{2.107573in}{2.890457in}}%
\pgfpathlineto{\pgfqpoint{3.343161in}{2.890573in}}%
\pgfpathlineto{\pgfqpoint{4.043615in}{2.888941in}}%
\pgfpathlineto{\pgfqpoint{4.289417in}{2.886404in}}%
\pgfpathlineto{\pgfqpoint{4.413375in}{2.883093in}}%
\pgfpathlineto{\pgfqpoint{4.489424in}{2.878997in}}%
\pgfpathlineto{\pgfqpoint{4.541451in}{2.874081in}}%
\pgfpathlineto{\pgfqpoint{4.578100in}{2.868470in}}%
\pgfpathlineto{\pgfqpoint{4.605818in}{2.862092in}}%
\pgfpathlineto{\pgfqpoint{4.626725in}{2.855245in}}%
\pgfpathlineto{\pgfqpoint{4.644925in}{2.847018in}}%
\pgfpathlineto{\pgfqpoint{4.660241in}{2.837590in}}%
\pgfpathlineto{\pgfqpoint{4.672623in}{2.827468in}}%
\pgfpathlineto{\pgfqpoint{4.683751in}{2.815592in}}%
\pgfpathlineto{\pgfqpoint{4.693406in}{2.802135in}}%
\pgfpathlineto{\pgfqpoint{4.702740in}{2.785343in}}%
\pgfpathlineto{\pgfqpoint{4.711277in}{2.765194in}}%
\pgfpathlineto{\pgfqpoint{4.719482in}{2.739484in}}%
\pgfpathlineto{\pgfqpoint{4.726293in}{2.710657in}}%
\pgfpathlineto{\pgfqpoint{4.733259in}{2.671643in}}%
\pgfpathlineto{\pgfqpoint{4.739604in}{2.622396in}}%
\pgfpathlineto{\pgfqpoint{4.745236in}{2.560504in}}%
\pgfpathlineto{\pgfqpoint{4.750164in}{2.481052in}}%
\pgfpathlineto{\pgfqpoint{4.754367in}{2.376618in}}%
\pgfpathlineto{\pgfqpoint{4.757443in}{2.242249in}}%
\pgfpathlineto{\pgfqpoint{4.758977in}{2.075483in}}%
\pgfpathlineto{\pgfqpoint{4.758447in}{1.888795in}}%
\pgfpathlineto{\pgfqpoint{4.755756in}{1.707111in}}%
\pgfpathlineto{\pgfqpoint{4.750925in}{1.532957in}}%
\pgfpathlineto{\pgfqpoint{4.744785in}{1.398726in}}%
\pgfpathlineto{\pgfqpoint{4.737575in}{1.289516in}}%
\pgfpathlineto{\pgfqpoint{4.728714in}{1.190470in}}%
\pgfpathlineto{\pgfqpoint{4.719652in}{1.116521in}}%
\pgfpathlineto{\pgfqpoint{4.710036in}{1.055276in}}%
\pgfpathlineto{\pgfqpoint{4.699503in}{1.001861in}}%
\pgfpathlineto{\pgfqpoint{4.689040in}{0.958690in}}%
\pgfpathlineto{\pgfqpoint{4.677219in}{0.918600in}}%
\pgfpathlineto{\pgfqpoint{4.664034in}{0.881749in}}%
\pgfpathlineto{\pgfqpoint{4.650584in}{0.850492in}}%
\pgfpathlineto{\pgfqpoint{4.636303in}{0.822570in}}%
\pgfpathlineto{\pgfqpoint{4.620207in}{0.795974in}}%
\pgfpathlineto{\pgfqpoint{4.603640in}{0.772901in}}%
\pgfpathlineto{\pgfqpoint{4.585488in}{0.751446in}}%
\pgfpathlineto{\pgfqpoint{4.565874in}{0.731749in}}%
\pgfpathlineto{\pgfqpoint{4.544964in}{0.713879in}}%
\pgfpathlineto{\pgfqpoint{4.522958in}{0.697824in}}%
\pgfpathlineto{\pgfqpoint{4.496157in}{0.681290in}}%
\pgfpathlineto{\pgfqpoint{4.470397in}{0.667953in}}%
\pgfpathlineto{\pgfqpoint{4.439961in}{0.654509in}}%
\pgfpathlineto{\pgfqpoint{4.406841in}{0.642281in}}%
\pgfpathlineto{\pgfqpoint{4.369009in}{0.630748in}}%
\pgfpathlineto{\pgfqpoint{4.326489in}{0.620226in}}%
\pgfpathlineto{\pgfqpoint{4.279327in}{0.610949in}}%
\pgfpathlineto{\pgfqpoint{4.227576in}{0.603085in}}%
\pgfpathlineto{\pgfqpoint{4.173450in}{0.597063in}}%
\pgfpathlineto{\pgfqpoint{4.110511in}{0.592203in}}%
\pgfpathlineto{\pgfqpoint{4.047471in}{0.589537in}}%
\pgfpathlineto{\pgfqpoint{3.977867in}{0.588624in}}%
\pgfpathlineto{\pgfqpoint{3.906093in}{0.589934in}}%
\pgfpathlineto{\pgfqpoint{3.834377in}{0.593496in}}%
\pgfpathlineto{\pgfqpoint{3.767120in}{0.599067in}}%
\pgfpathlineto{\pgfqpoint{3.704364in}{0.606392in}}%
\pgfpathlineto{\pgfqpoint{3.678516in}{0.610510in}}%
\pgfpathlineto{\pgfqpoint{3.620438in}{0.620500in}}%
\pgfpathlineto{\pgfqpoint{3.586319in}{0.628207in}}%
\pgfpathlineto{\pgfqpoint{3.495240in}{0.652428in}}%
\pgfpathlineto{\pgfqpoint{3.451528in}{0.667583in}}%
\pgfpathlineto{\pgfqpoint{3.408538in}{0.685220in}}%
\pgfpathlineto{\pgfqpoint{3.374594in}{0.702001in}}%
\pgfpathlineto{\pgfqpoint{3.345407in}{0.718682in}}%
\pgfpathlineto{\pgfqpoint{3.315236in}{0.738520in}}%
\pgfpathlineto{\pgfqpoint{3.288127in}{0.759290in}}%
\pgfpathlineto{\pgfqpoint{3.264004in}{0.780551in}}%
\pgfpathlineto{\pgfqpoint{3.241208in}{0.803648in}}%
\pgfpathlineto{\pgfqpoint{3.219894in}{0.828530in}}%
\pgfpathlineto{\pgfqpoint{3.200189in}{0.855091in}}%
\pgfpathlineto{\pgfqpoint{3.182177in}{0.883182in}}%
\pgfpathlineto{\pgfqpoint{3.165906in}{0.912633in}}%
\pgfpathlineto{\pgfqpoint{3.150351in}{0.945448in}}%
\pgfpathlineto{\pgfqpoint{3.136682in}{0.979345in}}%
\pgfpathlineto{\pgfqpoint{3.124073in}{1.016460in}}%
\pgfpathlineto{\pgfqpoint{3.112834in}{1.056769in}}%
\pgfpathlineto{\pgfqpoint{3.103046in}{1.100146in}}%
\pgfpathlineto{\pgfqpoint{3.095343in}{1.144071in}}%
\pgfpathlineto{\pgfqpoint{3.089208in}{1.190837in}}%
\pgfpathlineto{\pgfqpoint{3.084595in}{1.242838in}}%
\pgfpathlineto{\pgfqpoint{3.082137in}{1.295031in}}%
\pgfpathlineto{\pgfqpoint{3.081687in}{1.349787in}}%
\pgfpathlineto{\pgfqpoint{3.083451in}{1.406998in}}%
\pgfpathlineto{\pgfqpoint{3.087181in}{1.461589in}}%
\pgfpathlineto{\pgfqpoint{3.093485in}{1.520888in}}%
\pgfpathlineto{\pgfqpoint{3.101823in}{1.577334in}}%
\pgfpathlineto{\pgfqpoint{3.111930in}{1.630856in}}%
\pgfpathlineto{\pgfqpoint{3.124690in}{1.686208in}}%
\pgfpathlineto{\pgfqpoint{3.139178in}{1.738395in}}%
\pgfpathlineto{\pgfqpoint{3.155145in}{1.787366in}}%
\pgfpathlineto{\pgfqpoint{3.172353in}{1.833085in}}%
\pgfpathlineto{\pgfqpoint{3.191618in}{1.877716in}}%
\pgfpathlineto{\pgfqpoint{3.214026in}{1.923261in}}%
\pgfpathlineto{\pgfqpoint{3.236214in}{1.963157in}}%
\pgfpathlineto{\pgfqpoint{3.260178in}{2.001684in}}%
\pgfpathlineto{\pgfqpoint{3.285814in}{2.038776in}}%
\pgfpathlineto{\pgfqpoint{3.314415in}{2.076285in}}%
\pgfpathlineto{\pgfqpoint{3.348944in}{2.117711in}}%
\pgfpathlineto{\pgfqpoint{3.417133in}{2.198022in}}%
\pgfpathlineto{\pgfqpoint{3.426053in}{2.212128in}}%
\pgfpathlineto{\pgfqpoint{3.430798in}{2.223297in}}%
\pgfpathlineto{\pgfqpoint{3.432034in}{2.230603in}}%
\pgfpathlineto{\pgfqpoint{3.430773in}{2.237856in}}%
\pgfpathlineto{\pgfqpoint{3.426621in}{2.243526in}}%
\pgfpathlineto{\pgfqpoint{3.420908in}{2.247084in}}%
\pgfpathlineto{\pgfqpoint{3.412501in}{2.249583in}}%
\pgfpathlineto{\pgfqpoint{3.399499in}{2.250689in}}%
\pgfpathlineto{\pgfqpoint{3.384305in}{2.249671in}}%
\pgfpathlineto{\pgfqpoint{3.364985in}{2.246098in}}%
\pgfpathlineto{\pgfqpoint{3.341804in}{2.239342in}}%
\pgfpathlineto{\pgfqpoint{3.317109in}{2.229682in}}%
\pgfpathlineto{\pgfqpoint{3.291104in}{2.216986in}}%
\pgfpathlineto{\pgfqpoint{3.265928in}{2.202261in}}%
\pgfpathlineto{\pgfqpoint{3.239805in}{2.184361in}}%
\pgfpathlineto{\pgfqpoint{3.214775in}{2.164519in}}%
\pgfpathlineto{\pgfqpoint{3.190900in}{2.142893in}}%
\pgfpathlineto{\pgfqpoint{3.166657in}{2.117912in}}%
\pgfpathlineto{\pgfqpoint{3.143835in}{2.091233in}}%
\pgfpathlineto{\pgfqpoint{3.121079in}{2.061107in}}%
\pgfpathlineto{\pgfqpoint{3.099952in}{2.029463in}}%
\pgfpathlineto{\pgfqpoint{3.079251in}{1.994406in}}%
\pgfpathlineto{\pgfqpoint{3.059218in}{1.955915in}}%
\pgfpathlineto{\pgfqpoint{3.040058in}{1.914015in}}%
\pgfpathlineto{\pgfqpoint{3.022809in}{1.871041in}}%
\pgfpathlineto{\pgfqpoint{3.005790in}{1.822536in}}%
\pgfpathlineto{\pgfqpoint{2.990067in}{1.770819in}}%
\pgfpathlineto{\pgfqpoint{2.975708in}{1.715979in}}%
\pgfpathlineto{\pgfqpoint{2.962284in}{1.655680in}}%
\pgfpathlineto{\pgfqpoint{2.950496in}{1.592386in}}%
\pgfpathlineto{\pgfqpoint{2.940383in}{1.526185in}}%
\pgfpathlineto{\pgfqpoint{2.931745in}{1.454681in}}%
\pgfpathlineto{\pgfqpoint{2.925082in}{1.380399in}}%
\pgfpathlineto{\pgfqpoint{2.920647in}{1.305899in}}%
\pgfpathlineto{\pgfqpoint{2.918444in}{1.231270in}}%
\pgfpathlineto{\pgfqpoint{2.918545in}{1.159087in}}%
\pgfpathlineto{\pgfqpoint{2.920787in}{1.091931in}}%
\pgfpathlineto{\pgfqpoint{2.925177in}{1.027412in}}%
\pgfpathlineto{\pgfqpoint{2.931192in}{0.970580in}}%
\pgfpathlineto{\pgfqpoint{2.938760in}{0.919034in}}%
\pgfpathlineto{\pgfqpoint{2.947651in}{0.872852in}}%
\pgfpathlineto{\pgfqpoint{2.958213in}{0.829714in}}%
\pgfpathlineto{\pgfqpoint{2.969670in}{0.792114in}}%
\pgfpathlineto{\pgfqpoint{2.982463in}{0.757773in}}%
\pgfpathlineto{\pgfqpoint{2.996425in}{0.726812in}}%
\pgfpathlineto{\pgfqpoint{3.011299in}{0.699300in}}%
\pgfpathlineto{\pgfqpoint{3.026739in}{0.675225in}}%
\pgfpathlineto{\pgfqpoint{3.043828in}{0.652656in}}%
\pgfpathlineto{\pgfqpoint{3.062495in}{0.631788in}}%
\pgfpathlineto{\pgfqpoint{3.082602in}{0.612753in}}%
\pgfpathlineto{\pgfqpoint{3.103961in}{0.595592in}}%
\pgfpathlineto{\pgfqpoint{3.128268in}{0.579069in}}%
\pgfpathlineto{\pgfqpoint{3.153537in}{0.564554in}}%
\pgfpathlineto{\pgfqpoint{3.181571in}{0.550952in}}%
\pgfpathlineto{\pgfqpoint{3.214371in}{0.537647in}}%
\pgfpathlineto{\pgfqpoint{3.249846in}{0.525712in}}%
\pgfpathlineto{\pgfqpoint{3.290011in}{0.514571in}}%
\pgfpathlineto{\pgfqpoint{3.334820in}{0.504423in}}%
\pgfpathlineto{\pgfqpoint{3.386372in}{0.494999in}}%
\pgfpathlineto{\pgfqpoint{3.446798in}{0.486257in}}%
\pgfpathlineto{\pgfqpoint{3.518243in}{0.478282in}}%
\pgfpathlineto{\pgfqpoint{3.600685in}{0.471409in}}%
\pgfpathlineto{\pgfqpoint{3.696268in}{0.465713in}}%
\pgfpathlineto{\pgfqpoint{3.807144in}{0.461369in}}%
\pgfpathlineto{\pgfqpoint{3.933291in}{0.458719in}}%
\pgfpathlineto{\pgfqpoint{4.063808in}{0.458211in}}%
\pgfpathlineto{\pgfqpoint{4.187792in}{0.459914in}}%
\pgfpathlineto{\pgfqpoint{4.294335in}{0.463521in}}%
\pgfpathlineto{\pgfqpoint{4.381234in}{0.468574in}}%
\pgfpathlineto{\pgfqpoint{4.450636in}{0.474701in}}%
\pgfpathlineto{\pgfqpoint{4.506850in}{0.481799in}}%
\pgfpathlineto{\pgfqpoint{4.552009in}{0.489658in}}%
\pgfpathlineto{\pgfqpoint{4.588239in}{0.498115in}}%
\pgfpathlineto{\pgfqpoint{4.617656in}{0.507110in}}%
\pgfpathlineto{\pgfqpoint{4.642328in}{0.516843in}}%
\pgfpathlineto{\pgfqpoint{4.664194in}{0.527940in}}%
\pgfpathlineto{\pgfqpoint{4.681238in}{0.538945in}}%
\pgfpathlineto{\pgfqpoint{4.697164in}{0.551953in}}%
\pgfpathlineto{\pgfqpoint{4.710076in}{0.565289in}}%
\pgfpathlineto{\pgfqpoint{4.721578in}{0.580218in}}%
\pgfpathlineto{\pgfqpoint{4.731557in}{0.596521in}}%
\pgfpathlineto{\pgfqpoint{4.741000in}{0.616134in}}%
\pgfpathlineto{\pgfqpoint{4.749521in}{0.639027in}}%
\pgfpathlineto{\pgfqpoint{4.757522in}{0.667450in}}%
\pgfpathlineto{\pgfqpoint{4.764572in}{0.701345in}}%
\pgfpathlineto{\pgfqpoint{4.770840in}{0.743043in}}%
\pgfpathlineto{\pgfqpoint{4.776327in}{0.794934in}}%
\pgfpathlineto{\pgfqpoint{4.781278in}{0.864398in}}%
\pgfpathlineto{\pgfqpoint{4.785468in}{0.956371in}}%
\pgfpathlineto{\pgfqpoint{4.789000in}{1.085745in}}%
\pgfpathlineto{\pgfqpoint{4.791852in}{1.277385in}}%
\pgfpathlineto{\pgfqpoint{4.793959in}{1.581057in}}%
\pgfpathlineto{\pgfqpoint{4.794962in}{2.071429in}}%
\pgfpathlineto{\pgfqpoint{4.793967in}{2.559311in}}%
\pgfpathlineto{\pgfqpoint{4.791733in}{2.745981in}}%
\pgfpathlineto{\pgfqpoint{4.788955in}{2.818091in}}%
\pgfpathlineto{\pgfqpoint{4.785731in}{2.850227in}}%
\pgfpathlineto{\pgfqpoint{4.781879in}{2.867057in}}%
\pgfpathlineto{\pgfqpoint{4.777744in}{2.875780in}}%
\pgfpathlineto{\pgfqpoint{4.773097in}{2.880982in}}%
\pgfpathlineto{\pgfqpoint{4.767363in}{2.884504in}}%
\pgfpathlineto{\pgfqpoint{4.756853in}{2.887622in}}%
\pgfpathlineto{\pgfqpoint{4.739548in}{2.889639in}}%
\pgfpathlineto{\pgfqpoint{4.704762in}{2.890882in}}%
\pgfpathlineto{\pgfqpoint{4.602524in}{2.891538in}}%
\pgfpathlineto{\pgfqpoint{3.952100in}{2.891742in}}%
\pgfpathlineto{\pgfqpoint{0.617321in}{2.890753in}}%
\pgfpathlineto{\pgfqpoint{0.549910in}{2.888858in}}%
\pgfpathlineto{\pgfqpoint{0.521735in}{2.886179in}}%
\pgfpathlineto{\pgfqpoint{0.504666in}{2.882389in}}%
\pgfpathlineto{\pgfqpoint{0.494501in}{2.878011in}}%
\pgfpathlineto{\pgfqpoint{0.487180in}{2.872667in}}%
\pgfpathlineto{\pgfqpoint{0.481152in}{2.865519in}}%
\pgfpathlineto{\pgfqpoint{0.475664in}{2.854804in}}%
\pgfpathlineto{\pgfqpoint{0.471318in}{2.840737in}}%
\pgfpathlineto{\pgfqpoint{0.467301in}{2.818823in}}%
\pgfpathlineto{\pgfqpoint{0.463927in}{2.786700in}}%
\pgfpathlineto{\pgfqpoint{0.460918in}{2.734544in}}%
\pgfpathlineto{\pgfqpoint{0.458363in}{2.647473in}}%
\pgfpathlineto{\pgfqpoint{0.456575in}{2.523031in}}%
\pgfpathlineto{\pgfqpoint{0.456575in}{2.523031in}}%
\pgfusepath{stroke}%
\end{pgfscope}%
\begin{pgfscope}%
\pgfpathrectangle{\pgfqpoint{0.448634in}{0.402556in}}{\pgfqpoint{4.350661in}{2.489204in}} %
\pgfusepath{clip}%
\pgfsetbuttcap%
\pgfsetroundjoin%
\pgfsetlinewidth{1.003750pt}%
\definecolor{currentstroke}{rgb}{0.000000,0.000000,0.000000}%
\pgfsetstrokecolor{currentstroke}%
\pgfsetdash{{1.000000pt}{1.650000pt}}{0.000000pt}%
\pgfpathmoveto{\pgfqpoint{0.456424in}{1.370137in}}%
\pgfpathlineto{\pgfqpoint{0.459610in}{1.118755in}}%
\pgfpathlineto{\pgfqpoint{0.463695in}{0.962007in}}%
\pgfpathlineto{\pgfqpoint{0.468519in}{0.857610in}}%
\pgfpathlineto{\pgfqpoint{0.474082in}{0.783210in}}%
\pgfpathlineto{\pgfqpoint{0.480226in}{0.728906in}}%
\pgfpathlineto{\pgfqpoint{0.486970in}{0.687306in}}%
\pgfpathlineto{\pgfqpoint{0.494537in}{0.653558in}}%
\pgfpathlineto{\pgfqpoint{0.503107in}{0.625355in}}%
\pgfpathlineto{\pgfqpoint{0.512193in}{0.602749in}}%
\pgfpathlineto{\pgfqpoint{0.522200in}{0.583508in}}%
\pgfpathlineto{\pgfqpoint{0.534108in}{0.565743in}}%
\pgfpathlineto{\pgfqpoint{0.546263in}{0.551507in}}%
\pgfpathlineto{\pgfqpoint{0.559728in}{0.538907in}}%
\pgfpathlineto{\pgfqpoint{0.576129in}{0.526693in}}%
\pgfpathlineto{\pgfqpoint{0.595483in}{0.515351in}}%
\pgfpathlineto{\pgfqpoint{0.617681in}{0.505147in}}%
\pgfpathlineto{\pgfqpoint{0.642568in}{0.496153in}}%
\pgfpathlineto{\pgfqpoint{0.672126in}{0.487778in}}%
\pgfpathlineto{\pgfqpoint{0.708443in}{0.479824in}}%
\pgfpathlineto{\pgfqpoint{0.753649in}{0.472325in}}%
\pgfpathlineto{\pgfqpoint{0.807717in}{0.465660in}}%
\pgfpathlineto{\pgfqpoint{0.877116in}{0.459475in}}%
\pgfpathlineto{\pgfqpoint{0.961828in}{0.454230in}}%
\pgfpathlineto{\pgfqpoint{1.068351in}{0.449916in}}%
\pgfpathlineto{\pgfqpoint{1.201018in}{0.446839in}}%
\pgfpathlineto{\pgfqpoint{1.357637in}{0.445481in}}%
\pgfpathlineto{\pgfqpoint{1.525135in}{0.446232in}}%
\pgfpathlineto{\pgfqpoint{1.686088in}{0.449142in}}%
\pgfpathlineto{\pgfqpoint{1.823074in}{0.453747in}}%
\pgfpathlineto{\pgfqpoint{1.938245in}{0.459764in}}%
\pgfpathlineto{\pgfqpoint{2.031582in}{0.466759in}}%
\pgfpathlineto{\pgfqpoint{2.109580in}{0.474745in}}%
\pgfpathlineto{\pgfqpoint{2.174384in}{0.483535in}}%
\pgfpathlineto{\pgfqpoint{2.228139in}{0.492940in}}%
\pgfpathlineto{\pgfqpoint{2.275119in}{0.503356in}}%
\pgfpathlineto{\pgfqpoint{2.315282in}{0.514501in}}%
\pgfpathlineto{\pgfqpoint{2.350698in}{0.526659in}}%
\pgfpathlineto{\pgfqpoint{2.381320in}{0.539536in}}%
\pgfpathlineto{\pgfqpoint{2.407164in}{0.552659in}}%
\pgfpathlineto{\pgfqpoint{2.430226in}{0.566639in}}%
\pgfpathlineto{\pgfqpoint{2.452282in}{0.582602in}}%
\pgfpathlineto{\pgfqpoint{2.471391in}{0.599069in}}%
\pgfpathlineto{\pgfqpoint{2.489240in}{0.617293in}}%
\pgfpathlineto{\pgfqpoint{2.505678in}{0.637180in}}%
\pgfpathlineto{\pgfqpoint{2.520620in}{0.658557in}}%
\pgfpathlineto{\pgfqpoint{2.535213in}{0.683314in}}%
\pgfpathlineto{\pgfqpoint{2.549115in}{0.711484in}}%
\pgfpathlineto{\pgfqpoint{2.562091in}{0.743004in}}%
\pgfpathlineto{\pgfqpoint{2.574020in}{0.777751in}}%
\pgfpathlineto{\pgfqpoint{2.585502in}{0.817970in}}%
\pgfpathlineto{\pgfqpoint{2.596809in}{0.866038in}}%
\pgfpathlineto{\pgfqpoint{2.607562in}{0.921948in}}%
\pgfpathlineto{\pgfqpoint{2.617925in}{0.988098in}}%
\pgfpathlineto{\pgfqpoint{2.627958in}{1.066918in}}%
\pgfpathlineto{\pgfqpoint{2.637941in}{1.163320in}}%
\pgfpathlineto{\pgfqpoint{2.648424in}{1.287199in}}%
\pgfpathlineto{\pgfqpoint{2.660103in}{1.453438in}}%
\pgfpathlineto{\pgfqpoint{2.674773in}{1.696801in}}%
\pgfpathlineto{\pgfqpoint{2.687716in}{1.945279in}}%
\pgfpathlineto{\pgfqpoint{2.692670in}{2.079573in}}%
\pgfpathlineto{\pgfqpoint{2.693829in}{2.166682in}}%
\pgfpathlineto{\pgfqpoint{2.692565in}{2.233870in}}%
\pgfpathlineto{\pgfqpoint{2.689436in}{2.286015in}}%
\pgfpathlineto{\pgfqpoint{2.684859in}{2.327999in}}%
\pgfpathlineto{\pgfqpoint{2.678725in}{2.364664in}}%
\pgfpathlineto{\pgfqpoint{2.671356in}{2.395897in}}%
\pgfpathlineto{\pgfqpoint{2.662489in}{2.423981in}}%
\pgfpathlineto{\pgfqpoint{2.652361in}{2.448778in}}%
\pgfpathlineto{\pgfqpoint{2.641365in}{2.470245in}}%
\pgfpathlineto{\pgfqpoint{2.628643in}{2.490425in}}%
\pgfpathlineto{\pgfqpoint{2.614279in}{2.509106in}}%
\pgfpathlineto{\pgfqpoint{2.598443in}{2.526159in}}%
\pgfpathlineto{\pgfqpoint{2.579590in}{2.543005in}}%
\pgfpathlineto{\pgfqpoint{2.559532in}{2.557923in}}%
\pgfpathlineto{\pgfqpoint{2.536602in}{2.572183in}}%
\pgfpathlineto{\pgfqpoint{2.510850in}{2.585538in}}%
\pgfpathlineto{\pgfqpoint{2.482360in}{2.597837in}}%
\pgfpathlineto{\pgfqpoint{2.449134in}{2.609683in}}%
\pgfpathlineto{\pgfqpoint{2.411184in}{2.620696in}}%
\pgfpathlineto{\pgfqpoint{2.368552in}{2.630606in}}%
\pgfpathlineto{\pgfqpoint{2.321294in}{2.639221in}}%
\pgfpathlineto{\pgfqpoint{2.269467in}{2.646399in}}%
\pgfpathlineto{\pgfqpoint{2.210954in}{2.652193in}}%
\pgfpathlineto{\pgfqpoint{2.147967in}{2.656153in}}%
\pgfpathlineto{\pgfqpoint{2.080556in}{2.658135in}}%
\pgfpathlineto{\pgfqpoint{2.010948in}{2.657971in}}%
\pgfpathlineto{\pgfqpoint{1.939195in}{2.655572in}}%
\pgfpathlineto{\pgfqpoint{1.867527in}{2.650913in}}%
\pgfpathlineto{\pgfqpoint{1.798171in}{2.644140in}}%
\pgfpathlineto{\pgfqpoint{1.733341in}{2.635606in}}%
\pgfpathlineto{\pgfqpoint{1.673075in}{2.625521in}}%
\pgfpathlineto{\pgfqpoint{1.615274in}{2.613610in}}%
\pgfpathlineto{\pgfqpoint{1.562133in}{2.600402in}}%
\pgfpathlineto{\pgfqpoint{1.513681in}{2.586139in}}%
\pgfpathlineto{\pgfqpoint{1.467862in}{2.570344in}}%
\pgfpathlineto{\pgfqpoint{1.426794in}{2.553923in}}%
\pgfpathlineto{\pgfqpoint{1.388447in}{2.536289in}}%
\pgfpathlineto{\pgfqpoint{1.352878in}{2.517566in}}%
\pgfpathlineto{\pgfqpoint{1.320128in}{2.497922in}}%
\pgfpathlineto{\pgfqpoint{1.288379in}{2.476236in}}%
\pgfpathlineto{\pgfqpoint{1.259592in}{2.453861in}}%
\pgfpathlineto{\pgfqpoint{1.232050in}{2.429520in}}%
\pgfpathlineto{\pgfqpoint{1.207527in}{2.404898in}}%
\pgfpathlineto{\pgfqpoint{1.184409in}{2.378557in}}%
\pgfpathlineto{\pgfqpoint{1.162828in}{2.350561in}}%
\pgfpathlineto{\pgfqpoint{1.142891in}{2.321011in}}%
\pgfpathlineto{\pgfqpoint{1.124675in}{2.290041in}}%
\pgfpathlineto{\pgfqpoint{1.108225in}{2.257802in}}%
\pgfpathlineto{\pgfqpoint{1.092639in}{2.222199in}}%
\pgfpathlineto{\pgfqpoint{1.079059in}{2.185535in}}%
\pgfpathlineto{\pgfqpoint{1.067443in}{2.147998in}}%
\pgfpathlineto{\pgfqpoint{1.057187in}{2.107348in}}%
\pgfpathlineto{\pgfqpoint{1.049004in}{2.066086in}}%
\pgfpathlineto{\pgfqpoint{1.042513in}{2.021906in}}%
\pgfpathlineto{\pgfqpoint{1.038177in}{1.977382in}}%
\pgfpathlineto{\pgfqpoint{1.035866in}{1.930167in}}%
\pgfpathlineto{\pgfqpoint{1.035826in}{1.882878in}}%
\pgfpathlineto{\pgfqpoint{1.038031in}{1.835656in}}%
\pgfpathlineto{\pgfqpoint{1.042474in}{1.788641in}}%
\pgfpathlineto{\pgfqpoint{1.049176in}{1.741979in}}%
\pgfpathlineto{\pgfqpoint{1.057644in}{1.698239in}}%
\pgfpathlineto{\pgfqpoint{1.068221in}{1.655105in}}%
\pgfpathlineto{\pgfqpoint{1.080962in}{1.612745in}}%
\pgfpathlineto{\pgfqpoint{1.095031in}{1.573617in}}%
\pgfpathlineto{\pgfqpoint{1.111115in}{1.535520in}}%
\pgfpathlineto{\pgfqpoint{1.128118in}{1.500775in}}%
\pgfpathlineto{\pgfqpoint{1.146930in}{1.467274in}}%
\pgfpathlineto{\pgfqpoint{1.167531in}{1.435181in}}%
\pgfpathlineto{\pgfqpoint{1.189874in}{1.404652in}}%
\pgfpathlineto{\pgfqpoint{1.213884in}{1.375828in}}%
\pgfpathlineto{\pgfqpoint{1.237817in}{1.350457in}}%
\pgfpathlineto{\pgfqpoint{1.264748in}{1.325237in}}%
\pgfpathlineto{\pgfqpoint{1.292991in}{1.301972in}}%
\pgfpathlineto{\pgfqpoint{1.322398in}{1.280678in}}%
\pgfpathlineto{\pgfqpoint{1.352820in}{1.261340in}}%
\pgfpathlineto{\pgfqpoint{1.386095in}{1.242889in}}%
\pgfpathlineto{\pgfqpoint{1.420190in}{1.226516in}}%
\pgfpathlineto{\pgfqpoint{1.457024in}{1.211329in}}%
\pgfpathlineto{\pgfqpoint{1.496554in}{1.197536in}}%
\pgfpathlineto{\pgfqpoint{1.538719in}{1.185287in}}%
\pgfpathlineto{\pgfqpoint{1.583441in}{1.174641in}}%
\pgfpathlineto{\pgfqpoint{1.634929in}{1.164775in}}%
\pgfpathlineto{\pgfqpoint{1.706063in}{1.153745in}}%
\pgfpathlineto{\pgfqpoint{1.768492in}{1.143417in}}%
\pgfpathlineto{\pgfqpoint{1.796122in}{1.136567in}}%
\pgfpathlineto{\pgfqpoint{1.812683in}{1.130481in}}%
\pgfpathlineto{\pgfqpoint{1.824471in}{1.124102in}}%
\pgfpathlineto{\pgfqpoint{1.833209in}{1.116741in}}%
\pgfpathlineto{\pgfqpoint{1.838498in}{1.108890in}}%
\pgfpathlineto{\pgfqpoint{1.840588in}{1.101849in}}%
\pgfpathlineto{\pgfqpoint{1.840619in}{1.094412in}}%
\pgfpathlineto{\pgfqpoint{1.837931in}{1.084986in}}%
\pgfpathlineto{\pgfqpoint{1.833246in}{1.076615in}}%
\pgfpathlineto{\pgfqpoint{1.825819in}{1.067542in}}%
\pgfpathlineto{\pgfqpoint{1.813813in}{1.056850in}}%
\pgfpathlineto{\pgfqpoint{1.798819in}{1.046763in}}%
\pgfpathlineto{\pgfqpoint{1.781016in}{1.037462in}}%
\pgfpathlineto{\pgfqpoint{1.758447in}{1.028391in}}%
\pgfpathlineto{\pgfqpoint{1.733203in}{1.020815in}}%
\pgfpathlineto{\pgfqpoint{1.705410in}{1.014872in}}%
\pgfpathlineto{\pgfqpoint{1.675178in}{1.010714in}}%
\pgfpathlineto{\pgfqpoint{1.642610in}{1.008507in}}%
\pgfpathlineto{\pgfqpoint{1.607809in}{1.008432in}}%
\pgfpathlineto{\pgfqpoint{1.570886in}{1.010691in}}%
\pgfpathlineto{\pgfqpoint{1.534118in}{1.015181in}}%
\pgfpathlineto{\pgfqpoint{1.495454in}{1.022233in}}%
\pgfpathlineto{\pgfqpoint{1.457161in}{1.031563in}}%
\pgfpathlineto{\pgfqpoint{1.419337in}{1.043132in}}%
\pgfpathlineto{\pgfqpoint{1.382089in}{1.056929in}}%
\pgfpathlineto{\pgfqpoint{1.347544in}{1.072019in}}%
\pgfpathlineto{\pgfqpoint{1.313727in}{1.089133in}}%
\pgfpathlineto{\pgfqpoint{1.280762in}{1.108299in}}%
\pgfpathlineto{\pgfqpoint{1.248782in}{1.129536in}}%
\pgfpathlineto{\pgfqpoint{1.219708in}{1.151422in}}%
\pgfpathlineto{\pgfqpoint{1.191752in}{1.175138in}}%
\pgfpathlineto{\pgfqpoint{1.165031in}{1.200649in}}%
\pgfpathlineto{\pgfqpoint{1.139653in}{1.227898in}}%
\pgfpathlineto{\pgfqpoint{1.115714in}{1.256800in}}%
\pgfpathlineto{\pgfqpoint{1.093288in}{1.287251in}}%
\pgfpathlineto{\pgfqpoint{1.071178in}{1.321163in}}%
\pgfpathlineto{\pgfqpoint{1.050868in}{1.356520in}}%
\pgfpathlineto{\pgfqpoint{1.032365in}{1.393152in}}%
\pgfpathlineto{\pgfqpoint{1.014718in}{1.433142in}}%
\pgfpathlineto{\pgfqpoint{0.999024in}{1.474185in}}%
\pgfpathlineto{\pgfqpoint{0.984506in}{1.518461in}}%
\pgfpathlineto{\pgfqpoint{0.972010in}{1.563537in}}%
\pgfpathlineto{\pgfqpoint{0.960944in}{1.611678in}}%
\pgfpathlineto{\pgfqpoint{0.951530in}{1.662824in}}%
\pgfpathlineto{\pgfqpoint{0.944286in}{1.714431in}}%
\pgfpathlineto{\pgfqpoint{0.938950in}{1.768847in}}%
\pgfpathlineto{\pgfqpoint{0.935870in}{1.823491in}}%
\pgfpathlineto{\pgfqpoint{0.935034in}{1.878240in}}%
\pgfpathlineto{\pgfqpoint{0.936466in}{1.932973in}}%
\pgfpathlineto{\pgfqpoint{0.940005in}{1.985084in}}%
\pgfpathlineto{\pgfqpoint{0.945759in}{2.036935in}}%
\pgfpathlineto{\pgfqpoint{0.953410in}{2.085938in}}%
\pgfpathlineto{\pgfqpoint{0.962764in}{2.132000in}}%
\pgfpathlineto{\pgfqpoint{0.974287in}{2.177414in}}%
\pgfpathlineto{\pgfqpoint{0.987332in}{2.219653in}}%
\pgfpathlineto{\pgfqpoint{1.001667in}{2.258654in}}%
\pgfpathlineto{\pgfqpoint{1.018051in}{2.296583in}}%
\pgfpathlineto{\pgfqpoint{1.035401in}{2.331101in}}%
\pgfpathlineto{\pgfqpoint{1.054650in}{2.364275in}}%
\pgfpathlineto{\pgfqpoint{1.074406in}{2.393984in}}%
\pgfpathlineto{\pgfqpoint{1.095771in}{2.422197in}}%
\pgfpathlineto{\pgfqpoint{1.118662in}{2.448797in}}%
\pgfpathlineto{\pgfqpoint{1.142967in}{2.473701in}}%
\pgfpathlineto{\pgfqpoint{1.168550in}{2.496867in}}%
\pgfpathlineto{\pgfqpoint{1.197085in}{2.519662in}}%
\pgfpathlineto{\pgfqpoint{1.226727in}{2.540526in}}%
\pgfpathlineto{\pgfqpoint{1.259242in}{2.560673in}}%
\pgfpathlineto{\pgfqpoint{1.294612in}{2.579881in}}%
\pgfpathlineto{\pgfqpoint{1.332792in}{2.597982in}}%
\pgfpathlineto{\pgfqpoint{1.373719in}{2.614859in}}%
\pgfpathlineto{\pgfqpoint{1.417319in}{2.630445in}}%
\pgfpathlineto{\pgfqpoint{1.465632in}{2.645312in}}%
\pgfpathlineto{\pgfqpoint{1.518640in}{2.659204in}}%
\pgfpathlineto{\pgfqpoint{1.576309in}{2.671929in}}%
\pgfpathlineto{\pgfqpoint{1.638597in}{2.683344in}}%
\pgfpathlineto{\pgfqpoint{1.705462in}{2.693343in}}%
\pgfpathlineto{\pgfqpoint{1.779027in}{2.702064in}}%
\pgfpathlineto{\pgfqpoint{1.857097in}{2.709077in}}%
\pgfpathlineto{\pgfqpoint{1.939633in}{2.714280in}}%
\pgfpathlineto{\pgfqpoint{2.026598in}{2.717513in}}%
\pgfpathlineto{\pgfqpoint{2.113605in}{2.718523in}}%
\pgfpathlineto{\pgfqpoint{2.198435in}{2.717303in}}%
\pgfpathlineto{\pgfqpoint{2.278866in}{2.713929in}}%
\pgfpathlineto{\pgfqpoint{2.352678in}{2.708598in}}%
\pgfpathlineto{\pgfqpoint{2.417657in}{2.701709in}}%
\pgfpathlineto{\pgfqpoint{2.473770in}{2.693630in}}%
\pgfpathlineto{\pgfqpoint{2.523140in}{2.684368in}}%
\pgfpathlineto{\pgfqpoint{2.565726in}{2.674202in}}%
\pgfpathlineto{\pgfqpoint{2.601510in}{2.663544in}}%
\pgfpathlineto{\pgfqpoint{2.632577in}{2.652142in}}%
\pgfpathlineto{\pgfqpoint{2.658899in}{2.640331in}}%
\pgfpathlineto{\pgfqpoint{2.682438in}{2.627436in}}%
\pgfpathlineto{\pgfqpoint{2.703062in}{2.613571in}}%
\pgfpathlineto{\pgfqpoint{2.720674in}{2.598978in}}%
\pgfpathlineto{\pgfqpoint{2.735263in}{2.584053in}}%
\pgfpathlineto{\pgfqpoint{2.748320in}{2.567377in}}%
\pgfpathlineto{\pgfqpoint{2.759553in}{2.549046in}}%
\pgfpathlineto{\pgfqpoint{2.768788in}{2.529306in}}%
\pgfpathlineto{\pgfqpoint{2.776017in}{2.508498in}}%
\pgfpathlineto{\pgfqpoint{2.781884in}{2.484540in}}%
\pgfpathlineto{\pgfqpoint{2.786102in}{2.457597in}}%
\pgfpathlineto{\pgfqpoint{2.788720in}{2.425384in}}%
\pgfpathlineto{\pgfqpoint{2.789427in}{2.388061in}}%
\pgfpathlineto{\pgfqpoint{2.787962in}{2.340801in}}%
\pgfpathlineto{\pgfqpoint{2.783672in}{2.278768in}}%
\pgfpathlineto{\pgfqpoint{2.774289in}{2.179783in}}%
\pgfpathlineto{\pgfqpoint{2.743611in}{1.868119in}}%
\pgfpathlineto{\pgfqpoint{2.730112in}{1.702060in}}%
\pgfpathlineto{\pgfqpoint{2.717287in}{1.515949in}}%
\pgfpathlineto{\pgfqpoint{2.702602in}{1.267597in}}%
\pgfpathlineto{\pgfqpoint{2.684434in}{0.964630in}}%
\pgfpathlineto{\pgfqpoint{2.675374in}{0.850600in}}%
\pgfpathlineto{\pgfqpoint{2.667030in}{0.771523in}}%
\pgfpathlineto{\pgfqpoint{2.658752in}{0.712543in}}%
\pgfpathlineto{\pgfqpoint{2.650176in}{0.666284in}}%
\pgfpathlineto{\pgfqpoint{2.640820in}{0.627931in}}%
\pgfpathlineto{\pgfqpoint{2.631145in}{0.597534in}}%
\pgfpathlineto{\pgfqpoint{2.621004in}{0.572745in}}%
\pgfpathlineto{\pgfqpoint{2.609856in}{0.551383in}}%
\pgfpathlineto{\pgfqpoint{2.598042in}{0.533534in}}%
\pgfpathlineto{\pgfqpoint{2.584496in}{0.517378in}}%
\pgfpathlineto{\pgfqpoint{2.571109in}{0.504669in}}%
\pgfpathlineto{\pgfqpoint{2.554789in}{0.492313in}}%
\pgfpathlineto{\pgfqpoint{2.537457in}{0.481914in}}%
\pgfpathlineto{\pgfqpoint{2.517374in}{0.472367in}}%
\pgfpathlineto{\pgfqpoint{2.492542in}{0.463178in}}%
\pgfpathlineto{\pgfqpoint{2.462979in}{0.454833in}}%
\pgfpathlineto{\pgfqpoint{2.428766in}{0.447542in}}%
\pgfpathlineto{\pgfqpoint{2.385671in}{0.440735in}}%
\pgfpathlineto{\pgfqpoint{2.331557in}{0.434581in}}%
\pgfpathlineto{\pgfqpoint{2.262115in}{0.429077in}}%
\pgfpathlineto{\pgfqpoint{2.170851in}{0.424236in}}%
\pgfpathlineto{\pgfqpoint{2.049086in}{0.420134in}}%
\pgfpathlineto{\pgfqpoint{1.879436in}{0.416783in}}%
\pgfpathlineto{\pgfqpoint{1.640159in}{0.414418in}}%
\pgfpathlineto{\pgfqpoint{1.322562in}{0.413569in}}%
\pgfpathlineto{\pgfqpoint{1.020194in}{0.414850in}}%
\pgfpathlineto{\pgfqpoint{0.822256in}{0.417715in}}%
\pgfpathlineto{\pgfqpoint{0.704835in}{0.421430in}}%
\pgfpathlineto{\pgfqpoint{0.630976in}{0.425829in}}%
\pgfpathlineto{\pgfqpoint{0.583316in}{0.430734in}}%
\pgfpathlineto{\pgfqpoint{0.551033in}{0.436123in}}%
\pgfpathlineto{\pgfqpoint{0.527708in}{0.442189in}}%
\pgfpathlineto{\pgfqpoint{0.511250in}{0.448625in}}%
\pgfpathlineto{\pgfqpoint{0.499549in}{0.455216in}}%
\pgfpathlineto{\pgfqpoint{0.488916in}{0.463841in}}%
\pgfpathlineto{\pgfqpoint{0.481322in}{0.472730in}}%
\pgfpathlineto{\pgfqpoint{0.474078in}{0.485127in}}%
\pgfpathlineto{\pgfqpoint{0.468753in}{0.498748in}}%
\pgfpathlineto{\pgfqpoint{0.463870in}{0.517848in}}%
\pgfpathlineto{\pgfqpoint{0.459679in}{0.544796in}}%
\pgfpathlineto{\pgfqpoint{0.456386in}{0.581938in}}%
\pgfpathlineto{\pgfqpoint{0.453731in}{0.639106in}}%
\pgfpathlineto{\pgfqpoint{0.451681in}{0.736155in}}%
\pgfpathlineto{\pgfqpoint{0.450220in}{0.927815in}}%
\pgfpathlineto{\pgfqpoint{0.449345in}{1.403252in}}%
\pgfpathlineto{\pgfqpoint{0.449543in}{2.682703in}}%
\pgfpathlineto{\pgfqpoint{0.451011in}{2.856932in}}%
\pgfpathlineto{\pgfqpoint{0.452802in}{2.879219in}}%
\pgfpathlineto{\pgfqpoint{0.455188in}{2.886108in}}%
\pgfpathlineto{\pgfqpoint{0.458626in}{2.889028in}}%
\pgfpathlineto{\pgfqpoint{0.464996in}{2.890553in}}%
\pgfpathlineto{\pgfqpoint{0.482377in}{2.891423in}}%
\pgfpathlineto{\pgfqpoint{0.565038in}{2.891729in}}%
\pgfpathlineto{\pgfqpoint{2.733842in}{2.891760in}}%
\pgfpathlineto{\pgfqpoint{4.789510in}{2.890885in}}%
\pgfpathlineto{\pgfqpoint{4.793727in}{2.889730in}}%
\pgfpathlineto{\pgfqpoint{4.795481in}{2.888307in}}%
\pgfpathlineto{\pgfqpoint{4.797106in}{2.881145in}}%
\pgfpathlineto{\pgfqpoint{4.797997in}{2.858771in}}%
\pgfpathlineto{\pgfqpoint{4.798039in}{2.856283in}}%
\pgfpathlineto{\pgfqpoint{4.798039in}{2.856283in}}%
\pgfusepath{stroke}%
\end{pgfscope}%
\begin{pgfscope}%
\pgfpathrectangle{\pgfqpoint{0.448634in}{0.402556in}}{\pgfqpoint{4.350661in}{2.489204in}} %
\pgfusepath{clip}%
\pgfsetbuttcap%
\pgfsetroundjoin%
\pgfsetlinewidth{1.003750pt}%
\definecolor{currentstroke}{rgb}{0.000000,0.000000,0.000000}%
\pgfsetstrokecolor{currentstroke}%
\pgfsetdash{{1.000000pt}{1.650000pt}}{0.000000pt}%
\pgfpathmoveto{\pgfqpoint{3.428772in}{0.402610in}}%
\pgfpathlineto{\pgfqpoint{2.806632in}{0.403760in}}%
\pgfpathlineto{\pgfqpoint{2.769692in}{0.405578in}}%
\pgfpathlineto{\pgfqpoint{2.754632in}{0.408064in}}%
\pgfpathlineto{\pgfqpoint{2.746391in}{0.411198in}}%
\pgfpathlineto{\pgfqpoint{2.740943in}{0.415265in}}%
\pgfpathlineto{\pgfqpoint{2.736784in}{0.420984in}}%
\pgfpathlineto{\pgfqpoint{2.733281in}{0.430071in}}%
\pgfpathlineto{\pgfqpoint{2.730449in}{0.444636in}}%
\pgfpathlineto{\pgfqpoint{2.728238in}{0.469392in}}%
\pgfpathlineto{\pgfqpoint{2.726470in}{0.519131in}}%
\pgfpathlineto{\pgfqpoint{2.725711in}{0.613715in}}%
\pgfpathlineto{\pgfqpoint{2.726842in}{0.768038in}}%
\pgfpathlineto{\pgfqpoint{2.730556in}{0.962148in}}%
\pgfpathlineto{\pgfqpoint{2.736611in}{1.158670in}}%
\pgfpathlineto{\pgfqpoint{2.744092in}{1.327718in}}%
\pgfpathlineto{\pgfqpoint{2.753201in}{1.484189in}}%
\pgfpathlineto{\pgfqpoint{2.763257in}{1.620609in}}%
\pgfpathlineto{\pgfqpoint{2.776118in}{1.764216in}}%
\pgfpathlineto{\pgfqpoint{2.788914in}{1.877776in}}%
\pgfpathlineto{\pgfqpoint{2.805748in}{2.005740in}}%
\pgfpathlineto{\pgfqpoint{2.821176in}{2.101198in}}%
\pgfpathlineto{\pgfqpoint{2.838359in}{2.193718in}}%
\pgfpathlineto{\pgfqpoint{2.859135in}{2.292966in}}%
\pgfpathlineto{\pgfqpoint{2.887209in}{2.425960in}}%
\pgfpathlineto{\pgfqpoint{2.896991in}{2.479559in}}%
\pgfpathlineto{\pgfqpoint{2.901543in}{2.516523in}}%
\pgfpathlineto{\pgfqpoint{2.902849in}{2.543854in}}%
\pgfpathlineto{\pgfqpoint{2.901957in}{2.566223in}}%
\pgfpathlineto{\pgfqpoint{2.899151in}{2.585863in}}%
\pgfpathlineto{\pgfqpoint{2.894794in}{2.602546in}}%
\pgfpathlineto{\pgfqpoint{2.888484in}{2.618388in}}%
\pgfpathlineto{\pgfqpoint{2.880257in}{2.633033in}}%
\pgfpathlineto{\pgfqpoint{2.870348in}{2.646246in}}%
\pgfpathlineto{\pgfqpoint{2.857400in}{2.659530in}}%
\pgfpathlineto{\pgfqpoint{2.843189in}{2.671010in}}%
\pgfpathlineto{\pgfqpoint{2.824237in}{2.683209in}}%
\pgfpathlineto{\pgfqpoint{2.802413in}{2.694418in}}%
\pgfpathlineto{\pgfqpoint{2.775809in}{2.705369in}}%
\pgfpathlineto{\pgfqpoint{2.744461in}{2.715715in}}%
\pgfpathlineto{\pgfqpoint{2.708436in}{2.725252in}}%
\pgfpathlineto{\pgfqpoint{2.665655in}{2.734289in}}%
\pgfpathlineto{\pgfqpoint{2.613991in}{2.742869in}}%
\pgfpathlineto{\pgfqpoint{2.553459in}{2.750589in}}%
\pgfpathlineto{\pgfqpoint{2.481920in}{2.757365in}}%
\pgfpathlineto{\pgfqpoint{2.399398in}{2.762839in}}%
\pgfpathlineto{\pgfqpoint{2.310269in}{2.766482in}}%
\pgfpathlineto{\pgfqpoint{2.175416in}{2.768725in}}%
\pgfpathlineto{\pgfqpoint{2.066653in}{2.767942in}}%
\pgfpathlineto{\pgfqpoint{1.953570in}{2.764859in}}%
\pgfpathlineto{\pgfqpoint{1.851429in}{2.759759in}}%
\pgfpathlineto{\pgfqpoint{1.745051in}{2.752169in}}%
\pgfpathlineto{\pgfqpoint{1.658373in}{2.743453in}}%
\pgfpathlineto{\pgfqpoint{1.580552in}{2.733461in}}%
\pgfpathlineto{\pgfqpoint{1.490057in}{2.719338in}}%
\pgfpathlineto{\pgfqpoint{1.417231in}{2.704698in}}%
\pgfpathlineto{\pgfqpoint{1.361992in}{2.690818in}}%
\pgfpathlineto{\pgfqpoint{1.311460in}{2.675819in}}%
\pgfpathlineto{\pgfqpoint{1.265667in}{2.659924in}}%
\pgfpathlineto{\pgfqpoint{1.222575in}{2.642586in}}%
\pgfpathlineto{\pgfqpoint{1.184324in}{2.624682in}}%
\pgfpathlineto{\pgfqpoint{1.148892in}{2.605623in}}%
\pgfpathlineto{\pgfqpoint{1.116331in}{2.585573in}}%
\pgfpathlineto{\pgfqpoint{1.092327in}{2.568512in}}%
\pgfpathlineto{\pgfqpoint{1.079760in}{2.558686in}}%
\pgfpathlineto{\pgfqpoint{1.051544in}{2.535379in}}%
\pgfpathlineto{\pgfqpoint{1.026312in}{2.511712in}}%
\pgfpathlineto{\pgfqpoint{1.002399in}{2.486318in}}%
\pgfpathlineto{\pgfqpoint{0.979913in}{2.459269in}}%
\pgfpathlineto{\pgfqpoint{0.958934in}{2.430678in}}%
\pgfpathlineto{\pgfqpoint{0.938264in}{2.398643in}}%
\pgfpathlineto{\pgfqpoint{0.923047in}{2.371385in}}%
\pgfpathlineto{\pgfqpoint{0.904513in}{2.334774in}}%
\pgfpathlineto{\pgfqpoint{0.887854in}{2.297001in}}%
\pgfpathlineto{\pgfqpoint{0.872131in}{2.255971in}}%
\pgfpathlineto{\pgfqpoint{0.857508in}{2.211741in}}%
\pgfpathlineto{\pgfqpoint{0.844762in}{2.166757in}}%
\pgfpathlineto{\pgfqpoint{0.838624in}{2.140306in}}%
\pgfpathlineto{\pgfqpoint{0.826982in}{2.087194in}}%
\pgfpathlineto{\pgfqpoint{0.816322in}{2.028715in}}%
\pgfpathlineto{\pgfqpoint{0.810087in}{1.984495in}}%
\pgfpathlineto{\pgfqpoint{0.808026in}{1.967238in}}%
\pgfpathlineto{\pgfqpoint{0.800076in}{1.898140in}}%
\pgfpathlineto{\pgfqpoint{0.793713in}{1.823823in}}%
\pgfpathlineto{\pgfqpoint{0.788799in}{1.741875in}}%
\pgfpathlineto{\pgfqpoint{0.786199in}{1.677225in}}%
\pgfpathlineto{\pgfqpoint{0.776951in}{1.453481in}}%
\pgfpathlineto{\pgfqpoint{0.773280in}{1.418894in}}%
\pgfpathlineto{\pgfqpoint{0.768298in}{1.389582in}}%
\pgfpathlineto{\pgfqpoint{0.762752in}{1.368108in}}%
\pgfpathlineto{\pgfqpoint{0.756722in}{1.352123in}}%
\pgfpathlineto{\pgfqpoint{0.749752in}{1.339519in}}%
\pgfpathlineto{\pgfqpoint{0.742201in}{1.330599in}}%
\pgfpathlineto{\pgfqpoint{0.734854in}{1.325312in}}%
\pgfpathlineto{\pgfqpoint{0.726558in}{1.322419in}}%
\pgfpathlineto{\pgfqpoint{0.717884in}{1.322223in}}%
\pgfpathlineto{\pgfqpoint{0.709412in}{1.324411in}}%
\pgfpathlineto{\pgfqpoint{0.699548in}{1.329604in}}%
\pgfpathlineto{\pgfqpoint{0.688894in}{1.338203in}}%
\pgfpathlineto{\pgfqpoint{0.677907in}{1.350248in}}%
\pgfpathlineto{\pgfqpoint{0.666886in}{1.365647in}}%
\pgfpathlineto{\pgfqpoint{0.654913in}{1.386417in}}%
\pgfpathlineto{\pgfqpoint{0.642574in}{1.412730in}}%
\pgfpathlineto{\pgfqpoint{0.630328in}{1.444629in}}%
\pgfpathlineto{\pgfqpoint{0.618504in}{1.482081in}}%
\pgfpathlineto{\pgfqpoint{0.608613in}{1.520256in}}%
\pgfpathlineto{\pgfqpoint{0.590203in}{1.612445in}}%
\pgfpathlineto{\pgfqpoint{0.581848in}{1.668884in}}%
\pgfpathlineto{\pgfqpoint{0.573137in}{1.740376in}}%
\pgfpathlineto{\pgfqpoint{0.567062in}{1.807213in}}%
\pgfpathlineto{\pgfqpoint{0.560532in}{1.896510in}}%
\pgfpathlineto{\pgfqpoint{0.555526in}{1.995910in}}%
\pgfpathlineto{\pgfqpoint{0.552564in}{2.097908in}}%
\pgfpathlineto{\pgfqpoint{0.551526in}{2.204935in}}%
\pgfpathlineto{\pgfqpoint{0.552728in}{2.309470in}}%
\pgfpathlineto{\pgfqpoint{0.556011in}{2.403981in}}%
\pgfpathlineto{\pgfqpoint{0.560953in}{2.483430in}}%
\pgfpathlineto{\pgfqpoint{0.567303in}{2.550240in}}%
\pgfpathlineto{\pgfqpoint{0.574928in}{2.606817in}}%
\pgfpathlineto{\pgfqpoint{0.582988in}{2.650657in}}%
\pgfpathlineto{\pgfqpoint{0.592756in}{2.691452in}}%
\pgfpathlineto{\pgfqpoint{0.602650in}{2.721756in}}%
\pgfpathlineto{\pgfqpoint{0.612983in}{2.746441in}}%
\pgfpathlineto{\pgfqpoint{0.624292in}{2.767692in}}%
\pgfpathlineto{\pgfqpoint{0.636231in}{2.785433in}}%
\pgfpathlineto{\pgfqpoint{0.649892in}{2.801461in}}%
\pgfpathlineto{\pgfqpoint{0.663386in}{2.814020in}}%
\pgfpathlineto{\pgfqpoint{0.679842in}{2.826135in}}%
\pgfpathlineto{\pgfqpoint{0.697326in}{2.836197in}}%
\pgfpathlineto{\pgfqpoint{0.715574in}{2.844285in}}%
\pgfpathlineto{\pgfqpoint{0.738439in}{2.852335in}}%
\pgfpathlineto{\pgfqpoint{0.765983in}{2.859639in}}%
\pgfpathlineto{\pgfqpoint{0.800300in}{2.866256in}}%
\pgfpathlineto{\pgfqpoint{0.841340in}{2.871832in}}%
\pgfpathlineto{\pgfqpoint{0.895547in}{2.876803in}}%
\pgfpathlineto{\pgfqpoint{0.969413in}{2.881069in}}%
\pgfpathlineto{\pgfqpoint{1.071608in}{2.884501in}}%
\pgfpathlineto{\pgfqpoint{1.219512in}{2.887074in}}%
\pgfpathlineto{\pgfqpoint{1.471844in}{2.889091in}}%
\pgfpathlineto{\pgfqpoint{1.956941in}{2.890384in}}%
\pgfpathlineto{\pgfqpoint{3.096814in}{2.890781in}}%
\pgfpathlineto{\pgfqpoint{3.995224in}{2.889388in}}%
\pgfpathlineto{\pgfqpoint{4.275833in}{2.887011in}}%
\pgfpathlineto{\pgfqpoint{4.412847in}{2.883743in}}%
\pgfpathlineto{\pgfqpoint{4.491081in}{2.879810in}}%
\pgfpathlineto{\pgfqpoint{4.543127in}{2.875163in}}%
\pgfpathlineto{\pgfqpoint{4.579810in}{2.869841in}}%
\pgfpathlineto{\pgfqpoint{4.607580in}{2.863763in}}%
\pgfpathlineto{\pgfqpoint{4.630623in}{2.856424in}}%
\pgfpathlineto{\pgfqpoint{4.648833in}{2.848228in}}%
\pgfpathlineto{\pgfqpoint{4.664136in}{2.838773in}}%
\pgfpathlineto{\pgfqpoint{4.676470in}{2.828576in}}%
\pgfpathlineto{\pgfqpoint{4.687502in}{2.816585in}}%
\pgfpathlineto{\pgfqpoint{4.697051in}{2.803027in}}%
\pgfpathlineto{\pgfqpoint{4.706194in}{2.786098in}}%
\pgfpathlineto{\pgfqpoint{4.714508in}{2.765827in}}%
\pgfpathlineto{\pgfqpoint{4.722462in}{2.740013in}}%
\pgfpathlineto{\pgfqpoint{4.729577in}{2.708703in}}%
\pgfpathlineto{\pgfqpoint{4.736162in}{2.669601in}}%
\pgfpathlineto{\pgfqpoint{4.742419in}{2.617826in}}%
\pgfpathlineto{\pgfqpoint{4.747859in}{2.553410in}}%
\pgfpathlineto{\pgfqpoint{4.752661in}{2.468958in}}%
\pgfpathlineto{\pgfqpoint{4.756610in}{2.359528in}}%
\pgfpathlineto{\pgfqpoint{4.759416in}{2.217681in}}%
\pgfpathlineto{\pgfqpoint{4.760596in}{2.043444in}}%
\pgfpathlineto{\pgfqpoint{4.759662in}{1.851779in}}%
\pgfpathlineto{\pgfqpoint{4.756587in}{1.667613in}}%
\pgfpathlineto{\pgfqpoint{4.751596in}{1.503428in}}%
\pgfpathlineto{\pgfqpoint{4.745410in}{1.374185in}}%
\pgfpathlineto{\pgfqpoint{4.738113in}{1.267479in}}%
\pgfpathlineto{\pgfqpoint{4.729621in}{1.175896in}}%
\pgfpathlineto{\pgfqpoint{4.720762in}{1.104428in}}%
\pgfpathlineto{\pgfqpoint{4.711045in}{1.043204in}}%
\pgfpathlineto{\pgfqpoint{4.700364in}{0.989829in}}%
\pgfpathlineto{\pgfqpoint{4.689055in}{0.944345in}}%
\pgfpathlineto{\pgfqpoint{4.676881in}{0.904394in}}%
\pgfpathlineto{\pgfqpoint{4.676095in}{0.902073in}}%
\pgfpathlineto{\pgfqpoint{4.676095in}{0.902073in}}%
\pgfusepath{stroke}%
\end{pgfscope}%
\begin{pgfscope}%
\pgfpathrectangle{\pgfqpoint{0.448634in}{0.402556in}}{\pgfqpoint{4.350661in}{2.489204in}} %
\pgfusepath{clip}%
\pgfsetbuttcap%
\pgfsetroundjoin%
\pgfsetlinewidth{1.003750pt}%
\definecolor{currentstroke}{rgb}{0.000000,0.000000,0.000000}%
\pgfsetstrokecolor{currentstroke}%
\pgfsetdash{{1.000000pt}{1.650000pt}}{0.000000pt}%
\pgfpathmoveto{\pgfqpoint{2.795520in}{1.982745in}}%
\pgfpathlineto{\pgfqpoint{2.781780in}{1.874357in}}%
\pgfpathlineto{\pgfqpoint{2.769351in}{1.758234in}}%
\pgfpathlineto{\pgfqpoint{2.758095in}{1.631942in}}%
\pgfpathlineto{\pgfqpoint{2.747786in}{1.490551in}}%
\pgfpathlineto{\pgfqpoint{2.738644in}{1.334082in}}%
\pgfpathlineto{\pgfqpoint{2.730580in}{1.157591in}}%
\pgfpathlineto{\pgfqpoint{2.723334in}{0.948663in}}%
\pgfpathlineto{\pgfqpoint{2.709783in}{0.530788in}}%
\pgfpathlineto{\pgfqpoint{2.705868in}{0.488716in}}%
\pgfpathlineto{\pgfqpoint{2.701769in}{0.464281in}}%
\pgfpathlineto{\pgfqpoint{2.697021in}{0.447744in}}%
\pgfpathlineto{\pgfqpoint{2.691859in}{0.436812in}}%
\pgfpathlineto{\pgfqpoint{2.686245in}{0.429229in}}%
\pgfpathlineto{\pgfqpoint{2.679348in}{0.423188in}}%
\pgfpathlineto{\pgfqpoint{2.669540in}{0.417856in}}%
\pgfpathlineto{\pgfqpoint{2.656987in}{0.413810in}}%
\pgfpathlineto{\pgfqpoint{2.637654in}{0.410337in}}%
\pgfpathlineto{\pgfqpoint{2.607297in}{0.407617in}}%
\pgfpathlineto{\pgfqpoint{2.555121in}{0.405574in}}%
\pgfpathlineto{\pgfqpoint{2.450714in}{0.404139in}}%
\pgfpathlineto{\pgfqpoint{2.176624in}{0.403275in}}%
\pgfpathlineto{\pgfqpoint{1.130290in}{0.402953in}}%
\pgfpathlineto{\pgfqpoint{0.516849in}{0.404175in}}%
\pgfpathlineto{\pgfqpoint{0.466848in}{0.405970in}}%
\pgfpathlineto{\pgfqpoint{0.456130in}{0.407931in}}%
\pgfpathlineto{\pgfqpoint{0.452340in}{0.410303in}}%
\pgfpathlineto{\pgfqpoint{0.450346in}{0.414662in}}%
\pgfpathlineto{\pgfqpoint{0.449266in}{0.424524in}}%
\pgfpathlineto{\pgfqpoint{0.448771in}{0.464344in}}%
\pgfpathlineto{\pgfqpoint{0.448640in}{0.850171in}}%
\pgfpathlineto{\pgfqpoint{0.448653in}{2.891318in}}%
\pgfpathlineto{\pgfqpoint{0.448653in}{2.891318in}}%
\pgfusepath{stroke}%
\end{pgfscope}%
\begin{pgfscope}%
\pgfpathrectangle{\pgfqpoint{0.448634in}{0.402556in}}{\pgfqpoint{4.350661in}{2.489204in}} %
\pgfusepath{clip}%
\pgfsetbuttcap%
\pgfsetroundjoin%
\pgfsetlinewidth{1.003750pt}%
\definecolor{currentstroke}{rgb}{0.000000,0.000000,0.000000}%
\pgfsetstrokecolor{currentstroke}%
\pgfsetdash{{1.000000pt}{1.650000pt}}{0.000000pt}%
\pgfpathmoveto{\pgfqpoint{3.428189in}{0.402586in}}%
\pgfpathlineto{\pgfqpoint{2.782121in}{0.403701in}}%
\pgfpathlineto{\pgfqpoint{2.753906in}{0.405674in}}%
\pgfpathlineto{\pgfqpoint{2.743328in}{0.408443in}}%
\pgfpathlineto{\pgfqpoint{2.737717in}{0.412188in}}%
\pgfpathlineto{\pgfqpoint{2.733668in}{0.417995in}}%
\pgfpathlineto{\pgfqpoint{2.730649in}{0.427307in}}%
\pgfpathlineto{\pgfqpoint{2.728388in}{0.442004in}}%
\pgfpathlineto{\pgfqpoint{2.726544in}{0.471794in}}%
\pgfpathlineto{\pgfqpoint{2.725216in}{0.534003in}}%
\pgfpathlineto{\pgfqpoint{2.725169in}{0.655973in}}%
\pgfpathlineto{\pgfqpoint{2.727377in}{0.832687in}}%
\pgfpathlineto{\pgfqpoint{2.732259in}{1.041703in}}%
\pgfpathlineto{\pgfqpoint{2.738851in}{1.223257in}}%
\pgfpathlineto{\pgfqpoint{2.747078in}{1.389766in}}%
\pgfpathlineto{\pgfqpoint{2.756608in}{1.538717in}}%
\pgfpathlineto{\pgfqpoint{2.768955in}{1.694887in}}%
\pgfpathlineto{\pgfqpoint{2.781228in}{1.816044in}}%
\pgfpathlineto{\pgfqpoint{2.794401in}{1.924524in}}%
\pgfpathlineto{\pgfqpoint{2.812737in}{2.054722in}}%
\pgfpathlineto{\pgfqpoint{2.828774in}{2.147512in}}%
\pgfpathlineto{\pgfqpoint{2.847382in}{2.242224in}}%
\pgfpathlineto{\pgfqpoint{2.895818in}{2.479699in}}%
\pgfpathlineto{\pgfqpoint{2.900204in}{2.516689in}}%
\pgfpathlineto{\pgfqpoint{2.901346in}{2.544029in}}%
\pgfpathlineto{\pgfqpoint{2.900291in}{2.566388in}}%
\pgfpathlineto{\pgfqpoint{2.897334in}{2.585999in}}%
\pgfpathlineto{\pgfqpoint{2.892836in}{2.602633in}}%
\pgfpathlineto{\pgfqpoint{2.886394in}{2.618405in}}%
\pgfpathlineto{\pgfqpoint{2.878058in}{2.632969in}}%
\pgfpathlineto{\pgfqpoint{2.868065in}{2.646100in}}%
\pgfpathlineto{\pgfqpoint{2.855050in}{2.659300in}}%
\pgfpathlineto{\pgfqpoint{2.840801in}{2.670717in}}%
\pgfpathlineto{\pgfqpoint{2.821822in}{2.682861in}}%
\pgfpathlineto{\pgfqpoint{2.799980in}{2.694026in}}%
\pgfpathlineto{\pgfqpoint{2.773366in}{2.704944in}}%
\pgfpathlineto{\pgfqpoint{2.742012in}{2.715266in}}%
\pgfpathlineto{\pgfqpoint{2.705983in}{2.724785in}}%
\pgfpathlineto{\pgfqpoint{2.663200in}{2.733810in}}%
\pgfpathlineto{\pgfqpoint{2.611535in}{2.742379in}}%
\pgfpathlineto{\pgfqpoint{2.551002in}{2.750090in}}%
\pgfpathlineto{\pgfqpoint{2.481632in}{2.756682in}}%
\pgfpathlineto{\pgfqpoint{2.399112in}{2.762200in}}%
\pgfpathlineto{\pgfqpoint{2.309985in}{2.765886in}}%
\pgfpathlineto{\pgfqpoint{2.188184in}{2.768096in}}%
\pgfpathlineto{\pgfqpoint{2.081595in}{2.767619in}}%
\pgfpathlineto{\pgfqpoint{1.968506in}{2.764840in}}%
\pgfpathlineto{\pgfqpoint{1.864180in}{2.759918in}}%
\pgfpathlineto{\pgfqpoint{1.757786in}{2.752593in}}%
\pgfpathlineto{\pgfqpoint{1.671087in}{2.744171in}}%
\pgfpathlineto{\pgfqpoint{1.591076in}{2.734193in}}%
\pgfpathlineto{\pgfqpoint{1.502689in}{2.720717in}}%
\pgfpathlineto{\pgfqpoint{1.427655in}{2.706083in}}%
\pgfpathlineto{\pgfqpoint{1.372350in}{2.692544in}}%
\pgfpathlineto{\pgfqpoint{1.321734in}{2.677921in}}%
\pgfpathlineto{\pgfqpoint{1.273765in}{2.661664in}}%
\pgfpathlineto{\pgfqpoint{1.230567in}{2.644672in}}%
\pgfpathlineto{\pgfqpoint{1.192197in}{2.627106in}}%
\pgfpathlineto{\pgfqpoint{1.156620in}{2.608403in}}%
\pgfpathlineto{\pgfqpoint{1.123890in}{2.588716in}}%
\pgfpathlineto{\pgfqpoint{1.095883in}{2.569568in}}%
\pgfpathlineto{\pgfqpoint{1.063936in}{2.543701in}}%
\pgfpathlineto{\pgfqpoint{1.038217in}{2.520732in}}%
\pgfpathlineto{\pgfqpoint{1.013766in}{2.496016in}}%
\pgfpathlineto{\pgfqpoint{0.990704in}{2.469610in}}%
\pgfpathlineto{\pgfqpoint{0.969124in}{2.441612in}}%
\pgfpathlineto{\pgfqpoint{0.949083in}{2.412154in}}%
\pgfpathlineto{\pgfqpoint{0.930604in}{2.381387in}}%
\pgfpathlineto{\pgfqpoint{0.906555in}{2.334052in}}%
\pgfpathlineto{\pgfqpoint{0.889925in}{2.296262in}}%
\pgfpathlineto{\pgfqpoint{0.874241in}{2.255213in}}%
\pgfpathlineto{\pgfqpoint{0.859667in}{2.210961in}}%
\pgfpathlineto{\pgfqpoint{0.846986in}{2.165954in}}%
\pgfpathlineto{\pgfqpoint{0.839633in}{2.134715in}}%
\pgfpathlineto{\pgfqpoint{0.828238in}{2.081532in}}%
\pgfpathlineto{\pgfqpoint{0.817866in}{2.022986in}}%
\pgfpathlineto{\pgfqpoint{0.810784in}{1.971352in}}%
\pgfpathlineto{\pgfqpoint{0.802846in}{1.902252in}}%
\pgfpathlineto{\pgfqpoint{0.796554in}{1.827927in}}%
\pgfpathlineto{\pgfqpoint{0.791696in}{1.743480in}}%
\pgfpathlineto{\pgfqpoint{0.787773in}{1.621595in}}%
\pgfpathlineto{\pgfqpoint{0.785408in}{1.522064in}}%
\pgfpathlineto{\pgfqpoint{0.785408in}{1.522064in}}%
\pgfusepath{stroke}%
\end{pgfscope}%
\begin{pgfscope}%
\pgfpathrectangle{\pgfqpoint{0.448634in}{0.402556in}}{\pgfqpoint{4.350661in}{2.489204in}} %
\pgfusepath{clip}%
\pgfsetbuttcap%
\pgfsetroundjoin%
\pgfsetlinewidth{1.003750pt}%
\definecolor{currentstroke}{rgb}{0.000000,0.000000,0.000000}%
\pgfsetstrokecolor{currentstroke}%
\pgfsetdash{{1.000000pt}{1.650000pt}}{0.000000pt}%
\pgfpathmoveto{\pgfqpoint{2.028735in}{0.425754in}}%
\pgfpathlineto{\pgfqpoint{1.878677in}{0.421879in}}%
\pgfpathlineto{\pgfqpoint{1.676387in}{0.418997in}}%
\pgfpathlineto{\pgfqpoint{1.413176in}{0.417558in}}%
\pgfpathlineto{\pgfqpoint{1.134735in}{0.418204in}}%
\pgfpathlineto{\pgfqpoint{0.921565in}{0.420769in}}%
\pgfpathlineto{\pgfqpoint{0.782384in}{0.424523in}}%
\pgfpathlineto{\pgfqpoint{0.693283in}{0.428974in}}%
\pgfpathlineto{\pgfqpoint{0.632541in}{0.434091in}}%
\pgfpathlineto{\pgfqpoint{0.591492in}{0.439564in}}%
\pgfpathlineto{\pgfqpoint{0.561503in}{0.445595in}}%
\pgfpathlineto{\pgfqpoint{0.538349in}{0.452466in}}%
\pgfpathlineto{\pgfqpoint{0.522042in}{0.459394in}}%
\pgfpathlineto{\pgfqpoint{0.508540in}{0.467420in}}%
\pgfpathlineto{\pgfqpoint{0.497973in}{0.476161in}}%
\pgfpathlineto{\pgfqpoint{0.488790in}{0.486750in}}%
\pgfpathlineto{\pgfqpoint{0.481284in}{0.498948in}}%
\pgfpathlineto{\pgfqpoint{0.474590in}{0.514580in}}%
\pgfpathlineto{\pgfqpoint{0.469106in}{0.533467in}}%
\pgfpathlineto{\pgfqpoint{0.464439in}{0.557771in}}%
\pgfpathlineto{\pgfqpoint{0.460297in}{0.592289in}}%
\pgfpathlineto{\pgfqpoint{0.456856in}{0.641912in}}%
\pgfpathlineto{\pgfqpoint{0.454122in}{0.716520in}}%
\pgfpathlineto{\pgfqpoint{0.451978in}{0.843444in}}%
\pgfpathlineto{\pgfqpoint{0.450459in}{1.087380in}}%
\pgfpathlineto{\pgfqpoint{0.449596in}{1.657406in}}%
\pgfpathlineto{\pgfqpoint{0.450150in}{2.687936in}}%
\pgfpathlineto{\pgfqpoint{0.451781in}{2.839761in}}%
\pgfpathlineto{\pgfqpoint{0.453975in}{2.872003in}}%
\pgfpathlineto{\pgfqpoint{0.456339in}{2.881553in}}%
\pgfpathlineto{\pgfqpoint{0.458888in}{2.885549in}}%
\pgfpathlineto{\pgfqpoint{0.462554in}{2.888171in}}%
\pgfpathlineto{\pgfqpoint{0.471046in}{2.890205in}}%
\pgfpathlineto{\pgfqpoint{0.490597in}{2.891263in}}%
\pgfpathlineto{\pgfqpoint{0.564556in}{2.891692in}}%
\pgfpathlineto{\pgfqpoint{1.569559in}{2.891759in}}%
\pgfpathlineto{\pgfqpoint{4.784679in}{2.890785in}}%
\pgfpathlineto{\pgfqpoint{4.791005in}{2.889098in}}%
\pgfpathlineto{\pgfqpoint{4.793910in}{2.885555in}}%
\pgfpathlineto{\pgfqpoint{4.795579in}{2.878366in}}%
\pgfpathlineto{\pgfqpoint{4.796850in}{2.858513in}}%
\pgfpathlineto{\pgfqpoint{4.796850in}{2.858513in}}%
\pgfusepath{stroke}%
\end{pgfscope}%
\begin{pgfscope}%
\pgfsetrectcap%
\pgfsetmiterjoin%
\pgfsetlinewidth{0.803000pt}%
\definecolor{currentstroke}{rgb}{0.000000,0.000000,0.000000}%
\pgfsetstrokecolor{currentstroke}%
\pgfsetdash{}{0pt}%
\pgfpathmoveto{\pgfqpoint{0.448634in}{0.402556in}}%
\pgfpathlineto{\pgfqpoint{0.448634in}{2.891760in}}%
\pgfusepath{stroke}%
\end{pgfscope}%
\begin{pgfscope}%
\pgfsetrectcap%
\pgfsetmiterjoin%
\pgfsetlinewidth{0.803000pt}%
\definecolor{currentstroke}{rgb}{0.000000,0.000000,0.000000}%
\pgfsetstrokecolor{currentstroke}%
\pgfsetdash{}{0pt}%
\pgfpathmoveto{\pgfqpoint{4.799294in}{0.402556in}}%
\pgfpathlineto{\pgfqpoint{4.799294in}{2.891760in}}%
\pgfusepath{stroke}%
\end{pgfscope}%
\begin{pgfscope}%
\pgfsetrectcap%
\pgfsetmiterjoin%
\pgfsetlinewidth{0.803000pt}%
\definecolor{currentstroke}{rgb}{0.000000,0.000000,0.000000}%
\pgfsetstrokecolor{currentstroke}%
\pgfsetdash{}{0pt}%
\pgfpathmoveto{\pgfqpoint{0.448634in}{0.402556in}}%
\pgfpathlineto{\pgfqpoint{4.799294in}{0.402556in}}%
\pgfusepath{stroke}%
\end{pgfscope}%
\begin{pgfscope}%
\pgfsetrectcap%
\pgfsetmiterjoin%
\pgfsetlinewidth{0.803000pt}%
\definecolor{currentstroke}{rgb}{0.000000,0.000000,0.000000}%
\pgfsetstrokecolor{currentstroke}%
\pgfsetdash{}{0pt}%
\pgfpathmoveto{\pgfqpoint{0.448634in}{2.891760in}}%
\pgfpathlineto{\pgfqpoint{4.799294in}{2.891760in}}%
\pgfusepath{stroke}%
\end{pgfscope}%
\begin{pgfscope}%
\pgfsetbuttcap%
\pgfsetmiterjoin%
\definecolor{currentfill}{rgb}{1.000000,1.000000,1.000000}%
\pgfsetfillcolor{currentfill}%
\pgfsetfillopacity{0.500000}%
\pgfsetlinewidth{1.003750pt}%
\definecolor{currentstroke}{rgb}{0.800000,0.800000,0.800000}%
\pgfsetstrokecolor{currentstroke}%
\pgfsetstrokeopacity{0.500000}%
\pgfsetdash{}{0pt}%
\pgfpathmoveto{\pgfqpoint{3.700085in}{0.761312in}}%
\pgfpathlineto{\pgfqpoint{4.547129in}{0.761312in}}%
\pgfpathquadraticcurveto{\pgfqpoint{4.574907in}{0.761312in}}{\pgfqpoint{4.574907in}{0.789090in}}%
\pgfpathlineto{\pgfqpoint{4.574907in}{2.769646in}}%
\pgfpathquadraticcurveto{\pgfqpoint{4.574907in}{2.797424in}}{\pgfqpoint{4.547129in}{2.797424in}}%
\pgfpathlineto{\pgfqpoint{3.700085in}{2.797424in}}%
\pgfpathquadraticcurveto{\pgfqpoint{3.672307in}{2.797424in}}{\pgfqpoint{3.672307in}{2.769646in}}%
\pgfpathlineto{\pgfqpoint{3.672307in}{0.789090in}}%
\pgfpathquadraticcurveto{\pgfqpoint{3.672307in}{0.761312in}}{\pgfqpoint{3.700085in}{0.761312in}}%
\pgfpathclose%
\pgfusepath{stroke,fill}%
\end{pgfscope}%
\begin{pgfscope}%
\pgfsetrectcap%
\pgfsetroundjoin%
\pgfsetlinewidth{1.003750pt}%
\definecolor{currentstroke}{rgb}{0.121569,0.466667,0.705882}%
\pgfsetstrokecolor{currentstroke}%
\pgfsetdash{}{0pt}%
\pgfpathmoveto{\pgfqpoint{3.727863in}{2.693257in}}%
\pgfpathlineto{\pgfqpoint{3.797307in}{2.693257in}}%
\pgfusepath{stroke}%
\end{pgfscope}%
\begin{pgfscope}%
\pgftext[x=3.908418in,y=2.644646in,left,base]{\rmfamily\fontsize{10.000000}{12.000000}\selectfont \(\displaystyle \textnormal{tol}=10^{{-}10}\)}%
\end{pgfscope}%
\begin{pgfscope}%
\pgfsetrectcap%
\pgfsetroundjoin%
\pgfsetlinewidth{1.003750pt}%
\definecolor{currentstroke}{rgb}{1.000000,0.498039,0.054902}%
\pgfsetstrokecolor{currentstroke}%
\pgfsetdash{}{0pt}%
\pgfpathmoveto{\pgfqpoint{3.727863in}{2.493812in}}%
\pgfpathlineto{\pgfqpoint{3.797307in}{2.493812in}}%
\pgfusepath{stroke}%
\end{pgfscope}%
\begin{pgfscope}%
\pgftext[x=3.908418in,y=2.445201in,left,base]{\rmfamily\fontsize{10.000000}{12.000000}\selectfont \(\displaystyle \textnormal{tol}=10^{{-}9}\)}%
\end{pgfscope}%
\begin{pgfscope}%
\pgfsetrectcap%
\pgfsetroundjoin%
\pgfsetlinewidth{1.003750pt}%
\definecolor{currentstroke}{rgb}{0.172549,0.627451,0.172549}%
\pgfsetstrokecolor{currentstroke}%
\pgfsetdash{}{0pt}%
\pgfpathmoveto{\pgfqpoint{3.727863in}{2.294368in}}%
\pgfpathlineto{\pgfqpoint{3.797307in}{2.294368in}}%
\pgfusepath{stroke}%
\end{pgfscope}%
\begin{pgfscope}%
\pgftext[x=3.908418in,y=2.245757in,left,base]{\rmfamily\fontsize{10.000000}{12.000000}\selectfont \(\displaystyle \textnormal{tol}=10^{{-}8}\)}%
\end{pgfscope}%
\begin{pgfscope}%
\pgfsetrectcap%
\pgfsetroundjoin%
\pgfsetlinewidth{1.003750pt}%
\definecolor{currentstroke}{rgb}{0.839216,0.152941,0.156863}%
\pgfsetstrokecolor{currentstroke}%
\pgfsetdash{}{0pt}%
\pgfpathmoveto{\pgfqpoint{3.727863in}{2.094924in}}%
\pgfpathlineto{\pgfqpoint{3.797307in}{2.094924in}}%
\pgfusepath{stroke}%
\end{pgfscope}%
\begin{pgfscope}%
\pgftext[x=3.908418in,y=2.046312in,left,base]{\rmfamily\fontsize{10.000000}{12.000000}\selectfont \(\displaystyle \textnormal{tol}=10^{{-}7}\)}%
\end{pgfscope}%
\begin{pgfscope}%
\pgfsetrectcap%
\pgfsetroundjoin%
\pgfsetlinewidth{1.003750pt}%
\definecolor{currentstroke}{rgb}{0.580392,0.403922,0.741176}%
\pgfsetstrokecolor{currentstroke}%
\pgfsetdash{}{0pt}%
\pgfpathmoveto{\pgfqpoint{3.727863in}{1.895479in}}%
\pgfpathlineto{\pgfqpoint{3.797307in}{1.895479in}}%
\pgfusepath{stroke}%
\end{pgfscope}%
\begin{pgfscope}%
\pgftext[x=3.908418in,y=1.846868in,left,base]{\rmfamily\fontsize{10.000000}{12.000000}\selectfont \(\displaystyle \textnormal{tol}=10^{{-}6}\)}%
\end{pgfscope}%
\begin{pgfscope}%
\pgfsetrectcap%
\pgfsetroundjoin%
\pgfsetlinewidth{1.003750pt}%
\definecolor{currentstroke}{rgb}{0.549020,0.337255,0.294118}%
\pgfsetstrokecolor{currentstroke}%
\pgfsetdash{}{0pt}%
\pgfpathmoveto{\pgfqpoint{3.727863in}{1.696035in}}%
\pgfpathlineto{\pgfqpoint{3.797307in}{1.696035in}}%
\pgfusepath{stroke}%
\end{pgfscope}%
\begin{pgfscope}%
\pgftext[x=3.908418in,y=1.647424in,left,base]{\rmfamily\fontsize{10.000000}{12.000000}\selectfont \(\displaystyle \textnormal{tol}=10^{{-}5}\)}%
\end{pgfscope}%
\begin{pgfscope}%
\pgfsetrectcap%
\pgfsetroundjoin%
\pgfsetlinewidth{1.003750pt}%
\definecolor{currentstroke}{rgb}{0.890196,0.466667,0.760784}%
\pgfsetstrokecolor{currentstroke}%
\pgfsetdash{}{0pt}%
\pgfpathmoveto{\pgfqpoint{3.727863in}{1.496590in}}%
\pgfpathlineto{\pgfqpoint{3.797307in}{1.496590in}}%
\pgfusepath{stroke}%
\end{pgfscope}%
\begin{pgfscope}%
\pgftext[x=3.908418in,y=1.447979in,left,base]{\rmfamily\fontsize{10.000000}{12.000000}\selectfont \(\displaystyle \textnormal{tol}=10^{{-}4}\)}%
\end{pgfscope}%
\begin{pgfscope}%
\pgfsetrectcap%
\pgfsetroundjoin%
\pgfsetlinewidth{1.003750pt}%
\definecolor{currentstroke}{rgb}{0.498039,0.498039,0.498039}%
\pgfsetstrokecolor{currentstroke}%
\pgfsetdash{}{0pt}%
\pgfpathmoveto{\pgfqpoint{3.727863in}{1.297146in}}%
\pgfpathlineto{\pgfqpoint{3.797307in}{1.297146in}}%
\pgfusepath{stroke}%
\end{pgfscope}%
\begin{pgfscope}%
\pgftext[x=3.908418in,y=1.248535in,left,base]{\rmfamily\fontsize{10.000000}{12.000000}\selectfont \(\displaystyle \textnormal{tol}=10^{{-}3}\)}%
\end{pgfscope}%
\begin{pgfscope}%
\pgfsetrectcap%
\pgfsetroundjoin%
\pgfsetlinewidth{1.003750pt}%
\definecolor{currentstroke}{rgb}{0.737255,0.741176,0.133333}%
\pgfsetstrokecolor{currentstroke}%
\pgfsetdash{}{0pt}%
\pgfpathmoveto{\pgfqpoint{3.727863in}{1.097701in}}%
\pgfpathlineto{\pgfqpoint{3.797307in}{1.097701in}}%
\pgfusepath{stroke}%
\end{pgfscope}%
\begin{pgfscope}%
\pgftext[x=3.908418in,y=1.049090in,left,base]{\rmfamily\fontsize{10.000000}{12.000000}\selectfont \(\displaystyle \textnormal{tol}=10^{{-}2}\)}%
\end{pgfscope}%
\begin{pgfscope}%
\pgfsetbuttcap%
\pgfsetroundjoin%
\pgfsetlinewidth{1.003750pt}%
\definecolor{currentstroke}{rgb}{0.000000,0.000000,0.000000}%
\pgfsetstrokecolor{currentstroke}%
\pgfsetdash{{1.000000pt}{1.650000pt}}{0.000000pt}%
\pgfpathmoveto{\pgfqpoint{3.727863in}{0.898257in}}%
\pgfpathlineto{\pgfqpoint{3.797307in}{0.898257in}}%
\pgfusepath{stroke}%
\end{pgfscope}%
\begin{pgfscope}%
\pgftext[x=3.908418in,y=0.849646in,left,base]{\rmfamily\fontsize{10.000000}{12.000000}\selectfont \textnormal{Reference}}%
\end{pgfscope}%
\end{pgfpicture}%
\makeatother%
\endgroup%

    \caption[LCS curves found by means of the Bogacki-Shampine 5(4) integration
    scheme]{
        LCS curves found by means of the Bogacki-Shampine 5(4) integration
        scheme. The reference LCS, as shown by itself in figure
        \ref{fig:referencelcs}, is dashed on the top layer. Note that
        the LCS for the lowest tolerance level considered, that is,
        $\textnormal{tol}=0.1$, is not included. This is because the
        corresponding $\mathcal{U}_{0}$ domain, shown in figure
        \ref{fig:u0_dom_err_bs54}, and the reference $\mathcal{U}_{0}$, shown
        in figure~\ref{fig:u0_domain} are very dissimilar. Here, there are visible
        discrepancies for all tolerance levels $\textnormal{tol}>10^{-6}$.
        These are prominent all over the domain.}
    \label{fig:lcs_rkbs54}
\end{figure}


%\clearpage
\begin{figure}[htpb]
    \centering
    \includegraphics[width=0.9\linewidth]{figures/lcs_figures/rkdp54.pdf}
    \caption[LCS curves found by means of the Dormand-Prince 5(4) integration
    scheme]{
        LCS curves found by means of the Dormand-Prince 5(4) integration
        scheme. The reference LCS, as shown by itself in figure
        \ref{fig:referencelcs}, is plotted on the bottom layer. Note that
        the LCS for the lowest tolerance level considered, that is,
        $\textnormal{tol}=0.1$, is not included. This is because the
        corresponding $\mathcal{U}_{0}$ domain, shown in figure
        \ref{fig:u0_dp54}, and the reference $\mathcal{U}_{0}$, shown in figure
        \ref{fig:u0_domain} are dissimilar. Here, there are visible
        disparities for for all tolerance levels $\textnormal{tol}>10^{-6}$.}
    \label{fig:lcs_rkdp54}
\end{figure}


%\clearpage
\begin{figure}[htpb]
    \centering
    %% Creator: Matplotlib, PGF backend
%%
%% To include the figure in your LaTeX document, write
%%   \input{<filename>.pgf}
%%
%% Make sure the required packages are loaded in your preamble
%%   \usepackage{pgf}
%%
%% Figures using additional raster images can only be included by \input if
%% they are in the same directory as the main LaTeX file. For loading figures
%% from other directories you can use the `import` package
%%   \usepackage{import}
%% and then include the figures with
%%   \import{<path to file>}{<filename>.pgf}
%%
%% Matplotlib used the following preamble
%%   \usepackage[utf8x]{inputenc}
%%   \usepackage[T1]{fontenc}
%%   \usepackage[]{libertine}\usepackage[libertine]{newtxmath}
%%
\begingroup%
\makeatletter%
\begin{pgfpicture}%
\pgfpathrectangle{\pgfpointorigin}{\pgfqpoint{5.050000in}{3.100000in}}%
\pgfusepath{use as bounding box, clip}%
\begin{pgfscope}%
\pgfsetbuttcap%
\pgfsetmiterjoin%
\definecolor{currentfill}{rgb}{1.000000,1.000000,1.000000}%
\pgfsetfillcolor{currentfill}%
\pgfsetlinewidth{0.000000pt}%
\definecolor{currentstroke}{rgb}{1.000000,1.000000,1.000000}%
\pgfsetstrokecolor{currentstroke}%
\pgfsetdash{}{0pt}%
\pgfpathmoveto{\pgfqpoint{0.000000in}{0.000000in}}%
\pgfpathlineto{\pgfqpoint{5.050000in}{0.000000in}}%
\pgfpathlineto{\pgfqpoint{5.050000in}{3.100000in}}%
\pgfpathlineto{\pgfqpoint{0.000000in}{3.100000in}}%
\pgfpathclose%
\pgfusepath{fill}%
\end{pgfscope}%
\begin{pgfscope}%
\pgfsetbuttcap%
\pgfsetmiterjoin%
\definecolor{currentfill}{rgb}{1.000000,1.000000,1.000000}%
\pgfsetfillcolor{currentfill}%
\pgfsetlinewidth{0.000000pt}%
\definecolor{currentstroke}{rgb}{0.000000,0.000000,0.000000}%
\pgfsetstrokecolor{currentstroke}%
\pgfsetstrokeopacity{0.000000}%
\pgfsetdash{}{0pt}%
\pgfpathmoveto{\pgfqpoint{0.448634in}{0.402556in}}%
\pgfpathlineto{\pgfqpoint{4.799294in}{0.402556in}}%
\pgfpathlineto{\pgfqpoint{4.799294in}{2.891760in}}%
\pgfpathlineto{\pgfqpoint{0.448634in}{2.891760in}}%
\pgfpathclose%
\pgfusepath{fill}%
\end{pgfscope}%
\begin{pgfscope}%
\pgfsetbuttcap%
\pgfsetroundjoin%
\definecolor{currentfill}{rgb}{0.000000,0.000000,0.000000}%
\pgfsetfillcolor{currentfill}%
\pgfsetlinewidth{0.803000pt}%
\definecolor{currentstroke}{rgb}{0.000000,0.000000,0.000000}%
\pgfsetstrokecolor{currentstroke}%
\pgfsetdash{}{0pt}%
\pgfsys@defobject{currentmarker}{\pgfqpoint{0.000000in}{-0.048611in}}{\pgfqpoint{0.000000in}{0.000000in}}{%
\pgfpathmoveto{\pgfqpoint{0.000000in}{0.000000in}}%
\pgfpathlineto{\pgfqpoint{0.000000in}{-0.048611in}}%
\pgfusepath{stroke,fill}%
}%
\begin{pgfscope}%
\pgfsys@transformshift{0.448634in}{0.402556in}%
\pgfsys@useobject{currentmarker}{}%
\end{pgfscope}%
\end{pgfscope}%
\begin{pgfscope}%
\pgftext[x=0.448634in,y=0.305334in,,top]{\rmfamily\fontsize{12.000000}{14.400000}\selectfont \(\displaystyle 0.00\)}%
\end{pgfscope}%
\begin{pgfscope}%
\pgfsetbuttcap%
\pgfsetroundjoin%
\definecolor{currentfill}{rgb}{0.000000,0.000000,0.000000}%
\pgfsetfillcolor{currentfill}%
\pgfsetlinewidth{0.803000pt}%
\definecolor{currentstroke}{rgb}{0.000000,0.000000,0.000000}%
\pgfsetstrokecolor{currentstroke}%
\pgfsetdash{}{0pt}%
\pgfsys@defobject{currentmarker}{\pgfqpoint{0.000000in}{-0.048611in}}{\pgfqpoint{0.000000in}{0.000000in}}{%
\pgfpathmoveto{\pgfqpoint{0.000000in}{0.000000in}}%
\pgfpathlineto{\pgfqpoint{0.000000in}{-0.048611in}}%
\pgfusepath{stroke,fill}%
}%
\begin{pgfscope}%
\pgfsys@transformshift{0.992466in}{0.402556in}%
\pgfsys@useobject{currentmarker}{}%
\end{pgfscope}%
\end{pgfscope}%
\begin{pgfscope}%
\pgftext[x=0.992466in,y=0.305334in,,top]{\rmfamily\fontsize{12.000000}{14.400000}\selectfont \(\displaystyle 0.25\)}%
\end{pgfscope}%
\begin{pgfscope}%
\pgfsetbuttcap%
\pgfsetroundjoin%
\definecolor{currentfill}{rgb}{0.000000,0.000000,0.000000}%
\pgfsetfillcolor{currentfill}%
\pgfsetlinewidth{0.803000pt}%
\definecolor{currentstroke}{rgb}{0.000000,0.000000,0.000000}%
\pgfsetstrokecolor{currentstroke}%
\pgfsetdash{}{0pt}%
\pgfsys@defobject{currentmarker}{\pgfqpoint{0.000000in}{-0.048611in}}{\pgfqpoint{0.000000in}{0.000000in}}{%
\pgfpathmoveto{\pgfqpoint{0.000000in}{0.000000in}}%
\pgfpathlineto{\pgfqpoint{0.000000in}{-0.048611in}}%
\pgfusepath{stroke,fill}%
}%
\begin{pgfscope}%
\pgfsys@transformshift{1.536299in}{0.402556in}%
\pgfsys@useobject{currentmarker}{}%
\end{pgfscope}%
\end{pgfscope}%
\begin{pgfscope}%
\pgftext[x=1.536299in,y=0.305334in,,top]{\rmfamily\fontsize{12.000000}{14.400000}\selectfont \(\displaystyle 0.50\)}%
\end{pgfscope}%
\begin{pgfscope}%
\pgfsetbuttcap%
\pgfsetroundjoin%
\definecolor{currentfill}{rgb}{0.000000,0.000000,0.000000}%
\pgfsetfillcolor{currentfill}%
\pgfsetlinewidth{0.803000pt}%
\definecolor{currentstroke}{rgb}{0.000000,0.000000,0.000000}%
\pgfsetstrokecolor{currentstroke}%
\pgfsetdash{}{0pt}%
\pgfsys@defobject{currentmarker}{\pgfqpoint{0.000000in}{-0.048611in}}{\pgfqpoint{0.000000in}{0.000000in}}{%
\pgfpathmoveto{\pgfqpoint{0.000000in}{0.000000in}}%
\pgfpathlineto{\pgfqpoint{0.000000in}{-0.048611in}}%
\pgfusepath{stroke,fill}%
}%
\begin{pgfscope}%
\pgfsys@transformshift{2.080131in}{0.402556in}%
\pgfsys@useobject{currentmarker}{}%
\end{pgfscope}%
\end{pgfscope}%
\begin{pgfscope}%
\pgftext[x=2.080131in,y=0.305334in,,top]{\rmfamily\fontsize{12.000000}{14.400000}\selectfont \(\displaystyle 0.75\)}%
\end{pgfscope}%
\begin{pgfscope}%
\pgfsetbuttcap%
\pgfsetroundjoin%
\definecolor{currentfill}{rgb}{0.000000,0.000000,0.000000}%
\pgfsetfillcolor{currentfill}%
\pgfsetlinewidth{0.803000pt}%
\definecolor{currentstroke}{rgb}{0.000000,0.000000,0.000000}%
\pgfsetstrokecolor{currentstroke}%
\pgfsetdash{}{0pt}%
\pgfsys@defobject{currentmarker}{\pgfqpoint{0.000000in}{-0.048611in}}{\pgfqpoint{0.000000in}{0.000000in}}{%
\pgfpathmoveto{\pgfqpoint{0.000000in}{0.000000in}}%
\pgfpathlineto{\pgfqpoint{0.000000in}{-0.048611in}}%
\pgfusepath{stroke,fill}%
}%
\begin{pgfscope}%
\pgfsys@transformshift{2.623964in}{0.402556in}%
\pgfsys@useobject{currentmarker}{}%
\end{pgfscope}%
\end{pgfscope}%
\begin{pgfscope}%
\pgftext[x=2.623964in,y=0.305334in,,top]{\rmfamily\fontsize{12.000000}{14.400000}\selectfont \(\displaystyle 1.00\)}%
\end{pgfscope}%
\begin{pgfscope}%
\pgfsetbuttcap%
\pgfsetroundjoin%
\definecolor{currentfill}{rgb}{0.000000,0.000000,0.000000}%
\pgfsetfillcolor{currentfill}%
\pgfsetlinewidth{0.803000pt}%
\definecolor{currentstroke}{rgb}{0.000000,0.000000,0.000000}%
\pgfsetstrokecolor{currentstroke}%
\pgfsetdash{}{0pt}%
\pgfsys@defobject{currentmarker}{\pgfqpoint{0.000000in}{-0.048611in}}{\pgfqpoint{0.000000in}{0.000000in}}{%
\pgfpathmoveto{\pgfqpoint{0.000000in}{0.000000in}}%
\pgfpathlineto{\pgfqpoint{0.000000in}{-0.048611in}}%
\pgfusepath{stroke,fill}%
}%
\begin{pgfscope}%
\pgfsys@transformshift{3.167797in}{0.402556in}%
\pgfsys@useobject{currentmarker}{}%
\end{pgfscope}%
\end{pgfscope}%
\begin{pgfscope}%
\pgftext[x=3.167797in,y=0.305334in,,top]{\rmfamily\fontsize{12.000000}{14.400000}\selectfont \(\displaystyle 1.25\)}%
\end{pgfscope}%
\begin{pgfscope}%
\pgfsetbuttcap%
\pgfsetroundjoin%
\definecolor{currentfill}{rgb}{0.000000,0.000000,0.000000}%
\pgfsetfillcolor{currentfill}%
\pgfsetlinewidth{0.803000pt}%
\definecolor{currentstroke}{rgb}{0.000000,0.000000,0.000000}%
\pgfsetstrokecolor{currentstroke}%
\pgfsetdash{}{0pt}%
\pgfsys@defobject{currentmarker}{\pgfqpoint{0.000000in}{-0.048611in}}{\pgfqpoint{0.000000in}{0.000000in}}{%
\pgfpathmoveto{\pgfqpoint{0.000000in}{0.000000in}}%
\pgfpathlineto{\pgfqpoint{0.000000in}{-0.048611in}}%
\pgfusepath{stroke,fill}%
}%
\begin{pgfscope}%
\pgfsys@transformshift{3.711629in}{0.402556in}%
\pgfsys@useobject{currentmarker}{}%
\end{pgfscope}%
\end{pgfscope}%
\begin{pgfscope}%
\pgftext[x=3.711629in,y=0.305334in,,top]{\rmfamily\fontsize{12.000000}{14.400000}\selectfont \(\displaystyle 1.50\)}%
\end{pgfscope}%
\begin{pgfscope}%
\pgfsetbuttcap%
\pgfsetroundjoin%
\definecolor{currentfill}{rgb}{0.000000,0.000000,0.000000}%
\pgfsetfillcolor{currentfill}%
\pgfsetlinewidth{0.803000pt}%
\definecolor{currentstroke}{rgb}{0.000000,0.000000,0.000000}%
\pgfsetstrokecolor{currentstroke}%
\pgfsetdash{}{0pt}%
\pgfsys@defobject{currentmarker}{\pgfqpoint{0.000000in}{-0.048611in}}{\pgfqpoint{0.000000in}{0.000000in}}{%
\pgfpathmoveto{\pgfqpoint{0.000000in}{0.000000in}}%
\pgfpathlineto{\pgfqpoint{0.000000in}{-0.048611in}}%
\pgfusepath{stroke,fill}%
}%
\begin{pgfscope}%
\pgfsys@transformshift{4.255462in}{0.402556in}%
\pgfsys@useobject{currentmarker}{}%
\end{pgfscope}%
\end{pgfscope}%
\begin{pgfscope}%
\pgftext[x=4.255462in,y=0.305334in,,top]{\rmfamily\fontsize{12.000000}{14.400000}\selectfont \(\displaystyle 1.75\)}%
\end{pgfscope}%
\begin{pgfscope}%
\pgfsetbuttcap%
\pgfsetroundjoin%
\definecolor{currentfill}{rgb}{0.000000,0.000000,0.000000}%
\pgfsetfillcolor{currentfill}%
\pgfsetlinewidth{0.803000pt}%
\definecolor{currentstroke}{rgb}{0.000000,0.000000,0.000000}%
\pgfsetstrokecolor{currentstroke}%
\pgfsetdash{}{0pt}%
\pgfsys@defobject{currentmarker}{\pgfqpoint{0.000000in}{-0.048611in}}{\pgfqpoint{0.000000in}{0.000000in}}{%
\pgfpathmoveto{\pgfqpoint{0.000000in}{0.000000in}}%
\pgfpathlineto{\pgfqpoint{0.000000in}{-0.048611in}}%
\pgfusepath{stroke,fill}%
}%
\begin{pgfscope}%
\pgfsys@transformshift{4.799294in}{0.402556in}%
\pgfsys@useobject{currentmarker}{}%
\end{pgfscope}%
\end{pgfscope}%
\begin{pgfscope}%
\pgftext[x=4.799294in,y=0.305334in,,top]{\rmfamily\fontsize{12.000000}{14.400000}\selectfont \(\displaystyle 2.00\)}%
\end{pgfscope}%
\begin{pgfscope}%
\pgfsetbuttcap%
\pgfsetroundjoin%
\definecolor{currentfill}{rgb}{0.000000,0.000000,0.000000}%
\pgfsetfillcolor{currentfill}%
\pgfsetlinewidth{0.803000pt}%
\definecolor{currentstroke}{rgb}{0.000000,0.000000,0.000000}%
\pgfsetstrokecolor{currentstroke}%
\pgfsetdash{}{0pt}%
\pgfsys@defobject{currentmarker}{\pgfqpoint{-0.048611in}{0.000000in}}{\pgfqpoint{0.000000in}{0.000000in}}{%
\pgfpathmoveto{\pgfqpoint{0.000000in}{0.000000in}}%
\pgfpathlineto{\pgfqpoint{-0.048611in}{0.000000in}}%
\pgfusepath{stroke,fill}%
}%
\begin{pgfscope}%
\pgfsys@transformshift{0.448634in}{0.402556in}%
\pgfsys@useobject{currentmarker}{}%
\end{pgfscope}%
\end{pgfscope}%
\begin{pgfscope}%
\pgftext[x=0.149245in,y=0.345015in,left,base]{\rmfamily\fontsize{12.000000}{14.400000}\selectfont \(\displaystyle 0.0\)}%
\end{pgfscope}%
\begin{pgfscope}%
\pgfsetbuttcap%
\pgfsetroundjoin%
\definecolor{currentfill}{rgb}{0.000000,0.000000,0.000000}%
\pgfsetfillcolor{currentfill}%
\pgfsetlinewidth{0.803000pt}%
\definecolor{currentstroke}{rgb}{0.000000,0.000000,0.000000}%
\pgfsetstrokecolor{currentstroke}%
\pgfsetdash{}{0pt}%
\pgfsys@defobject{currentmarker}{\pgfqpoint{-0.048611in}{0.000000in}}{\pgfqpoint{0.000000in}{0.000000in}}{%
\pgfpathmoveto{\pgfqpoint{0.000000in}{0.000000in}}%
\pgfpathlineto{\pgfqpoint{-0.048611in}{0.000000in}}%
\pgfusepath{stroke,fill}%
}%
\begin{pgfscope}%
\pgfsys@transformshift{0.448634in}{0.900397in}%
\pgfsys@useobject{currentmarker}{}%
\end{pgfscope}%
\end{pgfscope}%
\begin{pgfscope}%
\pgftext[x=0.149245in,y=0.842855in,left,base]{\rmfamily\fontsize{12.000000}{14.400000}\selectfont \(\displaystyle 0.2\)}%
\end{pgfscope}%
\begin{pgfscope}%
\pgfsetbuttcap%
\pgfsetroundjoin%
\definecolor{currentfill}{rgb}{0.000000,0.000000,0.000000}%
\pgfsetfillcolor{currentfill}%
\pgfsetlinewidth{0.803000pt}%
\definecolor{currentstroke}{rgb}{0.000000,0.000000,0.000000}%
\pgfsetstrokecolor{currentstroke}%
\pgfsetdash{}{0pt}%
\pgfsys@defobject{currentmarker}{\pgfqpoint{-0.048611in}{0.000000in}}{\pgfqpoint{0.000000in}{0.000000in}}{%
\pgfpathmoveto{\pgfqpoint{0.000000in}{0.000000in}}%
\pgfpathlineto{\pgfqpoint{-0.048611in}{0.000000in}}%
\pgfusepath{stroke,fill}%
}%
\begin{pgfscope}%
\pgfsys@transformshift{0.448634in}{1.398238in}%
\pgfsys@useobject{currentmarker}{}%
\end{pgfscope}%
\end{pgfscope}%
\begin{pgfscope}%
\pgftext[x=0.149245in,y=1.340696in,left,base]{\rmfamily\fontsize{12.000000}{14.400000}\selectfont \(\displaystyle 0.4\)}%
\end{pgfscope}%
\begin{pgfscope}%
\pgfsetbuttcap%
\pgfsetroundjoin%
\definecolor{currentfill}{rgb}{0.000000,0.000000,0.000000}%
\pgfsetfillcolor{currentfill}%
\pgfsetlinewidth{0.803000pt}%
\definecolor{currentstroke}{rgb}{0.000000,0.000000,0.000000}%
\pgfsetstrokecolor{currentstroke}%
\pgfsetdash{}{0pt}%
\pgfsys@defobject{currentmarker}{\pgfqpoint{-0.048611in}{0.000000in}}{\pgfqpoint{0.000000in}{0.000000in}}{%
\pgfpathmoveto{\pgfqpoint{0.000000in}{0.000000in}}%
\pgfpathlineto{\pgfqpoint{-0.048611in}{0.000000in}}%
\pgfusepath{stroke,fill}%
}%
\begin{pgfscope}%
\pgfsys@transformshift{0.448634in}{1.896079in}%
\pgfsys@useobject{currentmarker}{}%
\end{pgfscope}%
\end{pgfscope}%
\begin{pgfscope}%
\pgftext[x=0.149245in,y=1.838537in,left,base]{\rmfamily\fontsize{12.000000}{14.400000}\selectfont \(\displaystyle 0.6\)}%
\end{pgfscope}%
\begin{pgfscope}%
\pgfsetbuttcap%
\pgfsetroundjoin%
\definecolor{currentfill}{rgb}{0.000000,0.000000,0.000000}%
\pgfsetfillcolor{currentfill}%
\pgfsetlinewidth{0.803000pt}%
\definecolor{currentstroke}{rgb}{0.000000,0.000000,0.000000}%
\pgfsetstrokecolor{currentstroke}%
\pgfsetdash{}{0pt}%
\pgfsys@defobject{currentmarker}{\pgfqpoint{-0.048611in}{0.000000in}}{\pgfqpoint{0.000000in}{0.000000in}}{%
\pgfpathmoveto{\pgfqpoint{0.000000in}{0.000000in}}%
\pgfpathlineto{\pgfqpoint{-0.048611in}{0.000000in}}%
\pgfusepath{stroke,fill}%
}%
\begin{pgfscope}%
\pgfsys@transformshift{0.448634in}{2.393919in}%
\pgfsys@useobject{currentmarker}{}%
\end{pgfscope}%
\end{pgfscope}%
\begin{pgfscope}%
\pgftext[x=0.149245in,y=2.336378in,left,base]{\rmfamily\fontsize{12.000000}{14.400000}\selectfont \(\displaystyle 0.8\)}%
\end{pgfscope}%
\begin{pgfscope}%
\pgfsetbuttcap%
\pgfsetroundjoin%
\definecolor{currentfill}{rgb}{0.000000,0.000000,0.000000}%
\pgfsetfillcolor{currentfill}%
\pgfsetlinewidth{0.803000pt}%
\definecolor{currentstroke}{rgb}{0.000000,0.000000,0.000000}%
\pgfsetstrokecolor{currentstroke}%
\pgfsetdash{}{0pt}%
\pgfsys@defobject{currentmarker}{\pgfqpoint{-0.048611in}{0.000000in}}{\pgfqpoint{0.000000in}{0.000000in}}{%
\pgfpathmoveto{\pgfqpoint{0.000000in}{0.000000in}}%
\pgfpathlineto{\pgfqpoint{-0.048611in}{0.000000in}}%
\pgfusepath{stroke,fill}%
}%
\begin{pgfscope}%
\pgfsys@transformshift{0.448634in}{2.891760in}%
\pgfsys@useobject{currentmarker}{}%
\end{pgfscope}%
\end{pgfscope}%
\begin{pgfscope}%
\pgftext[x=0.149245in,y=2.834219in,left,base]{\rmfamily\fontsize{12.000000}{14.400000}\selectfont \(\displaystyle 1.0\)}%
\end{pgfscope}%
\begin{pgfscope}%
\pgfpathrectangle{\pgfqpoint{0.448634in}{0.402556in}}{\pgfqpoint{4.350661in}{2.489204in}} %
\pgfusepath{clip}%
\pgfsetrectcap%
\pgfsetroundjoin%
\pgfsetlinewidth{1.003750pt}%
\definecolor{currentstroke}{rgb}{1.000000,0.388235,0.278431}%
\pgfsetstrokecolor{currentstroke}%
\pgfsetdash{}{0pt}%
\pgfpathmoveto{\pgfqpoint{1.127319in}{2.572074in}}%
\pgfpathlineto{\pgfqpoint{1.159575in}{2.592758in}}%
\pgfpathlineto{\pgfqpoint{1.192763in}{2.611414in}}%
\pgfpathlineto{\pgfqpoint{1.228726in}{2.629126in}}%
\pgfpathlineto{\pgfqpoint{1.267413in}{2.645758in}}%
\pgfpathlineto{\pgfqpoint{1.310846in}{2.661945in}}%
\pgfpathlineto{\pgfqpoint{1.356920in}{2.676740in}}%
\pgfpathlineto{\pgfqpoint{1.407680in}{2.690702in}}%
\pgfpathlineto{\pgfqpoint{1.463094in}{2.703640in}}%
\pgfpathlineto{\pgfqpoint{1.525273in}{2.715813in}}%
\pgfpathlineto{\pgfqpoint{1.594199in}{2.726937in}}%
\pgfpathlineto{\pgfqpoint{1.669843in}{2.736808in}}%
\pgfpathlineto{\pgfqpoint{1.752172in}{2.745271in}}%
\pgfpathlineto{\pgfqpoint{1.843325in}{2.752344in}}%
\pgfpathlineto{\pgfqpoint{1.941103in}{2.757656in}}%
\pgfpathlineto{\pgfqpoint{2.043301in}{2.760987in}}%
\pgfpathlineto{\pgfqpoint{2.147710in}{2.762199in}}%
\pgfpathlineto{\pgfqpoint{2.249945in}{2.761215in}}%
\pgfpathlineto{\pgfqpoint{2.345620in}{2.758145in}}%
\pgfpathlineto{\pgfqpoint{2.432525in}{2.753210in}}%
\pgfpathlineto{\pgfqpoint{2.508451in}{2.746766in}}%
\pgfpathlineto{\pgfqpoint{2.573368in}{2.739156in}}%
\pgfpathlineto{\pgfqpoint{2.629410in}{2.730451in}}%
\pgfpathlineto{\pgfqpoint{2.676543in}{2.720985in}}%
\pgfpathlineto{\pgfqpoint{2.716874in}{2.710666in}}%
\pgfpathlineto{\pgfqpoint{2.750366in}{2.699848in}}%
\pgfpathlineto{\pgfqpoint{2.779059in}{2.688192in}}%
\pgfpathlineto{\pgfqpoint{2.802882in}{2.676004in}}%
\pgfpathlineto{\pgfqpoint{2.821842in}{2.663819in}}%
\pgfpathlineto{\pgfqpoint{2.837815in}{2.650886in}}%
\pgfpathlineto{\pgfqpoint{2.850736in}{2.637564in}}%
\pgfpathlineto{\pgfqpoint{2.860694in}{2.624398in}}%
\pgfpathlineto{\pgfqpoint{2.869084in}{2.609873in}}%
\pgfpathlineto{\pgfqpoint{2.875698in}{2.594192in}}%
\pgfpathlineto{\pgfqpoint{2.881035in}{2.575255in}}%
\pgfpathlineto{\pgfqpoint{2.884200in}{2.555685in}}%
\pgfpathlineto{\pgfqpoint{2.885619in}{2.533351in}}%
\pgfpathlineto{\pgfqpoint{2.885038in}{2.505987in}}%
\pgfpathlineto{\pgfqpoint{2.882112in}{2.473807in}}%
\pgfpathlineto{\pgfqpoint{2.875657in}{2.429620in}}%
\pgfpathlineto{\pgfqpoint{2.863489in}{2.363873in}}%
\pgfpathlineto{\pgfqpoint{2.821102in}{2.142619in}}%
\pgfpathlineto{\pgfqpoint{2.804859in}{2.042271in}}%
\pgfpathlineto{\pgfqpoint{2.790421in}{1.939040in}}%
\pgfpathlineto{\pgfqpoint{2.777207in}{1.828054in}}%
\pgfpathlineto{\pgfqpoint{2.765338in}{1.709349in}}%
\pgfpathlineto{\pgfqpoint{2.754471in}{1.578010in}}%
\pgfpathlineto{\pgfqpoint{2.744640in}{1.431580in}}%
\pgfpathlineto{\pgfqpoint{2.735914in}{1.267598in}}%
\pgfpathlineto{\pgfqpoint{2.728277in}{1.081114in}}%
\pgfpathlineto{\pgfqpoint{2.721437in}{0.857223in}}%
\pgfpathlineto{\pgfqpoint{2.711961in}{0.541290in}}%
\pgfpathlineto{\pgfqpoint{2.708250in}{0.491694in}}%
\pgfpathlineto{\pgfqpoint{2.703951in}{0.462246in}}%
\pgfpathlineto{\pgfqpoint{2.699504in}{0.445599in}}%
\pgfpathlineto{\pgfqpoint{2.694517in}{0.434563in}}%
\pgfpathlineto{\pgfqpoint{2.688942in}{0.426947in}}%
\pgfpathlineto{\pgfqpoint{2.681980in}{0.421009in}}%
\pgfpathlineto{\pgfqpoint{2.672064in}{0.415948in}}%
\pgfpathlineto{\pgfqpoint{2.659429in}{0.412247in}}%
\pgfpathlineto{\pgfqpoint{2.640044in}{0.409163in}}%
\pgfpathlineto{\pgfqpoint{2.607490in}{0.406692in}}%
\pgfpathlineto{\pgfqpoint{2.548779in}{0.404894in}}%
\pgfpathlineto{\pgfqpoint{2.422615in}{0.403701in}}%
\pgfpathlineto{\pgfqpoint{2.026705in}{0.403016in}}%
\pgfpathlineto{\pgfqpoint{0.623617in}{0.403253in}}%
\pgfpathlineto{\pgfqpoint{0.477880in}{0.404742in}}%
\pgfpathlineto{\pgfqpoint{0.458368in}{0.406382in}}%
\pgfpathlineto{\pgfqpoint{0.452304in}{0.408937in}}%
\pgfpathlineto{\pgfqpoint{0.450213in}{0.413215in}}%
\pgfpathlineto{\pgfqpoint{0.449165in}{0.423080in}}%
\pgfpathlineto{\pgfqpoint{0.448735in}{0.465392in}}%
\pgfpathlineto{\pgfqpoint{0.448637in}{0.983146in}}%
\pgfpathlineto{\pgfqpoint{0.448652in}{2.889876in}}%
\pgfpathlineto{\pgfqpoint{0.448652in}{2.889876in}}%
\pgfusepath{stroke}%
\end{pgfscope}%
\begin{pgfscope}%
\pgfpathrectangle{\pgfqpoint{0.448634in}{0.402556in}}{\pgfqpoint{4.350661in}{2.489204in}} %
\pgfusepath{clip}%
\pgfsetrectcap%
\pgfsetroundjoin%
\pgfsetlinewidth{1.003750pt}%
\definecolor{currentstroke}{rgb}{1.000000,0.388235,0.278431}%
\pgfsetstrokecolor{currentstroke}%
\pgfsetdash{}{0pt}%
\pgfpathmoveto{\pgfqpoint{0.448634in}{2.896245in}}%
\pgfpathlineto{\pgfqpoint{0.448593in}{0.407043in}}%
\pgfpathlineto{\pgfqpoint{0.448593in}{0.407043in}}%
\pgfusepath{stroke}%
\end{pgfscope}%
\begin{pgfscope}%
\pgfpathrectangle{\pgfqpoint{0.448634in}{0.402556in}}{\pgfqpoint{4.350661in}{2.489204in}} %
\pgfusepath{clip}%
\pgfsetrectcap%
\pgfsetroundjoin%
\pgfsetlinewidth{1.003750pt}%
\definecolor{currentstroke}{rgb}{1.000000,0.388235,0.278431}%
\pgfsetstrokecolor{currentstroke}%
\pgfsetdash{}{0pt}%
\pgfpathmoveto{\pgfqpoint{0.576853in}{1.760817in}}%
\pgfpathlineto{\pgfqpoint{0.569394in}{1.840010in}}%
\pgfpathlineto{\pgfqpoint{0.563209in}{1.929338in}}%
\pgfpathlineto{\pgfqpoint{0.558592in}{2.028764in}}%
\pgfpathlineto{\pgfqpoint{0.555985in}{2.133265in}}%
\pgfpathlineto{\pgfqpoint{0.555566in}{2.237808in}}%
\pgfpathlineto{\pgfqpoint{0.557371in}{2.337352in}}%
\pgfpathlineto{\pgfqpoint{0.561096in}{2.424366in}}%
\pgfpathlineto{\pgfqpoint{0.566403in}{2.498791in}}%
\pgfpathlineto{\pgfqpoint{0.572909in}{2.560570in}}%
\pgfpathlineto{\pgfqpoint{0.580458in}{2.612119in}}%
\pgfpathlineto{\pgfqpoint{0.589086in}{2.655816in}}%
\pgfpathlineto{\pgfqpoint{0.598406in}{2.691589in}}%
\pgfpathlineto{\pgfqpoint{0.608613in}{2.721757in}}%
\pgfpathlineto{\pgfqpoint{0.619241in}{2.746278in}}%
\pgfpathlineto{\pgfqpoint{0.630817in}{2.767339in}}%
\pgfpathlineto{\pgfqpoint{0.642975in}{2.784884in}}%
\pgfpathlineto{\pgfqpoint{0.656813in}{2.800712in}}%
\pgfpathlineto{\pgfqpoint{0.672197in}{2.814549in}}%
\pgfpathlineto{\pgfqpoint{0.688853in}{2.826301in}}%
\pgfpathlineto{\pgfqpoint{0.706461in}{2.836076in}}%
\pgfpathlineto{\pgfqpoint{0.726804in}{2.844875in}}%
\pgfpathlineto{\pgfqpoint{0.751866in}{2.853203in}}%
\pgfpathlineto{\pgfqpoint{0.781631in}{2.860547in}}%
\pgfpathlineto{\pgfqpoint{0.818168in}{2.867054in}}%
\pgfpathlineto{\pgfqpoint{0.863581in}{2.872685in}}%
\pgfpathlineto{\pgfqpoint{0.922161in}{2.877518in}}%
\pgfpathlineto{\pgfqpoint{1.000391in}{2.881567in}}%
\pgfpathlineto{\pgfqpoint{1.111294in}{2.884881in}}%
\pgfpathlineto{\pgfqpoint{1.274428in}{2.887367in}}%
\pgfpathlineto{\pgfqpoint{1.552865in}{2.889263in}}%
\pgfpathlineto{\pgfqpoint{2.107573in}{2.890457in}}%
\pgfpathlineto{\pgfqpoint{3.343161in}{2.890573in}}%
\pgfpathlineto{\pgfqpoint{4.043615in}{2.888941in}}%
\pgfpathlineto{\pgfqpoint{4.289417in}{2.886404in}}%
\pgfpathlineto{\pgfqpoint{4.413375in}{2.883093in}}%
\pgfpathlineto{\pgfqpoint{4.489424in}{2.878997in}}%
\pgfpathlineto{\pgfqpoint{4.541451in}{2.874081in}}%
\pgfpathlineto{\pgfqpoint{4.578100in}{2.868470in}}%
\pgfpathlineto{\pgfqpoint{4.605818in}{2.862092in}}%
\pgfpathlineto{\pgfqpoint{4.626725in}{2.855245in}}%
\pgfpathlineto{\pgfqpoint{4.644925in}{2.847018in}}%
\pgfpathlineto{\pgfqpoint{4.660241in}{2.837590in}}%
\pgfpathlineto{\pgfqpoint{4.672623in}{2.827468in}}%
\pgfpathlineto{\pgfqpoint{4.683751in}{2.815592in}}%
\pgfpathlineto{\pgfqpoint{4.693406in}{2.802135in}}%
\pgfpathlineto{\pgfqpoint{4.702740in}{2.785343in}}%
\pgfpathlineto{\pgfqpoint{4.711277in}{2.765194in}}%
\pgfpathlineto{\pgfqpoint{4.719482in}{2.739484in}}%
\pgfpathlineto{\pgfqpoint{4.726293in}{2.710657in}}%
\pgfpathlineto{\pgfqpoint{4.733259in}{2.671643in}}%
\pgfpathlineto{\pgfqpoint{4.739604in}{2.622396in}}%
\pgfpathlineto{\pgfqpoint{4.745236in}{2.560504in}}%
\pgfpathlineto{\pgfqpoint{4.750164in}{2.481052in}}%
\pgfpathlineto{\pgfqpoint{4.754367in}{2.376618in}}%
\pgfpathlineto{\pgfqpoint{4.757443in}{2.242249in}}%
\pgfpathlineto{\pgfqpoint{4.758977in}{2.075483in}}%
\pgfpathlineto{\pgfqpoint{4.758447in}{1.888795in}}%
\pgfpathlineto{\pgfqpoint{4.755756in}{1.707111in}}%
\pgfpathlineto{\pgfqpoint{4.750925in}{1.532957in}}%
\pgfpathlineto{\pgfqpoint{4.744785in}{1.398726in}}%
\pgfpathlineto{\pgfqpoint{4.737575in}{1.289516in}}%
\pgfpathlineto{\pgfqpoint{4.728714in}{1.190470in}}%
\pgfpathlineto{\pgfqpoint{4.719652in}{1.116521in}}%
\pgfpathlineto{\pgfqpoint{4.710036in}{1.055276in}}%
\pgfpathlineto{\pgfqpoint{4.699503in}{1.001861in}}%
\pgfpathlineto{\pgfqpoint{4.689040in}{0.958690in}}%
\pgfpathlineto{\pgfqpoint{4.677219in}{0.918600in}}%
\pgfpathlineto{\pgfqpoint{4.664034in}{0.881749in}}%
\pgfpathlineto{\pgfqpoint{4.650584in}{0.850492in}}%
\pgfpathlineto{\pgfqpoint{4.636303in}{0.822570in}}%
\pgfpathlineto{\pgfqpoint{4.620207in}{0.795974in}}%
\pgfpathlineto{\pgfqpoint{4.603640in}{0.772901in}}%
\pgfpathlineto{\pgfqpoint{4.585488in}{0.751446in}}%
\pgfpathlineto{\pgfqpoint{4.565874in}{0.731749in}}%
\pgfpathlineto{\pgfqpoint{4.544964in}{0.713879in}}%
\pgfpathlineto{\pgfqpoint{4.522958in}{0.697824in}}%
\pgfpathlineto{\pgfqpoint{4.496157in}{0.681290in}}%
\pgfpathlineto{\pgfqpoint{4.470397in}{0.667953in}}%
\pgfpathlineto{\pgfqpoint{4.439961in}{0.654509in}}%
\pgfpathlineto{\pgfqpoint{4.406841in}{0.642281in}}%
\pgfpathlineto{\pgfqpoint{4.369009in}{0.630748in}}%
\pgfpathlineto{\pgfqpoint{4.326489in}{0.620226in}}%
\pgfpathlineto{\pgfqpoint{4.279327in}{0.610949in}}%
\pgfpathlineto{\pgfqpoint{4.227576in}{0.603085in}}%
\pgfpathlineto{\pgfqpoint{4.173450in}{0.597063in}}%
\pgfpathlineto{\pgfqpoint{4.110511in}{0.592203in}}%
\pgfpathlineto{\pgfqpoint{4.047471in}{0.589537in}}%
\pgfpathlineto{\pgfqpoint{3.977867in}{0.588624in}}%
\pgfpathlineto{\pgfqpoint{3.906093in}{0.589934in}}%
\pgfpathlineto{\pgfqpoint{3.834377in}{0.593496in}}%
\pgfpathlineto{\pgfqpoint{3.767120in}{0.599067in}}%
\pgfpathlineto{\pgfqpoint{3.704364in}{0.606392in}}%
\pgfpathlineto{\pgfqpoint{3.678516in}{0.610510in}}%
\pgfpathlineto{\pgfqpoint{3.620438in}{0.620500in}}%
\pgfpathlineto{\pgfqpoint{3.586319in}{0.628207in}}%
\pgfpathlineto{\pgfqpoint{3.495240in}{0.652428in}}%
\pgfpathlineto{\pgfqpoint{3.451528in}{0.667583in}}%
\pgfpathlineto{\pgfqpoint{3.408538in}{0.685220in}}%
\pgfpathlineto{\pgfqpoint{3.374594in}{0.702001in}}%
\pgfpathlineto{\pgfqpoint{3.345407in}{0.718682in}}%
\pgfpathlineto{\pgfqpoint{3.315236in}{0.738520in}}%
\pgfpathlineto{\pgfqpoint{3.288127in}{0.759290in}}%
\pgfpathlineto{\pgfqpoint{3.264004in}{0.780551in}}%
\pgfpathlineto{\pgfqpoint{3.241208in}{0.803648in}}%
\pgfpathlineto{\pgfqpoint{3.219894in}{0.828530in}}%
\pgfpathlineto{\pgfqpoint{3.200189in}{0.855091in}}%
\pgfpathlineto{\pgfqpoint{3.182177in}{0.883182in}}%
\pgfpathlineto{\pgfqpoint{3.165906in}{0.912633in}}%
\pgfpathlineto{\pgfqpoint{3.150351in}{0.945448in}}%
\pgfpathlineto{\pgfqpoint{3.136682in}{0.979345in}}%
\pgfpathlineto{\pgfqpoint{3.124073in}{1.016460in}}%
\pgfpathlineto{\pgfqpoint{3.112834in}{1.056769in}}%
\pgfpathlineto{\pgfqpoint{3.103046in}{1.100146in}}%
\pgfpathlineto{\pgfqpoint{3.095343in}{1.144071in}}%
\pgfpathlineto{\pgfqpoint{3.089208in}{1.190837in}}%
\pgfpathlineto{\pgfqpoint{3.084595in}{1.242838in}}%
\pgfpathlineto{\pgfqpoint{3.082137in}{1.295031in}}%
\pgfpathlineto{\pgfqpoint{3.081687in}{1.349787in}}%
\pgfpathlineto{\pgfqpoint{3.083451in}{1.406998in}}%
\pgfpathlineto{\pgfqpoint{3.087181in}{1.461589in}}%
\pgfpathlineto{\pgfqpoint{3.093485in}{1.520888in}}%
\pgfpathlineto{\pgfqpoint{3.101823in}{1.577334in}}%
\pgfpathlineto{\pgfqpoint{3.111930in}{1.630856in}}%
\pgfpathlineto{\pgfqpoint{3.124690in}{1.686208in}}%
\pgfpathlineto{\pgfqpoint{3.139178in}{1.738395in}}%
\pgfpathlineto{\pgfqpoint{3.155145in}{1.787366in}}%
\pgfpathlineto{\pgfqpoint{3.172353in}{1.833085in}}%
\pgfpathlineto{\pgfqpoint{3.191618in}{1.877716in}}%
\pgfpathlineto{\pgfqpoint{3.214026in}{1.923261in}}%
\pgfpathlineto{\pgfqpoint{3.236214in}{1.963157in}}%
\pgfpathlineto{\pgfqpoint{3.260178in}{2.001684in}}%
\pgfpathlineto{\pgfqpoint{3.285814in}{2.038776in}}%
\pgfpathlineto{\pgfqpoint{3.314415in}{2.076285in}}%
\pgfpathlineto{\pgfqpoint{3.348944in}{2.117711in}}%
\pgfpathlineto{\pgfqpoint{3.417133in}{2.198022in}}%
\pgfpathlineto{\pgfqpoint{3.426053in}{2.212128in}}%
\pgfpathlineto{\pgfqpoint{3.430798in}{2.223297in}}%
\pgfpathlineto{\pgfqpoint{3.432034in}{2.230603in}}%
\pgfpathlineto{\pgfqpoint{3.430773in}{2.237856in}}%
\pgfpathlineto{\pgfqpoint{3.426621in}{2.243526in}}%
\pgfpathlineto{\pgfqpoint{3.420908in}{2.247084in}}%
\pgfpathlineto{\pgfqpoint{3.412501in}{2.249583in}}%
\pgfpathlineto{\pgfqpoint{3.399499in}{2.250689in}}%
\pgfpathlineto{\pgfqpoint{3.384305in}{2.249671in}}%
\pgfpathlineto{\pgfqpoint{3.364985in}{2.246098in}}%
\pgfpathlineto{\pgfqpoint{3.341804in}{2.239342in}}%
\pgfpathlineto{\pgfqpoint{3.317109in}{2.229682in}}%
\pgfpathlineto{\pgfqpoint{3.291104in}{2.216986in}}%
\pgfpathlineto{\pgfqpoint{3.265928in}{2.202261in}}%
\pgfpathlineto{\pgfqpoint{3.239805in}{2.184361in}}%
\pgfpathlineto{\pgfqpoint{3.214775in}{2.164519in}}%
\pgfpathlineto{\pgfqpoint{3.190900in}{2.142893in}}%
\pgfpathlineto{\pgfqpoint{3.166657in}{2.117912in}}%
\pgfpathlineto{\pgfqpoint{3.143835in}{2.091233in}}%
\pgfpathlineto{\pgfqpoint{3.121079in}{2.061107in}}%
\pgfpathlineto{\pgfqpoint{3.099952in}{2.029463in}}%
\pgfpathlineto{\pgfqpoint{3.079251in}{1.994406in}}%
\pgfpathlineto{\pgfqpoint{3.059218in}{1.955915in}}%
\pgfpathlineto{\pgfqpoint{3.040058in}{1.914015in}}%
\pgfpathlineto{\pgfqpoint{3.022809in}{1.871041in}}%
\pgfpathlineto{\pgfqpoint{3.005790in}{1.822536in}}%
\pgfpathlineto{\pgfqpoint{2.990067in}{1.770819in}}%
\pgfpathlineto{\pgfqpoint{2.975708in}{1.715979in}}%
\pgfpathlineto{\pgfqpoint{2.962284in}{1.655680in}}%
\pgfpathlineto{\pgfqpoint{2.950496in}{1.592386in}}%
\pgfpathlineto{\pgfqpoint{2.940383in}{1.526185in}}%
\pgfpathlineto{\pgfqpoint{2.931745in}{1.454681in}}%
\pgfpathlineto{\pgfqpoint{2.925082in}{1.380399in}}%
\pgfpathlineto{\pgfqpoint{2.920647in}{1.305899in}}%
\pgfpathlineto{\pgfqpoint{2.918444in}{1.231270in}}%
\pgfpathlineto{\pgfqpoint{2.918545in}{1.159087in}}%
\pgfpathlineto{\pgfqpoint{2.920787in}{1.091931in}}%
\pgfpathlineto{\pgfqpoint{2.925177in}{1.027412in}}%
\pgfpathlineto{\pgfqpoint{2.931192in}{0.970580in}}%
\pgfpathlineto{\pgfqpoint{2.938760in}{0.919034in}}%
\pgfpathlineto{\pgfqpoint{2.947651in}{0.872852in}}%
\pgfpathlineto{\pgfqpoint{2.958213in}{0.829714in}}%
\pgfpathlineto{\pgfqpoint{2.969670in}{0.792114in}}%
\pgfpathlineto{\pgfqpoint{2.982463in}{0.757773in}}%
\pgfpathlineto{\pgfqpoint{2.996425in}{0.726812in}}%
\pgfpathlineto{\pgfqpoint{3.011299in}{0.699300in}}%
\pgfpathlineto{\pgfqpoint{3.026739in}{0.675225in}}%
\pgfpathlineto{\pgfqpoint{3.043828in}{0.652656in}}%
\pgfpathlineto{\pgfqpoint{3.062495in}{0.631788in}}%
\pgfpathlineto{\pgfqpoint{3.082602in}{0.612753in}}%
\pgfpathlineto{\pgfqpoint{3.103961in}{0.595592in}}%
\pgfpathlineto{\pgfqpoint{3.128268in}{0.579069in}}%
\pgfpathlineto{\pgfqpoint{3.153537in}{0.564554in}}%
\pgfpathlineto{\pgfqpoint{3.181571in}{0.550952in}}%
\pgfpathlineto{\pgfqpoint{3.214371in}{0.537647in}}%
\pgfpathlineto{\pgfqpoint{3.249846in}{0.525712in}}%
\pgfpathlineto{\pgfqpoint{3.290011in}{0.514571in}}%
\pgfpathlineto{\pgfqpoint{3.334820in}{0.504423in}}%
\pgfpathlineto{\pgfqpoint{3.386372in}{0.494999in}}%
\pgfpathlineto{\pgfqpoint{3.446798in}{0.486257in}}%
\pgfpathlineto{\pgfqpoint{3.518243in}{0.478282in}}%
\pgfpathlineto{\pgfqpoint{3.600685in}{0.471409in}}%
\pgfpathlineto{\pgfqpoint{3.696268in}{0.465713in}}%
\pgfpathlineto{\pgfqpoint{3.807144in}{0.461369in}}%
\pgfpathlineto{\pgfqpoint{3.933291in}{0.458719in}}%
\pgfpathlineto{\pgfqpoint{4.063808in}{0.458211in}}%
\pgfpathlineto{\pgfqpoint{4.187792in}{0.459914in}}%
\pgfpathlineto{\pgfqpoint{4.294335in}{0.463521in}}%
\pgfpathlineto{\pgfqpoint{4.381234in}{0.468574in}}%
\pgfpathlineto{\pgfqpoint{4.450636in}{0.474701in}}%
\pgfpathlineto{\pgfqpoint{4.506850in}{0.481799in}}%
\pgfpathlineto{\pgfqpoint{4.552009in}{0.489658in}}%
\pgfpathlineto{\pgfqpoint{4.588239in}{0.498115in}}%
\pgfpathlineto{\pgfqpoint{4.617656in}{0.507110in}}%
\pgfpathlineto{\pgfqpoint{4.642328in}{0.516843in}}%
\pgfpathlineto{\pgfqpoint{4.664194in}{0.527940in}}%
\pgfpathlineto{\pgfqpoint{4.681238in}{0.538945in}}%
\pgfpathlineto{\pgfqpoint{4.697164in}{0.551953in}}%
\pgfpathlineto{\pgfqpoint{4.710076in}{0.565289in}}%
\pgfpathlineto{\pgfqpoint{4.721578in}{0.580218in}}%
\pgfpathlineto{\pgfqpoint{4.731557in}{0.596521in}}%
\pgfpathlineto{\pgfqpoint{4.741000in}{0.616134in}}%
\pgfpathlineto{\pgfqpoint{4.749521in}{0.639027in}}%
\pgfpathlineto{\pgfqpoint{4.757522in}{0.667450in}}%
\pgfpathlineto{\pgfqpoint{4.764572in}{0.701345in}}%
\pgfpathlineto{\pgfqpoint{4.770840in}{0.743043in}}%
\pgfpathlineto{\pgfqpoint{4.776327in}{0.794934in}}%
\pgfpathlineto{\pgfqpoint{4.781278in}{0.864398in}}%
\pgfpathlineto{\pgfqpoint{4.785468in}{0.956371in}}%
\pgfpathlineto{\pgfqpoint{4.789000in}{1.085745in}}%
\pgfpathlineto{\pgfqpoint{4.791852in}{1.277385in}}%
\pgfpathlineto{\pgfqpoint{4.793959in}{1.581057in}}%
\pgfpathlineto{\pgfqpoint{4.794962in}{2.071429in}}%
\pgfpathlineto{\pgfqpoint{4.793967in}{2.559311in}}%
\pgfpathlineto{\pgfqpoint{4.791733in}{2.745981in}}%
\pgfpathlineto{\pgfqpoint{4.788955in}{2.818091in}}%
\pgfpathlineto{\pgfqpoint{4.785731in}{2.850227in}}%
\pgfpathlineto{\pgfqpoint{4.781879in}{2.867057in}}%
\pgfpathlineto{\pgfqpoint{4.777744in}{2.875780in}}%
\pgfpathlineto{\pgfqpoint{4.773097in}{2.880982in}}%
\pgfpathlineto{\pgfqpoint{4.767363in}{2.884504in}}%
\pgfpathlineto{\pgfqpoint{4.756853in}{2.887622in}}%
\pgfpathlineto{\pgfqpoint{4.739548in}{2.889639in}}%
\pgfpathlineto{\pgfqpoint{4.704762in}{2.890882in}}%
\pgfpathlineto{\pgfqpoint{4.602524in}{2.891538in}}%
\pgfpathlineto{\pgfqpoint{3.952100in}{2.891742in}}%
\pgfpathlineto{\pgfqpoint{0.617321in}{2.890753in}}%
\pgfpathlineto{\pgfqpoint{0.549910in}{2.888858in}}%
\pgfpathlineto{\pgfqpoint{0.521735in}{2.886179in}}%
\pgfpathlineto{\pgfqpoint{0.504666in}{2.882389in}}%
\pgfpathlineto{\pgfqpoint{0.494501in}{2.878011in}}%
\pgfpathlineto{\pgfqpoint{0.487180in}{2.872667in}}%
\pgfpathlineto{\pgfqpoint{0.481152in}{2.865519in}}%
\pgfpathlineto{\pgfqpoint{0.475664in}{2.854804in}}%
\pgfpathlineto{\pgfqpoint{0.471318in}{2.840737in}}%
\pgfpathlineto{\pgfqpoint{0.467301in}{2.818823in}}%
\pgfpathlineto{\pgfqpoint{0.463927in}{2.786700in}}%
\pgfpathlineto{\pgfqpoint{0.460918in}{2.734544in}}%
\pgfpathlineto{\pgfqpoint{0.458363in}{2.647473in}}%
\pgfpathlineto{\pgfqpoint{0.456575in}{2.523031in}}%
\pgfpathlineto{\pgfqpoint{0.456575in}{2.523031in}}%
\pgfusepath{stroke}%
\end{pgfscope}%
\begin{pgfscope}%
\pgfpathrectangle{\pgfqpoint{0.448634in}{0.402556in}}{\pgfqpoint{4.350661in}{2.489204in}} %
\pgfusepath{clip}%
\pgfsetrectcap%
\pgfsetroundjoin%
\pgfsetlinewidth{1.003750pt}%
\definecolor{currentstroke}{rgb}{1.000000,0.388235,0.278431}%
\pgfsetstrokecolor{currentstroke}%
\pgfsetdash{}{0pt}%
\pgfpathmoveto{\pgfqpoint{0.456424in}{1.370137in}}%
\pgfpathlineto{\pgfqpoint{0.459610in}{1.118755in}}%
\pgfpathlineto{\pgfqpoint{0.463695in}{0.962007in}}%
\pgfpathlineto{\pgfqpoint{0.468519in}{0.857610in}}%
\pgfpathlineto{\pgfqpoint{0.474082in}{0.783210in}}%
\pgfpathlineto{\pgfqpoint{0.480226in}{0.728906in}}%
\pgfpathlineto{\pgfqpoint{0.486970in}{0.687306in}}%
\pgfpathlineto{\pgfqpoint{0.494537in}{0.653558in}}%
\pgfpathlineto{\pgfqpoint{0.503107in}{0.625355in}}%
\pgfpathlineto{\pgfqpoint{0.512193in}{0.602749in}}%
\pgfpathlineto{\pgfqpoint{0.522200in}{0.583508in}}%
\pgfpathlineto{\pgfqpoint{0.534108in}{0.565743in}}%
\pgfpathlineto{\pgfqpoint{0.546263in}{0.551507in}}%
\pgfpathlineto{\pgfqpoint{0.559728in}{0.538907in}}%
\pgfpathlineto{\pgfqpoint{0.576129in}{0.526693in}}%
\pgfpathlineto{\pgfqpoint{0.595483in}{0.515351in}}%
\pgfpathlineto{\pgfqpoint{0.617681in}{0.505147in}}%
\pgfpathlineto{\pgfqpoint{0.642568in}{0.496153in}}%
\pgfpathlineto{\pgfqpoint{0.672126in}{0.487778in}}%
\pgfpathlineto{\pgfqpoint{0.708443in}{0.479824in}}%
\pgfpathlineto{\pgfqpoint{0.753649in}{0.472325in}}%
\pgfpathlineto{\pgfqpoint{0.807717in}{0.465660in}}%
\pgfpathlineto{\pgfqpoint{0.877116in}{0.459475in}}%
\pgfpathlineto{\pgfqpoint{0.961828in}{0.454230in}}%
\pgfpathlineto{\pgfqpoint{1.068351in}{0.449916in}}%
\pgfpathlineto{\pgfqpoint{1.201018in}{0.446839in}}%
\pgfpathlineto{\pgfqpoint{1.357637in}{0.445481in}}%
\pgfpathlineto{\pgfqpoint{1.525135in}{0.446232in}}%
\pgfpathlineto{\pgfqpoint{1.686088in}{0.449142in}}%
\pgfpathlineto{\pgfqpoint{1.823074in}{0.453747in}}%
\pgfpathlineto{\pgfqpoint{1.938245in}{0.459764in}}%
\pgfpathlineto{\pgfqpoint{2.031582in}{0.466759in}}%
\pgfpathlineto{\pgfqpoint{2.109580in}{0.474745in}}%
\pgfpathlineto{\pgfqpoint{2.174384in}{0.483535in}}%
\pgfpathlineto{\pgfqpoint{2.228139in}{0.492940in}}%
\pgfpathlineto{\pgfqpoint{2.275119in}{0.503356in}}%
\pgfpathlineto{\pgfqpoint{2.315282in}{0.514501in}}%
\pgfpathlineto{\pgfqpoint{2.350698in}{0.526659in}}%
\pgfpathlineto{\pgfqpoint{2.381320in}{0.539536in}}%
\pgfpathlineto{\pgfqpoint{2.407164in}{0.552659in}}%
\pgfpathlineto{\pgfqpoint{2.430226in}{0.566639in}}%
\pgfpathlineto{\pgfqpoint{2.452282in}{0.582602in}}%
\pgfpathlineto{\pgfqpoint{2.471391in}{0.599069in}}%
\pgfpathlineto{\pgfqpoint{2.489240in}{0.617293in}}%
\pgfpathlineto{\pgfqpoint{2.505678in}{0.637180in}}%
\pgfpathlineto{\pgfqpoint{2.520620in}{0.658557in}}%
\pgfpathlineto{\pgfqpoint{2.535213in}{0.683314in}}%
\pgfpathlineto{\pgfqpoint{2.549115in}{0.711484in}}%
\pgfpathlineto{\pgfqpoint{2.562091in}{0.743004in}}%
\pgfpathlineto{\pgfqpoint{2.574020in}{0.777751in}}%
\pgfpathlineto{\pgfqpoint{2.585502in}{0.817970in}}%
\pgfpathlineto{\pgfqpoint{2.596809in}{0.866038in}}%
\pgfpathlineto{\pgfqpoint{2.607562in}{0.921948in}}%
\pgfpathlineto{\pgfqpoint{2.617925in}{0.988098in}}%
\pgfpathlineto{\pgfqpoint{2.627958in}{1.066918in}}%
\pgfpathlineto{\pgfqpoint{2.637941in}{1.163320in}}%
\pgfpathlineto{\pgfqpoint{2.648424in}{1.287199in}}%
\pgfpathlineto{\pgfqpoint{2.660103in}{1.453438in}}%
\pgfpathlineto{\pgfqpoint{2.674773in}{1.696801in}}%
\pgfpathlineto{\pgfqpoint{2.687716in}{1.945279in}}%
\pgfpathlineto{\pgfqpoint{2.692670in}{2.079573in}}%
\pgfpathlineto{\pgfqpoint{2.693829in}{2.166682in}}%
\pgfpathlineto{\pgfqpoint{2.692565in}{2.233870in}}%
\pgfpathlineto{\pgfqpoint{2.689436in}{2.286015in}}%
\pgfpathlineto{\pgfqpoint{2.684859in}{2.327999in}}%
\pgfpathlineto{\pgfqpoint{2.678725in}{2.364664in}}%
\pgfpathlineto{\pgfqpoint{2.671356in}{2.395897in}}%
\pgfpathlineto{\pgfqpoint{2.662489in}{2.423981in}}%
\pgfpathlineto{\pgfqpoint{2.652361in}{2.448778in}}%
\pgfpathlineto{\pgfqpoint{2.641365in}{2.470245in}}%
\pgfpathlineto{\pgfqpoint{2.628643in}{2.490425in}}%
\pgfpathlineto{\pgfqpoint{2.614279in}{2.509106in}}%
\pgfpathlineto{\pgfqpoint{2.598443in}{2.526159in}}%
\pgfpathlineto{\pgfqpoint{2.579590in}{2.543005in}}%
\pgfpathlineto{\pgfqpoint{2.559532in}{2.557923in}}%
\pgfpathlineto{\pgfqpoint{2.536602in}{2.572183in}}%
\pgfpathlineto{\pgfqpoint{2.510850in}{2.585538in}}%
\pgfpathlineto{\pgfqpoint{2.482360in}{2.597837in}}%
\pgfpathlineto{\pgfqpoint{2.449134in}{2.609683in}}%
\pgfpathlineto{\pgfqpoint{2.411184in}{2.620696in}}%
\pgfpathlineto{\pgfqpoint{2.368552in}{2.630606in}}%
\pgfpathlineto{\pgfqpoint{2.321294in}{2.639221in}}%
\pgfpathlineto{\pgfqpoint{2.269467in}{2.646399in}}%
\pgfpathlineto{\pgfqpoint{2.210954in}{2.652193in}}%
\pgfpathlineto{\pgfqpoint{2.147967in}{2.656153in}}%
\pgfpathlineto{\pgfqpoint{2.080556in}{2.658135in}}%
\pgfpathlineto{\pgfqpoint{2.010948in}{2.657971in}}%
\pgfpathlineto{\pgfqpoint{1.939195in}{2.655572in}}%
\pgfpathlineto{\pgfqpoint{1.867527in}{2.650913in}}%
\pgfpathlineto{\pgfqpoint{1.798171in}{2.644140in}}%
\pgfpathlineto{\pgfqpoint{1.733341in}{2.635606in}}%
\pgfpathlineto{\pgfqpoint{1.673075in}{2.625521in}}%
\pgfpathlineto{\pgfqpoint{1.615274in}{2.613610in}}%
\pgfpathlineto{\pgfqpoint{1.562133in}{2.600402in}}%
\pgfpathlineto{\pgfqpoint{1.513681in}{2.586139in}}%
\pgfpathlineto{\pgfqpoint{1.467862in}{2.570344in}}%
\pgfpathlineto{\pgfqpoint{1.426794in}{2.553923in}}%
\pgfpathlineto{\pgfqpoint{1.388447in}{2.536289in}}%
\pgfpathlineto{\pgfqpoint{1.352878in}{2.517566in}}%
\pgfpathlineto{\pgfqpoint{1.320128in}{2.497922in}}%
\pgfpathlineto{\pgfqpoint{1.288379in}{2.476236in}}%
\pgfpathlineto{\pgfqpoint{1.259592in}{2.453861in}}%
\pgfpathlineto{\pgfqpoint{1.232050in}{2.429520in}}%
\pgfpathlineto{\pgfqpoint{1.207527in}{2.404898in}}%
\pgfpathlineto{\pgfqpoint{1.184409in}{2.378557in}}%
\pgfpathlineto{\pgfqpoint{1.162828in}{2.350561in}}%
\pgfpathlineto{\pgfqpoint{1.142891in}{2.321011in}}%
\pgfpathlineto{\pgfqpoint{1.124675in}{2.290041in}}%
\pgfpathlineto{\pgfqpoint{1.108225in}{2.257802in}}%
\pgfpathlineto{\pgfqpoint{1.092639in}{2.222199in}}%
\pgfpathlineto{\pgfqpoint{1.079059in}{2.185535in}}%
\pgfpathlineto{\pgfqpoint{1.067443in}{2.147998in}}%
\pgfpathlineto{\pgfqpoint{1.057187in}{2.107348in}}%
\pgfpathlineto{\pgfqpoint{1.049004in}{2.066086in}}%
\pgfpathlineto{\pgfqpoint{1.042513in}{2.021906in}}%
\pgfpathlineto{\pgfqpoint{1.038177in}{1.977382in}}%
\pgfpathlineto{\pgfqpoint{1.035866in}{1.930167in}}%
\pgfpathlineto{\pgfqpoint{1.035826in}{1.882878in}}%
\pgfpathlineto{\pgfqpoint{1.038031in}{1.835656in}}%
\pgfpathlineto{\pgfqpoint{1.042474in}{1.788641in}}%
\pgfpathlineto{\pgfqpoint{1.049176in}{1.741979in}}%
\pgfpathlineto{\pgfqpoint{1.057644in}{1.698239in}}%
\pgfpathlineto{\pgfqpoint{1.068221in}{1.655105in}}%
\pgfpathlineto{\pgfqpoint{1.080962in}{1.612745in}}%
\pgfpathlineto{\pgfqpoint{1.095031in}{1.573617in}}%
\pgfpathlineto{\pgfqpoint{1.111115in}{1.535520in}}%
\pgfpathlineto{\pgfqpoint{1.128118in}{1.500775in}}%
\pgfpathlineto{\pgfqpoint{1.146930in}{1.467274in}}%
\pgfpathlineto{\pgfqpoint{1.167531in}{1.435181in}}%
\pgfpathlineto{\pgfqpoint{1.189874in}{1.404652in}}%
\pgfpathlineto{\pgfqpoint{1.213884in}{1.375828in}}%
\pgfpathlineto{\pgfqpoint{1.237817in}{1.350457in}}%
\pgfpathlineto{\pgfqpoint{1.264748in}{1.325237in}}%
\pgfpathlineto{\pgfqpoint{1.292991in}{1.301972in}}%
\pgfpathlineto{\pgfqpoint{1.322398in}{1.280678in}}%
\pgfpathlineto{\pgfqpoint{1.352820in}{1.261340in}}%
\pgfpathlineto{\pgfqpoint{1.386095in}{1.242889in}}%
\pgfpathlineto{\pgfqpoint{1.420190in}{1.226516in}}%
\pgfpathlineto{\pgfqpoint{1.457024in}{1.211329in}}%
\pgfpathlineto{\pgfqpoint{1.496554in}{1.197536in}}%
\pgfpathlineto{\pgfqpoint{1.538719in}{1.185287in}}%
\pgfpathlineto{\pgfqpoint{1.583441in}{1.174641in}}%
\pgfpathlineto{\pgfqpoint{1.634929in}{1.164775in}}%
\pgfpathlineto{\pgfqpoint{1.706063in}{1.153745in}}%
\pgfpathlineto{\pgfqpoint{1.768492in}{1.143417in}}%
\pgfpathlineto{\pgfqpoint{1.796122in}{1.136567in}}%
\pgfpathlineto{\pgfqpoint{1.812683in}{1.130481in}}%
\pgfpathlineto{\pgfqpoint{1.824471in}{1.124102in}}%
\pgfpathlineto{\pgfqpoint{1.833209in}{1.116741in}}%
\pgfpathlineto{\pgfqpoint{1.838498in}{1.108890in}}%
\pgfpathlineto{\pgfqpoint{1.840588in}{1.101849in}}%
\pgfpathlineto{\pgfqpoint{1.840619in}{1.094412in}}%
\pgfpathlineto{\pgfqpoint{1.837931in}{1.084986in}}%
\pgfpathlineto{\pgfqpoint{1.833246in}{1.076615in}}%
\pgfpathlineto{\pgfqpoint{1.825819in}{1.067542in}}%
\pgfpathlineto{\pgfqpoint{1.813813in}{1.056850in}}%
\pgfpathlineto{\pgfqpoint{1.798819in}{1.046763in}}%
\pgfpathlineto{\pgfqpoint{1.781016in}{1.037462in}}%
\pgfpathlineto{\pgfqpoint{1.758447in}{1.028391in}}%
\pgfpathlineto{\pgfqpoint{1.733203in}{1.020815in}}%
\pgfpathlineto{\pgfqpoint{1.705410in}{1.014872in}}%
\pgfpathlineto{\pgfqpoint{1.675178in}{1.010714in}}%
\pgfpathlineto{\pgfqpoint{1.642610in}{1.008507in}}%
\pgfpathlineto{\pgfqpoint{1.607809in}{1.008432in}}%
\pgfpathlineto{\pgfqpoint{1.570886in}{1.010691in}}%
\pgfpathlineto{\pgfqpoint{1.534118in}{1.015181in}}%
\pgfpathlineto{\pgfqpoint{1.495454in}{1.022233in}}%
\pgfpathlineto{\pgfqpoint{1.457161in}{1.031563in}}%
\pgfpathlineto{\pgfqpoint{1.419337in}{1.043132in}}%
\pgfpathlineto{\pgfqpoint{1.382089in}{1.056929in}}%
\pgfpathlineto{\pgfqpoint{1.347544in}{1.072019in}}%
\pgfpathlineto{\pgfqpoint{1.313727in}{1.089133in}}%
\pgfpathlineto{\pgfqpoint{1.280762in}{1.108299in}}%
\pgfpathlineto{\pgfqpoint{1.248782in}{1.129536in}}%
\pgfpathlineto{\pgfqpoint{1.219708in}{1.151422in}}%
\pgfpathlineto{\pgfqpoint{1.191752in}{1.175138in}}%
\pgfpathlineto{\pgfqpoint{1.165031in}{1.200649in}}%
\pgfpathlineto{\pgfqpoint{1.139653in}{1.227898in}}%
\pgfpathlineto{\pgfqpoint{1.115714in}{1.256800in}}%
\pgfpathlineto{\pgfqpoint{1.093288in}{1.287251in}}%
\pgfpathlineto{\pgfqpoint{1.071178in}{1.321163in}}%
\pgfpathlineto{\pgfqpoint{1.050868in}{1.356520in}}%
\pgfpathlineto{\pgfqpoint{1.032365in}{1.393152in}}%
\pgfpathlineto{\pgfqpoint{1.014718in}{1.433142in}}%
\pgfpathlineto{\pgfqpoint{0.999024in}{1.474185in}}%
\pgfpathlineto{\pgfqpoint{0.984506in}{1.518461in}}%
\pgfpathlineto{\pgfqpoint{0.972010in}{1.563537in}}%
\pgfpathlineto{\pgfqpoint{0.960944in}{1.611678in}}%
\pgfpathlineto{\pgfqpoint{0.951530in}{1.662824in}}%
\pgfpathlineto{\pgfqpoint{0.944286in}{1.714431in}}%
\pgfpathlineto{\pgfqpoint{0.938950in}{1.768847in}}%
\pgfpathlineto{\pgfqpoint{0.935870in}{1.823491in}}%
\pgfpathlineto{\pgfqpoint{0.935034in}{1.878240in}}%
\pgfpathlineto{\pgfqpoint{0.936466in}{1.932973in}}%
\pgfpathlineto{\pgfqpoint{0.940005in}{1.985084in}}%
\pgfpathlineto{\pgfqpoint{0.945759in}{2.036935in}}%
\pgfpathlineto{\pgfqpoint{0.953410in}{2.085938in}}%
\pgfpathlineto{\pgfqpoint{0.962764in}{2.132000in}}%
\pgfpathlineto{\pgfqpoint{0.974287in}{2.177414in}}%
\pgfpathlineto{\pgfqpoint{0.987332in}{2.219653in}}%
\pgfpathlineto{\pgfqpoint{1.001667in}{2.258654in}}%
\pgfpathlineto{\pgfqpoint{1.018051in}{2.296583in}}%
\pgfpathlineto{\pgfqpoint{1.035401in}{2.331101in}}%
\pgfpathlineto{\pgfqpoint{1.054650in}{2.364275in}}%
\pgfpathlineto{\pgfqpoint{1.074406in}{2.393984in}}%
\pgfpathlineto{\pgfqpoint{1.095771in}{2.422197in}}%
\pgfpathlineto{\pgfqpoint{1.118662in}{2.448797in}}%
\pgfpathlineto{\pgfqpoint{1.142967in}{2.473701in}}%
\pgfpathlineto{\pgfqpoint{1.168550in}{2.496867in}}%
\pgfpathlineto{\pgfqpoint{1.197085in}{2.519662in}}%
\pgfpathlineto{\pgfqpoint{1.226727in}{2.540526in}}%
\pgfpathlineto{\pgfqpoint{1.259242in}{2.560673in}}%
\pgfpathlineto{\pgfqpoint{1.294612in}{2.579881in}}%
\pgfpathlineto{\pgfqpoint{1.332792in}{2.597982in}}%
\pgfpathlineto{\pgfqpoint{1.373719in}{2.614859in}}%
\pgfpathlineto{\pgfqpoint{1.417319in}{2.630445in}}%
\pgfpathlineto{\pgfqpoint{1.465632in}{2.645312in}}%
\pgfpathlineto{\pgfqpoint{1.518640in}{2.659204in}}%
\pgfpathlineto{\pgfqpoint{1.576309in}{2.671929in}}%
\pgfpathlineto{\pgfqpoint{1.638597in}{2.683344in}}%
\pgfpathlineto{\pgfqpoint{1.705462in}{2.693343in}}%
\pgfpathlineto{\pgfqpoint{1.779027in}{2.702064in}}%
\pgfpathlineto{\pgfqpoint{1.857097in}{2.709077in}}%
\pgfpathlineto{\pgfqpoint{1.939633in}{2.714280in}}%
\pgfpathlineto{\pgfqpoint{2.026598in}{2.717513in}}%
\pgfpathlineto{\pgfqpoint{2.113605in}{2.718523in}}%
\pgfpathlineto{\pgfqpoint{2.198435in}{2.717303in}}%
\pgfpathlineto{\pgfqpoint{2.278866in}{2.713929in}}%
\pgfpathlineto{\pgfqpoint{2.352678in}{2.708598in}}%
\pgfpathlineto{\pgfqpoint{2.417657in}{2.701709in}}%
\pgfpathlineto{\pgfqpoint{2.473770in}{2.693630in}}%
\pgfpathlineto{\pgfqpoint{2.523140in}{2.684368in}}%
\pgfpathlineto{\pgfqpoint{2.565726in}{2.674202in}}%
\pgfpathlineto{\pgfqpoint{2.601510in}{2.663544in}}%
\pgfpathlineto{\pgfqpoint{2.632577in}{2.652142in}}%
\pgfpathlineto{\pgfqpoint{2.658899in}{2.640331in}}%
\pgfpathlineto{\pgfqpoint{2.682438in}{2.627436in}}%
\pgfpathlineto{\pgfqpoint{2.703062in}{2.613571in}}%
\pgfpathlineto{\pgfqpoint{2.720674in}{2.598978in}}%
\pgfpathlineto{\pgfqpoint{2.735263in}{2.584053in}}%
\pgfpathlineto{\pgfqpoint{2.748320in}{2.567377in}}%
\pgfpathlineto{\pgfqpoint{2.759553in}{2.549046in}}%
\pgfpathlineto{\pgfqpoint{2.768788in}{2.529306in}}%
\pgfpathlineto{\pgfqpoint{2.776017in}{2.508498in}}%
\pgfpathlineto{\pgfqpoint{2.781884in}{2.484540in}}%
\pgfpathlineto{\pgfqpoint{2.786102in}{2.457597in}}%
\pgfpathlineto{\pgfqpoint{2.788720in}{2.425384in}}%
\pgfpathlineto{\pgfqpoint{2.789427in}{2.388061in}}%
\pgfpathlineto{\pgfqpoint{2.787962in}{2.340801in}}%
\pgfpathlineto{\pgfqpoint{2.783672in}{2.278768in}}%
\pgfpathlineto{\pgfqpoint{2.774289in}{2.179783in}}%
\pgfpathlineto{\pgfqpoint{2.743611in}{1.868119in}}%
\pgfpathlineto{\pgfqpoint{2.730112in}{1.702060in}}%
\pgfpathlineto{\pgfqpoint{2.717287in}{1.515949in}}%
\pgfpathlineto{\pgfqpoint{2.702602in}{1.267597in}}%
\pgfpathlineto{\pgfqpoint{2.684434in}{0.964630in}}%
\pgfpathlineto{\pgfqpoint{2.675374in}{0.850600in}}%
\pgfpathlineto{\pgfqpoint{2.667030in}{0.771523in}}%
\pgfpathlineto{\pgfqpoint{2.658752in}{0.712543in}}%
\pgfpathlineto{\pgfqpoint{2.650176in}{0.666284in}}%
\pgfpathlineto{\pgfqpoint{2.640820in}{0.627931in}}%
\pgfpathlineto{\pgfqpoint{2.631145in}{0.597534in}}%
\pgfpathlineto{\pgfqpoint{2.621004in}{0.572745in}}%
\pgfpathlineto{\pgfqpoint{2.609856in}{0.551383in}}%
\pgfpathlineto{\pgfqpoint{2.598042in}{0.533534in}}%
\pgfpathlineto{\pgfqpoint{2.584496in}{0.517378in}}%
\pgfpathlineto{\pgfqpoint{2.571109in}{0.504669in}}%
\pgfpathlineto{\pgfqpoint{2.554789in}{0.492313in}}%
\pgfpathlineto{\pgfqpoint{2.537457in}{0.481914in}}%
\pgfpathlineto{\pgfqpoint{2.517374in}{0.472367in}}%
\pgfpathlineto{\pgfqpoint{2.492542in}{0.463178in}}%
\pgfpathlineto{\pgfqpoint{2.462979in}{0.454833in}}%
\pgfpathlineto{\pgfqpoint{2.428766in}{0.447542in}}%
\pgfpathlineto{\pgfqpoint{2.385671in}{0.440735in}}%
\pgfpathlineto{\pgfqpoint{2.331557in}{0.434581in}}%
\pgfpathlineto{\pgfqpoint{2.262115in}{0.429077in}}%
\pgfpathlineto{\pgfqpoint{2.170851in}{0.424236in}}%
\pgfpathlineto{\pgfqpoint{2.049086in}{0.420134in}}%
\pgfpathlineto{\pgfqpoint{1.879436in}{0.416783in}}%
\pgfpathlineto{\pgfqpoint{1.640159in}{0.414418in}}%
\pgfpathlineto{\pgfqpoint{1.322562in}{0.413569in}}%
\pgfpathlineto{\pgfqpoint{1.020194in}{0.414850in}}%
\pgfpathlineto{\pgfqpoint{0.822256in}{0.417715in}}%
\pgfpathlineto{\pgfqpoint{0.704835in}{0.421430in}}%
\pgfpathlineto{\pgfqpoint{0.630976in}{0.425829in}}%
\pgfpathlineto{\pgfqpoint{0.583316in}{0.430734in}}%
\pgfpathlineto{\pgfqpoint{0.551033in}{0.436123in}}%
\pgfpathlineto{\pgfqpoint{0.527708in}{0.442189in}}%
\pgfpathlineto{\pgfqpoint{0.511250in}{0.448625in}}%
\pgfpathlineto{\pgfqpoint{0.499549in}{0.455216in}}%
\pgfpathlineto{\pgfqpoint{0.488916in}{0.463841in}}%
\pgfpathlineto{\pgfqpoint{0.481322in}{0.472730in}}%
\pgfpathlineto{\pgfqpoint{0.474078in}{0.485127in}}%
\pgfpathlineto{\pgfqpoint{0.468753in}{0.498748in}}%
\pgfpathlineto{\pgfqpoint{0.463870in}{0.517848in}}%
\pgfpathlineto{\pgfqpoint{0.459679in}{0.544796in}}%
\pgfpathlineto{\pgfqpoint{0.456386in}{0.581938in}}%
\pgfpathlineto{\pgfqpoint{0.453731in}{0.639106in}}%
\pgfpathlineto{\pgfqpoint{0.451681in}{0.736155in}}%
\pgfpathlineto{\pgfqpoint{0.450220in}{0.927815in}}%
\pgfpathlineto{\pgfqpoint{0.449345in}{1.403252in}}%
\pgfpathlineto{\pgfqpoint{0.449543in}{2.682703in}}%
\pgfpathlineto{\pgfqpoint{0.451011in}{2.856932in}}%
\pgfpathlineto{\pgfqpoint{0.452802in}{2.879219in}}%
\pgfpathlineto{\pgfqpoint{0.455188in}{2.886108in}}%
\pgfpathlineto{\pgfqpoint{0.458626in}{2.889028in}}%
\pgfpathlineto{\pgfqpoint{0.464996in}{2.890553in}}%
\pgfpathlineto{\pgfqpoint{0.482377in}{2.891423in}}%
\pgfpathlineto{\pgfqpoint{0.565038in}{2.891729in}}%
\pgfpathlineto{\pgfqpoint{2.733842in}{2.891760in}}%
\pgfpathlineto{\pgfqpoint{4.789510in}{2.890885in}}%
\pgfpathlineto{\pgfqpoint{4.793727in}{2.889730in}}%
\pgfpathlineto{\pgfqpoint{4.795481in}{2.888307in}}%
\pgfpathlineto{\pgfqpoint{4.797106in}{2.881145in}}%
\pgfpathlineto{\pgfqpoint{4.797997in}{2.858771in}}%
\pgfpathlineto{\pgfqpoint{4.798039in}{2.856283in}}%
\pgfpathlineto{\pgfqpoint{4.798039in}{2.856283in}}%
\pgfusepath{stroke}%
\end{pgfscope}%
\begin{pgfscope}%
\pgfpathrectangle{\pgfqpoint{0.448634in}{0.402556in}}{\pgfqpoint{4.350661in}{2.489204in}} %
\pgfusepath{clip}%
\pgfsetrectcap%
\pgfsetroundjoin%
\pgfsetlinewidth{1.003750pt}%
\definecolor{currentstroke}{rgb}{1.000000,0.388235,0.278431}%
\pgfsetstrokecolor{currentstroke}%
\pgfsetdash{}{0pt}%
\pgfpathmoveto{\pgfqpoint{3.428772in}{0.402610in}}%
\pgfpathlineto{\pgfqpoint{2.806632in}{0.403760in}}%
\pgfpathlineto{\pgfqpoint{2.769692in}{0.405578in}}%
\pgfpathlineto{\pgfqpoint{2.754632in}{0.408064in}}%
\pgfpathlineto{\pgfqpoint{2.746391in}{0.411198in}}%
\pgfpathlineto{\pgfqpoint{2.740943in}{0.415265in}}%
\pgfpathlineto{\pgfqpoint{2.736784in}{0.420984in}}%
\pgfpathlineto{\pgfqpoint{2.733281in}{0.430071in}}%
\pgfpathlineto{\pgfqpoint{2.730449in}{0.444636in}}%
\pgfpathlineto{\pgfqpoint{2.728238in}{0.469392in}}%
\pgfpathlineto{\pgfqpoint{2.726470in}{0.519131in}}%
\pgfpathlineto{\pgfqpoint{2.725711in}{0.613715in}}%
\pgfpathlineto{\pgfqpoint{2.726842in}{0.768038in}}%
\pgfpathlineto{\pgfqpoint{2.730556in}{0.962148in}}%
\pgfpathlineto{\pgfqpoint{2.736611in}{1.158670in}}%
\pgfpathlineto{\pgfqpoint{2.744092in}{1.327718in}}%
\pgfpathlineto{\pgfqpoint{2.753201in}{1.484189in}}%
\pgfpathlineto{\pgfqpoint{2.763257in}{1.620609in}}%
\pgfpathlineto{\pgfqpoint{2.776118in}{1.764216in}}%
\pgfpathlineto{\pgfqpoint{2.788914in}{1.877776in}}%
\pgfpathlineto{\pgfqpoint{2.805748in}{2.005740in}}%
\pgfpathlineto{\pgfqpoint{2.821176in}{2.101198in}}%
\pgfpathlineto{\pgfqpoint{2.838359in}{2.193718in}}%
\pgfpathlineto{\pgfqpoint{2.859135in}{2.292966in}}%
\pgfpathlineto{\pgfqpoint{2.887209in}{2.425960in}}%
\pgfpathlineto{\pgfqpoint{2.896991in}{2.479559in}}%
\pgfpathlineto{\pgfqpoint{2.901543in}{2.516523in}}%
\pgfpathlineto{\pgfqpoint{2.902849in}{2.543854in}}%
\pgfpathlineto{\pgfqpoint{2.901957in}{2.566223in}}%
\pgfpathlineto{\pgfqpoint{2.899151in}{2.585863in}}%
\pgfpathlineto{\pgfqpoint{2.894794in}{2.602546in}}%
\pgfpathlineto{\pgfqpoint{2.888484in}{2.618388in}}%
\pgfpathlineto{\pgfqpoint{2.880257in}{2.633033in}}%
\pgfpathlineto{\pgfqpoint{2.870348in}{2.646246in}}%
\pgfpathlineto{\pgfqpoint{2.857400in}{2.659530in}}%
\pgfpathlineto{\pgfqpoint{2.843189in}{2.671010in}}%
\pgfpathlineto{\pgfqpoint{2.824237in}{2.683209in}}%
\pgfpathlineto{\pgfqpoint{2.802413in}{2.694418in}}%
\pgfpathlineto{\pgfqpoint{2.775809in}{2.705369in}}%
\pgfpathlineto{\pgfqpoint{2.744461in}{2.715715in}}%
\pgfpathlineto{\pgfqpoint{2.708436in}{2.725252in}}%
\pgfpathlineto{\pgfqpoint{2.665655in}{2.734289in}}%
\pgfpathlineto{\pgfqpoint{2.613991in}{2.742869in}}%
\pgfpathlineto{\pgfqpoint{2.553459in}{2.750589in}}%
\pgfpathlineto{\pgfqpoint{2.481920in}{2.757365in}}%
\pgfpathlineto{\pgfqpoint{2.399398in}{2.762839in}}%
\pgfpathlineto{\pgfqpoint{2.310269in}{2.766482in}}%
\pgfpathlineto{\pgfqpoint{2.175416in}{2.768725in}}%
\pgfpathlineto{\pgfqpoint{2.066653in}{2.767942in}}%
\pgfpathlineto{\pgfqpoint{1.953570in}{2.764859in}}%
\pgfpathlineto{\pgfqpoint{1.851429in}{2.759759in}}%
\pgfpathlineto{\pgfqpoint{1.745051in}{2.752169in}}%
\pgfpathlineto{\pgfqpoint{1.658373in}{2.743453in}}%
\pgfpathlineto{\pgfqpoint{1.580552in}{2.733461in}}%
\pgfpathlineto{\pgfqpoint{1.490057in}{2.719338in}}%
\pgfpathlineto{\pgfqpoint{1.417231in}{2.704698in}}%
\pgfpathlineto{\pgfqpoint{1.361992in}{2.690818in}}%
\pgfpathlineto{\pgfqpoint{1.311460in}{2.675819in}}%
\pgfpathlineto{\pgfqpoint{1.265667in}{2.659924in}}%
\pgfpathlineto{\pgfqpoint{1.222575in}{2.642586in}}%
\pgfpathlineto{\pgfqpoint{1.184324in}{2.624682in}}%
\pgfpathlineto{\pgfqpoint{1.148892in}{2.605623in}}%
\pgfpathlineto{\pgfqpoint{1.116331in}{2.585573in}}%
\pgfpathlineto{\pgfqpoint{1.092327in}{2.568512in}}%
\pgfpathlineto{\pgfqpoint{1.079760in}{2.558686in}}%
\pgfpathlineto{\pgfqpoint{1.051544in}{2.535379in}}%
\pgfpathlineto{\pgfqpoint{1.026312in}{2.511712in}}%
\pgfpathlineto{\pgfqpoint{1.002399in}{2.486318in}}%
\pgfpathlineto{\pgfqpoint{0.979913in}{2.459269in}}%
\pgfpathlineto{\pgfqpoint{0.958934in}{2.430678in}}%
\pgfpathlineto{\pgfqpoint{0.938264in}{2.398643in}}%
\pgfpathlineto{\pgfqpoint{0.923047in}{2.371385in}}%
\pgfpathlineto{\pgfqpoint{0.904513in}{2.334774in}}%
\pgfpathlineto{\pgfqpoint{0.887854in}{2.297001in}}%
\pgfpathlineto{\pgfqpoint{0.872131in}{2.255971in}}%
\pgfpathlineto{\pgfqpoint{0.857508in}{2.211741in}}%
\pgfpathlineto{\pgfqpoint{0.844762in}{2.166757in}}%
\pgfpathlineto{\pgfqpoint{0.838624in}{2.140306in}}%
\pgfpathlineto{\pgfqpoint{0.826982in}{2.087194in}}%
\pgfpathlineto{\pgfqpoint{0.816322in}{2.028715in}}%
\pgfpathlineto{\pgfqpoint{0.810087in}{1.984495in}}%
\pgfpathlineto{\pgfqpoint{0.808026in}{1.967238in}}%
\pgfpathlineto{\pgfqpoint{0.800076in}{1.898140in}}%
\pgfpathlineto{\pgfqpoint{0.793713in}{1.823823in}}%
\pgfpathlineto{\pgfqpoint{0.788799in}{1.741875in}}%
\pgfpathlineto{\pgfqpoint{0.786199in}{1.677225in}}%
\pgfpathlineto{\pgfqpoint{0.776951in}{1.453481in}}%
\pgfpathlineto{\pgfqpoint{0.773280in}{1.418894in}}%
\pgfpathlineto{\pgfqpoint{0.768298in}{1.389582in}}%
\pgfpathlineto{\pgfqpoint{0.762752in}{1.368108in}}%
\pgfpathlineto{\pgfqpoint{0.756722in}{1.352123in}}%
\pgfpathlineto{\pgfqpoint{0.749752in}{1.339519in}}%
\pgfpathlineto{\pgfqpoint{0.742201in}{1.330599in}}%
\pgfpathlineto{\pgfqpoint{0.734854in}{1.325312in}}%
\pgfpathlineto{\pgfqpoint{0.726558in}{1.322419in}}%
\pgfpathlineto{\pgfqpoint{0.717884in}{1.322223in}}%
\pgfpathlineto{\pgfqpoint{0.709412in}{1.324411in}}%
\pgfpathlineto{\pgfqpoint{0.699548in}{1.329604in}}%
\pgfpathlineto{\pgfqpoint{0.688894in}{1.338203in}}%
\pgfpathlineto{\pgfqpoint{0.677907in}{1.350248in}}%
\pgfpathlineto{\pgfqpoint{0.666886in}{1.365647in}}%
\pgfpathlineto{\pgfqpoint{0.654913in}{1.386417in}}%
\pgfpathlineto{\pgfqpoint{0.642574in}{1.412730in}}%
\pgfpathlineto{\pgfqpoint{0.630328in}{1.444629in}}%
\pgfpathlineto{\pgfqpoint{0.618504in}{1.482081in}}%
\pgfpathlineto{\pgfqpoint{0.608613in}{1.520256in}}%
\pgfpathlineto{\pgfqpoint{0.590203in}{1.612445in}}%
\pgfpathlineto{\pgfqpoint{0.581848in}{1.668884in}}%
\pgfpathlineto{\pgfqpoint{0.573137in}{1.740376in}}%
\pgfpathlineto{\pgfqpoint{0.567062in}{1.807213in}}%
\pgfpathlineto{\pgfqpoint{0.560532in}{1.896510in}}%
\pgfpathlineto{\pgfqpoint{0.555526in}{1.995910in}}%
\pgfpathlineto{\pgfqpoint{0.552564in}{2.097908in}}%
\pgfpathlineto{\pgfqpoint{0.551526in}{2.204935in}}%
\pgfpathlineto{\pgfqpoint{0.552728in}{2.309470in}}%
\pgfpathlineto{\pgfqpoint{0.556011in}{2.403981in}}%
\pgfpathlineto{\pgfqpoint{0.560953in}{2.483430in}}%
\pgfpathlineto{\pgfqpoint{0.567303in}{2.550240in}}%
\pgfpathlineto{\pgfqpoint{0.574928in}{2.606817in}}%
\pgfpathlineto{\pgfqpoint{0.582988in}{2.650657in}}%
\pgfpathlineto{\pgfqpoint{0.592756in}{2.691452in}}%
\pgfpathlineto{\pgfqpoint{0.602650in}{2.721756in}}%
\pgfpathlineto{\pgfqpoint{0.612983in}{2.746441in}}%
\pgfpathlineto{\pgfqpoint{0.624292in}{2.767692in}}%
\pgfpathlineto{\pgfqpoint{0.636231in}{2.785433in}}%
\pgfpathlineto{\pgfqpoint{0.649892in}{2.801461in}}%
\pgfpathlineto{\pgfqpoint{0.663386in}{2.814020in}}%
\pgfpathlineto{\pgfqpoint{0.679842in}{2.826135in}}%
\pgfpathlineto{\pgfqpoint{0.697326in}{2.836197in}}%
\pgfpathlineto{\pgfqpoint{0.715574in}{2.844285in}}%
\pgfpathlineto{\pgfqpoint{0.738439in}{2.852335in}}%
\pgfpathlineto{\pgfqpoint{0.765983in}{2.859639in}}%
\pgfpathlineto{\pgfqpoint{0.800300in}{2.866256in}}%
\pgfpathlineto{\pgfqpoint{0.841340in}{2.871832in}}%
\pgfpathlineto{\pgfqpoint{0.895547in}{2.876803in}}%
\pgfpathlineto{\pgfqpoint{0.969413in}{2.881069in}}%
\pgfpathlineto{\pgfqpoint{1.071608in}{2.884501in}}%
\pgfpathlineto{\pgfqpoint{1.219512in}{2.887074in}}%
\pgfpathlineto{\pgfqpoint{1.471844in}{2.889091in}}%
\pgfpathlineto{\pgfqpoint{1.956941in}{2.890384in}}%
\pgfpathlineto{\pgfqpoint{3.096814in}{2.890781in}}%
\pgfpathlineto{\pgfqpoint{3.995224in}{2.889388in}}%
\pgfpathlineto{\pgfqpoint{4.275833in}{2.887011in}}%
\pgfpathlineto{\pgfqpoint{4.412847in}{2.883743in}}%
\pgfpathlineto{\pgfqpoint{4.491081in}{2.879810in}}%
\pgfpathlineto{\pgfqpoint{4.543127in}{2.875163in}}%
\pgfpathlineto{\pgfqpoint{4.579810in}{2.869841in}}%
\pgfpathlineto{\pgfqpoint{4.607580in}{2.863763in}}%
\pgfpathlineto{\pgfqpoint{4.630623in}{2.856424in}}%
\pgfpathlineto{\pgfqpoint{4.648833in}{2.848228in}}%
\pgfpathlineto{\pgfqpoint{4.664136in}{2.838773in}}%
\pgfpathlineto{\pgfqpoint{4.676470in}{2.828576in}}%
\pgfpathlineto{\pgfqpoint{4.687502in}{2.816585in}}%
\pgfpathlineto{\pgfqpoint{4.697051in}{2.803027in}}%
\pgfpathlineto{\pgfqpoint{4.706194in}{2.786098in}}%
\pgfpathlineto{\pgfqpoint{4.714508in}{2.765827in}}%
\pgfpathlineto{\pgfqpoint{4.722462in}{2.740013in}}%
\pgfpathlineto{\pgfqpoint{4.729577in}{2.708703in}}%
\pgfpathlineto{\pgfqpoint{4.736162in}{2.669601in}}%
\pgfpathlineto{\pgfqpoint{4.742419in}{2.617826in}}%
\pgfpathlineto{\pgfqpoint{4.747859in}{2.553410in}}%
\pgfpathlineto{\pgfqpoint{4.752661in}{2.468958in}}%
\pgfpathlineto{\pgfqpoint{4.756610in}{2.359528in}}%
\pgfpathlineto{\pgfqpoint{4.759416in}{2.217681in}}%
\pgfpathlineto{\pgfqpoint{4.760596in}{2.043444in}}%
\pgfpathlineto{\pgfqpoint{4.759662in}{1.851779in}}%
\pgfpathlineto{\pgfqpoint{4.756587in}{1.667613in}}%
\pgfpathlineto{\pgfqpoint{4.751596in}{1.503428in}}%
\pgfpathlineto{\pgfqpoint{4.745410in}{1.374185in}}%
\pgfpathlineto{\pgfqpoint{4.738113in}{1.267479in}}%
\pgfpathlineto{\pgfqpoint{4.729621in}{1.175896in}}%
\pgfpathlineto{\pgfqpoint{4.720762in}{1.104428in}}%
\pgfpathlineto{\pgfqpoint{4.711045in}{1.043204in}}%
\pgfpathlineto{\pgfqpoint{4.700364in}{0.989829in}}%
\pgfpathlineto{\pgfqpoint{4.689055in}{0.944345in}}%
\pgfpathlineto{\pgfqpoint{4.676881in}{0.904394in}}%
\pgfpathlineto{\pgfqpoint{4.676095in}{0.902073in}}%
\pgfpathlineto{\pgfqpoint{4.676095in}{0.902073in}}%
\pgfusepath{stroke}%
\end{pgfscope}%
\begin{pgfscope}%
\pgfpathrectangle{\pgfqpoint{0.448634in}{0.402556in}}{\pgfqpoint{4.350661in}{2.489204in}} %
\pgfusepath{clip}%
\pgfsetrectcap%
\pgfsetroundjoin%
\pgfsetlinewidth{1.003750pt}%
\definecolor{currentstroke}{rgb}{1.000000,0.388235,0.278431}%
\pgfsetstrokecolor{currentstroke}%
\pgfsetdash{}{0pt}%
\pgfpathmoveto{\pgfqpoint{2.795520in}{1.982745in}}%
\pgfpathlineto{\pgfqpoint{2.781780in}{1.874357in}}%
\pgfpathlineto{\pgfqpoint{2.769351in}{1.758234in}}%
\pgfpathlineto{\pgfqpoint{2.758095in}{1.631942in}}%
\pgfpathlineto{\pgfqpoint{2.747786in}{1.490551in}}%
\pgfpathlineto{\pgfqpoint{2.738644in}{1.334082in}}%
\pgfpathlineto{\pgfqpoint{2.730580in}{1.157591in}}%
\pgfpathlineto{\pgfqpoint{2.723334in}{0.948663in}}%
\pgfpathlineto{\pgfqpoint{2.709783in}{0.530788in}}%
\pgfpathlineto{\pgfqpoint{2.705868in}{0.488716in}}%
\pgfpathlineto{\pgfqpoint{2.701769in}{0.464281in}}%
\pgfpathlineto{\pgfqpoint{2.697021in}{0.447744in}}%
\pgfpathlineto{\pgfqpoint{2.691859in}{0.436812in}}%
\pgfpathlineto{\pgfqpoint{2.686245in}{0.429229in}}%
\pgfpathlineto{\pgfqpoint{2.679348in}{0.423188in}}%
\pgfpathlineto{\pgfqpoint{2.669540in}{0.417856in}}%
\pgfpathlineto{\pgfqpoint{2.656987in}{0.413810in}}%
\pgfpathlineto{\pgfqpoint{2.637654in}{0.410337in}}%
\pgfpathlineto{\pgfqpoint{2.607297in}{0.407617in}}%
\pgfpathlineto{\pgfqpoint{2.555121in}{0.405574in}}%
\pgfpathlineto{\pgfqpoint{2.450714in}{0.404139in}}%
\pgfpathlineto{\pgfqpoint{2.176624in}{0.403275in}}%
\pgfpathlineto{\pgfqpoint{1.130290in}{0.402953in}}%
\pgfpathlineto{\pgfqpoint{0.516849in}{0.404175in}}%
\pgfpathlineto{\pgfqpoint{0.466848in}{0.405970in}}%
\pgfpathlineto{\pgfqpoint{0.456130in}{0.407931in}}%
\pgfpathlineto{\pgfqpoint{0.452340in}{0.410303in}}%
\pgfpathlineto{\pgfqpoint{0.450346in}{0.414662in}}%
\pgfpathlineto{\pgfqpoint{0.449266in}{0.424524in}}%
\pgfpathlineto{\pgfqpoint{0.448771in}{0.464344in}}%
\pgfpathlineto{\pgfqpoint{0.448640in}{0.850171in}}%
\pgfpathlineto{\pgfqpoint{0.448653in}{2.891318in}}%
\pgfpathlineto{\pgfqpoint{0.448653in}{2.891318in}}%
\pgfusepath{stroke}%
\end{pgfscope}%
\begin{pgfscope}%
\pgfpathrectangle{\pgfqpoint{0.448634in}{0.402556in}}{\pgfqpoint{4.350661in}{2.489204in}} %
\pgfusepath{clip}%
\pgfsetrectcap%
\pgfsetroundjoin%
\pgfsetlinewidth{1.003750pt}%
\definecolor{currentstroke}{rgb}{1.000000,0.388235,0.278431}%
\pgfsetstrokecolor{currentstroke}%
\pgfsetdash{}{0pt}%
\pgfpathmoveto{\pgfqpoint{3.428189in}{0.402586in}}%
\pgfpathlineto{\pgfqpoint{2.782121in}{0.403701in}}%
\pgfpathlineto{\pgfqpoint{2.753906in}{0.405674in}}%
\pgfpathlineto{\pgfqpoint{2.743328in}{0.408443in}}%
\pgfpathlineto{\pgfqpoint{2.737717in}{0.412188in}}%
\pgfpathlineto{\pgfqpoint{2.733668in}{0.417995in}}%
\pgfpathlineto{\pgfqpoint{2.730649in}{0.427307in}}%
\pgfpathlineto{\pgfqpoint{2.728388in}{0.442004in}}%
\pgfpathlineto{\pgfqpoint{2.726544in}{0.471794in}}%
\pgfpathlineto{\pgfqpoint{2.725216in}{0.534003in}}%
\pgfpathlineto{\pgfqpoint{2.725169in}{0.655973in}}%
\pgfpathlineto{\pgfqpoint{2.727377in}{0.832687in}}%
\pgfpathlineto{\pgfqpoint{2.732259in}{1.041703in}}%
\pgfpathlineto{\pgfqpoint{2.738851in}{1.223257in}}%
\pgfpathlineto{\pgfqpoint{2.747078in}{1.389766in}}%
\pgfpathlineto{\pgfqpoint{2.756608in}{1.538717in}}%
\pgfpathlineto{\pgfqpoint{2.768955in}{1.694887in}}%
\pgfpathlineto{\pgfqpoint{2.781228in}{1.816044in}}%
\pgfpathlineto{\pgfqpoint{2.794401in}{1.924524in}}%
\pgfpathlineto{\pgfqpoint{2.812737in}{2.054722in}}%
\pgfpathlineto{\pgfqpoint{2.828774in}{2.147512in}}%
\pgfpathlineto{\pgfqpoint{2.847382in}{2.242224in}}%
\pgfpathlineto{\pgfqpoint{2.895818in}{2.479699in}}%
\pgfpathlineto{\pgfqpoint{2.900204in}{2.516689in}}%
\pgfpathlineto{\pgfqpoint{2.901346in}{2.544029in}}%
\pgfpathlineto{\pgfqpoint{2.900291in}{2.566388in}}%
\pgfpathlineto{\pgfqpoint{2.897334in}{2.585999in}}%
\pgfpathlineto{\pgfqpoint{2.892836in}{2.602633in}}%
\pgfpathlineto{\pgfqpoint{2.886394in}{2.618405in}}%
\pgfpathlineto{\pgfqpoint{2.878058in}{2.632969in}}%
\pgfpathlineto{\pgfqpoint{2.868065in}{2.646100in}}%
\pgfpathlineto{\pgfqpoint{2.855050in}{2.659300in}}%
\pgfpathlineto{\pgfqpoint{2.840801in}{2.670717in}}%
\pgfpathlineto{\pgfqpoint{2.821822in}{2.682861in}}%
\pgfpathlineto{\pgfqpoint{2.799980in}{2.694026in}}%
\pgfpathlineto{\pgfqpoint{2.773366in}{2.704944in}}%
\pgfpathlineto{\pgfqpoint{2.742012in}{2.715266in}}%
\pgfpathlineto{\pgfqpoint{2.705983in}{2.724785in}}%
\pgfpathlineto{\pgfqpoint{2.663200in}{2.733810in}}%
\pgfpathlineto{\pgfqpoint{2.611535in}{2.742379in}}%
\pgfpathlineto{\pgfqpoint{2.551002in}{2.750090in}}%
\pgfpathlineto{\pgfqpoint{2.481632in}{2.756682in}}%
\pgfpathlineto{\pgfqpoint{2.399112in}{2.762200in}}%
\pgfpathlineto{\pgfqpoint{2.309985in}{2.765886in}}%
\pgfpathlineto{\pgfqpoint{2.188184in}{2.768096in}}%
\pgfpathlineto{\pgfqpoint{2.081595in}{2.767619in}}%
\pgfpathlineto{\pgfqpoint{1.968506in}{2.764840in}}%
\pgfpathlineto{\pgfqpoint{1.864180in}{2.759918in}}%
\pgfpathlineto{\pgfqpoint{1.757786in}{2.752593in}}%
\pgfpathlineto{\pgfqpoint{1.671087in}{2.744171in}}%
\pgfpathlineto{\pgfqpoint{1.591076in}{2.734193in}}%
\pgfpathlineto{\pgfqpoint{1.502689in}{2.720717in}}%
\pgfpathlineto{\pgfqpoint{1.427655in}{2.706083in}}%
\pgfpathlineto{\pgfqpoint{1.372350in}{2.692544in}}%
\pgfpathlineto{\pgfqpoint{1.321734in}{2.677921in}}%
\pgfpathlineto{\pgfqpoint{1.273765in}{2.661664in}}%
\pgfpathlineto{\pgfqpoint{1.230567in}{2.644672in}}%
\pgfpathlineto{\pgfqpoint{1.192197in}{2.627106in}}%
\pgfpathlineto{\pgfqpoint{1.156620in}{2.608403in}}%
\pgfpathlineto{\pgfqpoint{1.123890in}{2.588716in}}%
\pgfpathlineto{\pgfqpoint{1.095883in}{2.569568in}}%
\pgfpathlineto{\pgfqpoint{1.063936in}{2.543701in}}%
\pgfpathlineto{\pgfqpoint{1.038217in}{2.520732in}}%
\pgfpathlineto{\pgfqpoint{1.013766in}{2.496016in}}%
\pgfpathlineto{\pgfqpoint{0.990704in}{2.469610in}}%
\pgfpathlineto{\pgfqpoint{0.969124in}{2.441612in}}%
\pgfpathlineto{\pgfqpoint{0.949083in}{2.412154in}}%
\pgfpathlineto{\pgfqpoint{0.930604in}{2.381387in}}%
\pgfpathlineto{\pgfqpoint{0.906555in}{2.334052in}}%
\pgfpathlineto{\pgfqpoint{0.889925in}{2.296262in}}%
\pgfpathlineto{\pgfqpoint{0.874241in}{2.255213in}}%
\pgfpathlineto{\pgfqpoint{0.859667in}{2.210961in}}%
\pgfpathlineto{\pgfqpoint{0.846986in}{2.165954in}}%
\pgfpathlineto{\pgfqpoint{0.839633in}{2.134715in}}%
\pgfpathlineto{\pgfqpoint{0.828238in}{2.081532in}}%
\pgfpathlineto{\pgfqpoint{0.817866in}{2.022986in}}%
\pgfpathlineto{\pgfqpoint{0.810784in}{1.971352in}}%
\pgfpathlineto{\pgfqpoint{0.802846in}{1.902252in}}%
\pgfpathlineto{\pgfqpoint{0.796554in}{1.827927in}}%
\pgfpathlineto{\pgfqpoint{0.791696in}{1.743480in}}%
\pgfpathlineto{\pgfqpoint{0.787773in}{1.621595in}}%
\pgfpathlineto{\pgfqpoint{0.785408in}{1.522064in}}%
\pgfpathlineto{\pgfqpoint{0.785408in}{1.522064in}}%
\pgfusepath{stroke}%
\end{pgfscope}%
\begin{pgfscope}%
\pgfpathrectangle{\pgfqpoint{0.448634in}{0.402556in}}{\pgfqpoint{4.350661in}{2.489204in}} %
\pgfusepath{clip}%
\pgfsetrectcap%
\pgfsetroundjoin%
\pgfsetlinewidth{1.003750pt}%
\definecolor{currentstroke}{rgb}{1.000000,0.388235,0.278431}%
\pgfsetstrokecolor{currentstroke}%
\pgfsetdash{}{0pt}%
\pgfpathmoveto{\pgfqpoint{2.028735in}{0.425754in}}%
\pgfpathlineto{\pgfqpoint{1.878677in}{0.421879in}}%
\pgfpathlineto{\pgfqpoint{1.676387in}{0.418997in}}%
\pgfpathlineto{\pgfqpoint{1.413176in}{0.417558in}}%
\pgfpathlineto{\pgfqpoint{1.134735in}{0.418204in}}%
\pgfpathlineto{\pgfqpoint{0.921565in}{0.420769in}}%
\pgfpathlineto{\pgfqpoint{0.782384in}{0.424523in}}%
\pgfpathlineto{\pgfqpoint{0.693283in}{0.428974in}}%
\pgfpathlineto{\pgfqpoint{0.632541in}{0.434091in}}%
\pgfpathlineto{\pgfqpoint{0.591492in}{0.439564in}}%
\pgfpathlineto{\pgfqpoint{0.561503in}{0.445595in}}%
\pgfpathlineto{\pgfqpoint{0.538349in}{0.452466in}}%
\pgfpathlineto{\pgfqpoint{0.522042in}{0.459394in}}%
\pgfpathlineto{\pgfqpoint{0.508540in}{0.467420in}}%
\pgfpathlineto{\pgfqpoint{0.497973in}{0.476161in}}%
\pgfpathlineto{\pgfqpoint{0.488790in}{0.486750in}}%
\pgfpathlineto{\pgfqpoint{0.481284in}{0.498948in}}%
\pgfpathlineto{\pgfqpoint{0.474590in}{0.514580in}}%
\pgfpathlineto{\pgfqpoint{0.469106in}{0.533467in}}%
\pgfpathlineto{\pgfqpoint{0.464439in}{0.557771in}}%
\pgfpathlineto{\pgfqpoint{0.460297in}{0.592289in}}%
\pgfpathlineto{\pgfqpoint{0.456856in}{0.641912in}}%
\pgfpathlineto{\pgfqpoint{0.454122in}{0.716520in}}%
\pgfpathlineto{\pgfqpoint{0.451978in}{0.843444in}}%
\pgfpathlineto{\pgfqpoint{0.450459in}{1.087380in}}%
\pgfpathlineto{\pgfqpoint{0.449596in}{1.657406in}}%
\pgfpathlineto{\pgfqpoint{0.450150in}{2.687936in}}%
\pgfpathlineto{\pgfqpoint{0.451781in}{2.839761in}}%
\pgfpathlineto{\pgfqpoint{0.453975in}{2.872003in}}%
\pgfpathlineto{\pgfqpoint{0.456339in}{2.881553in}}%
\pgfpathlineto{\pgfqpoint{0.458888in}{2.885549in}}%
\pgfpathlineto{\pgfqpoint{0.462554in}{2.888171in}}%
\pgfpathlineto{\pgfqpoint{0.471046in}{2.890205in}}%
\pgfpathlineto{\pgfqpoint{0.490597in}{2.891263in}}%
\pgfpathlineto{\pgfqpoint{0.564556in}{2.891692in}}%
\pgfpathlineto{\pgfqpoint{1.569559in}{2.891759in}}%
\pgfpathlineto{\pgfqpoint{4.784679in}{2.890785in}}%
\pgfpathlineto{\pgfqpoint{4.791005in}{2.889098in}}%
\pgfpathlineto{\pgfqpoint{4.793910in}{2.885555in}}%
\pgfpathlineto{\pgfqpoint{4.795579in}{2.878366in}}%
\pgfpathlineto{\pgfqpoint{4.796850in}{2.858513in}}%
\pgfpathlineto{\pgfqpoint{4.796850in}{2.858513in}}%
\pgfusepath{stroke}%
\end{pgfscope}%
\begin{pgfscope}%
\pgfpathrectangle{\pgfqpoint{0.448634in}{0.402556in}}{\pgfqpoint{4.350661in}{2.489204in}} %
\pgfusepath{clip}%
\pgfsetrectcap%
\pgfsetroundjoin%
\pgfsetlinewidth{1.003750pt}%
\definecolor{currentstroke}{rgb}{0.121569,0.466667,0.705882}%
\pgfsetstrokecolor{currentstroke}%
\pgfsetdash{}{0pt}%
\pgfpathmoveto{\pgfqpoint{0.448634in}{2.896245in}}%
\pgfpathlineto{\pgfqpoint{0.448593in}{0.407043in}}%
\pgfpathlineto{\pgfqpoint{0.448593in}{0.407043in}}%
\pgfusepath{stroke}%
\end{pgfscope}%
\begin{pgfscope}%
\pgfpathrectangle{\pgfqpoint{0.448634in}{0.402556in}}{\pgfqpoint{4.350661in}{2.489204in}} %
\pgfusepath{clip}%
\pgfsetrectcap%
\pgfsetroundjoin%
\pgfsetlinewidth{1.003750pt}%
\definecolor{currentstroke}{rgb}{0.121569,0.466667,0.705882}%
\pgfsetstrokecolor{currentstroke}%
\pgfsetdash{}{0pt}%
\pgfpathmoveto{\pgfqpoint{0.576853in}{1.760817in}}%
\pgfpathlineto{\pgfqpoint{0.569394in}{1.840010in}}%
\pgfpathlineto{\pgfqpoint{0.563209in}{1.929338in}}%
\pgfpathlineto{\pgfqpoint{0.558592in}{2.028764in}}%
\pgfpathlineto{\pgfqpoint{0.555985in}{2.133265in}}%
\pgfpathlineto{\pgfqpoint{0.555566in}{2.237808in}}%
\pgfpathlineto{\pgfqpoint{0.557371in}{2.337352in}}%
\pgfpathlineto{\pgfqpoint{0.561096in}{2.424366in}}%
\pgfpathlineto{\pgfqpoint{0.566403in}{2.498791in}}%
\pgfpathlineto{\pgfqpoint{0.572909in}{2.560570in}}%
\pgfpathlineto{\pgfqpoint{0.580458in}{2.612119in}}%
\pgfpathlineto{\pgfqpoint{0.589086in}{2.655816in}}%
\pgfpathlineto{\pgfqpoint{0.598406in}{2.691589in}}%
\pgfpathlineto{\pgfqpoint{0.608613in}{2.721757in}}%
\pgfpathlineto{\pgfqpoint{0.619241in}{2.746278in}}%
\pgfpathlineto{\pgfqpoint{0.630817in}{2.767339in}}%
\pgfpathlineto{\pgfqpoint{0.642975in}{2.784884in}}%
\pgfpathlineto{\pgfqpoint{0.656813in}{2.800712in}}%
\pgfpathlineto{\pgfqpoint{0.672197in}{2.814549in}}%
\pgfpathlineto{\pgfqpoint{0.688853in}{2.826301in}}%
\pgfpathlineto{\pgfqpoint{0.706461in}{2.836076in}}%
\pgfpathlineto{\pgfqpoint{0.726804in}{2.844875in}}%
\pgfpathlineto{\pgfqpoint{0.751866in}{2.853203in}}%
\pgfpathlineto{\pgfqpoint{0.781631in}{2.860547in}}%
\pgfpathlineto{\pgfqpoint{0.818168in}{2.867054in}}%
\pgfpathlineto{\pgfqpoint{0.863581in}{2.872685in}}%
\pgfpathlineto{\pgfqpoint{0.922161in}{2.877518in}}%
\pgfpathlineto{\pgfqpoint{1.000391in}{2.881567in}}%
\pgfpathlineto{\pgfqpoint{1.111294in}{2.884881in}}%
\pgfpathlineto{\pgfqpoint{1.274428in}{2.887367in}}%
\pgfpathlineto{\pgfqpoint{1.552865in}{2.889263in}}%
\pgfpathlineto{\pgfqpoint{2.107573in}{2.890457in}}%
\pgfpathlineto{\pgfqpoint{3.343161in}{2.890573in}}%
\pgfpathlineto{\pgfqpoint{4.043615in}{2.888941in}}%
\pgfpathlineto{\pgfqpoint{4.289417in}{2.886404in}}%
\pgfpathlineto{\pgfqpoint{4.413375in}{2.883093in}}%
\pgfpathlineto{\pgfqpoint{4.489424in}{2.878997in}}%
\pgfpathlineto{\pgfqpoint{4.541451in}{2.874081in}}%
\pgfpathlineto{\pgfqpoint{4.578100in}{2.868470in}}%
\pgfpathlineto{\pgfqpoint{4.605818in}{2.862092in}}%
\pgfpathlineto{\pgfqpoint{4.626725in}{2.855245in}}%
\pgfpathlineto{\pgfqpoint{4.644925in}{2.847018in}}%
\pgfpathlineto{\pgfqpoint{4.660241in}{2.837590in}}%
\pgfpathlineto{\pgfqpoint{4.672623in}{2.827468in}}%
\pgfpathlineto{\pgfqpoint{4.683751in}{2.815592in}}%
\pgfpathlineto{\pgfqpoint{4.693406in}{2.802135in}}%
\pgfpathlineto{\pgfqpoint{4.702740in}{2.785343in}}%
\pgfpathlineto{\pgfqpoint{4.711277in}{2.765194in}}%
\pgfpathlineto{\pgfqpoint{4.719482in}{2.739484in}}%
\pgfpathlineto{\pgfqpoint{4.726293in}{2.710657in}}%
\pgfpathlineto{\pgfqpoint{4.733259in}{2.671643in}}%
\pgfpathlineto{\pgfqpoint{4.739604in}{2.622396in}}%
\pgfpathlineto{\pgfqpoint{4.745236in}{2.560504in}}%
\pgfpathlineto{\pgfqpoint{4.750164in}{2.481052in}}%
\pgfpathlineto{\pgfqpoint{4.754367in}{2.376618in}}%
\pgfpathlineto{\pgfqpoint{4.757443in}{2.242249in}}%
\pgfpathlineto{\pgfqpoint{4.758977in}{2.075483in}}%
\pgfpathlineto{\pgfqpoint{4.758447in}{1.888795in}}%
\pgfpathlineto{\pgfqpoint{4.755756in}{1.707111in}}%
\pgfpathlineto{\pgfqpoint{4.750925in}{1.532957in}}%
\pgfpathlineto{\pgfqpoint{4.744785in}{1.398726in}}%
\pgfpathlineto{\pgfqpoint{4.737575in}{1.289516in}}%
\pgfpathlineto{\pgfqpoint{4.728714in}{1.190470in}}%
\pgfpathlineto{\pgfqpoint{4.719652in}{1.116521in}}%
\pgfpathlineto{\pgfqpoint{4.710036in}{1.055276in}}%
\pgfpathlineto{\pgfqpoint{4.699503in}{1.001861in}}%
\pgfpathlineto{\pgfqpoint{4.689040in}{0.958690in}}%
\pgfpathlineto{\pgfqpoint{4.677219in}{0.918600in}}%
\pgfpathlineto{\pgfqpoint{4.664034in}{0.881749in}}%
\pgfpathlineto{\pgfqpoint{4.650584in}{0.850492in}}%
\pgfpathlineto{\pgfqpoint{4.636303in}{0.822570in}}%
\pgfpathlineto{\pgfqpoint{4.620207in}{0.795974in}}%
\pgfpathlineto{\pgfqpoint{4.603640in}{0.772901in}}%
\pgfpathlineto{\pgfqpoint{4.585488in}{0.751446in}}%
\pgfpathlineto{\pgfqpoint{4.565874in}{0.731749in}}%
\pgfpathlineto{\pgfqpoint{4.544964in}{0.713879in}}%
\pgfpathlineto{\pgfqpoint{4.522958in}{0.697824in}}%
\pgfpathlineto{\pgfqpoint{4.496157in}{0.681290in}}%
\pgfpathlineto{\pgfqpoint{4.470397in}{0.667953in}}%
\pgfpathlineto{\pgfqpoint{4.439961in}{0.654509in}}%
\pgfpathlineto{\pgfqpoint{4.406841in}{0.642281in}}%
\pgfpathlineto{\pgfqpoint{4.369009in}{0.630748in}}%
\pgfpathlineto{\pgfqpoint{4.326489in}{0.620226in}}%
\pgfpathlineto{\pgfqpoint{4.279327in}{0.610949in}}%
\pgfpathlineto{\pgfqpoint{4.227576in}{0.603085in}}%
\pgfpathlineto{\pgfqpoint{4.173450in}{0.597063in}}%
\pgfpathlineto{\pgfqpoint{4.110511in}{0.592203in}}%
\pgfpathlineto{\pgfqpoint{4.047471in}{0.589537in}}%
\pgfpathlineto{\pgfqpoint{3.977867in}{0.588624in}}%
\pgfpathlineto{\pgfqpoint{3.906093in}{0.589934in}}%
\pgfpathlineto{\pgfqpoint{3.834377in}{0.593496in}}%
\pgfpathlineto{\pgfqpoint{3.767120in}{0.599067in}}%
\pgfpathlineto{\pgfqpoint{3.704364in}{0.606392in}}%
\pgfpathlineto{\pgfqpoint{3.678516in}{0.610510in}}%
\pgfpathlineto{\pgfqpoint{3.620438in}{0.620500in}}%
\pgfpathlineto{\pgfqpoint{3.586319in}{0.628207in}}%
\pgfpathlineto{\pgfqpoint{3.495240in}{0.652428in}}%
\pgfpathlineto{\pgfqpoint{3.451528in}{0.667583in}}%
\pgfpathlineto{\pgfqpoint{3.408538in}{0.685220in}}%
\pgfpathlineto{\pgfqpoint{3.374594in}{0.702001in}}%
\pgfpathlineto{\pgfqpoint{3.345407in}{0.718682in}}%
\pgfpathlineto{\pgfqpoint{3.315236in}{0.738520in}}%
\pgfpathlineto{\pgfqpoint{3.288127in}{0.759290in}}%
\pgfpathlineto{\pgfqpoint{3.264004in}{0.780551in}}%
\pgfpathlineto{\pgfqpoint{3.241208in}{0.803648in}}%
\pgfpathlineto{\pgfqpoint{3.219894in}{0.828530in}}%
\pgfpathlineto{\pgfqpoint{3.200189in}{0.855091in}}%
\pgfpathlineto{\pgfqpoint{3.182177in}{0.883182in}}%
\pgfpathlineto{\pgfqpoint{3.165906in}{0.912633in}}%
\pgfpathlineto{\pgfqpoint{3.150351in}{0.945448in}}%
\pgfpathlineto{\pgfqpoint{3.136682in}{0.979345in}}%
\pgfpathlineto{\pgfqpoint{3.124073in}{1.016460in}}%
\pgfpathlineto{\pgfqpoint{3.112834in}{1.056769in}}%
\pgfpathlineto{\pgfqpoint{3.103046in}{1.100146in}}%
\pgfpathlineto{\pgfqpoint{3.095343in}{1.144071in}}%
\pgfpathlineto{\pgfqpoint{3.089208in}{1.190837in}}%
\pgfpathlineto{\pgfqpoint{3.084595in}{1.242838in}}%
\pgfpathlineto{\pgfqpoint{3.082137in}{1.295031in}}%
\pgfpathlineto{\pgfqpoint{3.081687in}{1.349787in}}%
\pgfpathlineto{\pgfqpoint{3.083451in}{1.406998in}}%
\pgfpathlineto{\pgfqpoint{3.087181in}{1.461589in}}%
\pgfpathlineto{\pgfqpoint{3.093485in}{1.520888in}}%
\pgfpathlineto{\pgfqpoint{3.101823in}{1.577334in}}%
\pgfpathlineto{\pgfqpoint{3.111930in}{1.630856in}}%
\pgfpathlineto{\pgfqpoint{3.124690in}{1.686208in}}%
\pgfpathlineto{\pgfqpoint{3.139178in}{1.738395in}}%
\pgfpathlineto{\pgfqpoint{3.155145in}{1.787366in}}%
\pgfpathlineto{\pgfqpoint{3.172353in}{1.833085in}}%
\pgfpathlineto{\pgfqpoint{3.191618in}{1.877716in}}%
\pgfpathlineto{\pgfqpoint{3.214026in}{1.923261in}}%
\pgfpathlineto{\pgfqpoint{3.236214in}{1.963157in}}%
\pgfpathlineto{\pgfqpoint{3.260178in}{2.001684in}}%
\pgfpathlineto{\pgfqpoint{3.285814in}{2.038776in}}%
\pgfpathlineto{\pgfqpoint{3.314415in}{2.076285in}}%
\pgfpathlineto{\pgfqpoint{3.348944in}{2.117711in}}%
\pgfpathlineto{\pgfqpoint{3.417133in}{2.198022in}}%
\pgfpathlineto{\pgfqpoint{3.426053in}{2.212128in}}%
\pgfpathlineto{\pgfqpoint{3.430798in}{2.223297in}}%
\pgfpathlineto{\pgfqpoint{3.432034in}{2.230603in}}%
\pgfpathlineto{\pgfqpoint{3.430773in}{2.237856in}}%
\pgfpathlineto{\pgfqpoint{3.426621in}{2.243526in}}%
\pgfpathlineto{\pgfqpoint{3.420908in}{2.247084in}}%
\pgfpathlineto{\pgfqpoint{3.412501in}{2.249583in}}%
\pgfpathlineto{\pgfqpoint{3.399499in}{2.250689in}}%
\pgfpathlineto{\pgfqpoint{3.384305in}{2.249671in}}%
\pgfpathlineto{\pgfqpoint{3.364985in}{2.246098in}}%
\pgfpathlineto{\pgfqpoint{3.341804in}{2.239342in}}%
\pgfpathlineto{\pgfqpoint{3.317109in}{2.229682in}}%
\pgfpathlineto{\pgfqpoint{3.291104in}{2.216986in}}%
\pgfpathlineto{\pgfqpoint{3.265928in}{2.202261in}}%
\pgfpathlineto{\pgfqpoint{3.239805in}{2.184361in}}%
\pgfpathlineto{\pgfqpoint{3.214775in}{2.164519in}}%
\pgfpathlineto{\pgfqpoint{3.190900in}{2.142893in}}%
\pgfpathlineto{\pgfqpoint{3.166657in}{2.117912in}}%
\pgfpathlineto{\pgfqpoint{3.143835in}{2.091233in}}%
\pgfpathlineto{\pgfqpoint{3.121079in}{2.061107in}}%
\pgfpathlineto{\pgfqpoint{3.099952in}{2.029463in}}%
\pgfpathlineto{\pgfqpoint{3.079251in}{1.994406in}}%
\pgfpathlineto{\pgfqpoint{3.059218in}{1.955915in}}%
\pgfpathlineto{\pgfqpoint{3.040058in}{1.914015in}}%
\pgfpathlineto{\pgfqpoint{3.022809in}{1.871041in}}%
\pgfpathlineto{\pgfqpoint{3.005790in}{1.822536in}}%
\pgfpathlineto{\pgfqpoint{2.990067in}{1.770819in}}%
\pgfpathlineto{\pgfqpoint{2.975708in}{1.715979in}}%
\pgfpathlineto{\pgfqpoint{2.962284in}{1.655680in}}%
\pgfpathlineto{\pgfqpoint{2.950496in}{1.592386in}}%
\pgfpathlineto{\pgfqpoint{2.940383in}{1.526185in}}%
\pgfpathlineto{\pgfqpoint{2.931745in}{1.454681in}}%
\pgfpathlineto{\pgfqpoint{2.925082in}{1.380399in}}%
\pgfpathlineto{\pgfqpoint{2.920647in}{1.305899in}}%
\pgfpathlineto{\pgfqpoint{2.918444in}{1.231270in}}%
\pgfpathlineto{\pgfqpoint{2.918545in}{1.159087in}}%
\pgfpathlineto{\pgfqpoint{2.920787in}{1.091931in}}%
\pgfpathlineto{\pgfqpoint{2.925177in}{1.027412in}}%
\pgfpathlineto{\pgfqpoint{2.931192in}{0.970580in}}%
\pgfpathlineto{\pgfqpoint{2.938760in}{0.919034in}}%
\pgfpathlineto{\pgfqpoint{2.947651in}{0.872852in}}%
\pgfpathlineto{\pgfqpoint{2.958213in}{0.829714in}}%
\pgfpathlineto{\pgfqpoint{2.969670in}{0.792114in}}%
\pgfpathlineto{\pgfqpoint{2.982463in}{0.757773in}}%
\pgfpathlineto{\pgfqpoint{2.996425in}{0.726812in}}%
\pgfpathlineto{\pgfqpoint{3.011299in}{0.699300in}}%
\pgfpathlineto{\pgfqpoint{3.026739in}{0.675225in}}%
\pgfpathlineto{\pgfqpoint{3.043828in}{0.652656in}}%
\pgfpathlineto{\pgfqpoint{3.062495in}{0.631788in}}%
\pgfpathlineto{\pgfqpoint{3.082602in}{0.612753in}}%
\pgfpathlineto{\pgfqpoint{3.103961in}{0.595592in}}%
\pgfpathlineto{\pgfqpoint{3.128268in}{0.579069in}}%
\pgfpathlineto{\pgfqpoint{3.153537in}{0.564554in}}%
\pgfpathlineto{\pgfqpoint{3.181571in}{0.550952in}}%
\pgfpathlineto{\pgfqpoint{3.214371in}{0.537647in}}%
\pgfpathlineto{\pgfqpoint{3.249846in}{0.525712in}}%
\pgfpathlineto{\pgfqpoint{3.290011in}{0.514571in}}%
\pgfpathlineto{\pgfqpoint{3.334820in}{0.504423in}}%
\pgfpathlineto{\pgfqpoint{3.386372in}{0.494999in}}%
\pgfpathlineto{\pgfqpoint{3.446798in}{0.486257in}}%
\pgfpathlineto{\pgfqpoint{3.518243in}{0.478282in}}%
\pgfpathlineto{\pgfqpoint{3.600685in}{0.471409in}}%
\pgfpathlineto{\pgfqpoint{3.696268in}{0.465713in}}%
\pgfpathlineto{\pgfqpoint{3.807144in}{0.461369in}}%
\pgfpathlineto{\pgfqpoint{3.933291in}{0.458719in}}%
\pgfpathlineto{\pgfqpoint{4.063808in}{0.458211in}}%
\pgfpathlineto{\pgfqpoint{4.187792in}{0.459914in}}%
\pgfpathlineto{\pgfqpoint{4.294335in}{0.463521in}}%
\pgfpathlineto{\pgfqpoint{4.381234in}{0.468574in}}%
\pgfpathlineto{\pgfqpoint{4.450636in}{0.474701in}}%
\pgfpathlineto{\pgfqpoint{4.506850in}{0.481799in}}%
\pgfpathlineto{\pgfqpoint{4.552009in}{0.489658in}}%
\pgfpathlineto{\pgfqpoint{4.588239in}{0.498115in}}%
\pgfpathlineto{\pgfqpoint{4.617656in}{0.507110in}}%
\pgfpathlineto{\pgfqpoint{4.642328in}{0.516843in}}%
\pgfpathlineto{\pgfqpoint{4.664194in}{0.527940in}}%
\pgfpathlineto{\pgfqpoint{4.681238in}{0.538945in}}%
\pgfpathlineto{\pgfqpoint{4.697164in}{0.551953in}}%
\pgfpathlineto{\pgfqpoint{4.710076in}{0.565289in}}%
\pgfpathlineto{\pgfqpoint{4.721578in}{0.580218in}}%
\pgfpathlineto{\pgfqpoint{4.731557in}{0.596521in}}%
\pgfpathlineto{\pgfqpoint{4.741000in}{0.616134in}}%
\pgfpathlineto{\pgfqpoint{4.749521in}{0.639027in}}%
\pgfpathlineto{\pgfqpoint{4.757522in}{0.667450in}}%
\pgfpathlineto{\pgfqpoint{4.764572in}{0.701345in}}%
\pgfpathlineto{\pgfqpoint{4.770840in}{0.743043in}}%
\pgfpathlineto{\pgfqpoint{4.776327in}{0.794934in}}%
\pgfpathlineto{\pgfqpoint{4.781278in}{0.864398in}}%
\pgfpathlineto{\pgfqpoint{4.785468in}{0.956371in}}%
\pgfpathlineto{\pgfqpoint{4.789000in}{1.085745in}}%
\pgfpathlineto{\pgfqpoint{4.791852in}{1.277385in}}%
\pgfpathlineto{\pgfqpoint{4.793959in}{1.581057in}}%
\pgfpathlineto{\pgfqpoint{4.794962in}{2.071429in}}%
\pgfpathlineto{\pgfqpoint{4.793967in}{2.559311in}}%
\pgfpathlineto{\pgfqpoint{4.791733in}{2.745981in}}%
\pgfpathlineto{\pgfqpoint{4.788955in}{2.818091in}}%
\pgfpathlineto{\pgfqpoint{4.785731in}{2.850227in}}%
\pgfpathlineto{\pgfqpoint{4.781879in}{2.867057in}}%
\pgfpathlineto{\pgfqpoint{4.777744in}{2.875780in}}%
\pgfpathlineto{\pgfqpoint{4.773097in}{2.880982in}}%
\pgfpathlineto{\pgfqpoint{4.767363in}{2.884504in}}%
\pgfpathlineto{\pgfqpoint{4.756853in}{2.887622in}}%
\pgfpathlineto{\pgfqpoint{4.739548in}{2.889639in}}%
\pgfpathlineto{\pgfqpoint{4.704762in}{2.890882in}}%
\pgfpathlineto{\pgfqpoint{4.602524in}{2.891538in}}%
\pgfpathlineto{\pgfqpoint{3.952100in}{2.891742in}}%
\pgfpathlineto{\pgfqpoint{0.617321in}{2.890753in}}%
\pgfpathlineto{\pgfqpoint{0.549910in}{2.888858in}}%
\pgfpathlineto{\pgfqpoint{0.521735in}{2.886179in}}%
\pgfpathlineto{\pgfqpoint{0.504666in}{2.882389in}}%
\pgfpathlineto{\pgfqpoint{0.494501in}{2.878011in}}%
\pgfpathlineto{\pgfqpoint{0.487180in}{2.872667in}}%
\pgfpathlineto{\pgfqpoint{0.481152in}{2.865519in}}%
\pgfpathlineto{\pgfqpoint{0.475664in}{2.854804in}}%
\pgfpathlineto{\pgfqpoint{0.471318in}{2.840737in}}%
\pgfpathlineto{\pgfqpoint{0.467301in}{2.818823in}}%
\pgfpathlineto{\pgfqpoint{0.463927in}{2.786700in}}%
\pgfpathlineto{\pgfqpoint{0.460918in}{2.734544in}}%
\pgfpathlineto{\pgfqpoint{0.458363in}{2.647473in}}%
\pgfpathlineto{\pgfqpoint{0.456575in}{2.523031in}}%
\pgfpathlineto{\pgfqpoint{0.456575in}{2.523031in}}%
\pgfusepath{stroke}%
\end{pgfscope}%
\begin{pgfscope}%
\pgfpathrectangle{\pgfqpoint{0.448634in}{0.402556in}}{\pgfqpoint{4.350661in}{2.489204in}} %
\pgfusepath{clip}%
\pgfsetrectcap%
\pgfsetroundjoin%
\pgfsetlinewidth{1.003750pt}%
\definecolor{currentstroke}{rgb}{0.121569,0.466667,0.705882}%
\pgfsetstrokecolor{currentstroke}%
\pgfsetdash{}{0pt}%
\pgfpathmoveto{\pgfqpoint{4.798840in}{2.852369in}}%
\pgfpathlineto{\pgfqpoint{4.797564in}{2.889610in}}%
\pgfpathlineto{\pgfqpoint{4.796215in}{2.891483in}}%
\pgfpathlineto{\pgfqpoint{4.787551in}{2.891760in}}%
\pgfpathlineto{\pgfqpoint{0.452128in}{2.891659in}}%
\pgfpathlineto{\pgfqpoint{0.450530in}{2.890082in}}%
\pgfpathlineto{\pgfqpoint{0.449454in}{2.882763in}}%
\pgfpathlineto{\pgfqpoint{0.448970in}{2.845432in}}%
\pgfpathlineto{\pgfqpoint{0.448743in}{2.494454in}}%
\pgfpathlineto{\pgfqpoint{0.449624in}{0.615107in}}%
\pgfpathlineto{\pgfqpoint{0.451433in}{0.510586in}}%
\pgfpathlineto{\pgfqpoint{0.453993in}{0.473374in}}%
\pgfpathlineto{\pgfqpoint{0.457406in}{0.453868in}}%
\pgfpathlineto{\pgfqpoint{0.461540in}{0.442384in}}%
\pgfpathlineto{\pgfqpoint{0.466739in}{0.434437in}}%
\pgfpathlineto{\pgfqpoint{0.473595in}{0.428350in}}%
\pgfpathlineto{\pgfqpoint{0.483492in}{0.423244in}}%
\pgfpathlineto{\pgfqpoint{0.491854in}{0.420501in}}%
\pgfpathlineto{\pgfqpoint{0.491854in}{0.420501in}}%
\pgfusepath{stroke}%
\end{pgfscope}%
\begin{pgfscope}%
\pgfpathrectangle{\pgfqpoint{0.448634in}{0.402556in}}{\pgfqpoint{4.350661in}{2.489204in}} %
\pgfusepath{clip}%
\pgfsetrectcap%
\pgfsetroundjoin%
\pgfsetlinewidth{1.003750pt}%
\definecolor{currentstroke}{rgb}{0.121569,0.466667,0.705882}%
\pgfsetstrokecolor{currentstroke}%
\pgfsetdash{}{0pt}%
\pgfpathmoveto{\pgfqpoint{0.456424in}{1.370137in}}%
\pgfpathlineto{\pgfqpoint{0.459610in}{1.118755in}}%
\pgfpathlineto{\pgfqpoint{0.463695in}{0.962007in}}%
\pgfpathlineto{\pgfqpoint{0.468519in}{0.857610in}}%
\pgfpathlineto{\pgfqpoint{0.474082in}{0.783210in}}%
\pgfpathlineto{\pgfqpoint{0.480226in}{0.728906in}}%
\pgfpathlineto{\pgfqpoint{0.486970in}{0.687306in}}%
\pgfpathlineto{\pgfqpoint{0.494537in}{0.653558in}}%
\pgfpathlineto{\pgfqpoint{0.503107in}{0.625355in}}%
\pgfpathlineto{\pgfqpoint{0.512193in}{0.602750in}}%
\pgfpathlineto{\pgfqpoint{0.522200in}{0.583508in}}%
\pgfpathlineto{\pgfqpoint{0.534108in}{0.565743in}}%
\pgfpathlineto{\pgfqpoint{0.546263in}{0.551507in}}%
\pgfpathlineto{\pgfqpoint{0.559728in}{0.538907in}}%
\pgfpathlineto{\pgfqpoint{0.576129in}{0.526693in}}%
\pgfpathlineto{\pgfqpoint{0.595483in}{0.515351in}}%
\pgfpathlineto{\pgfqpoint{0.617681in}{0.505147in}}%
\pgfpathlineto{\pgfqpoint{0.642568in}{0.496153in}}%
\pgfpathlineto{\pgfqpoint{0.672126in}{0.487778in}}%
\pgfpathlineto{\pgfqpoint{0.708443in}{0.479824in}}%
\pgfpathlineto{\pgfqpoint{0.753649in}{0.472325in}}%
\pgfpathlineto{\pgfqpoint{0.807717in}{0.465660in}}%
\pgfpathlineto{\pgfqpoint{0.877116in}{0.459475in}}%
\pgfpathlineto{\pgfqpoint{0.961828in}{0.454230in}}%
\pgfpathlineto{\pgfqpoint{1.068351in}{0.449916in}}%
\pgfpathlineto{\pgfqpoint{1.201018in}{0.446839in}}%
\pgfpathlineto{\pgfqpoint{1.357637in}{0.445481in}}%
\pgfpathlineto{\pgfqpoint{1.525135in}{0.446232in}}%
\pgfpathlineto{\pgfqpoint{1.686088in}{0.449142in}}%
\pgfpathlineto{\pgfqpoint{1.823074in}{0.453747in}}%
\pgfpathlineto{\pgfqpoint{1.938245in}{0.459764in}}%
\pgfpathlineto{\pgfqpoint{2.031582in}{0.466759in}}%
\pgfpathlineto{\pgfqpoint{2.109580in}{0.474745in}}%
\pgfpathlineto{\pgfqpoint{2.174384in}{0.483535in}}%
\pgfpathlineto{\pgfqpoint{2.228139in}{0.492940in}}%
\pgfpathlineto{\pgfqpoint{2.275119in}{0.503356in}}%
\pgfpathlineto{\pgfqpoint{2.315282in}{0.514501in}}%
\pgfpathlineto{\pgfqpoint{2.350698in}{0.526659in}}%
\pgfpathlineto{\pgfqpoint{2.381320in}{0.539536in}}%
\pgfpathlineto{\pgfqpoint{2.407164in}{0.552659in}}%
\pgfpathlineto{\pgfqpoint{2.430226in}{0.566639in}}%
\pgfpathlineto{\pgfqpoint{2.452282in}{0.582602in}}%
\pgfpathlineto{\pgfqpoint{2.471391in}{0.599069in}}%
\pgfpathlineto{\pgfqpoint{2.489240in}{0.617293in}}%
\pgfpathlineto{\pgfqpoint{2.505678in}{0.637180in}}%
\pgfpathlineto{\pgfqpoint{2.520620in}{0.658557in}}%
\pgfpathlineto{\pgfqpoint{2.535213in}{0.683314in}}%
\pgfpathlineto{\pgfqpoint{2.549115in}{0.711484in}}%
\pgfpathlineto{\pgfqpoint{2.562091in}{0.743004in}}%
\pgfpathlineto{\pgfqpoint{2.574020in}{0.777751in}}%
\pgfpathlineto{\pgfqpoint{2.585502in}{0.817970in}}%
\pgfpathlineto{\pgfqpoint{2.596809in}{0.866038in}}%
\pgfpathlineto{\pgfqpoint{2.607562in}{0.921948in}}%
\pgfpathlineto{\pgfqpoint{2.617925in}{0.988098in}}%
\pgfpathlineto{\pgfqpoint{2.627958in}{1.066918in}}%
\pgfpathlineto{\pgfqpoint{2.637941in}{1.163320in}}%
\pgfpathlineto{\pgfqpoint{2.648424in}{1.287199in}}%
\pgfpathlineto{\pgfqpoint{2.660103in}{1.453438in}}%
\pgfpathlineto{\pgfqpoint{2.674773in}{1.696801in}}%
\pgfpathlineto{\pgfqpoint{2.687716in}{1.945279in}}%
\pgfpathlineto{\pgfqpoint{2.692670in}{2.079573in}}%
\pgfpathlineto{\pgfqpoint{2.693829in}{2.166682in}}%
\pgfpathlineto{\pgfqpoint{2.692565in}{2.233870in}}%
\pgfpathlineto{\pgfqpoint{2.689436in}{2.286015in}}%
\pgfpathlineto{\pgfqpoint{2.684859in}{2.327999in}}%
\pgfpathlineto{\pgfqpoint{2.678725in}{2.364664in}}%
\pgfpathlineto{\pgfqpoint{2.671356in}{2.395897in}}%
\pgfpathlineto{\pgfqpoint{2.662489in}{2.423981in}}%
\pgfpathlineto{\pgfqpoint{2.652361in}{2.448778in}}%
\pgfpathlineto{\pgfqpoint{2.641365in}{2.470245in}}%
\pgfpathlineto{\pgfqpoint{2.628643in}{2.490425in}}%
\pgfpathlineto{\pgfqpoint{2.614279in}{2.509106in}}%
\pgfpathlineto{\pgfqpoint{2.598443in}{2.526159in}}%
\pgfpathlineto{\pgfqpoint{2.579590in}{2.543005in}}%
\pgfpathlineto{\pgfqpoint{2.559532in}{2.557923in}}%
\pgfpathlineto{\pgfqpoint{2.536602in}{2.572183in}}%
\pgfpathlineto{\pgfqpoint{2.510850in}{2.585538in}}%
\pgfpathlineto{\pgfqpoint{2.482360in}{2.597837in}}%
\pgfpathlineto{\pgfqpoint{2.449134in}{2.609683in}}%
\pgfpathlineto{\pgfqpoint{2.411184in}{2.620696in}}%
\pgfpathlineto{\pgfqpoint{2.368552in}{2.630606in}}%
\pgfpathlineto{\pgfqpoint{2.321294in}{2.639221in}}%
\pgfpathlineto{\pgfqpoint{2.269467in}{2.646399in}}%
\pgfpathlineto{\pgfqpoint{2.210954in}{2.652193in}}%
\pgfpathlineto{\pgfqpoint{2.147967in}{2.656153in}}%
\pgfpathlineto{\pgfqpoint{2.080556in}{2.658135in}}%
\pgfpathlineto{\pgfqpoint{2.010948in}{2.657971in}}%
\pgfpathlineto{\pgfqpoint{1.939195in}{2.655572in}}%
\pgfpathlineto{\pgfqpoint{1.867527in}{2.650913in}}%
\pgfpathlineto{\pgfqpoint{1.798171in}{2.644140in}}%
\pgfpathlineto{\pgfqpoint{1.733341in}{2.635606in}}%
\pgfpathlineto{\pgfqpoint{1.673075in}{2.625521in}}%
\pgfpathlineto{\pgfqpoint{1.615274in}{2.613610in}}%
\pgfpathlineto{\pgfqpoint{1.562133in}{2.600402in}}%
\pgfpathlineto{\pgfqpoint{1.513681in}{2.586139in}}%
\pgfpathlineto{\pgfqpoint{1.467862in}{2.570344in}}%
\pgfpathlineto{\pgfqpoint{1.426794in}{2.553923in}}%
\pgfpathlineto{\pgfqpoint{1.388447in}{2.536289in}}%
\pgfpathlineto{\pgfqpoint{1.352878in}{2.517566in}}%
\pgfpathlineto{\pgfqpoint{1.320128in}{2.497922in}}%
\pgfpathlineto{\pgfqpoint{1.288379in}{2.476236in}}%
\pgfpathlineto{\pgfqpoint{1.259592in}{2.453861in}}%
\pgfpathlineto{\pgfqpoint{1.232050in}{2.429520in}}%
\pgfpathlineto{\pgfqpoint{1.207527in}{2.404898in}}%
\pgfpathlineto{\pgfqpoint{1.184409in}{2.378557in}}%
\pgfpathlineto{\pgfqpoint{1.162828in}{2.350561in}}%
\pgfpathlineto{\pgfqpoint{1.142891in}{2.321011in}}%
\pgfpathlineto{\pgfqpoint{1.124675in}{2.290041in}}%
\pgfpathlineto{\pgfqpoint{1.108225in}{2.257802in}}%
\pgfpathlineto{\pgfqpoint{1.092639in}{2.222199in}}%
\pgfpathlineto{\pgfqpoint{1.079059in}{2.185535in}}%
\pgfpathlineto{\pgfqpoint{1.067443in}{2.147998in}}%
\pgfpathlineto{\pgfqpoint{1.057187in}{2.107348in}}%
\pgfpathlineto{\pgfqpoint{1.049004in}{2.066086in}}%
\pgfpathlineto{\pgfqpoint{1.042513in}{2.021906in}}%
\pgfpathlineto{\pgfqpoint{1.038177in}{1.977382in}}%
\pgfpathlineto{\pgfqpoint{1.035866in}{1.930167in}}%
\pgfpathlineto{\pgfqpoint{1.035826in}{1.882878in}}%
\pgfpathlineto{\pgfqpoint{1.038031in}{1.835656in}}%
\pgfpathlineto{\pgfqpoint{1.042474in}{1.788641in}}%
\pgfpathlineto{\pgfqpoint{1.049176in}{1.741979in}}%
\pgfpathlineto{\pgfqpoint{1.057644in}{1.698239in}}%
\pgfpathlineto{\pgfqpoint{1.068221in}{1.655105in}}%
\pgfpathlineto{\pgfqpoint{1.080962in}{1.612745in}}%
\pgfpathlineto{\pgfqpoint{1.095031in}{1.573617in}}%
\pgfpathlineto{\pgfqpoint{1.111115in}{1.535520in}}%
\pgfpathlineto{\pgfqpoint{1.128118in}{1.500775in}}%
\pgfpathlineto{\pgfqpoint{1.146930in}{1.467274in}}%
\pgfpathlineto{\pgfqpoint{1.167531in}{1.435181in}}%
\pgfpathlineto{\pgfqpoint{1.189874in}{1.404652in}}%
\pgfpathlineto{\pgfqpoint{1.213884in}{1.375828in}}%
\pgfpathlineto{\pgfqpoint{1.237817in}{1.350457in}}%
\pgfpathlineto{\pgfqpoint{1.264748in}{1.325237in}}%
\pgfpathlineto{\pgfqpoint{1.292991in}{1.301972in}}%
\pgfpathlineto{\pgfqpoint{1.322398in}{1.280678in}}%
\pgfpathlineto{\pgfqpoint{1.352820in}{1.261340in}}%
\pgfpathlineto{\pgfqpoint{1.386095in}{1.242889in}}%
\pgfpathlineto{\pgfqpoint{1.420190in}{1.226516in}}%
\pgfpathlineto{\pgfqpoint{1.457024in}{1.211329in}}%
\pgfpathlineto{\pgfqpoint{1.496554in}{1.197536in}}%
\pgfpathlineto{\pgfqpoint{1.538719in}{1.185287in}}%
\pgfpathlineto{\pgfqpoint{1.583441in}{1.174641in}}%
\pgfpathlineto{\pgfqpoint{1.634929in}{1.164775in}}%
\pgfpathlineto{\pgfqpoint{1.706063in}{1.153745in}}%
\pgfpathlineto{\pgfqpoint{1.768492in}{1.143417in}}%
\pgfpathlineto{\pgfqpoint{1.796122in}{1.136567in}}%
\pgfpathlineto{\pgfqpoint{1.812683in}{1.130481in}}%
\pgfpathlineto{\pgfqpoint{1.824471in}{1.124102in}}%
\pgfpathlineto{\pgfqpoint{1.833209in}{1.116741in}}%
\pgfpathlineto{\pgfqpoint{1.838498in}{1.108890in}}%
\pgfpathlineto{\pgfqpoint{1.840588in}{1.101849in}}%
\pgfpathlineto{\pgfqpoint{1.840619in}{1.094412in}}%
\pgfpathlineto{\pgfqpoint{1.837931in}{1.084986in}}%
\pgfpathlineto{\pgfqpoint{1.833246in}{1.076615in}}%
\pgfpathlineto{\pgfqpoint{1.825819in}{1.067542in}}%
\pgfpathlineto{\pgfqpoint{1.813813in}{1.056850in}}%
\pgfpathlineto{\pgfqpoint{1.798819in}{1.046763in}}%
\pgfpathlineto{\pgfqpoint{1.781016in}{1.037462in}}%
\pgfpathlineto{\pgfqpoint{1.758447in}{1.028391in}}%
\pgfpathlineto{\pgfqpoint{1.733203in}{1.020815in}}%
\pgfpathlineto{\pgfqpoint{1.705410in}{1.014872in}}%
\pgfpathlineto{\pgfqpoint{1.675178in}{1.010714in}}%
\pgfpathlineto{\pgfqpoint{1.642610in}{1.008507in}}%
\pgfpathlineto{\pgfqpoint{1.607809in}{1.008432in}}%
\pgfpathlineto{\pgfqpoint{1.570886in}{1.010691in}}%
\pgfpathlineto{\pgfqpoint{1.534118in}{1.015181in}}%
\pgfpathlineto{\pgfqpoint{1.495454in}{1.022233in}}%
\pgfpathlineto{\pgfqpoint{1.457161in}{1.031563in}}%
\pgfpathlineto{\pgfqpoint{1.419337in}{1.043132in}}%
\pgfpathlineto{\pgfqpoint{1.382089in}{1.056929in}}%
\pgfpathlineto{\pgfqpoint{1.347544in}{1.072019in}}%
\pgfpathlineto{\pgfqpoint{1.313727in}{1.089133in}}%
\pgfpathlineto{\pgfqpoint{1.280762in}{1.108299in}}%
\pgfpathlineto{\pgfqpoint{1.248782in}{1.129536in}}%
\pgfpathlineto{\pgfqpoint{1.219708in}{1.151422in}}%
\pgfpathlineto{\pgfqpoint{1.191752in}{1.175138in}}%
\pgfpathlineto{\pgfqpoint{1.165031in}{1.200649in}}%
\pgfpathlineto{\pgfqpoint{1.139653in}{1.227898in}}%
\pgfpathlineto{\pgfqpoint{1.115714in}{1.256800in}}%
\pgfpathlineto{\pgfqpoint{1.093288in}{1.287251in}}%
\pgfpathlineto{\pgfqpoint{1.071178in}{1.321163in}}%
\pgfpathlineto{\pgfqpoint{1.050868in}{1.356520in}}%
\pgfpathlineto{\pgfqpoint{1.032365in}{1.393152in}}%
\pgfpathlineto{\pgfqpoint{1.014718in}{1.433142in}}%
\pgfpathlineto{\pgfqpoint{0.999024in}{1.474185in}}%
\pgfpathlineto{\pgfqpoint{0.984506in}{1.518461in}}%
\pgfpathlineto{\pgfqpoint{0.972010in}{1.563537in}}%
\pgfpathlineto{\pgfqpoint{0.960944in}{1.611678in}}%
\pgfpathlineto{\pgfqpoint{0.951530in}{1.662824in}}%
\pgfpathlineto{\pgfqpoint{0.944286in}{1.714431in}}%
\pgfpathlineto{\pgfqpoint{0.938950in}{1.768847in}}%
\pgfpathlineto{\pgfqpoint{0.935870in}{1.823491in}}%
\pgfpathlineto{\pgfqpoint{0.935034in}{1.878240in}}%
\pgfpathlineto{\pgfqpoint{0.936466in}{1.932973in}}%
\pgfpathlineto{\pgfqpoint{0.940005in}{1.985084in}}%
\pgfpathlineto{\pgfqpoint{0.945759in}{2.036935in}}%
\pgfpathlineto{\pgfqpoint{0.953410in}{2.085938in}}%
\pgfpathlineto{\pgfqpoint{0.962764in}{2.132000in}}%
\pgfpathlineto{\pgfqpoint{0.974287in}{2.177414in}}%
\pgfpathlineto{\pgfqpoint{0.987332in}{2.219653in}}%
\pgfpathlineto{\pgfqpoint{1.001667in}{2.258654in}}%
\pgfpathlineto{\pgfqpoint{1.018051in}{2.296583in}}%
\pgfpathlineto{\pgfqpoint{1.035401in}{2.331101in}}%
\pgfpathlineto{\pgfqpoint{1.054650in}{2.364275in}}%
\pgfpathlineto{\pgfqpoint{1.074406in}{2.393984in}}%
\pgfpathlineto{\pgfqpoint{1.095771in}{2.422197in}}%
\pgfpathlineto{\pgfqpoint{1.118662in}{2.448797in}}%
\pgfpathlineto{\pgfqpoint{1.142967in}{2.473701in}}%
\pgfpathlineto{\pgfqpoint{1.168550in}{2.496867in}}%
\pgfpathlineto{\pgfqpoint{1.197085in}{2.519662in}}%
\pgfpathlineto{\pgfqpoint{1.226727in}{2.540526in}}%
\pgfpathlineto{\pgfqpoint{1.259242in}{2.560673in}}%
\pgfpathlineto{\pgfqpoint{1.294612in}{2.579881in}}%
\pgfpathlineto{\pgfqpoint{1.332792in}{2.597982in}}%
\pgfpathlineto{\pgfqpoint{1.373719in}{2.614859in}}%
\pgfpathlineto{\pgfqpoint{1.417319in}{2.630445in}}%
\pgfpathlineto{\pgfqpoint{1.465632in}{2.645312in}}%
\pgfpathlineto{\pgfqpoint{1.518640in}{2.659204in}}%
\pgfpathlineto{\pgfqpoint{1.576309in}{2.671929in}}%
\pgfpathlineto{\pgfqpoint{1.638597in}{2.683344in}}%
\pgfpathlineto{\pgfqpoint{1.705462in}{2.693343in}}%
\pgfpathlineto{\pgfqpoint{1.779027in}{2.702064in}}%
\pgfpathlineto{\pgfqpoint{1.857097in}{2.709077in}}%
\pgfpathlineto{\pgfqpoint{1.939633in}{2.714280in}}%
\pgfpathlineto{\pgfqpoint{2.026598in}{2.717513in}}%
\pgfpathlineto{\pgfqpoint{2.113605in}{2.718523in}}%
\pgfpathlineto{\pgfqpoint{2.198435in}{2.717303in}}%
\pgfpathlineto{\pgfqpoint{2.278866in}{2.713929in}}%
\pgfpathlineto{\pgfqpoint{2.352678in}{2.708598in}}%
\pgfpathlineto{\pgfqpoint{2.417657in}{2.701709in}}%
\pgfpathlineto{\pgfqpoint{2.473770in}{2.693630in}}%
\pgfpathlineto{\pgfqpoint{2.523140in}{2.684368in}}%
\pgfpathlineto{\pgfqpoint{2.565726in}{2.674202in}}%
\pgfpathlineto{\pgfqpoint{2.601510in}{2.663544in}}%
\pgfpathlineto{\pgfqpoint{2.632577in}{2.652142in}}%
\pgfpathlineto{\pgfqpoint{2.658899in}{2.640331in}}%
\pgfpathlineto{\pgfqpoint{2.682438in}{2.627436in}}%
\pgfpathlineto{\pgfqpoint{2.703062in}{2.613571in}}%
\pgfpathlineto{\pgfqpoint{2.720674in}{2.598978in}}%
\pgfpathlineto{\pgfqpoint{2.735263in}{2.584053in}}%
\pgfpathlineto{\pgfqpoint{2.748320in}{2.567377in}}%
\pgfpathlineto{\pgfqpoint{2.759553in}{2.549046in}}%
\pgfpathlineto{\pgfqpoint{2.768788in}{2.529306in}}%
\pgfpathlineto{\pgfqpoint{2.776017in}{2.508498in}}%
\pgfpathlineto{\pgfqpoint{2.781884in}{2.484540in}}%
\pgfpathlineto{\pgfqpoint{2.786102in}{2.457597in}}%
\pgfpathlineto{\pgfqpoint{2.788720in}{2.425384in}}%
\pgfpathlineto{\pgfqpoint{2.789427in}{2.388061in}}%
\pgfpathlineto{\pgfqpoint{2.787962in}{2.340801in}}%
\pgfpathlineto{\pgfqpoint{2.783672in}{2.278768in}}%
\pgfpathlineto{\pgfqpoint{2.774289in}{2.179783in}}%
\pgfpathlineto{\pgfqpoint{2.743611in}{1.868119in}}%
\pgfpathlineto{\pgfqpoint{2.730112in}{1.702060in}}%
\pgfpathlineto{\pgfqpoint{2.717287in}{1.515949in}}%
\pgfpathlineto{\pgfqpoint{2.702602in}{1.267597in}}%
\pgfpathlineto{\pgfqpoint{2.684434in}{0.964630in}}%
\pgfpathlineto{\pgfqpoint{2.675374in}{0.850600in}}%
\pgfpathlineto{\pgfqpoint{2.667030in}{0.771523in}}%
\pgfpathlineto{\pgfqpoint{2.658752in}{0.712543in}}%
\pgfpathlineto{\pgfqpoint{2.650176in}{0.666284in}}%
\pgfpathlineto{\pgfqpoint{2.640820in}{0.627931in}}%
\pgfpathlineto{\pgfqpoint{2.631145in}{0.597534in}}%
\pgfpathlineto{\pgfqpoint{2.621004in}{0.572745in}}%
\pgfpathlineto{\pgfqpoint{2.609856in}{0.551383in}}%
\pgfpathlineto{\pgfqpoint{2.598042in}{0.533534in}}%
\pgfpathlineto{\pgfqpoint{2.584496in}{0.517378in}}%
\pgfpathlineto{\pgfqpoint{2.571109in}{0.504669in}}%
\pgfpathlineto{\pgfqpoint{2.554789in}{0.492313in}}%
\pgfpathlineto{\pgfqpoint{2.537457in}{0.481914in}}%
\pgfpathlineto{\pgfqpoint{2.517374in}{0.472367in}}%
\pgfpathlineto{\pgfqpoint{2.492542in}{0.463178in}}%
\pgfpathlineto{\pgfqpoint{2.462979in}{0.454833in}}%
\pgfpathlineto{\pgfqpoint{2.428766in}{0.447542in}}%
\pgfpathlineto{\pgfqpoint{2.385671in}{0.440735in}}%
\pgfpathlineto{\pgfqpoint{2.331557in}{0.434581in}}%
\pgfpathlineto{\pgfqpoint{2.262115in}{0.429077in}}%
\pgfpathlineto{\pgfqpoint{2.170851in}{0.424236in}}%
\pgfpathlineto{\pgfqpoint{2.049086in}{0.420134in}}%
\pgfpathlineto{\pgfqpoint{1.879436in}{0.416783in}}%
\pgfpathlineto{\pgfqpoint{1.640159in}{0.414418in}}%
\pgfpathlineto{\pgfqpoint{1.322562in}{0.413569in}}%
\pgfpathlineto{\pgfqpoint{1.020194in}{0.414850in}}%
\pgfpathlineto{\pgfqpoint{0.822256in}{0.417715in}}%
\pgfpathlineto{\pgfqpoint{0.704835in}{0.421430in}}%
\pgfpathlineto{\pgfqpoint{0.630976in}{0.425829in}}%
\pgfpathlineto{\pgfqpoint{0.583316in}{0.430734in}}%
\pgfpathlineto{\pgfqpoint{0.551033in}{0.436123in}}%
\pgfpathlineto{\pgfqpoint{0.527708in}{0.442189in}}%
\pgfpathlineto{\pgfqpoint{0.511250in}{0.448625in}}%
\pgfpathlineto{\pgfqpoint{0.499549in}{0.455216in}}%
\pgfpathlineto{\pgfqpoint{0.488916in}{0.463841in}}%
\pgfpathlineto{\pgfqpoint{0.481322in}{0.472730in}}%
\pgfpathlineto{\pgfqpoint{0.474078in}{0.485127in}}%
\pgfpathlineto{\pgfqpoint{0.468753in}{0.498748in}}%
\pgfpathlineto{\pgfqpoint{0.463870in}{0.517848in}}%
\pgfpathlineto{\pgfqpoint{0.459679in}{0.544796in}}%
\pgfpathlineto{\pgfqpoint{0.456386in}{0.581938in}}%
\pgfpathlineto{\pgfqpoint{0.453731in}{0.639106in}}%
\pgfpathlineto{\pgfqpoint{0.451681in}{0.736155in}}%
\pgfpathlineto{\pgfqpoint{0.450220in}{0.927815in}}%
\pgfpathlineto{\pgfqpoint{0.449345in}{1.403252in}}%
\pgfpathlineto{\pgfqpoint{0.449543in}{2.682703in}}%
\pgfpathlineto{\pgfqpoint{0.451011in}{2.856932in}}%
\pgfpathlineto{\pgfqpoint{0.452802in}{2.879219in}}%
\pgfpathlineto{\pgfqpoint{0.455188in}{2.886108in}}%
\pgfpathlineto{\pgfqpoint{0.458626in}{2.889028in}}%
\pgfpathlineto{\pgfqpoint{0.464996in}{2.890553in}}%
\pgfpathlineto{\pgfqpoint{0.482377in}{2.891423in}}%
\pgfpathlineto{\pgfqpoint{0.565038in}{2.891729in}}%
\pgfpathlineto{\pgfqpoint{2.733842in}{2.891760in}}%
\pgfpathlineto{\pgfqpoint{4.789510in}{2.890885in}}%
\pgfpathlineto{\pgfqpoint{4.793727in}{2.889730in}}%
\pgfpathlineto{\pgfqpoint{4.795481in}{2.888307in}}%
\pgfpathlineto{\pgfqpoint{4.797106in}{2.881145in}}%
\pgfpathlineto{\pgfqpoint{4.797997in}{2.858771in}}%
\pgfpathlineto{\pgfqpoint{4.798039in}{2.856283in}}%
\pgfpathlineto{\pgfqpoint{4.798039in}{2.856283in}}%
\pgfusepath{stroke}%
\end{pgfscope}%
\begin{pgfscope}%
\pgfpathrectangle{\pgfqpoint{0.448634in}{0.402556in}}{\pgfqpoint{4.350661in}{2.489204in}} %
\pgfusepath{clip}%
\pgfsetrectcap%
\pgfsetroundjoin%
\pgfsetlinewidth{1.003750pt}%
\definecolor{currentstroke}{rgb}{0.121569,0.466667,0.705882}%
\pgfsetstrokecolor{currentstroke}%
\pgfsetdash{}{0pt}%
\pgfpathmoveto{\pgfqpoint{3.428772in}{0.402610in}}%
\pgfpathlineto{\pgfqpoint{2.806632in}{0.403760in}}%
\pgfpathlineto{\pgfqpoint{2.769692in}{0.405578in}}%
\pgfpathlineto{\pgfqpoint{2.754632in}{0.408064in}}%
\pgfpathlineto{\pgfqpoint{2.746391in}{0.411198in}}%
\pgfpathlineto{\pgfqpoint{2.740943in}{0.415265in}}%
\pgfpathlineto{\pgfqpoint{2.736784in}{0.420984in}}%
\pgfpathlineto{\pgfqpoint{2.733281in}{0.430071in}}%
\pgfpathlineto{\pgfqpoint{2.730449in}{0.444636in}}%
\pgfpathlineto{\pgfqpoint{2.728238in}{0.469392in}}%
\pgfpathlineto{\pgfqpoint{2.726470in}{0.519131in}}%
\pgfpathlineto{\pgfqpoint{2.725711in}{0.613715in}}%
\pgfpathlineto{\pgfqpoint{2.726842in}{0.768038in}}%
\pgfpathlineto{\pgfqpoint{2.730556in}{0.962148in}}%
\pgfpathlineto{\pgfqpoint{2.736611in}{1.158670in}}%
\pgfpathlineto{\pgfqpoint{2.744092in}{1.327718in}}%
\pgfpathlineto{\pgfqpoint{2.753201in}{1.484189in}}%
\pgfpathlineto{\pgfqpoint{2.763257in}{1.620609in}}%
\pgfpathlineto{\pgfqpoint{2.776118in}{1.764216in}}%
\pgfpathlineto{\pgfqpoint{2.788914in}{1.877776in}}%
\pgfpathlineto{\pgfqpoint{2.805748in}{2.005740in}}%
\pgfpathlineto{\pgfqpoint{2.821176in}{2.101198in}}%
\pgfpathlineto{\pgfqpoint{2.838359in}{2.193718in}}%
\pgfpathlineto{\pgfqpoint{2.859135in}{2.292966in}}%
\pgfpathlineto{\pgfqpoint{2.887209in}{2.425960in}}%
\pgfpathlineto{\pgfqpoint{2.896991in}{2.479559in}}%
\pgfpathlineto{\pgfqpoint{2.901543in}{2.516523in}}%
\pgfpathlineto{\pgfqpoint{2.902849in}{2.543854in}}%
\pgfpathlineto{\pgfqpoint{2.901957in}{2.566223in}}%
\pgfpathlineto{\pgfqpoint{2.899151in}{2.585863in}}%
\pgfpathlineto{\pgfqpoint{2.894794in}{2.602546in}}%
\pgfpathlineto{\pgfqpoint{2.888484in}{2.618388in}}%
\pgfpathlineto{\pgfqpoint{2.880257in}{2.633033in}}%
\pgfpathlineto{\pgfqpoint{2.870348in}{2.646246in}}%
\pgfpathlineto{\pgfqpoint{2.857400in}{2.659530in}}%
\pgfpathlineto{\pgfqpoint{2.843189in}{2.671010in}}%
\pgfpathlineto{\pgfqpoint{2.824237in}{2.683209in}}%
\pgfpathlineto{\pgfqpoint{2.802413in}{2.694418in}}%
\pgfpathlineto{\pgfqpoint{2.775809in}{2.705369in}}%
\pgfpathlineto{\pgfqpoint{2.744461in}{2.715715in}}%
\pgfpathlineto{\pgfqpoint{2.708436in}{2.725252in}}%
\pgfpathlineto{\pgfqpoint{2.665655in}{2.734289in}}%
\pgfpathlineto{\pgfqpoint{2.613991in}{2.742869in}}%
\pgfpathlineto{\pgfqpoint{2.553459in}{2.750589in}}%
\pgfpathlineto{\pgfqpoint{2.481920in}{2.757365in}}%
\pgfpathlineto{\pgfqpoint{2.399398in}{2.762839in}}%
\pgfpathlineto{\pgfqpoint{2.310269in}{2.766482in}}%
\pgfpathlineto{\pgfqpoint{2.175416in}{2.768725in}}%
\pgfpathlineto{\pgfqpoint{2.066653in}{2.767942in}}%
\pgfpathlineto{\pgfqpoint{1.953570in}{2.764859in}}%
\pgfpathlineto{\pgfqpoint{1.851429in}{2.759759in}}%
\pgfpathlineto{\pgfqpoint{1.745051in}{2.752169in}}%
\pgfpathlineto{\pgfqpoint{1.658373in}{2.743453in}}%
\pgfpathlineto{\pgfqpoint{1.580552in}{2.733461in}}%
\pgfpathlineto{\pgfqpoint{1.490057in}{2.719338in}}%
\pgfpathlineto{\pgfqpoint{1.417231in}{2.704698in}}%
\pgfpathlineto{\pgfqpoint{1.361992in}{2.690818in}}%
\pgfpathlineto{\pgfqpoint{1.311460in}{2.675819in}}%
\pgfpathlineto{\pgfqpoint{1.265667in}{2.659924in}}%
\pgfpathlineto{\pgfqpoint{1.222575in}{2.642586in}}%
\pgfpathlineto{\pgfqpoint{1.184324in}{2.624682in}}%
\pgfpathlineto{\pgfqpoint{1.148892in}{2.605623in}}%
\pgfpathlineto{\pgfqpoint{1.116331in}{2.585573in}}%
\pgfpathlineto{\pgfqpoint{1.092327in}{2.568512in}}%
\pgfpathlineto{\pgfqpoint{1.079760in}{2.558686in}}%
\pgfpathlineto{\pgfqpoint{1.051544in}{2.535379in}}%
\pgfpathlineto{\pgfqpoint{1.026312in}{2.511712in}}%
\pgfpathlineto{\pgfqpoint{1.002399in}{2.486318in}}%
\pgfpathlineto{\pgfqpoint{0.979913in}{2.459269in}}%
\pgfpathlineto{\pgfqpoint{0.958934in}{2.430678in}}%
\pgfpathlineto{\pgfqpoint{0.938264in}{2.398643in}}%
\pgfpathlineto{\pgfqpoint{0.923047in}{2.371385in}}%
\pgfpathlineto{\pgfqpoint{0.904513in}{2.334774in}}%
\pgfpathlineto{\pgfqpoint{0.887854in}{2.297001in}}%
\pgfpathlineto{\pgfqpoint{0.872131in}{2.255971in}}%
\pgfpathlineto{\pgfqpoint{0.857508in}{2.211741in}}%
\pgfpathlineto{\pgfqpoint{0.844762in}{2.166757in}}%
\pgfpathlineto{\pgfqpoint{0.838624in}{2.140306in}}%
\pgfpathlineto{\pgfqpoint{0.826982in}{2.087194in}}%
\pgfpathlineto{\pgfqpoint{0.816322in}{2.028715in}}%
\pgfpathlineto{\pgfqpoint{0.810087in}{1.984495in}}%
\pgfpathlineto{\pgfqpoint{0.808026in}{1.967238in}}%
\pgfpathlineto{\pgfqpoint{0.800076in}{1.898140in}}%
\pgfpathlineto{\pgfqpoint{0.793713in}{1.823823in}}%
\pgfpathlineto{\pgfqpoint{0.788799in}{1.741875in}}%
\pgfpathlineto{\pgfqpoint{0.786199in}{1.677225in}}%
\pgfpathlineto{\pgfqpoint{0.776951in}{1.453481in}}%
\pgfpathlineto{\pgfqpoint{0.773280in}{1.418894in}}%
\pgfpathlineto{\pgfqpoint{0.768298in}{1.389582in}}%
\pgfpathlineto{\pgfqpoint{0.762752in}{1.368108in}}%
\pgfpathlineto{\pgfqpoint{0.756722in}{1.352123in}}%
\pgfpathlineto{\pgfqpoint{0.749752in}{1.339519in}}%
\pgfpathlineto{\pgfqpoint{0.742201in}{1.330599in}}%
\pgfpathlineto{\pgfqpoint{0.734854in}{1.325312in}}%
\pgfpathlineto{\pgfqpoint{0.726558in}{1.322419in}}%
\pgfpathlineto{\pgfqpoint{0.717884in}{1.322223in}}%
\pgfpathlineto{\pgfqpoint{0.709412in}{1.324411in}}%
\pgfpathlineto{\pgfqpoint{0.699548in}{1.329604in}}%
\pgfpathlineto{\pgfqpoint{0.688894in}{1.338203in}}%
\pgfpathlineto{\pgfqpoint{0.677907in}{1.350248in}}%
\pgfpathlineto{\pgfqpoint{0.666886in}{1.365647in}}%
\pgfpathlineto{\pgfqpoint{0.654913in}{1.386417in}}%
\pgfpathlineto{\pgfqpoint{0.642574in}{1.412730in}}%
\pgfpathlineto{\pgfqpoint{0.630328in}{1.444629in}}%
\pgfpathlineto{\pgfqpoint{0.618504in}{1.482081in}}%
\pgfpathlineto{\pgfqpoint{0.608613in}{1.520256in}}%
\pgfpathlineto{\pgfqpoint{0.590203in}{1.612445in}}%
\pgfpathlineto{\pgfqpoint{0.581848in}{1.668884in}}%
\pgfpathlineto{\pgfqpoint{0.573137in}{1.740376in}}%
\pgfpathlineto{\pgfqpoint{0.567062in}{1.807213in}}%
\pgfpathlineto{\pgfqpoint{0.560532in}{1.896510in}}%
\pgfpathlineto{\pgfqpoint{0.555526in}{1.995910in}}%
\pgfpathlineto{\pgfqpoint{0.552564in}{2.097908in}}%
\pgfpathlineto{\pgfqpoint{0.551526in}{2.204935in}}%
\pgfpathlineto{\pgfqpoint{0.552728in}{2.309470in}}%
\pgfpathlineto{\pgfqpoint{0.556011in}{2.403981in}}%
\pgfpathlineto{\pgfqpoint{0.560953in}{2.483430in}}%
\pgfpathlineto{\pgfqpoint{0.567303in}{2.550240in}}%
\pgfpathlineto{\pgfqpoint{0.574928in}{2.606817in}}%
\pgfpathlineto{\pgfqpoint{0.582988in}{2.650657in}}%
\pgfpathlineto{\pgfqpoint{0.592756in}{2.691452in}}%
\pgfpathlineto{\pgfqpoint{0.602650in}{2.721756in}}%
\pgfpathlineto{\pgfqpoint{0.612983in}{2.746441in}}%
\pgfpathlineto{\pgfqpoint{0.624292in}{2.767692in}}%
\pgfpathlineto{\pgfqpoint{0.636231in}{2.785433in}}%
\pgfpathlineto{\pgfqpoint{0.649892in}{2.801461in}}%
\pgfpathlineto{\pgfqpoint{0.663386in}{2.814020in}}%
\pgfpathlineto{\pgfqpoint{0.679842in}{2.826135in}}%
\pgfpathlineto{\pgfqpoint{0.697326in}{2.836197in}}%
\pgfpathlineto{\pgfqpoint{0.715574in}{2.844285in}}%
\pgfpathlineto{\pgfqpoint{0.738439in}{2.852335in}}%
\pgfpathlineto{\pgfqpoint{0.765983in}{2.859639in}}%
\pgfpathlineto{\pgfqpoint{0.800300in}{2.866256in}}%
\pgfpathlineto{\pgfqpoint{0.841340in}{2.871832in}}%
\pgfpathlineto{\pgfqpoint{0.895547in}{2.876803in}}%
\pgfpathlineto{\pgfqpoint{0.969413in}{2.881069in}}%
\pgfpathlineto{\pgfqpoint{1.071608in}{2.884501in}}%
\pgfpathlineto{\pgfqpoint{1.219512in}{2.887074in}}%
\pgfpathlineto{\pgfqpoint{1.471844in}{2.889091in}}%
\pgfpathlineto{\pgfqpoint{1.956941in}{2.890384in}}%
\pgfpathlineto{\pgfqpoint{3.096814in}{2.890781in}}%
\pgfpathlineto{\pgfqpoint{3.995224in}{2.889388in}}%
\pgfpathlineto{\pgfqpoint{4.275833in}{2.887011in}}%
\pgfpathlineto{\pgfqpoint{4.412847in}{2.883743in}}%
\pgfpathlineto{\pgfqpoint{4.491081in}{2.879810in}}%
\pgfpathlineto{\pgfqpoint{4.543127in}{2.875163in}}%
\pgfpathlineto{\pgfqpoint{4.579810in}{2.869841in}}%
\pgfpathlineto{\pgfqpoint{4.607580in}{2.863763in}}%
\pgfpathlineto{\pgfqpoint{4.630623in}{2.856424in}}%
\pgfpathlineto{\pgfqpoint{4.648833in}{2.848228in}}%
\pgfpathlineto{\pgfqpoint{4.664136in}{2.838773in}}%
\pgfpathlineto{\pgfqpoint{4.676470in}{2.828576in}}%
\pgfpathlineto{\pgfqpoint{4.687502in}{2.816585in}}%
\pgfpathlineto{\pgfqpoint{4.697051in}{2.803027in}}%
\pgfpathlineto{\pgfqpoint{4.706194in}{2.786098in}}%
\pgfpathlineto{\pgfqpoint{4.714508in}{2.765827in}}%
\pgfpathlineto{\pgfqpoint{4.722462in}{2.740013in}}%
\pgfpathlineto{\pgfqpoint{4.729577in}{2.708703in}}%
\pgfpathlineto{\pgfqpoint{4.736162in}{2.669601in}}%
\pgfpathlineto{\pgfqpoint{4.742419in}{2.617826in}}%
\pgfpathlineto{\pgfqpoint{4.747859in}{2.553410in}}%
\pgfpathlineto{\pgfqpoint{4.752661in}{2.468958in}}%
\pgfpathlineto{\pgfqpoint{4.756610in}{2.359528in}}%
\pgfpathlineto{\pgfqpoint{4.759416in}{2.217681in}}%
\pgfpathlineto{\pgfqpoint{4.760596in}{2.043444in}}%
\pgfpathlineto{\pgfqpoint{4.759662in}{1.851779in}}%
\pgfpathlineto{\pgfqpoint{4.756587in}{1.667613in}}%
\pgfpathlineto{\pgfqpoint{4.751596in}{1.503428in}}%
\pgfpathlineto{\pgfqpoint{4.745410in}{1.374185in}}%
\pgfpathlineto{\pgfqpoint{4.738113in}{1.267479in}}%
\pgfpathlineto{\pgfqpoint{4.729621in}{1.175896in}}%
\pgfpathlineto{\pgfqpoint{4.720762in}{1.104428in}}%
\pgfpathlineto{\pgfqpoint{4.711045in}{1.043204in}}%
\pgfpathlineto{\pgfqpoint{4.700364in}{0.989829in}}%
\pgfpathlineto{\pgfqpoint{4.689055in}{0.944345in}}%
\pgfpathlineto{\pgfqpoint{4.676881in}{0.904394in}}%
\pgfpathlineto{\pgfqpoint{4.676095in}{0.902073in}}%
\pgfpathlineto{\pgfqpoint{4.676095in}{0.902073in}}%
\pgfusepath{stroke}%
\end{pgfscope}%
\begin{pgfscope}%
\pgfpathrectangle{\pgfqpoint{0.448634in}{0.402556in}}{\pgfqpoint{4.350661in}{2.489204in}} %
\pgfusepath{clip}%
\pgfsetrectcap%
\pgfsetroundjoin%
\pgfsetlinewidth{1.003750pt}%
\definecolor{currentstroke}{rgb}{0.121569,0.466667,0.705882}%
\pgfsetstrokecolor{currentstroke}%
\pgfsetdash{}{0pt}%
\pgfpathmoveto{\pgfqpoint{2.795520in}{1.982745in}}%
\pgfpathlineto{\pgfqpoint{2.781780in}{1.874357in}}%
\pgfpathlineto{\pgfqpoint{2.769351in}{1.758234in}}%
\pgfpathlineto{\pgfqpoint{2.758095in}{1.631942in}}%
\pgfpathlineto{\pgfqpoint{2.747786in}{1.490551in}}%
\pgfpathlineto{\pgfqpoint{2.738644in}{1.334082in}}%
\pgfpathlineto{\pgfqpoint{2.730580in}{1.157591in}}%
\pgfpathlineto{\pgfqpoint{2.723334in}{0.948663in}}%
\pgfpathlineto{\pgfqpoint{2.709783in}{0.530788in}}%
\pgfpathlineto{\pgfqpoint{2.705868in}{0.488716in}}%
\pgfpathlineto{\pgfqpoint{2.701769in}{0.464281in}}%
\pgfpathlineto{\pgfqpoint{2.697021in}{0.447744in}}%
\pgfpathlineto{\pgfqpoint{2.691859in}{0.436812in}}%
\pgfpathlineto{\pgfqpoint{2.686245in}{0.429229in}}%
\pgfpathlineto{\pgfqpoint{2.679348in}{0.423188in}}%
\pgfpathlineto{\pgfqpoint{2.669540in}{0.417856in}}%
\pgfpathlineto{\pgfqpoint{2.656987in}{0.413810in}}%
\pgfpathlineto{\pgfqpoint{2.637654in}{0.410337in}}%
\pgfpathlineto{\pgfqpoint{2.607297in}{0.407617in}}%
\pgfpathlineto{\pgfqpoint{2.555121in}{0.405574in}}%
\pgfpathlineto{\pgfqpoint{2.450714in}{0.404139in}}%
\pgfpathlineto{\pgfqpoint{2.176624in}{0.403275in}}%
\pgfpathlineto{\pgfqpoint{1.130290in}{0.402953in}}%
\pgfpathlineto{\pgfqpoint{0.516849in}{0.404175in}}%
\pgfpathlineto{\pgfqpoint{0.466848in}{0.405970in}}%
\pgfpathlineto{\pgfqpoint{0.456130in}{0.407931in}}%
\pgfpathlineto{\pgfqpoint{0.452340in}{0.410303in}}%
\pgfpathlineto{\pgfqpoint{0.450346in}{0.414662in}}%
\pgfpathlineto{\pgfqpoint{0.449266in}{0.424524in}}%
\pgfpathlineto{\pgfqpoint{0.448771in}{0.464344in}}%
\pgfpathlineto{\pgfqpoint{0.448640in}{0.850171in}}%
\pgfpathlineto{\pgfqpoint{0.448679in}{2.891318in}}%
\pgfpathlineto{\pgfqpoint{0.448679in}{2.891318in}}%
\pgfusepath{stroke}%
\end{pgfscope}%
\begin{pgfscope}%
\pgfpathrectangle{\pgfqpoint{0.448634in}{0.402556in}}{\pgfqpoint{4.350661in}{2.489204in}} %
\pgfusepath{clip}%
\pgfsetrectcap%
\pgfsetroundjoin%
\pgfsetlinewidth{1.003750pt}%
\definecolor{currentstroke}{rgb}{0.121569,0.466667,0.705882}%
\pgfsetstrokecolor{currentstroke}%
\pgfsetdash{}{0pt}%
\pgfpathmoveto{\pgfqpoint{3.428189in}{0.402586in}}%
\pgfpathlineto{\pgfqpoint{2.782121in}{0.403701in}}%
\pgfpathlineto{\pgfqpoint{2.753906in}{0.405674in}}%
\pgfpathlineto{\pgfqpoint{2.743328in}{0.408443in}}%
\pgfpathlineto{\pgfqpoint{2.737717in}{0.412188in}}%
\pgfpathlineto{\pgfqpoint{2.733668in}{0.417995in}}%
\pgfpathlineto{\pgfqpoint{2.730649in}{0.427307in}}%
\pgfpathlineto{\pgfqpoint{2.728388in}{0.442004in}}%
\pgfpathlineto{\pgfqpoint{2.726544in}{0.471794in}}%
\pgfpathlineto{\pgfqpoint{2.725216in}{0.534003in}}%
\pgfpathlineto{\pgfqpoint{2.725169in}{0.655973in}}%
\pgfpathlineto{\pgfqpoint{2.727377in}{0.832687in}}%
\pgfpathlineto{\pgfqpoint{2.732259in}{1.041703in}}%
\pgfpathlineto{\pgfqpoint{2.738851in}{1.223257in}}%
\pgfpathlineto{\pgfqpoint{2.747078in}{1.389766in}}%
\pgfpathlineto{\pgfqpoint{2.756608in}{1.538717in}}%
\pgfpathlineto{\pgfqpoint{2.768955in}{1.694887in}}%
\pgfpathlineto{\pgfqpoint{2.781228in}{1.816044in}}%
\pgfpathlineto{\pgfqpoint{2.794401in}{1.924524in}}%
\pgfpathlineto{\pgfqpoint{2.812737in}{2.054722in}}%
\pgfpathlineto{\pgfqpoint{2.828774in}{2.147512in}}%
\pgfpathlineto{\pgfqpoint{2.847382in}{2.242224in}}%
\pgfpathlineto{\pgfqpoint{2.895818in}{2.479699in}}%
\pgfpathlineto{\pgfqpoint{2.900204in}{2.516689in}}%
\pgfpathlineto{\pgfqpoint{2.901346in}{2.544029in}}%
\pgfpathlineto{\pgfqpoint{2.900291in}{2.566388in}}%
\pgfpathlineto{\pgfqpoint{2.897334in}{2.585999in}}%
\pgfpathlineto{\pgfqpoint{2.892836in}{2.602633in}}%
\pgfpathlineto{\pgfqpoint{2.886394in}{2.618405in}}%
\pgfpathlineto{\pgfqpoint{2.878058in}{2.632969in}}%
\pgfpathlineto{\pgfqpoint{2.868065in}{2.646100in}}%
\pgfpathlineto{\pgfqpoint{2.855050in}{2.659300in}}%
\pgfpathlineto{\pgfqpoint{2.840801in}{2.670717in}}%
\pgfpathlineto{\pgfqpoint{2.821822in}{2.682861in}}%
\pgfpathlineto{\pgfqpoint{2.799980in}{2.694026in}}%
\pgfpathlineto{\pgfqpoint{2.773366in}{2.704944in}}%
\pgfpathlineto{\pgfqpoint{2.742012in}{2.715266in}}%
\pgfpathlineto{\pgfqpoint{2.705983in}{2.724785in}}%
\pgfpathlineto{\pgfqpoint{2.663200in}{2.733810in}}%
\pgfpathlineto{\pgfqpoint{2.611535in}{2.742379in}}%
\pgfpathlineto{\pgfqpoint{2.551002in}{2.750090in}}%
\pgfpathlineto{\pgfqpoint{2.481632in}{2.756682in}}%
\pgfpathlineto{\pgfqpoint{2.399112in}{2.762200in}}%
\pgfpathlineto{\pgfqpoint{2.309985in}{2.765886in}}%
\pgfpathlineto{\pgfqpoint{2.188184in}{2.768096in}}%
\pgfpathlineto{\pgfqpoint{2.081595in}{2.767619in}}%
\pgfpathlineto{\pgfqpoint{1.968506in}{2.764840in}}%
\pgfpathlineto{\pgfqpoint{1.864180in}{2.759918in}}%
\pgfpathlineto{\pgfqpoint{1.757786in}{2.752593in}}%
\pgfpathlineto{\pgfqpoint{1.671087in}{2.744171in}}%
\pgfpathlineto{\pgfqpoint{1.591076in}{2.734193in}}%
\pgfpathlineto{\pgfqpoint{1.502689in}{2.720717in}}%
\pgfpathlineto{\pgfqpoint{1.427655in}{2.706083in}}%
\pgfpathlineto{\pgfqpoint{1.372350in}{2.692544in}}%
\pgfpathlineto{\pgfqpoint{1.321734in}{2.677921in}}%
\pgfpathlineto{\pgfqpoint{1.273765in}{2.661664in}}%
\pgfpathlineto{\pgfqpoint{1.230567in}{2.644672in}}%
\pgfpathlineto{\pgfqpoint{1.192197in}{2.627106in}}%
\pgfpathlineto{\pgfqpoint{1.156620in}{2.608403in}}%
\pgfpathlineto{\pgfqpoint{1.123890in}{2.588716in}}%
\pgfpathlineto{\pgfqpoint{1.095883in}{2.569568in}}%
\pgfpathlineto{\pgfqpoint{1.063936in}{2.543701in}}%
\pgfpathlineto{\pgfqpoint{1.038217in}{2.520732in}}%
\pgfpathlineto{\pgfqpoint{1.013766in}{2.496016in}}%
\pgfpathlineto{\pgfqpoint{0.990704in}{2.469610in}}%
\pgfpathlineto{\pgfqpoint{0.969124in}{2.441612in}}%
\pgfpathlineto{\pgfqpoint{0.949083in}{2.412154in}}%
\pgfpathlineto{\pgfqpoint{0.930604in}{2.381387in}}%
\pgfpathlineto{\pgfqpoint{0.906555in}{2.334052in}}%
\pgfpathlineto{\pgfqpoint{0.889925in}{2.296262in}}%
\pgfpathlineto{\pgfqpoint{0.874241in}{2.255213in}}%
\pgfpathlineto{\pgfqpoint{0.859667in}{2.210961in}}%
\pgfpathlineto{\pgfqpoint{0.846986in}{2.165954in}}%
\pgfpathlineto{\pgfqpoint{0.839633in}{2.134715in}}%
\pgfpathlineto{\pgfqpoint{0.828238in}{2.081532in}}%
\pgfpathlineto{\pgfqpoint{0.817866in}{2.022986in}}%
\pgfpathlineto{\pgfqpoint{0.810784in}{1.971352in}}%
\pgfpathlineto{\pgfqpoint{0.802846in}{1.902252in}}%
\pgfpathlineto{\pgfqpoint{0.796554in}{1.827927in}}%
\pgfpathlineto{\pgfqpoint{0.791696in}{1.743480in}}%
\pgfpathlineto{\pgfqpoint{0.787773in}{1.621595in}}%
\pgfpathlineto{\pgfqpoint{0.785408in}{1.522064in}}%
\pgfpathlineto{\pgfqpoint{0.785408in}{1.522064in}}%
\pgfusepath{stroke}%
\end{pgfscope}%
\begin{pgfscope}%
\pgfpathrectangle{\pgfqpoint{0.448634in}{0.402556in}}{\pgfqpoint{4.350661in}{2.489204in}} %
\pgfusepath{clip}%
\pgfsetrectcap%
\pgfsetroundjoin%
\pgfsetlinewidth{1.003750pt}%
\definecolor{currentstroke}{rgb}{1.000000,0.498039,0.054902}%
\pgfsetstrokecolor{currentstroke}%
\pgfsetdash{}{0pt}%
\pgfpathmoveto{\pgfqpoint{0.448634in}{2.896245in}}%
\pgfpathlineto{\pgfqpoint{0.448593in}{0.407043in}}%
\pgfpathlineto{\pgfqpoint{0.448593in}{0.407043in}}%
\pgfusepath{stroke}%
\end{pgfscope}%
\begin{pgfscope}%
\pgfpathrectangle{\pgfqpoint{0.448634in}{0.402556in}}{\pgfqpoint{4.350661in}{2.489204in}} %
\pgfusepath{clip}%
\pgfsetrectcap%
\pgfsetroundjoin%
\pgfsetlinewidth{1.003750pt}%
\definecolor{currentstroke}{rgb}{1.000000,0.498039,0.054902}%
\pgfsetstrokecolor{currentstroke}%
\pgfsetdash{}{0pt}%
\pgfpathmoveto{\pgfqpoint{0.576853in}{1.760817in}}%
\pgfpathlineto{\pgfqpoint{0.569394in}{1.840010in}}%
\pgfpathlineto{\pgfqpoint{0.563209in}{1.929338in}}%
\pgfpathlineto{\pgfqpoint{0.558592in}{2.028764in}}%
\pgfpathlineto{\pgfqpoint{0.555985in}{2.133265in}}%
\pgfpathlineto{\pgfqpoint{0.555566in}{2.237808in}}%
\pgfpathlineto{\pgfqpoint{0.557371in}{2.337352in}}%
\pgfpathlineto{\pgfqpoint{0.561096in}{2.424366in}}%
\pgfpathlineto{\pgfqpoint{0.566403in}{2.498791in}}%
\pgfpathlineto{\pgfqpoint{0.572909in}{2.560570in}}%
\pgfpathlineto{\pgfqpoint{0.580458in}{2.612119in}}%
\pgfpathlineto{\pgfqpoint{0.589086in}{2.655816in}}%
\pgfpathlineto{\pgfqpoint{0.598406in}{2.691589in}}%
\pgfpathlineto{\pgfqpoint{0.608613in}{2.721757in}}%
\pgfpathlineto{\pgfqpoint{0.619241in}{2.746278in}}%
\pgfpathlineto{\pgfqpoint{0.630817in}{2.767339in}}%
\pgfpathlineto{\pgfqpoint{0.642975in}{2.784884in}}%
\pgfpathlineto{\pgfqpoint{0.656813in}{2.800712in}}%
\pgfpathlineto{\pgfqpoint{0.672197in}{2.814549in}}%
\pgfpathlineto{\pgfqpoint{0.688853in}{2.826301in}}%
\pgfpathlineto{\pgfqpoint{0.706461in}{2.836076in}}%
\pgfpathlineto{\pgfqpoint{0.726804in}{2.844875in}}%
\pgfpathlineto{\pgfqpoint{0.751866in}{2.853203in}}%
\pgfpathlineto{\pgfqpoint{0.781631in}{2.860547in}}%
\pgfpathlineto{\pgfqpoint{0.818168in}{2.867054in}}%
\pgfpathlineto{\pgfqpoint{0.863581in}{2.872685in}}%
\pgfpathlineto{\pgfqpoint{0.922161in}{2.877518in}}%
\pgfpathlineto{\pgfqpoint{1.000391in}{2.881567in}}%
\pgfpathlineto{\pgfqpoint{1.111294in}{2.884881in}}%
\pgfpathlineto{\pgfqpoint{1.274428in}{2.887367in}}%
\pgfpathlineto{\pgfqpoint{1.552865in}{2.889263in}}%
\pgfpathlineto{\pgfqpoint{2.107573in}{2.890457in}}%
\pgfpathlineto{\pgfqpoint{3.343161in}{2.890573in}}%
\pgfpathlineto{\pgfqpoint{4.043615in}{2.888941in}}%
\pgfpathlineto{\pgfqpoint{4.289417in}{2.886404in}}%
\pgfpathlineto{\pgfqpoint{4.413375in}{2.883093in}}%
\pgfpathlineto{\pgfqpoint{4.489424in}{2.878997in}}%
\pgfpathlineto{\pgfqpoint{4.541451in}{2.874081in}}%
\pgfpathlineto{\pgfqpoint{4.578100in}{2.868470in}}%
\pgfpathlineto{\pgfqpoint{4.605818in}{2.862092in}}%
\pgfpathlineto{\pgfqpoint{4.626725in}{2.855245in}}%
\pgfpathlineto{\pgfqpoint{4.644925in}{2.847018in}}%
\pgfpathlineto{\pgfqpoint{4.660241in}{2.837590in}}%
\pgfpathlineto{\pgfqpoint{4.672623in}{2.827468in}}%
\pgfpathlineto{\pgfqpoint{4.683751in}{2.815592in}}%
\pgfpathlineto{\pgfqpoint{4.693406in}{2.802135in}}%
\pgfpathlineto{\pgfqpoint{4.702740in}{2.785343in}}%
\pgfpathlineto{\pgfqpoint{4.711277in}{2.765194in}}%
\pgfpathlineto{\pgfqpoint{4.719482in}{2.739484in}}%
\pgfpathlineto{\pgfqpoint{4.726293in}{2.710657in}}%
\pgfpathlineto{\pgfqpoint{4.733259in}{2.671643in}}%
\pgfpathlineto{\pgfqpoint{4.739604in}{2.622396in}}%
\pgfpathlineto{\pgfqpoint{4.745236in}{2.560504in}}%
\pgfpathlineto{\pgfqpoint{4.750164in}{2.481052in}}%
\pgfpathlineto{\pgfqpoint{4.754367in}{2.376618in}}%
\pgfpathlineto{\pgfqpoint{4.757443in}{2.242249in}}%
\pgfpathlineto{\pgfqpoint{4.758977in}{2.075483in}}%
\pgfpathlineto{\pgfqpoint{4.758447in}{1.888795in}}%
\pgfpathlineto{\pgfqpoint{4.755756in}{1.707111in}}%
\pgfpathlineto{\pgfqpoint{4.750925in}{1.532957in}}%
\pgfpathlineto{\pgfqpoint{4.744785in}{1.398726in}}%
\pgfpathlineto{\pgfqpoint{4.737575in}{1.289516in}}%
\pgfpathlineto{\pgfqpoint{4.728714in}{1.190470in}}%
\pgfpathlineto{\pgfqpoint{4.719652in}{1.116521in}}%
\pgfpathlineto{\pgfqpoint{4.710036in}{1.055276in}}%
\pgfpathlineto{\pgfqpoint{4.699503in}{1.001861in}}%
\pgfpathlineto{\pgfqpoint{4.689040in}{0.958690in}}%
\pgfpathlineto{\pgfqpoint{4.677219in}{0.918600in}}%
\pgfpathlineto{\pgfqpoint{4.664034in}{0.881749in}}%
\pgfpathlineto{\pgfqpoint{4.650584in}{0.850492in}}%
\pgfpathlineto{\pgfqpoint{4.636303in}{0.822570in}}%
\pgfpathlineto{\pgfqpoint{4.620207in}{0.795974in}}%
\pgfpathlineto{\pgfqpoint{4.603640in}{0.772901in}}%
\pgfpathlineto{\pgfqpoint{4.585488in}{0.751446in}}%
\pgfpathlineto{\pgfqpoint{4.565874in}{0.731749in}}%
\pgfpathlineto{\pgfqpoint{4.544964in}{0.713879in}}%
\pgfpathlineto{\pgfqpoint{4.522958in}{0.697824in}}%
\pgfpathlineto{\pgfqpoint{4.496157in}{0.681290in}}%
\pgfpathlineto{\pgfqpoint{4.470397in}{0.667953in}}%
\pgfpathlineto{\pgfqpoint{4.439961in}{0.654509in}}%
\pgfpathlineto{\pgfqpoint{4.406841in}{0.642281in}}%
\pgfpathlineto{\pgfqpoint{4.369009in}{0.630748in}}%
\pgfpathlineto{\pgfqpoint{4.326489in}{0.620226in}}%
\pgfpathlineto{\pgfqpoint{4.279327in}{0.610949in}}%
\pgfpathlineto{\pgfqpoint{4.227576in}{0.603085in}}%
\pgfpathlineto{\pgfqpoint{4.173450in}{0.597063in}}%
\pgfpathlineto{\pgfqpoint{4.110511in}{0.592203in}}%
\pgfpathlineto{\pgfqpoint{4.047471in}{0.589537in}}%
\pgfpathlineto{\pgfqpoint{3.977867in}{0.588624in}}%
\pgfpathlineto{\pgfqpoint{3.906093in}{0.589934in}}%
\pgfpathlineto{\pgfqpoint{3.834377in}{0.593496in}}%
\pgfpathlineto{\pgfqpoint{3.767120in}{0.599067in}}%
\pgfpathlineto{\pgfqpoint{3.704364in}{0.606392in}}%
\pgfpathlineto{\pgfqpoint{3.678516in}{0.610510in}}%
\pgfpathlineto{\pgfqpoint{3.620438in}{0.620500in}}%
\pgfpathlineto{\pgfqpoint{3.586319in}{0.628207in}}%
\pgfpathlineto{\pgfqpoint{3.495240in}{0.652428in}}%
\pgfpathlineto{\pgfqpoint{3.451528in}{0.667583in}}%
\pgfpathlineto{\pgfqpoint{3.408538in}{0.685220in}}%
\pgfpathlineto{\pgfqpoint{3.374594in}{0.702001in}}%
\pgfpathlineto{\pgfqpoint{3.345407in}{0.718682in}}%
\pgfpathlineto{\pgfqpoint{3.315236in}{0.738520in}}%
\pgfpathlineto{\pgfqpoint{3.288127in}{0.759290in}}%
\pgfpathlineto{\pgfqpoint{3.264004in}{0.780551in}}%
\pgfpathlineto{\pgfqpoint{3.241208in}{0.803648in}}%
\pgfpathlineto{\pgfqpoint{3.219894in}{0.828530in}}%
\pgfpathlineto{\pgfqpoint{3.200189in}{0.855091in}}%
\pgfpathlineto{\pgfqpoint{3.182177in}{0.883182in}}%
\pgfpathlineto{\pgfqpoint{3.165906in}{0.912633in}}%
\pgfpathlineto{\pgfqpoint{3.150351in}{0.945448in}}%
\pgfpathlineto{\pgfqpoint{3.136682in}{0.979345in}}%
\pgfpathlineto{\pgfqpoint{3.124073in}{1.016460in}}%
\pgfpathlineto{\pgfqpoint{3.112834in}{1.056769in}}%
\pgfpathlineto{\pgfqpoint{3.103046in}{1.100146in}}%
\pgfpathlineto{\pgfqpoint{3.095343in}{1.144071in}}%
\pgfpathlineto{\pgfqpoint{3.089208in}{1.190837in}}%
\pgfpathlineto{\pgfqpoint{3.084595in}{1.242838in}}%
\pgfpathlineto{\pgfqpoint{3.082137in}{1.295031in}}%
\pgfpathlineto{\pgfqpoint{3.081687in}{1.349787in}}%
\pgfpathlineto{\pgfqpoint{3.083451in}{1.406998in}}%
\pgfpathlineto{\pgfqpoint{3.087181in}{1.461589in}}%
\pgfpathlineto{\pgfqpoint{3.093485in}{1.520888in}}%
\pgfpathlineto{\pgfqpoint{3.101823in}{1.577334in}}%
\pgfpathlineto{\pgfqpoint{3.111930in}{1.630856in}}%
\pgfpathlineto{\pgfqpoint{3.124690in}{1.686208in}}%
\pgfpathlineto{\pgfqpoint{3.139178in}{1.738395in}}%
\pgfpathlineto{\pgfqpoint{3.155145in}{1.787366in}}%
\pgfpathlineto{\pgfqpoint{3.172353in}{1.833085in}}%
\pgfpathlineto{\pgfqpoint{3.191618in}{1.877716in}}%
\pgfpathlineto{\pgfqpoint{3.214026in}{1.923261in}}%
\pgfpathlineto{\pgfqpoint{3.236214in}{1.963157in}}%
\pgfpathlineto{\pgfqpoint{3.260178in}{2.001684in}}%
\pgfpathlineto{\pgfqpoint{3.285814in}{2.038776in}}%
\pgfpathlineto{\pgfqpoint{3.314415in}{2.076285in}}%
\pgfpathlineto{\pgfqpoint{3.348944in}{2.117711in}}%
\pgfpathlineto{\pgfqpoint{3.417133in}{2.198022in}}%
\pgfpathlineto{\pgfqpoint{3.426053in}{2.212128in}}%
\pgfpathlineto{\pgfqpoint{3.430798in}{2.223297in}}%
\pgfpathlineto{\pgfqpoint{3.432034in}{2.230603in}}%
\pgfpathlineto{\pgfqpoint{3.430773in}{2.237856in}}%
\pgfpathlineto{\pgfqpoint{3.426621in}{2.243526in}}%
\pgfpathlineto{\pgfqpoint{3.420908in}{2.247084in}}%
\pgfpathlineto{\pgfqpoint{3.412501in}{2.249583in}}%
\pgfpathlineto{\pgfqpoint{3.399499in}{2.250689in}}%
\pgfpathlineto{\pgfqpoint{3.384305in}{2.249671in}}%
\pgfpathlineto{\pgfqpoint{3.364985in}{2.246098in}}%
\pgfpathlineto{\pgfqpoint{3.341804in}{2.239342in}}%
\pgfpathlineto{\pgfqpoint{3.317109in}{2.229682in}}%
\pgfpathlineto{\pgfqpoint{3.291104in}{2.216986in}}%
\pgfpathlineto{\pgfqpoint{3.265928in}{2.202261in}}%
\pgfpathlineto{\pgfqpoint{3.239805in}{2.184361in}}%
\pgfpathlineto{\pgfqpoint{3.214775in}{2.164519in}}%
\pgfpathlineto{\pgfqpoint{3.190900in}{2.142893in}}%
\pgfpathlineto{\pgfqpoint{3.166657in}{2.117912in}}%
\pgfpathlineto{\pgfqpoint{3.143835in}{2.091233in}}%
\pgfpathlineto{\pgfqpoint{3.121079in}{2.061107in}}%
\pgfpathlineto{\pgfqpoint{3.099952in}{2.029463in}}%
\pgfpathlineto{\pgfqpoint{3.079251in}{1.994406in}}%
\pgfpathlineto{\pgfqpoint{3.059218in}{1.955915in}}%
\pgfpathlineto{\pgfqpoint{3.040058in}{1.914015in}}%
\pgfpathlineto{\pgfqpoint{3.022809in}{1.871041in}}%
\pgfpathlineto{\pgfqpoint{3.005790in}{1.822536in}}%
\pgfpathlineto{\pgfqpoint{2.990067in}{1.770819in}}%
\pgfpathlineto{\pgfqpoint{2.975708in}{1.715979in}}%
\pgfpathlineto{\pgfqpoint{2.962284in}{1.655680in}}%
\pgfpathlineto{\pgfqpoint{2.950496in}{1.592386in}}%
\pgfpathlineto{\pgfqpoint{2.940383in}{1.526185in}}%
\pgfpathlineto{\pgfqpoint{2.931745in}{1.454681in}}%
\pgfpathlineto{\pgfqpoint{2.925082in}{1.380399in}}%
\pgfpathlineto{\pgfqpoint{2.920647in}{1.305899in}}%
\pgfpathlineto{\pgfqpoint{2.918444in}{1.231270in}}%
\pgfpathlineto{\pgfqpoint{2.918545in}{1.159087in}}%
\pgfpathlineto{\pgfqpoint{2.920787in}{1.091931in}}%
\pgfpathlineto{\pgfqpoint{2.925177in}{1.027412in}}%
\pgfpathlineto{\pgfqpoint{2.931192in}{0.970580in}}%
\pgfpathlineto{\pgfqpoint{2.938760in}{0.919034in}}%
\pgfpathlineto{\pgfqpoint{2.947651in}{0.872852in}}%
\pgfpathlineto{\pgfqpoint{2.958213in}{0.829714in}}%
\pgfpathlineto{\pgfqpoint{2.969670in}{0.792114in}}%
\pgfpathlineto{\pgfqpoint{2.982463in}{0.757773in}}%
\pgfpathlineto{\pgfqpoint{2.996425in}{0.726812in}}%
\pgfpathlineto{\pgfqpoint{3.011299in}{0.699300in}}%
\pgfpathlineto{\pgfqpoint{3.026739in}{0.675225in}}%
\pgfpathlineto{\pgfqpoint{3.043828in}{0.652656in}}%
\pgfpathlineto{\pgfqpoint{3.062495in}{0.631788in}}%
\pgfpathlineto{\pgfqpoint{3.082602in}{0.612753in}}%
\pgfpathlineto{\pgfqpoint{3.103961in}{0.595592in}}%
\pgfpathlineto{\pgfqpoint{3.128268in}{0.579069in}}%
\pgfpathlineto{\pgfqpoint{3.153537in}{0.564554in}}%
\pgfpathlineto{\pgfqpoint{3.181571in}{0.550952in}}%
\pgfpathlineto{\pgfqpoint{3.214371in}{0.537647in}}%
\pgfpathlineto{\pgfqpoint{3.249846in}{0.525712in}}%
\pgfpathlineto{\pgfqpoint{3.290011in}{0.514571in}}%
\pgfpathlineto{\pgfqpoint{3.334820in}{0.504423in}}%
\pgfpathlineto{\pgfqpoint{3.386372in}{0.494999in}}%
\pgfpathlineto{\pgfqpoint{3.446798in}{0.486257in}}%
\pgfpathlineto{\pgfqpoint{3.518243in}{0.478282in}}%
\pgfpathlineto{\pgfqpoint{3.600685in}{0.471409in}}%
\pgfpathlineto{\pgfqpoint{3.696268in}{0.465713in}}%
\pgfpathlineto{\pgfqpoint{3.807144in}{0.461369in}}%
\pgfpathlineto{\pgfqpoint{3.933291in}{0.458719in}}%
\pgfpathlineto{\pgfqpoint{4.063808in}{0.458211in}}%
\pgfpathlineto{\pgfqpoint{4.187792in}{0.459914in}}%
\pgfpathlineto{\pgfqpoint{4.294335in}{0.463521in}}%
\pgfpathlineto{\pgfqpoint{4.381234in}{0.468574in}}%
\pgfpathlineto{\pgfqpoint{4.450636in}{0.474701in}}%
\pgfpathlineto{\pgfqpoint{4.506850in}{0.481799in}}%
\pgfpathlineto{\pgfqpoint{4.552009in}{0.489658in}}%
\pgfpathlineto{\pgfqpoint{4.588239in}{0.498115in}}%
\pgfpathlineto{\pgfqpoint{4.617656in}{0.507110in}}%
\pgfpathlineto{\pgfqpoint{4.642328in}{0.516843in}}%
\pgfpathlineto{\pgfqpoint{4.664194in}{0.527940in}}%
\pgfpathlineto{\pgfqpoint{4.681238in}{0.538945in}}%
\pgfpathlineto{\pgfqpoint{4.697164in}{0.551953in}}%
\pgfpathlineto{\pgfqpoint{4.710076in}{0.565289in}}%
\pgfpathlineto{\pgfqpoint{4.721578in}{0.580218in}}%
\pgfpathlineto{\pgfqpoint{4.731557in}{0.596521in}}%
\pgfpathlineto{\pgfqpoint{4.741000in}{0.616134in}}%
\pgfpathlineto{\pgfqpoint{4.749521in}{0.639027in}}%
\pgfpathlineto{\pgfqpoint{4.757522in}{0.667450in}}%
\pgfpathlineto{\pgfqpoint{4.764572in}{0.701345in}}%
\pgfpathlineto{\pgfqpoint{4.770840in}{0.743043in}}%
\pgfpathlineto{\pgfqpoint{4.776327in}{0.794934in}}%
\pgfpathlineto{\pgfqpoint{4.781278in}{0.864398in}}%
\pgfpathlineto{\pgfqpoint{4.785468in}{0.956371in}}%
\pgfpathlineto{\pgfqpoint{4.789000in}{1.085745in}}%
\pgfpathlineto{\pgfqpoint{4.791852in}{1.277385in}}%
\pgfpathlineto{\pgfqpoint{4.793959in}{1.581057in}}%
\pgfpathlineto{\pgfqpoint{4.794962in}{2.071429in}}%
\pgfpathlineto{\pgfqpoint{4.793967in}{2.559311in}}%
\pgfpathlineto{\pgfqpoint{4.791733in}{2.745981in}}%
\pgfpathlineto{\pgfqpoint{4.788955in}{2.818091in}}%
\pgfpathlineto{\pgfqpoint{4.785731in}{2.850227in}}%
\pgfpathlineto{\pgfqpoint{4.781879in}{2.867057in}}%
\pgfpathlineto{\pgfqpoint{4.777744in}{2.875780in}}%
\pgfpathlineto{\pgfqpoint{4.773097in}{2.880982in}}%
\pgfpathlineto{\pgfqpoint{4.767363in}{2.884504in}}%
\pgfpathlineto{\pgfqpoint{4.756853in}{2.887622in}}%
\pgfpathlineto{\pgfqpoint{4.739548in}{2.889639in}}%
\pgfpathlineto{\pgfqpoint{4.704762in}{2.890882in}}%
\pgfpathlineto{\pgfqpoint{4.602524in}{2.891538in}}%
\pgfpathlineto{\pgfqpoint{3.952100in}{2.891742in}}%
\pgfpathlineto{\pgfqpoint{0.617321in}{2.890753in}}%
\pgfpathlineto{\pgfqpoint{0.549910in}{2.888858in}}%
\pgfpathlineto{\pgfqpoint{0.521735in}{2.886179in}}%
\pgfpathlineto{\pgfqpoint{0.504666in}{2.882389in}}%
\pgfpathlineto{\pgfqpoint{0.494501in}{2.878011in}}%
\pgfpathlineto{\pgfqpoint{0.487180in}{2.872667in}}%
\pgfpathlineto{\pgfqpoint{0.481152in}{2.865519in}}%
\pgfpathlineto{\pgfqpoint{0.475664in}{2.854804in}}%
\pgfpathlineto{\pgfqpoint{0.471318in}{2.840737in}}%
\pgfpathlineto{\pgfqpoint{0.467301in}{2.818823in}}%
\pgfpathlineto{\pgfqpoint{0.463927in}{2.786700in}}%
\pgfpathlineto{\pgfqpoint{0.460918in}{2.734544in}}%
\pgfpathlineto{\pgfqpoint{0.458363in}{2.647473in}}%
\pgfpathlineto{\pgfqpoint{0.456575in}{2.523031in}}%
\pgfpathlineto{\pgfqpoint{0.456575in}{2.523031in}}%
\pgfusepath{stroke}%
\end{pgfscope}%
\begin{pgfscope}%
\pgfpathrectangle{\pgfqpoint{0.448634in}{0.402556in}}{\pgfqpoint{4.350661in}{2.489204in}} %
\pgfusepath{clip}%
\pgfsetrectcap%
\pgfsetroundjoin%
\pgfsetlinewidth{1.003750pt}%
\definecolor{currentstroke}{rgb}{1.000000,0.498039,0.054902}%
\pgfsetstrokecolor{currentstroke}%
\pgfsetdash{}{0pt}%
\pgfpathmoveto{\pgfqpoint{4.798840in}{2.852369in}}%
\pgfpathlineto{\pgfqpoint{4.797564in}{2.889610in}}%
\pgfpathlineto{\pgfqpoint{4.796215in}{2.891483in}}%
\pgfpathlineto{\pgfqpoint{4.787551in}{2.891760in}}%
\pgfpathlineto{\pgfqpoint{0.452128in}{2.891659in}}%
\pgfpathlineto{\pgfqpoint{0.450530in}{2.890082in}}%
\pgfpathlineto{\pgfqpoint{0.449454in}{2.882763in}}%
\pgfpathlineto{\pgfqpoint{0.448970in}{2.845432in}}%
\pgfpathlineto{\pgfqpoint{0.448743in}{2.494454in}}%
\pgfpathlineto{\pgfqpoint{0.449624in}{0.615107in}}%
\pgfpathlineto{\pgfqpoint{0.451433in}{0.510586in}}%
\pgfpathlineto{\pgfqpoint{0.453993in}{0.473374in}}%
\pgfpathlineto{\pgfqpoint{0.457406in}{0.453868in}}%
\pgfpathlineto{\pgfqpoint{0.461540in}{0.442384in}}%
\pgfpathlineto{\pgfqpoint{0.466739in}{0.434437in}}%
\pgfpathlineto{\pgfqpoint{0.473595in}{0.428350in}}%
\pgfpathlineto{\pgfqpoint{0.483492in}{0.423244in}}%
\pgfpathlineto{\pgfqpoint{0.491854in}{0.420501in}}%
\pgfpathlineto{\pgfqpoint{0.491854in}{0.420501in}}%
\pgfusepath{stroke}%
\end{pgfscope}%
\begin{pgfscope}%
\pgfpathrectangle{\pgfqpoint{0.448634in}{0.402556in}}{\pgfqpoint{4.350661in}{2.489204in}} %
\pgfusepath{clip}%
\pgfsetrectcap%
\pgfsetroundjoin%
\pgfsetlinewidth{1.003750pt}%
\definecolor{currentstroke}{rgb}{1.000000,0.498039,0.054902}%
\pgfsetstrokecolor{currentstroke}%
\pgfsetdash{}{0pt}%
\pgfpathmoveto{\pgfqpoint{0.456424in}{1.370137in}}%
\pgfpathlineto{\pgfqpoint{0.459610in}{1.118755in}}%
\pgfpathlineto{\pgfqpoint{0.463695in}{0.962007in}}%
\pgfpathlineto{\pgfqpoint{0.468519in}{0.857610in}}%
\pgfpathlineto{\pgfqpoint{0.474082in}{0.783210in}}%
\pgfpathlineto{\pgfqpoint{0.480226in}{0.728906in}}%
\pgfpathlineto{\pgfqpoint{0.486970in}{0.687306in}}%
\pgfpathlineto{\pgfqpoint{0.494537in}{0.653558in}}%
\pgfpathlineto{\pgfqpoint{0.503107in}{0.625355in}}%
\pgfpathlineto{\pgfqpoint{0.512193in}{0.602749in}}%
\pgfpathlineto{\pgfqpoint{0.522200in}{0.583508in}}%
\pgfpathlineto{\pgfqpoint{0.534108in}{0.565743in}}%
\pgfpathlineto{\pgfqpoint{0.546263in}{0.551507in}}%
\pgfpathlineto{\pgfqpoint{0.559728in}{0.538907in}}%
\pgfpathlineto{\pgfqpoint{0.576129in}{0.526693in}}%
\pgfpathlineto{\pgfqpoint{0.595483in}{0.515351in}}%
\pgfpathlineto{\pgfqpoint{0.617681in}{0.505147in}}%
\pgfpathlineto{\pgfqpoint{0.642568in}{0.496153in}}%
\pgfpathlineto{\pgfqpoint{0.672126in}{0.487778in}}%
\pgfpathlineto{\pgfqpoint{0.708443in}{0.479824in}}%
\pgfpathlineto{\pgfqpoint{0.753649in}{0.472325in}}%
\pgfpathlineto{\pgfqpoint{0.807717in}{0.465660in}}%
\pgfpathlineto{\pgfqpoint{0.877116in}{0.459475in}}%
\pgfpathlineto{\pgfqpoint{0.961828in}{0.454230in}}%
\pgfpathlineto{\pgfqpoint{1.068351in}{0.449916in}}%
\pgfpathlineto{\pgfqpoint{1.201018in}{0.446839in}}%
\pgfpathlineto{\pgfqpoint{1.357637in}{0.445481in}}%
\pgfpathlineto{\pgfqpoint{1.525135in}{0.446232in}}%
\pgfpathlineto{\pgfqpoint{1.686088in}{0.449142in}}%
\pgfpathlineto{\pgfqpoint{1.823074in}{0.453747in}}%
\pgfpathlineto{\pgfqpoint{1.938245in}{0.459764in}}%
\pgfpathlineto{\pgfqpoint{2.031582in}{0.466759in}}%
\pgfpathlineto{\pgfqpoint{2.109580in}{0.474745in}}%
\pgfpathlineto{\pgfqpoint{2.174384in}{0.483535in}}%
\pgfpathlineto{\pgfqpoint{2.228139in}{0.492940in}}%
\pgfpathlineto{\pgfqpoint{2.275119in}{0.503356in}}%
\pgfpathlineto{\pgfqpoint{2.315282in}{0.514501in}}%
\pgfpathlineto{\pgfqpoint{2.350698in}{0.526659in}}%
\pgfpathlineto{\pgfqpoint{2.381320in}{0.539536in}}%
\pgfpathlineto{\pgfqpoint{2.407164in}{0.552659in}}%
\pgfpathlineto{\pgfqpoint{2.430226in}{0.566639in}}%
\pgfpathlineto{\pgfqpoint{2.452282in}{0.582602in}}%
\pgfpathlineto{\pgfqpoint{2.471391in}{0.599069in}}%
\pgfpathlineto{\pgfqpoint{2.489240in}{0.617293in}}%
\pgfpathlineto{\pgfqpoint{2.505678in}{0.637180in}}%
\pgfpathlineto{\pgfqpoint{2.520620in}{0.658557in}}%
\pgfpathlineto{\pgfqpoint{2.535213in}{0.683314in}}%
\pgfpathlineto{\pgfqpoint{2.549115in}{0.711484in}}%
\pgfpathlineto{\pgfqpoint{2.562091in}{0.743004in}}%
\pgfpathlineto{\pgfqpoint{2.574020in}{0.777751in}}%
\pgfpathlineto{\pgfqpoint{2.585502in}{0.817970in}}%
\pgfpathlineto{\pgfqpoint{2.596809in}{0.866038in}}%
\pgfpathlineto{\pgfqpoint{2.607562in}{0.921948in}}%
\pgfpathlineto{\pgfqpoint{2.617925in}{0.988098in}}%
\pgfpathlineto{\pgfqpoint{2.627958in}{1.066918in}}%
\pgfpathlineto{\pgfqpoint{2.637941in}{1.163320in}}%
\pgfpathlineto{\pgfqpoint{2.648424in}{1.287199in}}%
\pgfpathlineto{\pgfqpoint{2.660103in}{1.453438in}}%
\pgfpathlineto{\pgfqpoint{2.674773in}{1.696801in}}%
\pgfpathlineto{\pgfqpoint{2.687716in}{1.945279in}}%
\pgfpathlineto{\pgfqpoint{2.692670in}{2.079573in}}%
\pgfpathlineto{\pgfqpoint{2.693829in}{2.166682in}}%
\pgfpathlineto{\pgfqpoint{2.692565in}{2.233870in}}%
\pgfpathlineto{\pgfqpoint{2.689436in}{2.286015in}}%
\pgfpathlineto{\pgfqpoint{2.684859in}{2.327999in}}%
\pgfpathlineto{\pgfqpoint{2.678725in}{2.364664in}}%
\pgfpathlineto{\pgfqpoint{2.671356in}{2.395897in}}%
\pgfpathlineto{\pgfqpoint{2.662489in}{2.423981in}}%
\pgfpathlineto{\pgfqpoint{2.652361in}{2.448778in}}%
\pgfpathlineto{\pgfqpoint{2.641365in}{2.470245in}}%
\pgfpathlineto{\pgfqpoint{2.628643in}{2.490425in}}%
\pgfpathlineto{\pgfqpoint{2.614279in}{2.509106in}}%
\pgfpathlineto{\pgfqpoint{2.598443in}{2.526159in}}%
\pgfpathlineto{\pgfqpoint{2.579590in}{2.543005in}}%
\pgfpathlineto{\pgfqpoint{2.559532in}{2.557923in}}%
\pgfpathlineto{\pgfqpoint{2.536602in}{2.572183in}}%
\pgfpathlineto{\pgfqpoint{2.510850in}{2.585538in}}%
\pgfpathlineto{\pgfqpoint{2.482360in}{2.597837in}}%
\pgfpathlineto{\pgfqpoint{2.449134in}{2.609683in}}%
\pgfpathlineto{\pgfqpoint{2.411184in}{2.620696in}}%
\pgfpathlineto{\pgfqpoint{2.368552in}{2.630606in}}%
\pgfpathlineto{\pgfqpoint{2.321294in}{2.639221in}}%
\pgfpathlineto{\pgfqpoint{2.269467in}{2.646399in}}%
\pgfpathlineto{\pgfqpoint{2.210954in}{2.652193in}}%
\pgfpathlineto{\pgfqpoint{2.147967in}{2.656153in}}%
\pgfpathlineto{\pgfqpoint{2.080556in}{2.658135in}}%
\pgfpathlineto{\pgfqpoint{2.010948in}{2.657971in}}%
\pgfpathlineto{\pgfqpoint{1.939195in}{2.655572in}}%
\pgfpathlineto{\pgfqpoint{1.867527in}{2.650913in}}%
\pgfpathlineto{\pgfqpoint{1.798171in}{2.644140in}}%
\pgfpathlineto{\pgfqpoint{1.733341in}{2.635606in}}%
\pgfpathlineto{\pgfqpoint{1.673075in}{2.625521in}}%
\pgfpathlineto{\pgfqpoint{1.615274in}{2.613610in}}%
\pgfpathlineto{\pgfqpoint{1.562133in}{2.600402in}}%
\pgfpathlineto{\pgfqpoint{1.513681in}{2.586139in}}%
\pgfpathlineto{\pgfqpoint{1.467862in}{2.570344in}}%
\pgfpathlineto{\pgfqpoint{1.426794in}{2.553923in}}%
\pgfpathlineto{\pgfqpoint{1.388447in}{2.536289in}}%
\pgfpathlineto{\pgfqpoint{1.352878in}{2.517566in}}%
\pgfpathlineto{\pgfqpoint{1.320128in}{2.497922in}}%
\pgfpathlineto{\pgfqpoint{1.288379in}{2.476236in}}%
\pgfpathlineto{\pgfqpoint{1.259592in}{2.453861in}}%
\pgfpathlineto{\pgfqpoint{1.232050in}{2.429520in}}%
\pgfpathlineto{\pgfqpoint{1.207527in}{2.404898in}}%
\pgfpathlineto{\pgfqpoint{1.184409in}{2.378557in}}%
\pgfpathlineto{\pgfqpoint{1.162828in}{2.350561in}}%
\pgfpathlineto{\pgfqpoint{1.142891in}{2.321011in}}%
\pgfpathlineto{\pgfqpoint{1.124675in}{2.290041in}}%
\pgfpathlineto{\pgfqpoint{1.108225in}{2.257802in}}%
\pgfpathlineto{\pgfqpoint{1.092639in}{2.222199in}}%
\pgfpathlineto{\pgfqpoint{1.079059in}{2.185535in}}%
\pgfpathlineto{\pgfqpoint{1.067443in}{2.147998in}}%
\pgfpathlineto{\pgfqpoint{1.057187in}{2.107348in}}%
\pgfpathlineto{\pgfqpoint{1.049004in}{2.066086in}}%
\pgfpathlineto{\pgfqpoint{1.042513in}{2.021906in}}%
\pgfpathlineto{\pgfqpoint{1.038177in}{1.977382in}}%
\pgfpathlineto{\pgfqpoint{1.035866in}{1.930167in}}%
\pgfpathlineto{\pgfqpoint{1.035826in}{1.882878in}}%
\pgfpathlineto{\pgfqpoint{1.038031in}{1.835656in}}%
\pgfpathlineto{\pgfqpoint{1.042474in}{1.788641in}}%
\pgfpathlineto{\pgfqpoint{1.049176in}{1.741979in}}%
\pgfpathlineto{\pgfqpoint{1.057644in}{1.698239in}}%
\pgfpathlineto{\pgfqpoint{1.068221in}{1.655105in}}%
\pgfpathlineto{\pgfqpoint{1.080962in}{1.612745in}}%
\pgfpathlineto{\pgfqpoint{1.095031in}{1.573617in}}%
\pgfpathlineto{\pgfqpoint{1.111115in}{1.535520in}}%
\pgfpathlineto{\pgfqpoint{1.128118in}{1.500775in}}%
\pgfpathlineto{\pgfqpoint{1.146930in}{1.467274in}}%
\pgfpathlineto{\pgfqpoint{1.167531in}{1.435181in}}%
\pgfpathlineto{\pgfqpoint{1.189874in}{1.404652in}}%
\pgfpathlineto{\pgfqpoint{1.213884in}{1.375828in}}%
\pgfpathlineto{\pgfqpoint{1.237817in}{1.350457in}}%
\pgfpathlineto{\pgfqpoint{1.264748in}{1.325237in}}%
\pgfpathlineto{\pgfqpoint{1.292991in}{1.301972in}}%
\pgfpathlineto{\pgfqpoint{1.322398in}{1.280678in}}%
\pgfpathlineto{\pgfqpoint{1.352820in}{1.261340in}}%
\pgfpathlineto{\pgfqpoint{1.386095in}{1.242889in}}%
\pgfpathlineto{\pgfqpoint{1.420190in}{1.226516in}}%
\pgfpathlineto{\pgfqpoint{1.457024in}{1.211329in}}%
\pgfpathlineto{\pgfqpoint{1.496554in}{1.197536in}}%
\pgfpathlineto{\pgfqpoint{1.538719in}{1.185287in}}%
\pgfpathlineto{\pgfqpoint{1.583441in}{1.174641in}}%
\pgfpathlineto{\pgfqpoint{1.634929in}{1.164775in}}%
\pgfpathlineto{\pgfqpoint{1.706063in}{1.153745in}}%
\pgfpathlineto{\pgfqpoint{1.768492in}{1.143417in}}%
\pgfpathlineto{\pgfqpoint{1.796122in}{1.136567in}}%
\pgfpathlineto{\pgfqpoint{1.812683in}{1.130481in}}%
\pgfpathlineto{\pgfqpoint{1.824471in}{1.124102in}}%
\pgfpathlineto{\pgfqpoint{1.833209in}{1.116741in}}%
\pgfpathlineto{\pgfqpoint{1.838498in}{1.108890in}}%
\pgfpathlineto{\pgfqpoint{1.840588in}{1.101849in}}%
\pgfpathlineto{\pgfqpoint{1.840619in}{1.094412in}}%
\pgfpathlineto{\pgfqpoint{1.837931in}{1.084986in}}%
\pgfpathlineto{\pgfqpoint{1.833246in}{1.076615in}}%
\pgfpathlineto{\pgfqpoint{1.825819in}{1.067542in}}%
\pgfpathlineto{\pgfqpoint{1.813813in}{1.056850in}}%
\pgfpathlineto{\pgfqpoint{1.798819in}{1.046763in}}%
\pgfpathlineto{\pgfqpoint{1.781016in}{1.037462in}}%
\pgfpathlineto{\pgfqpoint{1.758447in}{1.028391in}}%
\pgfpathlineto{\pgfqpoint{1.733203in}{1.020815in}}%
\pgfpathlineto{\pgfqpoint{1.705410in}{1.014872in}}%
\pgfpathlineto{\pgfqpoint{1.675178in}{1.010714in}}%
\pgfpathlineto{\pgfqpoint{1.642610in}{1.008507in}}%
\pgfpathlineto{\pgfqpoint{1.607809in}{1.008432in}}%
\pgfpathlineto{\pgfqpoint{1.570886in}{1.010691in}}%
\pgfpathlineto{\pgfqpoint{1.534118in}{1.015181in}}%
\pgfpathlineto{\pgfqpoint{1.495454in}{1.022233in}}%
\pgfpathlineto{\pgfqpoint{1.457161in}{1.031563in}}%
\pgfpathlineto{\pgfqpoint{1.419337in}{1.043132in}}%
\pgfpathlineto{\pgfqpoint{1.382089in}{1.056929in}}%
\pgfpathlineto{\pgfqpoint{1.347544in}{1.072019in}}%
\pgfpathlineto{\pgfqpoint{1.313727in}{1.089133in}}%
\pgfpathlineto{\pgfqpoint{1.280762in}{1.108299in}}%
\pgfpathlineto{\pgfqpoint{1.248782in}{1.129536in}}%
\pgfpathlineto{\pgfqpoint{1.219708in}{1.151422in}}%
\pgfpathlineto{\pgfqpoint{1.191752in}{1.175138in}}%
\pgfpathlineto{\pgfqpoint{1.165031in}{1.200649in}}%
\pgfpathlineto{\pgfqpoint{1.139653in}{1.227898in}}%
\pgfpathlineto{\pgfqpoint{1.115714in}{1.256800in}}%
\pgfpathlineto{\pgfqpoint{1.093288in}{1.287251in}}%
\pgfpathlineto{\pgfqpoint{1.071178in}{1.321163in}}%
\pgfpathlineto{\pgfqpoint{1.050868in}{1.356520in}}%
\pgfpathlineto{\pgfqpoint{1.032365in}{1.393152in}}%
\pgfpathlineto{\pgfqpoint{1.014718in}{1.433142in}}%
\pgfpathlineto{\pgfqpoint{0.999024in}{1.474185in}}%
\pgfpathlineto{\pgfqpoint{0.984506in}{1.518461in}}%
\pgfpathlineto{\pgfqpoint{0.972010in}{1.563537in}}%
\pgfpathlineto{\pgfqpoint{0.960944in}{1.611678in}}%
\pgfpathlineto{\pgfqpoint{0.951530in}{1.662824in}}%
\pgfpathlineto{\pgfqpoint{0.944286in}{1.714431in}}%
\pgfpathlineto{\pgfqpoint{0.938950in}{1.768847in}}%
\pgfpathlineto{\pgfqpoint{0.935870in}{1.823491in}}%
\pgfpathlineto{\pgfqpoint{0.935034in}{1.878240in}}%
\pgfpathlineto{\pgfqpoint{0.936466in}{1.932973in}}%
\pgfpathlineto{\pgfqpoint{0.940005in}{1.985084in}}%
\pgfpathlineto{\pgfqpoint{0.945759in}{2.036935in}}%
\pgfpathlineto{\pgfqpoint{0.953410in}{2.085938in}}%
\pgfpathlineto{\pgfqpoint{0.962764in}{2.132000in}}%
\pgfpathlineto{\pgfqpoint{0.974287in}{2.177414in}}%
\pgfpathlineto{\pgfqpoint{0.987332in}{2.219653in}}%
\pgfpathlineto{\pgfqpoint{1.001667in}{2.258654in}}%
\pgfpathlineto{\pgfqpoint{1.018051in}{2.296583in}}%
\pgfpathlineto{\pgfqpoint{1.035401in}{2.331101in}}%
\pgfpathlineto{\pgfqpoint{1.054650in}{2.364275in}}%
\pgfpathlineto{\pgfqpoint{1.074406in}{2.393984in}}%
\pgfpathlineto{\pgfqpoint{1.095771in}{2.422197in}}%
\pgfpathlineto{\pgfqpoint{1.118662in}{2.448797in}}%
\pgfpathlineto{\pgfqpoint{1.142967in}{2.473701in}}%
\pgfpathlineto{\pgfqpoint{1.168550in}{2.496867in}}%
\pgfpathlineto{\pgfqpoint{1.197085in}{2.519662in}}%
\pgfpathlineto{\pgfqpoint{1.226727in}{2.540526in}}%
\pgfpathlineto{\pgfqpoint{1.259242in}{2.560673in}}%
\pgfpathlineto{\pgfqpoint{1.294612in}{2.579881in}}%
\pgfpathlineto{\pgfqpoint{1.332792in}{2.597982in}}%
\pgfpathlineto{\pgfqpoint{1.373719in}{2.614859in}}%
\pgfpathlineto{\pgfqpoint{1.417319in}{2.630445in}}%
\pgfpathlineto{\pgfqpoint{1.465632in}{2.645312in}}%
\pgfpathlineto{\pgfqpoint{1.518640in}{2.659204in}}%
\pgfpathlineto{\pgfqpoint{1.576309in}{2.671929in}}%
\pgfpathlineto{\pgfqpoint{1.638597in}{2.683344in}}%
\pgfpathlineto{\pgfqpoint{1.705462in}{2.693343in}}%
\pgfpathlineto{\pgfqpoint{1.779027in}{2.702064in}}%
\pgfpathlineto{\pgfqpoint{1.857097in}{2.709077in}}%
\pgfpathlineto{\pgfqpoint{1.939633in}{2.714280in}}%
\pgfpathlineto{\pgfqpoint{2.026598in}{2.717513in}}%
\pgfpathlineto{\pgfqpoint{2.113605in}{2.718523in}}%
\pgfpathlineto{\pgfqpoint{2.198435in}{2.717303in}}%
\pgfpathlineto{\pgfqpoint{2.278866in}{2.713929in}}%
\pgfpathlineto{\pgfqpoint{2.352678in}{2.708598in}}%
\pgfpathlineto{\pgfqpoint{2.417657in}{2.701709in}}%
\pgfpathlineto{\pgfqpoint{2.473770in}{2.693630in}}%
\pgfpathlineto{\pgfqpoint{2.523140in}{2.684368in}}%
\pgfpathlineto{\pgfqpoint{2.565726in}{2.674202in}}%
\pgfpathlineto{\pgfqpoint{2.601510in}{2.663544in}}%
\pgfpathlineto{\pgfqpoint{2.632577in}{2.652142in}}%
\pgfpathlineto{\pgfqpoint{2.658899in}{2.640331in}}%
\pgfpathlineto{\pgfqpoint{2.682438in}{2.627436in}}%
\pgfpathlineto{\pgfqpoint{2.703062in}{2.613571in}}%
\pgfpathlineto{\pgfqpoint{2.720674in}{2.598978in}}%
\pgfpathlineto{\pgfqpoint{2.735263in}{2.584053in}}%
\pgfpathlineto{\pgfqpoint{2.748320in}{2.567377in}}%
\pgfpathlineto{\pgfqpoint{2.759553in}{2.549046in}}%
\pgfpathlineto{\pgfqpoint{2.768788in}{2.529306in}}%
\pgfpathlineto{\pgfqpoint{2.776017in}{2.508498in}}%
\pgfpathlineto{\pgfqpoint{2.781884in}{2.484540in}}%
\pgfpathlineto{\pgfqpoint{2.786102in}{2.457597in}}%
\pgfpathlineto{\pgfqpoint{2.788720in}{2.425384in}}%
\pgfpathlineto{\pgfqpoint{2.789427in}{2.388061in}}%
\pgfpathlineto{\pgfqpoint{2.787962in}{2.340801in}}%
\pgfpathlineto{\pgfqpoint{2.783672in}{2.278768in}}%
\pgfpathlineto{\pgfqpoint{2.774289in}{2.179783in}}%
\pgfpathlineto{\pgfqpoint{2.743611in}{1.868119in}}%
\pgfpathlineto{\pgfqpoint{2.730112in}{1.702060in}}%
\pgfpathlineto{\pgfqpoint{2.717287in}{1.515949in}}%
\pgfpathlineto{\pgfqpoint{2.702602in}{1.267597in}}%
\pgfpathlineto{\pgfqpoint{2.684434in}{0.964630in}}%
\pgfpathlineto{\pgfqpoint{2.675374in}{0.850600in}}%
\pgfpathlineto{\pgfqpoint{2.667030in}{0.771523in}}%
\pgfpathlineto{\pgfqpoint{2.658752in}{0.712543in}}%
\pgfpathlineto{\pgfqpoint{2.650176in}{0.666284in}}%
\pgfpathlineto{\pgfqpoint{2.640820in}{0.627931in}}%
\pgfpathlineto{\pgfqpoint{2.631145in}{0.597534in}}%
\pgfpathlineto{\pgfqpoint{2.621004in}{0.572745in}}%
\pgfpathlineto{\pgfqpoint{2.609856in}{0.551383in}}%
\pgfpathlineto{\pgfqpoint{2.598042in}{0.533534in}}%
\pgfpathlineto{\pgfqpoint{2.584496in}{0.517378in}}%
\pgfpathlineto{\pgfqpoint{2.571109in}{0.504669in}}%
\pgfpathlineto{\pgfqpoint{2.554789in}{0.492313in}}%
\pgfpathlineto{\pgfqpoint{2.537457in}{0.481914in}}%
\pgfpathlineto{\pgfqpoint{2.517374in}{0.472367in}}%
\pgfpathlineto{\pgfqpoint{2.492542in}{0.463178in}}%
\pgfpathlineto{\pgfqpoint{2.462979in}{0.454833in}}%
\pgfpathlineto{\pgfqpoint{2.428766in}{0.447542in}}%
\pgfpathlineto{\pgfqpoint{2.385671in}{0.440735in}}%
\pgfpathlineto{\pgfqpoint{2.331557in}{0.434581in}}%
\pgfpathlineto{\pgfqpoint{2.262115in}{0.429077in}}%
\pgfpathlineto{\pgfqpoint{2.170851in}{0.424236in}}%
\pgfpathlineto{\pgfqpoint{2.049086in}{0.420134in}}%
\pgfpathlineto{\pgfqpoint{1.879436in}{0.416783in}}%
\pgfpathlineto{\pgfqpoint{1.640159in}{0.414418in}}%
\pgfpathlineto{\pgfqpoint{1.322562in}{0.413569in}}%
\pgfpathlineto{\pgfqpoint{1.020194in}{0.414850in}}%
\pgfpathlineto{\pgfqpoint{0.822256in}{0.417715in}}%
\pgfpathlineto{\pgfqpoint{0.704835in}{0.421430in}}%
\pgfpathlineto{\pgfqpoint{0.630976in}{0.425829in}}%
\pgfpathlineto{\pgfqpoint{0.583316in}{0.430734in}}%
\pgfpathlineto{\pgfqpoint{0.551033in}{0.436123in}}%
\pgfpathlineto{\pgfqpoint{0.527708in}{0.442189in}}%
\pgfpathlineto{\pgfqpoint{0.511250in}{0.448625in}}%
\pgfpathlineto{\pgfqpoint{0.499549in}{0.455216in}}%
\pgfpathlineto{\pgfqpoint{0.488916in}{0.463841in}}%
\pgfpathlineto{\pgfqpoint{0.481322in}{0.472730in}}%
\pgfpathlineto{\pgfqpoint{0.474078in}{0.485127in}}%
\pgfpathlineto{\pgfqpoint{0.468753in}{0.498748in}}%
\pgfpathlineto{\pgfqpoint{0.463870in}{0.517848in}}%
\pgfpathlineto{\pgfqpoint{0.459679in}{0.544796in}}%
\pgfpathlineto{\pgfqpoint{0.456386in}{0.581938in}}%
\pgfpathlineto{\pgfqpoint{0.453731in}{0.639106in}}%
\pgfpathlineto{\pgfqpoint{0.451681in}{0.736155in}}%
\pgfpathlineto{\pgfqpoint{0.450220in}{0.927815in}}%
\pgfpathlineto{\pgfqpoint{0.449345in}{1.403252in}}%
\pgfpathlineto{\pgfqpoint{0.449543in}{2.682703in}}%
\pgfpathlineto{\pgfqpoint{0.451011in}{2.856932in}}%
\pgfpathlineto{\pgfqpoint{0.452802in}{2.879219in}}%
\pgfpathlineto{\pgfqpoint{0.455188in}{2.886108in}}%
\pgfpathlineto{\pgfqpoint{0.458626in}{2.889028in}}%
\pgfpathlineto{\pgfqpoint{0.464996in}{2.890553in}}%
\pgfpathlineto{\pgfqpoint{0.482377in}{2.891423in}}%
\pgfpathlineto{\pgfqpoint{0.565038in}{2.891729in}}%
\pgfpathlineto{\pgfqpoint{2.733842in}{2.891760in}}%
\pgfpathlineto{\pgfqpoint{4.789510in}{2.890885in}}%
\pgfpathlineto{\pgfqpoint{4.793727in}{2.889730in}}%
\pgfpathlineto{\pgfqpoint{4.795481in}{2.888307in}}%
\pgfpathlineto{\pgfqpoint{4.797106in}{2.881145in}}%
\pgfpathlineto{\pgfqpoint{4.797997in}{2.858771in}}%
\pgfpathlineto{\pgfqpoint{4.798039in}{2.856283in}}%
\pgfpathlineto{\pgfqpoint{4.798039in}{2.856283in}}%
\pgfusepath{stroke}%
\end{pgfscope}%
\begin{pgfscope}%
\pgfpathrectangle{\pgfqpoint{0.448634in}{0.402556in}}{\pgfqpoint{4.350661in}{2.489204in}} %
\pgfusepath{clip}%
\pgfsetrectcap%
\pgfsetroundjoin%
\pgfsetlinewidth{1.003750pt}%
\definecolor{currentstroke}{rgb}{1.000000,0.498039,0.054902}%
\pgfsetstrokecolor{currentstroke}%
\pgfsetdash{}{0pt}%
\pgfpathmoveto{\pgfqpoint{3.428772in}{0.402610in}}%
\pgfpathlineto{\pgfqpoint{2.806632in}{0.403760in}}%
\pgfpathlineto{\pgfqpoint{2.769692in}{0.405578in}}%
\pgfpathlineto{\pgfqpoint{2.754632in}{0.408064in}}%
\pgfpathlineto{\pgfqpoint{2.746391in}{0.411198in}}%
\pgfpathlineto{\pgfqpoint{2.740943in}{0.415265in}}%
\pgfpathlineto{\pgfqpoint{2.736784in}{0.420984in}}%
\pgfpathlineto{\pgfqpoint{2.733281in}{0.430071in}}%
\pgfpathlineto{\pgfqpoint{2.730449in}{0.444636in}}%
\pgfpathlineto{\pgfqpoint{2.728238in}{0.469392in}}%
\pgfpathlineto{\pgfqpoint{2.726470in}{0.519131in}}%
\pgfpathlineto{\pgfqpoint{2.725711in}{0.613715in}}%
\pgfpathlineto{\pgfqpoint{2.726842in}{0.768038in}}%
\pgfpathlineto{\pgfqpoint{2.730556in}{0.962148in}}%
\pgfpathlineto{\pgfqpoint{2.736611in}{1.158670in}}%
\pgfpathlineto{\pgfqpoint{2.744092in}{1.327718in}}%
\pgfpathlineto{\pgfqpoint{2.753201in}{1.484189in}}%
\pgfpathlineto{\pgfqpoint{2.763257in}{1.620609in}}%
\pgfpathlineto{\pgfqpoint{2.776118in}{1.764216in}}%
\pgfpathlineto{\pgfqpoint{2.788914in}{1.877776in}}%
\pgfpathlineto{\pgfqpoint{2.805748in}{2.005740in}}%
\pgfpathlineto{\pgfqpoint{2.821176in}{2.101198in}}%
\pgfpathlineto{\pgfqpoint{2.838359in}{2.193718in}}%
\pgfpathlineto{\pgfqpoint{2.859135in}{2.292966in}}%
\pgfpathlineto{\pgfqpoint{2.887209in}{2.425960in}}%
\pgfpathlineto{\pgfqpoint{2.896991in}{2.479559in}}%
\pgfpathlineto{\pgfqpoint{2.901543in}{2.516523in}}%
\pgfpathlineto{\pgfqpoint{2.902849in}{2.543854in}}%
\pgfpathlineto{\pgfqpoint{2.901957in}{2.566223in}}%
\pgfpathlineto{\pgfqpoint{2.899151in}{2.585863in}}%
\pgfpathlineto{\pgfqpoint{2.894794in}{2.602546in}}%
\pgfpathlineto{\pgfqpoint{2.888484in}{2.618388in}}%
\pgfpathlineto{\pgfqpoint{2.880257in}{2.633033in}}%
\pgfpathlineto{\pgfqpoint{2.870348in}{2.646246in}}%
\pgfpathlineto{\pgfqpoint{2.857400in}{2.659530in}}%
\pgfpathlineto{\pgfqpoint{2.843189in}{2.671010in}}%
\pgfpathlineto{\pgfqpoint{2.824237in}{2.683209in}}%
\pgfpathlineto{\pgfqpoint{2.802413in}{2.694418in}}%
\pgfpathlineto{\pgfqpoint{2.775809in}{2.705369in}}%
\pgfpathlineto{\pgfqpoint{2.744461in}{2.715715in}}%
\pgfpathlineto{\pgfqpoint{2.708436in}{2.725252in}}%
\pgfpathlineto{\pgfqpoint{2.665655in}{2.734289in}}%
\pgfpathlineto{\pgfqpoint{2.613991in}{2.742869in}}%
\pgfpathlineto{\pgfqpoint{2.553459in}{2.750589in}}%
\pgfpathlineto{\pgfqpoint{2.481920in}{2.757365in}}%
\pgfpathlineto{\pgfqpoint{2.399398in}{2.762839in}}%
\pgfpathlineto{\pgfqpoint{2.310269in}{2.766482in}}%
\pgfpathlineto{\pgfqpoint{2.175416in}{2.768725in}}%
\pgfpathlineto{\pgfqpoint{2.066653in}{2.767942in}}%
\pgfpathlineto{\pgfqpoint{1.953570in}{2.764859in}}%
\pgfpathlineto{\pgfqpoint{1.851429in}{2.759759in}}%
\pgfpathlineto{\pgfqpoint{1.745051in}{2.752169in}}%
\pgfpathlineto{\pgfqpoint{1.658373in}{2.743453in}}%
\pgfpathlineto{\pgfqpoint{1.580552in}{2.733461in}}%
\pgfpathlineto{\pgfqpoint{1.490057in}{2.719338in}}%
\pgfpathlineto{\pgfqpoint{1.417231in}{2.704698in}}%
\pgfpathlineto{\pgfqpoint{1.361992in}{2.690818in}}%
\pgfpathlineto{\pgfqpoint{1.311460in}{2.675819in}}%
\pgfpathlineto{\pgfqpoint{1.265667in}{2.659924in}}%
\pgfpathlineto{\pgfqpoint{1.222575in}{2.642586in}}%
\pgfpathlineto{\pgfqpoint{1.184324in}{2.624682in}}%
\pgfpathlineto{\pgfqpoint{1.148892in}{2.605623in}}%
\pgfpathlineto{\pgfqpoint{1.116331in}{2.585573in}}%
\pgfpathlineto{\pgfqpoint{1.092327in}{2.568512in}}%
\pgfpathlineto{\pgfqpoint{1.079760in}{2.558686in}}%
\pgfpathlineto{\pgfqpoint{1.051544in}{2.535379in}}%
\pgfpathlineto{\pgfqpoint{1.026312in}{2.511712in}}%
\pgfpathlineto{\pgfqpoint{1.002399in}{2.486318in}}%
\pgfpathlineto{\pgfqpoint{0.979913in}{2.459269in}}%
\pgfpathlineto{\pgfqpoint{0.958934in}{2.430678in}}%
\pgfpathlineto{\pgfqpoint{0.938264in}{2.398643in}}%
\pgfpathlineto{\pgfqpoint{0.923047in}{2.371385in}}%
\pgfpathlineto{\pgfqpoint{0.904513in}{2.334774in}}%
\pgfpathlineto{\pgfqpoint{0.887854in}{2.297001in}}%
\pgfpathlineto{\pgfqpoint{0.872131in}{2.255971in}}%
\pgfpathlineto{\pgfqpoint{0.857508in}{2.211741in}}%
\pgfpathlineto{\pgfqpoint{0.844762in}{2.166757in}}%
\pgfpathlineto{\pgfqpoint{0.838624in}{2.140306in}}%
\pgfpathlineto{\pgfqpoint{0.826982in}{2.087194in}}%
\pgfpathlineto{\pgfqpoint{0.816322in}{2.028715in}}%
\pgfpathlineto{\pgfqpoint{0.810087in}{1.984495in}}%
\pgfpathlineto{\pgfqpoint{0.808026in}{1.967238in}}%
\pgfpathlineto{\pgfqpoint{0.800076in}{1.898140in}}%
\pgfpathlineto{\pgfqpoint{0.793713in}{1.823823in}}%
\pgfpathlineto{\pgfqpoint{0.788799in}{1.741875in}}%
\pgfpathlineto{\pgfqpoint{0.786199in}{1.677225in}}%
\pgfpathlineto{\pgfqpoint{0.776951in}{1.453481in}}%
\pgfpathlineto{\pgfqpoint{0.773280in}{1.418894in}}%
\pgfpathlineto{\pgfqpoint{0.768298in}{1.389582in}}%
\pgfpathlineto{\pgfqpoint{0.762752in}{1.368108in}}%
\pgfpathlineto{\pgfqpoint{0.756722in}{1.352123in}}%
\pgfpathlineto{\pgfqpoint{0.749752in}{1.339519in}}%
\pgfpathlineto{\pgfqpoint{0.742201in}{1.330599in}}%
\pgfpathlineto{\pgfqpoint{0.734854in}{1.325312in}}%
\pgfpathlineto{\pgfqpoint{0.726558in}{1.322419in}}%
\pgfpathlineto{\pgfqpoint{0.717884in}{1.322223in}}%
\pgfpathlineto{\pgfqpoint{0.709412in}{1.324411in}}%
\pgfpathlineto{\pgfqpoint{0.699548in}{1.329604in}}%
\pgfpathlineto{\pgfqpoint{0.688894in}{1.338203in}}%
\pgfpathlineto{\pgfqpoint{0.677907in}{1.350248in}}%
\pgfpathlineto{\pgfqpoint{0.666886in}{1.365647in}}%
\pgfpathlineto{\pgfqpoint{0.654913in}{1.386417in}}%
\pgfpathlineto{\pgfqpoint{0.642574in}{1.412730in}}%
\pgfpathlineto{\pgfqpoint{0.630328in}{1.444629in}}%
\pgfpathlineto{\pgfqpoint{0.618504in}{1.482081in}}%
\pgfpathlineto{\pgfqpoint{0.608613in}{1.520256in}}%
\pgfpathlineto{\pgfqpoint{0.590203in}{1.612445in}}%
\pgfpathlineto{\pgfqpoint{0.581848in}{1.668884in}}%
\pgfpathlineto{\pgfqpoint{0.573137in}{1.740376in}}%
\pgfpathlineto{\pgfqpoint{0.567062in}{1.807213in}}%
\pgfpathlineto{\pgfqpoint{0.560532in}{1.896510in}}%
\pgfpathlineto{\pgfqpoint{0.555526in}{1.995910in}}%
\pgfpathlineto{\pgfqpoint{0.552564in}{2.097908in}}%
\pgfpathlineto{\pgfqpoint{0.551526in}{2.204935in}}%
\pgfpathlineto{\pgfqpoint{0.552728in}{2.309470in}}%
\pgfpathlineto{\pgfqpoint{0.556011in}{2.403981in}}%
\pgfpathlineto{\pgfqpoint{0.560953in}{2.483430in}}%
\pgfpathlineto{\pgfqpoint{0.567303in}{2.550240in}}%
\pgfpathlineto{\pgfqpoint{0.574928in}{2.606817in}}%
\pgfpathlineto{\pgfqpoint{0.582988in}{2.650657in}}%
\pgfpathlineto{\pgfqpoint{0.592756in}{2.691452in}}%
\pgfpathlineto{\pgfqpoint{0.602650in}{2.721756in}}%
\pgfpathlineto{\pgfqpoint{0.612983in}{2.746441in}}%
\pgfpathlineto{\pgfqpoint{0.624292in}{2.767692in}}%
\pgfpathlineto{\pgfqpoint{0.636231in}{2.785433in}}%
\pgfpathlineto{\pgfqpoint{0.649892in}{2.801461in}}%
\pgfpathlineto{\pgfqpoint{0.663386in}{2.814020in}}%
\pgfpathlineto{\pgfqpoint{0.679842in}{2.826135in}}%
\pgfpathlineto{\pgfqpoint{0.697326in}{2.836197in}}%
\pgfpathlineto{\pgfqpoint{0.715574in}{2.844285in}}%
\pgfpathlineto{\pgfqpoint{0.738439in}{2.852335in}}%
\pgfpathlineto{\pgfqpoint{0.765983in}{2.859639in}}%
\pgfpathlineto{\pgfqpoint{0.800300in}{2.866256in}}%
\pgfpathlineto{\pgfqpoint{0.841340in}{2.871832in}}%
\pgfpathlineto{\pgfqpoint{0.895547in}{2.876803in}}%
\pgfpathlineto{\pgfqpoint{0.969413in}{2.881069in}}%
\pgfpathlineto{\pgfqpoint{1.071608in}{2.884501in}}%
\pgfpathlineto{\pgfqpoint{1.219512in}{2.887074in}}%
\pgfpathlineto{\pgfqpoint{1.471844in}{2.889091in}}%
\pgfpathlineto{\pgfqpoint{1.956941in}{2.890384in}}%
\pgfpathlineto{\pgfqpoint{3.096814in}{2.890781in}}%
\pgfpathlineto{\pgfqpoint{3.995224in}{2.889388in}}%
\pgfpathlineto{\pgfqpoint{4.275833in}{2.887011in}}%
\pgfpathlineto{\pgfqpoint{4.412847in}{2.883743in}}%
\pgfpathlineto{\pgfqpoint{4.491081in}{2.879810in}}%
\pgfpathlineto{\pgfqpoint{4.543127in}{2.875163in}}%
\pgfpathlineto{\pgfqpoint{4.579810in}{2.869841in}}%
\pgfpathlineto{\pgfqpoint{4.607580in}{2.863763in}}%
\pgfpathlineto{\pgfqpoint{4.630623in}{2.856424in}}%
\pgfpathlineto{\pgfqpoint{4.648833in}{2.848228in}}%
\pgfpathlineto{\pgfqpoint{4.664136in}{2.838773in}}%
\pgfpathlineto{\pgfqpoint{4.676470in}{2.828576in}}%
\pgfpathlineto{\pgfqpoint{4.687502in}{2.816585in}}%
\pgfpathlineto{\pgfqpoint{4.697051in}{2.803027in}}%
\pgfpathlineto{\pgfqpoint{4.706194in}{2.786098in}}%
\pgfpathlineto{\pgfqpoint{4.714508in}{2.765827in}}%
\pgfpathlineto{\pgfqpoint{4.722462in}{2.740013in}}%
\pgfpathlineto{\pgfqpoint{4.729577in}{2.708703in}}%
\pgfpathlineto{\pgfqpoint{4.736162in}{2.669601in}}%
\pgfpathlineto{\pgfqpoint{4.742419in}{2.617826in}}%
\pgfpathlineto{\pgfqpoint{4.747859in}{2.553410in}}%
\pgfpathlineto{\pgfqpoint{4.752661in}{2.468958in}}%
\pgfpathlineto{\pgfqpoint{4.756610in}{2.359528in}}%
\pgfpathlineto{\pgfqpoint{4.759416in}{2.217681in}}%
\pgfpathlineto{\pgfqpoint{4.760596in}{2.043444in}}%
\pgfpathlineto{\pgfqpoint{4.759662in}{1.851779in}}%
\pgfpathlineto{\pgfqpoint{4.756587in}{1.667613in}}%
\pgfpathlineto{\pgfqpoint{4.751596in}{1.503428in}}%
\pgfpathlineto{\pgfqpoint{4.745410in}{1.374185in}}%
\pgfpathlineto{\pgfqpoint{4.738113in}{1.267479in}}%
\pgfpathlineto{\pgfqpoint{4.729621in}{1.175896in}}%
\pgfpathlineto{\pgfqpoint{4.720762in}{1.104428in}}%
\pgfpathlineto{\pgfqpoint{4.711045in}{1.043204in}}%
\pgfpathlineto{\pgfqpoint{4.700364in}{0.989829in}}%
\pgfpathlineto{\pgfqpoint{4.689055in}{0.944345in}}%
\pgfpathlineto{\pgfqpoint{4.676881in}{0.904394in}}%
\pgfpathlineto{\pgfqpoint{4.676095in}{0.902073in}}%
\pgfpathlineto{\pgfqpoint{4.676095in}{0.902073in}}%
\pgfusepath{stroke}%
\end{pgfscope}%
\begin{pgfscope}%
\pgfpathrectangle{\pgfqpoint{0.448634in}{0.402556in}}{\pgfqpoint{4.350661in}{2.489204in}} %
\pgfusepath{clip}%
\pgfsetrectcap%
\pgfsetroundjoin%
\pgfsetlinewidth{1.003750pt}%
\definecolor{currentstroke}{rgb}{1.000000,0.498039,0.054902}%
\pgfsetstrokecolor{currentstroke}%
\pgfsetdash{}{0pt}%
\pgfpathmoveto{\pgfqpoint{2.795520in}{1.982745in}}%
\pgfpathlineto{\pgfqpoint{2.781780in}{1.874357in}}%
\pgfpathlineto{\pgfqpoint{2.769351in}{1.758234in}}%
\pgfpathlineto{\pgfqpoint{2.758095in}{1.631942in}}%
\pgfpathlineto{\pgfqpoint{2.747786in}{1.490551in}}%
\pgfpathlineto{\pgfqpoint{2.738644in}{1.334082in}}%
\pgfpathlineto{\pgfqpoint{2.730580in}{1.157591in}}%
\pgfpathlineto{\pgfqpoint{2.723334in}{0.948663in}}%
\pgfpathlineto{\pgfqpoint{2.709783in}{0.530788in}}%
\pgfpathlineto{\pgfqpoint{2.705868in}{0.488716in}}%
\pgfpathlineto{\pgfqpoint{2.701769in}{0.464281in}}%
\pgfpathlineto{\pgfqpoint{2.697021in}{0.447744in}}%
\pgfpathlineto{\pgfqpoint{2.691859in}{0.436812in}}%
\pgfpathlineto{\pgfqpoint{2.686245in}{0.429229in}}%
\pgfpathlineto{\pgfqpoint{2.679348in}{0.423188in}}%
\pgfpathlineto{\pgfqpoint{2.669540in}{0.417856in}}%
\pgfpathlineto{\pgfqpoint{2.656987in}{0.413810in}}%
\pgfpathlineto{\pgfqpoint{2.637654in}{0.410337in}}%
\pgfpathlineto{\pgfqpoint{2.607297in}{0.407617in}}%
\pgfpathlineto{\pgfqpoint{2.555121in}{0.405574in}}%
\pgfpathlineto{\pgfqpoint{2.450714in}{0.404139in}}%
\pgfpathlineto{\pgfqpoint{2.176624in}{0.403275in}}%
\pgfpathlineto{\pgfqpoint{1.130290in}{0.402953in}}%
\pgfpathlineto{\pgfqpoint{0.516849in}{0.404175in}}%
\pgfpathlineto{\pgfqpoint{0.466848in}{0.405970in}}%
\pgfpathlineto{\pgfqpoint{0.456130in}{0.407931in}}%
\pgfpathlineto{\pgfqpoint{0.452340in}{0.410303in}}%
\pgfpathlineto{\pgfqpoint{0.450346in}{0.414662in}}%
\pgfpathlineto{\pgfqpoint{0.449266in}{0.424524in}}%
\pgfpathlineto{\pgfqpoint{0.448771in}{0.464344in}}%
\pgfpathlineto{\pgfqpoint{0.448640in}{0.850171in}}%
\pgfpathlineto{\pgfqpoint{0.448679in}{2.891318in}}%
\pgfpathlineto{\pgfqpoint{0.448679in}{2.891318in}}%
\pgfusepath{stroke}%
\end{pgfscope}%
\begin{pgfscope}%
\pgfpathrectangle{\pgfqpoint{0.448634in}{0.402556in}}{\pgfqpoint{4.350661in}{2.489204in}} %
\pgfusepath{clip}%
\pgfsetrectcap%
\pgfsetroundjoin%
\pgfsetlinewidth{1.003750pt}%
\definecolor{currentstroke}{rgb}{1.000000,0.498039,0.054902}%
\pgfsetstrokecolor{currentstroke}%
\pgfsetdash{}{0pt}%
\pgfpathmoveto{\pgfqpoint{3.428189in}{0.402586in}}%
\pgfpathlineto{\pgfqpoint{2.782121in}{0.403701in}}%
\pgfpathlineto{\pgfqpoint{2.753906in}{0.405674in}}%
\pgfpathlineto{\pgfqpoint{2.743328in}{0.408443in}}%
\pgfpathlineto{\pgfqpoint{2.737717in}{0.412188in}}%
\pgfpathlineto{\pgfqpoint{2.733668in}{0.417995in}}%
\pgfpathlineto{\pgfqpoint{2.730649in}{0.427307in}}%
\pgfpathlineto{\pgfqpoint{2.728388in}{0.442004in}}%
\pgfpathlineto{\pgfqpoint{2.726544in}{0.471794in}}%
\pgfpathlineto{\pgfqpoint{2.725216in}{0.534003in}}%
\pgfpathlineto{\pgfqpoint{2.725169in}{0.655973in}}%
\pgfpathlineto{\pgfqpoint{2.727377in}{0.832687in}}%
\pgfpathlineto{\pgfqpoint{2.732259in}{1.041703in}}%
\pgfpathlineto{\pgfqpoint{2.738851in}{1.223257in}}%
\pgfpathlineto{\pgfqpoint{2.747078in}{1.389766in}}%
\pgfpathlineto{\pgfqpoint{2.756608in}{1.538717in}}%
\pgfpathlineto{\pgfqpoint{2.768955in}{1.694887in}}%
\pgfpathlineto{\pgfqpoint{2.781228in}{1.816044in}}%
\pgfpathlineto{\pgfqpoint{2.794401in}{1.924524in}}%
\pgfpathlineto{\pgfqpoint{2.812737in}{2.054722in}}%
\pgfpathlineto{\pgfqpoint{2.828774in}{2.147512in}}%
\pgfpathlineto{\pgfqpoint{2.847382in}{2.242224in}}%
\pgfpathlineto{\pgfqpoint{2.895818in}{2.479699in}}%
\pgfpathlineto{\pgfqpoint{2.900204in}{2.516689in}}%
\pgfpathlineto{\pgfqpoint{2.901346in}{2.544029in}}%
\pgfpathlineto{\pgfqpoint{2.900291in}{2.566388in}}%
\pgfpathlineto{\pgfqpoint{2.897334in}{2.585999in}}%
\pgfpathlineto{\pgfqpoint{2.892836in}{2.602633in}}%
\pgfpathlineto{\pgfqpoint{2.886394in}{2.618405in}}%
\pgfpathlineto{\pgfqpoint{2.878058in}{2.632969in}}%
\pgfpathlineto{\pgfqpoint{2.868065in}{2.646100in}}%
\pgfpathlineto{\pgfqpoint{2.855050in}{2.659300in}}%
\pgfpathlineto{\pgfqpoint{2.840801in}{2.670717in}}%
\pgfpathlineto{\pgfqpoint{2.821822in}{2.682861in}}%
\pgfpathlineto{\pgfqpoint{2.799980in}{2.694026in}}%
\pgfpathlineto{\pgfqpoint{2.773366in}{2.704944in}}%
\pgfpathlineto{\pgfqpoint{2.742012in}{2.715266in}}%
\pgfpathlineto{\pgfqpoint{2.705983in}{2.724785in}}%
\pgfpathlineto{\pgfqpoint{2.663200in}{2.733810in}}%
\pgfpathlineto{\pgfqpoint{2.611535in}{2.742379in}}%
\pgfpathlineto{\pgfqpoint{2.551002in}{2.750090in}}%
\pgfpathlineto{\pgfqpoint{2.481632in}{2.756682in}}%
\pgfpathlineto{\pgfqpoint{2.399112in}{2.762200in}}%
\pgfpathlineto{\pgfqpoint{2.309985in}{2.765886in}}%
\pgfpathlineto{\pgfqpoint{2.188184in}{2.768096in}}%
\pgfpathlineto{\pgfqpoint{2.081595in}{2.767619in}}%
\pgfpathlineto{\pgfqpoint{1.968506in}{2.764840in}}%
\pgfpathlineto{\pgfqpoint{1.864180in}{2.759918in}}%
\pgfpathlineto{\pgfqpoint{1.757786in}{2.752593in}}%
\pgfpathlineto{\pgfqpoint{1.671087in}{2.744171in}}%
\pgfpathlineto{\pgfqpoint{1.591076in}{2.734193in}}%
\pgfpathlineto{\pgfqpoint{1.502689in}{2.720717in}}%
\pgfpathlineto{\pgfqpoint{1.427655in}{2.706083in}}%
\pgfpathlineto{\pgfqpoint{1.372350in}{2.692544in}}%
\pgfpathlineto{\pgfqpoint{1.321734in}{2.677921in}}%
\pgfpathlineto{\pgfqpoint{1.273765in}{2.661664in}}%
\pgfpathlineto{\pgfqpoint{1.230567in}{2.644672in}}%
\pgfpathlineto{\pgfqpoint{1.192197in}{2.627106in}}%
\pgfpathlineto{\pgfqpoint{1.156620in}{2.608403in}}%
\pgfpathlineto{\pgfqpoint{1.123890in}{2.588716in}}%
\pgfpathlineto{\pgfqpoint{1.095883in}{2.569568in}}%
\pgfpathlineto{\pgfqpoint{1.063936in}{2.543701in}}%
\pgfpathlineto{\pgfqpoint{1.038217in}{2.520732in}}%
\pgfpathlineto{\pgfqpoint{1.013766in}{2.496016in}}%
\pgfpathlineto{\pgfqpoint{0.990704in}{2.469610in}}%
\pgfpathlineto{\pgfqpoint{0.969124in}{2.441612in}}%
\pgfpathlineto{\pgfqpoint{0.949083in}{2.412154in}}%
\pgfpathlineto{\pgfqpoint{0.930604in}{2.381387in}}%
\pgfpathlineto{\pgfqpoint{0.906555in}{2.334052in}}%
\pgfpathlineto{\pgfqpoint{0.889925in}{2.296262in}}%
\pgfpathlineto{\pgfqpoint{0.874241in}{2.255213in}}%
\pgfpathlineto{\pgfqpoint{0.859667in}{2.210961in}}%
\pgfpathlineto{\pgfqpoint{0.846986in}{2.165954in}}%
\pgfpathlineto{\pgfqpoint{0.839633in}{2.134715in}}%
\pgfpathlineto{\pgfqpoint{0.828238in}{2.081532in}}%
\pgfpathlineto{\pgfqpoint{0.817866in}{2.022986in}}%
\pgfpathlineto{\pgfqpoint{0.810784in}{1.971352in}}%
\pgfpathlineto{\pgfqpoint{0.802846in}{1.902252in}}%
\pgfpathlineto{\pgfqpoint{0.796554in}{1.827927in}}%
\pgfpathlineto{\pgfqpoint{0.791696in}{1.743480in}}%
\pgfpathlineto{\pgfqpoint{0.787773in}{1.621595in}}%
\pgfpathlineto{\pgfqpoint{0.785408in}{1.522064in}}%
\pgfpathlineto{\pgfqpoint{0.785408in}{1.522064in}}%
\pgfusepath{stroke}%
\end{pgfscope}%
\begin{pgfscope}%
\pgfpathrectangle{\pgfqpoint{0.448634in}{0.402556in}}{\pgfqpoint{4.350661in}{2.489204in}} %
\pgfusepath{clip}%
\pgfsetrectcap%
\pgfsetroundjoin%
\pgfsetlinewidth{1.003750pt}%
\definecolor{currentstroke}{rgb}{0.172549,0.627451,0.172549}%
\pgfsetstrokecolor{currentstroke}%
\pgfsetdash{}{0pt}%
\pgfpathmoveto{\pgfqpoint{0.448634in}{2.896245in}}%
\pgfpathlineto{\pgfqpoint{0.448593in}{0.407043in}}%
\pgfpathlineto{\pgfqpoint{0.448593in}{0.407043in}}%
\pgfusepath{stroke}%
\end{pgfscope}%
\begin{pgfscope}%
\pgfpathrectangle{\pgfqpoint{0.448634in}{0.402556in}}{\pgfqpoint{4.350661in}{2.489204in}} %
\pgfusepath{clip}%
\pgfsetrectcap%
\pgfsetroundjoin%
\pgfsetlinewidth{1.003750pt}%
\definecolor{currentstroke}{rgb}{0.172549,0.627451,0.172549}%
\pgfsetstrokecolor{currentstroke}%
\pgfsetdash{}{0pt}%
\pgfpathmoveto{\pgfqpoint{0.576853in}{1.760817in}}%
\pgfpathlineto{\pgfqpoint{0.569394in}{1.840010in}}%
\pgfpathlineto{\pgfqpoint{0.563209in}{1.929338in}}%
\pgfpathlineto{\pgfqpoint{0.558592in}{2.028764in}}%
\pgfpathlineto{\pgfqpoint{0.555985in}{2.133265in}}%
\pgfpathlineto{\pgfqpoint{0.555566in}{2.237808in}}%
\pgfpathlineto{\pgfqpoint{0.557371in}{2.337352in}}%
\pgfpathlineto{\pgfqpoint{0.561096in}{2.424366in}}%
\pgfpathlineto{\pgfqpoint{0.566403in}{2.498791in}}%
\pgfpathlineto{\pgfqpoint{0.572909in}{2.560570in}}%
\pgfpathlineto{\pgfqpoint{0.580458in}{2.612119in}}%
\pgfpathlineto{\pgfqpoint{0.589086in}{2.655816in}}%
\pgfpathlineto{\pgfqpoint{0.598406in}{2.691589in}}%
\pgfpathlineto{\pgfqpoint{0.608613in}{2.721757in}}%
\pgfpathlineto{\pgfqpoint{0.619241in}{2.746278in}}%
\pgfpathlineto{\pgfqpoint{0.630817in}{2.767339in}}%
\pgfpathlineto{\pgfqpoint{0.642975in}{2.784884in}}%
\pgfpathlineto{\pgfqpoint{0.656813in}{2.800712in}}%
\pgfpathlineto{\pgfqpoint{0.672197in}{2.814549in}}%
\pgfpathlineto{\pgfqpoint{0.688853in}{2.826301in}}%
\pgfpathlineto{\pgfqpoint{0.706461in}{2.836076in}}%
\pgfpathlineto{\pgfqpoint{0.726804in}{2.844875in}}%
\pgfpathlineto{\pgfqpoint{0.751866in}{2.853203in}}%
\pgfpathlineto{\pgfqpoint{0.781631in}{2.860547in}}%
\pgfpathlineto{\pgfqpoint{0.818168in}{2.867054in}}%
\pgfpathlineto{\pgfqpoint{0.863581in}{2.872685in}}%
\pgfpathlineto{\pgfqpoint{0.922161in}{2.877518in}}%
\pgfpathlineto{\pgfqpoint{1.000391in}{2.881567in}}%
\pgfpathlineto{\pgfqpoint{1.111294in}{2.884881in}}%
\pgfpathlineto{\pgfqpoint{1.274428in}{2.887367in}}%
\pgfpathlineto{\pgfqpoint{1.552865in}{2.889263in}}%
\pgfpathlineto{\pgfqpoint{2.107573in}{2.890457in}}%
\pgfpathlineto{\pgfqpoint{3.343161in}{2.890573in}}%
\pgfpathlineto{\pgfqpoint{4.043615in}{2.888941in}}%
\pgfpathlineto{\pgfqpoint{4.289417in}{2.886404in}}%
\pgfpathlineto{\pgfqpoint{4.413375in}{2.883093in}}%
\pgfpathlineto{\pgfqpoint{4.489424in}{2.878997in}}%
\pgfpathlineto{\pgfqpoint{4.541451in}{2.874081in}}%
\pgfpathlineto{\pgfqpoint{4.578100in}{2.868470in}}%
\pgfpathlineto{\pgfqpoint{4.605818in}{2.862092in}}%
\pgfpathlineto{\pgfqpoint{4.626725in}{2.855245in}}%
\pgfpathlineto{\pgfqpoint{4.644925in}{2.847018in}}%
\pgfpathlineto{\pgfqpoint{4.660241in}{2.837590in}}%
\pgfpathlineto{\pgfqpoint{4.672623in}{2.827468in}}%
\pgfpathlineto{\pgfqpoint{4.683751in}{2.815592in}}%
\pgfpathlineto{\pgfqpoint{4.693406in}{2.802135in}}%
\pgfpathlineto{\pgfqpoint{4.702740in}{2.785343in}}%
\pgfpathlineto{\pgfqpoint{4.711277in}{2.765194in}}%
\pgfpathlineto{\pgfqpoint{4.719482in}{2.739484in}}%
\pgfpathlineto{\pgfqpoint{4.726293in}{2.710657in}}%
\pgfpathlineto{\pgfqpoint{4.733259in}{2.671643in}}%
\pgfpathlineto{\pgfqpoint{4.739604in}{2.622396in}}%
\pgfpathlineto{\pgfqpoint{4.745236in}{2.560504in}}%
\pgfpathlineto{\pgfqpoint{4.750164in}{2.481052in}}%
\pgfpathlineto{\pgfqpoint{4.754367in}{2.376618in}}%
\pgfpathlineto{\pgfqpoint{4.757443in}{2.242249in}}%
\pgfpathlineto{\pgfqpoint{4.758977in}{2.075483in}}%
\pgfpathlineto{\pgfqpoint{4.758447in}{1.888795in}}%
\pgfpathlineto{\pgfqpoint{4.755756in}{1.707111in}}%
\pgfpathlineto{\pgfqpoint{4.750925in}{1.532957in}}%
\pgfpathlineto{\pgfqpoint{4.744785in}{1.398726in}}%
\pgfpathlineto{\pgfqpoint{4.737575in}{1.289516in}}%
\pgfpathlineto{\pgfqpoint{4.728714in}{1.190470in}}%
\pgfpathlineto{\pgfqpoint{4.719652in}{1.116521in}}%
\pgfpathlineto{\pgfqpoint{4.710036in}{1.055276in}}%
\pgfpathlineto{\pgfqpoint{4.699503in}{1.001861in}}%
\pgfpathlineto{\pgfqpoint{4.689040in}{0.958690in}}%
\pgfpathlineto{\pgfqpoint{4.677220in}{0.918600in}}%
\pgfpathlineto{\pgfqpoint{4.664034in}{0.881749in}}%
\pgfpathlineto{\pgfqpoint{4.650584in}{0.850492in}}%
\pgfpathlineto{\pgfqpoint{4.636303in}{0.822570in}}%
\pgfpathlineto{\pgfqpoint{4.620207in}{0.795974in}}%
\pgfpathlineto{\pgfqpoint{4.603640in}{0.772901in}}%
\pgfpathlineto{\pgfqpoint{4.585488in}{0.751446in}}%
\pgfpathlineto{\pgfqpoint{4.565874in}{0.731749in}}%
\pgfpathlineto{\pgfqpoint{4.544964in}{0.713879in}}%
\pgfpathlineto{\pgfqpoint{4.522958in}{0.697824in}}%
\pgfpathlineto{\pgfqpoint{4.496157in}{0.681290in}}%
\pgfpathlineto{\pgfqpoint{4.470397in}{0.667953in}}%
\pgfpathlineto{\pgfqpoint{4.439961in}{0.654509in}}%
\pgfpathlineto{\pgfqpoint{4.406841in}{0.642281in}}%
\pgfpathlineto{\pgfqpoint{4.369009in}{0.630748in}}%
\pgfpathlineto{\pgfqpoint{4.326489in}{0.620226in}}%
\pgfpathlineto{\pgfqpoint{4.279327in}{0.610949in}}%
\pgfpathlineto{\pgfqpoint{4.227576in}{0.603085in}}%
\pgfpathlineto{\pgfqpoint{4.173450in}{0.597063in}}%
\pgfpathlineto{\pgfqpoint{4.110511in}{0.592203in}}%
\pgfpathlineto{\pgfqpoint{4.047471in}{0.589537in}}%
\pgfpathlineto{\pgfqpoint{3.977867in}{0.588624in}}%
\pgfpathlineto{\pgfqpoint{3.906093in}{0.589934in}}%
\pgfpathlineto{\pgfqpoint{3.834377in}{0.593496in}}%
\pgfpathlineto{\pgfqpoint{3.767120in}{0.599067in}}%
\pgfpathlineto{\pgfqpoint{3.704364in}{0.606392in}}%
\pgfpathlineto{\pgfqpoint{3.678516in}{0.610510in}}%
\pgfpathlineto{\pgfqpoint{3.620438in}{0.620500in}}%
\pgfpathlineto{\pgfqpoint{3.586319in}{0.628207in}}%
\pgfpathlineto{\pgfqpoint{3.495240in}{0.652428in}}%
\pgfpathlineto{\pgfqpoint{3.451528in}{0.667583in}}%
\pgfpathlineto{\pgfqpoint{3.408538in}{0.685220in}}%
\pgfpathlineto{\pgfqpoint{3.374594in}{0.702001in}}%
\pgfpathlineto{\pgfqpoint{3.345407in}{0.718682in}}%
\pgfpathlineto{\pgfqpoint{3.315236in}{0.738520in}}%
\pgfpathlineto{\pgfqpoint{3.288127in}{0.759290in}}%
\pgfpathlineto{\pgfqpoint{3.264004in}{0.780551in}}%
\pgfpathlineto{\pgfqpoint{3.241208in}{0.803648in}}%
\pgfpathlineto{\pgfqpoint{3.219894in}{0.828530in}}%
\pgfpathlineto{\pgfqpoint{3.200189in}{0.855091in}}%
\pgfpathlineto{\pgfqpoint{3.182177in}{0.883182in}}%
\pgfpathlineto{\pgfqpoint{3.165906in}{0.912633in}}%
\pgfpathlineto{\pgfqpoint{3.150351in}{0.945448in}}%
\pgfpathlineto{\pgfqpoint{3.136682in}{0.979345in}}%
\pgfpathlineto{\pgfqpoint{3.124073in}{1.016460in}}%
\pgfpathlineto{\pgfqpoint{3.112834in}{1.056769in}}%
\pgfpathlineto{\pgfqpoint{3.103046in}{1.100146in}}%
\pgfpathlineto{\pgfqpoint{3.095343in}{1.144071in}}%
\pgfpathlineto{\pgfqpoint{3.089208in}{1.190837in}}%
\pgfpathlineto{\pgfqpoint{3.084595in}{1.242838in}}%
\pgfpathlineto{\pgfqpoint{3.082137in}{1.295031in}}%
\pgfpathlineto{\pgfqpoint{3.081687in}{1.349787in}}%
\pgfpathlineto{\pgfqpoint{3.083451in}{1.406998in}}%
\pgfpathlineto{\pgfqpoint{3.087181in}{1.461589in}}%
\pgfpathlineto{\pgfqpoint{3.093485in}{1.520888in}}%
\pgfpathlineto{\pgfqpoint{3.101823in}{1.577334in}}%
\pgfpathlineto{\pgfqpoint{3.111930in}{1.630856in}}%
\pgfpathlineto{\pgfqpoint{3.124690in}{1.686208in}}%
\pgfpathlineto{\pgfqpoint{3.139178in}{1.738395in}}%
\pgfpathlineto{\pgfqpoint{3.155145in}{1.787366in}}%
\pgfpathlineto{\pgfqpoint{3.172353in}{1.833085in}}%
\pgfpathlineto{\pgfqpoint{3.191618in}{1.877716in}}%
\pgfpathlineto{\pgfqpoint{3.214026in}{1.923261in}}%
\pgfpathlineto{\pgfqpoint{3.236214in}{1.963157in}}%
\pgfpathlineto{\pgfqpoint{3.260178in}{2.001684in}}%
\pgfpathlineto{\pgfqpoint{3.285814in}{2.038776in}}%
\pgfpathlineto{\pgfqpoint{3.314415in}{2.076285in}}%
\pgfpathlineto{\pgfqpoint{3.348944in}{2.117711in}}%
\pgfpathlineto{\pgfqpoint{3.417133in}{2.198022in}}%
\pgfpathlineto{\pgfqpoint{3.426053in}{2.212128in}}%
\pgfpathlineto{\pgfqpoint{3.430798in}{2.223297in}}%
\pgfpathlineto{\pgfqpoint{3.432034in}{2.230603in}}%
\pgfpathlineto{\pgfqpoint{3.430773in}{2.237856in}}%
\pgfpathlineto{\pgfqpoint{3.426621in}{2.243526in}}%
\pgfpathlineto{\pgfqpoint{3.420908in}{2.247084in}}%
\pgfpathlineto{\pgfqpoint{3.412501in}{2.249583in}}%
\pgfpathlineto{\pgfqpoint{3.399499in}{2.250689in}}%
\pgfpathlineto{\pgfqpoint{3.384305in}{2.249671in}}%
\pgfpathlineto{\pgfqpoint{3.364985in}{2.246098in}}%
\pgfpathlineto{\pgfqpoint{3.341804in}{2.239342in}}%
\pgfpathlineto{\pgfqpoint{3.317109in}{2.229682in}}%
\pgfpathlineto{\pgfqpoint{3.291104in}{2.216986in}}%
\pgfpathlineto{\pgfqpoint{3.265928in}{2.202261in}}%
\pgfpathlineto{\pgfqpoint{3.239805in}{2.184361in}}%
\pgfpathlineto{\pgfqpoint{3.214775in}{2.164519in}}%
\pgfpathlineto{\pgfqpoint{3.190900in}{2.142893in}}%
\pgfpathlineto{\pgfqpoint{3.166657in}{2.117912in}}%
\pgfpathlineto{\pgfqpoint{3.143835in}{2.091233in}}%
\pgfpathlineto{\pgfqpoint{3.121079in}{2.061107in}}%
\pgfpathlineto{\pgfqpoint{3.099952in}{2.029463in}}%
\pgfpathlineto{\pgfqpoint{3.079251in}{1.994406in}}%
\pgfpathlineto{\pgfqpoint{3.059218in}{1.955915in}}%
\pgfpathlineto{\pgfqpoint{3.040058in}{1.914015in}}%
\pgfpathlineto{\pgfqpoint{3.022809in}{1.871041in}}%
\pgfpathlineto{\pgfqpoint{3.005790in}{1.822536in}}%
\pgfpathlineto{\pgfqpoint{2.990067in}{1.770819in}}%
\pgfpathlineto{\pgfqpoint{2.975708in}{1.715979in}}%
\pgfpathlineto{\pgfqpoint{2.962284in}{1.655680in}}%
\pgfpathlineto{\pgfqpoint{2.950496in}{1.592386in}}%
\pgfpathlineto{\pgfqpoint{2.940383in}{1.526185in}}%
\pgfpathlineto{\pgfqpoint{2.931745in}{1.454681in}}%
\pgfpathlineto{\pgfqpoint{2.925082in}{1.380399in}}%
\pgfpathlineto{\pgfqpoint{2.920647in}{1.305899in}}%
\pgfpathlineto{\pgfqpoint{2.918444in}{1.231270in}}%
\pgfpathlineto{\pgfqpoint{2.918545in}{1.159087in}}%
\pgfpathlineto{\pgfqpoint{2.920787in}{1.091931in}}%
\pgfpathlineto{\pgfqpoint{2.925177in}{1.027412in}}%
\pgfpathlineto{\pgfqpoint{2.931192in}{0.970580in}}%
\pgfpathlineto{\pgfqpoint{2.938760in}{0.919034in}}%
\pgfpathlineto{\pgfqpoint{2.947651in}{0.872852in}}%
\pgfpathlineto{\pgfqpoint{2.958213in}{0.829714in}}%
\pgfpathlineto{\pgfqpoint{2.969670in}{0.792114in}}%
\pgfpathlineto{\pgfqpoint{2.982463in}{0.757773in}}%
\pgfpathlineto{\pgfqpoint{2.996425in}{0.726812in}}%
\pgfpathlineto{\pgfqpoint{3.011299in}{0.699300in}}%
\pgfpathlineto{\pgfqpoint{3.026739in}{0.675225in}}%
\pgfpathlineto{\pgfqpoint{3.043828in}{0.652656in}}%
\pgfpathlineto{\pgfqpoint{3.062495in}{0.631788in}}%
\pgfpathlineto{\pgfqpoint{3.082602in}{0.612753in}}%
\pgfpathlineto{\pgfqpoint{3.103961in}{0.595592in}}%
\pgfpathlineto{\pgfqpoint{3.128268in}{0.579069in}}%
\pgfpathlineto{\pgfqpoint{3.153537in}{0.564554in}}%
\pgfpathlineto{\pgfqpoint{3.181571in}{0.550952in}}%
\pgfpathlineto{\pgfqpoint{3.214371in}{0.537647in}}%
\pgfpathlineto{\pgfqpoint{3.249846in}{0.525712in}}%
\pgfpathlineto{\pgfqpoint{3.290011in}{0.514571in}}%
\pgfpathlineto{\pgfqpoint{3.334820in}{0.504423in}}%
\pgfpathlineto{\pgfqpoint{3.386372in}{0.494999in}}%
\pgfpathlineto{\pgfqpoint{3.446798in}{0.486257in}}%
\pgfpathlineto{\pgfqpoint{3.518243in}{0.478282in}}%
\pgfpathlineto{\pgfqpoint{3.600685in}{0.471409in}}%
\pgfpathlineto{\pgfqpoint{3.696268in}{0.465713in}}%
\pgfpathlineto{\pgfqpoint{3.807144in}{0.461369in}}%
\pgfpathlineto{\pgfqpoint{3.933291in}{0.458719in}}%
\pgfpathlineto{\pgfqpoint{4.063808in}{0.458211in}}%
\pgfpathlineto{\pgfqpoint{4.187792in}{0.459914in}}%
\pgfpathlineto{\pgfqpoint{4.294335in}{0.463521in}}%
\pgfpathlineto{\pgfqpoint{4.381234in}{0.468574in}}%
\pgfpathlineto{\pgfqpoint{4.450636in}{0.474701in}}%
\pgfpathlineto{\pgfqpoint{4.506850in}{0.481799in}}%
\pgfpathlineto{\pgfqpoint{4.552009in}{0.489658in}}%
\pgfpathlineto{\pgfqpoint{4.588239in}{0.498115in}}%
\pgfpathlineto{\pgfqpoint{4.617656in}{0.507110in}}%
\pgfpathlineto{\pgfqpoint{4.642328in}{0.516843in}}%
\pgfpathlineto{\pgfqpoint{4.664194in}{0.527940in}}%
\pgfpathlineto{\pgfqpoint{4.681238in}{0.538945in}}%
\pgfpathlineto{\pgfqpoint{4.697164in}{0.551953in}}%
\pgfpathlineto{\pgfqpoint{4.710076in}{0.565289in}}%
\pgfpathlineto{\pgfqpoint{4.721578in}{0.580218in}}%
\pgfpathlineto{\pgfqpoint{4.731557in}{0.596521in}}%
\pgfpathlineto{\pgfqpoint{4.741000in}{0.616134in}}%
\pgfpathlineto{\pgfqpoint{4.749521in}{0.639027in}}%
\pgfpathlineto{\pgfqpoint{4.757522in}{0.667450in}}%
\pgfpathlineto{\pgfqpoint{4.764572in}{0.701345in}}%
\pgfpathlineto{\pgfqpoint{4.770840in}{0.743043in}}%
\pgfpathlineto{\pgfqpoint{4.776327in}{0.794934in}}%
\pgfpathlineto{\pgfqpoint{4.781278in}{0.864398in}}%
\pgfpathlineto{\pgfqpoint{4.785468in}{0.956371in}}%
\pgfpathlineto{\pgfqpoint{4.789000in}{1.085745in}}%
\pgfpathlineto{\pgfqpoint{4.791852in}{1.277385in}}%
\pgfpathlineto{\pgfqpoint{4.793959in}{1.581057in}}%
\pgfpathlineto{\pgfqpoint{4.794962in}{2.071429in}}%
\pgfpathlineto{\pgfqpoint{4.793967in}{2.559311in}}%
\pgfpathlineto{\pgfqpoint{4.791733in}{2.745981in}}%
\pgfpathlineto{\pgfqpoint{4.788955in}{2.818091in}}%
\pgfpathlineto{\pgfqpoint{4.785731in}{2.850227in}}%
\pgfpathlineto{\pgfqpoint{4.781879in}{2.867057in}}%
\pgfpathlineto{\pgfqpoint{4.777744in}{2.875780in}}%
\pgfpathlineto{\pgfqpoint{4.773097in}{2.880982in}}%
\pgfpathlineto{\pgfqpoint{4.767363in}{2.884504in}}%
\pgfpathlineto{\pgfqpoint{4.756853in}{2.887622in}}%
\pgfpathlineto{\pgfqpoint{4.739548in}{2.889639in}}%
\pgfpathlineto{\pgfqpoint{4.704762in}{2.890882in}}%
\pgfpathlineto{\pgfqpoint{4.602524in}{2.891538in}}%
\pgfpathlineto{\pgfqpoint{3.952100in}{2.891742in}}%
\pgfpathlineto{\pgfqpoint{0.617321in}{2.890753in}}%
\pgfpathlineto{\pgfqpoint{0.549910in}{2.888858in}}%
\pgfpathlineto{\pgfqpoint{0.521735in}{2.886179in}}%
\pgfpathlineto{\pgfqpoint{0.504666in}{2.882389in}}%
\pgfpathlineto{\pgfqpoint{0.494501in}{2.878011in}}%
\pgfpathlineto{\pgfqpoint{0.487180in}{2.872667in}}%
\pgfpathlineto{\pgfqpoint{0.481152in}{2.865519in}}%
\pgfpathlineto{\pgfqpoint{0.475664in}{2.854804in}}%
\pgfpathlineto{\pgfqpoint{0.471318in}{2.840737in}}%
\pgfpathlineto{\pgfqpoint{0.467301in}{2.818823in}}%
\pgfpathlineto{\pgfqpoint{0.463927in}{2.786700in}}%
\pgfpathlineto{\pgfqpoint{0.460918in}{2.734544in}}%
\pgfpathlineto{\pgfqpoint{0.458363in}{2.647474in}}%
\pgfpathlineto{\pgfqpoint{0.456575in}{2.523031in}}%
\pgfpathlineto{\pgfqpoint{0.456575in}{2.523031in}}%
\pgfusepath{stroke}%
\end{pgfscope}%
\begin{pgfscope}%
\pgfpathrectangle{\pgfqpoint{0.448634in}{0.402556in}}{\pgfqpoint{4.350661in}{2.489204in}} %
\pgfusepath{clip}%
\pgfsetrectcap%
\pgfsetroundjoin%
\pgfsetlinewidth{1.003750pt}%
\definecolor{currentstroke}{rgb}{0.172549,0.627451,0.172549}%
\pgfsetstrokecolor{currentstroke}%
\pgfsetdash{}{0pt}%
\pgfpathmoveto{\pgfqpoint{4.798840in}{2.852369in}}%
\pgfpathlineto{\pgfqpoint{4.797564in}{2.889610in}}%
\pgfpathlineto{\pgfqpoint{4.796215in}{2.891483in}}%
\pgfpathlineto{\pgfqpoint{4.787551in}{2.891760in}}%
\pgfpathlineto{\pgfqpoint{0.452128in}{2.891659in}}%
\pgfpathlineto{\pgfqpoint{0.450530in}{2.890082in}}%
\pgfpathlineto{\pgfqpoint{0.449454in}{2.882763in}}%
\pgfpathlineto{\pgfqpoint{0.448970in}{2.845432in}}%
\pgfpathlineto{\pgfqpoint{0.448743in}{2.494454in}}%
\pgfpathlineto{\pgfqpoint{0.449624in}{0.615107in}}%
\pgfpathlineto{\pgfqpoint{0.451433in}{0.510586in}}%
\pgfpathlineto{\pgfqpoint{0.453993in}{0.473374in}}%
\pgfpathlineto{\pgfqpoint{0.457406in}{0.453868in}}%
\pgfpathlineto{\pgfqpoint{0.461540in}{0.442384in}}%
\pgfpathlineto{\pgfqpoint{0.466739in}{0.434437in}}%
\pgfpathlineto{\pgfqpoint{0.473595in}{0.428350in}}%
\pgfpathlineto{\pgfqpoint{0.483492in}{0.423244in}}%
\pgfpathlineto{\pgfqpoint{0.491854in}{0.420501in}}%
\pgfpathlineto{\pgfqpoint{0.491854in}{0.420501in}}%
\pgfusepath{stroke}%
\end{pgfscope}%
\begin{pgfscope}%
\pgfpathrectangle{\pgfqpoint{0.448634in}{0.402556in}}{\pgfqpoint{4.350661in}{2.489204in}} %
\pgfusepath{clip}%
\pgfsetrectcap%
\pgfsetroundjoin%
\pgfsetlinewidth{1.003750pt}%
\definecolor{currentstroke}{rgb}{0.172549,0.627451,0.172549}%
\pgfsetstrokecolor{currentstroke}%
\pgfsetdash{}{0pt}%
\pgfpathmoveto{\pgfqpoint{0.456424in}{1.370137in}}%
\pgfpathlineto{\pgfqpoint{0.459610in}{1.118755in}}%
\pgfpathlineto{\pgfqpoint{0.463695in}{0.962007in}}%
\pgfpathlineto{\pgfqpoint{0.468519in}{0.857610in}}%
\pgfpathlineto{\pgfqpoint{0.474082in}{0.783210in}}%
\pgfpathlineto{\pgfqpoint{0.480226in}{0.728906in}}%
\pgfpathlineto{\pgfqpoint{0.486970in}{0.687306in}}%
\pgfpathlineto{\pgfqpoint{0.494537in}{0.653559in}}%
\pgfpathlineto{\pgfqpoint{0.503107in}{0.625355in}}%
\pgfpathlineto{\pgfqpoint{0.512193in}{0.602750in}}%
\pgfpathlineto{\pgfqpoint{0.522200in}{0.583508in}}%
\pgfpathlineto{\pgfqpoint{0.534108in}{0.565743in}}%
\pgfpathlineto{\pgfqpoint{0.546263in}{0.551507in}}%
\pgfpathlineto{\pgfqpoint{0.559728in}{0.538907in}}%
\pgfpathlineto{\pgfqpoint{0.576129in}{0.526693in}}%
\pgfpathlineto{\pgfqpoint{0.595483in}{0.515351in}}%
\pgfpathlineto{\pgfqpoint{0.617681in}{0.505147in}}%
\pgfpathlineto{\pgfqpoint{0.642568in}{0.496153in}}%
\pgfpathlineto{\pgfqpoint{0.672125in}{0.487778in}}%
\pgfpathlineto{\pgfqpoint{0.708443in}{0.479824in}}%
\pgfpathlineto{\pgfqpoint{0.753649in}{0.472325in}}%
\pgfpathlineto{\pgfqpoint{0.807717in}{0.465660in}}%
\pgfpathlineto{\pgfqpoint{0.877116in}{0.459475in}}%
\pgfpathlineto{\pgfqpoint{0.961828in}{0.454230in}}%
\pgfpathlineto{\pgfqpoint{1.068351in}{0.449916in}}%
\pgfpathlineto{\pgfqpoint{1.201018in}{0.446839in}}%
\pgfpathlineto{\pgfqpoint{1.357637in}{0.445481in}}%
\pgfpathlineto{\pgfqpoint{1.525135in}{0.446232in}}%
\pgfpathlineto{\pgfqpoint{1.686088in}{0.449142in}}%
\pgfpathlineto{\pgfqpoint{1.823074in}{0.453747in}}%
\pgfpathlineto{\pgfqpoint{1.938245in}{0.459764in}}%
\pgfpathlineto{\pgfqpoint{2.031582in}{0.466759in}}%
\pgfpathlineto{\pgfqpoint{2.109580in}{0.474745in}}%
\pgfpathlineto{\pgfqpoint{2.174384in}{0.483535in}}%
\pgfpathlineto{\pgfqpoint{2.228139in}{0.492940in}}%
\pgfpathlineto{\pgfqpoint{2.275119in}{0.503356in}}%
\pgfpathlineto{\pgfqpoint{2.315282in}{0.514501in}}%
\pgfpathlineto{\pgfqpoint{2.350698in}{0.526659in}}%
\pgfpathlineto{\pgfqpoint{2.381320in}{0.539536in}}%
\pgfpathlineto{\pgfqpoint{2.407164in}{0.552659in}}%
\pgfpathlineto{\pgfqpoint{2.430226in}{0.566639in}}%
\pgfpathlineto{\pgfqpoint{2.452282in}{0.582602in}}%
\pgfpathlineto{\pgfqpoint{2.471391in}{0.599069in}}%
\pgfpathlineto{\pgfqpoint{2.489240in}{0.617293in}}%
\pgfpathlineto{\pgfqpoint{2.505678in}{0.637180in}}%
\pgfpathlineto{\pgfqpoint{2.520620in}{0.658557in}}%
\pgfpathlineto{\pgfqpoint{2.535213in}{0.683314in}}%
\pgfpathlineto{\pgfqpoint{2.549115in}{0.711484in}}%
\pgfpathlineto{\pgfqpoint{2.562091in}{0.743004in}}%
\pgfpathlineto{\pgfqpoint{2.574020in}{0.777751in}}%
\pgfpathlineto{\pgfqpoint{2.585502in}{0.817970in}}%
\pgfpathlineto{\pgfqpoint{2.596809in}{0.866038in}}%
\pgfpathlineto{\pgfqpoint{2.607562in}{0.921948in}}%
\pgfpathlineto{\pgfqpoint{2.617925in}{0.988098in}}%
\pgfpathlineto{\pgfqpoint{2.627958in}{1.066918in}}%
\pgfpathlineto{\pgfqpoint{2.637941in}{1.163320in}}%
\pgfpathlineto{\pgfqpoint{2.648424in}{1.287199in}}%
\pgfpathlineto{\pgfqpoint{2.660103in}{1.453438in}}%
\pgfpathlineto{\pgfqpoint{2.674773in}{1.696801in}}%
\pgfpathlineto{\pgfqpoint{2.687716in}{1.945279in}}%
\pgfpathlineto{\pgfqpoint{2.692670in}{2.079573in}}%
\pgfpathlineto{\pgfqpoint{2.693829in}{2.166682in}}%
\pgfpathlineto{\pgfqpoint{2.692565in}{2.233870in}}%
\pgfpathlineto{\pgfqpoint{2.689436in}{2.286014in}}%
\pgfpathlineto{\pgfqpoint{2.684859in}{2.327999in}}%
\pgfpathlineto{\pgfqpoint{2.678725in}{2.364664in}}%
\pgfpathlineto{\pgfqpoint{2.671356in}{2.395897in}}%
\pgfpathlineto{\pgfqpoint{2.662489in}{2.423981in}}%
\pgfpathlineto{\pgfqpoint{2.652361in}{2.448778in}}%
\pgfpathlineto{\pgfqpoint{2.641365in}{2.470245in}}%
\pgfpathlineto{\pgfqpoint{2.628643in}{2.490425in}}%
\pgfpathlineto{\pgfqpoint{2.614279in}{2.509105in}}%
\pgfpathlineto{\pgfqpoint{2.598443in}{2.526159in}}%
\pgfpathlineto{\pgfqpoint{2.579590in}{2.543005in}}%
\pgfpathlineto{\pgfqpoint{2.559532in}{2.557923in}}%
\pgfpathlineto{\pgfqpoint{2.536602in}{2.572183in}}%
\pgfpathlineto{\pgfqpoint{2.510850in}{2.585538in}}%
\pgfpathlineto{\pgfqpoint{2.482360in}{2.597837in}}%
\pgfpathlineto{\pgfqpoint{2.449134in}{2.609683in}}%
\pgfpathlineto{\pgfqpoint{2.411184in}{2.620696in}}%
\pgfpathlineto{\pgfqpoint{2.368552in}{2.630606in}}%
\pgfpathlineto{\pgfqpoint{2.321294in}{2.639221in}}%
\pgfpathlineto{\pgfqpoint{2.269467in}{2.646399in}}%
\pgfpathlineto{\pgfqpoint{2.210954in}{2.652193in}}%
\pgfpathlineto{\pgfqpoint{2.147967in}{2.656153in}}%
\pgfpathlineto{\pgfqpoint{2.080556in}{2.658135in}}%
\pgfpathlineto{\pgfqpoint{2.010948in}{2.657971in}}%
\pgfpathlineto{\pgfqpoint{1.939195in}{2.655572in}}%
\pgfpathlineto{\pgfqpoint{1.867527in}{2.650913in}}%
\pgfpathlineto{\pgfqpoint{1.798171in}{2.644140in}}%
\pgfpathlineto{\pgfqpoint{1.733341in}{2.635606in}}%
\pgfpathlineto{\pgfqpoint{1.673075in}{2.625521in}}%
\pgfpathlineto{\pgfqpoint{1.615274in}{2.613610in}}%
\pgfpathlineto{\pgfqpoint{1.562133in}{2.600402in}}%
\pgfpathlineto{\pgfqpoint{1.513681in}{2.586139in}}%
\pgfpathlineto{\pgfqpoint{1.467862in}{2.570344in}}%
\pgfpathlineto{\pgfqpoint{1.426794in}{2.553923in}}%
\pgfpathlineto{\pgfqpoint{1.388447in}{2.536289in}}%
\pgfpathlineto{\pgfqpoint{1.352878in}{2.517566in}}%
\pgfpathlineto{\pgfqpoint{1.320128in}{2.497922in}}%
\pgfpathlineto{\pgfqpoint{1.288379in}{2.476236in}}%
\pgfpathlineto{\pgfqpoint{1.259592in}{2.453861in}}%
\pgfpathlineto{\pgfqpoint{1.232050in}{2.429520in}}%
\pgfpathlineto{\pgfqpoint{1.207527in}{2.404898in}}%
\pgfpathlineto{\pgfqpoint{1.184409in}{2.378557in}}%
\pgfpathlineto{\pgfqpoint{1.162828in}{2.350561in}}%
\pgfpathlineto{\pgfqpoint{1.142891in}{2.321012in}}%
\pgfpathlineto{\pgfqpoint{1.124675in}{2.290041in}}%
\pgfpathlineto{\pgfqpoint{1.108225in}{2.257802in}}%
\pgfpathlineto{\pgfqpoint{1.092639in}{2.222199in}}%
\pgfpathlineto{\pgfqpoint{1.079059in}{2.185535in}}%
\pgfpathlineto{\pgfqpoint{1.067443in}{2.147998in}}%
\pgfpathlineto{\pgfqpoint{1.057187in}{2.107348in}}%
\pgfpathlineto{\pgfqpoint{1.049004in}{2.066086in}}%
\pgfpathlineto{\pgfqpoint{1.042513in}{2.021906in}}%
\pgfpathlineto{\pgfqpoint{1.038177in}{1.977382in}}%
\pgfpathlineto{\pgfqpoint{1.035866in}{1.930167in}}%
\pgfpathlineto{\pgfqpoint{1.035826in}{1.882878in}}%
\pgfpathlineto{\pgfqpoint{1.038031in}{1.835656in}}%
\pgfpathlineto{\pgfqpoint{1.042474in}{1.788641in}}%
\pgfpathlineto{\pgfqpoint{1.049176in}{1.741979in}}%
\pgfpathlineto{\pgfqpoint{1.057644in}{1.698239in}}%
\pgfpathlineto{\pgfqpoint{1.068221in}{1.655105in}}%
\pgfpathlineto{\pgfqpoint{1.080962in}{1.612745in}}%
\pgfpathlineto{\pgfqpoint{1.095031in}{1.573617in}}%
\pgfpathlineto{\pgfqpoint{1.111115in}{1.535520in}}%
\pgfpathlineto{\pgfqpoint{1.128117in}{1.500775in}}%
\pgfpathlineto{\pgfqpoint{1.146930in}{1.467274in}}%
\pgfpathlineto{\pgfqpoint{1.167531in}{1.435181in}}%
\pgfpathlineto{\pgfqpoint{1.189874in}{1.404652in}}%
\pgfpathlineto{\pgfqpoint{1.213884in}{1.375828in}}%
\pgfpathlineto{\pgfqpoint{1.237817in}{1.350457in}}%
\pgfpathlineto{\pgfqpoint{1.264748in}{1.325237in}}%
\pgfpathlineto{\pgfqpoint{1.292991in}{1.301972in}}%
\pgfpathlineto{\pgfqpoint{1.322398in}{1.280678in}}%
\pgfpathlineto{\pgfqpoint{1.352820in}{1.261340in}}%
\pgfpathlineto{\pgfqpoint{1.386095in}{1.242889in}}%
\pgfpathlineto{\pgfqpoint{1.420190in}{1.226516in}}%
\pgfpathlineto{\pgfqpoint{1.457024in}{1.211329in}}%
\pgfpathlineto{\pgfqpoint{1.496554in}{1.197536in}}%
\pgfpathlineto{\pgfqpoint{1.538719in}{1.185287in}}%
\pgfpathlineto{\pgfqpoint{1.583441in}{1.174641in}}%
\pgfpathlineto{\pgfqpoint{1.634929in}{1.164775in}}%
\pgfpathlineto{\pgfqpoint{1.706063in}{1.153745in}}%
\pgfpathlineto{\pgfqpoint{1.768492in}{1.143417in}}%
\pgfpathlineto{\pgfqpoint{1.796122in}{1.136567in}}%
\pgfpathlineto{\pgfqpoint{1.812683in}{1.130481in}}%
\pgfpathlineto{\pgfqpoint{1.824471in}{1.124102in}}%
\pgfpathlineto{\pgfqpoint{1.833209in}{1.116741in}}%
\pgfpathlineto{\pgfqpoint{1.838498in}{1.108890in}}%
\pgfpathlineto{\pgfqpoint{1.840588in}{1.101849in}}%
\pgfpathlineto{\pgfqpoint{1.840619in}{1.094412in}}%
\pgfpathlineto{\pgfqpoint{1.837931in}{1.084986in}}%
\pgfpathlineto{\pgfqpoint{1.833246in}{1.076615in}}%
\pgfpathlineto{\pgfqpoint{1.825819in}{1.067542in}}%
\pgfpathlineto{\pgfqpoint{1.813813in}{1.056850in}}%
\pgfpathlineto{\pgfqpoint{1.798819in}{1.046763in}}%
\pgfpathlineto{\pgfqpoint{1.781016in}{1.037462in}}%
\pgfpathlineto{\pgfqpoint{1.758447in}{1.028391in}}%
\pgfpathlineto{\pgfqpoint{1.733203in}{1.020815in}}%
\pgfpathlineto{\pgfqpoint{1.705410in}{1.014872in}}%
\pgfpathlineto{\pgfqpoint{1.675178in}{1.010714in}}%
\pgfpathlineto{\pgfqpoint{1.642610in}{1.008507in}}%
\pgfpathlineto{\pgfqpoint{1.607809in}{1.008432in}}%
\pgfpathlineto{\pgfqpoint{1.570886in}{1.010691in}}%
\pgfpathlineto{\pgfqpoint{1.534118in}{1.015181in}}%
\pgfpathlineto{\pgfqpoint{1.495454in}{1.022233in}}%
\pgfpathlineto{\pgfqpoint{1.457161in}{1.031563in}}%
\pgfpathlineto{\pgfqpoint{1.419337in}{1.043132in}}%
\pgfpathlineto{\pgfqpoint{1.382089in}{1.056929in}}%
\pgfpathlineto{\pgfqpoint{1.347544in}{1.072019in}}%
\pgfpathlineto{\pgfqpoint{1.313727in}{1.089133in}}%
\pgfpathlineto{\pgfqpoint{1.280762in}{1.108299in}}%
\pgfpathlineto{\pgfqpoint{1.248782in}{1.129536in}}%
\pgfpathlineto{\pgfqpoint{1.219708in}{1.151422in}}%
\pgfpathlineto{\pgfqpoint{1.191752in}{1.175138in}}%
\pgfpathlineto{\pgfqpoint{1.165031in}{1.200649in}}%
\pgfpathlineto{\pgfqpoint{1.139653in}{1.227898in}}%
\pgfpathlineto{\pgfqpoint{1.115714in}{1.256800in}}%
\pgfpathlineto{\pgfqpoint{1.093288in}{1.287251in}}%
\pgfpathlineto{\pgfqpoint{1.071178in}{1.321163in}}%
\pgfpathlineto{\pgfqpoint{1.050868in}{1.356520in}}%
\pgfpathlineto{\pgfqpoint{1.032365in}{1.393152in}}%
\pgfpathlineto{\pgfqpoint{1.014718in}{1.433142in}}%
\pgfpathlineto{\pgfqpoint{0.999024in}{1.474185in}}%
\pgfpathlineto{\pgfqpoint{0.984506in}{1.518461in}}%
\pgfpathlineto{\pgfqpoint{0.972009in}{1.563537in}}%
\pgfpathlineto{\pgfqpoint{0.960944in}{1.611678in}}%
\pgfpathlineto{\pgfqpoint{0.951530in}{1.662824in}}%
\pgfpathlineto{\pgfqpoint{0.944286in}{1.714431in}}%
\pgfpathlineto{\pgfqpoint{0.938950in}{1.768847in}}%
\pgfpathlineto{\pgfqpoint{0.935870in}{1.823491in}}%
\pgfpathlineto{\pgfqpoint{0.935034in}{1.878240in}}%
\pgfpathlineto{\pgfqpoint{0.936466in}{1.932973in}}%
\pgfpathlineto{\pgfqpoint{0.940005in}{1.985084in}}%
\pgfpathlineto{\pgfqpoint{0.945759in}{2.036935in}}%
\pgfpathlineto{\pgfqpoint{0.953410in}{2.085938in}}%
\pgfpathlineto{\pgfqpoint{0.962764in}{2.132000in}}%
\pgfpathlineto{\pgfqpoint{0.974287in}{2.177414in}}%
\pgfpathlineto{\pgfqpoint{0.987332in}{2.219653in}}%
\pgfpathlineto{\pgfqpoint{1.001667in}{2.258654in}}%
\pgfpathlineto{\pgfqpoint{1.018051in}{2.296583in}}%
\pgfpathlineto{\pgfqpoint{1.035401in}{2.331101in}}%
\pgfpathlineto{\pgfqpoint{1.054650in}{2.364275in}}%
\pgfpathlineto{\pgfqpoint{1.074406in}{2.393984in}}%
\pgfpathlineto{\pgfqpoint{1.095771in}{2.422197in}}%
\pgfpathlineto{\pgfqpoint{1.118662in}{2.448797in}}%
\pgfpathlineto{\pgfqpoint{1.142967in}{2.473701in}}%
\pgfpathlineto{\pgfqpoint{1.168550in}{2.496867in}}%
\pgfpathlineto{\pgfqpoint{1.197085in}{2.519662in}}%
\pgfpathlineto{\pgfqpoint{1.226727in}{2.540526in}}%
\pgfpathlineto{\pgfqpoint{1.259242in}{2.560673in}}%
\pgfpathlineto{\pgfqpoint{1.294612in}{2.579881in}}%
\pgfpathlineto{\pgfqpoint{1.332792in}{2.597982in}}%
\pgfpathlineto{\pgfqpoint{1.373719in}{2.614859in}}%
\pgfpathlineto{\pgfqpoint{1.417319in}{2.630445in}}%
\pgfpathlineto{\pgfqpoint{1.465632in}{2.645312in}}%
\pgfpathlineto{\pgfqpoint{1.518640in}{2.659204in}}%
\pgfpathlineto{\pgfqpoint{1.576309in}{2.671929in}}%
\pgfpathlineto{\pgfqpoint{1.638597in}{2.683344in}}%
\pgfpathlineto{\pgfqpoint{1.705462in}{2.693343in}}%
\pgfpathlineto{\pgfqpoint{1.779027in}{2.702064in}}%
\pgfpathlineto{\pgfqpoint{1.857097in}{2.709077in}}%
\pgfpathlineto{\pgfqpoint{1.939633in}{2.714280in}}%
\pgfpathlineto{\pgfqpoint{2.026598in}{2.717513in}}%
\pgfpathlineto{\pgfqpoint{2.113605in}{2.718523in}}%
\pgfpathlineto{\pgfqpoint{2.198435in}{2.717303in}}%
\pgfpathlineto{\pgfqpoint{2.278866in}{2.713929in}}%
\pgfpathlineto{\pgfqpoint{2.352678in}{2.708598in}}%
\pgfpathlineto{\pgfqpoint{2.417657in}{2.701709in}}%
\pgfpathlineto{\pgfqpoint{2.473770in}{2.693630in}}%
\pgfpathlineto{\pgfqpoint{2.523140in}{2.684368in}}%
\pgfpathlineto{\pgfqpoint{2.565726in}{2.674202in}}%
\pgfpathlineto{\pgfqpoint{2.601510in}{2.663544in}}%
\pgfpathlineto{\pgfqpoint{2.632577in}{2.652142in}}%
\pgfpathlineto{\pgfqpoint{2.658899in}{2.640331in}}%
\pgfpathlineto{\pgfqpoint{2.682438in}{2.627436in}}%
\pgfpathlineto{\pgfqpoint{2.703062in}{2.613571in}}%
\pgfpathlineto{\pgfqpoint{2.720674in}{2.598978in}}%
\pgfpathlineto{\pgfqpoint{2.735263in}{2.584053in}}%
\pgfpathlineto{\pgfqpoint{2.748320in}{2.567377in}}%
\pgfpathlineto{\pgfqpoint{2.759553in}{2.549046in}}%
\pgfpathlineto{\pgfqpoint{2.768788in}{2.529306in}}%
\pgfpathlineto{\pgfqpoint{2.776017in}{2.508498in}}%
\pgfpathlineto{\pgfqpoint{2.781884in}{2.484540in}}%
\pgfpathlineto{\pgfqpoint{2.786102in}{2.457597in}}%
\pgfpathlineto{\pgfqpoint{2.788720in}{2.425384in}}%
\pgfpathlineto{\pgfqpoint{2.789427in}{2.388061in}}%
\pgfpathlineto{\pgfqpoint{2.787962in}{2.340801in}}%
\pgfpathlineto{\pgfqpoint{2.783672in}{2.278768in}}%
\pgfpathlineto{\pgfqpoint{2.774289in}{2.179783in}}%
\pgfpathlineto{\pgfqpoint{2.743611in}{1.868119in}}%
\pgfpathlineto{\pgfqpoint{2.730112in}{1.702060in}}%
\pgfpathlineto{\pgfqpoint{2.717287in}{1.515949in}}%
\pgfpathlineto{\pgfqpoint{2.702602in}{1.267597in}}%
\pgfpathlineto{\pgfqpoint{2.684434in}{0.964630in}}%
\pgfpathlineto{\pgfqpoint{2.675374in}{0.850600in}}%
\pgfpathlineto{\pgfqpoint{2.667030in}{0.771523in}}%
\pgfpathlineto{\pgfqpoint{2.658752in}{0.712543in}}%
\pgfpathlineto{\pgfqpoint{2.650176in}{0.666284in}}%
\pgfpathlineto{\pgfqpoint{2.640820in}{0.627931in}}%
\pgfpathlineto{\pgfqpoint{2.631145in}{0.597534in}}%
\pgfpathlineto{\pgfqpoint{2.621004in}{0.572745in}}%
\pgfpathlineto{\pgfqpoint{2.609856in}{0.551383in}}%
\pgfpathlineto{\pgfqpoint{2.598042in}{0.533534in}}%
\pgfpathlineto{\pgfqpoint{2.584496in}{0.517378in}}%
\pgfpathlineto{\pgfqpoint{2.571109in}{0.504669in}}%
\pgfpathlineto{\pgfqpoint{2.554789in}{0.492313in}}%
\pgfpathlineto{\pgfqpoint{2.537457in}{0.481914in}}%
\pgfpathlineto{\pgfqpoint{2.517374in}{0.472367in}}%
\pgfpathlineto{\pgfqpoint{2.492542in}{0.463178in}}%
\pgfpathlineto{\pgfqpoint{2.462979in}{0.454833in}}%
\pgfpathlineto{\pgfqpoint{2.428766in}{0.447542in}}%
\pgfpathlineto{\pgfqpoint{2.385671in}{0.440735in}}%
\pgfpathlineto{\pgfqpoint{2.331557in}{0.434581in}}%
\pgfpathlineto{\pgfqpoint{2.262115in}{0.429077in}}%
\pgfpathlineto{\pgfqpoint{2.170851in}{0.424236in}}%
\pgfpathlineto{\pgfqpoint{2.049086in}{0.420134in}}%
\pgfpathlineto{\pgfqpoint{1.879436in}{0.416783in}}%
\pgfpathlineto{\pgfqpoint{1.640159in}{0.414418in}}%
\pgfpathlineto{\pgfqpoint{1.322562in}{0.413569in}}%
\pgfpathlineto{\pgfqpoint{1.020194in}{0.414850in}}%
\pgfpathlineto{\pgfqpoint{0.822256in}{0.417715in}}%
\pgfpathlineto{\pgfqpoint{0.704835in}{0.421430in}}%
\pgfpathlineto{\pgfqpoint{0.630977in}{0.425829in}}%
\pgfpathlineto{\pgfqpoint{0.583316in}{0.430734in}}%
\pgfpathlineto{\pgfqpoint{0.551033in}{0.436123in}}%
\pgfpathlineto{\pgfqpoint{0.527708in}{0.442189in}}%
\pgfpathlineto{\pgfqpoint{0.511250in}{0.448625in}}%
\pgfpathlineto{\pgfqpoint{0.499549in}{0.455216in}}%
\pgfpathlineto{\pgfqpoint{0.488916in}{0.463841in}}%
\pgfpathlineto{\pgfqpoint{0.481322in}{0.472730in}}%
\pgfpathlineto{\pgfqpoint{0.474078in}{0.485127in}}%
\pgfpathlineto{\pgfqpoint{0.468753in}{0.498748in}}%
\pgfpathlineto{\pgfqpoint{0.463870in}{0.517848in}}%
\pgfpathlineto{\pgfqpoint{0.459679in}{0.544796in}}%
\pgfpathlineto{\pgfqpoint{0.456386in}{0.581938in}}%
\pgfpathlineto{\pgfqpoint{0.453731in}{0.639106in}}%
\pgfpathlineto{\pgfqpoint{0.451681in}{0.736155in}}%
\pgfpathlineto{\pgfqpoint{0.450220in}{0.927815in}}%
\pgfpathlineto{\pgfqpoint{0.449345in}{1.403252in}}%
\pgfpathlineto{\pgfqpoint{0.449543in}{2.682703in}}%
\pgfpathlineto{\pgfqpoint{0.451011in}{2.856932in}}%
\pgfpathlineto{\pgfqpoint{0.452802in}{2.879219in}}%
\pgfpathlineto{\pgfqpoint{0.455188in}{2.886107in}}%
\pgfpathlineto{\pgfqpoint{0.458626in}{2.889028in}}%
\pgfpathlineto{\pgfqpoint{0.464996in}{2.890553in}}%
\pgfpathlineto{\pgfqpoint{0.482376in}{2.891423in}}%
\pgfpathlineto{\pgfqpoint{0.565038in}{2.891729in}}%
\pgfpathlineto{\pgfqpoint{2.733842in}{2.891760in}}%
\pgfpathlineto{\pgfqpoint{4.789510in}{2.890885in}}%
\pgfpathlineto{\pgfqpoint{4.793727in}{2.889730in}}%
\pgfpathlineto{\pgfqpoint{4.795481in}{2.888307in}}%
\pgfpathlineto{\pgfqpoint{4.797106in}{2.881145in}}%
\pgfpathlineto{\pgfqpoint{4.797997in}{2.858771in}}%
\pgfpathlineto{\pgfqpoint{4.798039in}{2.856283in}}%
\pgfpathlineto{\pgfqpoint{4.798039in}{2.856283in}}%
\pgfusepath{stroke}%
\end{pgfscope}%
\begin{pgfscope}%
\pgfpathrectangle{\pgfqpoint{0.448634in}{0.402556in}}{\pgfqpoint{4.350661in}{2.489204in}} %
\pgfusepath{clip}%
\pgfsetrectcap%
\pgfsetroundjoin%
\pgfsetlinewidth{1.003750pt}%
\definecolor{currentstroke}{rgb}{0.172549,0.627451,0.172549}%
\pgfsetstrokecolor{currentstroke}%
\pgfsetdash{}{0pt}%
\pgfpathmoveto{\pgfqpoint{3.428772in}{0.402610in}}%
\pgfpathlineto{\pgfqpoint{2.806632in}{0.403760in}}%
\pgfpathlineto{\pgfqpoint{2.769692in}{0.405578in}}%
\pgfpathlineto{\pgfqpoint{2.754632in}{0.408064in}}%
\pgfpathlineto{\pgfqpoint{2.746391in}{0.411198in}}%
\pgfpathlineto{\pgfqpoint{2.740943in}{0.415265in}}%
\pgfpathlineto{\pgfqpoint{2.736784in}{0.420984in}}%
\pgfpathlineto{\pgfqpoint{2.733281in}{0.430071in}}%
\pgfpathlineto{\pgfqpoint{2.730449in}{0.444636in}}%
\pgfpathlineto{\pgfqpoint{2.728238in}{0.469392in}}%
\pgfpathlineto{\pgfqpoint{2.726470in}{0.519131in}}%
\pgfpathlineto{\pgfqpoint{2.725711in}{0.613715in}}%
\pgfpathlineto{\pgfqpoint{2.726842in}{0.768038in}}%
\pgfpathlineto{\pgfqpoint{2.730556in}{0.962148in}}%
\pgfpathlineto{\pgfqpoint{2.736611in}{1.158670in}}%
\pgfpathlineto{\pgfqpoint{2.744092in}{1.327718in}}%
\pgfpathlineto{\pgfqpoint{2.753201in}{1.484189in}}%
\pgfpathlineto{\pgfqpoint{2.763257in}{1.620609in}}%
\pgfpathlineto{\pgfqpoint{2.776118in}{1.764216in}}%
\pgfpathlineto{\pgfqpoint{2.788914in}{1.877776in}}%
\pgfpathlineto{\pgfqpoint{2.805748in}{2.005740in}}%
\pgfpathlineto{\pgfqpoint{2.821176in}{2.101198in}}%
\pgfpathlineto{\pgfqpoint{2.838359in}{2.193718in}}%
\pgfpathlineto{\pgfqpoint{2.859135in}{2.292966in}}%
\pgfpathlineto{\pgfqpoint{2.887209in}{2.425960in}}%
\pgfpathlineto{\pgfqpoint{2.896991in}{2.479560in}}%
\pgfpathlineto{\pgfqpoint{2.901543in}{2.516523in}}%
\pgfpathlineto{\pgfqpoint{2.902849in}{2.543854in}}%
\pgfpathlineto{\pgfqpoint{2.901957in}{2.566223in}}%
\pgfpathlineto{\pgfqpoint{2.899151in}{2.585863in}}%
\pgfpathlineto{\pgfqpoint{2.894794in}{2.602546in}}%
\pgfpathlineto{\pgfqpoint{2.888484in}{2.618388in}}%
\pgfpathlineto{\pgfqpoint{2.880257in}{2.633033in}}%
\pgfpathlineto{\pgfqpoint{2.870348in}{2.646246in}}%
\pgfpathlineto{\pgfqpoint{2.857400in}{2.659530in}}%
\pgfpathlineto{\pgfqpoint{2.843189in}{2.671010in}}%
\pgfpathlineto{\pgfqpoint{2.824237in}{2.683209in}}%
\pgfpathlineto{\pgfqpoint{2.802413in}{2.694418in}}%
\pgfpathlineto{\pgfqpoint{2.775809in}{2.705369in}}%
\pgfpathlineto{\pgfqpoint{2.744461in}{2.715715in}}%
\pgfpathlineto{\pgfqpoint{2.708436in}{2.725252in}}%
\pgfpathlineto{\pgfqpoint{2.665655in}{2.734289in}}%
\pgfpathlineto{\pgfqpoint{2.613991in}{2.742869in}}%
\pgfpathlineto{\pgfqpoint{2.553459in}{2.750589in}}%
\pgfpathlineto{\pgfqpoint{2.481920in}{2.757365in}}%
\pgfpathlineto{\pgfqpoint{2.399398in}{2.762839in}}%
\pgfpathlineto{\pgfqpoint{2.310269in}{2.766482in}}%
\pgfpathlineto{\pgfqpoint{2.175416in}{2.768725in}}%
\pgfpathlineto{\pgfqpoint{2.066653in}{2.767942in}}%
\pgfpathlineto{\pgfqpoint{1.953570in}{2.764859in}}%
\pgfpathlineto{\pgfqpoint{1.851429in}{2.759759in}}%
\pgfpathlineto{\pgfqpoint{1.745051in}{2.752169in}}%
\pgfpathlineto{\pgfqpoint{1.658373in}{2.743453in}}%
\pgfpathlineto{\pgfqpoint{1.580552in}{2.733461in}}%
\pgfpathlineto{\pgfqpoint{1.490057in}{2.719338in}}%
\pgfpathlineto{\pgfqpoint{1.417231in}{2.704698in}}%
\pgfpathlineto{\pgfqpoint{1.361992in}{2.690818in}}%
\pgfpathlineto{\pgfqpoint{1.311460in}{2.675819in}}%
\pgfpathlineto{\pgfqpoint{1.265667in}{2.659924in}}%
\pgfpathlineto{\pgfqpoint{1.222575in}{2.642586in}}%
\pgfpathlineto{\pgfqpoint{1.184324in}{2.624682in}}%
\pgfpathlineto{\pgfqpoint{1.148892in}{2.605623in}}%
\pgfpathlineto{\pgfqpoint{1.116331in}{2.585573in}}%
\pgfpathlineto{\pgfqpoint{1.092327in}{2.568512in}}%
\pgfpathlineto{\pgfqpoint{1.079760in}{2.558686in}}%
\pgfpathlineto{\pgfqpoint{1.051544in}{2.535379in}}%
\pgfpathlineto{\pgfqpoint{1.026312in}{2.511712in}}%
\pgfpathlineto{\pgfqpoint{1.002399in}{2.486318in}}%
\pgfpathlineto{\pgfqpoint{0.979913in}{2.459269in}}%
\pgfpathlineto{\pgfqpoint{0.958934in}{2.430678in}}%
\pgfpathlineto{\pgfqpoint{0.938264in}{2.398643in}}%
\pgfpathlineto{\pgfqpoint{0.923047in}{2.371385in}}%
\pgfpathlineto{\pgfqpoint{0.904513in}{2.334774in}}%
\pgfpathlineto{\pgfqpoint{0.887854in}{2.297001in}}%
\pgfpathlineto{\pgfqpoint{0.872131in}{2.255971in}}%
\pgfpathlineto{\pgfqpoint{0.857508in}{2.211741in}}%
\pgfpathlineto{\pgfqpoint{0.844762in}{2.166757in}}%
\pgfpathlineto{\pgfqpoint{0.838624in}{2.140306in}}%
\pgfpathlineto{\pgfqpoint{0.826982in}{2.087194in}}%
\pgfpathlineto{\pgfqpoint{0.816322in}{2.028715in}}%
\pgfpathlineto{\pgfqpoint{0.810087in}{1.984495in}}%
\pgfpathlineto{\pgfqpoint{0.808026in}{1.967238in}}%
\pgfpathlineto{\pgfqpoint{0.800076in}{1.898140in}}%
\pgfpathlineto{\pgfqpoint{0.793713in}{1.823823in}}%
\pgfpathlineto{\pgfqpoint{0.788799in}{1.741875in}}%
\pgfpathlineto{\pgfqpoint{0.786199in}{1.677225in}}%
\pgfpathlineto{\pgfqpoint{0.776951in}{1.453481in}}%
\pgfpathlineto{\pgfqpoint{0.773280in}{1.418894in}}%
\pgfpathlineto{\pgfqpoint{0.768298in}{1.389582in}}%
\pgfpathlineto{\pgfqpoint{0.762752in}{1.368108in}}%
\pgfpathlineto{\pgfqpoint{0.756722in}{1.352123in}}%
\pgfpathlineto{\pgfqpoint{0.749752in}{1.339519in}}%
\pgfpathlineto{\pgfqpoint{0.742201in}{1.330599in}}%
\pgfpathlineto{\pgfqpoint{0.734854in}{1.325312in}}%
\pgfpathlineto{\pgfqpoint{0.726558in}{1.322419in}}%
\pgfpathlineto{\pgfqpoint{0.717884in}{1.322223in}}%
\pgfpathlineto{\pgfqpoint{0.709412in}{1.324411in}}%
\pgfpathlineto{\pgfqpoint{0.699548in}{1.329604in}}%
\pgfpathlineto{\pgfqpoint{0.688894in}{1.338203in}}%
\pgfpathlineto{\pgfqpoint{0.677907in}{1.350248in}}%
\pgfpathlineto{\pgfqpoint{0.666886in}{1.365647in}}%
\pgfpathlineto{\pgfqpoint{0.654913in}{1.386417in}}%
\pgfpathlineto{\pgfqpoint{0.642574in}{1.412730in}}%
\pgfpathlineto{\pgfqpoint{0.630328in}{1.444629in}}%
\pgfpathlineto{\pgfqpoint{0.618504in}{1.482081in}}%
\pgfpathlineto{\pgfqpoint{0.608613in}{1.520256in}}%
\pgfpathlineto{\pgfqpoint{0.590203in}{1.612445in}}%
\pgfpathlineto{\pgfqpoint{0.581848in}{1.668884in}}%
\pgfpathlineto{\pgfqpoint{0.573137in}{1.740376in}}%
\pgfpathlineto{\pgfqpoint{0.567062in}{1.807213in}}%
\pgfpathlineto{\pgfqpoint{0.560532in}{1.896510in}}%
\pgfpathlineto{\pgfqpoint{0.555526in}{1.995910in}}%
\pgfpathlineto{\pgfqpoint{0.552564in}{2.097908in}}%
\pgfpathlineto{\pgfqpoint{0.551526in}{2.204935in}}%
\pgfpathlineto{\pgfqpoint{0.552728in}{2.309470in}}%
\pgfpathlineto{\pgfqpoint{0.556011in}{2.403981in}}%
\pgfpathlineto{\pgfqpoint{0.560953in}{2.483430in}}%
\pgfpathlineto{\pgfqpoint{0.567303in}{2.550240in}}%
\pgfpathlineto{\pgfqpoint{0.574928in}{2.606817in}}%
\pgfpathlineto{\pgfqpoint{0.582988in}{2.650657in}}%
\pgfpathlineto{\pgfqpoint{0.592756in}{2.691452in}}%
\pgfpathlineto{\pgfqpoint{0.602650in}{2.721756in}}%
\pgfpathlineto{\pgfqpoint{0.612983in}{2.746441in}}%
\pgfpathlineto{\pgfqpoint{0.624292in}{2.767692in}}%
\pgfpathlineto{\pgfqpoint{0.636231in}{2.785433in}}%
\pgfpathlineto{\pgfqpoint{0.649892in}{2.801461in}}%
\pgfpathlineto{\pgfqpoint{0.663386in}{2.814020in}}%
\pgfpathlineto{\pgfqpoint{0.679842in}{2.826135in}}%
\pgfpathlineto{\pgfqpoint{0.697326in}{2.836197in}}%
\pgfpathlineto{\pgfqpoint{0.715574in}{2.844285in}}%
\pgfpathlineto{\pgfqpoint{0.738439in}{2.852335in}}%
\pgfpathlineto{\pgfqpoint{0.765983in}{2.859639in}}%
\pgfpathlineto{\pgfqpoint{0.800300in}{2.866256in}}%
\pgfpathlineto{\pgfqpoint{0.841340in}{2.871832in}}%
\pgfpathlineto{\pgfqpoint{0.895547in}{2.876803in}}%
\pgfpathlineto{\pgfqpoint{0.969413in}{2.881069in}}%
\pgfpathlineto{\pgfqpoint{1.071608in}{2.884501in}}%
\pgfpathlineto{\pgfqpoint{1.219512in}{2.887074in}}%
\pgfpathlineto{\pgfqpoint{1.471844in}{2.889091in}}%
\pgfpathlineto{\pgfqpoint{1.956941in}{2.890384in}}%
\pgfpathlineto{\pgfqpoint{3.096814in}{2.890781in}}%
\pgfpathlineto{\pgfqpoint{3.995224in}{2.889388in}}%
\pgfpathlineto{\pgfqpoint{4.275833in}{2.887011in}}%
\pgfpathlineto{\pgfqpoint{4.412847in}{2.883743in}}%
\pgfpathlineto{\pgfqpoint{4.491081in}{2.879810in}}%
\pgfpathlineto{\pgfqpoint{4.543127in}{2.875163in}}%
\pgfpathlineto{\pgfqpoint{4.579810in}{2.869841in}}%
\pgfpathlineto{\pgfqpoint{4.607580in}{2.863763in}}%
\pgfpathlineto{\pgfqpoint{4.630623in}{2.856424in}}%
\pgfpathlineto{\pgfqpoint{4.648833in}{2.848228in}}%
\pgfpathlineto{\pgfqpoint{4.664136in}{2.838773in}}%
\pgfpathlineto{\pgfqpoint{4.676470in}{2.828576in}}%
\pgfpathlineto{\pgfqpoint{4.687502in}{2.816585in}}%
\pgfpathlineto{\pgfqpoint{4.697051in}{2.803027in}}%
\pgfpathlineto{\pgfqpoint{4.706194in}{2.786098in}}%
\pgfpathlineto{\pgfqpoint{4.714508in}{2.765827in}}%
\pgfpathlineto{\pgfqpoint{4.722462in}{2.740013in}}%
\pgfpathlineto{\pgfqpoint{4.729577in}{2.708703in}}%
\pgfpathlineto{\pgfqpoint{4.736162in}{2.669601in}}%
\pgfpathlineto{\pgfqpoint{4.742419in}{2.617826in}}%
\pgfpathlineto{\pgfqpoint{4.747859in}{2.553410in}}%
\pgfpathlineto{\pgfqpoint{4.752661in}{2.468958in}}%
\pgfpathlineto{\pgfqpoint{4.756610in}{2.359528in}}%
\pgfpathlineto{\pgfqpoint{4.759416in}{2.217681in}}%
\pgfpathlineto{\pgfqpoint{4.760596in}{2.043444in}}%
\pgfpathlineto{\pgfqpoint{4.759662in}{1.851779in}}%
\pgfpathlineto{\pgfqpoint{4.756587in}{1.667613in}}%
\pgfpathlineto{\pgfqpoint{4.751596in}{1.503428in}}%
\pgfpathlineto{\pgfqpoint{4.745410in}{1.374185in}}%
\pgfpathlineto{\pgfqpoint{4.738113in}{1.267479in}}%
\pgfpathlineto{\pgfqpoint{4.729621in}{1.175896in}}%
\pgfpathlineto{\pgfqpoint{4.720762in}{1.104428in}}%
\pgfpathlineto{\pgfqpoint{4.711045in}{1.043204in}}%
\pgfpathlineto{\pgfqpoint{4.700364in}{0.989829in}}%
\pgfpathlineto{\pgfqpoint{4.689055in}{0.944345in}}%
\pgfpathlineto{\pgfqpoint{4.676881in}{0.904394in}}%
\pgfpathlineto{\pgfqpoint{4.676095in}{0.902073in}}%
\pgfpathlineto{\pgfqpoint{4.676095in}{0.902073in}}%
\pgfusepath{stroke}%
\end{pgfscope}%
\begin{pgfscope}%
\pgfpathrectangle{\pgfqpoint{0.448634in}{0.402556in}}{\pgfqpoint{4.350661in}{2.489204in}} %
\pgfusepath{clip}%
\pgfsetrectcap%
\pgfsetroundjoin%
\pgfsetlinewidth{1.003750pt}%
\definecolor{currentstroke}{rgb}{0.172549,0.627451,0.172549}%
\pgfsetstrokecolor{currentstroke}%
\pgfsetdash{}{0pt}%
\pgfpathmoveto{\pgfqpoint{2.795520in}{1.982745in}}%
\pgfpathlineto{\pgfqpoint{2.781780in}{1.874357in}}%
\pgfpathlineto{\pgfqpoint{2.769351in}{1.758234in}}%
\pgfpathlineto{\pgfqpoint{2.758095in}{1.631942in}}%
\pgfpathlineto{\pgfqpoint{2.747786in}{1.490551in}}%
\pgfpathlineto{\pgfqpoint{2.738644in}{1.334082in}}%
\pgfpathlineto{\pgfqpoint{2.730580in}{1.157591in}}%
\pgfpathlineto{\pgfqpoint{2.723334in}{0.948663in}}%
\pgfpathlineto{\pgfqpoint{2.709783in}{0.530788in}}%
\pgfpathlineto{\pgfqpoint{2.705868in}{0.488716in}}%
\pgfpathlineto{\pgfqpoint{2.701769in}{0.464281in}}%
\pgfpathlineto{\pgfqpoint{2.697021in}{0.447744in}}%
\pgfpathlineto{\pgfqpoint{2.691859in}{0.436812in}}%
\pgfpathlineto{\pgfqpoint{2.686245in}{0.429229in}}%
\pgfpathlineto{\pgfqpoint{2.679348in}{0.423188in}}%
\pgfpathlineto{\pgfqpoint{2.669540in}{0.417856in}}%
\pgfpathlineto{\pgfqpoint{2.656987in}{0.413810in}}%
\pgfpathlineto{\pgfqpoint{2.637654in}{0.410337in}}%
\pgfpathlineto{\pgfqpoint{2.607297in}{0.407617in}}%
\pgfpathlineto{\pgfqpoint{2.555121in}{0.405574in}}%
\pgfpathlineto{\pgfqpoint{2.450714in}{0.404139in}}%
\pgfpathlineto{\pgfqpoint{2.176624in}{0.403275in}}%
\pgfpathlineto{\pgfqpoint{1.130290in}{0.402953in}}%
\pgfpathlineto{\pgfqpoint{0.516849in}{0.404175in}}%
\pgfpathlineto{\pgfqpoint{0.466848in}{0.405970in}}%
\pgfpathlineto{\pgfqpoint{0.456130in}{0.407931in}}%
\pgfpathlineto{\pgfqpoint{0.452340in}{0.410303in}}%
\pgfpathlineto{\pgfqpoint{0.450346in}{0.414662in}}%
\pgfpathlineto{\pgfqpoint{0.449266in}{0.424524in}}%
\pgfpathlineto{\pgfqpoint{0.448771in}{0.464344in}}%
\pgfpathlineto{\pgfqpoint{0.448640in}{0.850171in}}%
\pgfpathlineto{\pgfqpoint{0.448679in}{2.891318in}}%
\pgfpathlineto{\pgfqpoint{0.448679in}{2.891318in}}%
\pgfusepath{stroke}%
\end{pgfscope}%
\begin{pgfscope}%
\pgfpathrectangle{\pgfqpoint{0.448634in}{0.402556in}}{\pgfqpoint{4.350661in}{2.489204in}} %
\pgfusepath{clip}%
\pgfsetrectcap%
\pgfsetroundjoin%
\pgfsetlinewidth{1.003750pt}%
\definecolor{currentstroke}{rgb}{0.172549,0.627451,0.172549}%
\pgfsetstrokecolor{currentstroke}%
\pgfsetdash{}{0pt}%
\pgfpathmoveto{\pgfqpoint{3.428189in}{0.402586in}}%
\pgfpathlineto{\pgfqpoint{2.782121in}{0.403701in}}%
\pgfpathlineto{\pgfqpoint{2.753906in}{0.405674in}}%
\pgfpathlineto{\pgfqpoint{2.743328in}{0.408444in}}%
\pgfpathlineto{\pgfqpoint{2.737717in}{0.412188in}}%
\pgfpathlineto{\pgfqpoint{2.733668in}{0.417995in}}%
\pgfpathlineto{\pgfqpoint{2.730649in}{0.427307in}}%
\pgfpathlineto{\pgfqpoint{2.728388in}{0.442004in}}%
\pgfpathlineto{\pgfqpoint{2.726544in}{0.471794in}}%
\pgfpathlineto{\pgfqpoint{2.725216in}{0.534003in}}%
\pgfpathlineto{\pgfqpoint{2.725169in}{0.655973in}}%
\pgfpathlineto{\pgfqpoint{2.727377in}{0.832687in}}%
\pgfpathlineto{\pgfqpoint{2.732259in}{1.041703in}}%
\pgfpathlineto{\pgfqpoint{2.738851in}{1.223257in}}%
\pgfpathlineto{\pgfqpoint{2.747078in}{1.389766in}}%
\pgfpathlineto{\pgfqpoint{2.756608in}{1.538717in}}%
\pgfpathlineto{\pgfqpoint{2.768955in}{1.694887in}}%
\pgfpathlineto{\pgfqpoint{2.781228in}{1.816044in}}%
\pgfpathlineto{\pgfqpoint{2.794401in}{1.924524in}}%
\pgfpathlineto{\pgfqpoint{2.812737in}{2.054722in}}%
\pgfpathlineto{\pgfqpoint{2.828774in}{2.147512in}}%
\pgfpathlineto{\pgfqpoint{2.847382in}{2.242224in}}%
\pgfpathlineto{\pgfqpoint{2.895818in}{2.479699in}}%
\pgfpathlineto{\pgfqpoint{2.900204in}{2.516689in}}%
\pgfpathlineto{\pgfqpoint{2.901346in}{2.544029in}}%
\pgfpathlineto{\pgfqpoint{2.900291in}{2.566388in}}%
\pgfpathlineto{\pgfqpoint{2.897334in}{2.585999in}}%
\pgfpathlineto{\pgfqpoint{2.892836in}{2.602633in}}%
\pgfpathlineto{\pgfqpoint{2.886394in}{2.618406in}}%
\pgfpathlineto{\pgfqpoint{2.878058in}{2.632970in}}%
\pgfpathlineto{\pgfqpoint{2.868065in}{2.646100in}}%
\pgfpathlineto{\pgfqpoint{2.855050in}{2.659300in}}%
\pgfpathlineto{\pgfqpoint{2.840801in}{2.670717in}}%
\pgfpathlineto{\pgfqpoint{2.821822in}{2.682861in}}%
\pgfpathlineto{\pgfqpoint{2.799980in}{2.694026in}}%
\pgfpathlineto{\pgfqpoint{2.773366in}{2.704944in}}%
\pgfpathlineto{\pgfqpoint{2.742012in}{2.715266in}}%
\pgfpathlineto{\pgfqpoint{2.705983in}{2.724785in}}%
\pgfpathlineto{\pgfqpoint{2.663200in}{2.733810in}}%
\pgfpathlineto{\pgfqpoint{2.611535in}{2.742379in}}%
\pgfpathlineto{\pgfqpoint{2.551002in}{2.750090in}}%
\pgfpathlineto{\pgfqpoint{2.481632in}{2.756682in}}%
\pgfpathlineto{\pgfqpoint{2.399112in}{2.762200in}}%
\pgfpathlineto{\pgfqpoint{2.309985in}{2.765886in}}%
\pgfpathlineto{\pgfqpoint{2.188184in}{2.768096in}}%
\pgfpathlineto{\pgfqpoint{2.081595in}{2.767619in}}%
\pgfpathlineto{\pgfqpoint{1.968506in}{2.764840in}}%
\pgfpathlineto{\pgfqpoint{1.864180in}{2.759918in}}%
\pgfpathlineto{\pgfqpoint{1.757786in}{2.752593in}}%
\pgfpathlineto{\pgfqpoint{1.671087in}{2.744171in}}%
\pgfpathlineto{\pgfqpoint{1.591076in}{2.734193in}}%
\pgfpathlineto{\pgfqpoint{1.502689in}{2.720717in}}%
\pgfpathlineto{\pgfqpoint{1.427655in}{2.706083in}}%
\pgfpathlineto{\pgfqpoint{1.372350in}{2.692544in}}%
\pgfpathlineto{\pgfqpoint{1.321734in}{2.677921in}}%
\pgfpathlineto{\pgfqpoint{1.273765in}{2.661664in}}%
\pgfpathlineto{\pgfqpoint{1.230567in}{2.644672in}}%
\pgfpathlineto{\pgfqpoint{1.192197in}{2.627106in}}%
\pgfpathlineto{\pgfqpoint{1.156620in}{2.608403in}}%
\pgfpathlineto{\pgfqpoint{1.123890in}{2.588716in}}%
\pgfpathlineto{\pgfqpoint{1.095883in}{2.569568in}}%
\pgfpathlineto{\pgfqpoint{1.063936in}{2.543701in}}%
\pgfpathlineto{\pgfqpoint{1.038217in}{2.520732in}}%
\pgfpathlineto{\pgfqpoint{1.013766in}{2.496016in}}%
\pgfpathlineto{\pgfqpoint{0.990704in}{2.469610in}}%
\pgfpathlineto{\pgfqpoint{0.969124in}{2.441612in}}%
\pgfpathlineto{\pgfqpoint{0.949083in}{2.412154in}}%
\pgfpathlineto{\pgfqpoint{0.930604in}{2.381387in}}%
\pgfpathlineto{\pgfqpoint{0.906555in}{2.334052in}}%
\pgfpathlineto{\pgfqpoint{0.889925in}{2.296262in}}%
\pgfpathlineto{\pgfqpoint{0.874241in}{2.255213in}}%
\pgfpathlineto{\pgfqpoint{0.859667in}{2.210961in}}%
\pgfpathlineto{\pgfqpoint{0.846986in}{2.165954in}}%
\pgfpathlineto{\pgfqpoint{0.839633in}{2.134715in}}%
\pgfpathlineto{\pgfqpoint{0.828238in}{2.081532in}}%
\pgfpathlineto{\pgfqpoint{0.817866in}{2.022986in}}%
\pgfpathlineto{\pgfqpoint{0.810784in}{1.971352in}}%
\pgfpathlineto{\pgfqpoint{0.802845in}{1.902252in}}%
\pgfpathlineto{\pgfqpoint{0.796554in}{1.827927in}}%
\pgfpathlineto{\pgfqpoint{0.791696in}{1.743480in}}%
\pgfpathlineto{\pgfqpoint{0.787773in}{1.621595in}}%
\pgfpathlineto{\pgfqpoint{0.785407in}{1.522064in}}%
\pgfpathlineto{\pgfqpoint{0.785407in}{1.522064in}}%
\pgfusepath{stroke}%
\end{pgfscope}%
\begin{pgfscope}%
\pgfpathrectangle{\pgfqpoint{0.448634in}{0.402556in}}{\pgfqpoint{4.350661in}{2.489204in}} %
\pgfusepath{clip}%
\pgfsetrectcap%
\pgfsetroundjoin%
\pgfsetlinewidth{1.003750pt}%
\definecolor{currentstroke}{rgb}{0.839216,0.152941,0.156863}%
\pgfsetstrokecolor{currentstroke}%
\pgfsetdash{}{0pt}%
\pgfpathmoveto{\pgfqpoint{1.127319in}{2.572074in}}%
\pgfpathlineto{\pgfqpoint{1.159575in}{2.592758in}}%
\pgfpathlineto{\pgfqpoint{1.192763in}{2.611414in}}%
\pgfpathlineto{\pgfqpoint{1.228726in}{2.629126in}}%
\pgfpathlineto{\pgfqpoint{1.267413in}{2.645758in}}%
\pgfpathlineto{\pgfqpoint{1.310846in}{2.661945in}}%
\pgfpathlineto{\pgfqpoint{1.356920in}{2.676740in}}%
\pgfpathlineto{\pgfqpoint{1.407680in}{2.690702in}}%
\pgfpathlineto{\pgfqpoint{1.463094in}{2.703640in}}%
\pgfpathlineto{\pgfqpoint{1.525273in}{2.715813in}}%
\pgfpathlineto{\pgfqpoint{1.594199in}{2.726937in}}%
\pgfpathlineto{\pgfqpoint{1.669843in}{2.736808in}}%
\pgfpathlineto{\pgfqpoint{1.752172in}{2.745271in}}%
\pgfpathlineto{\pgfqpoint{1.843325in}{2.752344in}}%
\pgfpathlineto{\pgfqpoint{1.941103in}{2.757656in}}%
\pgfpathlineto{\pgfqpoint{2.043301in}{2.760987in}}%
\pgfpathlineto{\pgfqpoint{2.147710in}{2.762199in}}%
\pgfpathlineto{\pgfqpoint{2.249945in}{2.761215in}}%
\pgfpathlineto{\pgfqpoint{2.345620in}{2.758145in}}%
\pgfpathlineto{\pgfqpoint{2.432525in}{2.753210in}}%
\pgfpathlineto{\pgfqpoint{2.508451in}{2.746766in}}%
\pgfpathlineto{\pgfqpoint{2.573368in}{2.739156in}}%
\pgfpathlineto{\pgfqpoint{2.629410in}{2.730451in}}%
\pgfpathlineto{\pgfqpoint{2.676543in}{2.720985in}}%
\pgfpathlineto{\pgfqpoint{2.716874in}{2.710666in}}%
\pgfpathlineto{\pgfqpoint{2.750366in}{2.699848in}}%
\pgfpathlineto{\pgfqpoint{2.779059in}{2.688192in}}%
\pgfpathlineto{\pgfqpoint{2.802882in}{2.676004in}}%
\pgfpathlineto{\pgfqpoint{2.821842in}{2.663819in}}%
\pgfpathlineto{\pgfqpoint{2.837815in}{2.650886in}}%
\pgfpathlineto{\pgfqpoint{2.850736in}{2.637564in}}%
\pgfpathlineto{\pgfqpoint{2.860694in}{2.624398in}}%
\pgfpathlineto{\pgfqpoint{2.869084in}{2.609873in}}%
\pgfpathlineto{\pgfqpoint{2.875698in}{2.594192in}}%
\pgfpathlineto{\pgfqpoint{2.881035in}{2.575255in}}%
\pgfpathlineto{\pgfqpoint{2.884200in}{2.555685in}}%
\pgfpathlineto{\pgfqpoint{2.885619in}{2.533351in}}%
\pgfpathlineto{\pgfqpoint{2.885038in}{2.505987in}}%
\pgfpathlineto{\pgfqpoint{2.882112in}{2.473807in}}%
\pgfpathlineto{\pgfqpoint{2.875657in}{2.429620in}}%
\pgfpathlineto{\pgfqpoint{2.863489in}{2.363873in}}%
\pgfpathlineto{\pgfqpoint{2.821102in}{2.142619in}}%
\pgfpathlineto{\pgfqpoint{2.804859in}{2.042271in}}%
\pgfpathlineto{\pgfqpoint{2.790421in}{1.939040in}}%
\pgfpathlineto{\pgfqpoint{2.777207in}{1.828054in}}%
\pgfpathlineto{\pgfqpoint{2.765338in}{1.709349in}}%
\pgfpathlineto{\pgfqpoint{2.754471in}{1.578010in}}%
\pgfpathlineto{\pgfqpoint{2.744640in}{1.431580in}}%
\pgfpathlineto{\pgfqpoint{2.735914in}{1.267598in}}%
\pgfpathlineto{\pgfqpoint{2.728277in}{1.081114in}}%
\pgfpathlineto{\pgfqpoint{2.721437in}{0.857223in}}%
\pgfpathlineto{\pgfqpoint{2.711961in}{0.541290in}}%
\pgfpathlineto{\pgfqpoint{2.708250in}{0.491694in}}%
\pgfpathlineto{\pgfqpoint{2.703951in}{0.462246in}}%
\pgfpathlineto{\pgfqpoint{2.699504in}{0.445599in}}%
\pgfpathlineto{\pgfqpoint{2.694517in}{0.434563in}}%
\pgfpathlineto{\pgfqpoint{2.688942in}{0.426947in}}%
\pgfpathlineto{\pgfqpoint{2.681980in}{0.421009in}}%
\pgfpathlineto{\pgfqpoint{2.672064in}{0.415948in}}%
\pgfpathlineto{\pgfqpoint{2.659429in}{0.412247in}}%
\pgfpathlineto{\pgfqpoint{2.640044in}{0.409163in}}%
\pgfpathlineto{\pgfqpoint{2.607490in}{0.406692in}}%
\pgfpathlineto{\pgfqpoint{2.548779in}{0.404894in}}%
\pgfpathlineto{\pgfqpoint{2.422615in}{0.403701in}}%
\pgfpathlineto{\pgfqpoint{2.026705in}{0.403016in}}%
\pgfpathlineto{\pgfqpoint{0.623617in}{0.403253in}}%
\pgfpathlineto{\pgfqpoint{0.477880in}{0.404742in}}%
\pgfpathlineto{\pgfqpoint{0.458368in}{0.406382in}}%
\pgfpathlineto{\pgfqpoint{0.452304in}{0.408937in}}%
\pgfpathlineto{\pgfqpoint{0.450213in}{0.413215in}}%
\pgfpathlineto{\pgfqpoint{0.449165in}{0.423080in}}%
\pgfpathlineto{\pgfqpoint{0.448735in}{0.465392in}}%
\pgfpathlineto{\pgfqpoint{0.448637in}{0.983146in}}%
\pgfpathlineto{\pgfqpoint{0.448652in}{2.889876in}}%
\pgfpathlineto{\pgfqpoint{0.448652in}{2.889876in}}%
\pgfusepath{stroke}%
\end{pgfscope}%
\begin{pgfscope}%
\pgfpathrectangle{\pgfqpoint{0.448634in}{0.402556in}}{\pgfqpoint{4.350661in}{2.489204in}} %
\pgfusepath{clip}%
\pgfsetrectcap%
\pgfsetroundjoin%
\pgfsetlinewidth{1.003750pt}%
\definecolor{currentstroke}{rgb}{0.839216,0.152941,0.156863}%
\pgfsetstrokecolor{currentstroke}%
\pgfsetdash{}{0pt}%
\pgfpathmoveto{\pgfqpoint{0.448634in}{2.896245in}}%
\pgfpathlineto{\pgfqpoint{0.448593in}{0.407043in}}%
\pgfpathlineto{\pgfqpoint{0.448593in}{0.407043in}}%
\pgfusepath{stroke}%
\end{pgfscope}%
\begin{pgfscope}%
\pgfpathrectangle{\pgfqpoint{0.448634in}{0.402556in}}{\pgfqpoint{4.350661in}{2.489204in}} %
\pgfusepath{clip}%
\pgfsetrectcap%
\pgfsetroundjoin%
\pgfsetlinewidth{1.003750pt}%
\definecolor{currentstroke}{rgb}{0.839216,0.152941,0.156863}%
\pgfsetstrokecolor{currentstroke}%
\pgfsetdash{}{0pt}%
\pgfpathmoveto{\pgfqpoint{0.576853in}{1.760817in}}%
\pgfpathlineto{\pgfqpoint{0.569394in}{1.840010in}}%
\pgfpathlineto{\pgfqpoint{0.563208in}{1.929339in}}%
\pgfpathlineto{\pgfqpoint{0.558592in}{2.028764in}}%
\pgfpathlineto{\pgfqpoint{0.555985in}{2.133265in}}%
\pgfpathlineto{\pgfqpoint{0.555565in}{2.237808in}}%
\pgfpathlineto{\pgfqpoint{0.557371in}{2.337352in}}%
\pgfpathlineto{\pgfqpoint{0.561096in}{2.424366in}}%
\pgfpathlineto{\pgfqpoint{0.566403in}{2.498791in}}%
\pgfpathlineto{\pgfqpoint{0.572909in}{2.560570in}}%
\pgfpathlineto{\pgfqpoint{0.580458in}{2.612119in}}%
\pgfpathlineto{\pgfqpoint{0.589086in}{2.655816in}}%
\pgfpathlineto{\pgfqpoint{0.598406in}{2.691590in}}%
\pgfpathlineto{\pgfqpoint{0.608613in}{2.721757in}}%
\pgfpathlineto{\pgfqpoint{0.619241in}{2.746278in}}%
\pgfpathlineto{\pgfqpoint{0.630817in}{2.767339in}}%
\pgfpathlineto{\pgfqpoint{0.642975in}{2.784884in}}%
\pgfpathlineto{\pgfqpoint{0.656813in}{2.800713in}}%
\pgfpathlineto{\pgfqpoint{0.672197in}{2.814549in}}%
\pgfpathlineto{\pgfqpoint{0.688853in}{2.826301in}}%
\pgfpathlineto{\pgfqpoint{0.706461in}{2.836076in}}%
\pgfpathlineto{\pgfqpoint{0.726804in}{2.844876in}}%
\pgfpathlineto{\pgfqpoint{0.751866in}{2.853203in}}%
\pgfpathlineto{\pgfqpoint{0.781632in}{2.860547in}}%
\pgfpathlineto{\pgfqpoint{0.818168in}{2.867054in}}%
\pgfpathlineto{\pgfqpoint{0.863581in}{2.872685in}}%
\pgfpathlineto{\pgfqpoint{0.922161in}{2.877518in}}%
\pgfpathlineto{\pgfqpoint{1.000391in}{2.881567in}}%
\pgfpathlineto{\pgfqpoint{1.111294in}{2.884881in}}%
\pgfpathlineto{\pgfqpoint{1.274428in}{2.887367in}}%
\pgfpathlineto{\pgfqpoint{1.552865in}{2.889263in}}%
\pgfpathlineto{\pgfqpoint{2.107573in}{2.890457in}}%
\pgfpathlineto{\pgfqpoint{3.343161in}{2.890573in}}%
\pgfpathlineto{\pgfqpoint{4.043615in}{2.888941in}}%
\pgfpathlineto{\pgfqpoint{4.289417in}{2.886404in}}%
\pgfpathlineto{\pgfqpoint{4.413375in}{2.883093in}}%
\pgfpathlineto{\pgfqpoint{4.489425in}{2.878997in}}%
\pgfpathlineto{\pgfqpoint{4.541451in}{2.874081in}}%
\pgfpathlineto{\pgfqpoint{4.578100in}{2.868470in}}%
\pgfpathlineto{\pgfqpoint{4.605819in}{2.862092in}}%
\pgfpathlineto{\pgfqpoint{4.626726in}{2.855245in}}%
\pgfpathlineto{\pgfqpoint{4.644925in}{2.847018in}}%
\pgfpathlineto{\pgfqpoint{4.660241in}{2.837590in}}%
\pgfpathlineto{\pgfqpoint{4.672623in}{2.827468in}}%
\pgfpathlineto{\pgfqpoint{4.683751in}{2.815592in}}%
\pgfpathlineto{\pgfqpoint{4.693406in}{2.802135in}}%
\pgfpathlineto{\pgfqpoint{4.702740in}{2.785343in}}%
\pgfpathlineto{\pgfqpoint{4.711277in}{2.765194in}}%
\pgfpathlineto{\pgfqpoint{4.719483in}{2.739484in}}%
\pgfpathlineto{\pgfqpoint{4.726294in}{2.710657in}}%
\pgfpathlineto{\pgfqpoint{4.733260in}{2.671642in}}%
\pgfpathlineto{\pgfqpoint{4.739604in}{2.622396in}}%
\pgfpathlineto{\pgfqpoint{4.745236in}{2.560504in}}%
\pgfpathlineto{\pgfqpoint{4.750164in}{2.481051in}}%
\pgfpathlineto{\pgfqpoint{4.754367in}{2.376618in}}%
\pgfpathlineto{\pgfqpoint{4.757443in}{2.242249in}}%
\pgfpathlineto{\pgfqpoint{4.758977in}{2.075483in}}%
\pgfpathlineto{\pgfqpoint{4.758447in}{1.888795in}}%
\pgfpathlineto{\pgfqpoint{4.755756in}{1.707110in}}%
\pgfpathlineto{\pgfqpoint{4.750925in}{1.532957in}}%
\pgfpathlineto{\pgfqpoint{4.744785in}{1.398726in}}%
\pgfpathlineto{\pgfqpoint{4.737575in}{1.289515in}}%
\pgfpathlineto{\pgfqpoint{4.728714in}{1.190469in}}%
\pgfpathlineto{\pgfqpoint{4.719652in}{1.116521in}}%
\pgfpathlineto{\pgfqpoint{4.710036in}{1.055276in}}%
\pgfpathlineto{\pgfqpoint{4.699503in}{1.001861in}}%
\pgfpathlineto{\pgfqpoint{4.689040in}{0.958690in}}%
\pgfpathlineto{\pgfqpoint{4.677219in}{0.918600in}}%
\pgfpathlineto{\pgfqpoint{4.664034in}{0.881749in}}%
\pgfpathlineto{\pgfqpoint{4.650584in}{0.850491in}}%
\pgfpathlineto{\pgfqpoint{4.636303in}{0.822569in}}%
\pgfpathlineto{\pgfqpoint{4.620207in}{0.795974in}}%
\pgfpathlineto{\pgfqpoint{4.603640in}{0.772901in}}%
\pgfpathlineto{\pgfqpoint{4.585488in}{0.751446in}}%
\pgfpathlineto{\pgfqpoint{4.565874in}{0.731749in}}%
\pgfpathlineto{\pgfqpoint{4.544964in}{0.713879in}}%
\pgfpathlineto{\pgfqpoint{4.522958in}{0.697824in}}%
\pgfpathlineto{\pgfqpoint{4.496157in}{0.681290in}}%
\pgfpathlineto{\pgfqpoint{4.470397in}{0.667953in}}%
\pgfpathlineto{\pgfqpoint{4.439961in}{0.654509in}}%
\pgfpathlineto{\pgfqpoint{4.406841in}{0.642281in}}%
\pgfpathlineto{\pgfqpoint{4.369009in}{0.630748in}}%
\pgfpathlineto{\pgfqpoint{4.326489in}{0.620226in}}%
\pgfpathlineto{\pgfqpoint{4.279327in}{0.610949in}}%
\pgfpathlineto{\pgfqpoint{4.227576in}{0.603085in}}%
\pgfpathlineto{\pgfqpoint{4.173450in}{0.597063in}}%
\pgfpathlineto{\pgfqpoint{4.110511in}{0.592203in}}%
\pgfpathlineto{\pgfqpoint{4.047471in}{0.589537in}}%
\pgfpathlineto{\pgfqpoint{3.977867in}{0.588624in}}%
\pgfpathlineto{\pgfqpoint{3.906093in}{0.589934in}}%
\pgfpathlineto{\pgfqpoint{3.834377in}{0.593496in}}%
\pgfpathlineto{\pgfqpoint{3.767120in}{0.599067in}}%
\pgfpathlineto{\pgfqpoint{3.704364in}{0.606392in}}%
\pgfpathlineto{\pgfqpoint{3.678516in}{0.610510in}}%
\pgfpathlineto{\pgfqpoint{3.620438in}{0.620500in}}%
\pgfpathlineto{\pgfqpoint{3.586319in}{0.628207in}}%
\pgfpathlineto{\pgfqpoint{3.495240in}{0.652428in}}%
\pgfpathlineto{\pgfqpoint{3.451528in}{0.667583in}}%
\pgfpathlineto{\pgfqpoint{3.408538in}{0.685220in}}%
\pgfpathlineto{\pgfqpoint{3.374594in}{0.702001in}}%
\pgfpathlineto{\pgfqpoint{3.345407in}{0.718682in}}%
\pgfpathlineto{\pgfqpoint{3.315236in}{0.738520in}}%
\pgfpathlineto{\pgfqpoint{3.288127in}{0.759290in}}%
\pgfpathlineto{\pgfqpoint{3.264004in}{0.780551in}}%
\pgfpathlineto{\pgfqpoint{3.241208in}{0.803648in}}%
\pgfpathlineto{\pgfqpoint{3.219894in}{0.828530in}}%
\pgfpathlineto{\pgfqpoint{3.200189in}{0.855091in}}%
\pgfpathlineto{\pgfqpoint{3.182177in}{0.883183in}}%
\pgfpathlineto{\pgfqpoint{3.165906in}{0.912633in}}%
\pgfpathlineto{\pgfqpoint{3.150351in}{0.945448in}}%
\pgfpathlineto{\pgfqpoint{3.136682in}{0.979345in}}%
\pgfpathlineto{\pgfqpoint{3.124073in}{1.016460in}}%
\pgfpathlineto{\pgfqpoint{3.112834in}{1.056769in}}%
\pgfpathlineto{\pgfqpoint{3.103046in}{1.100146in}}%
\pgfpathlineto{\pgfqpoint{3.095343in}{1.144071in}}%
\pgfpathlineto{\pgfqpoint{3.089208in}{1.190837in}}%
\pgfpathlineto{\pgfqpoint{3.084595in}{1.242838in}}%
\pgfpathlineto{\pgfqpoint{3.082137in}{1.295031in}}%
\pgfpathlineto{\pgfqpoint{3.081687in}{1.349787in}}%
\pgfpathlineto{\pgfqpoint{3.083451in}{1.406998in}}%
\pgfpathlineto{\pgfqpoint{3.087181in}{1.461589in}}%
\pgfpathlineto{\pgfqpoint{3.093485in}{1.520888in}}%
\pgfpathlineto{\pgfqpoint{3.101823in}{1.577334in}}%
\pgfpathlineto{\pgfqpoint{3.111930in}{1.630856in}}%
\pgfpathlineto{\pgfqpoint{3.124690in}{1.686208in}}%
\pgfpathlineto{\pgfqpoint{3.139178in}{1.738395in}}%
\pgfpathlineto{\pgfqpoint{3.155145in}{1.787366in}}%
\pgfpathlineto{\pgfqpoint{3.172353in}{1.833085in}}%
\pgfpathlineto{\pgfqpoint{3.191618in}{1.877716in}}%
\pgfpathlineto{\pgfqpoint{3.214026in}{1.923261in}}%
\pgfpathlineto{\pgfqpoint{3.236214in}{1.963157in}}%
\pgfpathlineto{\pgfqpoint{3.260178in}{2.001684in}}%
\pgfpathlineto{\pgfqpoint{3.285814in}{2.038776in}}%
\pgfpathlineto{\pgfqpoint{3.314415in}{2.076285in}}%
\pgfpathlineto{\pgfqpoint{3.348944in}{2.117711in}}%
\pgfpathlineto{\pgfqpoint{3.417133in}{2.198022in}}%
\pgfpathlineto{\pgfqpoint{3.426053in}{2.212128in}}%
\pgfpathlineto{\pgfqpoint{3.430798in}{2.223297in}}%
\pgfpathlineto{\pgfqpoint{3.432034in}{2.230603in}}%
\pgfpathlineto{\pgfqpoint{3.430773in}{2.237856in}}%
\pgfpathlineto{\pgfqpoint{3.426621in}{2.243526in}}%
\pgfpathlineto{\pgfqpoint{3.420908in}{2.247084in}}%
\pgfpathlineto{\pgfqpoint{3.412501in}{2.249583in}}%
\pgfpathlineto{\pgfqpoint{3.399499in}{2.250689in}}%
\pgfpathlineto{\pgfqpoint{3.384305in}{2.249671in}}%
\pgfpathlineto{\pgfqpoint{3.364985in}{2.246098in}}%
\pgfpathlineto{\pgfqpoint{3.341804in}{2.239342in}}%
\pgfpathlineto{\pgfqpoint{3.317109in}{2.229682in}}%
\pgfpathlineto{\pgfqpoint{3.291104in}{2.216986in}}%
\pgfpathlineto{\pgfqpoint{3.265928in}{2.202261in}}%
\pgfpathlineto{\pgfqpoint{3.239805in}{2.184361in}}%
\pgfpathlineto{\pgfqpoint{3.214775in}{2.164519in}}%
\pgfpathlineto{\pgfqpoint{3.190900in}{2.142893in}}%
\pgfpathlineto{\pgfqpoint{3.166657in}{2.117912in}}%
\pgfpathlineto{\pgfqpoint{3.143835in}{2.091233in}}%
\pgfpathlineto{\pgfqpoint{3.121079in}{2.061107in}}%
\pgfpathlineto{\pgfqpoint{3.099952in}{2.029463in}}%
\pgfpathlineto{\pgfqpoint{3.079251in}{1.994406in}}%
\pgfpathlineto{\pgfqpoint{3.059218in}{1.955915in}}%
\pgfpathlineto{\pgfqpoint{3.040058in}{1.914015in}}%
\pgfpathlineto{\pgfqpoint{3.022809in}{1.871041in}}%
\pgfpathlineto{\pgfqpoint{3.005790in}{1.822536in}}%
\pgfpathlineto{\pgfqpoint{2.990067in}{1.770819in}}%
\pgfpathlineto{\pgfqpoint{2.975708in}{1.715979in}}%
\pgfpathlineto{\pgfqpoint{2.962284in}{1.655680in}}%
\pgfpathlineto{\pgfqpoint{2.950496in}{1.592386in}}%
\pgfpathlineto{\pgfqpoint{2.940383in}{1.526185in}}%
\pgfpathlineto{\pgfqpoint{2.931745in}{1.454681in}}%
\pgfpathlineto{\pgfqpoint{2.925082in}{1.380399in}}%
\pgfpathlineto{\pgfqpoint{2.920647in}{1.305899in}}%
\pgfpathlineto{\pgfqpoint{2.918444in}{1.231270in}}%
\pgfpathlineto{\pgfqpoint{2.918545in}{1.159087in}}%
\pgfpathlineto{\pgfqpoint{2.920787in}{1.091931in}}%
\pgfpathlineto{\pgfqpoint{2.925177in}{1.027412in}}%
\pgfpathlineto{\pgfqpoint{2.931192in}{0.970580in}}%
\pgfpathlineto{\pgfqpoint{2.938760in}{0.919034in}}%
\pgfpathlineto{\pgfqpoint{2.947651in}{0.872852in}}%
\pgfpathlineto{\pgfqpoint{2.958213in}{0.829714in}}%
\pgfpathlineto{\pgfqpoint{2.969670in}{0.792114in}}%
\pgfpathlineto{\pgfqpoint{2.982463in}{0.757773in}}%
\pgfpathlineto{\pgfqpoint{2.996425in}{0.726812in}}%
\pgfpathlineto{\pgfqpoint{3.011299in}{0.699300in}}%
\pgfpathlineto{\pgfqpoint{3.026739in}{0.675225in}}%
\pgfpathlineto{\pgfqpoint{3.043828in}{0.652656in}}%
\pgfpathlineto{\pgfqpoint{3.062495in}{0.631788in}}%
\pgfpathlineto{\pgfqpoint{3.082602in}{0.612753in}}%
\pgfpathlineto{\pgfqpoint{3.103961in}{0.595592in}}%
\pgfpathlineto{\pgfqpoint{3.128268in}{0.579069in}}%
\pgfpathlineto{\pgfqpoint{3.153537in}{0.564554in}}%
\pgfpathlineto{\pgfqpoint{3.181571in}{0.550952in}}%
\pgfpathlineto{\pgfqpoint{3.214371in}{0.537647in}}%
\pgfpathlineto{\pgfqpoint{3.249846in}{0.525712in}}%
\pgfpathlineto{\pgfqpoint{3.290011in}{0.514571in}}%
\pgfpathlineto{\pgfqpoint{3.334820in}{0.504423in}}%
\pgfpathlineto{\pgfqpoint{3.386372in}{0.494999in}}%
\pgfpathlineto{\pgfqpoint{3.446798in}{0.486257in}}%
\pgfpathlineto{\pgfqpoint{3.518243in}{0.478282in}}%
\pgfpathlineto{\pgfqpoint{3.600685in}{0.471409in}}%
\pgfpathlineto{\pgfqpoint{3.696268in}{0.465713in}}%
\pgfpathlineto{\pgfqpoint{3.807144in}{0.461369in}}%
\pgfpathlineto{\pgfqpoint{3.933291in}{0.458719in}}%
\pgfpathlineto{\pgfqpoint{4.063808in}{0.458211in}}%
\pgfpathlineto{\pgfqpoint{4.187792in}{0.459914in}}%
\pgfpathlineto{\pgfqpoint{4.294335in}{0.463521in}}%
\pgfpathlineto{\pgfqpoint{4.381234in}{0.468574in}}%
\pgfpathlineto{\pgfqpoint{4.450636in}{0.474701in}}%
\pgfpathlineto{\pgfqpoint{4.506850in}{0.481799in}}%
\pgfpathlineto{\pgfqpoint{4.552009in}{0.489658in}}%
\pgfpathlineto{\pgfqpoint{4.588239in}{0.498115in}}%
\pgfpathlineto{\pgfqpoint{4.617656in}{0.507110in}}%
\pgfpathlineto{\pgfqpoint{4.642328in}{0.516843in}}%
\pgfpathlineto{\pgfqpoint{4.664194in}{0.527940in}}%
\pgfpathlineto{\pgfqpoint{4.681238in}{0.538945in}}%
\pgfpathlineto{\pgfqpoint{4.697164in}{0.551953in}}%
\pgfpathlineto{\pgfqpoint{4.710076in}{0.565289in}}%
\pgfpathlineto{\pgfqpoint{4.721578in}{0.580218in}}%
\pgfpathlineto{\pgfqpoint{4.731557in}{0.596521in}}%
\pgfpathlineto{\pgfqpoint{4.741000in}{0.616134in}}%
\pgfpathlineto{\pgfqpoint{4.749521in}{0.639027in}}%
\pgfpathlineto{\pgfqpoint{4.757522in}{0.667450in}}%
\pgfpathlineto{\pgfqpoint{4.764572in}{0.701345in}}%
\pgfpathlineto{\pgfqpoint{4.770840in}{0.743043in}}%
\pgfpathlineto{\pgfqpoint{4.776327in}{0.794934in}}%
\pgfpathlineto{\pgfqpoint{4.781278in}{0.864398in}}%
\pgfpathlineto{\pgfqpoint{4.785468in}{0.956371in}}%
\pgfpathlineto{\pgfqpoint{4.789000in}{1.085745in}}%
\pgfpathlineto{\pgfqpoint{4.791852in}{1.277385in}}%
\pgfpathlineto{\pgfqpoint{4.793959in}{1.581057in}}%
\pgfpathlineto{\pgfqpoint{4.794962in}{2.071429in}}%
\pgfpathlineto{\pgfqpoint{4.793967in}{2.559311in}}%
\pgfpathlineto{\pgfqpoint{4.791733in}{2.745981in}}%
\pgfpathlineto{\pgfqpoint{4.788955in}{2.818091in}}%
\pgfpathlineto{\pgfqpoint{4.785731in}{2.850227in}}%
\pgfpathlineto{\pgfqpoint{4.781879in}{2.867057in}}%
\pgfpathlineto{\pgfqpoint{4.777744in}{2.875780in}}%
\pgfpathlineto{\pgfqpoint{4.773097in}{2.880982in}}%
\pgfpathlineto{\pgfqpoint{4.767363in}{2.884504in}}%
\pgfpathlineto{\pgfqpoint{4.756853in}{2.887622in}}%
\pgfpathlineto{\pgfqpoint{4.739548in}{2.889639in}}%
\pgfpathlineto{\pgfqpoint{4.704762in}{2.890882in}}%
\pgfpathlineto{\pgfqpoint{4.602524in}{2.891538in}}%
\pgfpathlineto{\pgfqpoint{3.952100in}{2.891742in}}%
\pgfpathlineto{\pgfqpoint{0.617321in}{2.890753in}}%
\pgfpathlineto{\pgfqpoint{0.549910in}{2.888858in}}%
\pgfpathlineto{\pgfqpoint{0.521735in}{2.886179in}}%
\pgfpathlineto{\pgfqpoint{0.504666in}{2.882389in}}%
\pgfpathlineto{\pgfqpoint{0.494501in}{2.878011in}}%
\pgfpathlineto{\pgfqpoint{0.487180in}{2.872667in}}%
\pgfpathlineto{\pgfqpoint{0.481152in}{2.865519in}}%
\pgfpathlineto{\pgfqpoint{0.475664in}{2.854804in}}%
\pgfpathlineto{\pgfqpoint{0.471318in}{2.840737in}}%
\pgfpathlineto{\pgfqpoint{0.467301in}{2.818823in}}%
\pgfpathlineto{\pgfqpoint{0.463927in}{2.786700in}}%
\pgfpathlineto{\pgfqpoint{0.460918in}{2.734544in}}%
\pgfpathlineto{\pgfqpoint{0.458363in}{2.647473in}}%
\pgfpathlineto{\pgfqpoint{0.456575in}{2.523031in}}%
\pgfpathlineto{\pgfqpoint{0.456575in}{2.523031in}}%
\pgfusepath{stroke}%
\end{pgfscope}%
\begin{pgfscope}%
\pgfpathrectangle{\pgfqpoint{0.448634in}{0.402556in}}{\pgfqpoint{4.350661in}{2.489204in}} %
\pgfusepath{clip}%
\pgfsetrectcap%
\pgfsetroundjoin%
\pgfsetlinewidth{1.003750pt}%
\definecolor{currentstroke}{rgb}{0.839216,0.152941,0.156863}%
\pgfsetstrokecolor{currentstroke}%
\pgfsetdash{}{0pt}%
\pgfpathmoveto{\pgfqpoint{0.456424in}{1.370137in}}%
\pgfpathlineto{\pgfqpoint{0.459610in}{1.118755in}}%
\pgfpathlineto{\pgfqpoint{0.463695in}{0.962007in}}%
\pgfpathlineto{\pgfqpoint{0.468519in}{0.857610in}}%
\pgfpathlineto{\pgfqpoint{0.474082in}{0.783210in}}%
\pgfpathlineto{\pgfqpoint{0.480226in}{0.728906in}}%
\pgfpathlineto{\pgfqpoint{0.486970in}{0.687306in}}%
\pgfpathlineto{\pgfqpoint{0.494537in}{0.653558in}}%
\pgfpathlineto{\pgfqpoint{0.503107in}{0.625355in}}%
\pgfpathlineto{\pgfqpoint{0.512193in}{0.602749in}}%
\pgfpathlineto{\pgfqpoint{0.522200in}{0.583508in}}%
\pgfpathlineto{\pgfqpoint{0.534108in}{0.565743in}}%
\pgfpathlineto{\pgfqpoint{0.546263in}{0.551507in}}%
\pgfpathlineto{\pgfqpoint{0.559728in}{0.538907in}}%
\pgfpathlineto{\pgfqpoint{0.576130in}{0.526693in}}%
\pgfpathlineto{\pgfqpoint{0.595483in}{0.515351in}}%
\pgfpathlineto{\pgfqpoint{0.617681in}{0.505147in}}%
\pgfpathlineto{\pgfqpoint{0.642568in}{0.496153in}}%
\pgfpathlineto{\pgfqpoint{0.672126in}{0.487778in}}%
\pgfpathlineto{\pgfqpoint{0.708443in}{0.479824in}}%
\pgfpathlineto{\pgfqpoint{0.753649in}{0.472325in}}%
\pgfpathlineto{\pgfqpoint{0.807718in}{0.465660in}}%
\pgfpathlineto{\pgfqpoint{0.877116in}{0.459475in}}%
\pgfpathlineto{\pgfqpoint{0.961828in}{0.454230in}}%
\pgfpathlineto{\pgfqpoint{1.068351in}{0.449916in}}%
\pgfpathlineto{\pgfqpoint{1.201018in}{0.446839in}}%
\pgfpathlineto{\pgfqpoint{1.357637in}{0.445481in}}%
\pgfpathlineto{\pgfqpoint{1.525135in}{0.446232in}}%
\pgfpathlineto{\pgfqpoint{1.686088in}{0.449142in}}%
\pgfpathlineto{\pgfqpoint{1.823074in}{0.453747in}}%
\pgfpathlineto{\pgfqpoint{1.938245in}{0.459764in}}%
\pgfpathlineto{\pgfqpoint{2.031582in}{0.466759in}}%
\pgfpathlineto{\pgfqpoint{2.109580in}{0.474745in}}%
\pgfpathlineto{\pgfqpoint{2.174384in}{0.483535in}}%
\pgfpathlineto{\pgfqpoint{2.228139in}{0.492940in}}%
\pgfpathlineto{\pgfqpoint{2.275119in}{0.503356in}}%
\pgfpathlineto{\pgfqpoint{2.315282in}{0.514501in}}%
\pgfpathlineto{\pgfqpoint{2.350698in}{0.526659in}}%
\pgfpathlineto{\pgfqpoint{2.381320in}{0.539536in}}%
\pgfpathlineto{\pgfqpoint{2.407164in}{0.552659in}}%
\pgfpathlineto{\pgfqpoint{2.430226in}{0.566639in}}%
\pgfpathlineto{\pgfqpoint{2.452282in}{0.582602in}}%
\pgfpathlineto{\pgfqpoint{2.471391in}{0.599069in}}%
\pgfpathlineto{\pgfqpoint{2.489240in}{0.617293in}}%
\pgfpathlineto{\pgfqpoint{2.505678in}{0.637180in}}%
\pgfpathlineto{\pgfqpoint{2.520620in}{0.658557in}}%
\pgfpathlineto{\pgfqpoint{2.535213in}{0.683314in}}%
\pgfpathlineto{\pgfqpoint{2.549115in}{0.711484in}}%
\pgfpathlineto{\pgfqpoint{2.562091in}{0.743004in}}%
\pgfpathlineto{\pgfqpoint{2.574020in}{0.777751in}}%
\pgfpathlineto{\pgfqpoint{2.585502in}{0.817970in}}%
\pgfpathlineto{\pgfqpoint{2.596809in}{0.866038in}}%
\pgfpathlineto{\pgfqpoint{2.607562in}{0.921948in}}%
\pgfpathlineto{\pgfqpoint{2.617925in}{0.988098in}}%
\pgfpathlineto{\pgfqpoint{2.627958in}{1.066918in}}%
\pgfpathlineto{\pgfqpoint{2.637941in}{1.163320in}}%
\pgfpathlineto{\pgfqpoint{2.648424in}{1.287199in}}%
\pgfpathlineto{\pgfqpoint{2.660103in}{1.453438in}}%
\pgfpathlineto{\pgfqpoint{2.674773in}{1.696801in}}%
\pgfpathlineto{\pgfqpoint{2.687716in}{1.945279in}}%
\pgfpathlineto{\pgfqpoint{2.692670in}{2.079573in}}%
\pgfpathlineto{\pgfqpoint{2.693829in}{2.166682in}}%
\pgfpathlineto{\pgfqpoint{2.692565in}{2.233870in}}%
\pgfpathlineto{\pgfqpoint{2.689436in}{2.286015in}}%
\pgfpathlineto{\pgfqpoint{2.684859in}{2.327999in}}%
\pgfpathlineto{\pgfqpoint{2.678725in}{2.364664in}}%
\pgfpathlineto{\pgfqpoint{2.671356in}{2.395897in}}%
\pgfpathlineto{\pgfqpoint{2.662489in}{2.423981in}}%
\pgfpathlineto{\pgfqpoint{2.652361in}{2.448778in}}%
\pgfpathlineto{\pgfqpoint{2.641365in}{2.470245in}}%
\pgfpathlineto{\pgfqpoint{2.628643in}{2.490425in}}%
\pgfpathlineto{\pgfqpoint{2.614279in}{2.509106in}}%
\pgfpathlineto{\pgfqpoint{2.598443in}{2.526159in}}%
\pgfpathlineto{\pgfqpoint{2.579590in}{2.543005in}}%
\pgfpathlineto{\pgfqpoint{2.559532in}{2.557923in}}%
\pgfpathlineto{\pgfqpoint{2.536602in}{2.572183in}}%
\pgfpathlineto{\pgfqpoint{2.510850in}{2.585538in}}%
\pgfpathlineto{\pgfqpoint{2.482360in}{2.597837in}}%
\pgfpathlineto{\pgfqpoint{2.449134in}{2.609683in}}%
\pgfpathlineto{\pgfqpoint{2.411184in}{2.620696in}}%
\pgfpathlineto{\pgfqpoint{2.368552in}{2.630606in}}%
\pgfpathlineto{\pgfqpoint{2.321294in}{2.639221in}}%
\pgfpathlineto{\pgfqpoint{2.269467in}{2.646399in}}%
\pgfpathlineto{\pgfqpoint{2.210954in}{2.652193in}}%
\pgfpathlineto{\pgfqpoint{2.147967in}{2.656153in}}%
\pgfpathlineto{\pgfqpoint{2.080556in}{2.658135in}}%
\pgfpathlineto{\pgfqpoint{2.010948in}{2.657971in}}%
\pgfpathlineto{\pgfqpoint{1.939195in}{2.655572in}}%
\pgfpathlineto{\pgfqpoint{1.867527in}{2.650913in}}%
\pgfpathlineto{\pgfqpoint{1.798171in}{2.644140in}}%
\pgfpathlineto{\pgfqpoint{1.733341in}{2.635606in}}%
\pgfpathlineto{\pgfqpoint{1.673075in}{2.625521in}}%
\pgfpathlineto{\pgfqpoint{1.615274in}{2.613610in}}%
\pgfpathlineto{\pgfqpoint{1.562133in}{2.600402in}}%
\pgfpathlineto{\pgfqpoint{1.513681in}{2.586139in}}%
\pgfpathlineto{\pgfqpoint{1.467862in}{2.570344in}}%
\pgfpathlineto{\pgfqpoint{1.426794in}{2.553923in}}%
\pgfpathlineto{\pgfqpoint{1.388447in}{2.536289in}}%
\pgfpathlineto{\pgfqpoint{1.352878in}{2.517566in}}%
\pgfpathlineto{\pgfqpoint{1.320128in}{2.497922in}}%
\pgfpathlineto{\pgfqpoint{1.288379in}{2.476236in}}%
\pgfpathlineto{\pgfqpoint{1.259592in}{2.453861in}}%
\pgfpathlineto{\pgfqpoint{1.232050in}{2.429520in}}%
\pgfpathlineto{\pgfqpoint{1.207527in}{2.404898in}}%
\pgfpathlineto{\pgfqpoint{1.184409in}{2.378557in}}%
\pgfpathlineto{\pgfqpoint{1.162828in}{2.350561in}}%
\pgfpathlineto{\pgfqpoint{1.142891in}{2.321011in}}%
\pgfpathlineto{\pgfqpoint{1.124675in}{2.290041in}}%
\pgfpathlineto{\pgfqpoint{1.108225in}{2.257802in}}%
\pgfpathlineto{\pgfqpoint{1.092639in}{2.222199in}}%
\pgfpathlineto{\pgfqpoint{1.079059in}{2.185535in}}%
\pgfpathlineto{\pgfqpoint{1.067443in}{2.147998in}}%
\pgfpathlineto{\pgfqpoint{1.057187in}{2.107347in}}%
\pgfpathlineto{\pgfqpoint{1.049004in}{2.066086in}}%
\pgfpathlineto{\pgfqpoint{1.042513in}{2.021906in}}%
\pgfpathlineto{\pgfqpoint{1.038177in}{1.977382in}}%
\pgfpathlineto{\pgfqpoint{1.035866in}{1.930167in}}%
\pgfpathlineto{\pgfqpoint{1.035826in}{1.882878in}}%
\pgfpathlineto{\pgfqpoint{1.038031in}{1.835656in}}%
\pgfpathlineto{\pgfqpoint{1.042474in}{1.788641in}}%
\pgfpathlineto{\pgfqpoint{1.049176in}{1.741979in}}%
\pgfpathlineto{\pgfqpoint{1.057644in}{1.698239in}}%
\pgfpathlineto{\pgfqpoint{1.068221in}{1.655105in}}%
\pgfpathlineto{\pgfqpoint{1.080962in}{1.612745in}}%
\pgfpathlineto{\pgfqpoint{1.095031in}{1.573617in}}%
\pgfpathlineto{\pgfqpoint{1.111115in}{1.535520in}}%
\pgfpathlineto{\pgfqpoint{1.128118in}{1.500775in}}%
\pgfpathlineto{\pgfqpoint{1.146930in}{1.467274in}}%
\pgfpathlineto{\pgfqpoint{1.167531in}{1.435181in}}%
\pgfpathlineto{\pgfqpoint{1.189874in}{1.404652in}}%
\pgfpathlineto{\pgfqpoint{1.213884in}{1.375828in}}%
\pgfpathlineto{\pgfqpoint{1.237817in}{1.350457in}}%
\pgfpathlineto{\pgfqpoint{1.264748in}{1.325237in}}%
\pgfpathlineto{\pgfqpoint{1.292991in}{1.301972in}}%
\pgfpathlineto{\pgfqpoint{1.322398in}{1.280678in}}%
\pgfpathlineto{\pgfqpoint{1.352820in}{1.261340in}}%
\pgfpathlineto{\pgfqpoint{1.386095in}{1.242889in}}%
\pgfpathlineto{\pgfqpoint{1.420190in}{1.226516in}}%
\pgfpathlineto{\pgfqpoint{1.457024in}{1.211329in}}%
\pgfpathlineto{\pgfqpoint{1.496554in}{1.197536in}}%
\pgfpathlineto{\pgfqpoint{1.538719in}{1.185287in}}%
\pgfpathlineto{\pgfqpoint{1.583441in}{1.174641in}}%
\pgfpathlineto{\pgfqpoint{1.634929in}{1.164775in}}%
\pgfpathlineto{\pgfqpoint{1.706063in}{1.153745in}}%
\pgfpathlineto{\pgfqpoint{1.768492in}{1.143417in}}%
\pgfpathlineto{\pgfqpoint{1.796122in}{1.136567in}}%
\pgfpathlineto{\pgfqpoint{1.812683in}{1.130481in}}%
\pgfpathlineto{\pgfqpoint{1.824471in}{1.124102in}}%
\pgfpathlineto{\pgfqpoint{1.833209in}{1.116741in}}%
\pgfpathlineto{\pgfqpoint{1.838498in}{1.108890in}}%
\pgfpathlineto{\pgfqpoint{1.840588in}{1.101849in}}%
\pgfpathlineto{\pgfqpoint{1.840619in}{1.094412in}}%
\pgfpathlineto{\pgfqpoint{1.837931in}{1.084986in}}%
\pgfpathlineto{\pgfqpoint{1.833246in}{1.076615in}}%
\pgfpathlineto{\pgfqpoint{1.825819in}{1.067542in}}%
\pgfpathlineto{\pgfqpoint{1.813813in}{1.056850in}}%
\pgfpathlineto{\pgfqpoint{1.798819in}{1.046763in}}%
\pgfpathlineto{\pgfqpoint{1.781016in}{1.037462in}}%
\pgfpathlineto{\pgfqpoint{1.758447in}{1.028391in}}%
\pgfpathlineto{\pgfqpoint{1.733203in}{1.020815in}}%
\pgfpathlineto{\pgfqpoint{1.705410in}{1.014872in}}%
\pgfpathlineto{\pgfqpoint{1.675178in}{1.010714in}}%
\pgfpathlineto{\pgfqpoint{1.642610in}{1.008507in}}%
\pgfpathlineto{\pgfqpoint{1.607809in}{1.008432in}}%
\pgfpathlineto{\pgfqpoint{1.570886in}{1.010691in}}%
\pgfpathlineto{\pgfqpoint{1.534118in}{1.015181in}}%
\pgfpathlineto{\pgfqpoint{1.495454in}{1.022233in}}%
\pgfpathlineto{\pgfqpoint{1.457161in}{1.031563in}}%
\pgfpathlineto{\pgfqpoint{1.419337in}{1.043132in}}%
\pgfpathlineto{\pgfqpoint{1.382089in}{1.056929in}}%
\pgfpathlineto{\pgfqpoint{1.347544in}{1.072019in}}%
\pgfpathlineto{\pgfqpoint{1.313727in}{1.089133in}}%
\pgfpathlineto{\pgfqpoint{1.280762in}{1.108299in}}%
\pgfpathlineto{\pgfqpoint{1.248782in}{1.129536in}}%
\pgfpathlineto{\pgfqpoint{1.219708in}{1.151422in}}%
\pgfpathlineto{\pgfqpoint{1.191752in}{1.175138in}}%
\pgfpathlineto{\pgfqpoint{1.165031in}{1.200649in}}%
\pgfpathlineto{\pgfqpoint{1.139653in}{1.227898in}}%
\pgfpathlineto{\pgfqpoint{1.115714in}{1.256800in}}%
\pgfpathlineto{\pgfqpoint{1.093288in}{1.287251in}}%
\pgfpathlineto{\pgfqpoint{1.071178in}{1.321163in}}%
\pgfpathlineto{\pgfqpoint{1.050868in}{1.356520in}}%
\pgfpathlineto{\pgfqpoint{1.032365in}{1.393152in}}%
\pgfpathlineto{\pgfqpoint{1.014718in}{1.433142in}}%
\pgfpathlineto{\pgfqpoint{0.999024in}{1.474185in}}%
\pgfpathlineto{\pgfqpoint{0.984506in}{1.518461in}}%
\pgfpathlineto{\pgfqpoint{0.972010in}{1.563537in}}%
\pgfpathlineto{\pgfqpoint{0.960944in}{1.611678in}}%
\pgfpathlineto{\pgfqpoint{0.951530in}{1.662824in}}%
\pgfpathlineto{\pgfqpoint{0.944286in}{1.714431in}}%
\pgfpathlineto{\pgfqpoint{0.938950in}{1.768847in}}%
\pgfpathlineto{\pgfqpoint{0.935870in}{1.823491in}}%
\pgfpathlineto{\pgfqpoint{0.935034in}{1.878240in}}%
\pgfpathlineto{\pgfqpoint{0.936466in}{1.932973in}}%
\pgfpathlineto{\pgfqpoint{0.940005in}{1.985084in}}%
\pgfpathlineto{\pgfqpoint{0.945759in}{2.036935in}}%
\pgfpathlineto{\pgfqpoint{0.953410in}{2.085938in}}%
\pgfpathlineto{\pgfqpoint{0.962764in}{2.132000in}}%
\pgfpathlineto{\pgfqpoint{0.974287in}{2.177414in}}%
\pgfpathlineto{\pgfqpoint{0.987332in}{2.219653in}}%
\pgfpathlineto{\pgfqpoint{1.001667in}{2.258654in}}%
\pgfpathlineto{\pgfqpoint{1.018051in}{2.296583in}}%
\pgfpathlineto{\pgfqpoint{1.035401in}{2.331101in}}%
\pgfpathlineto{\pgfqpoint{1.054650in}{2.364275in}}%
\pgfpathlineto{\pgfqpoint{1.074406in}{2.393984in}}%
\pgfpathlineto{\pgfqpoint{1.095771in}{2.422197in}}%
\pgfpathlineto{\pgfqpoint{1.118662in}{2.448797in}}%
\pgfpathlineto{\pgfqpoint{1.142967in}{2.473701in}}%
\pgfpathlineto{\pgfqpoint{1.168550in}{2.496867in}}%
\pgfpathlineto{\pgfqpoint{1.197085in}{2.519662in}}%
\pgfpathlineto{\pgfqpoint{1.226727in}{2.540526in}}%
\pgfpathlineto{\pgfqpoint{1.259242in}{2.560673in}}%
\pgfpathlineto{\pgfqpoint{1.294612in}{2.579881in}}%
\pgfpathlineto{\pgfqpoint{1.332792in}{2.597982in}}%
\pgfpathlineto{\pgfqpoint{1.373719in}{2.614859in}}%
\pgfpathlineto{\pgfqpoint{1.417319in}{2.630445in}}%
\pgfpathlineto{\pgfqpoint{1.465632in}{2.645312in}}%
\pgfpathlineto{\pgfqpoint{1.518640in}{2.659204in}}%
\pgfpathlineto{\pgfqpoint{1.576309in}{2.671929in}}%
\pgfpathlineto{\pgfqpoint{1.638597in}{2.683344in}}%
\pgfpathlineto{\pgfqpoint{1.705462in}{2.693343in}}%
\pgfpathlineto{\pgfqpoint{1.779027in}{2.702064in}}%
\pgfpathlineto{\pgfqpoint{1.857097in}{2.709077in}}%
\pgfpathlineto{\pgfqpoint{1.939633in}{2.714280in}}%
\pgfpathlineto{\pgfqpoint{2.026598in}{2.717513in}}%
\pgfpathlineto{\pgfqpoint{2.113605in}{2.718523in}}%
\pgfpathlineto{\pgfqpoint{2.198435in}{2.717303in}}%
\pgfpathlineto{\pgfqpoint{2.278866in}{2.713929in}}%
\pgfpathlineto{\pgfqpoint{2.352678in}{2.708598in}}%
\pgfpathlineto{\pgfqpoint{2.417657in}{2.701709in}}%
\pgfpathlineto{\pgfqpoint{2.473770in}{2.693630in}}%
\pgfpathlineto{\pgfqpoint{2.523140in}{2.684368in}}%
\pgfpathlineto{\pgfqpoint{2.565726in}{2.674202in}}%
\pgfpathlineto{\pgfqpoint{2.601510in}{2.663544in}}%
\pgfpathlineto{\pgfqpoint{2.632577in}{2.652142in}}%
\pgfpathlineto{\pgfqpoint{2.658899in}{2.640331in}}%
\pgfpathlineto{\pgfqpoint{2.682438in}{2.627436in}}%
\pgfpathlineto{\pgfqpoint{2.703062in}{2.613571in}}%
\pgfpathlineto{\pgfqpoint{2.720674in}{2.598978in}}%
\pgfpathlineto{\pgfqpoint{2.735263in}{2.584053in}}%
\pgfpathlineto{\pgfqpoint{2.748320in}{2.567377in}}%
\pgfpathlineto{\pgfqpoint{2.759553in}{2.549046in}}%
\pgfpathlineto{\pgfqpoint{2.768788in}{2.529306in}}%
\pgfpathlineto{\pgfqpoint{2.776017in}{2.508498in}}%
\pgfpathlineto{\pgfqpoint{2.781884in}{2.484540in}}%
\pgfpathlineto{\pgfqpoint{2.786102in}{2.457597in}}%
\pgfpathlineto{\pgfqpoint{2.788720in}{2.425384in}}%
\pgfpathlineto{\pgfqpoint{2.789427in}{2.388061in}}%
\pgfpathlineto{\pgfqpoint{2.787962in}{2.340801in}}%
\pgfpathlineto{\pgfqpoint{2.783672in}{2.278768in}}%
\pgfpathlineto{\pgfqpoint{2.774289in}{2.179783in}}%
\pgfpathlineto{\pgfqpoint{2.743611in}{1.868119in}}%
\pgfpathlineto{\pgfqpoint{2.730112in}{1.702060in}}%
\pgfpathlineto{\pgfqpoint{2.717287in}{1.515949in}}%
\pgfpathlineto{\pgfqpoint{2.702602in}{1.267597in}}%
\pgfpathlineto{\pgfqpoint{2.684434in}{0.964630in}}%
\pgfpathlineto{\pgfqpoint{2.675374in}{0.850600in}}%
\pgfpathlineto{\pgfqpoint{2.667030in}{0.771523in}}%
\pgfpathlineto{\pgfqpoint{2.658752in}{0.712543in}}%
\pgfpathlineto{\pgfqpoint{2.650176in}{0.666284in}}%
\pgfpathlineto{\pgfqpoint{2.640820in}{0.627931in}}%
\pgfpathlineto{\pgfqpoint{2.631145in}{0.597534in}}%
\pgfpathlineto{\pgfqpoint{2.621004in}{0.572745in}}%
\pgfpathlineto{\pgfqpoint{2.609856in}{0.551383in}}%
\pgfpathlineto{\pgfqpoint{2.598042in}{0.533534in}}%
\pgfpathlineto{\pgfqpoint{2.584496in}{0.517378in}}%
\pgfpathlineto{\pgfqpoint{2.571109in}{0.504669in}}%
\pgfpathlineto{\pgfqpoint{2.554789in}{0.492313in}}%
\pgfpathlineto{\pgfqpoint{2.537457in}{0.481914in}}%
\pgfpathlineto{\pgfqpoint{2.517374in}{0.472367in}}%
\pgfpathlineto{\pgfqpoint{2.492542in}{0.463178in}}%
\pgfpathlineto{\pgfqpoint{2.462979in}{0.454833in}}%
\pgfpathlineto{\pgfqpoint{2.428766in}{0.447542in}}%
\pgfpathlineto{\pgfqpoint{2.385671in}{0.440735in}}%
\pgfpathlineto{\pgfqpoint{2.331557in}{0.434581in}}%
\pgfpathlineto{\pgfqpoint{2.262115in}{0.429077in}}%
\pgfpathlineto{\pgfqpoint{2.170851in}{0.424236in}}%
\pgfpathlineto{\pgfqpoint{2.049086in}{0.420134in}}%
\pgfpathlineto{\pgfqpoint{1.879436in}{0.416783in}}%
\pgfpathlineto{\pgfqpoint{1.640159in}{0.414418in}}%
\pgfpathlineto{\pgfqpoint{1.322562in}{0.413569in}}%
\pgfpathlineto{\pgfqpoint{1.020194in}{0.414850in}}%
\pgfpathlineto{\pgfqpoint{0.822256in}{0.417715in}}%
\pgfpathlineto{\pgfqpoint{0.704835in}{0.421430in}}%
\pgfpathlineto{\pgfqpoint{0.630976in}{0.425829in}}%
\pgfpathlineto{\pgfqpoint{0.583316in}{0.430734in}}%
\pgfpathlineto{\pgfqpoint{0.551033in}{0.436123in}}%
\pgfpathlineto{\pgfqpoint{0.527708in}{0.442189in}}%
\pgfpathlineto{\pgfqpoint{0.511250in}{0.448625in}}%
\pgfpathlineto{\pgfqpoint{0.499549in}{0.455216in}}%
\pgfpathlineto{\pgfqpoint{0.488916in}{0.463841in}}%
\pgfpathlineto{\pgfqpoint{0.481322in}{0.472730in}}%
\pgfpathlineto{\pgfqpoint{0.474078in}{0.485127in}}%
\pgfpathlineto{\pgfqpoint{0.468753in}{0.498748in}}%
\pgfpathlineto{\pgfqpoint{0.463870in}{0.517848in}}%
\pgfpathlineto{\pgfqpoint{0.459679in}{0.544796in}}%
\pgfpathlineto{\pgfqpoint{0.456386in}{0.581938in}}%
\pgfpathlineto{\pgfqpoint{0.453731in}{0.639106in}}%
\pgfpathlineto{\pgfqpoint{0.451681in}{0.736155in}}%
\pgfpathlineto{\pgfqpoint{0.450220in}{0.927815in}}%
\pgfpathlineto{\pgfqpoint{0.449345in}{1.403252in}}%
\pgfpathlineto{\pgfqpoint{0.449543in}{2.682703in}}%
\pgfpathlineto{\pgfqpoint{0.451011in}{2.856932in}}%
\pgfpathlineto{\pgfqpoint{0.452802in}{2.879219in}}%
\pgfpathlineto{\pgfqpoint{0.455188in}{2.886108in}}%
\pgfpathlineto{\pgfqpoint{0.458626in}{2.889028in}}%
\pgfpathlineto{\pgfqpoint{0.464996in}{2.890553in}}%
\pgfpathlineto{\pgfqpoint{0.482376in}{2.891423in}}%
\pgfpathlineto{\pgfqpoint{0.565038in}{2.891729in}}%
\pgfpathlineto{\pgfqpoint{2.733842in}{2.891760in}}%
\pgfpathlineto{\pgfqpoint{4.789510in}{2.890885in}}%
\pgfpathlineto{\pgfqpoint{4.793727in}{2.889730in}}%
\pgfpathlineto{\pgfqpoint{4.795481in}{2.888307in}}%
\pgfpathlineto{\pgfqpoint{4.797106in}{2.881145in}}%
\pgfpathlineto{\pgfqpoint{4.797997in}{2.858771in}}%
\pgfpathlineto{\pgfqpoint{4.798039in}{2.856283in}}%
\pgfpathlineto{\pgfqpoint{4.798039in}{2.856283in}}%
\pgfusepath{stroke}%
\end{pgfscope}%
\begin{pgfscope}%
\pgfpathrectangle{\pgfqpoint{0.448634in}{0.402556in}}{\pgfqpoint{4.350661in}{2.489204in}} %
\pgfusepath{clip}%
\pgfsetrectcap%
\pgfsetroundjoin%
\pgfsetlinewidth{1.003750pt}%
\definecolor{currentstroke}{rgb}{0.839216,0.152941,0.156863}%
\pgfsetstrokecolor{currentstroke}%
\pgfsetdash{}{0pt}%
\pgfpathmoveto{\pgfqpoint{3.428772in}{0.402610in}}%
\pgfpathlineto{\pgfqpoint{2.806632in}{0.403760in}}%
\pgfpathlineto{\pgfqpoint{2.769692in}{0.405578in}}%
\pgfpathlineto{\pgfqpoint{2.754632in}{0.408064in}}%
\pgfpathlineto{\pgfqpoint{2.746391in}{0.411198in}}%
\pgfpathlineto{\pgfqpoint{2.740943in}{0.415265in}}%
\pgfpathlineto{\pgfqpoint{2.736785in}{0.420985in}}%
\pgfpathlineto{\pgfqpoint{2.733281in}{0.430071in}}%
\pgfpathlineto{\pgfqpoint{2.730449in}{0.444637in}}%
\pgfpathlineto{\pgfqpoint{2.728238in}{0.469392in}}%
\pgfpathlineto{\pgfqpoint{2.726470in}{0.519131in}}%
\pgfpathlineto{\pgfqpoint{2.725711in}{0.613715in}}%
\pgfpathlineto{\pgfqpoint{2.726842in}{0.768039in}}%
\pgfpathlineto{\pgfqpoint{2.730557in}{0.962149in}}%
\pgfpathlineto{\pgfqpoint{2.736611in}{1.158671in}}%
\pgfpathlineto{\pgfqpoint{2.744092in}{1.327719in}}%
\pgfpathlineto{\pgfqpoint{2.753202in}{1.484190in}}%
\pgfpathlineto{\pgfqpoint{2.763257in}{1.620610in}}%
\pgfpathlineto{\pgfqpoint{2.776118in}{1.764216in}}%
\pgfpathlineto{\pgfqpoint{2.788914in}{1.877777in}}%
\pgfpathlineto{\pgfqpoint{2.805748in}{2.005741in}}%
\pgfpathlineto{\pgfqpoint{2.821176in}{2.101198in}}%
\pgfpathlineto{\pgfqpoint{2.838360in}{2.193719in}}%
\pgfpathlineto{\pgfqpoint{2.859135in}{2.292966in}}%
\pgfpathlineto{\pgfqpoint{2.887209in}{2.425960in}}%
\pgfpathlineto{\pgfqpoint{2.896992in}{2.479560in}}%
\pgfpathlineto{\pgfqpoint{2.901543in}{2.516524in}}%
\pgfpathlineto{\pgfqpoint{2.902849in}{2.543855in}}%
\pgfpathlineto{\pgfqpoint{2.901958in}{2.566223in}}%
\pgfpathlineto{\pgfqpoint{2.899152in}{2.585863in}}%
\pgfpathlineto{\pgfqpoint{2.894794in}{2.602546in}}%
\pgfpathlineto{\pgfqpoint{2.888484in}{2.618388in}}%
\pgfpathlineto{\pgfqpoint{2.880257in}{2.633033in}}%
\pgfpathlineto{\pgfqpoint{2.870348in}{2.646246in}}%
\pgfpathlineto{\pgfqpoint{2.857400in}{2.659531in}}%
\pgfpathlineto{\pgfqpoint{2.843189in}{2.671010in}}%
\pgfpathlineto{\pgfqpoint{2.824238in}{2.683209in}}%
\pgfpathlineto{\pgfqpoint{2.802413in}{2.694419in}}%
\pgfpathlineto{\pgfqpoint{2.775809in}{2.705369in}}%
\pgfpathlineto{\pgfqpoint{2.744461in}{2.715715in}}%
\pgfpathlineto{\pgfqpoint{2.708436in}{2.725252in}}%
\pgfpathlineto{\pgfqpoint{2.665655in}{2.734289in}}%
\pgfpathlineto{\pgfqpoint{2.613991in}{2.742869in}}%
\pgfpathlineto{\pgfqpoint{2.553459in}{2.750589in}}%
\pgfpathlineto{\pgfqpoint{2.481920in}{2.757365in}}%
\pgfpathlineto{\pgfqpoint{2.399398in}{2.762839in}}%
\pgfpathlineto{\pgfqpoint{2.310269in}{2.766482in}}%
\pgfpathlineto{\pgfqpoint{2.175416in}{2.768725in}}%
\pgfpathlineto{\pgfqpoint{2.066653in}{2.767942in}}%
\pgfpathlineto{\pgfqpoint{1.953571in}{2.764859in}}%
\pgfpathlineto{\pgfqpoint{1.851429in}{2.759759in}}%
\pgfpathlineto{\pgfqpoint{1.745051in}{2.752169in}}%
\pgfpathlineto{\pgfqpoint{1.658374in}{2.743454in}}%
\pgfpathlineto{\pgfqpoint{1.580552in}{2.733461in}}%
\pgfpathlineto{\pgfqpoint{1.490058in}{2.719338in}}%
\pgfpathlineto{\pgfqpoint{1.417231in}{2.704698in}}%
\pgfpathlineto{\pgfqpoint{1.361992in}{2.690818in}}%
\pgfpathlineto{\pgfqpoint{1.311460in}{2.675819in}}%
\pgfpathlineto{\pgfqpoint{1.265667in}{2.659924in}}%
\pgfpathlineto{\pgfqpoint{1.222575in}{2.642586in}}%
\pgfpathlineto{\pgfqpoint{1.184324in}{2.624682in}}%
\pgfpathlineto{\pgfqpoint{1.148892in}{2.605623in}}%
\pgfpathlineto{\pgfqpoint{1.116332in}{2.585573in}}%
\pgfpathlineto{\pgfqpoint{1.092327in}{2.568512in}}%
\pgfpathlineto{\pgfqpoint{1.079760in}{2.558686in}}%
\pgfpathlineto{\pgfqpoint{1.051544in}{2.535379in}}%
\pgfpathlineto{\pgfqpoint{1.026312in}{2.511712in}}%
\pgfpathlineto{\pgfqpoint{1.002399in}{2.486318in}}%
\pgfpathlineto{\pgfqpoint{0.979913in}{2.459269in}}%
\pgfpathlineto{\pgfqpoint{0.958934in}{2.430678in}}%
\pgfpathlineto{\pgfqpoint{0.938264in}{2.398644in}}%
\pgfpathlineto{\pgfqpoint{0.923047in}{2.371385in}}%
\pgfpathlineto{\pgfqpoint{0.904513in}{2.334774in}}%
\pgfpathlineto{\pgfqpoint{0.887854in}{2.297001in}}%
\pgfpathlineto{\pgfqpoint{0.872131in}{2.255972in}}%
\pgfpathlineto{\pgfqpoint{0.857508in}{2.211741in}}%
\pgfpathlineto{\pgfqpoint{0.844762in}{2.166757in}}%
\pgfpathlineto{\pgfqpoint{0.838624in}{2.140306in}}%
\pgfpathlineto{\pgfqpoint{0.826982in}{2.087194in}}%
\pgfpathlineto{\pgfqpoint{0.816322in}{2.028716in}}%
\pgfpathlineto{\pgfqpoint{0.810087in}{1.984495in}}%
\pgfpathlineto{\pgfqpoint{0.808026in}{1.967238in}}%
\pgfpathlineto{\pgfqpoint{0.800076in}{1.898141in}}%
\pgfpathlineto{\pgfqpoint{0.793713in}{1.823823in}}%
\pgfpathlineto{\pgfqpoint{0.788799in}{1.741875in}}%
\pgfpathlineto{\pgfqpoint{0.786199in}{1.677225in}}%
\pgfpathlineto{\pgfqpoint{0.776951in}{1.453481in}}%
\pgfpathlineto{\pgfqpoint{0.773280in}{1.418894in}}%
\pgfpathlineto{\pgfqpoint{0.768298in}{1.389582in}}%
\pgfpathlineto{\pgfqpoint{0.762752in}{1.368108in}}%
\pgfpathlineto{\pgfqpoint{0.756722in}{1.352123in}}%
\pgfpathlineto{\pgfqpoint{0.749752in}{1.339519in}}%
\pgfpathlineto{\pgfqpoint{0.742201in}{1.330599in}}%
\pgfpathlineto{\pgfqpoint{0.734854in}{1.325312in}}%
\pgfpathlineto{\pgfqpoint{0.726558in}{1.322419in}}%
\pgfpathlineto{\pgfqpoint{0.717884in}{1.322223in}}%
\pgfpathlineto{\pgfqpoint{0.709412in}{1.324411in}}%
\pgfpathlineto{\pgfqpoint{0.699548in}{1.329604in}}%
\pgfpathlineto{\pgfqpoint{0.688894in}{1.338203in}}%
\pgfpathlineto{\pgfqpoint{0.677907in}{1.350248in}}%
\pgfpathlineto{\pgfqpoint{0.666886in}{1.365647in}}%
\pgfpathlineto{\pgfqpoint{0.654913in}{1.386417in}}%
\pgfpathlineto{\pgfqpoint{0.642574in}{1.412730in}}%
\pgfpathlineto{\pgfqpoint{0.630328in}{1.444629in}}%
\pgfpathlineto{\pgfqpoint{0.618504in}{1.482081in}}%
\pgfpathlineto{\pgfqpoint{0.608613in}{1.520256in}}%
\pgfpathlineto{\pgfqpoint{0.590203in}{1.612445in}}%
\pgfpathlineto{\pgfqpoint{0.581848in}{1.668884in}}%
\pgfpathlineto{\pgfqpoint{0.573137in}{1.740376in}}%
\pgfpathlineto{\pgfqpoint{0.567062in}{1.807213in}}%
\pgfpathlineto{\pgfqpoint{0.560532in}{1.896510in}}%
\pgfpathlineto{\pgfqpoint{0.555526in}{1.995910in}}%
\pgfpathlineto{\pgfqpoint{0.552564in}{2.097908in}}%
\pgfpathlineto{\pgfqpoint{0.551526in}{2.204935in}}%
\pgfpathlineto{\pgfqpoint{0.552728in}{2.309470in}}%
\pgfpathlineto{\pgfqpoint{0.556011in}{2.403981in}}%
\pgfpathlineto{\pgfqpoint{0.560953in}{2.483430in}}%
\pgfpathlineto{\pgfqpoint{0.567303in}{2.550240in}}%
\pgfpathlineto{\pgfqpoint{0.574928in}{2.606817in}}%
\pgfpathlineto{\pgfqpoint{0.582988in}{2.650657in}}%
\pgfpathlineto{\pgfqpoint{0.592756in}{2.691452in}}%
\pgfpathlineto{\pgfqpoint{0.602650in}{2.721756in}}%
\pgfpathlineto{\pgfqpoint{0.612983in}{2.746441in}}%
\pgfpathlineto{\pgfqpoint{0.624292in}{2.767692in}}%
\pgfpathlineto{\pgfqpoint{0.636231in}{2.785433in}}%
\pgfpathlineto{\pgfqpoint{0.649892in}{2.801461in}}%
\pgfpathlineto{\pgfqpoint{0.663386in}{2.814020in}}%
\pgfpathlineto{\pgfqpoint{0.679842in}{2.826135in}}%
\pgfpathlineto{\pgfqpoint{0.697326in}{2.836197in}}%
\pgfpathlineto{\pgfqpoint{0.715574in}{2.844285in}}%
\pgfpathlineto{\pgfqpoint{0.738439in}{2.852335in}}%
\pgfpathlineto{\pgfqpoint{0.765983in}{2.859639in}}%
\pgfpathlineto{\pgfqpoint{0.800300in}{2.866256in}}%
\pgfpathlineto{\pgfqpoint{0.841340in}{2.871832in}}%
\pgfpathlineto{\pgfqpoint{0.895547in}{2.876803in}}%
\pgfpathlineto{\pgfqpoint{0.969413in}{2.881069in}}%
\pgfpathlineto{\pgfqpoint{1.071608in}{2.884501in}}%
\pgfpathlineto{\pgfqpoint{1.219512in}{2.887074in}}%
\pgfpathlineto{\pgfqpoint{1.471844in}{2.889091in}}%
\pgfpathlineto{\pgfqpoint{1.956941in}{2.890384in}}%
\pgfpathlineto{\pgfqpoint{3.096814in}{2.890781in}}%
\pgfpathlineto{\pgfqpoint{3.995224in}{2.889388in}}%
\pgfpathlineto{\pgfqpoint{4.275833in}{2.887011in}}%
\pgfpathlineto{\pgfqpoint{4.412847in}{2.883743in}}%
\pgfpathlineto{\pgfqpoint{4.491081in}{2.879810in}}%
\pgfpathlineto{\pgfqpoint{4.543127in}{2.875163in}}%
\pgfpathlineto{\pgfqpoint{4.579810in}{2.869841in}}%
\pgfpathlineto{\pgfqpoint{4.607580in}{2.863763in}}%
\pgfpathlineto{\pgfqpoint{4.630623in}{2.856424in}}%
\pgfpathlineto{\pgfqpoint{4.648833in}{2.848228in}}%
\pgfpathlineto{\pgfqpoint{4.664136in}{2.838773in}}%
\pgfpathlineto{\pgfqpoint{4.676470in}{2.828576in}}%
\pgfpathlineto{\pgfqpoint{4.687502in}{2.816585in}}%
\pgfpathlineto{\pgfqpoint{4.697051in}{2.803027in}}%
\pgfpathlineto{\pgfqpoint{4.706194in}{2.786098in}}%
\pgfpathlineto{\pgfqpoint{4.714508in}{2.765827in}}%
\pgfpathlineto{\pgfqpoint{4.722462in}{2.740013in}}%
\pgfpathlineto{\pgfqpoint{4.729577in}{2.708703in}}%
\pgfpathlineto{\pgfqpoint{4.736162in}{2.669601in}}%
\pgfpathlineto{\pgfqpoint{4.742419in}{2.617826in}}%
\pgfpathlineto{\pgfqpoint{4.747859in}{2.553410in}}%
\pgfpathlineto{\pgfqpoint{4.752661in}{2.468958in}}%
\pgfpathlineto{\pgfqpoint{4.756610in}{2.359528in}}%
\pgfpathlineto{\pgfqpoint{4.759416in}{2.217681in}}%
\pgfpathlineto{\pgfqpoint{4.760596in}{2.043444in}}%
\pgfpathlineto{\pgfqpoint{4.759662in}{1.851779in}}%
\pgfpathlineto{\pgfqpoint{4.756587in}{1.667613in}}%
\pgfpathlineto{\pgfqpoint{4.751596in}{1.503428in}}%
\pgfpathlineto{\pgfqpoint{4.745410in}{1.374185in}}%
\pgfpathlineto{\pgfqpoint{4.738113in}{1.267479in}}%
\pgfpathlineto{\pgfqpoint{4.729621in}{1.175896in}}%
\pgfpathlineto{\pgfqpoint{4.720762in}{1.104428in}}%
\pgfpathlineto{\pgfqpoint{4.711045in}{1.043204in}}%
\pgfpathlineto{\pgfqpoint{4.700364in}{0.989829in}}%
\pgfpathlineto{\pgfqpoint{4.689055in}{0.944345in}}%
\pgfpathlineto{\pgfqpoint{4.676881in}{0.904394in}}%
\pgfpathlineto{\pgfqpoint{4.676095in}{0.902073in}}%
\pgfpathlineto{\pgfqpoint{4.676095in}{0.902073in}}%
\pgfusepath{stroke}%
\end{pgfscope}%
\begin{pgfscope}%
\pgfpathrectangle{\pgfqpoint{0.448634in}{0.402556in}}{\pgfqpoint{4.350661in}{2.489204in}} %
\pgfusepath{clip}%
\pgfsetrectcap%
\pgfsetroundjoin%
\pgfsetlinewidth{1.003750pt}%
\definecolor{currentstroke}{rgb}{0.839216,0.152941,0.156863}%
\pgfsetstrokecolor{currentstroke}%
\pgfsetdash{}{0pt}%
\pgfpathmoveto{\pgfqpoint{2.795520in}{1.982745in}}%
\pgfpathlineto{\pgfqpoint{2.781780in}{1.874357in}}%
\pgfpathlineto{\pgfqpoint{2.769351in}{1.758234in}}%
\pgfpathlineto{\pgfqpoint{2.758095in}{1.631942in}}%
\pgfpathlineto{\pgfqpoint{2.747786in}{1.490551in}}%
\pgfpathlineto{\pgfqpoint{2.738644in}{1.334082in}}%
\pgfpathlineto{\pgfqpoint{2.730580in}{1.157591in}}%
\pgfpathlineto{\pgfqpoint{2.723334in}{0.948663in}}%
\pgfpathlineto{\pgfqpoint{2.709783in}{0.530788in}}%
\pgfpathlineto{\pgfqpoint{2.705868in}{0.488716in}}%
\pgfpathlineto{\pgfqpoint{2.701769in}{0.464281in}}%
\pgfpathlineto{\pgfqpoint{2.697021in}{0.447744in}}%
\pgfpathlineto{\pgfqpoint{2.691859in}{0.436812in}}%
\pgfpathlineto{\pgfqpoint{2.686245in}{0.429229in}}%
\pgfpathlineto{\pgfqpoint{2.679348in}{0.423188in}}%
\pgfpathlineto{\pgfqpoint{2.669540in}{0.417856in}}%
\pgfpathlineto{\pgfqpoint{2.656987in}{0.413810in}}%
\pgfpathlineto{\pgfqpoint{2.637654in}{0.410337in}}%
\pgfpathlineto{\pgfqpoint{2.607297in}{0.407617in}}%
\pgfpathlineto{\pgfqpoint{2.555121in}{0.405574in}}%
\pgfpathlineto{\pgfqpoint{2.450714in}{0.404139in}}%
\pgfpathlineto{\pgfqpoint{2.176624in}{0.403275in}}%
\pgfpathlineto{\pgfqpoint{1.130290in}{0.402953in}}%
\pgfpathlineto{\pgfqpoint{0.516849in}{0.404175in}}%
\pgfpathlineto{\pgfqpoint{0.466848in}{0.405970in}}%
\pgfpathlineto{\pgfqpoint{0.456130in}{0.407931in}}%
\pgfpathlineto{\pgfqpoint{0.452340in}{0.410303in}}%
\pgfpathlineto{\pgfqpoint{0.450346in}{0.414662in}}%
\pgfpathlineto{\pgfqpoint{0.449266in}{0.424524in}}%
\pgfpathlineto{\pgfqpoint{0.448771in}{0.464344in}}%
\pgfpathlineto{\pgfqpoint{0.448640in}{0.850171in}}%
\pgfpathlineto{\pgfqpoint{0.448653in}{2.891318in}}%
\pgfpathlineto{\pgfqpoint{0.448653in}{2.891318in}}%
\pgfusepath{stroke}%
\end{pgfscope}%
\begin{pgfscope}%
\pgfpathrectangle{\pgfqpoint{0.448634in}{0.402556in}}{\pgfqpoint{4.350661in}{2.489204in}} %
\pgfusepath{clip}%
\pgfsetrectcap%
\pgfsetroundjoin%
\pgfsetlinewidth{1.003750pt}%
\definecolor{currentstroke}{rgb}{0.839216,0.152941,0.156863}%
\pgfsetstrokecolor{currentstroke}%
\pgfsetdash{}{0pt}%
\pgfpathmoveto{\pgfqpoint{3.428190in}{0.402586in}}%
\pgfpathlineto{\pgfqpoint{2.782122in}{0.403702in}}%
\pgfpathlineto{\pgfqpoint{2.753907in}{0.405674in}}%
\pgfpathlineto{\pgfqpoint{2.743329in}{0.408444in}}%
\pgfpathlineto{\pgfqpoint{2.737718in}{0.412189in}}%
\pgfpathlineto{\pgfqpoint{2.733668in}{0.417995in}}%
\pgfpathlineto{\pgfqpoint{2.730649in}{0.427308in}}%
\pgfpathlineto{\pgfqpoint{2.728388in}{0.442005in}}%
\pgfpathlineto{\pgfqpoint{2.726544in}{0.471795in}}%
\pgfpathlineto{\pgfqpoint{2.725216in}{0.534004in}}%
\pgfpathlineto{\pgfqpoint{2.725169in}{0.655973in}}%
\pgfpathlineto{\pgfqpoint{2.727377in}{0.832687in}}%
\pgfpathlineto{\pgfqpoint{2.732259in}{1.041703in}}%
\pgfpathlineto{\pgfqpoint{2.738851in}{1.223257in}}%
\pgfpathlineto{\pgfqpoint{2.747078in}{1.389766in}}%
\pgfpathlineto{\pgfqpoint{2.756608in}{1.538718in}}%
\pgfpathlineto{\pgfqpoint{2.768955in}{1.694887in}}%
\pgfpathlineto{\pgfqpoint{2.781228in}{1.816045in}}%
\pgfpathlineto{\pgfqpoint{2.794401in}{1.924525in}}%
\pgfpathlineto{\pgfqpoint{2.812737in}{2.054723in}}%
\pgfpathlineto{\pgfqpoint{2.828774in}{2.147513in}}%
\pgfpathlineto{\pgfqpoint{2.847382in}{2.242225in}}%
\pgfpathlineto{\pgfqpoint{2.895818in}{2.479700in}}%
\pgfpathlineto{\pgfqpoint{2.900204in}{2.516690in}}%
\pgfpathlineto{\pgfqpoint{2.901346in}{2.544030in}}%
\pgfpathlineto{\pgfqpoint{2.900292in}{2.566389in}}%
\pgfpathlineto{\pgfqpoint{2.897335in}{2.586000in}}%
\pgfpathlineto{\pgfqpoint{2.892836in}{2.602634in}}%
\pgfpathlineto{\pgfqpoint{2.886394in}{2.618406in}}%
\pgfpathlineto{\pgfqpoint{2.878058in}{2.632970in}}%
\pgfpathlineto{\pgfqpoint{2.868065in}{2.646101in}}%
\pgfpathlineto{\pgfqpoint{2.855050in}{2.659301in}}%
\pgfpathlineto{\pgfqpoint{2.840801in}{2.670717in}}%
\pgfpathlineto{\pgfqpoint{2.821822in}{2.682861in}}%
\pgfpathlineto{\pgfqpoint{2.799980in}{2.694026in}}%
\pgfpathlineto{\pgfqpoint{2.773366in}{2.704944in}}%
\pgfpathlineto{\pgfqpoint{2.742012in}{2.715266in}}%
\pgfpathlineto{\pgfqpoint{2.705983in}{2.724786in}}%
\pgfpathlineto{\pgfqpoint{2.663200in}{2.733811in}}%
\pgfpathlineto{\pgfqpoint{2.611535in}{2.742379in}}%
\pgfpathlineto{\pgfqpoint{2.551002in}{2.750090in}}%
\pgfpathlineto{\pgfqpoint{2.481632in}{2.756682in}}%
\pgfpathlineto{\pgfqpoint{2.399112in}{2.762200in}}%
\pgfpathlineto{\pgfqpoint{2.309985in}{2.765886in}}%
\pgfpathlineto{\pgfqpoint{2.188184in}{2.768097in}}%
\pgfpathlineto{\pgfqpoint{2.081595in}{2.767619in}}%
\pgfpathlineto{\pgfqpoint{1.968506in}{2.764840in}}%
\pgfpathlineto{\pgfqpoint{1.864180in}{2.759918in}}%
\pgfpathlineto{\pgfqpoint{1.757786in}{2.752593in}}%
\pgfpathlineto{\pgfqpoint{1.671087in}{2.744171in}}%
\pgfpathlineto{\pgfqpoint{1.591076in}{2.734193in}}%
\pgfpathlineto{\pgfqpoint{1.502689in}{2.720717in}}%
\pgfpathlineto{\pgfqpoint{1.427655in}{2.706083in}}%
\pgfpathlineto{\pgfqpoint{1.372350in}{2.692544in}}%
\pgfpathlineto{\pgfqpoint{1.321734in}{2.677921in}}%
\pgfpathlineto{\pgfqpoint{1.273765in}{2.661664in}}%
\pgfpathlineto{\pgfqpoint{1.230567in}{2.644672in}}%
\pgfpathlineto{\pgfqpoint{1.192197in}{2.627106in}}%
\pgfpathlineto{\pgfqpoint{1.156620in}{2.608403in}}%
\pgfpathlineto{\pgfqpoint{1.123890in}{2.588716in}}%
\pgfpathlineto{\pgfqpoint{1.095883in}{2.569568in}}%
\pgfpathlineto{\pgfqpoint{1.063936in}{2.543701in}}%
\pgfpathlineto{\pgfqpoint{1.038217in}{2.520732in}}%
\pgfpathlineto{\pgfqpoint{1.013766in}{2.496016in}}%
\pgfpathlineto{\pgfqpoint{0.990704in}{2.469610in}}%
\pgfpathlineto{\pgfqpoint{0.969124in}{2.441612in}}%
\pgfpathlineto{\pgfqpoint{0.949082in}{2.412154in}}%
\pgfpathlineto{\pgfqpoint{0.930604in}{2.381387in}}%
\pgfpathlineto{\pgfqpoint{0.906555in}{2.334052in}}%
\pgfpathlineto{\pgfqpoint{0.889925in}{2.296262in}}%
\pgfpathlineto{\pgfqpoint{0.874241in}{2.255213in}}%
\pgfpathlineto{\pgfqpoint{0.859667in}{2.210961in}}%
\pgfpathlineto{\pgfqpoint{0.846985in}{2.165954in}}%
\pgfpathlineto{\pgfqpoint{0.839633in}{2.134715in}}%
\pgfpathlineto{\pgfqpoint{0.828238in}{2.081532in}}%
\pgfpathlineto{\pgfqpoint{0.817866in}{2.022986in}}%
\pgfpathlineto{\pgfqpoint{0.810784in}{1.971352in}}%
\pgfpathlineto{\pgfqpoint{0.802845in}{1.902253in}}%
\pgfpathlineto{\pgfqpoint{0.796554in}{1.827927in}}%
\pgfpathlineto{\pgfqpoint{0.791696in}{1.743480in}}%
\pgfpathlineto{\pgfqpoint{0.787773in}{1.621595in}}%
\pgfpathlineto{\pgfqpoint{0.785407in}{1.522064in}}%
\pgfpathlineto{\pgfqpoint{0.785407in}{1.522064in}}%
\pgfusepath{stroke}%
\end{pgfscope}%
\begin{pgfscope}%
\pgfpathrectangle{\pgfqpoint{0.448634in}{0.402556in}}{\pgfqpoint{4.350661in}{2.489204in}} %
\pgfusepath{clip}%
\pgfsetrectcap%
\pgfsetroundjoin%
\pgfsetlinewidth{1.003750pt}%
\definecolor{currentstroke}{rgb}{0.839216,0.152941,0.156863}%
\pgfsetstrokecolor{currentstroke}%
\pgfsetdash{}{0pt}%
\pgfpathmoveto{\pgfqpoint{2.028735in}{0.425754in}}%
\pgfpathlineto{\pgfqpoint{1.878677in}{0.421879in}}%
\pgfpathlineto{\pgfqpoint{1.676387in}{0.418997in}}%
\pgfpathlineto{\pgfqpoint{1.413176in}{0.417558in}}%
\pgfpathlineto{\pgfqpoint{1.134735in}{0.418204in}}%
\pgfpathlineto{\pgfqpoint{0.921565in}{0.420769in}}%
\pgfpathlineto{\pgfqpoint{0.782384in}{0.424523in}}%
\pgfpathlineto{\pgfqpoint{0.693283in}{0.428974in}}%
\pgfpathlineto{\pgfqpoint{0.632541in}{0.434091in}}%
\pgfpathlineto{\pgfqpoint{0.591492in}{0.439564in}}%
\pgfpathlineto{\pgfqpoint{0.561503in}{0.445595in}}%
\pgfpathlineto{\pgfqpoint{0.538349in}{0.452466in}}%
\pgfpathlineto{\pgfqpoint{0.522042in}{0.459394in}}%
\pgfpathlineto{\pgfqpoint{0.508540in}{0.467420in}}%
\pgfpathlineto{\pgfqpoint{0.497973in}{0.476161in}}%
\pgfpathlineto{\pgfqpoint{0.488790in}{0.486750in}}%
\pgfpathlineto{\pgfqpoint{0.481284in}{0.498948in}}%
\pgfpathlineto{\pgfqpoint{0.474590in}{0.514580in}}%
\pgfpathlineto{\pgfqpoint{0.469106in}{0.533467in}}%
\pgfpathlineto{\pgfqpoint{0.464439in}{0.557771in}}%
\pgfpathlineto{\pgfqpoint{0.460297in}{0.592289in}}%
\pgfpathlineto{\pgfqpoint{0.456856in}{0.641912in}}%
\pgfpathlineto{\pgfqpoint{0.454122in}{0.716520in}}%
\pgfpathlineto{\pgfqpoint{0.451978in}{0.843444in}}%
\pgfpathlineto{\pgfqpoint{0.450459in}{1.087380in}}%
\pgfpathlineto{\pgfqpoint{0.449596in}{1.657406in}}%
\pgfpathlineto{\pgfqpoint{0.450150in}{2.687936in}}%
\pgfpathlineto{\pgfqpoint{0.451781in}{2.839761in}}%
\pgfpathlineto{\pgfqpoint{0.453975in}{2.872003in}}%
\pgfpathlineto{\pgfqpoint{0.456339in}{2.881553in}}%
\pgfpathlineto{\pgfqpoint{0.458888in}{2.885549in}}%
\pgfpathlineto{\pgfqpoint{0.462554in}{2.888171in}}%
\pgfpathlineto{\pgfqpoint{0.471046in}{2.890205in}}%
\pgfpathlineto{\pgfqpoint{0.490597in}{2.891263in}}%
\pgfpathlineto{\pgfqpoint{0.564556in}{2.891692in}}%
\pgfpathlineto{\pgfqpoint{1.569559in}{2.891759in}}%
\pgfpathlineto{\pgfqpoint{4.784679in}{2.890785in}}%
\pgfpathlineto{\pgfqpoint{4.791005in}{2.889098in}}%
\pgfpathlineto{\pgfqpoint{4.793910in}{2.885555in}}%
\pgfpathlineto{\pgfqpoint{4.795579in}{2.878366in}}%
\pgfpathlineto{\pgfqpoint{4.796850in}{2.858513in}}%
\pgfpathlineto{\pgfqpoint{4.796850in}{2.858513in}}%
\pgfusepath{stroke}%
\end{pgfscope}%
\begin{pgfscope}%
\pgfpathrectangle{\pgfqpoint{0.448634in}{0.402556in}}{\pgfqpoint{4.350661in}{2.489204in}} %
\pgfusepath{clip}%
\pgfsetrectcap%
\pgfsetroundjoin%
\pgfsetlinewidth{1.003750pt}%
\definecolor{currentstroke}{rgb}{0.580392,0.403922,0.741176}%
\pgfsetstrokecolor{currentstroke}%
\pgfsetdash{}{0pt}%
\pgfpathmoveto{\pgfqpoint{0.448634in}{2.896245in}}%
\pgfpathlineto{\pgfqpoint{0.448593in}{0.407043in}}%
\pgfpathlineto{\pgfqpoint{0.448593in}{0.407043in}}%
\pgfusepath{stroke}%
\end{pgfscope}%
\begin{pgfscope}%
\pgfpathrectangle{\pgfqpoint{0.448634in}{0.402556in}}{\pgfqpoint{4.350661in}{2.489204in}} %
\pgfusepath{clip}%
\pgfsetrectcap%
\pgfsetroundjoin%
\pgfsetlinewidth{1.003750pt}%
\definecolor{currentstroke}{rgb}{0.580392,0.403922,0.741176}%
\pgfsetstrokecolor{currentstroke}%
\pgfsetdash{}{0pt}%
\pgfpathmoveto{\pgfqpoint{0.576852in}{1.760819in}}%
\pgfpathlineto{\pgfqpoint{0.569393in}{1.840012in}}%
\pgfpathlineto{\pgfqpoint{0.563208in}{1.929341in}}%
\pgfpathlineto{\pgfqpoint{0.558592in}{2.028766in}}%
\pgfpathlineto{\pgfqpoint{0.555985in}{2.133267in}}%
\pgfpathlineto{\pgfqpoint{0.555565in}{2.237810in}}%
\pgfpathlineto{\pgfqpoint{0.557371in}{2.337354in}}%
\pgfpathlineto{\pgfqpoint{0.561096in}{2.424369in}}%
\pgfpathlineto{\pgfqpoint{0.566403in}{2.498793in}}%
\pgfpathlineto{\pgfqpoint{0.572908in}{2.560572in}}%
\pgfpathlineto{\pgfqpoint{0.580458in}{2.612121in}}%
\pgfpathlineto{\pgfqpoint{0.589086in}{2.655818in}}%
\pgfpathlineto{\pgfqpoint{0.598406in}{2.691592in}}%
\pgfpathlineto{\pgfqpoint{0.608613in}{2.721759in}}%
\pgfpathlineto{\pgfqpoint{0.619241in}{2.746280in}}%
\pgfpathlineto{\pgfqpoint{0.630817in}{2.767341in}}%
\pgfpathlineto{\pgfqpoint{0.642976in}{2.784886in}}%
\pgfpathlineto{\pgfqpoint{0.656814in}{2.800714in}}%
\pgfpathlineto{\pgfqpoint{0.672198in}{2.814550in}}%
\pgfpathlineto{\pgfqpoint{0.688854in}{2.826302in}}%
\pgfpathlineto{\pgfqpoint{0.706462in}{2.836077in}}%
\pgfpathlineto{\pgfqpoint{0.726805in}{2.844876in}}%
\pgfpathlineto{\pgfqpoint{0.751867in}{2.853203in}}%
\pgfpathlineto{\pgfqpoint{0.781633in}{2.860548in}}%
\pgfpathlineto{\pgfqpoint{0.818169in}{2.867054in}}%
\pgfpathlineto{\pgfqpoint{0.863582in}{2.872685in}}%
\pgfpathlineto{\pgfqpoint{0.922162in}{2.877518in}}%
\pgfpathlineto{\pgfqpoint{1.000392in}{2.881567in}}%
\pgfpathlineto{\pgfqpoint{1.111295in}{2.884881in}}%
\pgfpathlineto{\pgfqpoint{1.274430in}{2.887367in}}%
\pgfpathlineto{\pgfqpoint{1.552867in}{2.889263in}}%
\pgfpathlineto{\pgfqpoint{2.107575in}{2.890457in}}%
\pgfpathlineto{\pgfqpoint{3.343162in}{2.890573in}}%
\pgfpathlineto{\pgfqpoint{4.043617in}{2.888941in}}%
\pgfpathlineto{\pgfqpoint{4.289418in}{2.886404in}}%
\pgfpathlineto{\pgfqpoint{4.413376in}{2.883093in}}%
\pgfpathlineto{\pgfqpoint{4.489426in}{2.878997in}}%
\pgfpathlineto{\pgfqpoint{4.541453in}{2.874081in}}%
\pgfpathlineto{\pgfqpoint{4.578102in}{2.868470in}}%
\pgfpathlineto{\pgfqpoint{4.605820in}{2.862092in}}%
\pgfpathlineto{\pgfqpoint{4.626727in}{2.855245in}}%
\pgfpathlineto{\pgfqpoint{4.644927in}{2.847018in}}%
\pgfpathlineto{\pgfqpoint{4.660242in}{2.837589in}}%
\pgfpathlineto{\pgfqpoint{4.672625in}{2.827468in}}%
\pgfpathlineto{\pgfqpoint{4.683752in}{2.815592in}}%
\pgfpathlineto{\pgfqpoint{4.693407in}{2.802134in}}%
\pgfpathlineto{\pgfqpoint{4.702741in}{2.785342in}}%
\pgfpathlineto{\pgfqpoint{4.711278in}{2.765193in}}%
\pgfpathlineto{\pgfqpoint{4.719483in}{2.739483in}}%
\pgfpathlineto{\pgfqpoint{4.726294in}{2.710656in}}%
\pgfpathlineto{\pgfqpoint{4.733260in}{2.671642in}}%
\pgfpathlineto{\pgfqpoint{4.739604in}{2.622395in}}%
\pgfpathlineto{\pgfqpoint{4.745236in}{2.560503in}}%
\pgfpathlineto{\pgfqpoint{4.750164in}{2.481051in}}%
\pgfpathlineto{\pgfqpoint{4.754367in}{2.376617in}}%
\pgfpathlineto{\pgfqpoint{4.757443in}{2.242248in}}%
\pgfpathlineto{\pgfqpoint{4.758977in}{2.075482in}}%
\pgfpathlineto{\pgfqpoint{4.758447in}{1.888794in}}%
\pgfpathlineto{\pgfqpoint{4.755756in}{1.707109in}}%
\pgfpathlineto{\pgfqpoint{4.750925in}{1.532956in}}%
\pgfpathlineto{\pgfqpoint{4.744786in}{1.398725in}}%
\pgfpathlineto{\pgfqpoint{4.737575in}{1.289514in}}%
\pgfpathlineto{\pgfqpoint{4.728714in}{1.190468in}}%
\pgfpathlineto{\pgfqpoint{4.719653in}{1.116520in}}%
\pgfpathlineto{\pgfqpoint{4.710036in}{1.055275in}}%
\pgfpathlineto{\pgfqpoint{4.699504in}{1.001860in}}%
\pgfpathlineto{\pgfqpoint{4.689041in}{0.958689in}}%
\pgfpathlineto{\pgfqpoint{4.677220in}{0.918599in}}%
\pgfpathlineto{\pgfqpoint{4.664034in}{0.881748in}}%
\pgfpathlineto{\pgfqpoint{4.650584in}{0.850490in}}%
\pgfpathlineto{\pgfqpoint{4.636303in}{0.822568in}}%
\pgfpathlineto{\pgfqpoint{4.620207in}{0.795973in}}%
\pgfpathlineto{\pgfqpoint{4.603640in}{0.772900in}}%
\pgfpathlineto{\pgfqpoint{4.585488in}{0.751445in}}%
\pgfpathlineto{\pgfqpoint{4.565874in}{0.731748in}}%
\pgfpathlineto{\pgfqpoint{4.544964in}{0.713878in}}%
\pgfpathlineto{\pgfqpoint{4.522958in}{0.697823in}}%
\pgfpathlineto{\pgfqpoint{4.496157in}{0.681289in}}%
\pgfpathlineto{\pgfqpoint{4.470397in}{0.667952in}}%
\pgfpathlineto{\pgfqpoint{4.439961in}{0.654509in}}%
\pgfpathlineto{\pgfqpoint{4.406841in}{0.642281in}}%
\pgfpathlineto{\pgfqpoint{4.369009in}{0.630748in}}%
\pgfpathlineto{\pgfqpoint{4.326489in}{0.620225in}}%
\pgfpathlineto{\pgfqpoint{4.279327in}{0.610948in}}%
\pgfpathlineto{\pgfqpoint{4.227576in}{0.603084in}}%
\pgfpathlineto{\pgfqpoint{4.173450in}{0.597062in}}%
\pgfpathlineto{\pgfqpoint{4.110511in}{0.592202in}}%
\pgfpathlineto{\pgfqpoint{4.047471in}{0.589536in}}%
\pgfpathlineto{\pgfqpoint{3.977867in}{0.588623in}}%
\pgfpathlineto{\pgfqpoint{3.906093in}{0.589933in}}%
\pgfpathlineto{\pgfqpoint{3.834377in}{0.593496in}}%
\pgfpathlineto{\pgfqpoint{3.767120in}{0.599067in}}%
\pgfpathlineto{\pgfqpoint{3.704364in}{0.606392in}}%
\pgfpathlineto{\pgfqpoint{3.678516in}{0.610510in}}%
\pgfpathlineto{\pgfqpoint{3.620438in}{0.620500in}}%
\pgfpathlineto{\pgfqpoint{3.586319in}{0.628207in}}%
\pgfpathlineto{\pgfqpoint{3.495241in}{0.652428in}}%
\pgfpathlineto{\pgfqpoint{3.451528in}{0.667583in}}%
\pgfpathlineto{\pgfqpoint{3.408538in}{0.685220in}}%
\pgfpathlineto{\pgfqpoint{3.374594in}{0.702001in}}%
\pgfpathlineto{\pgfqpoint{3.345407in}{0.718682in}}%
\pgfpathlineto{\pgfqpoint{3.315236in}{0.738520in}}%
\pgfpathlineto{\pgfqpoint{3.288127in}{0.759290in}}%
\pgfpathlineto{\pgfqpoint{3.264004in}{0.780551in}}%
\pgfpathlineto{\pgfqpoint{3.241208in}{0.803648in}}%
\pgfpathlineto{\pgfqpoint{3.219894in}{0.828530in}}%
\pgfpathlineto{\pgfqpoint{3.200189in}{0.855091in}}%
\pgfpathlineto{\pgfqpoint{3.182177in}{0.883182in}}%
\pgfpathlineto{\pgfqpoint{3.165906in}{0.912633in}}%
\pgfpathlineto{\pgfqpoint{3.150351in}{0.945448in}}%
\pgfpathlineto{\pgfqpoint{3.136682in}{0.979345in}}%
\pgfpathlineto{\pgfqpoint{3.124073in}{1.016460in}}%
\pgfpathlineto{\pgfqpoint{3.112834in}{1.056769in}}%
\pgfpathlineto{\pgfqpoint{3.103046in}{1.100146in}}%
\pgfpathlineto{\pgfqpoint{3.095343in}{1.144071in}}%
\pgfpathlineto{\pgfqpoint{3.089208in}{1.190837in}}%
\pgfpathlineto{\pgfqpoint{3.084595in}{1.242838in}}%
\pgfpathlineto{\pgfqpoint{3.082137in}{1.295031in}}%
\pgfpathlineto{\pgfqpoint{3.081687in}{1.349787in}}%
\pgfpathlineto{\pgfqpoint{3.083451in}{1.406998in}}%
\pgfpathlineto{\pgfqpoint{3.087181in}{1.461589in}}%
\pgfpathlineto{\pgfqpoint{3.093485in}{1.520887in}}%
\pgfpathlineto{\pgfqpoint{3.101823in}{1.577334in}}%
\pgfpathlineto{\pgfqpoint{3.111930in}{1.630856in}}%
\pgfpathlineto{\pgfqpoint{3.124690in}{1.686208in}}%
\pgfpathlineto{\pgfqpoint{3.139178in}{1.738395in}}%
\pgfpathlineto{\pgfqpoint{3.155145in}{1.787366in}}%
\pgfpathlineto{\pgfqpoint{3.172353in}{1.833084in}}%
\pgfpathlineto{\pgfqpoint{3.191618in}{1.877716in}}%
\pgfpathlineto{\pgfqpoint{3.214026in}{1.923261in}}%
\pgfpathlineto{\pgfqpoint{3.236214in}{1.963157in}}%
\pgfpathlineto{\pgfqpoint{3.260178in}{2.001684in}}%
\pgfpathlineto{\pgfqpoint{3.285814in}{2.038776in}}%
\pgfpathlineto{\pgfqpoint{3.314415in}{2.076285in}}%
\pgfpathlineto{\pgfqpoint{3.348944in}{2.117711in}}%
\pgfpathlineto{\pgfqpoint{3.417133in}{2.198022in}}%
\pgfpathlineto{\pgfqpoint{3.426053in}{2.212128in}}%
\pgfpathlineto{\pgfqpoint{3.430798in}{2.223297in}}%
\pgfpathlineto{\pgfqpoint{3.432034in}{2.230603in}}%
\pgfpathlineto{\pgfqpoint{3.430773in}{2.237856in}}%
\pgfpathlineto{\pgfqpoint{3.426621in}{2.243526in}}%
\pgfpathlineto{\pgfqpoint{3.420908in}{2.247084in}}%
\pgfpathlineto{\pgfqpoint{3.412501in}{2.249583in}}%
\pgfpathlineto{\pgfqpoint{3.399499in}{2.250689in}}%
\pgfpathlineto{\pgfqpoint{3.384305in}{2.249671in}}%
\pgfpathlineto{\pgfqpoint{3.364985in}{2.246098in}}%
\pgfpathlineto{\pgfqpoint{3.341804in}{2.239342in}}%
\pgfpathlineto{\pgfqpoint{3.317109in}{2.229682in}}%
\pgfpathlineto{\pgfqpoint{3.291104in}{2.216986in}}%
\pgfpathlineto{\pgfqpoint{3.265928in}{2.202261in}}%
\pgfpathlineto{\pgfqpoint{3.239805in}{2.184361in}}%
\pgfpathlineto{\pgfqpoint{3.214775in}{2.164519in}}%
\pgfpathlineto{\pgfqpoint{3.190900in}{2.142893in}}%
\pgfpathlineto{\pgfqpoint{3.166657in}{2.117912in}}%
\pgfpathlineto{\pgfqpoint{3.143835in}{2.091233in}}%
\pgfpathlineto{\pgfqpoint{3.121079in}{2.061107in}}%
\pgfpathlineto{\pgfqpoint{3.099952in}{2.029463in}}%
\pgfpathlineto{\pgfqpoint{3.079251in}{1.994406in}}%
\pgfpathlineto{\pgfqpoint{3.059218in}{1.955915in}}%
\pgfpathlineto{\pgfqpoint{3.040058in}{1.914015in}}%
\pgfpathlineto{\pgfqpoint{3.022809in}{1.871041in}}%
\pgfpathlineto{\pgfqpoint{3.005790in}{1.822536in}}%
\pgfpathlineto{\pgfqpoint{2.990067in}{1.770819in}}%
\pgfpathlineto{\pgfqpoint{2.975708in}{1.715979in}}%
\pgfpathlineto{\pgfqpoint{2.962284in}{1.655680in}}%
\pgfpathlineto{\pgfqpoint{2.950496in}{1.592386in}}%
\pgfpathlineto{\pgfqpoint{2.940383in}{1.526185in}}%
\pgfpathlineto{\pgfqpoint{2.931745in}{1.454681in}}%
\pgfpathlineto{\pgfqpoint{2.925082in}{1.380399in}}%
\pgfpathlineto{\pgfqpoint{2.920647in}{1.305899in}}%
\pgfpathlineto{\pgfqpoint{2.918444in}{1.231270in}}%
\pgfpathlineto{\pgfqpoint{2.918545in}{1.159087in}}%
\pgfpathlineto{\pgfqpoint{2.920787in}{1.091931in}}%
\pgfpathlineto{\pgfqpoint{2.925177in}{1.027412in}}%
\pgfpathlineto{\pgfqpoint{2.931192in}{0.970580in}}%
\pgfpathlineto{\pgfqpoint{2.938760in}{0.919034in}}%
\pgfpathlineto{\pgfqpoint{2.947651in}{0.872852in}}%
\pgfpathlineto{\pgfqpoint{2.958213in}{0.829714in}}%
\pgfpathlineto{\pgfqpoint{2.969670in}{0.792114in}}%
\pgfpathlineto{\pgfqpoint{2.982463in}{0.757773in}}%
\pgfpathlineto{\pgfqpoint{2.996425in}{0.726812in}}%
\pgfpathlineto{\pgfqpoint{3.011299in}{0.699300in}}%
\pgfpathlineto{\pgfqpoint{3.026739in}{0.675225in}}%
\pgfpathlineto{\pgfqpoint{3.043828in}{0.652656in}}%
\pgfpathlineto{\pgfqpoint{3.062495in}{0.631788in}}%
\pgfpathlineto{\pgfqpoint{3.082602in}{0.612753in}}%
\pgfpathlineto{\pgfqpoint{3.103961in}{0.595592in}}%
\pgfpathlineto{\pgfqpoint{3.128268in}{0.579069in}}%
\pgfpathlineto{\pgfqpoint{3.153537in}{0.564554in}}%
\pgfpathlineto{\pgfqpoint{3.181571in}{0.550952in}}%
\pgfpathlineto{\pgfqpoint{3.214371in}{0.537647in}}%
\pgfpathlineto{\pgfqpoint{3.249846in}{0.525712in}}%
\pgfpathlineto{\pgfqpoint{3.290011in}{0.514571in}}%
\pgfpathlineto{\pgfqpoint{3.334820in}{0.504423in}}%
\pgfpathlineto{\pgfqpoint{3.386372in}{0.494999in}}%
\pgfpathlineto{\pgfqpoint{3.446798in}{0.486257in}}%
\pgfpathlineto{\pgfqpoint{3.518243in}{0.478282in}}%
\pgfpathlineto{\pgfqpoint{3.600685in}{0.471409in}}%
\pgfpathlineto{\pgfqpoint{3.696268in}{0.465713in}}%
\pgfpathlineto{\pgfqpoint{3.807144in}{0.461369in}}%
\pgfpathlineto{\pgfqpoint{3.933291in}{0.458719in}}%
\pgfpathlineto{\pgfqpoint{4.063808in}{0.458211in}}%
\pgfpathlineto{\pgfqpoint{4.187792in}{0.459914in}}%
\pgfpathlineto{\pgfqpoint{4.294335in}{0.463521in}}%
\pgfpathlineto{\pgfqpoint{4.381234in}{0.468574in}}%
\pgfpathlineto{\pgfqpoint{4.450636in}{0.474701in}}%
\pgfpathlineto{\pgfqpoint{4.506850in}{0.481799in}}%
\pgfpathlineto{\pgfqpoint{4.552009in}{0.489658in}}%
\pgfpathlineto{\pgfqpoint{4.588239in}{0.498115in}}%
\pgfpathlineto{\pgfqpoint{4.617656in}{0.507110in}}%
\pgfpathlineto{\pgfqpoint{4.642328in}{0.516843in}}%
\pgfpathlineto{\pgfqpoint{4.664194in}{0.527940in}}%
\pgfpathlineto{\pgfqpoint{4.681238in}{0.538945in}}%
\pgfpathlineto{\pgfqpoint{4.697164in}{0.551953in}}%
\pgfpathlineto{\pgfqpoint{4.710076in}{0.565289in}}%
\pgfpathlineto{\pgfqpoint{4.721578in}{0.580218in}}%
\pgfpathlineto{\pgfqpoint{4.731557in}{0.596521in}}%
\pgfpathlineto{\pgfqpoint{4.741000in}{0.616134in}}%
\pgfpathlineto{\pgfqpoint{4.749521in}{0.639027in}}%
\pgfpathlineto{\pgfqpoint{4.757522in}{0.667450in}}%
\pgfpathlineto{\pgfqpoint{4.764572in}{0.701345in}}%
\pgfpathlineto{\pgfqpoint{4.770840in}{0.743043in}}%
\pgfpathlineto{\pgfqpoint{4.776327in}{0.794934in}}%
\pgfpathlineto{\pgfqpoint{4.781278in}{0.864398in}}%
\pgfpathlineto{\pgfqpoint{4.785468in}{0.956371in}}%
\pgfpathlineto{\pgfqpoint{4.789000in}{1.085745in}}%
\pgfpathlineto{\pgfqpoint{4.791852in}{1.277385in}}%
\pgfpathlineto{\pgfqpoint{4.793959in}{1.581058in}}%
\pgfpathlineto{\pgfqpoint{4.794962in}{2.071429in}}%
\pgfpathlineto{\pgfqpoint{4.793967in}{2.559311in}}%
\pgfpathlineto{\pgfqpoint{4.791733in}{2.745981in}}%
\pgfpathlineto{\pgfqpoint{4.788955in}{2.818091in}}%
\pgfpathlineto{\pgfqpoint{4.785731in}{2.850227in}}%
\pgfpathlineto{\pgfqpoint{4.781879in}{2.867057in}}%
\pgfpathlineto{\pgfqpoint{4.777744in}{2.875780in}}%
\pgfpathlineto{\pgfqpoint{4.773097in}{2.880982in}}%
\pgfpathlineto{\pgfqpoint{4.767363in}{2.884504in}}%
\pgfpathlineto{\pgfqpoint{4.756853in}{2.887622in}}%
\pgfpathlineto{\pgfqpoint{4.739548in}{2.889639in}}%
\pgfpathlineto{\pgfqpoint{4.704762in}{2.890882in}}%
\pgfpathlineto{\pgfqpoint{4.602524in}{2.891538in}}%
\pgfpathlineto{\pgfqpoint{3.952100in}{2.891742in}}%
\pgfpathlineto{\pgfqpoint{0.617321in}{2.890753in}}%
\pgfpathlineto{\pgfqpoint{0.549910in}{2.888858in}}%
\pgfpathlineto{\pgfqpoint{0.521735in}{2.886179in}}%
\pgfpathlineto{\pgfqpoint{0.504666in}{2.882389in}}%
\pgfpathlineto{\pgfqpoint{0.494501in}{2.878011in}}%
\pgfpathlineto{\pgfqpoint{0.487180in}{2.872667in}}%
\pgfpathlineto{\pgfqpoint{0.481152in}{2.865519in}}%
\pgfpathlineto{\pgfqpoint{0.475664in}{2.854804in}}%
\pgfpathlineto{\pgfqpoint{0.471318in}{2.840737in}}%
\pgfpathlineto{\pgfqpoint{0.467301in}{2.818823in}}%
\pgfpathlineto{\pgfqpoint{0.463927in}{2.786700in}}%
\pgfpathlineto{\pgfqpoint{0.460918in}{2.734545in}}%
\pgfpathlineto{\pgfqpoint{0.458363in}{2.647474in}}%
\pgfpathlineto{\pgfqpoint{0.456575in}{2.523031in}}%
\pgfpathlineto{\pgfqpoint{0.456575in}{2.523031in}}%
\pgfusepath{stroke}%
\end{pgfscope}%
\begin{pgfscope}%
\pgfpathrectangle{\pgfqpoint{0.448634in}{0.402556in}}{\pgfqpoint{4.350661in}{2.489204in}} %
\pgfusepath{clip}%
\pgfsetrectcap%
\pgfsetroundjoin%
\pgfsetlinewidth{1.003750pt}%
\definecolor{currentstroke}{rgb}{0.580392,0.403922,0.741176}%
\pgfsetstrokecolor{currentstroke}%
\pgfsetdash{}{0pt}%
\pgfpathmoveto{\pgfqpoint{4.798840in}{2.852369in}}%
\pgfpathlineto{\pgfqpoint{4.797564in}{2.889610in}}%
\pgfpathlineto{\pgfqpoint{4.796215in}{2.891483in}}%
\pgfpathlineto{\pgfqpoint{4.787551in}{2.891760in}}%
\pgfpathlineto{\pgfqpoint{0.452128in}{2.891664in}}%
\pgfpathlineto{\pgfqpoint{0.450530in}{2.890087in}}%
\pgfpathlineto{\pgfqpoint{0.449453in}{2.882768in}}%
\pgfpathlineto{\pgfqpoint{0.448969in}{2.845437in}}%
\pgfpathlineto{\pgfqpoint{0.448742in}{2.491970in}}%
\pgfpathlineto{\pgfqpoint{0.449622in}{0.615112in}}%
\pgfpathlineto{\pgfqpoint{0.451429in}{0.510591in}}%
\pgfpathlineto{\pgfqpoint{0.453985in}{0.473379in}}%
\pgfpathlineto{\pgfqpoint{0.457393in}{0.453872in}}%
\pgfpathlineto{\pgfqpoint{0.461521in}{0.442385in}}%
\pgfpathlineto{\pgfqpoint{0.466715in}{0.434434in}}%
\pgfpathlineto{\pgfqpoint{0.473568in}{0.428343in}}%
\pgfpathlineto{\pgfqpoint{0.483465in}{0.423234in}}%
\pgfpathlineto{\pgfqpoint{0.489716in}{0.421095in}}%
\pgfpathlineto{\pgfqpoint{0.489716in}{0.421095in}}%
\pgfusepath{stroke}%
\end{pgfscope}%
\begin{pgfscope}%
\pgfpathrectangle{\pgfqpoint{0.448634in}{0.402556in}}{\pgfqpoint{4.350661in}{2.489204in}} %
\pgfusepath{clip}%
\pgfsetrectcap%
\pgfsetroundjoin%
\pgfsetlinewidth{1.003750pt}%
\definecolor{currentstroke}{rgb}{0.580392,0.403922,0.741176}%
\pgfsetstrokecolor{currentstroke}%
\pgfsetdash{}{0pt}%
\pgfpathmoveto{\pgfqpoint{0.456424in}{1.370141in}}%
\pgfpathlineto{\pgfqpoint{0.459610in}{1.118760in}}%
\pgfpathlineto{\pgfqpoint{0.463694in}{0.962012in}}%
\pgfpathlineto{\pgfqpoint{0.468518in}{0.857614in}}%
\pgfpathlineto{\pgfqpoint{0.474082in}{0.783215in}}%
\pgfpathlineto{\pgfqpoint{0.480225in}{0.728911in}}%
\pgfpathlineto{\pgfqpoint{0.486970in}{0.687310in}}%
\pgfpathlineto{\pgfqpoint{0.494536in}{0.653563in}}%
\pgfpathlineto{\pgfqpoint{0.503106in}{0.625359in}}%
\pgfpathlineto{\pgfqpoint{0.512191in}{0.602753in}}%
\pgfpathlineto{\pgfqpoint{0.522198in}{0.583511in}}%
\pgfpathlineto{\pgfqpoint{0.534105in}{0.565746in}}%
\pgfpathlineto{\pgfqpoint{0.546261in}{0.551510in}}%
\pgfpathlineto{\pgfqpoint{0.559725in}{0.538910in}}%
\pgfpathlineto{\pgfqpoint{0.576126in}{0.526696in}}%
\pgfpathlineto{\pgfqpoint{0.595480in}{0.515353in}}%
\pgfpathlineto{\pgfqpoint{0.617677in}{0.505148in}}%
\pgfpathlineto{\pgfqpoint{0.642564in}{0.496154in}}%
\pgfpathlineto{\pgfqpoint{0.672122in}{0.487779in}}%
\pgfpathlineto{\pgfqpoint{0.708439in}{0.479824in}}%
\pgfpathlineto{\pgfqpoint{0.753646in}{0.472326in}}%
\pgfpathlineto{\pgfqpoint{0.807714in}{0.465661in}}%
\pgfpathlineto{\pgfqpoint{0.877112in}{0.459475in}}%
\pgfpathlineto{\pgfqpoint{0.961824in}{0.454230in}}%
\pgfpathlineto{\pgfqpoint{1.068348in}{0.449916in}}%
\pgfpathlineto{\pgfqpoint{1.201014in}{0.446839in}}%
\pgfpathlineto{\pgfqpoint{1.357633in}{0.445481in}}%
\pgfpathlineto{\pgfqpoint{1.525131in}{0.446232in}}%
\pgfpathlineto{\pgfqpoint{1.686084in}{0.449142in}}%
\pgfpathlineto{\pgfqpoint{1.823070in}{0.453747in}}%
\pgfpathlineto{\pgfqpoint{1.938241in}{0.459764in}}%
\pgfpathlineto{\pgfqpoint{2.031578in}{0.466759in}}%
\pgfpathlineto{\pgfqpoint{2.109576in}{0.474745in}}%
\pgfpathlineto{\pgfqpoint{2.174380in}{0.483534in}}%
\pgfpathlineto{\pgfqpoint{2.228135in}{0.492939in}}%
\pgfpathlineto{\pgfqpoint{2.275115in}{0.503356in}}%
\pgfpathlineto{\pgfqpoint{2.315278in}{0.514500in}}%
\pgfpathlineto{\pgfqpoint{2.350694in}{0.526658in}}%
\pgfpathlineto{\pgfqpoint{2.381317in}{0.539534in}}%
\pgfpathlineto{\pgfqpoint{2.407161in}{0.552657in}}%
\pgfpathlineto{\pgfqpoint{2.430223in}{0.566637in}}%
\pgfpathlineto{\pgfqpoint{2.452278in}{0.582599in}}%
\pgfpathlineto{\pgfqpoint{2.471388in}{0.599066in}}%
\pgfpathlineto{\pgfqpoint{2.489238in}{0.617290in}}%
\pgfpathlineto{\pgfqpoint{2.505675in}{0.637177in}}%
\pgfpathlineto{\pgfqpoint{2.520618in}{0.658554in}}%
\pgfpathlineto{\pgfqpoint{2.535211in}{0.683310in}}%
\pgfpathlineto{\pgfqpoint{2.549114in}{0.711480in}}%
\pgfpathlineto{\pgfqpoint{2.562089in}{0.743000in}}%
\pgfpathlineto{\pgfqpoint{2.574019in}{0.777747in}}%
\pgfpathlineto{\pgfqpoint{2.585501in}{0.817966in}}%
\pgfpathlineto{\pgfqpoint{2.596808in}{0.866034in}}%
\pgfpathlineto{\pgfqpoint{2.607561in}{0.921943in}}%
\pgfpathlineto{\pgfqpoint{2.617924in}{0.988094in}}%
\pgfpathlineto{\pgfqpoint{2.627958in}{1.066914in}}%
\pgfpathlineto{\pgfqpoint{2.637941in}{1.163316in}}%
\pgfpathlineto{\pgfqpoint{2.648424in}{1.287195in}}%
\pgfpathlineto{\pgfqpoint{2.660103in}{1.453434in}}%
\pgfpathlineto{\pgfqpoint{2.674773in}{1.696797in}}%
\pgfpathlineto{\pgfqpoint{2.687716in}{1.945274in}}%
\pgfpathlineto{\pgfqpoint{2.692670in}{2.079569in}}%
\pgfpathlineto{\pgfqpoint{2.693829in}{2.166678in}}%
\pgfpathlineto{\pgfqpoint{2.692565in}{2.233866in}}%
\pgfpathlineto{\pgfqpoint{2.689437in}{2.286010in}}%
\pgfpathlineto{\pgfqpoint{2.684859in}{2.327995in}}%
\pgfpathlineto{\pgfqpoint{2.678725in}{2.364660in}}%
\pgfpathlineto{\pgfqpoint{2.671357in}{2.395893in}}%
\pgfpathlineto{\pgfqpoint{2.662490in}{2.423977in}}%
\pgfpathlineto{\pgfqpoint{2.652363in}{2.448774in}}%
\pgfpathlineto{\pgfqpoint{2.641367in}{2.470242in}}%
\pgfpathlineto{\pgfqpoint{2.628645in}{2.490421in}}%
\pgfpathlineto{\pgfqpoint{2.614281in}{2.509102in}}%
\pgfpathlineto{\pgfqpoint{2.598446in}{2.526156in}}%
\pgfpathlineto{\pgfqpoint{2.579592in}{2.543003in}}%
\pgfpathlineto{\pgfqpoint{2.559535in}{2.557921in}}%
\pgfpathlineto{\pgfqpoint{2.536605in}{2.572181in}}%
\pgfpathlineto{\pgfqpoint{2.510853in}{2.585537in}}%
\pgfpathlineto{\pgfqpoint{2.482363in}{2.597836in}}%
\pgfpathlineto{\pgfqpoint{2.449138in}{2.609682in}}%
\pgfpathlineto{\pgfqpoint{2.411187in}{2.620695in}}%
\pgfpathlineto{\pgfqpoint{2.368555in}{2.630605in}}%
\pgfpathlineto{\pgfqpoint{2.321297in}{2.639220in}}%
\pgfpathlineto{\pgfqpoint{2.269470in}{2.646398in}}%
\pgfpathlineto{\pgfqpoint{2.210958in}{2.652193in}}%
\pgfpathlineto{\pgfqpoint{2.147970in}{2.656153in}}%
\pgfpathlineto{\pgfqpoint{2.080560in}{2.658135in}}%
\pgfpathlineto{\pgfqpoint{2.010951in}{2.657971in}}%
\pgfpathlineto{\pgfqpoint{1.939198in}{2.655572in}}%
\pgfpathlineto{\pgfqpoint{1.867530in}{2.650913in}}%
\pgfpathlineto{\pgfqpoint{1.798174in}{2.644141in}}%
\pgfpathlineto{\pgfqpoint{1.733344in}{2.635606in}}%
\pgfpathlineto{\pgfqpoint{1.673079in}{2.625522in}}%
\pgfpathlineto{\pgfqpoint{1.615277in}{2.613611in}}%
\pgfpathlineto{\pgfqpoint{1.562136in}{2.600402in}}%
\pgfpathlineto{\pgfqpoint{1.513684in}{2.586140in}}%
\pgfpathlineto{\pgfqpoint{1.467865in}{2.570345in}}%
\pgfpathlineto{\pgfqpoint{1.426797in}{2.553924in}}%
\pgfpathlineto{\pgfqpoint{1.388450in}{2.536290in}}%
\pgfpathlineto{\pgfqpoint{1.352881in}{2.517568in}}%
\pgfpathlineto{\pgfqpoint{1.320131in}{2.497924in}}%
\pgfpathlineto{\pgfqpoint{1.288382in}{2.476238in}}%
\pgfpathlineto{\pgfqpoint{1.259595in}{2.453863in}}%
\pgfpathlineto{\pgfqpoint{1.232053in}{2.429522in}}%
\pgfpathlineto{\pgfqpoint{1.207530in}{2.404901in}}%
\pgfpathlineto{\pgfqpoint{1.184411in}{2.378560in}}%
\pgfpathlineto{\pgfqpoint{1.162830in}{2.350564in}}%
\pgfpathlineto{\pgfqpoint{1.142893in}{2.321015in}}%
\pgfpathlineto{\pgfqpoint{1.124677in}{2.290044in}}%
\pgfpathlineto{\pgfqpoint{1.108227in}{2.257805in}}%
\pgfpathlineto{\pgfqpoint{1.092641in}{2.222202in}}%
\pgfpathlineto{\pgfqpoint{1.079061in}{2.185539in}}%
\pgfpathlineto{\pgfqpoint{1.067444in}{2.148001in}}%
\pgfpathlineto{\pgfqpoint{1.057188in}{2.107351in}}%
\pgfpathlineto{\pgfqpoint{1.049005in}{2.066090in}}%
\pgfpathlineto{\pgfqpoint{1.042514in}{2.021910in}}%
\pgfpathlineto{\pgfqpoint{1.038177in}{1.977385in}}%
\pgfpathlineto{\pgfqpoint{1.035866in}{1.930171in}}%
\pgfpathlineto{\pgfqpoint{1.035826in}{1.882882in}}%
\pgfpathlineto{\pgfqpoint{1.038031in}{1.835660in}}%
\pgfpathlineto{\pgfqpoint{1.042474in}{1.788645in}}%
\pgfpathlineto{\pgfqpoint{1.049175in}{1.741982in}}%
\pgfpathlineto{\pgfqpoint{1.057643in}{1.698243in}}%
\pgfpathlineto{\pgfqpoint{1.068221in}{1.655109in}}%
\pgfpathlineto{\pgfqpoint{1.080962in}{1.612748in}}%
\pgfpathlineto{\pgfqpoint{1.095030in}{1.573620in}}%
\pgfpathlineto{\pgfqpoint{1.111114in}{1.535523in}}%
\pgfpathlineto{\pgfqpoint{1.128116in}{1.500778in}}%
\pgfpathlineto{\pgfqpoint{1.146928in}{1.467277in}}%
\pgfpathlineto{\pgfqpoint{1.167529in}{1.435184in}}%
\pgfpathlineto{\pgfqpoint{1.189872in}{1.404655in}}%
\pgfpathlineto{\pgfqpoint{1.213882in}{1.375830in}}%
\pgfpathlineto{\pgfqpoint{1.237815in}{1.350459in}}%
\pgfpathlineto{\pgfqpoint{1.264746in}{1.325240in}}%
\pgfpathlineto{\pgfqpoint{1.292988in}{1.301974in}}%
\pgfpathlineto{\pgfqpoint{1.322395in}{1.280680in}}%
\pgfpathlineto{\pgfqpoint{1.352817in}{1.261342in}}%
\pgfpathlineto{\pgfqpoint{1.386092in}{1.242890in}}%
\pgfpathlineto{\pgfqpoint{1.420188in}{1.226517in}}%
\pgfpathlineto{\pgfqpoint{1.457021in}{1.211330in}}%
\pgfpathlineto{\pgfqpoint{1.496551in}{1.197537in}}%
\pgfpathlineto{\pgfqpoint{1.538717in}{1.185287in}}%
\pgfpathlineto{\pgfqpoint{1.583438in}{1.174642in}}%
\pgfpathlineto{\pgfqpoint{1.634926in}{1.164775in}}%
\pgfpathlineto{\pgfqpoint{1.706060in}{1.153745in}}%
\pgfpathlineto{\pgfqpoint{1.768489in}{1.143417in}}%
\pgfpathlineto{\pgfqpoint{1.796119in}{1.136568in}}%
\pgfpathlineto{\pgfqpoint{1.812680in}{1.130482in}}%
\pgfpathlineto{\pgfqpoint{1.824468in}{1.124103in}}%
\pgfpathlineto{\pgfqpoint{1.833207in}{1.116743in}}%
\pgfpathlineto{\pgfqpoint{1.838496in}{1.108891in}}%
\pgfpathlineto{\pgfqpoint{1.840587in}{1.101851in}}%
\pgfpathlineto{\pgfqpoint{1.840618in}{1.094414in}}%
\pgfpathlineto{\pgfqpoint{1.837930in}{1.084987in}}%
\pgfpathlineto{\pgfqpoint{1.833246in}{1.076617in}}%
\pgfpathlineto{\pgfqpoint{1.825818in}{1.067544in}}%
\pgfpathlineto{\pgfqpoint{1.813813in}{1.056851in}}%
\pgfpathlineto{\pgfqpoint{1.798819in}{1.046764in}}%
\pgfpathlineto{\pgfqpoint{1.781016in}{1.037463in}}%
\pgfpathlineto{\pgfqpoint{1.758447in}{1.028392in}}%
\pgfpathlineto{\pgfqpoint{1.733203in}{1.020816in}}%
\pgfpathlineto{\pgfqpoint{1.705410in}{1.014872in}}%
\pgfpathlineto{\pgfqpoint{1.675178in}{1.010714in}}%
\pgfpathlineto{\pgfqpoint{1.642610in}{1.008508in}}%
\pgfpathlineto{\pgfqpoint{1.607809in}{1.008433in}}%
\pgfpathlineto{\pgfqpoint{1.570886in}{1.010692in}}%
\pgfpathlineto{\pgfqpoint{1.534118in}{1.015182in}}%
\pgfpathlineto{\pgfqpoint{1.495455in}{1.022233in}}%
\pgfpathlineto{\pgfqpoint{1.457161in}{1.031564in}}%
\pgfpathlineto{\pgfqpoint{1.419338in}{1.043132in}}%
\pgfpathlineto{\pgfqpoint{1.382089in}{1.056929in}}%
\pgfpathlineto{\pgfqpoint{1.347544in}{1.072019in}}%
\pgfpathlineto{\pgfqpoint{1.313727in}{1.089133in}}%
\pgfpathlineto{\pgfqpoint{1.280762in}{1.108299in}}%
\pgfpathlineto{\pgfqpoint{1.248782in}{1.129536in}}%
\pgfpathlineto{\pgfqpoint{1.219708in}{1.151422in}}%
\pgfpathlineto{\pgfqpoint{1.191752in}{1.175138in}}%
\pgfpathlineto{\pgfqpoint{1.165031in}{1.200649in}}%
\pgfpathlineto{\pgfqpoint{1.139653in}{1.227897in}}%
\pgfpathlineto{\pgfqpoint{1.115714in}{1.256800in}}%
\pgfpathlineto{\pgfqpoint{1.093288in}{1.287251in}}%
\pgfpathlineto{\pgfqpoint{1.071177in}{1.321163in}}%
\pgfpathlineto{\pgfqpoint{1.050868in}{1.356519in}}%
\pgfpathlineto{\pgfqpoint{1.032365in}{1.393151in}}%
\pgfpathlineto{\pgfqpoint{1.014718in}{1.433142in}}%
\pgfpathlineto{\pgfqpoint{0.999024in}{1.474185in}}%
\pgfpathlineto{\pgfqpoint{0.984506in}{1.518461in}}%
\pgfpathlineto{\pgfqpoint{0.972009in}{1.563537in}}%
\pgfpathlineto{\pgfqpoint{0.960943in}{1.611678in}}%
\pgfpathlineto{\pgfqpoint{0.951530in}{1.662824in}}%
\pgfpathlineto{\pgfqpoint{0.944286in}{1.714431in}}%
\pgfpathlineto{\pgfqpoint{0.938950in}{1.768847in}}%
\pgfpathlineto{\pgfqpoint{0.935870in}{1.823491in}}%
\pgfpathlineto{\pgfqpoint{0.935034in}{1.878240in}}%
\pgfpathlineto{\pgfqpoint{0.936465in}{1.932972in}}%
\pgfpathlineto{\pgfqpoint{0.940005in}{1.985084in}}%
\pgfpathlineto{\pgfqpoint{0.945758in}{2.036935in}}%
\pgfpathlineto{\pgfqpoint{0.953409in}{2.085938in}}%
\pgfpathlineto{\pgfqpoint{0.962764in}{2.132000in}}%
\pgfpathlineto{\pgfqpoint{0.974286in}{2.177413in}}%
\pgfpathlineto{\pgfqpoint{0.987332in}{2.219652in}}%
\pgfpathlineto{\pgfqpoint{1.001667in}{2.258654in}}%
\pgfpathlineto{\pgfqpoint{1.018050in}{2.296583in}}%
\pgfpathlineto{\pgfqpoint{1.035401in}{2.331101in}}%
\pgfpathlineto{\pgfqpoint{1.054650in}{2.364275in}}%
\pgfpathlineto{\pgfqpoint{1.074406in}{2.393983in}}%
\pgfpathlineto{\pgfqpoint{1.095771in}{2.422197in}}%
\pgfpathlineto{\pgfqpoint{1.118662in}{2.448796in}}%
\pgfpathlineto{\pgfqpoint{1.142966in}{2.473700in}}%
\pgfpathlineto{\pgfqpoint{1.168550in}{2.496867in}}%
\pgfpathlineto{\pgfqpoint{1.197084in}{2.519662in}}%
\pgfpathlineto{\pgfqpoint{1.226726in}{2.540526in}}%
\pgfpathlineto{\pgfqpoint{1.259241in}{2.560673in}}%
\pgfpathlineto{\pgfqpoint{1.294612in}{2.579881in}}%
\pgfpathlineto{\pgfqpoint{1.332792in}{2.597981in}}%
\pgfpathlineto{\pgfqpoint{1.373718in}{2.614859in}}%
\pgfpathlineto{\pgfqpoint{1.417319in}{2.630445in}}%
\pgfpathlineto{\pgfqpoint{1.465632in}{2.645312in}}%
\pgfpathlineto{\pgfqpoint{1.518640in}{2.659204in}}%
\pgfpathlineto{\pgfqpoint{1.576309in}{2.671929in}}%
\pgfpathlineto{\pgfqpoint{1.638597in}{2.683344in}}%
\pgfpathlineto{\pgfqpoint{1.705462in}{2.693343in}}%
\pgfpathlineto{\pgfqpoint{1.779027in}{2.702064in}}%
\pgfpathlineto{\pgfqpoint{1.857097in}{2.709076in}}%
\pgfpathlineto{\pgfqpoint{1.939633in}{2.714280in}}%
\pgfpathlineto{\pgfqpoint{2.026598in}{2.717513in}}%
\pgfpathlineto{\pgfqpoint{2.113605in}{2.718523in}}%
\pgfpathlineto{\pgfqpoint{2.198434in}{2.717303in}}%
\pgfpathlineto{\pgfqpoint{2.278865in}{2.713929in}}%
\pgfpathlineto{\pgfqpoint{2.352677in}{2.708598in}}%
\pgfpathlineto{\pgfqpoint{2.417656in}{2.701709in}}%
\pgfpathlineto{\pgfqpoint{2.473770in}{2.693630in}}%
\pgfpathlineto{\pgfqpoint{2.523140in}{2.684368in}}%
\pgfpathlineto{\pgfqpoint{2.565726in}{2.674202in}}%
\pgfpathlineto{\pgfqpoint{2.601510in}{2.663544in}}%
\pgfpathlineto{\pgfqpoint{2.632576in}{2.652142in}}%
\pgfpathlineto{\pgfqpoint{2.658899in}{2.640331in}}%
\pgfpathlineto{\pgfqpoint{2.682438in}{2.627436in}}%
\pgfpathlineto{\pgfqpoint{2.703062in}{2.613571in}}%
\pgfpathlineto{\pgfqpoint{2.720674in}{2.598978in}}%
\pgfpathlineto{\pgfqpoint{2.735262in}{2.584053in}}%
\pgfpathlineto{\pgfqpoint{2.748319in}{2.567377in}}%
\pgfpathlineto{\pgfqpoint{2.759553in}{2.549046in}}%
\pgfpathlineto{\pgfqpoint{2.768787in}{2.529306in}}%
\pgfpathlineto{\pgfqpoint{2.776016in}{2.508498in}}%
\pgfpathlineto{\pgfqpoint{2.781884in}{2.484540in}}%
\pgfpathlineto{\pgfqpoint{2.786102in}{2.457597in}}%
\pgfpathlineto{\pgfqpoint{2.788720in}{2.425384in}}%
\pgfpathlineto{\pgfqpoint{2.789427in}{2.388061in}}%
\pgfpathlineto{\pgfqpoint{2.787962in}{2.340801in}}%
\pgfpathlineto{\pgfqpoint{2.783672in}{2.278768in}}%
\pgfpathlineto{\pgfqpoint{2.774288in}{2.179783in}}%
\pgfpathlineto{\pgfqpoint{2.743611in}{1.868119in}}%
\pgfpathlineto{\pgfqpoint{2.730111in}{1.702060in}}%
\pgfpathlineto{\pgfqpoint{2.717287in}{1.515949in}}%
\pgfpathlineto{\pgfqpoint{2.702602in}{1.267597in}}%
\pgfpathlineto{\pgfqpoint{2.684434in}{0.964630in}}%
\pgfpathlineto{\pgfqpoint{2.675374in}{0.850600in}}%
\pgfpathlineto{\pgfqpoint{2.667030in}{0.771523in}}%
\pgfpathlineto{\pgfqpoint{2.658752in}{0.712543in}}%
\pgfpathlineto{\pgfqpoint{2.650176in}{0.666284in}}%
\pgfpathlineto{\pgfqpoint{2.640820in}{0.627931in}}%
\pgfpathlineto{\pgfqpoint{2.631144in}{0.597534in}}%
\pgfpathlineto{\pgfqpoint{2.621003in}{0.572745in}}%
\pgfpathlineto{\pgfqpoint{2.609856in}{0.551383in}}%
\pgfpathlineto{\pgfqpoint{2.598042in}{0.533534in}}%
\pgfpathlineto{\pgfqpoint{2.584495in}{0.517378in}}%
\pgfpathlineto{\pgfqpoint{2.571108in}{0.504669in}}%
\pgfpathlineto{\pgfqpoint{2.554789in}{0.492313in}}%
\pgfpathlineto{\pgfqpoint{2.537456in}{0.481914in}}%
\pgfpathlineto{\pgfqpoint{2.517373in}{0.472367in}}%
\pgfpathlineto{\pgfqpoint{2.492542in}{0.463178in}}%
\pgfpathlineto{\pgfqpoint{2.462979in}{0.454833in}}%
\pgfpathlineto{\pgfqpoint{2.428766in}{0.447542in}}%
\pgfpathlineto{\pgfqpoint{2.385671in}{0.440735in}}%
\pgfpathlineto{\pgfqpoint{2.331557in}{0.434581in}}%
\pgfpathlineto{\pgfqpoint{2.262115in}{0.429077in}}%
\pgfpathlineto{\pgfqpoint{2.170850in}{0.424236in}}%
\pgfpathlineto{\pgfqpoint{2.049086in}{0.420134in}}%
\pgfpathlineto{\pgfqpoint{1.879436in}{0.416783in}}%
\pgfpathlineto{\pgfqpoint{1.640159in}{0.414418in}}%
\pgfpathlineto{\pgfqpoint{1.322562in}{0.413569in}}%
\pgfpathlineto{\pgfqpoint{1.020194in}{0.414850in}}%
\pgfpathlineto{\pgfqpoint{0.822256in}{0.417715in}}%
\pgfpathlineto{\pgfqpoint{0.704834in}{0.421430in}}%
\pgfpathlineto{\pgfqpoint{0.630976in}{0.425829in}}%
\pgfpathlineto{\pgfqpoint{0.583315in}{0.430734in}}%
\pgfpathlineto{\pgfqpoint{0.551033in}{0.436124in}}%
\pgfpathlineto{\pgfqpoint{0.527708in}{0.442189in}}%
\pgfpathlineto{\pgfqpoint{0.511250in}{0.448626in}}%
\pgfpathlineto{\pgfqpoint{0.499548in}{0.455216in}}%
\pgfpathlineto{\pgfqpoint{0.488916in}{0.463842in}}%
\pgfpathlineto{\pgfqpoint{0.481322in}{0.472731in}}%
\pgfpathlineto{\pgfqpoint{0.474078in}{0.485127in}}%
\pgfpathlineto{\pgfqpoint{0.468753in}{0.498749in}}%
\pgfpathlineto{\pgfqpoint{0.463869in}{0.517849in}}%
\pgfpathlineto{\pgfqpoint{0.459679in}{0.544797in}}%
\pgfpathlineto{\pgfqpoint{0.456386in}{0.581939in}}%
\pgfpathlineto{\pgfqpoint{0.453731in}{0.639107in}}%
\pgfpathlineto{\pgfqpoint{0.451681in}{0.736156in}}%
\pgfpathlineto{\pgfqpoint{0.450220in}{0.927816in}}%
\pgfpathlineto{\pgfqpoint{0.449345in}{1.403253in}}%
\pgfpathlineto{\pgfqpoint{0.449543in}{2.682703in}}%
\pgfpathlineto{\pgfqpoint{0.451011in}{2.856933in}}%
\pgfpathlineto{\pgfqpoint{0.452802in}{2.879220in}}%
\pgfpathlineto{\pgfqpoint{0.455188in}{2.886108in}}%
\pgfpathlineto{\pgfqpoint{0.458626in}{2.889029in}}%
\pgfpathlineto{\pgfqpoint{0.464996in}{2.890553in}}%
\pgfpathlineto{\pgfqpoint{0.482377in}{2.891423in}}%
\pgfpathlineto{\pgfqpoint{0.565039in}{2.891729in}}%
\pgfpathlineto{\pgfqpoint{2.733843in}{2.891760in}}%
\pgfpathlineto{\pgfqpoint{4.789510in}{2.890885in}}%
\pgfpathlineto{\pgfqpoint{4.793727in}{2.889729in}}%
\pgfpathlineto{\pgfqpoint{4.795481in}{2.888304in}}%
\pgfpathlineto{\pgfqpoint{4.797105in}{2.881142in}}%
\pgfpathlineto{\pgfqpoint{4.797996in}{2.858769in}}%
\pgfpathlineto{\pgfqpoint{4.798039in}{2.856280in}}%
\pgfpathlineto{\pgfqpoint{4.798039in}{2.856280in}}%
\pgfusepath{stroke}%
\end{pgfscope}%
\begin{pgfscope}%
\pgfpathrectangle{\pgfqpoint{0.448634in}{0.402556in}}{\pgfqpoint{4.350661in}{2.489204in}} %
\pgfusepath{clip}%
\pgfsetrectcap%
\pgfsetroundjoin%
\pgfsetlinewidth{1.003750pt}%
\definecolor{currentstroke}{rgb}{0.580392,0.403922,0.741176}%
\pgfsetstrokecolor{currentstroke}%
\pgfsetdash{}{0pt}%
\pgfpathmoveto{\pgfqpoint{3.428775in}{0.402610in}}%
\pgfpathlineto{\pgfqpoint{2.806635in}{0.403761in}}%
\pgfpathlineto{\pgfqpoint{2.769695in}{0.405580in}}%
\pgfpathlineto{\pgfqpoint{2.754636in}{0.408067in}}%
\pgfpathlineto{\pgfqpoint{2.746395in}{0.411202in}}%
\pgfpathlineto{\pgfqpoint{2.740947in}{0.415270in}}%
\pgfpathlineto{\pgfqpoint{2.736788in}{0.420989in}}%
\pgfpathlineto{\pgfqpoint{2.733285in}{0.430076in}}%
\pgfpathlineto{\pgfqpoint{2.730452in}{0.444641in}}%
\pgfpathlineto{\pgfqpoint{2.728240in}{0.469397in}}%
\pgfpathlineto{\pgfqpoint{2.726472in}{0.519136in}}%
\pgfpathlineto{\pgfqpoint{2.725712in}{0.613719in}}%
\pgfpathlineto{\pgfqpoint{2.726843in}{0.768043in}}%
\pgfpathlineto{\pgfqpoint{2.730557in}{0.962153in}}%
\pgfpathlineto{\pgfqpoint{2.736612in}{1.158675in}}%
\pgfpathlineto{\pgfqpoint{2.744093in}{1.327723in}}%
\pgfpathlineto{\pgfqpoint{2.753202in}{1.484194in}}%
\pgfpathlineto{\pgfqpoint{2.763258in}{1.620614in}}%
\pgfpathlineto{\pgfqpoint{2.776119in}{1.764220in}}%
\pgfpathlineto{\pgfqpoint{2.788916in}{1.877781in}}%
\pgfpathlineto{\pgfqpoint{2.805750in}{2.005745in}}%
\pgfpathlineto{\pgfqpoint{2.821178in}{2.101202in}}%
\pgfpathlineto{\pgfqpoint{2.838362in}{2.193723in}}%
\pgfpathlineto{\pgfqpoint{2.859138in}{2.292970in}}%
\pgfpathlineto{\pgfqpoint{2.887212in}{2.425964in}}%
\pgfpathlineto{\pgfqpoint{2.896994in}{2.479564in}}%
\pgfpathlineto{\pgfqpoint{2.901546in}{2.516527in}}%
\pgfpathlineto{\pgfqpoint{2.902852in}{2.543858in}}%
\pgfpathlineto{\pgfqpoint{2.901960in}{2.566227in}}%
\pgfpathlineto{\pgfqpoint{2.899154in}{2.585867in}}%
\pgfpathlineto{\pgfqpoint{2.894796in}{2.602550in}}%
\pgfpathlineto{\pgfqpoint{2.888486in}{2.618392in}}%
\pgfpathlineto{\pgfqpoint{2.880259in}{2.633036in}}%
\pgfpathlineto{\pgfqpoint{2.870350in}{2.646250in}}%
\pgfpathlineto{\pgfqpoint{2.857401in}{2.659534in}}%
\pgfpathlineto{\pgfqpoint{2.843191in}{2.671013in}}%
\pgfpathlineto{\pgfqpoint{2.824239in}{2.683212in}}%
\pgfpathlineto{\pgfqpoint{2.802414in}{2.694421in}}%
\pgfpathlineto{\pgfqpoint{2.775810in}{2.705371in}}%
\pgfpathlineto{\pgfqpoint{2.744462in}{2.715717in}}%
\pgfpathlineto{\pgfqpoint{2.708437in}{2.725254in}}%
\pgfpathlineto{\pgfqpoint{2.665656in}{2.734291in}}%
\pgfpathlineto{\pgfqpoint{2.613992in}{2.742871in}}%
\pgfpathlineto{\pgfqpoint{2.553460in}{2.750590in}}%
\pgfpathlineto{\pgfqpoint{2.481921in}{2.757367in}}%
\pgfpathlineto{\pgfqpoint{2.399399in}{2.762840in}}%
\pgfpathlineto{\pgfqpoint{2.310270in}{2.766483in}}%
\pgfpathlineto{\pgfqpoint{2.175417in}{2.768726in}}%
\pgfpathlineto{\pgfqpoint{2.066654in}{2.767943in}}%
\pgfpathlineto{\pgfqpoint{1.953571in}{2.764861in}}%
\pgfpathlineto{\pgfqpoint{1.851430in}{2.759760in}}%
\pgfpathlineto{\pgfqpoint{1.745052in}{2.752170in}}%
\pgfpathlineto{\pgfqpoint{1.658374in}{2.743455in}}%
\pgfpathlineto{\pgfqpoint{1.580553in}{2.733463in}}%
\pgfpathlineto{\pgfqpoint{1.490058in}{2.719340in}}%
\pgfpathlineto{\pgfqpoint{1.417232in}{2.704698in}}%
\pgfpathlineto{\pgfqpoint{1.361993in}{2.690818in}}%
\pgfpathlineto{\pgfqpoint{1.311461in}{2.675819in}}%
\pgfpathlineto{\pgfqpoint{1.265668in}{2.659923in}}%
\pgfpathlineto{\pgfqpoint{1.222576in}{2.642586in}}%
\pgfpathlineto{\pgfqpoint{1.184325in}{2.624682in}}%
\pgfpathlineto{\pgfqpoint{1.148893in}{2.605623in}}%
\pgfpathlineto{\pgfqpoint{1.116332in}{2.585573in}}%
\pgfpathlineto{\pgfqpoint{1.092328in}{2.568512in}}%
\pgfpathlineto{\pgfqpoint{1.079761in}{2.558686in}}%
\pgfpathlineto{\pgfqpoint{1.051545in}{2.535379in}}%
\pgfpathlineto{\pgfqpoint{1.026313in}{2.511713in}}%
\pgfpathlineto{\pgfqpoint{1.002399in}{2.486318in}}%
\pgfpathlineto{\pgfqpoint{0.979913in}{2.459270in}}%
\pgfpathlineto{\pgfqpoint{0.958935in}{2.430679in}}%
\pgfpathlineto{\pgfqpoint{0.938265in}{2.398644in}}%
\pgfpathlineto{\pgfqpoint{0.923048in}{2.371386in}}%
\pgfpathlineto{\pgfqpoint{0.904514in}{2.334775in}}%
\pgfpathlineto{\pgfqpoint{0.887854in}{2.297002in}}%
\pgfpathlineto{\pgfqpoint{0.872132in}{2.255972in}}%
\pgfpathlineto{\pgfqpoint{0.857508in}{2.211742in}}%
\pgfpathlineto{\pgfqpoint{0.844762in}{2.166758in}}%
\pgfpathlineto{\pgfqpoint{0.838624in}{2.140307in}}%
\pgfpathlineto{\pgfqpoint{0.826982in}{2.087195in}}%
\pgfpathlineto{\pgfqpoint{0.816322in}{2.028716in}}%
\pgfpathlineto{\pgfqpoint{0.810087in}{1.984496in}}%
\pgfpathlineto{\pgfqpoint{0.808026in}{1.967239in}}%
\pgfpathlineto{\pgfqpoint{0.800076in}{1.898141in}}%
\pgfpathlineto{\pgfqpoint{0.793713in}{1.823824in}}%
\pgfpathlineto{\pgfqpoint{0.788798in}{1.741876in}}%
\pgfpathlineto{\pgfqpoint{0.786199in}{1.677226in}}%
\pgfpathlineto{\pgfqpoint{0.776951in}{1.453482in}}%
\pgfpathlineto{\pgfqpoint{0.773280in}{1.418895in}}%
\pgfpathlineto{\pgfqpoint{0.768298in}{1.389583in}}%
\pgfpathlineto{\pgfqpoint{0.762752in}{1.368109in}}%
\pgfpathlineto{\pgfqpoint{0.756722in}{1.352124in}}%
\pgfpathlineto{\pgfqpoint{0.749752in}{1.339520in}}%
\pgfpathlineto{\pgfqpoint{0.742201in}{1.330600in}}%
\pgfpathlineto{\pgfqpoint{0.734854in}{1.325312in}}%
\pgfpathlineto{\pgfqpoint{0.726558in}{1.322419in}}%
\pgfpathlineto{\pgfqpoint{0.717884in}{1.322223in}}%
\pgfpathlineto{\pgfqpoint{0.709413in}{1.324411in}}%
\pgfpathlineto{\pgfqpoint{0.699548in}{1.329605in}}%
\pgfpathlineto{\pgfqpoint{0.688894in}{1.338203in}}%
\pgfpathlineto{\pgfqpoint{0.677907in}{1.350248in}}%
\pgfpathlineto{\pgfqpoint{0.666886in}{1.365647in}}%
\pgfpathlineto{\pgfqpoint{0.654913in}{1.386417in}}%
\pgfpathlineto{\pgfqpoint{0.642574in}{1.412730in}}%
\pgfpathlineto{\pgfqpoint{0.630328in}{1.444629in}}%
\pgfpathlineto{\pgfqpoint{0.618505in}{1.482080in}}%
\pgfpathlineto{\pgfqpoint{0.608613in}{1.520256in}}%
\pgfpathlineto{\pgfqpoint{0.590203in}{1.612445in}}%
\pgfpathlineto{\pgfqpoint{0.581848in}{1.668884in}}%
\pgfpathlineto{\pgfqpoint{0.573138in}{1.740376in}}%
\pgfpathlineto{\pgfqpoint{0.567062in}{1.807213in}}%
\pgfpathlineto{\pgfqpoint{0.560532in}{1.896509in}}%
\pgfpathlineto{\pgfqpoint{0.555526in}{1.995910in}}%
\pgfpathlineto{\pgfqpoint{0.552564in}{2.097908in}}%
\pgfpathlineto{\pgfqpoint{0.551526in}{2.204935in}}%
\pgfpathlineto{\pgfqpoint{0.552728in}{2.309470in}}%
\pgfpathlineto{\pgfqpoint{0.556011in}{2.403981in}}%
\pgfpathlineto{\pgfqpoint{0.560953in}{2.483430in}}%
\pgfpathlineto{\pgfqpoint{0.567303in}{2.550240in}}%
\pgfpathlineto{\pgfqpoint{0.574928in}{2.606817in}}%
\pgfpathlineto{\pgfqpoint{0.582988in}{2.650657in}}%
\pgfpathlineto{\pgfqpoint{0.592756in}{2.691452in}}%
\pgfpathlineto{\pgfqpoint{0.602650in}{2.721756in}}%
\pgfpathlineto{\pgfqpoint{0.612983in}{2.746441in}}%
\pgfpathlineto{\pgfqpoint{0.624292in}{2.767692in}}%
\pgfpathlineto{\pgfqpoint{0.636231in}{2.785432in}}%
\pgfpathlineto{\pgfqpoint{0.649892in}{2.801461in}}%
\pgfpathlineto{\pgfqpoint{0.663386in}{2.814020in}}%
\pgfpathlineto{\pgfqpoint{0.679842in}{2.826135in}}%
\pgfpathlineto{\pgfqpoint{0.697326in}{2.836197in}}%
\pgfpathlineto{\pgfqpoint{0.715574in}{2.844285in}}%
\pgfpathlineto{\pgfqpoint{0.738439in}{2.852335in}}%
\pgfpathlineto{\pgfqpoint{0.765983in}{2.859639in}}%
\pgfpathlineto{\pgfqpoint{0.800300in}{2.866256in}}%
\pgfpathlineto{\pgfqpoint{0.841340in}{2.871832in}}%
\pgfpathlineto{\pgfqpoint{0.895547in}{2.876803in}}%
\pgfpathlineto{\pgfqpoint{0.969413in}{2.881069in}}%
\pgfpathlineto{\pgfqpoint{1.071608in}{2.884501in}}%
\pgfpathlineto{\pgfqpoint{1.219512in}{2.887074in}}%
\pgfpathlineto{\pgfqpoint{1.471844in}{2.889091in}}%
\pgfpathlineto{\pgfqpoint{1.956941in}{2.890384in}}%
\pgfpathlineto{\pgfqpoint{3.096814in}{2.890781in}}%
\pgfpathlineto{\pgfqpoint{3.995224in}{2.889388in}}%
\pgfpathlineto{\pgfqpoint{4.275833in}{2.887011in}}%
\pgfpathlineto{\pgfqpoint{4.412847in}{2.883743in}}%
\pgfpathlineto{\pgfqpoint{4.491081in}{2.879810in}}%
\pgfpathlineto{\pgfqpoint{4.543127in}{2.875163in}}%
\pgfpathlineto{\pgfqpoint{4.579810in}{2.869841in}}%
\pgfpathlineto{\pgfqpoint{4.607580in}{2.863763in}}%
\pgfpathlineto{\pgfqpoint{4.630623in}{2.856424in}}%
\pgfpathlineto{\pgfqpoint{4.648833in}{2.848228in}}%
\pgfpathlineto{\pgfqpoint{4.664136in}{2.838773in}}%
\pgfpathlineto{\pgfqpoint{4.676470in}{2.828576in}}%
\pgfpathlineto{\pgfqpoint{4.687502in}{2.816585in}}%
\pgfpathlineto{\pgfqpoint{4.697051in}{2.803027in}}%
\pgfpathlineto{\pgfqpoint{4.706194in}{2.786098in}}%
\pgfpathlineto{\pgfqpoint{4.714508in}{2.765827in}}%
\pgfpathlineto{\pgfqpoint{4.722462in}{2.740013in}}%
\pgfpathlineto{\pgfqpoint{4.729577in}{2.708703in}}%
\pgfpathlineto{\pgfqpoint{4.736162in}{2.669601in}}%
\pgfpathlineto{\pgfqpoint{4.742419in}{2.617826in}}%
\pgfpathlineto{\pgfqpoint{4.747859in}{2.553410in}}%
\pgfpathlineto{\pgfqpoint{4.752661in}{2.468958in}}%
\pgfpathlineto{\pgfqpoint{4.756610in}{2.359528in}}%
\pgfpathlineto{\pgfqpoint{4.759416in}{2.217681in}}%
\pgfpathlineto{\pgfqpoint{4.760596in}{2.043444in}}%
\pgfpathlineto{\pgfqpoint{4.759662in}{1.851779in}}%
\pgfpathlineto{\pgfqpoint{4.756587in}{1.667613in}}%
\pgfpathlineto{\pgfqpoint{4.751596in}{1.503428in}}%
\pgfpathlineto{\pgfqpoint{4.745410in}{1.374185in}}%
\pgfpathlineto{\pgfqpoint{4.738113in}{1.267479in}}%
\pgfpathlineto{\pgfqpoint{4.729621in}{1.175896in}}%
\pgfpathlineto{\pgfqpoint{4.720762in}{1.104428in}}%
\pgfpathlineto{\pgfqpoint{4.711045in}{1.043204in}}%
\pgfpathlineto{\pgfqpoint{4.700364in}{0.989829in}}%
\pgfpathlineto{\pgfqpoint{4.689055in}{0.944345in}}%
\pgfpathlineto{\pgfqpoint{4.676881in}{0.904394in}}%
\pgfpathlineto{\pgfqpoint{4.676095in}{0.902073in}}%
\pgfpathlineto{\pgfqpoint{4.676095in}{0.902073in}}%
\pgfusepath{stroke}%
\end{pgfscope}%
\begin{pgfscope}%
\pgfpathrectangle{\pgfqpoint{0.448634in}{0.402556in}}{\pgfqpoint{4.350661in}{2.489204in}} %
\pgfusepath{clip}%
\pgfsetrectcap%
\pgfsetroundjoin%
\pgfsetlinewidth{1.003750pt}%
\definecolor{currentstroke}{rgb}{0.580392,0.403922,0.741176}%
\pgfsetstrokecolor{currentstroke}%
\pgfsetdash{}{0pt}%
\pgfpathmoveto{\pgfqpoint{2.795520in}{1.982745in}}%
\pgfpathlineto{\pgfqpoint{2.781780in}{1.874357in}}%
\pgfpathlineto{\pgfqpoint{2.769351in}{1.758234in}}%
\pgfpathlineto{\pgfqpoint{2.758095in}{1.631942in}}%
\pgfpathlineto{\pgfqpoint{2.747786in}{1.490551in}}%
\pgfpathlineto{\pgfqpoint{2.738644in}{1.334082in}}%
\pgfpathlineto{\pgfqpoint{2.730580in}{1.157591in}}%
\pgfpathlineto{\pgfqpoint{2.723334in}{0.948663in}}%
\pgfpathlineto{\pgfqpoint{2.709783in}{0.530788in}}%
\pgfpathlineto{\pgfqpoint{2.705868in}{0.488716in}}%
\pgfpathlineto{\pgfqpoint{2.701769in}{0.464281in}}%
\pgfpathlineto{\pgfqpoint{2.697021in}{0.447744in}}%
\pgfpathlineto{\pgfqpoint{2.691859in}{0.436812in}}%
\pgfpathlineto{\pgfqpoint{2.686245in}{0.429229in}}%
\pgfpathlineto{\pgfqpoint{2.679348in}{0.423188in}}%
\pgfpathlineto{\pgfqpoint{2.669540in}{0.417856in}}%
\pgfpathlineto{\pgfqpoint{2.656987in}{0.413810in}}%
\pgfpathlineto{\pgfqpoint{2.637654in}{0.410337in}}%
\pgfpathlineto{\pgfqpoint{2.607297in}{0.407617in}}%
\pgfpathlineto{\pgfqpoint{2.555121in}{0.405574in}}%
\pgfpathlineto{\pgfqpoint{2.450714in}{0.404139in}}%
\pgfpathlineto{\pgfqpoint{2.176624in}{0.403275in}}%
\pgfpathlineto{\pgfqpoint{1.130290in}{0.402953in}}%
\pgfpathlineto{\pgfqpoint{0.516849in}{0.404175in}}%
\pgfpathlineto{\pgfqpoint{0.466848in}{0.405970in}}%
\pgfpathlineto{\pgfqpoint{0.456130in}{0.407931in}}%
\pgfpathlineto{\pgfqpoint{0.452340in}{0.410303in}}%
\pgfpathlineto{\pgfqpoint{0.450346in}{0.414662in}}%
\pgfpathlineto{\pgfqpoint{0.449266in}{0.424524in}}%
\pgfpathlineto{\pgfqpoint{0.448771in}{0.464344in}}%
\pgfpathlineto{\pgfqpoint{0.448640in}{0.850171in}}%
\pgfpathlineto{\pgfqpoint{0.448679in}{2.891318in}}%
\pgfpathlineto{\pgfqpoint{0.448679in}{2.891318in}}%
\pgfusepath{stroke}%
\end{pgfscope}%
\begin{pgfscope}%
\pgfpathrectangle{\pgfqpoint{0.448634in}{0.402556in}}{\pgfqpoint{4.350661in}{2.489204in}} %
\pgfusepath{clip}%
\pgfsetrectcap%
\pgfsetroundjoin%
\pgfsetlinewidth{1.003750pt}%
\definecolor{currentstroke}{rgb}{0.580392,0.403922,0.741176}%
\pgfsetstrokecolor{currentstroke}%
\pgfsetdash{}{0pt}%
\pgfpathmoveto{\pgfqpoint{3.428198in}{0.402586in}}%
\pgfpathlineto{\pgfqpoint{2.782130in}{0.403703in}}%
\pgfpathlineto{\pgfqpoint{2.753915in}{0.405679in}}%
\pgfpathlineto{\pgfqpoint{2.743337in}{0.408451in}}%
\pgfpathlineto{\pgfqpoint{2.737726in}{0.412196in}}%
\pgfpathlineto{\pgfqpoint{2.733676in}{0.418002in}}%
\pgfpathlineto{\pgfqpoint{2.730655in}{0.427314in}}%
\pgfpathlineto{\pgfqpoint{2.728393in}{0.442010in}}%
\pgfpathlineto{\pgfqpoint{2.726547in}{0.471800in}}%
\pgfpathlineto{\pgfqpoint{2.725218in}{0.534009in}}%
\pgfpathlineto{\pgfqpoint{2.725171in}{0.655978in}}%
\pgfpathlineto{\pgfqpoint{2.727378in}{0.832692in}}%
\pgfpathlineto{\pgfqpoint{2.732260in}{1.041709in}}%
\pgfpathlineto{\pgfqpoint{2.738852in}{1.223262in}}%
\pgfpathlineto{\pgfqpoint{2.747079in}{1.389771in}}%
\pgfpathlineto{\pgfqpoint{2.756610in}{1.538723in}}%
\pgfpathlineto{\pgfqpoint{2.768956in}{1.694893in}}%
\pgfpathlineto{\pgfqpoint{2.781229in}{1.816050in}}%
\pgfpathlineto{\pgfqpoint{2.794403in}{1.924530in}}%
\pgfpathlineto{\pgfqpoint{2.812739in}{2.054728in}}%
\pgfpathlineto{\pgfqpoint{2.828776in}{2.147518in}}%
\pgfpathlineto{\pgfqpoint{2.847385in}{2.242230in}}%
\pgfpathlineto{\pgfqpoint{2.895821in}{2.479705in}}%
\pgfpathlineto{\pgfqpoint{2.900207in}{2.516694in}}%
\pgfpathlineto{\pgfqpoint{2.901349in}{2.544034in}}%
\pgfpathlineto{\pgfqpoint{2.900294in}{2.566394in}}%
\pgfpathlineto{\pgfqpoint{2.897337in}{2.586004in}}%
\pgfpathlineto{\pgfqpoint{2.892838in}{2.602638in}}%
\pgfpathlineto{\pgfqpoint{2.886396in}{2.618410in}}%
\pgfpathlineto{\pgfqpoint{2.878059in}{2.632974in}}%
\pgfpathlineto{\pgfqpoint{2.868066in}{2.646105in}}%
\pgfpathlineto{\pgfqpoint{2.855051in}{2.659304in}}%
\pgfpathlineto{\pgfqpoint{2.840801in}{2.670720in}}%
\pgfpathlineto{\pgfqpoint{2.821823in}{2.682864in}}%
\pgfpathlineto{\pgfqpoint{2.799981in}{2.694029in}}%
\pgfpathlineto{\pgfqpoint{2.773366in}{2.704947in}}%
\pgfpathlineto{\pgfqpoint{2.742012in}{2.715268in}}%
\pgfpathlineto{\pgfqpoint{2.705983in}{2.724787in}}%
\pgfpathlineto{\pgfqpoint{2.663200in}{2.733812in}}%
\pgfpathlineto{\pgfqpoint{2.611535in}{2.742381in}}%
\pgfpathlineto{\pgfqpoint{2.551002in}{2.750092in}}%
\pgfpathlineto{\pgfqpoint{2.481632in}{2.756684in}}%
\pgfpathlineto{\pgfqpoint{2.399112in}{2.762202in}}%
\pgfpathlineto{\pgfqpoint{2.309985in}{2.765887in}}%
\pgfpathlineto{\pgfqpoint{2.188184in}{2.768098in}}%
\pgfpathlineto{\pgfqpoint{2.081595in}{2.767620in}}%
\pgfpathlineto{\pgfqpoint{1.968506in}{2.764841in}}%
\pgfpathlineto{\pgfqpoint{1.864180in}{2.759919in}}%
\pgfpathlineto{\pgfqpoint{1.757786in}{2.752594in}}%
\pgfpathlineto{\pgfqpoint{1.671087in}{2.744172in}}%
\pgfpathlineto{\pgfqpoint{1.591075in}{2.734194in}}%
\pgfpathlineto{\pgfqpoint{1.502689in}{2.720719in}}%
\pgfpathlineto{\pgfqpoint{1.427655in}{2.706083in}}%
\pgfpathlineto{\pgfqpoint{1.372350in}{2.692544in}}%
\pgfpathlineto{\pgfqpoint{1.321734in}{2.677921in}}%
\pgfpathlineto{\pgfqpoint{1.273765in}{2.661664in}}%
\pgfpathlineto{\pgfqpoint{1.230567in}{2.644672in}}%
\pgfpathlineto{\pgfqpoint{1.192197in}{2.627106in}}%
\pgfpathlineto{\pgfqpoint{1.156620in}{2.608403in}}%
\pgfpathlineto{\pgfqpoint{1.123890in}{2.588717in}}%
\pgfpathlineto{\pgfqpoint{1.095883in}{2.569568in}}%
\pgfpathlineto{\pgfqpoint{1.063936in}{2.543701in}}%
\pgfpathlineto{\pgfqpoint{1.038216in}{2.520733in}}%
\pgfpathlineto{\pgfqpoint{1.013766in}{2.496017in}}%
\pgfpathlineto{\pgfqpoint{0.990704in}{2.469611in}}%
\pgfpathlineto{\pgfqpoint{0.969124in}{2.441613in}}%
\pgfpathlineto{\pgfqpoint{0.949082in}{2.412155in}}%
\pgfpathlineto{\pgfqpoint{0.930603in}{2.381387in}}%
\pgfpathlineto{\pgfqpoint{0.906555in}{2.334053in}}%
\pgfpathlineto{\pgfqpoint{0.889924in}{2.296263in}}%
\pgfpathlineto{\pgfqpoint{0.874240in}{2.255214in}}%
\pgfpathlineto{\pgfqpoint{0.859667in}{2.210962in}}%
\pgfpathlineto{\pgfqpoint{0.846985in}{2.165955in}}%
\pgfpathlineto{\pgfqpoint{0.839632in}{2.134716in}}%
\pgfpathlineto{\pgfqpoint{0.828237in}{2.081533in}}%
\pgfpathlineto{\pgfqpoint{0.817866in}{2.022987in}}%
\pgfpathlineto{\pgfqpoint{0.810783in}{1.971353in}}%
\pgfpathlineto{\pgfqpoint{0.802845in}{1.902253in}}%
\pgfpathlineto{\pgfqpoint{0.796553in}{1.827928in}}%
\pgfpathlineto{\pgfqpoint{0.791695in}{1.743481in}}%
\pgfpathlineto{\pgfqpoint{0.787772in}{1.621596in}}%
\pgfpathlineto{\pgfqpoint{0.785406in}{1.522065in}}%
\pgfpathlineto{\pgfqpoint{0.785406in}{1.522065in}}%
\pgfusepath{stroke}%
\end{pgfscope}%
\begin{pgfscope}%
\pgfpathrectangle{\pgfqpoint{0.448634in}{0.402556in}}{\pgfqpoint{4.350661in}{2.489204in}} %
\pgfusepath{clip}%
\pgfsetrectcap%
\pgfsetroundjoin%
\pgfsetlinewidth{1.003750pt}%
\definecolor{currentstroke}{rgb}{0.549020,0.337255,0.294118}%
\pgfsetstrokecolor{currentstroke}%
\pgfsetdash{}{0pt}%
\pgfpathmoveto{\pgfqpoint{1.127319in}{2.572073in}}%
\pgfpathlineto{\pgfqpoint{1.159575in}{2.592758in}}%
\pgfpathlineto{\pgfqpoint{1.192763in}{2.611414in}}%
\pgfpathlineto{\pgfqpoint{1.228725in}{2.629126in}}%
\pgfpathlineto{\pgfqpoint{1.267413in}{2.645758in}}%
\pgfpathlineto{\pgfqpoint{1.310846in}{2.661945in}}%
\pgfpathlineto{\pgfqpoint{1.356920in}{2.676740in}}%
\pgfpathlineto{\pgfqpoint{1.407679in}{2.690702in}}%
\pgfpathlineto{\pgfqpoint{1.463094in}{2.703640in}}%
\pgfpathlineto{\pgfqpoint{1.525272in}{2.715812in}}%
\pgfpathlineto{\pgfqpoint{1.594198in}{2.726937in}}%
\pgfpathlineto{\pgfqpoint{1.669843in}{2.736808in}}%
\pgfpathlineto{\pgfqpoint{1.752172in}{2.745271in}}%
\pgfpathlineto{\pgfqpoint{1.843325in}{2.752344in}}%
\pgfpathlineto{\pgfqpoint{1.941103in}{2.757656in}}%
\pgfpathlineto{\pgfqpoint{2.043301in}{2.760987in}}%
\pgfpathlineto{\pgfqpoint{2.147710in}{2.762199in}}%
\pgfpathlineto{\pgfqpoint{2.249945in}{2.761215in}}%
\pgfpathlineto{\pgfqpoint{2.345620in}{2.758145in}}%
\pgfpathlineto{\pgfqpoint{2.432525in}{2.753210in}}%
\pgfpathlineto{\pgfqpoint{2.508450in}{2.746766in}}%
\pgfpathlineto{\pgfqpoint{2.573368in}{2.739156in}}%
\pgfpathlineto{\pgfqpoint{2.629410in}{2.730451in}}%
\pgfpathlineto{\pgfqpoint{2.676543in}{2.720985in}}%
\pgfpathlineto{\pgfqpoint{2.716874in}{2.710666in}}%
\pgfpathlineto{\pgfqpoint{2.750365in}{2.699848in}}%
\pgfpathlineto{\pgfqpoint{2.779059in}{2.688192in}}%
\pgfpathlineto{\pgfqpoint{2.802882in}{2.676004in}}%
\pgfpathlineto{\pgfqpoint{2.821841in}{2.663819in}}%
\pgfpathlineto{\pgfqpoint{2.837815in}{2.650886in}}%
\pgfpathlineto{\pgfqpoint{2.850735in}{2.637564in}}%
\pgfpathlineto{\pgfqpoint{2.860694in}{2.624398in}}%
\pgfpathlineto{\pgfqpoint{2.869084in}{2.609873in}}%
\pgfpathlineto{\pgfqpoint{2.875698in}{2.594192in}}%
\pgfpathlineto{\pgfqpoint{2.881035in}{2.575255in}}%
\pgfpathlineto{\pgfqpoint{2.884200in}{2.555685in}}%
\pgfpathlineto{\pgfqpoint{2.885619in}{2.533351in}}%
\pgfpathlineto{\pgfqpoint{2.885038in}{2.505987in}}%
\pgfpathlineto{\pgfqpoint{2.882112in}{2.473807in}}%
\pgfpathlineto{\pgfqpoint{2.875657in}{2.429620in}}%
\pgfpathlineto{\pgfqpoint{2.863489in}{2.363873in}}%
\pgfpathlineto{\pgfqpoint{2.821102in}{2.142619in}}%
\pgfpathlineto{\pgfqpoint{2.804859in}{2.042271in}}%
\pgfpathlineto{\pgfqpoint{2.790421in}{1.939040in}}%
\pgfpathlineto{\pgfqpoint{2.777207in}{1.828054in}}%
\pgfpathlineto{\pgfqpoint{2.765338in}{1.709349in}}%
\pgfpathlineto{\pgfqpoint{2.754471in}{1.578010in}}%
\pgfpathlineto{\pgfqpoint{2.744640in}{1.431580in}}%
\pgfpathlineto{\pgfqpoint{2.735914in}{1.267598in}}%
\pgfpathlineto{\pgfqpoint{2.728277in}{1.081114in}}%
\pgfpathlineto{\pgfqpoint{2.721436in}{0.857223in}}%
\pgfpathlineto{\pgfqpoint{2.711961in}{0.541290in}}%
\pgfpathlineto{\pgfqpoint{2.708250in}{0.491694in}}%
\pgfpathlineto{\pgfqpoint{2.703951in}{0.462246in}}%
\pgfpathlineto{\pgfqpoint{2.699504in}{0.445599in}}%
\pgfpathlineto{\pgfqpoint{2.694517in}{0.434563in}}%
\pgfpathlineto{\pgfqpoint{2.688941in}{0.426947in}}%
\pgfpathlineto{\pgfqpoint{2.681980in}{0.421009in}}%
\pgfpathlineto{\pgfqpoint{2.672063in}{0.415948in}}%
\pgfpathlineto{\pgfqpoint{2.659429in}{0.412247in}}%
\pgfpathlineto{\pgfqpoint{2.640044in}{0.409163in}}%
\pgfpathlineto{\pgfqpoint{2.607489in}{0.406692in}}%
\pgfpathlineto{\pgfqpoint{2.548778in}{0.404894in}}%
\pgfpathlineto{\pgfqpoint{2.422614in}{0.403701in}}%
\pgfpathlineto{\pgfqpoint{2.026705in}{0.403016in}}%
\pgfpathlineto{\pgfqpoint{0.623617in}{0.403253in}}%
\pgfpathlineto{\pgfqpoint{0.477879in}{0.404742in}}%
\pgfpathlineto{\pgfqpoint{0.458367in}{0.406382in}}%
\pgfpathlineto{\pgfqpoint{0.452303in}{0.408938in}}%
\pgfpathlineto{\pgfqpoint{0.450213in}{0.413215in}}%
\pgfpathlineto{\pgfqpoint{0.449165in}{0.423081in}}%
\pgfpathlineto{\pgfqpoint{0.448735in}{0.465392in}}%
\pgfpathlineto{\pgfqpoint{0.448637in}{0.983147in}}%
\pgfpathlineto{\pgfqpoint{0.448652in}{2.889877in}}%
\pgfpathlineto{\pgfqpoint{0.448652in}{2.889877in}}%
\pgfusepath{stroke}%
\end{pgfscope}%
\begin{pgfscope}%
\pgfpathrectangle{\pgfqpoint{0.448634in}{0.402556in}}{\pgfqpoint{4.350661in}{2.489204in}} %
\pgfusepath{clip}%
\pgfsetrectcap%
\pgfsetroundjoin%
\pgfsetlinewidth{1.003750pt}%
\definecolor{currentstroke}{rgb}{0.549020,0.337255,0.294118}%
\pgfsetstrokecolor{currentstroke}%
\pgfsetdash{}{0pt}%
\pgfpathmoveto{\pgfqpoint{0.448634in}{2.896245in}}%
\pgfpathlineto{\pgfqpoint{0.448593in}{0.407043in}}%
\pgfpathlineto{\pgfqpoint{0.448593in}{0.407043in}}%
\pgfusepath{stroke}%
\end{pgfscope}%
\begin{pgfscope}%
\pgfpathrectangle{\pgfqpoint{0.448634in}{0.402556in}}{\pgfqpoint{4.350661in}{2.489204in}} %
\pgfusepath{clip}%
\pgfsetrectcap%
\pgfsetroundjoin%
\pgfsetlinewidth{1.003750pt}%
\definecolor{currentstroke}{rgb}{0.549020,0.337255,0.294118}%
\pgfsetstrokecolor{currentstroke}%
\pgfsetdash{}{0pt}%
\pgfpathmoveto{\pgfqpoint{0.576842in}{1.760843in}}%
\pgfpathlineto{\pgfqpoint{0.569384in}{1.840036in}}%
\pgfpathlineto{\pgfqpoint{0.563200in}{1.929365in}}%
\pgfpathlineto{\pgfqpoint{0.558585in}{2.028790in}}%
\pgfpathlineto{\pgfqpoint{0.555979in}{2.133292in}}%
\pgfpathlineto{\pgfqpoint{0.555560in}{2.237835in}}%
\pgfpathlineto{\pgfqpoint{0.557366in}{2.337378in}}%
\pgfpathlineto{\pgfqpoint{0.561092in}{2.424393in}}%
\pgfpathlineto{\pgfqpoint{0.566399in}{2.498818in}}%
\pgfpathlineto{\pgfqpoint{0.572906in}{2.560596in}}%
\pgfpathlineto{\pgfqpoint{0.580456in}{2.612145in}}%
\pgfpathlineto{\pgfqpoint{0.589085in}{2.655842in}}%
\pgfpathlineto{\pgfqpoint{0.598406in}{2.691615in}}%
\pgfpathlineto{\pgfqpoint{0.608615in}{2.721782in}}%
\pgfpathlineto{\pgfqpoint{0.619244in}{2.746302in}}%
\pgfpathlineto{\pgfqpoint{0.630822in}{2.767362in}}%
\pgfpathlineto{\pgfqpoint{0.642982in}{2.784905in}}%
\pgfpathlineto{\pgfqpoint{0.656822in}{2.800731in}}%
\pgfpathlineto{\pgfqpoint{0.672207in}{2.814565in}}%
\pgfpathlineto{\pgfqpoint{0.688865in}{2.826316in}}%
\pgfpathlineto{\pgfqpoint{0.706474in}{2.836088in}}%
\pgfpathlineto{\pgfqpoint{0.726817in}{2.844885in}}%
\pgfpathlineto{\pgfqpoint{0.751880in}{2.853211in}}%
\pgfpathlineto{\pgfqpoint{0.781646in}{2.860554in}}%
\pgfpathlineto{\pgfqpoint{0.818182in}{2.867058in}}%
\pgfpathlineto{\pgfqpoint{0.863595in}{2.872688in}}%
\pgfpathlineto{\pgfqpoint{0.922175in}{2.877520in}}%
\pgfpathlineto{\pgfqpoint{1.000405in}{2.881568in}}%
\pgfpathlineto{\pgfqpoint{1.111308in}{2.884882in}}%
\pgfpathlineto{\pgfqpoint{1.274443in}{2.887368in}}%
\pgfpathlineto{\pgfqpoint{1.552880in}{2.889263in}}%
\pgfpathlineto{\pgfqpoint{2.107588in}{2.890457in}}%
\pgfpathlineto{\pgfqpoint{3.343175in}{2.890573in}}%
\pgfpathlineto{\pgfqpoint{4.043630in}{2.888941in}}%
\pgfpathlineto{\pgfqpoint{4.289431in}{2.886404in}}%
\pgfpathlineto{\pgfqpoint{4.413389in}{2.883093in}}%
\pgfpathlineto{\pgfqpoint{4.489439in}{2.878997in}}%
\pgfpathlineto{\pgfqpoint{4.541466in}{2.874080in}}%
\pgfpathlineto{\pgfqpoint{4.578115in}{2.868469in}}%
\pgfpathlineto{\pgfqpoint{4.605833in}{2.862091in}}%
\pgfpathlineto{\pgfqpoint{4.626740in}{2.855243in}}%
\pgfpathlineto{\pgfqpoint{4.644939in}{2.847015in}}%
\pgfpathlineto{\pgfqpoint{4.660254in}{2.837585in}}%
\pgfpathlineto{\pgfqpoint{4.672636in}{2.827463in}}%
\pgfpathlineto{\pgfqpoint{4.683762in}{2.815585in}}%
\pgfpathlineto{\pgfqpoint{4.693416in}{2.802126in}}%
\pgfpathlineto{\pgfqpoint{4.702749in}{2.785334in}}%
\pgfpathlineto{\pgfqpoint{4.711285in}{2.765184in}}%
\pgfpathlineto{\pgfqpoint{4.719489in}{2.739473in}}%
\pgfpathlineto{\pgfqpoint{4.726299in}{2.710647in}}%
\pgfpathlineto{\pgfqpoint{4.733264in}{2.671632in}}%
\pgfpathlineto{\pgfqpoint{4.739608in}{2.622385in}}%
\pgfpathlineto{\pgfqpoint{4.745239in}{2.560493in}}%
\pgfpathlineto{\pgfqpoint{4.750167in}{2.481040in}}%
\pgfpathlineto{\pgfqpoint{4.754369in}{2.376607in}}%
\pgfpathlineto{\pgfqpoint{4.757445in}{2.242238in}}%
\pgfpathlineto{\pgfqpoint{4.758979in}{2.075472in}}%
\pgfpathlineto{\pgfqpoint{4.758448in}{1.888784in}}%
\pgfpathlineto{\pgfqpoint{4.755757in}{1.707099in}}%
\pgfpathlineto{\pgfqpoint{4.750927in}{1.532946in}}%
\pgfpathlineto{\pgfqpoint{4.744787in}{1.398715in}}%
\pgfpathlineto{\pgfqpoint{4.737576in}{1.289504in}}%
\pgfpathlineto{\pgfqpoint{4.728716in}{1.190458in}}%
\pgfpathlineto{\pgfqpoint{4.719654in}{1.116510in}}%
\pgfpathlineto{\pgfqpoint{4.710038in}{1.055265in}}%
\pgfpathlineto{\pgfqpoint{4.699505in}{1.001850in}}%
\pgfpathlineto{\pgfqpoint{4.689042in}{0.958679in}}%
\pgfpathlineto{\pgfqpoint{4.677221in}{0.918589in}}%
\pgfpathlineto{\pgfqpoint{4.664036in}{0.881738in}}%
\pgfpathlineto{\pgfqpoint{4.650585in}{0.850480in}}%
\pgfpathlineto{\pgfqpoint{4.636304in}{0.822559in}}%
\pgfpathlineto{\pgfqpoint{4.620208in}{0.795964in}}%
\pgfpathlineto{\pgfqpoint{4.603641in}{0.772891in}}%
\pgfpathlineto{\pgfqpoint{4.585489in}{0.751436in}}%
\pgfpathlineto{\pgfqpoint{4.565874in}{0.731738in}}%
\pgfpathlineto{\pgfqpoint{4.544964in}{0.713869in}}%
\pgfpathlineto{\pgfqpoint{4.522958in}{0.697814in}}%
\pgfpathlineto{\pgfqpoint{4.496157in}{0.681281in}}%
\pgfpathlineto{\pgfqpoint{4.470398in}{0.667943in}}%
\pgfpathlineto{\pgfqpoint{4.439961in}{0.654500in}}%
\pgfpathlineto{\pgfqpoint{4.406841in}{0.642273in}}%
\pgfpathlineto{\pgfqpoint{4.369009in}{0.630740in}}%
\pgfpathlineto{\pgfqpoint{4.326489in}{0.620218in}}%
\pgfpathlineto{\pgfqpoint{4.279327in}{0.610941in}}%
\pgfpathlineto{\pgfqpoint{4.227576in}{0.603077in}}%
\pgfpathlineto{\pgfqpoint{4.173450in}{0.597055in}}%
\pgfpathlineto{\pgfqpoint{4.110511in}{0.592195in}}%
\pgfpathlineto{\pgfqpoint{4.047471in}{0.589529in}}%
\pgfpathlineto{\pgfqpoint{3.977867in}{0.588616in}}%
\pgfpathlineto{\pgfqpoint{3.906092in}{0.589925in}}%
\pgfpathlineto{\pgfqpoint{3.834376in}{0.593488in}}%
\pgfpathlineto{\pgfqpoint{3.767119in}{0.599059in}}%
\pgfpathlineto{\pgfqpoint{3.704364in}{0.606384in}}%
\pgfpathlineto{\pgfqpoint{3.678517in}{0.610507in}}%
\pgfpathlineto{\pgfqpoint{3.620439in}{0.620497in}}%
\pgfpathlineto{\pgfqpoint{3.586320in}{0.628204in}}%
\pgfpathlineto{\pgfqpoint{3.495242in}{0.652427in}}%
\pgfpathlineto{\pgfqpoint{3.451529in}{0.667582in}}%
\pgfpathlineto{\pgfqpoint{3.408539in}{0.685219in}}%
\pgfpathlineto{\pgfqpoint{3.374595in}{0.702000in}}%
\pgfpathlineto{\pgfqpoint{3.345408in}{0.718681in}}%
\pgfpathlineto{\pgfqpoint{3.315237in}{0.738519in}}%
\pgfpathlineto{\pgfqpoint{3.288128in}{0.759288in}}%
\pgfpathlineto{\pgfqpoint{3.264005in}{0.780549in}}%
\pgfpathlineto{\pgfqpoint{3.241209in}{0.803647in}}%
\pgfpathlineto{\pgfqpoint{3.219895in}{0.828528in}}%
\pgfpathlineto{\pgfqpoint{3.200190in}{0.855090in}}%
\pgfpathlineto{\pgfqpoint{3.182178in}{0.883181in}}%
\pgfpathlineto{\pgfqpoint{3.165907in}{0.912631in}}%
\pgfpathlineto{\pgfqpoint{3.150352in}{0.945446in}}%
\pgfpathlineto{\pgfqpoint{3.136683in}{0.979343in}}%
\pgfpathlineto{\pgfqpoint{3.124073in}{1.016458in}}%
\pgfpathlineto{\pgfqpoint{3.112834in}{1.056767in}}%
\pgfpathlineto{\pgfqpoint{3.103046in}{1.100144in}}%
\pgfpathlineto{\pgfqpoint{3.095344in}{1.144069in}}%
\pgfpathlineto{\pgfqpoint{3.089209in}{1.190836in}}%
\pgfpathlineto{\pgfqpoint{3.084596in}{1.242837in}}%
\pgfpathlineto{\pgfqpoint{3.082137in}{1.295030in}}%
\pgfpathlineto{\pgfqpoint{3.081687in}{1.349785in}}%
\pgfpathlineto{\pgfqpoint{3.083451in}{1.406997in}}%
\pgfpathlineto{\pgfqpoint{3.087182in}{1.461588in}}%
\pgfpathlineto{\pgfqpoint{3.093486in}{1.520886in}}%
\pgfpathlineto{\pgfqpoint{3.101824in}{1.577332in}}%
\pgfpathlineto{\pgfqpoint{3.111929in}{1.630855in}}%
\pgfpathlineto{\pgfqpoint{3.124690in}{1.686207in}}%
\pgfpathlineto{\pgfqpoint{3.139177in}{1.738394in}}%
\pgfpathlineto{\pgfqpoint{3.155144in}{1.787365in}}%
\pgfpathlineto{\pgfqpoint{3.172352in}{1.833083in}}%
\pgfpathlineto{\pgfqpoint{3.191617in}{1.877715in}}%
\pgfpathlineto{\pgfqpoint{3.214025in}{1.923260in}}%
\pgfpathlineto{\pgfqpoint{3.236213in}{1.963156in}}%
\pgfpathlineto{\pgfqpoint{3.260177in}{2.001683in}}%
\pgfpathlineto{\pgfqpoint{3.285813in}{2.038775in}}%
\pgfpathlineto{\pgfqpoint{3.314414in}{2.076284in}}%
\pgfpathlineto{\pgfqpoint{3.348943in}{2.117710in}}%
\pgfpathlineto{\pgfqpoint{3.417132in}{2.198021in}}%
\pgfpathlineto{\pgfqpoint{3.426053in}{2.212127in}}%
\pgfpathlineto{\pgfqpoint{3.430798in}{2.223296in}}%
\pgfpathlineto{\pgfqpoint{3.432034in}{2.230602in}}%
\pgfpathlineto{\pgfqpoint{3.430773in}{2.237855in}}%
\pgfpathlineto{\pgfqpoint{3.426622in}{2.243525in}}%
\pgfpathlineto{\pgfqpoint{3.420909in}{2.247083in}}%
\pgfpathlineto{\pgfqpoint{3.412501in}{2.249582in}}%
\pgfpathlineto{\pgfqpoint{3.399499in}{2.250688in}}%
\pgfpathlineto{\pgfqpoint{3.384306in}{2.249670in}}%
\pgfpathlineto{\pgfqpoint{3.364986in}{2.246097in}}%
\pgfpathlineto{\pgfqpoint{3.341804in}{2.239342in}}%
\pgfpathlineto{\pgfqpoint{3.317110in}{2.229681in}}%
\pgfpathlineto{\pgfqpoint{3.291105in}{2.216985in}}%
\pgfpathlineto{\pgfqpoint{3.265928in}{2.202261in}}%
\pgfpathlineto{\pgfqpoint{3.239805in}{2.184361in}}%
\pgfpathlineto{\pgfqpoint{3.214776in}{2.164518in}}%
\pgfpathlineto{\pgfqpoint{3.190901in}{2.142893in}}%
\pgfpathlineto{\pgfqpoint{3.166657in}{2.117912in}}%
\pgfpathlineto{\pgfqpoint{3.143836in}{2.091233in}}%
\pgfpathlineto{\pgfqpoint{3.121079in}{2.061107in}}%
\pgfpathlineto{\pgfqpoint{3.099952in}{2.029463in}}%
\pgfpathlineto{\pgfqpoint{3.079251in}{1.994405in}}%
\pgfpathlineto{\pgfqpoint{3.059218in}{1.955915in}}%
\pgfpathlineto{\pgfqpoint{3.040059in}{1.914015in}}%
\pgfpathlineto{\pgfqpoint{3.022809in}{1.871041in}}%
\pgfpathlineto{\pgfqpoint{3.005790in}{1.822536in}}%
\pgfpathlineto{\pgfqpoint{2.990067in}{1.770819in}}%
\pgfpathlineto{\pgfqpoint{2.975708in}{1.715979in}}%
\pgfpathlineto{\pgfqpoint{2.962284in}{1.655680in}}%
\pgfpathlineto{\pgfqpoint{2.950496in}{1.592386in}}%
\pgfpathlineto{\pgfqpoint{2.940383in}{1.526185in}}%
\pgfpathlineto{\pgfqpoint{2.931745in}{1.454681in}}%
\pgfpathlineto{\pgfqpoint{2.925082in}{1.380399in}}%
\pgfpathlineto{\pgfqpoint{2.920647in}{1.305899in}}%
\pgfpathlineto{\pgfqpoint{2.918444in}{1.231270in}}%
\pgfpathlineto{\pgfqpoint{2.918545in}{1.159087in}}%
\pgfpathlineto{\pgfqpoint{2.920787in}{1.091931in}}%
\pgfpathlineto{\pgfqpoint{2.925177in}{1.027412in}}%
\pgfpathlineto{\pgfqpoint{2.931192in}{0.970580in}}%
\pgfpathlineto{\pgfqpoint{2.938760in}{0.919034in}}%
\pgfpathlineto{\pgfqpoint{2.947651in}{0.872852in}}%
\pgfpathlineto{\pgfqpoint{2.958213in}{0.829714in}}%
\pgfpathlineto{\pgfqpoint{2.969670in}{0.792114in}}%
\pgfpathlineto{\pgfqpoint{2.982463in}{0.757774in}}%
\pgfpathlineto{\pgfqpoint{2.996425in}{0.726812in}}%
\pgfpathlineto{\pgfqpoint{3.011299in}{0.699300in}}%
\pgfpathlineto{\pgfqpoint{3.026739in}{0.675225in}}%
\pgfpathlineto{\pgfqpoint{3.043828in}{0.652656in}}%
\pgfpathlineto{\pgfqpoint{3.062495in}{0.631788in}}%
\pgfpathlineto{\pgfqpoint{3.082602in}{0.612753in}}%
\pgfpathlineto{\pgfqpoint{3.103961in}{0.595592in}}%
\pgfpathlineto{\pgfqpoint{3.128268in}{0.579069in}}%
\pgfpathlineto{\pgfqpoint{3.153537in}{0.564554in}}%
\pgfpathlineto{\pgfqpoint{3.181571in}{0.550952in}}%
\pgfpathlineto{\pgfqpoint{3.214371in}{0.537647in}}%
\pgfpathlineto{\pgfqpoint{3.249846in}{0.525712in}}%
\pgfpathlineto{\pgfqpoint{3.290011in}{0.514571in}}%
\pgfpathlineto{\pgfqpoint{3.334821in}{0.504423in}}%
\pgfpathlineto{\pgfqpoint{3.386372in}{0.494999in}}%
\pgfpathlineto{\pgfqpoint{3.446798in}{0.486257in}}%
\pgfpathlineto{\pgfqpoint{3.518243in}{0.478282in}}%
\pgfpathlineto{\pgfqpoint{3.600685in}{0.471409in}}%
\pgfpathlineto{\pgfqpoint{3.696269in}{0.465713in}}%
\pgfpathlineto{\pgfqpoint{3.807144in}{0.461369in}}%
\pgfpathlineto{\pgfqpoint{3.933291in}{0.458719in}}%
\pgfpathlineto{\pgfqpoint{4.063809in}{0.458211in}}%
\pgfpathlineto{\pgfqpoint{4.187792in}{0.459914in}}%
\pgfpathlineto{\pgfqpoint{4.294335in}{0.463522in}}%
\pgfpathlineto{\pgfqpoint{4.381234in}{0.468575in}}%
\pgfpathlineto{\pgfqpoint{4.450636in}{0.474702in}}%
\pgfpathlineto{\pgfqpoint{4.506850in}{0.481800in}}%
\pgfpathlineto{\pgfqpoint{4.552009in}{0.489659in}}%
\pgfpathlineto{\pgfqpoint{4.588239in}{0.498116in}}%
\pgfpathlineto{\pgfqpoint{4.617656in}{0.507111in}}%
\pgfpathlineto{\pgfqpoint{4.642329in}{0.516843in}}%
\pgfpathlineto{\pgfqpoint{4.664194in}{0.527941in}}%
\pgfpathlineto{\pgfqpoint{4.681238in}{0.538946in}}%
\pgfpathlineto{\pgfqpoint{4.697164in}{0.551955in}}%
\pgfpathlineto{\pgfqpoint{4.710076in}{0.565291in}}%
\pgfpathlineto{\pgfqpoint{4.721578in}{0.580220in}}%
\pgfpathlineto{\pgfqpoint{4.731557in}{0.596522in}}%
\pgfpathlineto{\pgfqpoint{4.741000in}{0.616135in}}%
\pgfpathlineto{\pgfqpoint{4.749521in}{0.639028in}}%
\pgfpathlineto{\pgfqpoint{4.757522in}{0.667451in}}%
\pgfpathlineto{\pgfqpoint{4.764572in}{0.701346in}}%
\pgfpathlineto{\pgfqpoint{4.770840in}{0.743044in}}%
\pgfpathlineto{\pgfqpoint{4.776327in}{0.794935in}}%
\pgfpathlineto{\pgfqpoint{4.781278in}{0.864399in}}%
\pgfpathlineto{\pgfqpoint{4.785468in}{0.956372in}}%
\pgfpathlineto{\pgfqpoint{4.789000in}{1.085746in}}%
\pgfpathlineto{\pgfqpoint{4.791852in}{1.277386in}}%
\pgfpathlineto{\pgfqpoint{4.793959in}{1.581059in}}%
\pgfpathlineto{\pgfqpoint{4.794962in}{2.071430in}}%
\pgfpathlineto{\pgfqpoint{4.793967in}{2.559312in}}%
\pgfpathlineto{\pgfqpoint{4.791733in}{2.745982in}}%
\pgfpathlineto{\pgfqpoint{4.788955in}{2.818092in}}%
\pgfpathlineto{\pgfqpoint{4.785731in}{2.850228in}}%
\pgfpathlineto{\pgfqpoint{4.781878in}{2.867058in}}%
\pgfpathlineto{\pgfqpoint{4.777743in}{2.875781in}}%
\pgfpathlineto{\pgfqpoint{4.773096in}{2.880983in}}%
\pgfpathlineto{\pgfqpoint{4.767362in}{2.884504in}}%
\pgfpathlineto{\pgfqpoint{4.756852in}{2.887622in}}%
\pgfpathlineto{\pgfqpoint{4.739547in}{2.889639in}}%
\pgfpathlineto{\pgfqpoint{4.704761in}{2.890882in}}%
\pgfpathlineto{\pgfqpoint{4.602523in}{2.891538in}}%
\pgfpathlineto{\pgfqpoint{3.952099in}{2.891742in}}%
\pgfpathlineto{\pgfqpoint{0.617319in}{2.890753in}}%
\pgfpathlineto{\pgfqpoint{0.549909in}{2.888858in}}%
\pgfpathlineto{\pgfqpoint{0.521734in}{2.886179in}}%
\pgfpathlineto{\pgfqpoint{0.504665in}{2.882389in}}%
\pgfpathlineto{\pgfqpoint{0.494500in}{2.878010in}}%
\pgfpathlineto{\pgfqpoint{0.487179in}{2.872665in}}%
\pgfpathlineto{\pgfqpoint{0.481151in}{2.865517in}}%
\pgfpathlineto{\pgfqpoint{0.475663in}{2.854802in}}%
\pgfpathlineto{\pgfqpoint{0.471318in}{2.840735in}}%
\pgfpathlineto{\pgfqpoint{0.467301in}{2.818821in}}%
\pgfpathlineto{\pgfqpoint{0.463927in}{2.786698in}}%
\pgfpathlineto{\pgfqpoint{0.460918in}{2.734542in}}%
\pgfpathlineto{\pgfqpoint{0.458363in}{2.647471in}}%
\pgfpathlineto{\pgfqpoint{0.456575in}{2.523029in}}%
\pgfpathlineto{\pgfqpoint{0.456575in}{2.523029in}}%
\pgfusepath{stroke}%
\end{pgfscope}%
\begin{pgfscope}%
\pgfpathrectangle{\pgfqpoint{0.448634in}{0.402556in}}{\pgfqpoint{4.350661in}{2.489204in}} %
\pgfusepath{clip}%
\pgfsetrectcap%
\pgfsetroundjoin%
\pgfsetlinewidth{1.003750pt}%
\definecolor{currentstroke}{rgb}{0.549020,0.337255,0.294118}%
\pgfsetstrokecolor{currentstroke}%
\pgfsetdash{}{0pt}%
\pgfpathmoveto{\pgfqpoint{0.456424in}{1.370135in}}%
\pgfpathlineto{\pgfqpoint{0.459610in}{1.118754in}}%
\pgfpathlineto{\pgfqpoint{0.463695in}{0.962006in}}%
\pgfpathlineto{\pgfqpoint{0.468519in}{0.857608in}}%
\pgfpathlineto{\pgfqpoint{0.474082in}{0.783209in}}%
\pgfpathlineto{\pgfqpoint{0.480226in}{0.728905in}}%
\pgfpathlineto{\pgfqpoint{0.486971in}{0.687305in}}%
\pgfpathlineto{\pgfqpoint{0.494537in}{0.653557in}}%
\pgfpathlineto{\pgfqpoint{0.503108in}{0.625354in}}%
\pgfpathlineto{\pgfqpoint{0.512193in}{0.602748in}}%
\pgfpathlineto{\pgfqpoint{0.522201in}{0.583506in}}%
\pgfpathlineto{\pgfqpoint{0.534109in}{0.565742in}}%
\pgfpathlineto{\pgfqpoint{0.546264in}{0.551506in}}%
\pgfpathlineto{\pgfqpoint{0.559729in}{0.538906in}}%
\pgfpathlineto{\pgfqpoint{0.576131in}{0.526693in}}%
\pgfpathlineto{\pgfqpoint{0.595484in}{0.515350in}}%
\pgfpathlineto{\pgfqpoint{0.617682in}{0.505146in}}%
\pgfpathlineto{\pgfqpoint{0.642569in}{0.496153in}}%
\pgfpathlineto{\pgfqpoint{0.672127in}{0.487777in}}%
\pgfpathlineto{\pgfqpoint{0.708444in}{0.479823in}}%
\pgfpathlineto{\pgfqpoint{0.753651in}{0.472325in}}%
\pgfpathlineto{\pgfqpoint{0.807719in}{0.465660in}}%
\pgfpathlineto{\pgfqpoint{0.877117in}{0.459475in}}%
\pgfpathlineto{\pgfqpoint{0.961829in}{0.454230in}}%
\pgfpathlineto{\pgfqpoint{1.068353in}{0.449916in}}%
\pgfpathlineto{\pgfqpoint{1.201020in}{0.446839in}}%
\pgfpathlineto{\pgfqpoint{1.357638in}{0.445481in}}%
\pgfpathlineto{\pgfqpoint{1.525136in}{0.446232in}}%
\pgfpathlineto{\pgfqpoint{1.686089in}{0.449142in}}%
\pgfpathlineto{\pgfqpoint{1.823075in}{0.453747in}}%
\pgfpathlineto{\pgfqpoint{1.938246in}{0.459765in}}%
\pgfpathlineto{\pgfqpoint{2.031583in}{0.466759in}}%
\pgfpathlineto{\pgfqpoint{2.109581in}{0.474745in}}%
\pgfpathlineto{\pgfqpoint{2.174385in}{0.483535in}}%
\pgfpathlineto{\pgfqpoint{2.228140in}{0.492940in}}%
\pgfpathlineto{\pgfqpoint{2.275120in}{0.503357in}}%
\pgfpathlineto{\pgfqpoint{2.315283in}{0.514502in}}%
\pgfpathlineto{\pgfqpoint{2.350699in}{0.526660in}}%
\pgfpathlineto{\pgfqpoint{2.381322in}{0.539536in}}%
\pgfpathlineto{\pgfqpoint{2.407165in}{0.552659in}}%
\pgfpathlineto{\pgfqpoint{2.430227in}{0.566640in}}%
\pgfpathlineto{\pgfqpoint{2.452283in}{0.582603in}}%
\pgfpathlineto{\pgfqpoint{2.471392in}{0.599070in}}%
\pgfpathlineto{\pgfqpoint{2.489241in}{0.617294in}}%
\pgfpathlineto{\pgfqpoint{2.505678in}{0.637182in}}%
\pgfpathlineto{\pgfqpoint{2.520621in}{0.658558in}}%
\pgfpathlineto{\pgfqpoint{2.535214in}{0.683315in}}%
\pgfpathlineto{\pgfqpoint{2.549116in}{0.711486in}}%
\pgfpathlineto{\pgfqpoint{2.562091in}{0.743006in}}%
\pgfpathlineto{\pgfqpoint{2.574020in}{0.777753in}}%
\pgfpathlineto{\pgfqpoint{2.585503in}{0.817972in}}%
\pgfpathlineto{\pgfqpoint{2.596809in}{0.866040in}}%
\pgfpathlineto{\pgfqpoint{2.607562in}{0.921949in}}%
\pgfpathlineto{\pgfqpoint{2.617925in}{0.988100in}}%
\pgfpathlineto{\pgfqpoint{2.627958in}{1.066920in}}%
\pgfpathlineto{\pgfqpoint{2.637941in}{1.163322in}}%
\pgfpathlineto{\pgfqpoint{2.648424in}{1.287201in}}%
\pgfpathlineto{\pgfqpoint{2.660103in}{1.453440in}}%
\pgfpathlineto{\pgfqpoint{2.674773in}{1.696803in}}%
\pgfpathlineto{\pgfqpoint{2.687717in}{1.945280in}}%
\pgfpathlineto{\pgfqpoint{2.692671in}{2.079575in}}%
\pgfpathlineto{\pgfqpoint{2.693829in}{2.166684in}}%
\pgfpathlineto{\pgfqpoint{2.692565in}{2.233871in}}%
\pgfpathlineto{\pgfqpoint{2.689436in}{2.286016in}}%
\pgfpathlineto{\pgfqpoint{2.684859in}{2.328001in}}%
\pgfpathlineto{\pgfqpoint{2.678725in}{2.364665in}}%
\pgfpathlineto{\pgfqpoint{2.671356in}{2.395899in}}%
\pgfpathlineto{\pgfqpoint{2.662489in}{2.423983in}}%
\pgfpathlineto{\pgfqpoint{2.652361in}{2.448780in}}%
\pgfpathlineto{\pgfqpoint{2.641364in}{2.470247in}}%
\pgfpathlineto{\pgfqpoint{2.628642in}{2.490426in}}%
\pgfpathlineto{\pgfqpoint{2.614278in}{2.509107in}}%
\pgfpathlineto{\pgfqpoint{2.598442in}{2.526160in}}%
\pgfpathlineto{\pgfqpoint{2.579589in}{2.543006in}}%
\pgfpathlineto{\pgfqpoint{2.559531in}{2.557924in}}%
\pgfpathlineto{\pgfqpoint{2.536601in}{2.572184in}}%
\pgfpathlineto{\pgfqpoint{2.510848in}{2.585539in}}%
\pgfpathlineto{\pgfqpoint{2.482359in}{2.597838in}}%
\pgfpathlineto{\pgfqpoint{2.449133in}{2.609684in}}%
\pgfpathlineto{\pgfqpoint{2.411183in}{2.620697in}}%
\pgfpathlineto{\pgfqpoint{2.368551in}{2.630607in}}%
\pgfpathlineto{\pgfqpoint{2.321293in}{2.639221in}}%
\pgfpathlineto{\pgfqpoint{2.269466in}{2.646399in}}%
\pgfpathlineto{\pgfqpoint{2.210953in}{2.652193in}}%
\pgfpathlineto{\pgfqpoint{2.147966in}{2.656153in}}%
\pgfpathlineto{\pgfqpoint{2.080555in}{2.658135in}}%
\pgfpathlineto{\pgfqpoint{2.010947in}{2.657971in}}%
\pgfpathlineto{\pgfqpoint{1.939194in}{2.655572in}}%
\pgfpathlineto{\pgfqpoint{1.867526in}{2.650913in}}%
\pgfpathlineto{\pgfqpoint{1.798170in}{2.644140in}}%
\pgfpathlineto{\pgfqpoint{1.733340in}{2.635606in}}%
\pgfpathlineto{\pgfqpoint{1.673074in}{2.625521in}}%
\pgfpathlineto{\pgfqpoint{1.615273in}{2.613610in}}%
\pgfpathlineto{\pgfqpoint{1.562132in}{2.600401in}}%
\pgfpathlineto{\pgfqpoint{1.513680in}{2.586139in}}%
\pgfpathlineto{\pgfqpoint{1.467860in}{2.570343in}}%
\pgfpathlineto{\pgfqpoint{1.426792in}{2.553922in}}%
\pgfpathlineto{\pgfqpoint{1.388446in}{2.536288in}}%
\pgfpathlineto{\pgfqpoint{1.352877in}{2.517565in}}%
\pgfpathlineto{\pgfqpoint{1.320127in}{2.497921in}}%
\pgfpathlineto{\pgfqpoint{1.288378in}{2.476235in}}%
\pgfpathlineto{\pgfqpoint{1.259591in}{2.453860in}}%
\pgfpathlineto{\pgfqpoint{1.232050in}{2.429519in}}%
\pgfpathlineto{\pgfqpoint{1.207526in}{2.404897in}}%
\pgfpathlineto{\pgfqpoint{1.184408in}{2.378556in}}%
\pgfpathlineto{\pgfqpoint{1.162827in}{2.350560in}}%
\pgfpathlineto{\pgfqpoint{1.142890in}{2.321010in}}%
\pgfpathlineto{\pgfqpoint{1.124674in}{2.290040in}}%
\pgfpathlineto{\pgfqpoint{1.108225in}{2.257801in}}%
\pgfpathlineto{\pgfqpoint{1.092639in}{2.222198in}}%
\pgfpathlineto{\pgfqpoint{1.079059in}{2.185534in}}%
\pgfpathlineto{\pgfqpoint{1.067443in}{2.147996in}}%
\pgfpathlineto{\pgfqpoint{1.057187in}{2.107346in}}%
\pgfpathlineto{\pgfqpoint{1.049004in}{2.066084in}}%
\pgfpathlineto{\pgfqpoint{1.042513in}{2.021905in}}%
\pgfpathlineto{\pgfqpoint{1.038177in}{1.977380in}}%
\pgfpathlineto{\pgfqpoint{1.035866in}{1.930165in}}%
\pgfpathlineto{\pgfqpoint{1.035826in}{1.882876in}}%
\pgfpathlineto{\pgfqpoint{1.038031in}{1.835655in}}%
\pgfpathlineto{\pgfqpoint{1.042475in}{1.788640in}}%
\pgfpathlineto{\pgfqpoint{1.049176in}{1.741977in}}%
\pgfpathlineto{\pgfqpoint{1.057644in}{1.698237in}}%
\pgfpathlineto{\pgfqpoint{1.068222in}{1.655104in}}%
\pgfpathlineto{\pgfqpoint{1.080963in}{1.612743in}}%
\pgfpathlineto{\pgfqpoint{1.095031in}{1.573615in}}%
\pgfpathlineto{\pgfqpoint{1.111116in}{1.535518in}}%
\pgfpathlineto{\pgfqpoint{1.128119in}{1.500773in}}%
\pgfpathlineto{\pgfqpoint{1.146931in}{1.467272in}}%
\pgfpathlineto{\pgfqpoint{1.167532in}{1.435180in}}%
\pgfpathlineto{\pgfqpoint{1.189875in}{1.404651in}}%
\pgfpathlineto{\pgfqpoint{1.213885in}{1.375827in}}%
\pgfpathlineto{\pgfqpoint{1.237819in}{1.350456in}}%
\pgfpathlineto{\pgfqpoint{1.264750in}{1.325237in}}%
\pgfpathlineto{\pgfqpoint{1.292992in}{1.301971in}}%
\pgfpathlineto{\pgfqpoint{1.322399in}{1.280677in}}%
\pgfpathlineto{\pgfqpoint{1.352822in}{1.261340in}}%
\pgfpathlineto{\pgfqpoint{1.386096in}{1.242889in}}%
\pgfpathlineto{\pgfqpoint{1.420192in}{1.226516in}}%
\pgfpathlineto{\pgfqpoint{1.457026in}{1.211329in}}%
\pgfpathlineto{\pgfqpoint{1.496556in}{1.197536in}}%
\pgfpathlineto{\pgfqpoint{1.538721in}{1.185287in}}%
\pgfpathlineto{\pgfqpoint{1.583443in}{1.174641in}}%
\pgfpathlineto{\pgfqpoint{1.634931in}{1.164775in}}%
\pgfpathlineto{\pgfqpoint{1.706065in}{1.153745in}}%
\pgfpathlineto{\pgfqpoint{1.768494in}{1.143417in}}%
\pgfpathlineto{\pgfqpoint{1.796124in}{1.136567in}}%
\pgfpathlineto{\pgfqpoint{1.812685in}{1.130481in}}%
\pgfpathlineto{\pgfqpoint{1.824472in}{1.124102in}}%
\pgfpathlineto{\pgfqpoint{1.833211in}{1.116741in}}%
\pgfpathlineto{\pgfqpoint{1.838500in}{1.108889in}}%
\pgfpathlineto{\pgfqpoint{1.840590in}{1.101848in}}%
\pgfpathlineto{\pgfqpoint{1.840620in}{1.094411in}}%
\pgfpathlineto{\pgfqpoint{1.837931in}{1.084985in}}%
\pgfpathlineto{\pgfqpoint{1.833247in}{1.076614in}}%
\pgfpathlineto{\pgfqpoint{1.825819in}{1.067542in}}%
\pgfpathlineto{\pgfqpoint{1.813813in}{1.056849in}}%
\pgfpathlineto{\pgfqpoint{1.798819in}{1.046763in}}%
\pgfpathlineto{\pgfqpoint{1.781016in}{1.037462in}}%
\pgfpathlineto{\pgfqpoint{1.758447in}{1.028391in}}%
\pgfpathlineto{\pgfqpoint{1.733203in}{1.020815in}}%
\pgfpathlineto{\pgfqpoint{1.705410in}{1.014872in}}%
\pgfpathlineto{\pgfqpoint{1.675178in}{1.010714in}}%
\pgfpathlineto{\pgfqpoint{1.642610in}{1.008507in}}%
\pgfpathlineto{\pgfqpoint{1.607809in}{1.008432in}}%
\pgfpathlineto{\pgfqpoint{1.570886in}{1.010691in}}%
\pgfpathlineto{\pgfqpoint{1.534118in}{1.015181in}}%
\pgfpathlineto{\pgfqpoint{1.495455in}{1.022233in}}%
\pgfpathlineto{\pgfqpoint{1.457161in}{1.031564in}}%
\pgfpathlineto{\pgfqpoint{1.419338in}{1.043132in}}%
\pgfpathlineto{\pgfqpoint{1.382089in}{1.056929in}}%
\pgfpathlineto{\pgfqpoint{1.347544in}{1.072019in}}%
\pgfpathlineto{\pgfqpoint{1.313727in}{1.089133in}}%
\pgfpathlineto{\pgfqpoint{1.280762in}{1.108299in}}%
\pgfpathlineto{\pgfqpoint{1.248782in}{1.129536in}}%
\pgfpathlineto{\pgfqpoint{1.219708in}{1.151422in}}%
\pgfpathlineto{\pgfqpoint{1.191752in}{1.175138in}}%
\pgfpathlineto{\pgfqpoint{1.165031in}{1.200649in}}%
\pgfpathlineto{\pgfqpoint{1.139653in}{1.227898in}}%
\pgfpathlineto{\pgfqpoint{1.115714in}{1.256800in}}%
\pgfpathlineto{\pgfqpoint{1.093288in}{1.287251in}}%
\pgfpathlineto{\pgfqpoint{1.071178in}{1.321163in}}%
\pgfpathlineto{\pgfqpoint{1.050868in}{1.356519in}}%
\pgfpathlineto{\pgfqpoint{1.032365in}{1.393151in}}%
\pgfpathlineto{\pgfqpoint{1.014717in}{1.433142in}}%
\pgfpathlineto{\pgfqpoint{0.999023in}{1.474185in}}%
\pgfpathlineto{\pgfqpoint{0.984506in}{1.518461in}}%
\pgfpathlineto{\pgfqpoint{0.972009in}{1.563537in}}%
\pgfpathlineto{\pgfqpoint{0.960943in}{1.611678in}}%
\pgfpathlineto{\pgfqpoint{0.951530in}{1.662824in}}%
\pgfpathlineto{\pgfqpoint{0.944286in}{1.714431in}}%
\pgfpathlineto{\pgfqpoint{0.938949in}{1.768847in}}%
\pgfpathlineto{\pgfqpoint{0.935870in}{1.823491in}}%
\pgfpathlineto{\pgfqpoint{0.935034in}{1.878240in}}%
\pgfpathlineto{\pgfqpoint{0.936465in}{1.932972in}}%
\pgfpathlineto{\pgfqpoint{0.940005in}{1.985084in}}%
\pgfpathlineto{\pgfqpoint{0.945758in}{2.036935in}}%
\pgfpathlineto{\pgfqpoint{0.953409in}{2.085938in}}%
\pgfpathlineto{\pgfqpoint{0.962764in}{2.132000in}}%
\pgfpathlineto{\pgfqpoint{0.974286in}{2.177413in}}%
\pgfpathlineto{\pgfqpoint{0.987332in}{2.219652in}}%
\pgfpathlineto{\pgfqpoint{1.001667in}{2.258654in}}%
\pgfpathlineto{\pgfqpoint{1.018050in}{2.296583in}}%
\pgfpathlineto{\pgfqpoint{1.035401in}{2.331101in}}%
\pgfpathlineto{\pgfqpoint{1.054649in}{2.364275in}}%
\pgfpathlineto{\pgfqpoint{1.074406in}{2.393983in}}%
\pgfpathlineto{\pgfqpoint{1.095770in}{2.422197in}}%
\pgfpathlineto{\pgfqpoint{1.118661in}{2.448797in}}%
\pgfpathlineto{\pgfqpoint{1.142966in}{2.473701in}}%
\pgfpathlineto{\pgfqpoint{1.168550in}{2.496867in}}%
\pgfpathlineto{\pgfqpoint{1.197084in}{2.519662in}}%
\pgfpathlineto{\pgfqpoint{1.226726in}{2.540526in}}%
\pgfpathlineto{\pgfqpoint{1.259241in}{2.560673in}}%
\pgfpathlineto{\pgfqpoint{1.294611in}{2.579881in}}%
\pgfpathlineto{\pgfqpoint{1.332791in}{2.597982in}}%
\pgfpathlineto{\pgfqpoint{1.373718in}{2.614859in}}%
\pgfpathlineto{\pgfqpoint{1.417318in}{2.630445in}}%
\pgfpathlineto{\pgfqpoint{1.465631in}{2.645312in}}%
\pgfpathlineto{\pgfqpoint{1.518639in}{2.659205in}}%
\pgfpathlineto{\pgfqpoint{1.576308in}{2.671930in}}%
\pgfpathlineto{\pgfqpoint{1.638596in}{2.683344in}}%
\pgfpathlineto{\pgfqpoint{1.705461in}{2.693343in}}%
\pgfpathlineto{\pgfqpoint{1.779026in}{2.702064in}}%
\pgfpathlineto{\pgfqpoint{1.857096in}{2.709077in}}%
\pgfpathlineto{\pgfqpoint{1.939632in}{2.714280in}}%
\pgfpathlineto{\pgfqpoint{2.026597in}{2.717514in}}%
\pgfpathlineto{\pgfqpoint{2.113604in}{2.718524in}}%
\pgfpathlineto{\pgfqpoint{2.198434in}{2.717303in}}%
\pgfpathlineto{\pgfqpoint{2.278865in}{2.713929in}}%
\pgfpathlineto{\pgfqpoint{2.352677in}{2.708598in}}%
\pgfpathlineto{\pgfqpoint{2.417656in}{2.701709in}}%
\pgfpathlineto{\pgfqpoint{2.473769in}{2.693631in}}%
\pgfpathlineto{\pgfqpoint{2.523139in}{2.684368in}}%
\pgfpathlineto{\pgfqpoint{2.565725in}{2.674203in}}%
\pgfpathlineto{\pgfqpoint{2.601509in}{2.663545in}}%
\pgfpathlineto{\pgfqpoint{2.632576in}{2.652143in}}%
\pgfpathlineto{\pgfqpoint{2.658898in}{2.640332in}}%
\pgfpathlineto{\pgfqpoint{2.682437in}{2.627437in}}%
\pgfpathlineto{\pgfqpoint{2.703062in}{2.613573in}}%
\pgfpathlineto{\pgfqpoint{2.720674in}{2.598980in}}%
\pgfpathlineto{\pgfqpoint{2.735262in}{2.584055in}}%
\pgfpathlineto{\pgfqpoint{2.748319in}{2.567379in}}%
\pgfpathlineto{\pgfqpoint{2.759553in}{2.549048in}}%
\pgfpathlineto{\pgfqpoint{2.768788in}{2.529308in}}%
\pgfpathlineto{\pgfqpoint{2.776017in}{2.508500in}}%
\pgfpathlineto{\pgfqpoint{2.781884in}{2.484542in}}%
\pgfpathlineto{\pgfqpoint{2.786103in}{2.457599in}}%
\pgfpathlineto{\pgfqpoint{2.788721in}{2.425387in}}%
\pgfpathlineto{\pgfqpoint{2.789428in}{2.388064in}}%
\pgfpathlineto{\pgfqpoint{2.787963in}{2.340804in}}%
\pgfpathlineto{\pgfqpoint{2.783673in}{2.278771in}}%
\pgfpathlineto{\pgfqpoint{2.774289in}{2.179785in}}%
\pgfpathlineto{\pgfqpoint{2.743611in}{1.868121in}}%
\pgfpathlineto{\pgfqpoint{2.730112in}{1.702063in}}%
\pgfpathlineto{\pgfqpoint{2.717287in}{1.515952in}}%
\pgfpathlineto{\pgfqpoint{2.702602in}{1.267600in}}%
\pgfpathlineto{\pgfqpoint{2.684434in}{0.964632in}}%
\pgfpathlineto{\pgfqpoint{2.675375in}{0.850602in}}%
\pgfpathlineto{\pgfqpoint{2.667030in}{0.771526in}}%
\pgfpathlineto{\pgfqpoint{2.658753in}{0.712545in}}%
\pgfpathlineto{\pgfqpoint{2.650177in}{0.666286in}}%
\pgfpathlineto{\pgfqpoint{2.640821in}{0.627933in}}%
\pgfpathlineto{\pgfqpoint{2.631146in}{0.597536in}}%
\pgfpathlineto{\pgfqpoint{2.621005in}{0.572747in}}%
\pgfpathlineto{\pgfqpoint{2.609858in}{0.551385in}}%
\pgfpathlineto{\pgfqpoint{2.598044in}{0.533535in}}%
\pgfpathlineto{\pgfqpoint{2.584497in}{0.517380in}}%
\pgfpathlineto{\pgfqpoint{2.571111in}{0.504670in}}%
\pgfpathlineto{\pgfqpoint{2.554791in}{0.492314in}}%
\pgfpathlineto{\pgfqpoint{2.537459in}{0.481915in}}%
\pgfpathlineto{\pgfqpoint{2.517376in}{0.472367in}}%
\pgfpathlineto{\pgfqpoint{2.492544in}{0.463179in}}%
\pgfpathlineto{\pgfqpoint{2.462982in}{0.454833in}}%
\pgfpathlineto{\pgfqpoint{2.428768in}{0.447542in}}%
\pgfpathlineto{\pgfqpoint{2.385674in}{0.440735in}}%
\pgfpathlineto{\pgfqpoint{2.331559in}{0.434582in}}%
\pgfpathlineto{\pgfqpoint{2.262117in}{0.429077in}}%
\pgfpathlineto{\pgfqpoint{2.170853in}{0.424236in}}%
\pgfpathlineto{\pgfqpoint{2.049088in}{0.420134in}}%
\pgfpathlineto{\pgfqpoint{1.879438in}{0.416783in}}%
\pgfpathlineto{\pgfqpoint{1.640162in}{0.414418in}}%
\pgfpathlineto{\pgfqpoint{1.322565in}{0.413569in}}%
\pgfpathlineto{\pgfqpoint{1.020196in}{0.414850in}}%
\pgfpathlineto{\pgfqpoint{0.822258in}{0.417714in}}%
\pgfpathlineto{\pgfqpoint{0.704837in}{0.421430in}}%
\pgfpathlineto{\pgfqpoint{0.630979in}{0.425828in}}%
\pgfpathlineto{\pgfqpoint{0.583318in}{0.430734in}}%
\pgfpathlineto{\pgfqpoint{0.551036in}{0.436123in}}%
\pgfpathlineto{\pgfqpoint{0.527710in}{0.442188in}}%
\pgfpathlineto{\pgfqpoint{0.511252in}{0.448624in}}%
\pgfpathlineto{\pgfqpoint{0.499551in}{0.455214in}}%
\pgfpathlineto{\pgfqpoint{0.488918in}{0.463840in}}%
\pgfpathlineto{\pgfqpoint{0.481324in}{0.472728in}}%
\pgfpathlineto{\pgfqpoint{0.474079in}{0.485124in}}%
\pgfpathlineto{\pgfqpoint{0.468754in}{0.498746in}}%
\pgfpathlineto{\pgfqpoint{0.463870in}{0.517846in}}%
\pgfpathlineto{\pgfqpoint{0.459679in}{0.544794in}}%
\pgfpathlineto{\pgfqpoint{0.456386in}{0.581936in}}%
\pgfpathlineto{\pgfqpoint{0.453731in}{0.639103in}}%
\pgfpathlineto{\pgfqpoint{0.451681in}{0.736152in}}%
\pgfpathlineto{\pgfqpoint{0.450220in}{0.927813in}}%
\pgfpathlineto{\pgfqpoint{0.449345in}{1.403249in}}%
\pgfpathlineto{\pgfqpoint{0.449542in}{2.682700in}}%
\pgfpathlineto{\pgfqpoint{0.451010in}{2.856929in}}%
\pgfpathlineto{\pgfqpoint{0.452802in}{2.879217in}}%
\pgfpathlineto{\pgfqpoint{0.455186in}{2.886105in}}%
\pgfpathlineto{\pgfqpoint{0.458623in}{2.889027in}}%
\pgfpathlineto{\pgfqpoint{0.464993in}{2.890553in}}%
\pgfpathlineto{\pgfqpoint{0.482374in}{2.891422in}}%
\pgfpathlineto{\pgfqpoint{0.565036in}{2.891729in}}%
\pgfpathlineto{\pgfqpoint{2.733840in}{2.891760in}}%
\pgfpathlineto{\pgfqpoint{4.789507in}{2.890886in}}%
\pgfpathlineto{\pgfqpoint{4.793724in}{2.889732in}}%
\pgfpathlineto{\pgfqpoint{4.795479in}{2.888309in}}%
\pgfpathlineto{\pgfqpoint{4.797105in}{2.881148in}}%
\pgfpathlineto{\pgfqpoint{4.797997in}{2.858774in}}%
\pgfpathlineto{\pgfqpoint{4.798039in}{2.856286in}}%
\pgfpathlineto{\pgfqpoint{4.798039in}{2.856286in}}%
\pgfusepath{stroke}%
\end{pgfscope}%
\begin{pgfscope}%
\pgfpathrectangle{\pgfqpoint{0.448634in}{0.402556in}}{\pgfqpoint{4.350661in}{2.489204in}} %
\pgfusepath{clip}%
\pgfsetrectcap%
\pgfsetroundjoin%
\pgfsetlinewidth{1.003750pt}%
\definecolor{currentstroke}{rgb}{0.549020,0.337255,0.294118}%
\pgfsetstrokecolor{currentstroke}%
\pgfsetdash{}{0pt}%
\pgfpathmoveto{\pgfqpoint{3.428768in}{0.402610in}}%
\pgfpathlineto{\pgfqpoint{2.806627in}{0.403757in}}%
\pgfpathlineto{\pgfqpoint{2.769687in}{0.405570in}}%
\pgfpathlineto{\pgfqpoint{2.754627in}{0.408050in}}%
\pgfpathlineto{\pgfqpoint{2.746384in}{0.411178in}}%
\pgfpathlineto{\pgfqpoint{2.740933in}{0.415240in}}%
\pgfpathlineto{\pgfqpoint{2.736772in}{0.420957in}}%
\pgfpathlineto{\pgfqpoint{2.733268in}{0.430044in}}%
\pgfpathlineto{\pgfqpoint{2.730437in}{0.444609in}}%
\pgfpathlineto{\pgfqpoint{2.728229in}{0.469365in}}%
\pgfpathlineto{\pgfqpoint{2.726463in}{0.519104in}}%
\pgfpathlineto{\pgfqpoint{2.725706in}{0.613688in}}%
\pgfpathlineto{\pgfqpoint{2.726838in}{0.768012in}}%
\pgfpathlineto{\pgfqpoint{2.730553in}{0.962122in}}%
\pgfpathlineto{\pgfqpoint{2.736608in}{1.158644in}}%
\pgfpathlineto{\pgfqpoint{2.744088in}{1.327692in}}%
\pgfpathlineto{\pgfqpoint{2.753197in}{1.484163in}}%
\pgfpathlineto{\pgfqpoint{2.763252in}{1.620583in}}%
\pgfpathlineto{\pgfqpoint{2.776113in}{1.764189in}}%
\pgfpathlineto{\pgfqpoint{2.788908in}{1.877750in}}%
\pgfpathlineto{\pgfqpoint{2.805740in}{2.005714in}}%
\pgfpathlineto{\pgfqpoint{2.821167in}{2.101172in}}%
\pgfpathlineto{\pgfqpoint{2.838349in}{2.193693in}}%
\pgfpathlineto{\pgfqpoint{2.859124in}{2.292941in}}%
\pgfpathlineto{\pgfqpoint{2.887197in}{2.425935in}}%
\pgfpathlineto{\pgfqpoint{2.896979in}{2.479535in}}%
\pgfpathlineto{\pgfqpoint{2.901531in}{2.516499in}}%
\pgfpathlineto{\pgfqpoint{2.902838in}{2.543829in}}%
\pgfpathlineto{\pgfqpoint{2.901948in}{2.566198in}}%
\pgfpathlineto{\pgfqpoint{2.899143in}{2.585839in}}%
\pgfpathlineto{\pgfqpoint{2.894787in}{2.602522in}}%
\pgfpathlineto{\pgfqpoint{2.888479in}{2.618365in}}%
\pgfpathlineto{\pgfqpoint{2.880255in}{2.633011in}}%
\pgfpathlineto{\pgfqpoint{2.870347in}{2.646226in}}%
\pgfpathlineto{\pgfqpoint{2.857400in}{2.659513in}}%
\pgfpathlineto{\pgfqpoint{2.843191in}{2.670994in}}%
\pgfpathlineto{\pgfqpoint{2.824240in}{2.683195in}}%
\pgfpathlineto{\pgfqpoint{2.802416in}{2.694406in}}%
\pgfpathlineto{\pgfqpoint{2.775812in}{2.705357in}}%
\pgfpathlineto{\pgfqpoint{2.744465in}{2.715705in}}%
\pgfpathlineto{\pgfqpoint{2.708440in}{2.725243in}}%
\pgfpathlineto{\pgfqpoint{2.665659in}{2.734281in}}%
\pgfpathlineto{\pgfqpoint{2.613996in}{2.742862in}}%
\pgfpathlineto{\pgfqpoint{2.553464in}{2.750582in}}%
\pgfpathlineto{\pgfqpoint{2.481924in}{2.757360in}}%
\pgfpathlineto{\pgfqpoint{2.399402in}{2.762834in}}%
\pgfpathlineto{\pgfqpoint{2.310274in}{2.766477in}}%
\pgfpathlineto{\pgfqpoint{2.175421in}{2.768721in}}%
\pgfpathlineto{\pgfqpoint{2.066658in}{2.767937in}}%
\pgfpathlineto{\pgfqpoint{1.953575in}{2.764855in}}%
\pgfpathlineto{\pgfqpoint{1.851433in}{2.759754in}}%
\pgfpathlineto{\pgfqpoint{1.745055in}{2.752164in}}%
\pgfpathlineto{\pgfqpoint{1.658378in}{2.743449in}}%
\pgfpathlineto{\pgfqpoint{1.580556in}{2.733456in}}%
\pgfpathlineto{\pgfqpoint{1.490062in}{2.719333in}}%
\pgfpathlineto{\pgfqpoint{1.417237in}{2.704685in}}%
\pgfpathlineto{\pgfqpoint{1.361997in}{2.690804in}}%
\pgfpathlineto{\pgfqpoint{1.311466in}{2.675803in}}%
\pgfpathlineto{\pgfqpoint{1.265673in}{2.659907in}}%
\pgfpathlineto{\pgfqpoint{1.222581in}{2.642570in}}%
\pgfpathlineto{\pgfqpoint{1.184330in}{2.624665in}}%
\pgfpathlineto{\pgfqpoint{1.148899in}{2.605605in}}%
\pgfpathlineto{\pgfqpoint{1.116339in}{2.585554in}}%
\pgfpathlineto{\pgfqpoint{1.092330in}{2.568498in}}%
\pgfpathlineto{\pgfqpoint{1.079759in}{2.558679in}}%
\pgfpathlineto{\pgfqpoint{1.051543in}{2.535372in}}%
\pgfpathlineto{\pgfqpoint{1.026311in}{2.511705in}}%
\pgfpathlineto{\pgfqpoint{1.002398in}{2.486310in}}%
\pgfpathlineto{\pgfqpoint{0.979913in}{2.459261in}}%
\pgfpathlineto{\pgfqpoint{0.958935in}{2.430669in}}%
\pgfpathlineto{\pgfqpoint{0.938265in}{2.398634in}}%
\pgfpathlineto{\pgfqpoint{0.923046in}{2.371377in}}%
\pgfpathlineto{\pgfqpoint{0.904513in}{2.334766in}}%
\pgfpathlineto{\pgfqpoint{0.887854in}{2.296992in}}%
\pgfpathlineto{\pgfqpoint{0.872132in}{2.255963in}}%
\pgfpathlineto{\pgfqpoint{0.857509in}{2.211732in}}%
\pgfpathlineto{\pgfqpoint{0.844763in}{2.166748in}}%
\pgfpathlineto{\pgfqpoint{0.838623in}{2.140298in}}%
\pgfpathlineto{\pgfqpoint{0.826981in}{2.087185in}}%
\pgfpathlineto{\pgfqpoint{0.816321in}{2.028707in}}%
\pgfpathlineto{\pgfqpoint{0.810088in}{1.984486in}}%
\pgfpathlineto{\pgfqpoint{0.808025in}{1.967229in}}%
\pgfpathlineto{\pgfqpoint{0.800076in}{1.898131in}}%
\pgfpathlineto{\pgfqpoint{0.793713in}{1.823814in}}%
\pgfpathlineto{\pgfqpoint{0.788799in}{1.741866in}}%
\pgfpathlineto{\pgfqpoint{0.786199in}{1.677216in}}%
\pgfpathlineto{\pgfqpoint{0.776950in}{1.453472in}}%
\pgfpathlineto{\pgfqpoint{0.773279in}{1.418885in}}%
\pgfpathlineto{\pgfqpoint{0.768298in}{1.389573in}}%
\pgfpathlineto{\pgfqpoint{0.762751in}{1.368099in}}%
\pgfpathlineto{\pgfqpoint{0.756720in}{1.352114in}}%
\pgfpathlineto{\pgfqpoint{0.749750in}{1.339511in}}%
\pgfpathlineto{\pgfqpoint{0.742198in}{1.330592in}}%
\pgfpathlineto{\pgfqpoint{0.734851in}{1.325306in}}%
\pgfpathlineto{\pgfqpoint{0.726554in}{1.322414in}}%
\pgfpathlineto{\pgfqpoint{0.717880in}{1.322220in}}%
\pgfpathlineto{\pgfqpoint{0.709409in}{1.324409in}}%
\pgfpathlineto{\pgfqpoint{0.699545in}{1.329603in}}%
\pgfpathlineto{\pgfqpoint{0.688891in}{1.338202in}}%
\pgfpathlineto{\pgfqpoint{0.677905in}{1.350248in}}%
\pgfpathlineto{\pgfqpoint{0.666884in}{1.365647in}}%
\pgfpathlineto{\pgfqpoint{0.654911in}{1.386417in}}%
\pgfpathlineto{\pgfqpoint{0.642572in}{1.412730in}}%
\pgfpathlineto{\pgfqpoint{0.630327in}{1.444629in}}%
\pgfpathlineto{\pgfqpoint{0.618503in}{1.482081in}}%
\pgfpathlineto{\pgfqpoint{0.608612in}{1.520256in}}%
\pgfpathlineto{\pgfqpoint{0.590202in}{1.612446in}}%
\pgfpathlineto{\pgfqpoint{0.581847in}{1.668885in}}%
\pgfpathlineto{\pgfqpoint{0.573137in}{1.740377in}}%
\pgfpathlineto{\pgfqpoint{0.567061in}{1.807214in}}%
\pgfpathlineto{\pgfqpoint{0.560531in}{1.896510in}}%
\pgfpathlineto{\pgfqpoint{0.555525in}{1.995911in}}%
\pgfpathlineto{\pgfqpoint{0.552564in}{2.097909in}}%
\pgfpathlineto{\pgfqpoint{0.551525in}{2.204936in}}%
\pgfpathlineto{\pgfqpoint{0.552727in}{2.309470in}}%
\pgfpathlineto{\pgfqpoint{0.556011in}{2.403982in}}%
\pgfpathlineto{\pgfqpoint{0.560952in}{2.483431in}}%
\pgfpathlineto{\pgfqpoint{0.567303in}{2.550241in}}%
\pgfpathlineto{\pgfqpoint{0.574927in}{2.606818in}}%
\pgfpathlineto{\pgfqpoint{0.582987in}{2.650657in}}%
\pgfpathlineto{\pgfqpoint{0.592755in}{2.691453in}}%
\pgfpathlineto{\pgfqpoint{0.602650in}{2.721757in}}%
\pgfpathlineto{\pgfqpoint{0.612983in}{2.746442in}}%
\pgfpathlineto{\pgfqpoint{0.624292in}{2.767693in}}%
\pgfpathlineto{\pgfqpoint{0.636231in}{2.785433in}}%
\pgfpathlineto{\pgfqpoint{0.649892in}{2.801462in}}%
\pgfpathlineto{\pgfqpoint{0.663386in}{2.814021in}}%
\pgfpathlineto{\pgfqpoint{0.679842in}{2.826135in}}%
\pgfpathlineto{\pgfqpoint{0.697326in}{2.836198in}}%
\pgfpathlineto{\pgfqpoint{0.715574in}{2.844285in}}%
\pgfpathlineto{\pgfqpoint{0.738439in}{2.852335in}}%
\pgfpathlineto{\pgfqpoint{0.765984in}{2.859640in}}%
\pgfpathlineto{\pgfqpoint{0.800301in}{2.866257in}}%
\pgfpathlineto{\pgfqpoint{0.841340in}{2.871832in}}%
\pgfpathlineto{\pgfqpoint{0.895547in}{2.876803in}}%
\pgfpathlineto{\pgfqpoint{0.969413in}{2.881069in}}%
\pgfpathlineto{\pgfqpoint{1.071608in}{2.884501in}}%
\pgfpathlineto{\pgfqpoint{1.219512in}{2.887074in}}%
\pgfpathlineto{\pgfqpoint{1.471844in}{2.889091in}}%
\pgfpathlineto{\pgfqpoint{1.956941in}{2.890384in}}%
\pgfpathlineto{\pgfqpoint{3.096814in}{2.890781in}}%
\pgfpathlineto{\pgfqpoint{3.995224in}{2.889388in}}%
\pgfpathlineto{\pgfqpoint{4.275833in}{2.887011in}}%
\pgfpathlineto{\pgfqpoint{4.412848in}{2.883743in}}%
\pgfpathlineto{\pgfqpoint{4.491081in}{2.879810in}}%
\pgfpathlineto{\pgfqpoint{4.543127in}{2.875163in}}%
\pgfpathlineto{\pgfqpoint{4.579810in}{2.869841in}}%
\pgfpathlineto{\pgfqpoint{4.607580in}{2.863763in}}%
\pgfpathlineto{\pgfqpoint{4.630623in}{2.856424in}}%
\pgfpathlineto{\pgfqpoint{4.648834in}{2.848228in}}%
\pgfpathlineto{\pgfqpoint{4.664136in}{2.838773in}}%
\pgfpathlineto{\pgfqpoint{4.676470in}{2.828576in}}%
\pgfpathlineto{\pgfqpoint{4.687502in}{2.816585in}}%
\pgfpathlineto{\pgfqpoint{4.697051in}{2.803027in}}%
\pgfpathlineto{\pgfqpoint{4.706194in}{2.786098in}}%
\pgfpathlineto{\pgfqpoint{4.714508in}{2.765827in}}%
\pgfpathlineto{\pgfqpoint{4.722462in}{2.740013in}}%
\pgfpathlineto{\pgfqpoint{4.729577in}{2.708703in}}%
\pgfpathlineto{\pgfqpoint{4.736162in}{2.669601in}}%
\pgfpathlineto{\pgfqpoint{4.742419in}{2.617826in}}%
\pgfpathlineto{\pgfqpoint{4.747859in}{2.553410in}}%
\pgfpathlineto{\pgfqpoint{4.752661in}{2.468958in}}%
\pgfpathlineto{\pgfqpoint{4.756610in}{2.359528in}}%
\pgfpathlineto{\pgfqpoint{4.759416in}{2.217681in}}%
\pgfpathlineto{\pgfqpoint{4.760596in}{2.043444in}}%
\pgfpathlineto{\pgfqpoint{4.759662in}{1.851779in}}%
\pgfpathlineto{\pgfqpoint{4.756587in}{1.667613in}}%
\pgfpathlineto{\pgfqpoint{4.751596in}{1.503428in}}%
\pgfpathlineto{\pgfqpoint{4.745410in}{1.374185in}}%
\pgfpathlineto{\pgfqpoint{4.738113in}{1.267479in}}%
\pgfpathlineto{\pgfqpoint{4.729621in}{1.175896in}}%
\pgfpathlineto{\pgfqpoint{4.720762in}{1.104428in}}%
\pgfpathlineto{\pgfqpoint{4.711045in}{1.043204in}}%
\pgfpathlineto{\pgfqpoint{4.700364in}{0.989829in}}%
\pgfpathlineto{\pgfqpoint{4.689055in}{0.944345in}}%
\pgfpathlineto{\pgfqpoint{4.676881in}{0.904394in}}%
\pgfpathlineto{\pgfqpoint{4.676095in}{0.902073in}}%
\pgfpathlineto{\pgfqpoint{4.676095in}{0.902073in}}%
\pgfusepath{stroke}%
\end{pgfscope}%
\begin{pgfscope}%
\pgfpathrectangle{\pgfqpoint{0.448634in}{0.402556in}}{\pgfqpoint{4.350661in}{2.489204in}} %
\pgfusepath{clip}%
\pgfsetrectcap%
\pgfsetroundjoin%
\pgfsetlinewidth{1.003750pt}%
\definecolor{currentstroke}{rgb}{0.549020,0.337255,0.294118}%
\pgfsetstrokecolor{currentstroke}%
\pgfsetdash{}{0pt}%
\pgfpathmoveto{\pgfqpoint{2.795520in}{1.982745in}}%
\pgfpathlineto{\pgfqpoint{2.781780in}{1.874357in}}%
\pgfpathlineto{\pgfqpoint{2.769351in}{1.758234in}}%
\pgfpathlineto{\pgfqpoint{2.758095in}{1.631942in}}%
\pgfpathlineto{\pgfqpoint{2.747786in}{1.490551in}}%
\pgfpathlineto{\pgfqpoint{2.738644in}{1.334082in}}%
\pgfpathlineto{\pgfqpoint{2.730580in}{1.157591in}}%
\pgfpathlineto{\pgfqpoint{2.723334in}{0.948663in}}%
\pgfpathlineto{\pgfqpoint{2.709783in}{0.530788in}}%
\pgfpathlineto{\pgfqpoint{2.705868in}{0.488716in}}%
\pgfpathlineto{\pgfqpoint{2.701769in}{0.464281in}}%
\pgfpathlineto{\pgfqpoint{2.697021in}{0.447744in}}%
\pgfpathlineto{\pgfqpoint{2.691859in}{0.436812in}}%
\pgfpathlineto{\pgfqpoint{2.686245in}{0.429229in}}%
\pgfpathlineto{\pgfqpoint{2.679348in}{0.423188in}}%
\pgfpathlineto{\pgfqpoint{2.669540in}{0.417856in}}%
\pgfpathlineto{\pgfqpoint{2.656987in}{0.413810in}}%
\pgfpathlineto{\pgfqpoint{2.637654in}{0.410337in}}%
\pgfpathlineto{\pgfqpoint{2.607297in}{0.407617in}}%
\pgfpathlineto{\pgfqpoint{2.555121in}{0.405574in}}%
\pgfpathlineto{\pgfqpoint{2.450714in}{0.404139in}}%
\pgfpathlineto{\pgfqpoint{2.176624in}{0.403275in}}%
\pgfpathlineto{\pgfqpoint{1.130290in}{0.402953in}}%
\pgfpathlineto{\pgfqpoint{0.516849in}{0.404175in}}%
\pgfpathlineto{\pgfqpoint{0.466848in}{0.405970in}}%
\pgfpathlineto{\pgfqpoint{0.456129in}{0.407931in}}%
\pgfpathlineto{\pgfqpoint{0.452340in}{0.410303in}}%
\pgfpathlineto{\pgfqpoint{0.450346in}{0.414662in}}%
\pgfpathlineto{\pgfqpoint{0.449266in}{0.424524in}}%
\pgfpathlineto{\pgfqpoint{0.448771in}{0.464345in}}%
\pgfpathlineto{\pgfqpoint{0.448640in}{0.850171in}}%
\pgfpathlineto{\pgfqpoint{0.448653in}{2.891318in}}%
\pgfpathlineto{\pgfqpoint{0.448653in}{2.891318in}}%
\pgfusepath{stroke}%
\end{pgfscope}%
\begin{pgfscope}%
\pgfpathrectangle{\pgfqpoint{0.448634in}{0.402556in}}{\pgfqpoint{4.350661in}{2.489204in}} %
\pgfusepath{clip}%
\pgfsetrectcap%
\pgfsetroundjoin%
\pgfsetlinewidth{1.003750pt}%
\definecolor{currentstroke}{rgb}{0.549020,0.337255,0.294118}%
\pgfsetstrokecolor{currentstroke}%
\pgfsetdash{}{0pt}%
\pgfpathmoveto{\pgfqpoint{3.428229in}{0.402586in}}%
\pgfpathlineto{\pgfqpoint{2.782161in}{0.403711in}}%
\pgfpathlineto{\pgfqpoint{2.753947in}{0.405697in}}%
\pgfpathlineto{\pgfqpoint{2.743371in}{0.408479in}}%
\pgfpathlineto{\pgfqpoint{2.737761in}{0.412225in}}%
\pgfpathlineto{\pgfqpoint{2.733707in}{0.418028in}}%
\pgfpathlineto{\pgfqpoint{2.730679in}{0.427337in}}%
\pgfpathlineto{\pgfqpoint{2.728410in}{0.442032in}}%
\pgfpathlineto{\pgfqpoint{2.726559in}{0.471822in}}%
\pgfpathlineto{\pgfqpoint{2.725226in}{0.534030in}}%
\pgfpathlineto{\pgfqpoint{2.725176in}{0.656000in}}%
\pgfpathlineto{\pgfqpoint{2.727382in}{0.832714in}}%
\pgfpathlineto{\pgfqpoint{2.732264in}{1.041730in}}%
\pgfpathlineto{\pgfqpoint{2.738856in}{1.223284in}}%
\pgfpathlineto{\pgfqpoint{2.747083in}{1.389793in}}%
\pgfpathlineto{\pgfqpoint{2.756614in}{1.538744in}}%
\pgfpathlineto{\pgfqpoint{2.768961in}{1.694914in}}%
\pgfpathlineto{\pgfqpoint{2.781235in}{1.816071in}}%
\pgfpathlineto{\pgfqpoint{2.794410in}{1.924551in}}%
\pgfpathlineto{\pgfqpoint{2.812747in}{2.054749in}}%
\pgfpathlineto{\pgfqpoint{2.828786in}{2.147538in}}%
\pgfpathlineto{\pgfqpoint{2.847396in}{2.242250in}}%
\pgfpathlineto{\pgfqpoint{2.895833in}{2.479724in}}%
\pgfpathlineto{\pgfqpoint{2.900218in}{2.516714in}}%
\pgfpathlineto{\pgfqpoint{2.901360in}{2.544054in}}%
\pgfpathlineto{\pgfqpoint{2.900304in}{2.566413in}}%
\pgfpathlineto{\pgfqpoint{2.897346in}{2.586024in}}%
\pgfpathlineto{\pgfqpoint{2.892846in}{2.602657in}}%
\pgfpathlineto{\pgfqpoint{2.886402in}{2.618429in}}%
\pgfpathlineto{\pgfqpoint{2.878064in}{2.632991in}}%
\pgfpathlineto{\pgfqpoint{2.868069in}{2.646120in}}%
\pgfpathlineto{\pgfqpoint{2.855054in}{2.659319in}}%
\pgfpathlineto{\pgfqpoint{2.840803in}{2.670733in}}%
\pgfpathlineto{\pgfqpoint{2.821824in}{2.682876in}}%
\pgfpathlineto{\pgfqpoint{2.799981in}{2.694039in}}%
\pgfpathlineto{\pgfqpoint{2.773366in}{2.704956in}}%
\pgfpathlineto{\pgfqpoint{2.742012in}{2.715277in}}%
\pgfpathlineto{\pgfqpoint{2.705983in}{2.724795in}}%
\pgfpathlineto{\pgfqpoint{2.663200in}{2.733819in}}%
\pgfpathlineto{\pgfqpoint{2.611535in}{2.742387in}}%
\pgfpathlineto{\pgfqpoint{2.551002in}{2.750097in}}%
\pgfpathlineto{\pgfqpoint{2.481632in}{2.756689in}}%
\pgfpathlineto{\pgfqpoint{2.399112in}{2.762206in}}%
\pgfpathlineto{\pgfqpoint{2.309984in}{2.765891in}}%
\pgfpathlineto{\pgfqpoint{2.188184in}{2.768102in}}%
\pgfpathlineto{\pgfqpoint{2.081595in}{2.767624in}}%
\pgfpathlineto{\pgfqpoint{1.968505in}{2.764845in}}%
\pgfpathlineto{\pgfqpoint{1.864180in}{2.759924in}}%
\pgfpathlineto{\pgfqpoint{1.757786in}{2.752599in}}%
\pgfpathlineto{\pgfqpoint{1.671086in}{2.744177in}}%
\pgfpathlineto{\pgfqpoint{1.591075in}{2.734200in}}%
\pgfpathlineto{\pgfqpoint{1.502689in}{2.720726in}}%
\pgfpathlineto{\pgfqpoint{1.427655in}{2.706083in}}%
\pgfpathlineto{\pgfqpoint{1.372350in}{2.692544in}}%
\pgfpathlineto{\pgfqpoint{1.321734in}{2.677921in}}%
\pgfpathlineto{\pgfqpoint{1.273765in}{2.661664in}}%
\pgfpathlineto{\pgfqpoint{1.230567in}{2.644673in}}%
\pgfpathlineto{\pgfqpoint{1.192196in}{2.627107in}}%
\pgfpathlineto{\pgfqpoint{1.156620in}{2.608404in}}%
\pgfpathlineto{\pgfqpoint{1.123890in}{2.588718in}}%
\pgfpathlineto{\pgfqpoint{1.095882in}{2.569570in}}%
\pgfpathlineto{\pgfqpoint{1.026625in}{2.509436in}}%
\pgfpathlineto{\pgfqpoint{1.002809in}{2.483922in}}%
\pgfpathlineto{\pgfqpoint{0.980428in}{2.456759in}}%
\pgfpathlineto{\pgfqpoint{0.959560in}{2.428062in}}%
\pgfpathlineto{\pgfqpoint{0.939013in}{2.395924in}}%
\pgfpathlineto{\pgfqpoint{0.923905in}{2.368586in}}%
\pgfpathlineto{\pgfqpoint{0.905505in}{2.331886in}}%
\pgfpathlineto{\pgfqpoint{0.888981in}{2.294035in}}%
\pgfpathlineto{\pgfqpoint{0.873402in}{2.252935in}}%
\pgfpathlineto{\pgfqpoint{0.858929in}{2.208639in}}%
\pgfpathlineto{\pgfqpoint{0.846473in}{2.163558in}}%
\pgfpathlineto{\pgfqpoint{0.841904in}{2.144348in}}%
\pgfpathlineto{\pgfqpoint{0.830146in}{2.091269in}}%
\pgfpathlineto{\pgfqpoint{0.819426in}{2.032806in}}%
\pgfpathlineto{\pgfqpoint{0.812048in}{1.981229in}}%
\pgfpathlineto{\pgfqpoint{0.800927in}{1.882487in}}%
\pgfpathlineto{\pgfqpoint{0.795032in}{1.805620in}}%
\pgfpathlineto{\pgfqpoint{0.790757in}{1.721132in}}%
\pgfpathlineto{\pgfqpoint{0.787501in}{1.611673in}}%
\pgfpathlineto{\pgfqpoint{0.785354in}{1.522095in}}%
\pgfpathlineto{\pgfqpoint{0.785354in}{1.522095in}}%
\pgfusepath{stroke}%
\end{pgfscope}%
\begin{pgfscope}%
\pgfpathrectangle{\pgfqpoint{0.448634in}{0.402556in}}{\pgfqpoint{4.350661in}{2.489204in}} %
\pgfusepath{clip}%
\pgfsetrectcap%
\pgfsetroundjoin%
\pgfsetlinewidth{1.003750pt}%
\definecolor{currentstroke}{rgb}{0.549020,0.337255,0.294118}%
\pgfsetstrokecolor{currentstroke}%
\pgfsetdash{}{0pt}%
\pgfpathmoveto{\pgfqpoint{2.028735in}{0.425754in}}%
\pgfpathlineto{\pgfqpoint{1.878677in}{0.421879in}}%
\pgfpathlineto{\pgfqpoint{1.676387in}{0.418997in}}%
\pgfpathlineto{\pgfqpoint{1.413176in}{0.417558in}}%
\pgfpathlineto{\pgfqpoint{1.134735in}{0.418204in}}%
\pgfpathlineto{\pgfqpoint{0.921565in}{0.420769in}}%
\pgfpathlineto{\pgfqpoint{0.782384in}{0.424523in}}%
\pgfpathlineto{\pgfqpoint{0.693283in}{0.428974in}}%
\pgfpathlineto{\pgfqpoint{0.632541in}{0.434091in}}%
\pgfpathlineto{\pgfqpoint{0.591492in}{0.439564in}}%
\pgfpathlineto{\pgfqpoint{0.561503in}{0.445595in}}%
\pgfpathlineto{\pgfqpoint{0.538349in}{0.452466in}}%
\pgfpathlineto{\pgfqpoint{0.522042in}{0.459394in}}%
\pgfpathlineto{\pgfqpoint{0.508540in}{0.467420in}}%
\pgfpathlineto{\pgfqpoint{0.497973in}{0.476161in}}%
\pgfpathlineto{\pgfqpoint{0.488790in}{0.486749in}}%
\pgfpathlineto{\pgfqpoint{0.481284in}{0.498948in}}%
\pgfpathlineto{\pgfqpoint{0.474590in}{0.514580in}}%
\pgfpathlineto{\pgfqpoint{0.469106in}{0.533467in}}%
\pgfpathlineto{\pgfqpoint{0.464439in}{0.557771in}}%
\pgfpathlineto{\pgfqpoint{0.460297in}{0.592289in}}%
\pgfpathlineto{\pgfqpoint{0.456855in}{0.641912in}}%
\pgfpathlineto{\pgfqpoint{0.454122in}{0.716520in}}%
\pgfpathlineto{\pgfqpoint{0.451978in}{0.843444in}}%
\pgfpathlineto{\pgfqpoint{0.450459in}{1.087379in}}%
\pgfpathlineto{\pgfqpoint{0.449596in}{1.657406in}}%
\pgfpathlineto{\pgfqpoint{0.450150in}{2.687936in}}%
\pgfpathlineto{\pgfqpoint{0.451781in}{2.839761in}}%
\pgfpathlineto{\pgfqpoint{0.453975in}{2.872003in}}%
\pgfpathlineto{\pgfqpoint{0.456339in}{2.881553in}}%
\pgfpathlineto{\pgfqpoint{0.458888in}{2.885549in}}%
\pgfpathlineto{\pgfqpoint{0.462554in}{2.888171in}}%
\pgfpathlineto{\pgfqpoint{0.471046in}{2.890205in}}%
\pgfpathlineto{\pgfqpoint{0.490597in}{2.891263in}}%
\pgfpathlineto{\pgfqpoint{0.564556in}{2.891692in}}%
\pgfpathlineto{\pgfqpoint{1.569559in}{2.891759in}}%
\pgfpathlineto{\pgfqpoint{4.784679in}{2.890785in}}%
\pgfpathlineto{\pgfqpoint{4.791004in}{2.889098in}}%
\pgfpathlineto{\pgfqpoint{4.793910in}{2.885555in}}%
\pgfpathlineto{\pgfqpoint{4.795579in}{2.878366in}}%
\pgfpathlineto{\pgfqpoint{4.796850in}{2.858514in}}%
\pgfpathlineto{\pgfqpoint{4.796850in}{2.858514in}}%
\pgfusepath{stroke}%
\end{pgfscope}%
\begin{pgfscope}%
\pgfpathrectangle{\pgfqpoint{0.448634in}{0.402556in}}{\pgfqpoint{4.350661in}{2.489204in}} %
\pgfusepath{clip}%
\pgfsetrectcap%
\pgfsetroundjoin%
\pgfsetlinewidth{1.003750pt}%
\definecolor{currentstroke}{rgb}{0.890196,0.466667,0.760784}%
\pgfsetstrokecolor{currentstroke}%
\pgfsetdash{}{0pt}%
\pgfpathmoveto{\pgfqpoint{0.448634in}{2.896245in}}%
\pgfpathlineto{\pgfqpoint{0.448593in}{0.407043in}}%
\pgfpathlineto{\pgfqpoint{0.448593in}{0.407043in}}%
\pgfusepath{stroke}%
\end{pgfscope}%
\begin{pgfscope}%
\pgfpathrectangle{\pgfqpoint{0.448634in}{0.402556in}}{\pgfqpoint{4.350661in}{2.489204in}} %
\pgfusepath{clip}%
\pgfsetrectcap%
\pgfsetroundjoin%
\pgfsetlinewidth{1.003750pt}%
\definecolor{currentstroke}{rgb}{0.890196,0.466667,0.760784}%
\pgfsetstrokecolor{currentstroke}%
\pgfsetdash{}{0pt}%
\pgfpathmoveto{\pgfqpoint{0.581265in}{1.721678in}}%
\pgfpathlineto{\pgfqpoint{0.573199in}{1.795779in}}%
\pgfpathlineto{\pgfqpoint{0.566265in}{1.880036in}}%
\pgfpathlineto{\pgfqpoint{0.560758in}{1.974413in}}%
\pgfpathlineto{\pgfqpoint{0.557054in}{2.076380in}}%
\pgfpathlineto{\pgfqpoint{0.555430in}{2.183397in}}%
\pgfpathlineto{\pgfqpoint{0.556065in}{2.285449in}}%
\pgfpathlineto{\pgfqpoint{0.558835in}{2.379982in}}%
\pgfpathlineto{\pgfqpoint{0.563404in}{2.461956in}}%
\pgfpathlineto{\pgfqpoint{0.569455in}{2.531304in}}%
\pgfpathlineto{\pgfqpoint{0.576540in}{2.587974in}}%
\pgfpathlineto{\pgfqpoint{0.584450in}{2.634388in}}%
\pgfpathlineto{\pgfqpoint{0.593140in}{2.672947in}}%
\pgfpathlineto{\pgfqpoint{0.602842in}{2.705971in}}%
\pgfpathlineto{\pgfqpoint{0.613219in}{2.733369in}}%
\pgfpathlineto{\pgfqpoint{0.623685in}{2.755179in}}%
\pgfpathlineto{\pgfqpoint{0.634752in}{2.773648in}}%
\pgfpathlineto{\pgfqpoint{0.647498in}{2.790636in}}%
\pgfpathlineto{\pgfqpoint{0.661897in}{2.805796in}}%
\pgfpathlineto{\pgfqpoint{0.677759in}{2.818907in}}%
\pgfpathlineto{\pgfqpoint{0.694783in}{2.829950in}}%
\pgfpathlineto{\pgfqpoint{0.714689in}{2.839960in}}%
\pgfpathlineto{\pgfqpoint{0.737349in}{2.848734in}}%
\pgfpathlineto{\pgfqpoint{0.764749in}{2.856713in}}%
\pgfpathlineto{\pgfqpoint{0.796820in}{2.863566in}}%
\pgfpathlineto{\pgfqpoint{0.835627in}{2.869508in}}%
\pgfpathlineto{\pgfqpoint{0.885445in}{2.874764in}}%
\pgfpathlineto{\pgfqpoint{0.950585in}{2.879248in}}%
\pgfpathlineto{\pgfqpoint{1.039713in}{2.882988in}}%
\pgfpathlineto{\pgfqpoint{1.168029in}{2.885982in}}%
\pgfpathlineto{\pgfqpoint{1.370325in}{2.888239in}}%
\pgfpathlineto{\pgfqpoint{1.733602in}{2.889852in}}%
\pgfpathlineto{\pgfqpoint{2.523247in}{2.890712in}}%
\pgfpathlineto{\pgfqpoint{3.769711in}{2.889981in}}%
\pgfpathlineto{\pgfqpoint{4.176493in}{2.887912in}}%
\pgfpathlineto{\pgfqpoint{4.354850in}{2.884989in}}%
\pgfpathlineto{\pgfqpoint{4.452685in}{2.881304in}}%
\pgfpathlineto{\pgfqpoint{4.515646in}{2.876851in}}%
\pgfpathlineto{\pgfqpoint{4.554602in}{2.872363in}}%
\pgfpathlineto{\pgfqpoint{4.588992in}{2.866278in}}%
\pgfpathlineto{\pgfqpoint{4.614424in}{2.859579in}}%
\pgfpathlineto{\pgfqpoint{4.635088in}{2.851836in}}%
\pgfpathlineto{\pgfqpoint{4.650960in}{2.843692in}}%
\pgfpathlineto{\pgfqpoint{4.665848in}{2.833410in}}%
\pgfpathlineto{\pgfqpoint{4.677713in}{2.822508in}}%
\pgfpathlineto{\pgfqpoint{4.688223in}{2.809917in}}%
\pgfpathlineto{\pgfqpoint{4.698420in}{2.793796in}}%
\pgfpathlineto{\pgfqpoint{4.706877in}{2.776403in}}%
\pgfpathlineto{\pgfqpoint{4.715364in}{2.753495in}}%
\pgfpathlineto{\pgfqpoint{4.723239in}{2.725026in}}%
\pgfpathlineto{\pgfqpoint{4.730107in}{2.691080in}}%
\pgfpathlineto{\pgfqpoint{4.736746in}{2.646929in}}%
\pgfpathlineto{\pgfqpoint{4.742604in}{2.592582in}}%
\pgfpathlineto{\pgfqpoint{4.747970in}{2.520661in}}%
\pgfpathlineto{\pgfqpoint{4.752552in}{2.428712in}}%
\pgfpathlineto{\pgfqpoint{4.756182in}{2.309304in}}%
\pgfpathlineto{\pgfqpoint{4.758511in}{2.157488in}}%
\pgfpathlineto{\pgfqpoint{4.758999in}{1.978267in}}%
\pgfpathlineto{\pgfqpoint{4.757315in}{1.791589in}}%
\pgfpathlineto{\pgfqpoint{4.753521in}{1.614910in}}%
\pgfpathlineto{\pgfqpoint{4.747937in}{1.460716in}}%
\pgfpathlineto{\pgfqpoint{4.741166in}{1.338995in}}%
\pgfpathlineto{\pgfqpoint{4.733254in}{1.237343in}}%
\pgfpathlineto{\pgfqpoint{4.724197in}{1.150846in}}%
\pgfpathlineto{\pgfqpoint{4.714951in}{1.084480in}}%
\pgfpathlineto{\pgfqpoint{4.704590in}{1.025933in}}%
\pgfpathlineto{\pgfqpoint{4.693365in}{0.975268in}}%
\pgfpathlineto{\pgfqpoint{4.681655in}{0.932520in}}%
\pgfpathlineto{\pgfqpoint{4.669262in}{0.895310in}}%
\pgfpathlineto{\pgfqpoint{4.655633in}{0.861393in}}%
\pgfpathlineto{\pgfqpoint{4.640912in}{0.830896in}}%
\pgfpathlineto{\pgfqpoint{4.625396in}{0.803852in}}%
\pgfpathlineto{\pgfqpoint{4.609363in}{0.780288in}}%
\pgfpathlineto{\pgfqpoint{4.591725in}{0.758280in}}%
\pgfpathlineto{\pgfqpoint{4.572578in}{0.737989in}}%
\pgfpathlineto{\pgfqpoint{4.552076in}{0.719512in}}%
\pgfpathlineto{\pgfqpoint{4.530410in}{0.702862in}}%
\pgfpathlineto{\pgfqpoint{4.505851in}{0.686832in}}%
\pgfpathlineto{\pgfqpoint{4.478416in}{0.671722in}}%
\pgfpathlineto{\pgfqpoint{4.439963in}{0.654407in}}%
\pgfpathlineto{\pgfqpoint{4.406842in}{0.642183in}}%
\pgfpathlineto{\pgfqpoint{4.369008in}{0.630654in}}%
\pgfpathlineto{\pgfqpoint{4.326488in}{0.620136in}}%
\pgfpathlineto{\pgfqpoint{4.279325in}{0.610862in}}%
\pgfpathlineto{\pgfqpoint{4.227574in}{0.603000in}}%
\pgfpathlineto{\pgfqpoint{4.173448in}{0.596978in}}%
\pgfpathlineto{\pgfqpoint{4.110509in}{0.592119in}}%
\pgfpathlineto{\pgfqpoint{4.047469in}{0.589450in}}%
\pgfpathlineto{\pgfqpoint{3.977865in}{0.588536in}}%
\pgfpathlineto{\pgfqpoint{3.906091in}{0.589843in}}%
\pgfpathlineto{\pgfqpoint{3.834375in}{0.593403in}}%
\pgfpathlineto{\pgfqpoint{3.767117in}{0.598970in}}%
\pgfpathlineto{\pgfqpoint{3.704362in}{0.606293in}}%
\pgfpathlineto{\pgfqpoint{3.682839in}{0.609829in}}%
\pgfpathlineto{\pgfqpoint{3.624737in}{0.619642in}}%
\pgfpathlineto{\pgfqpoint{3.588418in}{0.627483in}}%
\pgfpathlineto{\pgfqpoint{3.575673in}{0.630667in}}%
\pgfpathlineto{\pgfqpoint{3.522592in}{0.644152in}}%
\pgfpathlineto{\pgfqpoint{3.476421in}{0.658549in}}%
\pgfpathlineto{\pgfqpoint{3.437148in}{0.673272in}}%
\pgfpathlineto{\pgfqpoint{3.394483in}{0.691888in}}%
\pgfpathlineto{\pgfqpoint{3.360887in}{0.709559in}}%
\pgfpathlineto{\pgfqpoint{3.330177in}{0.728291in}}%
\pgfpathlineto{\pgfqpoint{3.302452in}{0.747967in}}%
\pgfpathlineto{\pgfqpoint{3.277652in}{0.768184in}}%
\pgfpathlineto{\pgfqpoint{3.254080in}{0.790239in}}%
\pgfpathlineto{\pgfqpoint{3.231904in}{0.814113in}}%
\pgfpathlineto{\pgfqpoint{3.211272in}{0.839736in}}%
\pgfpathlineto{\pgfqpoint{3.192276in}{0.866965in}}%
\pgfpathlineto{\pgfqpoint{3.175016in}{0.895668in}}%
\pgfpathlineto{\pgfqpoint{3.157408in}{0.930012in}}%
\pgfpathlineto{\pgfqpoint{3.142851in}{0.963422in}}%
\pgfpathlineto{\pgfqpoint{3.129375in}{1.000135in}}%
\pgfpathlineto{\pgfqpoint{3.117913in}{1.037734in}}%
\pgfpathlineto{\pgfqpoint{3.107163in}{1.080812in}}%
\pgfpathlineto{\pgfqpoint{3.098545in}{1.124513in}}%
\pgfpathlineto{\pgfqpoint{3.091608in}{1.171130in}}%
\pgfpathlineto{\pgfqpoint{3.086343in}{1.220544in}}%
\pgfpathlineto{\pgfqpoint{3.082974in}{1.272670in}}%
\pgfpathlineto{\pgfqpoint{3.081645in}{1.327406in}}%
\pgfpathlineto{\pgfqpoint{3.082529in}{1.384644in}}%
\pgfpathlineto{\pgfqpoint{3.085619in}{1.441782in}}%
\pgfpathlineto{\pgfqpoint{3.091167in}{1.501179in}}%
\pgfpathlineto{\pgfqpoint{3.099087in}{1.560222in}}%
\pgfpathlineto{\pgfqpoint{3.108986in}{1.616337in}}%
\pgfpathlineto{\pgfqpoint{3.120552in}{1.669472in}}%
\pgfpathlineto{\pgfqpoint{3.134326in}{1.721911in}}%
\pgfpathlineto{\pgfqpoint{3.149590in}{1.771175in}}%
\pgfpathlineto{\pgfqpoint{3.166110in}{1.817225in}}%
\pgfpathlineto{\pgfqpoint{3.184656in}{1.862253in}}%
\pgfpathlineto{\pgfqpoint{3.204162in}{1.903944in}}%
\pgfpathlineto{\pgfqpoint{3.225489in}{1.944450in}}%
\pgfpathlineto{\pgfqpoint{3.248626in}{1.983634in}}%
\pgfpathlineto{\pgfqpoint{3.273489in}{2.021408in}}%
\pgfpathlineto{\pgfqpoint{3.301375in}{2.059615in}}%
\pgfpathlineto{\pgfqpoint{3.332268in}{2.098116in}}%
\pgfpathlineto{\pgfqpoint{3.375168in}{2.147590in}}%
\pgfpathlineto{\pgfqpoint{3.410081in}{2.188587in}}%
\pgfpathlineto{\pgfqpoint{3.422453in}{2.205936in}}%
\pgfpathlineto{\pgfqpoint{3.429196in}{2.218699in}}%
\pgfpathlineto{\pgfqpoint{3.431848in}{2.228147in}}%
\pgfpathlineto{\pgfqpoint{3.431550in}{2.235551in}}%
\pgfpathlineto{\pgfqpoint{3.429667in}{2.240011in}}%
\pgfpathlineto{\pgfqpoint{3.424820in}{2.244932in}}%
\pgfpathlineto{\pgfqpoint{3.416758in}{2.248582in}}%
\pgfpathlineto{\pgfqpoint{3.406018in}{2.250440in}}%
\pgfpathlineto{\pgfqpoint{3.390803in}{2.250343in}}%
\pgfpathlineto{\pgfqpoint{3.373534in}{2.247944in}}%
\pgfpathlineto{\pgfqpoint{3.352277in}{2.242695in}}%
\pgfpathlineto{\pgfqpoint{3.329375in}{2.234783in}}%
\pgfpathlineto{\pgfqpoint{3.305006in}{2.224092in}}%
\pgfpathlineto{\pgfqpoint{3.279376in}{2.210431in}}%
\pgfpathlineto{\pgfqpoint{3.252732in}{2.193567in}}%
\pgfpathlineto{\pgfqpoint{3.227148in}{2.174674in}}%
\pgfpathlineto{\pgfqpoint{3.202689in}{2.153921in}}%
\pgfpathlineto{\pgfqpoint{3.177794in}{2.129793in}}%
\pgfpathlineto{\pgfqpoint{3.154306in}{2.103884in}}%
\pgfpathlineto{\pgfqpoint{3.132252in}{2.076373in}}%
\pgfpathlineto{\pgfqpoint{3.110313in}{2.045462in}}%
\pgfpathlineto{\pgfqpoint{3.088772in}{2.011074in}}%
\pgfpathlineto{\pgfqpoint{3.068998in}{1.975322in}}%
\pgfpathlineto{\pgfqpoint{3.049897in}{1.936214in}}%
\pgfpathlineto{\pgfqpoint{3.031658in}{1.893780in}}%
\pgfpathlineto{\pgfqpoint{3.014449in}{1.848061in}}%
\pgfpathlineto{\pgfqpoint{2.998389in}{1.799129in}}%
\pgfpathlineto{\pgfqpoint{2.983576in}{1.747061in}}%
\pgfpathlineto{\pgfqpoint{2.969537in}{1.689525in}}%
\pgfpathlineto{\pgfqpoint{2.957039in}{1.628965in}}%
\pgfpathlineto{\pgfqpoint{2.945744in}{1.563015in}}%
\pgfpathlineto{\pgfqpoint{2.936236in}{1.494175in}}%
\pgfpathlineto{\pgfqpoint{2.928581in}{1.422525in}}%
\pgfpathlineto{\pgfqpoint{2.922885in}{1.348138in}}%
\pgfpathlineto{\pgfqpoint{2.919413in}{1.273571in}}%
\pgfpathlineto{\pgfqpoint{2.918210in}{1.201401in}}%
\pgfpathlineto{\pgfqpoint{2.919192in}{1.131716in}}%
\pgfpathlineto{\pgfqpoint{2.922359in}{1.064610in}}%
\pgfpathlineto{\pgfqpoint{2.927525in}{1.002666in}}%
\pgfpathlineto{\pgfqpoint{2.934171in}{0.948439in}}%
\pgfpathlineto{\pgfqpoint{2.942707in}{0.897092in}}%
\pgfpathlineto{\pgfqpoint{2.952649in}{0.851193in}}%
\pgfpathlineto{\pgfqpoint{2.963679in}{0.810809in}}%
\pgfpathlineto{\pgfqpoint{2.975377in}{0.775960in}}%
\pgfpathlineto{\pgfqpoint{2.988197in}{0.744357in}}%
\pgfpathlineto{\pgfqpoint{3.001924in}{0.716074in}}%
\pgfpathlineto{\pgfqpoint{3.017533in}{0.689102in}}%
\pgfpathlineto{\pgfqpoint{3.033664in}{0.665627in}}%
\pgfpathlineto{\pgfqpoint{3.051423in}{0.643747in}}%
\pgfpathlineto{\pgfqpoint{3.070711in}{0.623632in}}%
\pgfpathlineto{\pgfqpoint{3.091365in}{0.605377in}}%
\pgfpathlineto{\pgfqpoint{3.113183in}{0.588991in}}%
\pgfpathlineto{\pgfqpoint{3.137889in}{0.573260in}}%
\pgfpathlineto{\pgfqpoint{3.165464in}{0.558484in}}%
\pgfpathlineto{\pgfqpoint{3.195840in}{0.544862in}}%
\pgfpathlineto{\pgfqpoint{3.228917in}{0.532485in}}%
\pgfpathlineto{\pgfqpoint{3.266707in}{0.520770in}}%
\pgfpathlineto{\pgfqpoint{3.309175in}{0.509977in}}%
\pgfpathlineto{\pgfqpoint{3.358416in}{0.499850in}}%
\pgfpathlineto{\pgfqpoint{3.414401in}{0.490680in}}%
\pgfpathlineto{\pgfqpoint{3.479253in}{0.482363in}}%
\pgfpathlineto{\pgfqpoint{3.555111in}{0.474946in}}%
\pgfpathlineto{\pgfqpoint{3.644124in}{0.468561in}}%
\pgfpathlineto{\pgfqpoint{3.746267in}{0.463491in}}%
\pgfpathlineto{\pgfqpoint{3.863692in}{0.459910in}}%
\pgfpathlineto{\pgfqpoint{3.992026in}{0.458217in}}%
\pgfpathlineto{\pgfqpoint{4.120369in}{0.458704in}}%
\pgfpathlineto{\pgfqpoint{4.237813in}{0.461322in}}%
\pgfpathlineto{\pgfqpoint{4.335629in}{0.465634in}}%
\pgfpathlineto{\pgfqpoint{4.415958in}{0.471353in}}%
\pgfpathlineto{\pgfqpoint{4.478775in}{0.477957in}}%
\pgfpathlineto{\pgfqpoint{4.528397in}{0.485252in}}%
\pgfpathlineto{\pgfqpoint{4.569112in}{0.493359in}}%
\pgfpathlineto{\pgfqpoint{4.603015in}{0.502343in}}%
\pgfpathlineto{\pgfqpoint{4.630078in}{0.511708in}}%
\pgfpathlineto{\pgfqpoint{4.652383in}{0.521599in}}%
\pgfpathlineto{\pgfqpoint{4.671887in}{0.532602in}}%
\pgfpathlineto{\pgfqpoint{4.688479in}{0.544473in}}%
\pgfpathlineto{\pgfqpoint{4.702160in}{0.556766in}}%
\pgfpathlineto{\pgfqpoint{4.714565in}{0.570717in}}%
\pgfpathlineto{\pgfqpoint{4.725501in}{0.586194in}}%
\pgfpathlineto{\pgfqpoint{4.735978in}{0.605104in}}%
\pgfpathlineto{\pgfqpoint{4.745497in}{0.627473in}}%
\pgfpathlineto{\pgfqpoint{4.753787in}{0.653148in}}%
\pgfpathlineto{\pgfqpoint{4.761305in}{0.684335in}}%
\pgfpathlineto{\pgfqpoint{4.767778in}{0.720925in}}%
\pgfpathlineto{\pgfqpoint{4.773698in}{0.767727in}}%
\pgfpathlineto{\pgfqpoint{4.778881in}{0.827169in}}%
\pgfpathlineto{\pgfqpoint{4.783445in}{0.906649in}}%
\pgfpathlineto{\pgfqpoint{4.787336in}{1.016082in}}%
\pgfpathlineto{\pgfqpoint{4.790532in}{1.172858in}}%
\pgfpathlineto{\pgfqpoint{4.792964in}{1.404336in}}%
\pgfpathlineto{\pgfqpoint{4.794628in}{1.790158in}}%
\pgfpathlineto{\pgfqpoint{4.794757in}{2.345250in}}%
\pgfpathlineto{\pgfqpoint{4.792993in}{2.671327in}}%
\pgfpathlineto{\pgfqpoint{4.790462in}{2.788280in}}%
\pgfpathlineto{\pgfqpoint{4.787557in}{2.835450in}}%
\pgfpathlineto{\pgfqpoint{4.784407in}{2.857547in}}%
\pgfpathlineto{\pgfqpoint{4.781030in}{2.869359in}}%
\pgfpathlineto{\pgfqpoint{4.776344in}{2.877698in}}%
\pgfpathlineto{\pgfqpoint{4.771269in}{2.882350in}}%
\pgfpathlineto{\pgfqpoint{4.763220in}{2.886057in}}%
\pgfpathlineto{\pgfqpoint{4.750376in}{2.888628in}}%
\pgfpathlineto{\pgfqpoint{4.728672in}{2.890224in}}%
\pgfpathlineto{\pgfqpoint{4.676473in}{2.891225in}}%
\pgfpathlineto{\pgfqpoint{4.485045in}{2.891665in}}%
\pgfpathlineto{\pgfqpoint{2.250981in}{2.891751in}}%
\pgfpathlineto{\pgfqpoint{0.608609in}{2.890634in}}%
\pgfpathlineto{\pgfqpoint{0.545554in}{2.888587in}}%
\pgfpathlineto{\pgfqpoint{0.519569in}{2.885838in}}%
\pgfpathlineto{\pgfqpoint{0.504655in}{2.882384in}}%
\pgfpathlineto{\pgfqpoint{0.494491in}{2.878003in}}%
\pgfpathlineto{\pgfqpoint{0.487172in}{2.872655in}}%
\pgfpathlineto{\pgfqpoint{0.481146in}{2.865505in}}%
\pgfpathlineto{\pgfqpoint{0.475661in}{2.854789in}}%
\pgfpathlineto{\pgfqpoint{0.471317in}{2.840721in}}%
\pgfpathlineto{\pgfqpoint{0.467301in}{2.818806in}}%
\pgfpathlineto{\pgfqpoint{0.463928in}{2.786684in}}%
\pgfpathlineto{\pgfqpoint{0.460919in}{2.734528in}}%
\pgfpathlineto{\pgfqpoint{0.458363in}{2.647457in}}%
\pgfpathlineto{\pgfqpoint{0.456575in}{2.523014in}}%
\pgfpathlineto{\pgfqpoint{0.456575in}{2.523014in}}%
\pgfusepath{stroke}%
\end{pgfscope}%
\begin{pgfscope}%
\pgfpathrectangle{\pgfqpoint{0.448634in}{0.402556in}}{\pgfqpoint{4.350661in}{2.489204in}} %
\pgfusepath{clip}%
\pgfsetrectcap%
\pgfsetroundjoin%
\pgfsetlinewidth{1.003750pt}%
\definecolor{currentstroke}{rgb}{0.890196,0.466667,0.760784}%
\pgfsetstrokecolor{currentstroke}%
\pgfsetdash{}{0pt}%
\pgfpathmoveto{\pgfqpoint{4.798840in}{2.852369in}}%
\pgfpathlineto{\pgfqpoint{4.797564in}{2.889610in}}%
\pgfpathlineto{\pgfqpoint{4.796215in}{2.891483in}}%
\pgfpathlineto{\pgfqpoint{4.787551in}{2.891760in}}%
\pgfpathlineto{\pgfqpoint{0.452129in}{2.891653in}}%
\pgfpathlineto{\pgfqpoint{0.450532in}{2.890074in}}%
\pgfpathlineto{\pgfqpoint{0.449456in}{2.882754in}}%
\pgfpathlineto{\pgfqpoint{0.448970in}{2.845423in}}%
\pgfpathlineto{\pgfqpoint{0.448743in}{2.494445in}}%
\pgfpathlineto{\pgfqpoint{0.449609in}{0.617587in}}%
\pgfpathlineto{\pgfqpoint{0.451440in}{0.510578in}}%
\pgfpathlineto{\pgfqpoint{0.454007in}{0.473366in}}%
\pgfpathlineto{\pgfqpoint{0.457429in}{0.453862in}}%
\pgfpathlineto{\pgfqpoint{0.461572in}{0.442383in}}%
\pgfpathlineto{\pgfqpoint{0.466779in}{0.434443in}}%
\pgfpathlineto{\pgfqpoint{0.473639in}{0.428362in}}%
\pgfpathlineto{\pgfqpoint{0.483539in}{0.423260in}}%
\pgfpathlineto{\pgfqpoint{0.491900in}{0.420518in}}%
\pgfpathlineto{\pgfqpoint{0.491900in}{0.420518in}}%
\pgfusepath{stroke}%
\end{pgfscope}%
\begin{pgfscope}%
\pgfpathrectangle{\pgfqpoint{0.448634in}{0.402556in}}{\pgfqpoint{4.350661in}{2.489204in}} %
\pgfusepath{clip}%
\pgfsetrectcap%
\pgfsetroundjoin%
\pgfsetlinewidth{1.003750pt}%
\definecolor{currentstroke}{rgb}{0.890196,0.466667,0.760784}%
\pgfsetstrokecolor{currentstroke}%
\pgfsetdash{}{0pt}%
\pgfpathmoveto{\pgfqpoint{0.456425in}{1.370076in}}%
\pgfpathlineto{\pgfqpoint{0.459613in}{1.118694in}}%
\pgfpathlineto{\pgfqpoint{0.463699in}{0.961947in}}%
\pgfpathlineto{\pgfqpoint{0.468525in}{0.857549in}}%
\pgfpathlineto{\pgfqpoint{0.474091in}{0.783150in}}%
\pgfpathlineto{\pgfqpoint{0.480237in}{0.728846in}}%
\pgfpathlineto{\pgfqpoint{0.486986in}{0.687247in}}%
\pgfpathlineto{\pgfqpoint{0.494557in}{0.653501in}}%
\pgfpathlineto{\pgfqpoint{0.503133in}{0.625299in}}%
\pgfpathlineto{\pgfqpoint{0.512224in}{0.602697in}}%
\pgfpathlineto{\pgfqpoint{0.522237in}{0.583459in}}%
\pgfpathlineto{\pgfqpoint{0.534150in}{0.565699in}}%
\pgfpathlineto{\pgfqpoint{0.546311in}{0.551469in}}%
\pgfpathlineto{\pgfqpoint{0.559779in}{0.538874in}}%
\pgfpathlineto{\pgfqpoint{0.576184in}{0.526666in}}%
\pgfpathlineto{\pgfqpoint{0.595540in}{0.515329in}}%
\pgfpathlineto{\pgfqpoint{0.617739in}{0.505130in}}%
\pgfpathlineto{\pgfqpoint{0.642628in}{0.496141in}}%
\pgfpathlineto{\pgfqpoint{0.672186in}{0.487769in}}%
\pgfpathlineto{\pgfqpoint{0.708504in}{0.479817in}}%
\pgfpathlineto{\pgfqpoint{0.753711in}{0.472321in}}%
\pgfpathlineto{\pgfqpoint{0.807779in}{0.465658in}}%
\pgfpathlineto{\pgfqpoint{0.877177in}{0.459474in}}%
\pgfpathlineto{\pgfqpoint{0.961890in}{0.454230in}}%
\pgfpathlineto{\pgfqpoint{1.068413in}{0.449917in}}%
\pgfpathlineto{\pgfqpoint{1.201080in}{0.446841in}}%
\pgfpathlineto{\pgfqpoint{1.357698in}{0.445484in}}%
\pgfpathlineto{\pgfqpoint{1.525197in}{0.446236in}}%
\pgfpathlineto{\pgfqpoint{1.686150in}{0.449147in}}%
\pgfpathlineto{\pgfqpoint{1.823135in}{0.453753in}}%
\pgfpathlineto{\pgfqpoint{1.938306in}{0.459773in}}%
\pgfpathlineto{\pgfqpoint{2.031644in}{0.466769in}}%
\pgfpathlineto{\pgfqpoint{2.109641in}{0.474758in}}%
\pgfpathlineto{\pgfqpoint{2.174445in}{0.483550in}}%
\pgfpathlineto{\pgfqpoint{2.228200in}{0.492959in}}%
\pgfpathlineto{\pgfqpoint{2.275179in}{0.503379in}}%
\pgfpathlineto{\pgfqpoint{2.315341in}{0.514529in}}%
\pgfpathlineto{\pgfqpoint{2.350756in}{0.526691in}}%
\pgfpathlineto{\pgfqpoint{2.381376in}{0.539573in}}%
\pgfpathlineto{\pgfqpoint{2.407218in}{0.552701in}}%
\pgfpathlineto{\pgfqpoint{2.430278in}{0.566687in}}%
\pgfpathlineto{\pgfqpoint{2.452330in}{0.582655in}}%
\pgfpathlineto{\pgfqpoint{2.471436in}{0.599128in}}%
\pgfpathlineto{\pgfqpoint{2.489281in}{0.617357in}}%
\pgfpathlineto{\pgfqpoint{2.505714in}{0.637249in}}%
\pgfpathlineto{\pgfqpoint{2.520653in}{0.658629in}}%
\pgfpathlineto{\pgfqpoint{2.535242in}{0.683389in}}%
\pgfpathlineto{\pgfqpoint{2.549139in}{0.711563in}}%
\pgfpathlineto{\pgfqpoint{2.562111in}{0.743084in}}%
\pgfpathlineto{\pgfqpoint{2.574037in}{0.777833in}}%
\pgfpathlineto{\pgfqpoint{2.585516in}{0.818054in}}%
\pgfpathlineto{\pgfqpoint{2.596819in}{0.866123in}}%
\pgfpathlineto{\pgfqpoint{2.607570in}{0.922033in}}%
\pgfpathlineto{\pgfqpoint{2.617930in}{0.988184in}}%
\pgfpathlineto{\pgfqpoint{2.627961in}{1.067004in}}%
\pgfpathlineto{\pgfqpoint{2.637943in}{1.163406in}}%
\pgfpathlineto{\pgfqpoint{2.648425in}{1.287285in}}%
\pgfpathlineto{\pgfqpoint{2.660105in}{1.453524in}}%
\pgfpathlineto{\pgfqpoint{2.674776in}{1.696887in}}%
\pgfpathlineto{\pgfqpoint{2.687720in}{1.945365in}}%
\pgfpathlineto{\pgfqpoint{2.692672in}{2.079659in}}%
\pgfpathlineto{\pgfqpoint{2.693829in}{2.166768in}}%
\pgfpathlineto{\pgfqpoint{2.692561in}{2.233956in}}%
\pgfpathlineto{\pgfqpoint{2.689428in}{2.286100in}}%
\pgfpathlineto{\pgfqpoint{2.684846in}{2.328084in}}%
\pgfpathlineto{\pgfqpoint{2.678706in}{2.364747in}}%
\pgfpathlineto{\pgfqpoint{2.671332in}{2.395979in}}%
\pgfpathlineto{\pgfqpoint{2.662458in}{2.424060in}}%
\pgfpathlineto{\pgfqpoint{2.652324in}{2.448854in}}%
\pgfpathlineto{\pgfqpoint{2.641321in}{2.470317in}}%
\pgfpathlineto{\pgfqpoint{2.628593in}{2.490491in}}%
\pgfpathlineto{\pgfqpoint{2.614223in}{2.509166in}}%
\pgfpathlineto{\pgfqpoint{2.598383in}{2.526214in}}%
\pgfpathlineto{\pgfqpoint{2.579525in}{2.543054in}}%
\pgfpathlineto{\pgfqpoint{2.559464in}{2.557966in}}%
\pgfpathlineto{\pgfqpoint{2.536531in}{2.572220in}}%
\pgfpathlineto{\pgfqpoint{2.510777in}{2.585570in}}%
\pgfpathlineto{\pgfqpoint{2.482285in}{2.597864in}}%
\pgfpathlineto{\pgfqpoint{2.449059in}{2.609705in}}%
\pgfpathlineto{\pgfqpoint{2.411108in}{2.620714in}}%
\pgfpathlineto{\pgfqpoint{2.368475in}{2.630620in}}%
\pgfpathlineto{\pgfqpoint{2.321216in}{2.639231in}}%
\pgfpathlineto{\pgfqpoint{2.269389in}{2.646406in}}%
\pgfpathlineto{\pgfqpoint{2.210876in}{2.652197in}}%
\pgfpathlineto{\pgfqpoint{2.147889in}{2.656155in}}%
\pgfpathlineto{\pgfqpoint{2.080478in}{2.658134in}}%
\pgfpathlineto{\pgfqpoint{2.010870in}{2.657968in}}%
\pgfpathlineto{\pgfqpoint{1.939117in}{2.655567in}}%
\pgfpathlineto{\pgfqpoint{1.867449in}{2.650905in}}%
\pgfpathlineto{\pgfqpoint{1.798093in}{2.644130in}}%
\pgfpathlineto{\pgfqpoint{1.733263in}{2.635592in}}%
\pgfpathlineto{\pgfqpoint{1.672998in}{2.625505in}}%
\pgfpathlineto{\pgfqpoint{1.615197in}{2.613591in}}%
\pgfpathlineto{\pgfqpoint{1.562057in}{2.600379in}}%
\pgfpathlineto{\pgfqpoint{1.513606in}{2.586112in}}%
\pgfpathlineto{\pgfqpoint{1.467787in}{2.570313in}}%
\pgfpathlineto{\pgfqpoint{1.426720in}{2.553888in}}%
\pgfpathlineto{\pgfqpoint{1.388375in}{2.536250in}}%
\pgfpathlineto{\pgfqpoint{1.352808in}{2.517523in}}%
\pgfpathlineto{\pgfqpoint{1.320060in}{2.497874in}}%
\pgfpathlineto{\pgfqpoint{1.288314in}{2.476183in}}%
\pgfpathlineto{\pgfqpoint{1.259529in}{2.453804in}}%
\pgfpathlineto{\pgfqpoint{1.231991in}{2.429458in}}%
\pgfpathlineto{\pgfqpoint{1.207471in}{2.404832in}}%
\pgfpathlineto{\pgfqpoint{1.184356in}{2.378487in}}%
\pgfpathlineto{\pgfqpoint{1.162779in}{2.350487in}}%
\pgfpathlineto{\pgfqpoint{1.142846in}{2.320934in}}%
\pgfpathlineto{\pgfqpoint{1.124634in}{2.289961in}}%
\pgfpathlineto{\pgfqpoint{1.108188in}{2.257719in}}%
\pgfpathlineto{\pgfqpoint{1.092606in}{2.222114in}}%
\pgfpathlineto{\pgfqpoint{1.079031in}{2.185448in}}%
\pgfpathlineto{\pgfqpoint{1.067418in}{2.147909in}}%
\pgfpathlineto{\pgfqpoint{1.057166in}{2.107257in}}%
\pgfpathlineto{\pgfqpoint{1.048987in}{2.065995in}}%
\pgfpathlineto{\pgfqpoint{1.042499in}{2.021814in}}%
\pgfpathlineto{\pgfqpoint{1.038167in}{1.977289in}}%
\pgfpathlineto{\pgfqpoint{1.035860in}{1.930074in}}%
\pgfpathlineto{\pgfqpoint{1.035825in}{1.882785in}}%
\pgfpathlineto{\pgfqpoint{1.038034in}{1.835564in}}%
\pgfpathlineto{\pgfqpoint{1.042481in}{1.788550in}}%
\pgfpathlineto{\pgfqpoint{1.049187in}{1.741888in}}%
\pgfpathlineto{\pgfqpoint{1.057660in}{1.698149in}}%
\pgfpathlineto{\pgfqpoint{1.068242in}{1.655017in}}%
\pgfpathlineto{\pgfqpoint{1.080988in}{1.612658in}}%
\pgfpathlineto{\pgfqpoint{1.095061in}{1.573532in}}%
\pgfpathlineto{\pgfqpoint{1.111151in}{1.535438in}}%
\pgfpathlineto{\pgfqpoint{1.128158in}{1.500696in}}%
\pgfpathlineto{\pgfqpoint{1.146974in}{1.467198in}}%
\pgfpathlineto{\pgfqpoint{1.167580in}{1.435109in}}%
\pgfpathlineto{\pgfqpoint{1.189927in}{1.404584in}}%
\pgfpathlineto{\pgfqpoint{1.213941in}{1.375764in}}%
\pgfpathlineto{\pgfqpoint{1.237877in}{1.350397in}}%
\pgfpathlineto{\pgfqpoint{1.264812in}{1.325183in}}%
\pgfpathlineto{\pgfqpoint{1.293057in}{1.301922in}}%
\pgfpathlineto{\pgfqpoint{1.322467in}{1.280633in}}%
\pgfpathlineto{\pgfqpoint{1.352892in}{1.261301in}}%
\pgfpathlineto{\pgfqpoint{1.386169in}{1.242855in}}%
\pgfpathlineto{\pgfqpoint{1.420267in}{1.226488in}}%
\pgfpathlineto{\pgfqpoint{1.457102in}{1.211307in}}%
\pgfpathlineto{\pgfqpoint{1.496634in}{1.197521in}}%
\pgfpathlineto{\pgfqpoint{1.538801in}{1.185277in}}%
\pgfpathlineto{\pgfqpoint{1.583523in}{1.174638in}}%
\pgfpathlineto{\pgfqpoint{1.635012in}{1.164779in}}%
\pgfpathlineto{\pgfqpoint{1.706147in}{1.153754in}}%
\pgfpathlineto{\pgfqpoint{1.768576in}{1.143424in}}%
\pgfpathlineto{\pgfqpoint{1.796205in}{1.136568in}}%
\pgfpathlineto{\pgfqpoint{1.812764in}{1.130474in}}%
\pgfpathlineto{\pgfqpoint{1.824548in}{1.124087in}}%
\pgfpathlineto{\pgfqpoint{1.833279in}{1.116714in}}%
\pgfpathlineto{\pgfqpoint{1.838555in}{1.108852in}}%
\pgfpathlineto{\pgfqpoint{1.840632in}{1.101806in}}%
\pgfpathlineto{\pgfqpoint{1.840649in}{1.094369in}}%
\pgfpathlineto{\pgfqpoint{1.837948in}{1.084947in}}%
\pgfpathlineto{\pgfqpoint{1.833257in}{1.076581in}}%
\pgfpathlineto{\pgfqpoint{1.825825in}{1.067513in}}%
\pgfpathlineto{\pgfqpoint{1.813816in}{1.056825in}}%
\pgfpathlineto{\pgfqpoint{1.798821in}{1.046741in}}%
\pgfpathlineto{\pgfqpoint{1.781016in}{1.037443in}}%
\pgfpathlineto{\pgfqpoint{1.758447in}{1.028374in}}%
\pgfpathlineto{\pgfqpoint{1.733202in}{1.020800in}}%
\pgfpathlineto{\pgfqpoint{1.705409in}{1.014858in}}%
\pgfpathlineto{\pgfqpoint{1.675177in}{1.010702in}}%
\pgfpathlineto{\pgfqpoint{1.642609in}{1.008497in}}%
\pgfpathlineto{\pgfqpoint{1.607808in}{1.008425in}}%
\pgfpathlineto{\pgfqpoint{1.570885in}{1.010686in}}%
\pgfpathlineto{\pgfqpoint{1.534117in}{1.015177in}}%
\pgfpathlineto{\pgfqpoint{1.495454in}{1.022230in}}%
\pgfpathlineto{\pgfqpoint{1.457161in}{1.031561in}}%
\pgfpathlineto{\pgfqpoint{1.419337in}{1.043130in}}%
\pgfpathlineto{\pgfqpoint{1.382089in}{1.056928in}}%
\pgfpathlineto{\pgfqpoint{1.347544in}{1.072018in}}%
\pgfpathlineto{\pgfqpoint{1.313727in}{1.089133in}}%
\pgfpathlineto{\pgfqpoint{1.280762in}{1.108299in}}%
\pgfpathlineto{\pgfqpoint{1.248782in}{1.129537in}}%
\pgfpathlineto{\pgfqpoint{1.219709in}{1.151423in}}%
\pgfpathlineto{\pgfqpoint{1.191753in}{1.175139in}}%
\pgfpathlineto{\pgfqpoint{1.165031in}{1.200651in}}%
\pgfpathlineto{\pgfqpoint{1.139654in}{1.227900in}}%
\pgfpathlineto{\pgfqpoint{1.115716in}{1.256803in}}%
\pgfpathlineto{\pgfqpoint{1.093291in}{1.287255in}}%
\pgfpathlineto{\pgfqpoint{1.071181in}{1.321167in}}%
\pgfpathlineto{\pgfqpoint{1.050873in}{1.356525in}}%
\pgfpathlineto{\pgfqpoint{1.032371in}{1.393158in}}%
\pgfpathlineto{\pgfqpoint{1.014725in}{1.433149in}}%
\pgfpathlineto{\pgfqpoint{0.999033in}{1.474193in}}%
\pgfpathlineto{\pgfqpoint{0.984516in}{1.518469in}}%
\pgfpathlineto{\pgfqpoint{0.972020in}{1.563545in}}%
\pgfpathlineto{\pgfqpoint{0.960955in}{1.611687in}}%
\pgfpathlineto{\pgfqpoint{0.951542in}{1.662833in}}%
\pgfpathlineto{\pgfqpoint{0.944298in}{1.714440in}}%
\pgfpathlineto{\pgfqpoint{0.938962in}{1.768856in}}%
\pgfpathlineto{\pgfqpoint{0.935883in}{1.823500in}}%
\pgfpathlineto{\pgfqpoint{0.935047in}{1.878249in}}%
\pgfpathlineto{\pgfqpoint{0.936479in}{1.932981in}}%
\pgfpathlineto{\pgfqpoint{0.940019in}{1.985093in}}%
\pgfpathlineto{\pgfqpoint{0.945773in}{2.036944in}}%
\pgfpathlineto{\pgfqpoint{0.953424in}{2.085947in}}%
\pgfpathlineto{\pgfqpoint{0.962780in}{2.132009in}}%
\pgfpathlineto{\pgfqpoint{0.974303in}{2.177422in}}%
\pgfpathlineto{\pgfqpoint{0.987350in}{2.219660in}}%
\pgfpathlineto{\pgfqpoint{1.001686in}{2.258661in}}%
\pgfpathlineto{\pgfqpoint{1.018070in}{2.296590in}}%
\pgfpathlineto{\pgfqpoint{1.035422in}{2.331107in}}%
\pgfpathlineto{\pgfqpoint{1.054671in}{2.364281in}}%
\pgfpathlineto{\pgfqpoint{1.074428in}{2.393988in}}%
\pgfpathlineto{\pgfqpoint{1.095794in}{2.422201in}}%
\pgfpathlineto{\pgfqpoint{1.118685in}{2.448800in}}%
\pgfpathlineto{\pgfqpoint{1.142990in}{2.473703in}}%
\pgfpathlineto{\pgfqpoint{1.168575in}{2.496869in}}%
\pgfpathlineto{\pgfqpoint{1.197109in}{2.519663in}}%
\pgfpathlineto{\pgfqpoint{1.226751in}{2.540527in}}%
\pgfpathlineto{\pgfqpoint{1.259267in}{2.560673in}}%
\pgfpathlineto{\pgfqpoint{1.294638in}{2.579881in}}%
\pgfpathlineto{\pgfqpoint{1.332818in}{2.597981in}}%
\pgfpathlineto{\pgfqpoint{1.373744in}{2.614858in}}%
\pgfpathlineto{\pgfqpoint{1.417345in}{2.630443in}}%
\pgfpathlineto{\pgfqpoint{1.465658in}{2.645309in}}%
\pgfpathlineto{\pgfqpoint{1.518666in}{2.659201in}}%
\pgfpathlineto{\pgfqpoint{1.576335in}{2.671926in}}%
\pgfpathlineto{\pgfqpoint{1.638624in}{2.683340in}}%
\pgfpathlineto{\pgfqpoint{1.705488in}{2.693338in}}%
\pgfpathlineto{\pgfqpoint{1.779054in}{2.702059in}}%
\pgfpathlineto{\pgfqpoint{1.857124in}{2.709071in}}%
\pgfpathlineto{\pgfqpoint{1.939659in}{2.714274in}}%
\pgfpathlineto{\pgfqpoint{2.026625in}{2.717507in}}%
\pgfpathlineto{\pgfqpoint{2.113632in}{2.718516in}}%
\pgfpathlineto{\pgfqpoint{2.198461in}{2.717294in}}%
\pgfpathlineto{\pgfqpoint{2.278892in}{2.713919in}}%
\pgfpathlineto{\pgfqpoint{2.352704in}{2.708586in}}%
\pgfpathlineto{\pgfqpoint{2.417683in}{2.701696in}}%
\pgfpathlineto{\pgfqpoint{2.473796in}{2.693615in}}%
\pgfpathlineto{\pgfqpoint{2.523166in}{2.684350in}}%
\pgfpathlineto{\pgfqpoint{2.565751in}{2.674182in}}%
\pgfpathlineto{\pgfqpoint{2.601534in}{2.663520in}}%
\pgfpathlineto{\pgfqpoint{2.632600in}{2.652116in}}%
\pgfpathlineto{\pgfqpoint{2.658921in}{2.640301in}}%
\pgfpathlineto{\pgfqpoint{2.682458in}{2.627402in}}%
\pgfpathlineto{\pgfqpoint{2.703081in}{2.613533in}}%
\pgfpathlineto{\pgfqpoint{2.720690in}{2.598936in}}%
\pgfpathlineto{\pgfqpoint{2.735275in}{2.584006in}}%
\pgfpathlineto{\pgfqpoint{2.748328in}{2.567326in}}%
\pgfpathlineto{\pgfqpoint{2.759558in}{2.548992in}}%
\pgfpathlineto{\pgfqpoint{2.768788in}{2.529250in}}%
\pgfpathlineto{\pgfqpoint{2.776013in}{2.508440in}}%
\pgfpathlineto{\pgfqpoint{2.781876in}{2.484480in}}%
\pgfpathlineto{\pgfqpoint{2.786091in}{2.457536in}}%
\pgfpathlineto{\pgfqpoint{2.788706in}{2.425324in}}%
\pgfpathlineto{\pgfqpoint{2.789412in}{2.388001in}}%
\pgfpathlineto{\pgfqpoint{2.787946in}{2.340740in}}%
\pgfpathlineto{\pgfqpoint{2.783655in}{2.278708in}}%
\pgfpathlineto{\pgfqpoint{2.774273in}{2.179722in}}%
\pgfpathlineto{\pgfqpoint{2.743600in}{1.868057in}}%
\pgfpathlineto{\pgfqpoint{2.730102in}{1.701999in}}%
\pgfpathlineto{\pgfqpoint{2.717277in}{1.515888in}}%
\pgfpathlineto{\pgfqpoint{2.702592in}{1.267536in}}%
\pgfpathlineto{\pgfqpoint{2.684421in}{0.964569in}}%
\pgfpathlineto{\pgfqpoint{2.675359in}{0.850539in}}%
\pgfpathlineto{\pgfqpoint{2.667011in}{0.771463in}}%
\pgfpathlineto{\pgfqpoint{2.658730in}{0.712483in}}%
\pgfpathlineto{\pgfqpoint{2.650149in}{0.666225in}}%
\pgfpathlineto{\pgfqpoint{2.640788in}{0.627874in}}%
\pgfpathlineto{\pgfqpoint{2.631107in}{0.597479in}}%
\pgfpathlineto{\pgfqpoint{2.620960in}{0.572693in}}%
\pgfpathlineto{\pgfqpoint{2.609806in}{0.551336in}}%
\pgfpathlineto{\pgfqpoint{2.597986in}{0.533492in}}%
\pgfpathlineto{\pgfqpoint{2.584434in}{0.517342in}}%
\pgfpathlineto{\pgfqpoint{2.571043in}{0.504639in}}%
\pgfpathlineto{\pgfqpoint{2.554719in}{0.492290in}}%
\pgfpathlineto{\pgfqpoint{2.537384in}{0.481896in}}%
\pgfpathlineto{\pgfqpoint{2.517300in}{0.472354in}}%
\pgfpathlineto{\pgfqpoint{2.492467in}{0.463170in}}%
\pgfpathlineto{\pgfqpoint{2.462903in}{0.454828in}}%
\pgfpathlineto{\pgfqpoint{2.428689in}{0.447540in}}%
\pgfpathlineto{\pgfqpoint{2.385594in}{0.440735in}}%
\pgfpathlineto{\pgfqpoint{2.331480in}{0.434582in}}%
\pgfpathlineto{\pgfqpoint{2.262038in}{0.429078in}}%
\pgfpathlineto{\pgfqpoint{2.170774in}{0.424238in}}%
\pgfpathlineto{\pgfqpoint{2.049009in}{0.420136in}}%
\pgfpathlineto{\pgfqpoint{1.879359in}{0.416786in}}%
\pgfpathlineto{\pgfqpoint{1.640082in}{0.414420in}}%
\pgfpathlineto{\pgfqpoint{1.322485in}{0.413572in}}%
\pgfpathlineto{\pgfqpoint{1.020117in}{0.414853in}}%
\pgfpathlineto{\pgfqpoint{0.822179in}{0.417720in}}%
\pgfpathlineto{\pgfqpoint{0.704758in}{0.421438in}}%
\pgfpathlineto{\pgfqpoint{0.630900in}{0.425840in}}%
\pgfpathlineto{\pgfqpoint{0.583239in}{0.430751in}}%
\pgfpathlineto{\pgfqpoint{0.550958in}{0.436148in}}%
\pgfpathlineto{\pgfqpoint{0.527634in}{0.442223in}}%
\pgfpathlineto{\pgfqpoint{0.511180in}{0.448670in}}%
\pgfpathlineto{\pgfqpoint{0.499484in}{0.455272in}}%
\pgfpathlineto{\pgfqpoint{0.488860in}{0.463912in}}%
\pgfpathlineto{\pgfqpoint{0.481276in}{0.472812in}}%
\pgfpathlineto{\pgfqpoint{0.474044in}{0.485218in}}%
\pgfpathlineto{\pgfqpoint{0.468730in}{0.498845in}}%
\pgfpathlineto{\pgfqpoint{0.463855in}{0.517948in}}%
\pgfpathlineto{\pgfqpoint{0.459671in}{0.544898in}}%
\pgfpathlineto{\pgfqpoint{0.456382in}{0.582040in}}%
\pgfpathlineto{\pgfqpoint{0.453730in}{0.639208in}}%
\pgfpathlineto{\pgfqpoint{0.451648in}{0.738746in}}%
\pgfpathlineto{\pgfqpoint{0.450199in}{0.932896in}}%
\pgfpathlineto{\pgfqpoint{0.449337in}{1.415800in}}%
\pgfpathlineto{\pgfqpoint{0.449566in}{2.692761in}}%
\pgfpathlineto{\pgfqpoint{0.451015in}{2.857034in}}%
\pgfpathlineto{\pgfqpoint{0.452823in}{2.879319in}}%
\pgfpathlineto{\pgfqpoint{0.455242in}{2.886190in}}%
\pgfpathlineto{\pgfqpoint{0.458712in}{2.889064in}}%
\pgfpathlineto{\pgfqpoint{0.465087in}{2.890564in}}%
\pgfpathlineto{\pgfqpoint{0.482469in}{2.891424in}}%
\pgfpathlineto{\pgfqpoint{0.567306in}{2.891731in}}%
\pgfpathlineto{\pgfqpoint{2.860104in}{2.891760in}}%
\pgfpathlineto{\pgfqpoint{4.789601in}{2.890872in}}%
\pgfpathlineto{\pgfqpoint{4.793810in}{2.889686in}}%
\pgfpathlineto{\pgfqpoint{4.795539in}{2.888226in}}%
\pgfpathlineto{\pgfqpoint{4.797116in}{2.881043in}}%
\pgfpathlineto{\pgfqpoint{4.798041in}{2.856180in}}%
\pgfpathlineto{\pgfqpoint{4.798041in}{2.856180in}}%
\pgfusepath{stroke}%
\end{pgfscope}%
\begin{pgfscope}%
\pgfpathrectangle{\pgfqpoint{0.448634in}{0.402556in}}{\pgfqpoint{4.350661in}{2.489204in}} %
\pgfusepath{clip}%
\pgfsetrectcap%
\pgfsetroundjoin%
\pgfsetlinewidth{1.003750pt}%
\definecolor{currentstroke}{rgb}{0.890196,0.466667,0.760784}%
\pgfsetstrokecolor{currentstroke}%
\pgfsetdash{}{0pt}%
\pgfpathmoveto{\pgfqpoint{3.430047in}{0.402666in}}%
\pgfpathlineto{\pgfqpoint{2.847062in}{0.403860in}}%
\pgfpathlineto{\pgfqpoint{2.792711in}{0.405833in}}%
\pgfpathlineto{\pgfqpoint{2.771087in}{0.408445in}}%
\pgfpathlineto{\pgfqpoint{2.758426in}{0.411997in}}%
\pgfpathlineto{\pgfqpoint{2.750610in}{0.416321in}}%
\pgfpathlineto{\pgfqpoint{2.744166in}{0.422951in}}%
\pgfpathlineto{\pgfqpoint{2.739632in}{0.431419in}}%
\pgfpathlineto{\pgfqpoint{2.736051in}{0.443154in}}%
\pgfpathlineto{\pgfqpoint{2.732802in}{0.462705in}}%
\pgfpathlineto{\pgfqpoint{2.730155in}{0.494916in}}%
\pgfpathlineto{\pgfqpoint{2.728235in}{0.549631in}}%
\pgfpathlineto{\pgfqpoint{2.727366in}{0.644213in}}%
\pgfpathlineto{\pgfqpoint{2.728341in}{0.791071in}}%
\pgfpathlineto{\pgfqpoint{2.731829in}{0.975227in}}%
\pgfpathlineto{\pgfqpoint{2.737684in}{1.164285in}}%
\pgfpathlineto{\pgfqpoint{2.745203in}{1.333331in}}%
\pgfpathlineto{\pgfqpoint{2.754068in}{1.484831in}}%
\pgfpathlineto{\pgfqpoint{2.764160in}{1.621247in}}%
\pgfpathlineto{\pgfqpoint{2.776679in}{1.759894in}}%
\pgfpathlineto{\pgfqpoint{2.789167in}{1.870991in}}%
\pgfpathlineto{\pgfqpoint{2.805060in}{1.991582in}}%
\pgfpathlineto{\pgfqpoint{2.820332in}{2.087072in}}%
\pgfpathlineto{\pgfqpoint{2.837403in}{2.179620in}}%
\pgfpathlineto{\pgfqpoint{2.857133in}{2.274034in}}%
\pgfpathlineto{\pgfqpoint{2.901327in}{2.479444in}}%
\pgfpathlineto{\pgfqpoint{2.906431in}{2.516314in}}%
\pgfpathlineto{\pgfqpoint{2.908301in}{2.546096in}}%
\pgfpathlineto{\pgfqpoint{2.907676in}{2.568477in}}%
\pgfpathlineto{\pgfqpoint{2.905105in}{2.588159in}}%
\pgfpathlineto{\pgfqpoint{2.900926in}{2.604901in}}%
\pgfpathlineto{\pgfqpoint{2.894753in}{2.620813in}}%
\pgfpathlineto{\pgfqpoint{2.886614in}{2.635521in}}%
\pgfpathlineto{\pgfqpoint{2.876746in}{2.648775in}}%
\pgfpathlineto{\pgfqpoint{2.863810in}{2.662073in}}%
\pgfpathlineto{\pgfqpoint{2.849597in}{2.673548in}}%
\pgfpathlineto{\pgfqpoint{2.832571in}{2.684593in}}%
\pgfpathlineto{\pgfqpoint{2.810800in}{2.695935in}}%
\pgfpathlineto{\pgfqpoint{2.784228in}{2.706989in}}%
\pgfpathlineto{\pgfqpoint{2.752898in}{2.717401in}}%
\pgfpathlineto{\pgfqpoint{2.716881in}{2.726980in}}%
\pgfpathlineto{\pgfqpoint{2.674105in}{2.736045in}}%
\pgfpathlineto{\pgfqpoint{2.622443in}{2.744642in}}%
\pgfpathlineto{\pgfqpoint{2.561912in}{2.752377in}}%
\pgfpathlineto{\pgfqpoint{2.490375in}{2.759176in}}%
\pgfpathlineto{\pgfqpoint{2.407855in}{2.764686in}}%
\pgfpathlineto{\pgfqpoint{2.314380in}{2.768610in}}%
\pgfpathlineto{\pgfqpoint{2.096865in}{2.770441in}}%
\pgfpathlineto{\pgfqpoint{1.983769in}{2.767993in}}%
\pgfpathlineto{\pgfqpoint{1.879430in}{2.763471in}}%
\pgfpathlineto{\pgfqpoint{1.770839in}{2.756462in}}%
\pgfpathlineto{\pgfqpoint{1.681938in}{2.748298in}}%
\pgfpathlineto{\pgfqpoint{1.601880in}{2.738824in}}%
\pgfpathlineto{\pgfqpoint{1.517732in}{2.726543in}}%
\pgfpathlineto{\pgfqpoint{1.481301in}{2.719532in}}%
\pgfpathlineto{\pgfqpoint{1.475248in}{2.716927in}}%
\pgfpathlineto{\pgfqpoint{1.415363in}{2.704217in}}%
\pgfpathlineto{\pgfqpoint{1.360139in}{2.690257in}}%
\pgfpathlineto{\pgfqpoint{1.309627in}{2.675170in}}%
\pgfpathlineto{\pgfqpoint{1.263858in}{2.659182in}}%
\pgfpathlineto{\pgfqpoint{1.220795in}{2.641754in}}%
\pgfpathlineto{\pgfqpoint{1.182581in}{2.623746in}}%
\pgfpathlineto{\pgfqpoint{1.147195in}{2.604578in}}%
\pgfpathlineto{\pgfqpoint{1.114687in}{2.584416in}}%
\pgfpathlineto{\pgfqpoint{1.092532in}{2.568654in}}%
\pgfpathlineto{\pgfqpoint{1.079965in}{2.558827in}}%
\pgfpathlineto{\pgfqpoint{1.051740in}{2.535533in}}%
\pgfpathlineto{\pgfqpoint{1.026500in}{2.511878in}}%
\pgfpathlineto{\pgfqpoint{1.002577in}{2.486496in}}%
\pgfpathlineto{\pgfqpoint{0.980081in}{2.459458in}}%
\pgfpathlineto{\pgfqpoint{0.959092in}{2.430877in}}%
\pgfpathlineto{\pgfqpoint{0.938410in}{2.398852in}}%
\pgfpathlineto{\pgfqpoint{0.922028in}{2.369491in}}%
\pgfpathlineto{\pgfqpoint{0.903596in}{2.332813in}}%
\pgfpathlineto{\pgfqpoint{0.887032in}{2.294985in}}%
\pgfpathlineto{\pgfqpoint{0.871404in}{2.253908in}}%
\pgfpathlineto{\pgfqpoint{0.856871in}{2.209639in}}%
\pgfpathlineto{\pgfqpoint{0.844318in}{2.164591in}}%
\pgfpathlineto{\pgfqpoint{0.840367in}{2.147772in}}%
\pgfpathlineto{\pgfqpoint{0.828437in}{2.094743in}}%
\pgfpathlineto{\pgfqpoint{0.817505in}{2.036331in}}%
\pgfpathlineto{\pgfqpoint{0.810175in}{1.987265in}}%
\pgfpathlineto{\pgfqpoint{0.807069in}{1.960134in}}%
\pgfpathlineto{\pgfqpoint{0.799330in}{1.891004in}}%
\pgfpathlineto{\pgfqpoint{0.793154in}{1.816666in}}%
\pgfpathlineto{\pgfqpoint{0.787893in}{1.719783in}}%
\pgfpathlineto{\pgfqpoint{0.783944in}{1.605372in}}%
\pgfpathlineto{\pgfqpoint{0.779255in}{1.486016in}}%
\pgfpathlineto{\pgfqpoint{0.775089in}{1.433968in}}%
\pgfpathlineto{\pgfqpoint{0.770204in}{1.399579in}}%
\pgfpathlineto{\pgfqpoint{0.764173in}{1.373096in}}%
\pgfpathlineto{\pgfqpoint{0.757729in}{1.354615in}}%
\pgfpathlineto{\pgfqpoint{0.751103in}{1.341768in}}%
\pgfpathlineto{\pgfqpoint{0.743923in}{1.332450in}}%
\pgfpathlineto{\pgfqpoint{0.736880in}{1.326642in}}%
\pgfpathlineto{\pgfqpoint{0.728790in}{1.323060in}}%
\pgfpathlineto{\pgfqpoint{0.720152in}{1.322159in}}%
\pgfpathlineto{\pgfqpoint{0.711585in}{1.323773in}}%
\pgfpathlineto{\pgfqpoint{0.701534in}{1.328467in}}%
\pgfpathlineto{\pgfqpoint{0.690660in}{1.336692in}}%
\pgfpathlineto{\pgfqpoint{0.679459in}{1.348474in}}%
\pgfpathlineto{\pgfqpoint{0.668244in}{1.363687in}}%
\pgfpathlineto{\pgfqpoint{0.656083in}{1.384313in}}%
\pgfpathlineto{\pgfqpoint{0.643575in}{1.410521in}}%
\pgfpathlineto{\pgfqpoint{0.631996in}{1.440036in}}%
\pgfpathlineto{\pgfqpoint{0.620606in}{1.475019in}}%
\pgfpathlineto{\pgfqpoint{0.609684in}{1.515443in}}%
\pgfpathlineto{\pgfqpoint{0.594921in}{1.585615in}}%
\pgfpathlineto{\pgfqpoint{0.585196in}{1.644307in}}%
\pgfpathlineto{\pgfqpoint{0.577101in}{1.705839in}}%
\pgfpathlineto{\pgfqpoint{0.570356in}{1.770084in}}%
\pgfpathlineto{\pgfqpoint{0.567731in}{1.799801in}}%
\pgfpathlineto{\pgfqpoint{0.561180in}{1.886597in}}%
\pgfpathlineto{\pgfqpoint{0.556058in}{1.983496in}}%
\pgfpathlineto{\pgfqpoint{0.552853in}{2.085484in}}%
\pgfpathlineto{\pgfqpoint{0.551559in}{2.192507in}}%
\pgfpathlineto{\pgfqpoint{0.552485in}{2.297046in}}%
\pgfpathlineto{\pgfqpoint{0.555572in}{2.394057in}}%
\pgfpathlineto{\pgfqpoint{0.560558in}{2.478492in}}%
\pgfpathlineto{\pgfqpoint{0.566735in}{2.545324in}}%
\pgfpathlineto{\pgfqpoint{0.574150in}{2.601937in}}%
\pgfpathlineto{\pgfqpoint{0.582466in}{2.648258in}}%
\pgfpathlineto{\pgfqpoint{0.592075in}{2.689105in}}%
\pgfpathlineto{\pgfqpoint{0.601800in}{2.719481in}}%
\pgfpathlineto{\pgfqpoint{0.611952in}{2.744265in}}%
\pgfpathlineto{\pgfqpoint{0.623066in}{2.765650in}}%
\pgfpathlineto{\pgfqpoint{0.634815in}{2.783557in}}%
\pgfpathlineto{\pgfqpoint{0.648288in}{2.799792in}}%
\pgfpathlineto{\pgfqpoint{0.661634in}{2.812557in}}%
\pgfpathlineto{\pgfqpoint{0.677956in}{2.824907in}}%
\pgfpathlineto{\pgfqpoint{0.695342in}{2.835186in}}%
\pgfpathlineto{\pgfqpoint{0.713526in}{2.843458in}}%
\pgfpathlineto{\pgfqpoint{0.736341in}{2.851689in}}%
\pgfpathlineto{\pgfqpoint{0.763852in}{2.859154in}}%
\pgfpathlineto{\pgfqpoint{0.795996in}{2.865547in}}%
\pgfpathlineto{\pgfqpoint{0.837013in}{2.871337in}}%
\pgfpathlineto{\pgfqpoint{0.889037in}{2.876312in}}%
\pgfpathlineto{\pgfqpoint{0.958547in}{2.880567in}}%
\pgfpathlineto{\pgfqpoint{1.054212in}{2.884048in}}%
\pgfpathlineto{\pgfqpoint{1.193412in}{2.886754in}}%
\pgfpathlineto{\pgfqpoint{2.102690in}{2.890547in}}%
\pgfpathlineto{\pgfqpoint{3.383959in}{2.890630in}}%
\pgfpathlineto{\pgfqpoint{4.069187in}{2.889001in}}%
\pgfpathlineto{\pgfqpoint{4.308461in}{2.886463in}}%
\pgfpathlineto{\pgfqpoint{4.428068in}{2.883158in}}%
\pgfpathlineto{\pgfqpoint{4.501938in}{2.879023in}}%
\pgfpathlineto{\pgfqpoint{4.551781in}{2.874108in}}%
\pgfpathlineto{\pgfqpoint{4.586251in}{2.868631in}}%
\pgfpathlineto{\pgfqpoint{4.613923in}{2.862001in}}%
\pgfpathlineto{\pgfqpoint{4.634737in}{2.854798in}}%
\pgfpathlineto{\pgfqpoint{4.650803in}{2.847168in}}%
\pgfpathlineto{\pgfqpoint{4.665969in}{2.837430in}}%
\pgfpathlineto{\pgfqpoint{4.678133in}{2.826970in}}%
\pgfpathlineto{\pgfqpoint{4.688961in}{2.814737in}}%
\pgfpathlineto{\pgfqpoint{4.698296in}{2.800985in}}%
\pgfpathlineto{\pgfqpoint{4.707215in}{2.783900in}}%
\pgfpathlineto{\pgfqpoint{4.715323in}{2.763519in}}%
\pgfpathlineto{\pgfqpoint{4.723089in}{2.737630in}}%
\pgfpathlineto{\pgfqpoint{4.730052in}{2.706274in}}%
\pgfpathlineto{\pgfqpoint{4.736513in}{2.667144in}}%
\pgfpathlineto{\pgfqpoint{4.742669in}{2.615353in}}%
\pgfpathlineto{\pgfqpoint{4.748204in}{2.548447in}}%
\pgfpathlineto{\pgfqpoint{4.752889in}{2.463986in}}%
\pgfpathlineto{\pgfqpoint{4.756814in}{2.352064in}}%
\pgfpathlineto{\pgfqpoint{4.759513in}{2.210214in}}%
\pgfpathlineto{\pgfqpoint{4.760603in}{2.033487in}}%
\pgfpathlineto{\pgfqpoint{4.759552in}{1.841823in}}%
\pgfpathlineto{\pgfqpoint{4.756349in}{1.657660in}}%
\pgfpathlineto{\pgfqpoint{4.751296in}{1.495968in}}%
\pgfpathlineto{\pgfqpoint{4.744971in}{1.366734in}}%
\pgfpathlineto{\pgfqpoint{4.737707in}{1.262522in}}%
\pgfpathlineto{\pgfqpoint{4.729075in}{1.170957in}}%
\pgfpathlineto{\pgfqpoint{4.720060in}{1.099515in}}%
\pgfpathlineto{\pgfqpoint{4.710161in}{1.038329in}}%
\pgfpathlineto{\pgfqpoint{4.699271in}{0.985009in}}%
\pgfpathlineto{\pgfqpoint{4.687728in}{0.939603in}}%
\pgfpathlineto{\pgfqpoint{4.676092in}{0.902073in}}%
\pgfpathlineto{\pgfqpoint{4.676092in}{0.902073in}}%
\pgfusepath{stroke}%
\end{pgfscope}%
\begin{pgfscope}%
\pgfpathrectangle{\pgfqpoint{0.448634in}{0.402556in}}{\pgfqpoint{4.350661in}{2.489204in}} %
\pgfusepath{clip}%
\pgfsetrectcap%
\pgfsetroundjoin%
\pgfsetlinewidth{1.003750pt}%
\definecolor{currentstroke}{rgb}{0.890196,0.466667,0.760784}%
\pgfsetstrokecolor{currentstroke}%
\pgfsetdash{}{0pt}%
\pgfpathmoveto{\pgfqpoint{2.795521in}{1.982745in}}%
\pgfpathlineto{\pgfqpoint{2.781780in}{1.874357in}}%
\pgfpathlineto{\pgfqpoint{2.769352in}{1.758234in}}%
\pgfpathlineto{\pgfqpoint{2.758095in}{1.631942in}}%
\pgfpathlineto{\pgfqpoint{2.747786in}{1.490551in}}%
\pgfpathlineto{\pgfqpoint{2.738644in}{1.334082in}}%
\pgfpathlineto{\pgfqpoint{2.730580in}{1.157591in}}%
\pgfpathlineto{\pgfqpoint{2.723334in}{0.948663in}}%
\pgfpathlineto{\pgfqpoint{2.709783in}{0.530788in}}%
\pgfpathlineto{\pgfqpoint{2.705868in}{0.488716in}}%
\pgfpathlineto{\pgfqpoint{2.701769in}{0.464281in}}%
\pgfpathlineto{\pgfqpoint{2.697021in}{0.447744in}}%
\pgfpathlineto{\pgfqpoint{2.691859in}{0.436812in}}%
\pgfpathlineto{\pgfqpoint{2.686245in}{0.429229in}}%
\pgfpathlineto{\pgfqpoint{2.679348in}{0.423188in}}%
\pgfpathlineto{\pgfqpoint{2.669540in}{0.417856in}}%
\pgfpathlineto{\pgfqpoint{2.656987in}{0.413810in}}%
\pgfpathlineto{\pgfqpoint{2.637654in}{0.410337in}}%
\pgfpathlineto{\pgfqpoint{2.607297in}{0.407617in}}%
\pgfpathlineto{\pgfqpoint{2.555121in}{0.405574in}}%
\pgfpathlineto{\pgfqpoint{2.450714in}{0.404139in}}%
\pgfpathlineto{\pgfqpoint{2.176624in}{0.403275in}}%
\pgfpathlineto{\pgfqpoint{1.130290in}{0.402953in}}%
\pgfpathlineto{\pgfqpoint{0.516849in}{0.404175in}}%
\pgfpathlineto{\pgfqpoint{0.466848in}{0.405970in}}%
\pgfpathlineto{\pgfqpoint{0.456130in}{0.407931in}}%
\pgfpathlineto{\pgfqpoint{0.452340in}{0.410303in}}%
\pgfpathlineto{\pgfqpoint{0.450346in}{0.414662in}}%
\pgfpathlineto{\pgfqpoint{0.449266in}{0.424523in}}%
\pgfpathlineto{\pgfqpoint{0.448771in}{0.464344in}}%
\pgfpathlineto{\pgfqpoint{0.448640in}{0.850171in}}%
\pgfpathlineto{\pgfqpoint{0.448679in}{2.891318in}}%
\pgfpathlineto{\pgfqpoint{0.448679in}{2.891318in}}%
\pgfusepath{stroke}%
\end{pgfscope}%
\begin{pgfscope}%
\pgfpathrectangle{\pgfqpoint{0.448634in}{0.402556in}}{\pgfqpoint{4.350661in}{2.489204in}} %
\pgfusepath{clip}%
\pgfsetrectcap%
\pgfsetroundjoin%
\pgfsetlinewidth{1.003750pt}%
\definecolor{currentstroke}{rgb}{0.890196,0.466667,0.760784}%
\pgfsetstrokecolor{currentstroke}%
\pgfsetdash{}{0pt}%
\pgfpathmoveto{\pgfqpoint{3.429565in}{0.402653in}}%
\pgfpathlineto{\pgfqpoint{2.840053in}{0.403823in}}%
\pgfpathlineto{\pgfqpoint{2.787880in}{0.405806in}}%
\pgfpathlineto{\pgfqpoint{2.766286in}{0.408695in}}%
\pgfpathlineto{\pgfqpoint{2.755802in}{0.411949in}}%
\pgfpathlineto{\pgfqpoint{2.748124in}{0.416574in}}%
\pgfpathlineto{\pgfqpoint{2.743356in}{0.421646in}}%
\pgfpathlineto{\pgfqpoint{2.738772in}{0.430076in}}%
\pgfpathlineto{\pgfqpoint{2.735224in}{0.441823in}}%
\pgfpathlineto{\pgfqpoint{2.732078in}{0.461396in}}%
\pgfpathlineto{\pgfqpoint{2.729567in}{0.493622in}}%
\pgfpathlineto{\pgfqpoint{2.727736in}{0.550831in}}%
\pgfpathlineto{\pgfqpoint{2.727026in}{0.650394in}}%
\pgfpathlineto{\pgfqpoint{2.728278in}{0.804717in}}%
\pgfpathlineto{\pgfqpoint{2.732045in}{0.991356in}}%
\pgfpathlineto{\pgfqpoint{2.738219in}{1.182892in}}%
\pgfpathlineto{\pgfqpoint{2.745866in}{1.349437in}}%
\pgfpathlineto{\pgfqpoint{2.754971in}{1.500919in}}%
\pgfpathlineto{\pgfqpoint{2.765318in}{1.637310in}}%
\pgfpathlineto{\pgfqpoint{2.778131in}{1.775920in}}%
\pgfpathlineto{\pgfqpoint{2.790919in}{1.886972in}}%
\pgfpathlineto{\pgfqpoint{2.807519in}{2.009953in}}%
\pgfpathlineto{\pgfqpoint{2.823182in}{2.105360in}}%
\pgfpathlineto{\pgfqpoint{2.840638in}{2.197813in}}%
\pgfpathlineto{\pgfqpoint{2.861228in}{2.294551in}}%
\pgfpathlineto{\pgfqpoint{2.896286in}{2.456398in}}%
\pgfpathlineto{\pgfqpoint{2.903540in}{2.500421in}}%
\pgfpathlineto{\pgfqpoint{2.906686in}{2.532570in}}%
\pgfpathlineto{\pgfqpoint{2.907066in}{2.557447in}}%
\pgfpathlineto{\pgfqpoint{2.905198in}{2.579733in}}%
\pgfpathlineto{\pgfqpoint{2.901235in}{2.599106in}}%
\pgfpathlineto{\pgfqpoint{2.895692in}{2.615320in}}%
\pgfpathlineto{\pgfqpoint{2.888155in}{2.630442in}}%
\pgfpathlineto{\pgfqpoint{2.878805in}{2.644177in}}%
\pgfpathlineto{\pgfqpoint{2.867967in}{2.656401in}}%
\pgfpathlineto{\pgfqpoint{2.854222in}{2.668597in}}%
\pgfpathlineto{\pgfqpoint{2.837548in}{2.680319in}}%
\pgfpathlineto{\pgfqpoint{2.818043in}{2.691323in}}%
\pgfpathlineto{\pgfqpoint{2.793783in}{2.702329in}}%
\pgfpathlineto{\pgfqpoint{2.764761in}{2.712874in}}%
\pgfpathlineto{\pgfqpoint{2.731038in}{2.722708in}}%
\pgfpathlineto{\pgfqpoint{2.690539in}{2.732133in}}%
\pgfpathlineto{\pgfqpoint{2.643291in}{2.740826in}}%
\pgfpathlineto{\pgfqpoint{2.587173in}{2.748862in}}%
\pgfpathlineto{\pgfqpoint{2.520039in}{2.756121in}}%
\pgfpathlineto{\pgfqpoint{2.441910in}{2.762217in}}%
\pgfpathlineto{\pgfqpoint{2.352815in}{2.766842in}}%
\pgfpathlineto{\pgfqpoint{2.259306in}{2.769491in}}%
\pgfpathlineto{\pgfqpoint{2.152720in}{2.770418in}}%
\pgfpathlineto{\pgfqpoint{2.043960in}{2.769152in}}%
\pgfpathlineto{\pgfqpoint{1.930889in}{2.765585in}}%
\pgfpathlineto{\pgfqpoint{1.828764in}{2.760042in}}%
\pgfpathlineto{\pgfqpoint{1.724586in}{2.752093in}}%
\pgfpathlineto{\pgfqpoint{1.640107in}{2.743198in}}%
\pgfpathlineto{\pgfqpoint{1.562326in}{2.732802in}}%
\pgfpathlineto{\pgfqpoint{1.484801in}{2.720261in}}%
\pgfpathlineto{\pgfqpoint{1.480740in}{2.718577in}}%
\pgfpathlineto{\pgfqpoint{1.476869in}{2.716386in}}%
\pgfpathlineto{\pgfqpoint{1.399951in}{2.699589in}}%
\pgfpathlineto{\pgfqpoint{1.346981in}{2.685510in}}%
\pgfpathlineto{\pgfqpoint{1.298718in}{2.670433in}}%
\pgfpathlineto{\pgfqpoint{1.255180in}{2.654620in}}%
\pgfpathlineto{\pgfqpoint{1.212295in}{2.636631in}}%
\pgfpathlineto{\pgfqpoint{1.176275in}{2.619070in}}%
\pgfpathlineto{\pgfqpoint{1.141096in}{2.599410in}}%
\pgfpathlineto{\pgfqpoint{1.108829in}{2.578748in}}%
\pgfpathlineto{\pgfqpoint{1.090730in}{2.565005in}}%
\pgfpathlineto{\pgfqpoint{1.055192in}{2.536305in}}%
\pgfpathlineto{\pgfqpoint{1.029879in}{2.512754in}}%
\pgfpathlineto{\pgfqpoint{1.005877in}{2.487469in}}%
\pgfpathlineto{\pgfqpoint{0.983297in}{2.460522in}}%
\pgfpathlineto{\pgfqpoint{0.962223in}{2.432024in}}%
\pgfpathlineto{\pgfqpoint{0.941451in}{2.400075in}}%
\pgfpathlineto{\pgfqpoint{0.926153in}{2.372878in}}%
\pgfpathlineto{\pgfqpoint{0.907525in}{2.336330in}}%
\pgfpathlineto{\pgfqpoint{0.890783in}{2.298605in}}%
\pgfpathlineto{\pgfqpoint{0.874988in}{2.257612in}}%
\pgfpathlineto{\pgfqpoint{0.860308in}{2.213406in}}%
\pgfpathlineto{\pgfqpoint{0.846895in}{2.166054in}}%
\pgfpathlineto{\pgfqpoint{0.837275in}{2.125200in}}%
\pgfpathlineto{\pgfqpoint{0.825765in}{2.069488in}}%
\pgfpathlineto{\pgfqpoint{0.816268in}{2.013284in}}%
\pgfpathlineto{\pgfqpoint{0.802493in}{1.899890in}}%
\pgfpathlineto{\pgfqpoint{0.796258in}{1.825559in}}%
\pgfpathlineto{\pgfqpoint{0.791455in}{1.741107in}}%
\pgfpathlineto{\pgfqpoint{0.785207in}{1.522179in}}%
\pgfpathlineto{\pgfqpoint{0.785207in}{1.522179in}}%
\pgfusepath{stroke}%
\end{pgfscope}%
\begin{pgfscope}%
\pgfpathrectangle{\pgfqpoint{0.448634in}{0.402556in}}{\pgfqpoint{4.350661in}{2.489204in}} %
\pgfusepath{clip}%
\pgfsetrectcap%
\pgfsetroundjoin%
\pgfsetlinewidth{1.003750pt}%
\definecolor{currentstroke}{rgb}{0.498039,0.498039,0.498039}%
\pgfsetstrokecolor{currentstroke}%
\pgfsetdash{}{0pt}%
\pgfpathmoveto{\pgfqpoint{0.448634in}{2.896245in}}%
\pgfpathlineto{\pgfqpoint{0.448593in}{0.407043in}}%
\pgfpathlineto{\pgfqpoint{0.448593in}{0.407043in}}%
\pgfusepath{stroke}%
\end{pgfscope}%
\begin{pgfscope}%
\pgfpathrectangle{\pgfqpoint{0.448634in}{0.402556in}}{\pgfqpoint{4.350661in}{2.489204in}} %
\pgfusepath{clip}%
\pgfsetrectcap%
\pgfsetroundjoin%
\pgfsetlinewidth{1.003750pt}%
\definecolor{currentstroke}{rgb}{0.498039,0.498039,0.498039}%
\pgfsetstrokecolor{currentstroke}%
\pgfsetdash{}{0pt}%
\pgfpathmoveto{\pgfqpoint{3.429181in}{0.402624in}}%
\pgfpathlineto{\pgfqpoint{2.817917in}{0.403790in}}%
\pgfpathlineto{\pgfqpoint{2.774454in}{0.405815in}}%
\pgfpathlineto{\pgfqpoint{2.757248in}{0.408682in}}%
\pgfpathlineto{\pgfqpoint{2.749006in}{0.411825in}}%
\pgfpathlineto{\pgfqpoint{2.743465in}{0.415733in}}%
\pgfpathlineto{\pgfqpoint{2.739053in}{0.421204in}}%
\pgfpathlineto{\pgfqpoint{2.735137in}{0.430065in}}%
\pgfpathlineto{\pgfqpoint{2.731868in}{0.444506in}}%
\pgfpathlineto{\pgfqpoint{2.729413in}{0.466723in}}%
\pgfpathlineto{\pgfqpoint{2.727404in}{0.508973in}}%
\pgfpathlineto{\pgfqpoint{2.726247in}{0.586124in}}%
\pgfpathlineto{\pgfqpoint{2.726631in}{0.720539in}}%
\pgfpathlineto{\pgfqpoint{2.729429in}{0.902221in}}%
\pgfpathlineto{\pgfqpoint{2.734610in}{1.096287in}}%
\pgfpathlineto{\pgfqpoint{2.741633in}{1.272836in}}%
\pgfpathlineto{\pgfqpoint{2.750106in}{1.431848in}}%
\pgfpathlineto{\pgfqpoint{2.759931in}{1.575781in}}%
\pgfpathlineto{\pgfqpoint{2.772412in}{1.724438in}}%
\pgfpathlineto{\pgfqpoint{2.784744in}{1.840574in}}%
\pgfpathlineto{\pgfqpoint{2.800706in}{1.968694in}}%
\pgfpathlineto{\pgfqpoint{2.815746in}{2.066761in}}%
\pgfpathlineto{\pgfqpoint{2.832197in}{2.159455in}}%
\pgfpathlineto{\pgfqpoint{2.850752in}{2.251629in}}%
\pgfpathlineto{\pgfqpoint{2.897592in}{2.476737in}}%
\pgfpathlineto{\pgfqpoint{2.902464in}{2.513649in}}%
\pgfpathlineto{\pgfqpoint{2.904162in}{2.543444in}}%
\pgfpathlineto{\pgfqpoint{2.903432in}{2.565821in}}%
\pgfpathlineto{\pgfqpoint{2.900789in}{2.585491in}}%
\pgfpathlineto{\pgfqpoint{2.896584in}{2.602225in}}%
\pgfpathlineto{\pgfqpoint{2.890421in}{2.618142in}}%
\pgfpathlineto{\pgfqpoint{2.882321in}{2.632879in}}%
\pgfpathlineto{\pgfqpoint{2.872510in}{2.646188in}}%
\pgfpathlineto{\pgfqpoint{2.859642in}{2.659573in}}%
\pgfpathlineto{\pgfqpoint{2.845485in}{2.671139in}}%
\pgfpathlineto{\pgfqpoint{2.826570in}{2.683413in}}%
\pgfpathlineto{\pgfqpoint{2.804770in}{2.694682in}}%
\pgfpathlineto{\pgfqpoint{2.778180in}{2.705679in}}%
\pgfpathlineto{\pgfqpoint{2.746842in}{2.716061in}}%
\pgfpathlineto{\pgfqpoint{2.710822in}{2.725626in}}%
\pgfpathlineto{\pgfqpoint{2.668045in}{2.734683in}}%
\pgfpathlineto{\pgfqpoint{2.616383in}{2.743281in}}%
\pgfpathlineto{\pgfqpoint{2.555852in}{2.751014in}}%
\pgfpathlineto{\pgfqpoint{2.484314in}{2.757804in}}%
\pgfpathlineto{\pgfqpoint{2.401793in}{2.763294in}}%
\pgfpathlineto{\pgfqpoint{2.312665in}{2.766997in}}%
\pgfpathlineto{\pgfqpoint{2.234369in}{2.768714in}}%
\pgfpathlineto{\pgfqpoint{2.127780in}{2.769159in}}%
\pgfpathlineto{\pgfqpoint{2.019026in}{2.767391in}}%
\pgfpathlineto{\pgfqpoint{1.905969in}{2.763278in}}%
\pgfpathlineto{\pgfqpoint{1.795177in}{2.756759in}}%
\pgfpathlineto{\pgfqpoint{1.704074in}{2.748899in}}%
\pgfpathlineto{\pgfqpoint{1.621810in}{2.739641in}}%
\pgfpathlineto{\pgfqpoint{1.546254in}{2.728929in}}%
\pgfpathlineto{\pgfqpoint{1.449471in}{2.712478in}}%
\pgfpathlineto{\pgfqpoint{1.391867in}{2.699372in}}%
\pgfpathlineto{\pgfqpoint{1.338937in}{2.685096in}}%
\pgfpathlineto{\pgfqpoint{1.290725in}{2.669808in}}%
\pgfpathlineto{\pgfqpoint{1.249309in}{2.654575in}}%
\pgfpathlineto{\pgfqpoint{1.206483in}{2.636401in}}%
\pgfpathlineto{\pgfqpoint{1.170518in}{2.618695in}}%
\pgfpathlineto{\pgfqpoint{1.137332in}{2.600033in}}%
\pgfpathlineto{\pgfqpoint{1.105085in}{2.579331in}}%
\pgfpathlineto{\pgfqpoint{1.083232in}{2.563024in}}%
\pgfpathlineto{\pgfqpoint{1.054801in}{2.540060in}}%
\pgfpathlineto{\pgfqpoint{1.029341in}{2.516715in}}%
\pgfpathlineto{\pgfqpoint{1.005176in}{2.491635in}}%
\pgfpathlineto{\pgfqpoint{0.982419in}{2.464884in}}%
\pgfpathlineto{\pgfqpoint{0.961156in}{2.436570in}}%
\pgfpathlineto{\pgfqpoint{0.940174in}{2.404802in}}%
\pgfpathlineto{\pgfqpoint{0.921002in}{2.371569in}}%
\pgfpathlineto{\pgfqpoint{0.902472in}{2.334956in}}%
\pgfpathlineto{\pgfqpoint{0.885807in}{2.297186in}}%
\pgfpathlineto{\pgfqpoint{0.870069in}{2.256164in}}%
\pgfpathlineto{\pgfqpoint{0.855417in}{2.211946in}}%
\pgfpathlineto{\pgfqpoint{0.842022in}{2.164587in}}%
\pgfpathlineto{\pgfqpoint{0.821878in}{2.075436in}}%
\pgfpathlineto{\pgfqpoint{0.811528in}{2.016885in}}%
\pgfpathlineto{\pgfqpoint{0.797570in}{1.911059in}}%
\pgfpathlineto{\pgfqpoint{0.790252in}{1.834353in}}%
\pgfpathlineto{\pgfqpoint{0.784016in}{1.745029in}}%
\pgfpathlineto{\pgfqpoint{0.777222in}{1.610838in}}%
\pgfpathlineto{\pgfqpoint{0.771554in}{1.516476in}}%
\pgfpathlineto{\pgfqpoint{0.766437in}{1.467045in}}%
\pgfpathlineto{\pgfqpoint{0.760690in}{1.432832in}}%
\pgfpathlineto{\pgfqpoint{0.754521in}{1.408976in}}%
\pgfpathlineto{\pgfqpoint{0.748461in}{1.393004in}}%
\pgfpathlineto{\pgfqpoint{0.741461in}{1.380424in}}%
\pgfpathlineto{\pgfqpoint{0.733819in}{1.371612in}}%
\pgfpathlineto{\pgfqpoint{0.726322in}{1.366618in}}%
\pgfpathlineto{\pgfqpoint{0.720046in}{1.364633in}}%
\pgfpathlineto{\pgfqpoint{0.711377in}{1.364619in}}%
\pgfpathlineto{\pgfqpoint{0.702998in}{1.367218in}}%
\pgfpathlineto{\pgfqpoint{0.693399in}{1.373026in}}%
\pgfpathlineto{\pgfqpoint{0.683198in}{1.382314in}}%
\pgfpathlineto{\pgfqpoint{0.672798in}{1.395025in}}%
\pgfpathlineto{\pgfqpoint{0.661248in}{1.413099in}}%
\pgfpathlineto{\pgfqpoint{0.650220in}{1.434546in}}%
\pgfpathlineto{\pgfqpoint{0.638871in}{1.461438in}}%
\pgfpathlineto{\pgfqpoint{0.627602in}{1.493806in}}%
\pgfpathlineto{\pgfqpoint{0.616088in}{1.534013in}}%
\pgfpathlineto{\pgfqpoint{0.605373in}{1.579686in}}%
\pgfpathlineto{\pgfqpoint{0.595142in}{1.633178in}}%
\pgfpathlineto{\pgfqpoint{0.585702in}{1.694459in}}%
\pgfpathlineto{\pgfqpoint{0.577009in}{1.765954in}}%
\pgfpathlineto{\pgfqpoint{0.569419in}{1.847634in}}%
\pgfpathlineto{\pgfqpoint{0.563219in}{1.939458in}}%
\pgfpathlineto{\pgfqpoint{0.558783in}{2.038894in}}%
\pgfpathlineto{\pgfqpoint{0.556360in}{2.143401in}}%
\pgfpathlineto{\pgfqpoint{0.556145in}{2.247945in}}%
\pgfpathlineto{\pgfqpoint{0.558113in}{2.344995in}}%
\pgfpathlineto{\pgfqpoint{0.562048in}{2.431997in}}%
\pgfpathlineto{\pgfqpoint{0.567611in}{2.506397in}}%
\pgfpathlineto{\pgfqpoint{0.574423in}{2.568132in}}%
\pgfpathlineto{\pgfqpoint{0.582340in}{2.619608in}}%
\pgfpathlineto{\pgfqpoint{0.590842in}{2.660784in}}%
\pgfpathlineto{\pgfqpoint{0.599793in}{2.694086in}}%
\pgfpathlineto{\pgfqpoint{0.609332in}{2.721882in}}%
\pgfpathlineto{\pgfqpoint{0.620000in}{2.746380in}}%
\pgfpathlineto{\pgfqpoint{0.631615in}{2.767413in}}%
\pgfpathlineto{\pgfqpoint{0.643807in}{2.784927in}}%
\pgfpathlineto{\pgfqpoint{0.657674in}{2.800723in}}%
\pgfpathlineto{\pgfqpoint{0.673079in}{2.814528in}}%
\pgfpathlineto{\pgfqpoint{0.689750in}{2.826253in}}%
\pgfpathlineto{\pgfqpoint{0.707367in}{2.836008in}}%
\pgfpathlineto{\pgfqpoint{0.727710in}{2.844806in}}%
\pgfpathlineto{\pgfqpoint{0.752775in}{2.853124in}}%
\pgfpathlineto{\pgfqpoint{0.782541in}{2.860465in}}%
\pgfpathlineto{\pgfqpoint{0.819077in}{2.866974in}}%
\pgfpathlineto{\pgfqpoint{0.864489in}{2.872612in}}%
\pgfpathlineto{\pgfqpoint{0.923068in}{2.877455in}}%
\pgfpathlineto{\pgfqpoint{1.001298in}{2.881516in}}%
\pgfpathlineto{\pgfqpoint{1.112200in}{2.884842in}}%
\pgfpathlineto{\pgfqpoint{1.275335in}{2.887341in}}%
\pgfpathlineto{\pgfqpoint{1.553772in}{2.889246in}}%
\pgfpathlineto{\pgfqpoint{2.106304in}{2.890444in}}%
\pgfpathlineto{\pgfqpoint{3.335366in}{2.890569in}}%
\pgfpathlineto{\pgfqpoint{4.037996in}{2.888950in}}%
\pgfpathlineto{\pgfqpoint{4.285973in}{2.886419in}}%
\pgfpathlineto{\pgfqpoint{4.409932in}{2.883154in}}%
\pgfpathlineto{\pgfqpoint{4.485985in}{2.879137in}}%
\pgfpathlineto{\pgfqpoint{4.538021in}{2.874347in}}%
\pgfpathlineto{\pgfqpoint{4.574691in}{2.868910in}}%
\pgfpathlineto{\pgfqpoint{4.602449in}{2.862762in}}%
\pgfpathlineto{\pgfqpoint{4.625490in}{2.855409in}}%
\pgfpathlineto{\pgfqpoint{4.643721in}{2.847272in}}%
\pgfpathlineto{\pgfqpoint{4.659085in}{2.837948in}}%
\pgfpathlineto{\pgfqpoint{4.671533in}{2.827932in}}%
\pgfpathlineto{\pgfqpoint{4.682746in}{2.816161in}}%
\pgfpathlineto{\pgfqpoint{4.692498in}{2.802797in}}%
\pgfpathlineto{\pgfqpoint{4.701941in}{2.786084in}}%
\pgfpathlineto{\pgfqpoint{4.710583in}{2.765995in}}%
\pgfpathlineto{\pgfqpoint{4.718219in}{2.742695in}}%
\pgfpathlineto{\pgfqpoint{4.725284in}{2.713950in}}%
\pgfpathlineto{\pgfqpoint{4.732457in}{2.674984in}}%
\pgfpathlineto{\pgfqpoint{4.738699in}{2.628237in}}%
\pgfpathlineto{\pgfqpoint{4.744555in}{2.566372in}}%
\pgfpathlineto{\pgfqpoint{4.749529in}{2.489420in}}%
\pgfpathlineto{\pgfqpoint{4.753841in}{2.387484in}}%
\pgfpathlineto{\pgfqpoint{4.757022in}{2.258098in}}%
\pgfpathlineto{\pgfqpoint{4.758756in}{2.096314in}}%
\pgfpathlineto{\pgfqpoint{4.758503in}{1.912114in}}%
\pgfpathlineto{\pgfqpoint{4.756053in}{1.727936in}}%
\pgfpathlineto{\pgfqpoint{4.751618in}{1.558750in}}%
\pgfpathlineto{\pgfqpoint{4.745483in}{1.414550in}}%
\pgfpathlineto{\pgfqpoint{4.738386in}{1.302833in}}%
\pgfpathlineto{\pgfqpoint{4.730561in}{1.211174in}}%
\pgfpathlineto{\pgfqpoint{4.720259in}{1.119859in}}%
\pgfpathlineto{\pgfqpoint{4.710354in}{1.056145in}}%
\pgfpathlineto{\pgfqpoint{4.699876in}{1.002716in}}%
\pgfpathlineto{\pgfqpoint{4.690113in}{0.961906in}}%
\pgfpathlineto{\pgfqpoint{4.677721in}{0.919411in}}%
\pgfpathlineto{\pgfqpoint{4.664599in}{0.882530in}}%
\pgfpathlineto{\pgfqpoint{4.651210in}{0.851238in}}%
\pgfpathlineto{\pgfqpoint{4.636989in}{0.823276in}}%
\pgfpathlineto{\pgfqpoint{4.620957in}{0.796630in}}%
\pgfpathlineto{\pgfqpoint{4.604449in}{0.773502in}}%
\pgfpathlineto{\pgfqpoint{4.586354in}{0.751984in}}%
\pgfpathlineto{\pgfqpoint{4.566792in}{0.732218in}}%
\pgfpathlineto{\pgfqpoint{4.545928in}{0.714278in}}%
\pgfpathlineto{\pgfqpoint{4.522086in}{0.696891in}}%
\pgfpathlineto{\pgfqpoint{4.495247in}{0.680439in}}%
\pgfpathlineto{\pgfqpoint{4.469462in}{0.667166in}}%
\pgfpathlineto{\pgfqpoint{4.439001in}{0.653796in}}%
\pgfpathlineto{\pgfqpoint{4.405861in}{0.641637in}}%
\pgfpathlineto{\pgfqpoint{4.368014in}{0.630170in}}%
\pgfpathlineto{\pgfqpoint{4.325482in}{0.619709in}}%
\pgfpathlineto{\pgfqpoint{4.278312in}{0.610487in}}%
\pgfpathlineto{\pgfqpoint{4.226555in}{0.602673in}}%
\pgfpathlineto{\pgfqpoint{4.170259in}{0.596468in}}%
\pgfpathlineto{\pgfqpoint{4.107313in}{0.591731in}}%
\pgfpathlineto{\pgfqpoint{4.033395in}{0.588906in}}%
\pgfpathlineto{\pgfqpoint{3.963788in}{0.588448in}}%
\pgfpathlineto{\pgfqpoint{3.892021in}{0.590211in}}%
\pgfpathlineto{\pgfqpoint{3.818151in}{0.594330in}}%
\pgfpathlineto{\pgfqpoint{3.750929in}{0.600430in}}%
\pgfpathlineto{\pgfqpoint{3.701226in}{0.606824in}}%
\pgfpathlineto{\pgfqpoint{3.686145in}{0.609198in}}%
\pgfpathlineto{\pgfqpoint{3.628026in}{0.618869in}}%
\pgfpathlineto{\pgfqpoint{3.587436in}{0.627634in}}%
\pgfpathlineto{\pgfqpoint{3.568265in}{0.632160in}}%
\pgfpathlineto{\pgfqpoint{3.542748in}{0.638442in}}%
\pgfpathlineto{\pgfqpoint{3.496363in}{0.651905in}}%
\pgfpathlineto{\pgfqpoint{3.452633in}{0.666995in}}%
\pgfpathlineto{\pgfqpoint{3.409614in}{0.684540in}}%
\pgfpathlineto{\pgfqpoint{3.375643in}{0.701248in}}%
\pgfpathlineto{\pgfqpoint{3.348337in}{0.716669in}}%
\pgfpathlineto{\pgfqpoint{3.319922in}{0.735011in}}%
\pgfpathlineto{\pgfqpoint{3.292603in}{0.755417in}}%
\pgfpathlineto{\pgfqpoint{3.268254in}{0.776338in}}%
\pgfpathlineto{\pgfqpoint{3.245197in}{0.799094in}}%
\pgfpathlineto{\pgfqpoint{3.223594in}{0.823648in}}%
\pgfpathlineto{\pgfqpoint{3.203589in}{0.849913in}}%
\pgfpathlineto{\pgfqpoint{3.185238in}{0.877717in}}%
\pgfpathlineto{\pgfqpoint{3.168757in}{0.907013in}}%
\pgfpathlineto{\pgfqpoint{3.153924in}{0.937442in}}%
\pgfpathlineto{\pgfqpoint{3.139803in}{0.971094in}}%
\pgfpathlineto{\pgfqpoint{3.125255in}{1.012684in}}%
\pgfpathlineto{\pgfqpoint{3.113766in}{1.052900in}}%
\pgfpathlineto{\pgfqpoint{3.104308in}{1.093803in}}%
\pgfpathlineto{\pgfqpoint{3.095883in}{1.140099in}}%
\pgfpathlineto{\pgfqpoint{3.089287in}{1.189302in}}%
\pgfpathlineto{\pgfqpoint{3.084614in}{1.241296in}}%
\pgfpathlineto{\pgfqpoint{3.082116in}{1.293486in}}%
\pgfpathlineto{\pgfqpoint{3.081623in}{1.348241in}}%
\pgfpathlineto{\pgfqpoint{3.083351in}{1.405454in}}%
\pgfpathlineto{\pgfqpoint{3.087052in}{1.460048in}}%
\pgfpathlineto{\pgfqpoint{3.093317in}{1.519352in}}%
\pgfpathlineto{\pgfqpoint{3.101620in}{1.575805in}}%
\pgfpathlineto{\pgfqpoint{3.112163in}{1.631766in}}%
\pgfpathlineto{\pgfqpoint{3.124975in}{1.687102in}}%
\pgfpathlineto{\pgfqpoint{3.139513in}{1.739271in}}%
\pgfpathlineto{\pgfqpoint{3.155528in}{1.788221in}}%
\pgfpathlineto{\pgfqpoint{3.172784in}{1.833916in}}%
\pgfpathlineto{\pgfqpoint{3.192078in}{1.878530in}}%
\pgfpathlineto{\pgfqpoint{3.215662in}{1.926173in}}%
\pgfpathlineto{\pgfqpoint{3.237990in}{1.965966in}}%
\pgfpathlineto{\pgfqpoint{3.262091in}{2.004382in}}%
\pgfpathlineto{\pgfqpoint{3.287852in}{2.041360in}}%
\pgfpathlineto{\pgfqpoint{3.316568in}{2.078754in}}%
\pgfpathlineto{\pgfqpoint{3.351198in}{2.120071in}}%
\pgfpathlineto{\pgfqpoint{3.416495in}{2.196666in}}%
\pgfpathlineto{\pgfqpoint{3.426763in}{2.212721in}}%
\pgfpathlineto{\pgfqpoint{3.431431in}{2.223931in}}%
\pgfpathlineto{\pgfqpoint{3.432538in}{2.231261in}}%
\pgfpathlineto{\pgfqpoint{3.431062in}{2.238456in}}%
\pgfpathlineto{\pgfqpoint{3.426743in}{2.243962in}}%
\pgfpathlineto{\pgfqpoint{3.420965in}{2.247384in}}%
\pgfpathlineto{\pgfqpoint{3.412532in}{2.249773in}}%
\pgfpathlineto{\pgfqpoint{3.399523in}{2.250789in}}%
\pgfpathlineto{\pgfqpoint{3.384332in}{2.249716in}}%
\pgfpathlineto{\pgfqpoint{3.365018in}{2.246104in}}%
\pgfpathlineto{\pgfqpoint{3.341842in}{2.239323in}}%
\pgfpathlineto{\pgfqpoint{3.317151in}{2.229649in}}%
\pgfpathlineto{\pgfqpoint{3.291148in}{2.216948in}}%
\pgfpathlineto{\pgfqpoint{3.265972in}{2.202224in}}%
\pgfpathlineto{\pgfqpoint{3.239847in}{2.184327in}}%
\pgfpathlineto{\pgfqpoint{3.214814in}{2.164490in}}%
\pgfpathlineto{\pgfqpoint{3.190936in}{2.142869in}}%
\pgfpathlineto{\pgfqpoint{3.166687in}{2.117894in}}%
\pgfpathlineto{\pgfqpoint{3.143862in}{2.091220in}}%
\pgfpathlineto{\pgfqpoint{3.121100in}{2.061098in}}%
\pgfpathlineto{\pgfqpoint{3.099970in}{2.029458in}}%
\pgfpathlineto{\pgfqpoint{3.079265in}{1.994403in}}%
\pgfpathlineto{\pgfqpoint{3.059230in}{1.955913in}}%
\pgfpathlineto{\pgfqpoint{3.040070in}{1.914014in}}%
\pgfpathlineto{\pgfqpoint{3.022822in}{1.871040in}}%
\pgfpathlineto{\pgfqpoint{3.005804in}{1.822534in}}%
\pgfpathlineto{\pgfqpoint{2.990082in}{1.770816in}}%
\pgfpathlineto{\pgfqpoint{2.975725in}{1.715977in}}%
\pgfpathlineto{\pgfqpoint{2.962300in}{1.655677in}}%
\pgfpathlineto{\pgfqpoint{2.950510in}{1.592383in}}%
\pgfpathlineto{\pgfqpoint{2.940394in}{1.526183in}}%
\pgfpathlineto{\pgfqpoint{2.931750in}{1.454681in}}%
\pgfpathlineto{\pgfqpoint{2.925080in}{1.380399in}}%
\pgfpathlineto{\pgfqpoint{2.920639in}{1.305900in}}%
\pgfpathlineto{\pgfqpoint{2.918431in}{1.231271in}}%
\pgfpathlineto{\pgfqpoint{2.918530in}{1.159088in}}%
\pgfpathlineto{\pgfqpoint{2.920771in}{1.091932in}}%
\pgfpathlineto{\pgfqpoint{2.925162in}{1.027413in}}%
\pgfpathlineto{\pgfqpoint{2.931177in}{0.970581in}}%
\pgfpathlineto{\pgfqpoint{2.938744in}{0.919035in}}%
\pgfpathlineto{\pgfqpoint{2.947631in}{0.872852in}}%
\pgfpathlineto{\pgfqpoint{2.958188in}{0.829712in}}%
\pgfpathlineto{\pgfqpoint{2.969637in}{0.792109in}}%
\pgfpathlineto{\pgfqpoint{2.982421in}{0.757764in}}%
\pgfpathlineto{\pgfqpoint{2.996374in}{0.726797in}}%
\pgfpathlineto{\pgfqpoint{3.011239in}{0.699279in}}%
\pgfpathlineto{\pgfqpoint{3.026671in}{0.675197in}}%
\pgfpathlineto{\pgfqpoint{3.043753in}{0.652621in}}%
\pgfpathlineto{\pgfqpoint{3.062415in}{0.631747in}}%
\pgfpathlineto{\pgfqpoint{3.082518in}{0.612706in}}%
\pgfpathlineto{\pgfqpoint{3.103874in}{0.595542in}}%
\pgfpathlineto{\pgfqpoint{3.128180in}{0.579016in}}%
\pgfpathlineto{\pgfqpoint{3.153448in}{0.564499in}}%
\pgfpathlineto{\pgfqpoint{3.181482in}{0.550897in}}%
\pgfpathlineto{\pgfqpoint{3.214282in}{0.537593in}}%
\pgfpathlineto{\pgfqpoint{3.249758in}{0.525658in}}%
\pgfpathlineto{\pgfqpoint{3.289922in}{0.514517in}}%
\pgfpathlineto{\pgfqpoint{3.334732in}{0.504370in}}%
\pgfpathlineto{\pgfqpoint{3.386284in}{0.494947in}}%
\pgfpathlineto{\pgfqpoint{3.446710in}{0.486206in}}%
\pgfpathlineto{\pgfqpoint{3.518155in}{0.478232in}}%
\pgfpathlineto{\pgfqpoint{3.600597in}{0.471361in}}%
\pgfpathlineto{\pgfqpoint{3.696181in}{0.465666in}}%
\pgfpathlineto{\pgfqpoint{3.807056in}{0.461323in}}%
\pgfpathlineto{\pgfqpoint{3.931028in}{0.458706in}}%
\pgfpathlineto{\pgfqpoint{4.061546in}{0.458169in}}%
\pgfpathlineto{\pgfqpoint{4.185529in}{0.459841in}}%
\pgfpathlineto{\pgfqpoint{4.292073in}{0.463409in}}%
\pgfpathlineto{\pgfqpoint{4.378975in}{0.468410in}}%
\pgfpathlineto{\pgfqpoint{4.448381in}{0.474472in}}%
\pgfpathlineto{\pgfqpoint{4.504603in}{0.481490in}}%
\pgfpathlineto{\pgfqpoint{4.549775in}{0.489248in}}%
\pgfpathlineto{\pgfqpoint{4.586028in}{0.497578in}}%
\pgfpathlineto{\pgfqpoint{4.615483in}{0.506412in}}%
\pgfpathlineto{\pgfqpoint{4.640214in}{0.515949in}}%
\pgfpathlineto{\pgfqpoint{4.662170in}{0.526811in}}%
\pgfpathlineto{\pgfqpoint{4.679327in}{0.537585in}}%
\pgfpathlineto{\pgfqpoint{4.695408in}{0.550341in}}%
\pgfpathlineto{\pgfqpoint{4.708494in}{0.563453in}}%
\pgfpathlineto{\pgfqpoint{4.720195in}{0.578179in}}%
\pgfpathlineto{\pgfqpoint{4.730379in}{0.594315in}}%
\pgfpathlineto{\pgfqpoint{4.740035in}{0.613790in}}%
\pgfpathlineto{\pgfqpoint{4.748755in}{0.636584in}}%
\pgfpathlineto{\pgfqpoint{4.756335in}{0.662547in}}%
\pgfpathlineto{\pgfqpoint{4.763229in}{0.693924in}}%
\pgfpathlineto{\pgfqpoint{4.769533in}{0.733086in}}%
\pgfpathlineto{\pgfqpoint{4.775177in}{0.782445in}}%
\pgfpathlineto{\pgfqpoint{4.780207in}{0.846905in}}%
\pgfpathlineto{\pgfqpoint{4.784603in}{0.933879in}}%
\pgfpathlineto{\pgfqpoint{4.788273in}{1.053285in}}%
\pgfpathlineto{\pgfqpoint{4.791272in}{1.227494in}}%
\pgfpathlineto{\pgfqpoint{4.793571in}{1.501293in}}%
\pgfpathlineto{\pgfqpoint{4.794883in}{1.946858in}}%
\pgfpathlineto{\pgfqpoint{4.794284in}{2.499460in}}%
\pgfpathlineto{\pgfqpoint{4.792163in}{2.725962in}}%
\pgfpathlineto{\pgfqpoint{4.789427in}{2.810531in}}%
\pgfpathlineto{\pgfqpoint{4.786464in}{2.845205in}}%
\pgfpathlineto{\pgfqpoint{4.783332in}{2.862243in}}%
\pgfpathlineto{\pgfqpoint{4.779062in}{2.873649in}}%
\pgfpathlineto{\pgfqpoint{4.774896in}{2.879364in}}%
\pgfpathlineto{\pgfqpoint{4.769472in}{2.883473in}}%
\pgfpathlineto{\pgfqpoint{4.761241in}{2.886642in}}%
\pgfpathlineto{\pgfqpoint{4.748345in}{2.888875in}}%
\pgfpathlineto{\pgfqpoint{4.724457in}{2.890390in}}%
\pgfpathlineto{\pgfqpoint{4.665730in}{2.891305in}}%
\pgfpathlineto{\pgfqpoint{4.428620in}{2.891689in}}%
\pgfpathlineto{\pgfqpoint{1.072085in}{2.891701in}}%
\pgfpathlineto{\pgfqpoint{0.600041in}{2.890510in}}%
\pgfpathlineto{\pgfqpoint{0.543515in}{2.888485in}}%
\pgfpathlineto{\pgfqpoint{0.517545in}{2.885563in}}%
\pgfpathlineto{\pgfqpoint{0.502686in}{2.881825in}}%
\pgfpathlineto{\pgfqpoint{0.492659in}{2.877062in}}%
\pgfpathlineto{\pgfqpoint{0.485579in}{2.871311in}}%
\pgfpathlineto{\pgfqpoint{0.479882in}{2.863815in}}%
\pgfpathlineto{\pgfqpoint{0.474764in}{2.852859in}}%
\pgfpathlineto{\pgfqpoint{0.470700in}{2.838679in}}%
\pgfpathlineto{\pgfqpoint{0.466902in}{2.816712in}}%
\pgfpathlineto{\pgfqpoint{0.463483in}{2.782090in}}%
\pgfpathlineto{\pgfqpoint{0.460556in}{2.727433in}}%
\pgfpathlineto{\pgfqpoint{0.458022in}{2.632890in}}%
\pgfpathlineto{\pgfqpoint{0.456555in}{2.525868in}}%
\pgfpathlineto{\pgfqpoint{0.456555in}{2.525868in}}%
\pgfusepath{stroke}%
\end{pgfscope}%
\begin{pgfscope}%
\pgfpathrectangle{\pgfqpoint{0.448634in}{0.402556in}}{\pgfqpoint{4.350661in}{2.489204in}} %
\pgfusepath{clip}%
\pgfsetrectcap%
\pgfsetroundjoin%
\pgfsetlinewidth{1.003750pt}%
\definecolor{currentstroke}{rgb}{0.498039,0.498039,0.498039}%
\pgfsetstrokecolor{currentstroke}%
\pgfsetdash{}{0pt}%
\pgfpathmoveto{\pgfqpoint{0.456431in}{1.369072in}}%
\pgfpathlineto{\pgfqpoint{0.459628in}{1.117690in}}%
\pgfpathlineto{\pgfqpoint{0.463640in}{0.963430in}}%
\pgfpathlineto{\pgfqpoint{0.468429in}{0.859030in}}%
\pgfpathlineto{\pgfqpoint{0.473943in}{0.784626in}}%
\pgfpathlineto{\pgfqpoint{0.480025in}{0.730313in}}%
\pgfpathlineto{\pgfqpoint{0.486695in}{0.688697in}}%
\pgfpathlineto{\pgfqpoint{0.494170in}{0.654922in}}%
\pgfpathlineto{\pgfqpoint{0.502633in}{0.626676in}}%
\pgfpathlineto{\pgfqpoint{0.511604in}{0.604010in}}%
\pgfpathlineto{\pgfqpoint{0.521493in}{0.584688in}}%
\pgfpathlineto{\pgfqpoint{0.533277in}{0.566815in}}%
\pgfpathlineto{\pgfqpoint{0.545329in}{0.552466in}}%
\pgfpathlineto{\pgfqpoint{0.558707in}{0.539745in}}%
\pgfpathlineto{\pgfqpoint{0.575033in}{0.527401in}}%
\pgfpathlineto{\pgfqpoint{0.594330in}{0.515932in}}%
\pgfpathlineto{\pgfqpoint{0.616488in}{0.505616in}}%
\pgfpathlineto{\pgfqpoint{0.641349in}{0.496529in}}%
\pgfpathlineto{\pgfqpoint{0.670889in}{0.488074in}}%
\pgfpathlineto{\pgfqpoint{0.707195in}{0.480051in}}%
\pgfpathlineto{\pgfqpoint{0.752394in}{0.472495in}}%
\pgfpathlineto{\pgfqpoint{0.806458in}{0.465785in}}%
\pgfpathlineto{\pgfqpoint{0.873683in}{0.459726in}}%
\pgfpathlineto{\pgfqpoint{0.958391in}{0.454398in}}%
\pgfpathlineto{\pgfqpoint{1.064912in}{0.450016in}}%
\pgfpathlineto{\pgfqpoint{1.197578in}{0.446883in}}%
\pgfpathlineto{\pgfqpoint{1.354196in}{0.445475in}}%
\pgfpathlineto{\pgfqpoint{1.521694in}{0.446179in}}%
\pgfpathlineto{\pgfqpoint{1.682648in}{0.449038in}}%
\pgfpathlineto{\pgfqpoint{1.821809in}{0.453676in}}%
\pgfpathlineto{\pgfqpoint{1.936981in}{0.459668in}}%
\pgfpathlineto{\pgfqpoint{2.032489in}{0.466822in}}%
\pgfpathlineto{\pgfqpoint{2.110485in}{0.474834in}}%
\pgfpathlineto{\pgfqpoint{2.175286in}{0.483655in}}%
\pgfpathlineto{\pgfqpoint{2.229036in}{0.493096in}}%
\pgfpathlineto{\pgfqpoint{2.276009in}{0.503555in}}%
\pgfpathlineto{\pgfqpoint{2.316162in}{0.514746in}}%
\pgfpathlineto{\pgfqpoint{2.351564in}{0.526958in}}%
\pgfpathlineto{\pgfqpoint{2.382168in}{0.539891in}}%
\pgfpathlineto{\pgfqpoint{2.407989in}{0.553072in}}%
\pgfpathlineto{\pgfqpoint{2.431023in}{0.567112in}}%
\pgfpathlineto{\pgfqpoint{2.453041in}{0.583142in}}%
\pgfpathlineto{\pgfqpoint{2.472109in}{0.599672in}}%
\pgfpathlineto{\pgfqpoint{2.489911in}{0.617957in}}%
\pgfpathlineto{\pgfqpoint{2.506295in}{0.637901in}}%
\pgfpathlineto{\pgfqpoint{2.521181in}{0.659330in}}%
\pgfpathlineto{\pgfqpoint{2.535713in}{0.684134in}}%
\pgfpathlineto{\pgfqpoint{2.549552in}{0.712345in}}%
\pgfpathlineto{\pgfqpoint{2.562469in}{0.743896in}}%
\pgfpathlineto{\pgfqpoint{2.574345in}{0.778667in}}%
\pgfpathlineto{\pgfqpoint{2.585780in}{0.818904in}}%
\pgfpathlineto{\pgfqpoint{2.597042in}{0.866986in}}%
\pgfpathlineto{\pgfqpoint{2.607757in}{0.922905in}}%
\pgfpathlineto{\pgfqpoint{2.618088in}{0.989062in}}%
\pgfpathlineto{\pgfqpoint{2.628094in}{1.067887in}}%
\pgfpathlineto{\pgfqpoint{2.638281in}{1.166767in}}%
\pgfpathlineto{\pgfqpoint{2.648700in}{1.290654in}}%
\pgfpathlineto{\pgfqpoint{2.660495in}{1.459379in}}%
\pgfpathlineto{\pgfqpoint{2.675263in}{1.705229in}}%
\pgfpathlineto{\pgfqpoint{2.687904in}{1.948741in}}%
\pgfpathlineto{\pgfqpoint{2.692732in}{2.080550in}}%
\pgfpathlineto{\pgfqpoint{2.693862in}{2.167659in}}%
\pgfpathlineto{\pgfqpoint{2.692652in}{2.232359in}}%
\pgfpathlineto{\pgfqpoint{2.689584in}{2.284508in}}%
\pgfpathlineto{\pgfqpoint{2.684741in}{2.328963in}}%
\pgfpathlineto{\pgfqpoint{2.678540in}{2.365612in}}%
\pgfpathlineto{\pgfqpoint{2.671100in}{2.396823in}}%
\pgfpathlineto{\pgfqpoint{2.662157in}{2.424876in}}%
\pgfpathlineto{\pgfqpoint{2.651953in}{2.449631in}}%
\pgfpathlineto{\pgfqpoint{2.640884in}{2.471050in}}%
\pgfpathlineto{\pgfqpoint{2.628093in}{2.491172in}}%
\pgfpathlineto{\pgfqpoint{2.613666in}{2.509790in}}%
\pgfpathlineto{\pgfqpoint{2.597778in}{2.526779in}}%
\pgfpathlineto{\pgfqpoint{2.578877in}{2.543555in}}%
\pgfpathlineto{\pgfqpoint{2.558783in}{2.558409in}}%
\pgfpathlineto{\pgfqpoint{2.535823in}{2.572605in}}%
\pgfpathlineto{\pgfqpoint{2.510047in}{2.585901in}}%
\pgfpathlineto{\pgfqpoint{2.481539in}{2.598147in}}%
\pgfpathlineto{\pgfqpoint{2.448300in}{2.609942in}}%
\pgfpathlineto{\pgfqpoint{2.410340in}{2.620910in}}%
\pgfpathlineto{\pgfqpoint{2.367700in}{2.630777in}}%
\pgfpathlineto{\pgfqpoint{2.320437in}{2.639354in}}%
\pgfpathlineto{\pgfqpoint{2.268606in}{2.646499in}}%
\pgfpathlineto{\pgfqpoint{2.210091in}{2.652261in}}%
\pgfpathlineto{\pgfqpoint{2.147103in}{2.656193in}}%
\pgfpathlineto{\pgfqpoint{2.079691in}{2.658147in}}%
\pgfpathlineto{\pgfqpoint{2.010083in}{2.657956in}}%
\pgfpathlineto{\pgfqpoint{1.938331in}{2.655530in}}%
\pgfpathlineto{\pgfqpoint{1.866664in}{2.650842in}}%
\pgfpathlineto{\pgfqpoint{1.797310in}{2.644039in}}%
\pgfpathlineto{\pgfqpoint{1.732484in}{2.635474in}}%
\pgfpathlineto{\pgfqpoint{1.672222in}{2.625357in}}%
\pgfpathlineto{\pgfqpoint{1.614426in}{2.613411in}}%
\pgfpathlineto{\pgfqpoint{1.561292in}{2.600164in}}%
\pgfpathlineto{\pgfqpoint{1.512849in}{2.585863in}}%
\pgfpathlineto{\pgfqpoint{1.467041in}{2.570024in}}%
\pgfpathlineto{\pgfqpoint{1.425987in}{2.553559in}}%
\pgfpathlineto{\pgfqpoint{1.387657in}{2.535878in}}%
\pgfpathlineto{\pgfqpoint{1.352107in}{2.517107in}}%
\pgfpathlineto{\pgfqpoint{1.319380in}{2.497414in}}%
\pgfpathlineto{\pgfqpoint{1.287658in}{2.475675in}}%
\pgfpathlineto{\pgfqpoint{1.258901in}{2.453249in}}%
\pgfpathlineto{\pgfqpoint{1.231396in}{2.428856in}}%
\pgfpathlineto{\pgfqpoint{1.206910in}{2.404185in}}%
\pgfpathlineto{\pgfqpoint{1.183834in}{2.377796in}}%
\pgfpathlineto{\pgfqpoint{1.162297in}{2.349754in}}%
\pgfpathlineto{\pgfqpoint{1.142408in}{2.320162in}}%
\pgfpathlineto{\pgfqpoint{1.124243in}{2.289153in}}%
\pgfpathlineto{\pgfqpoint{1.107844in}{2.256880in}}%
\pgfpathlineto{\pgfqpoint{1.092313in}{2.221246in}}%
\pgfpathlineto{\pgfqpoint{1.078787in}{2.184556in}}%
\pgfpathlineto{\pgfqpoint{1.067223in}{2.146997in}}%
\pgfpathlineto{\pgfqpoint{1.057021in}{2.106330in}}%
\pgfpathlineto{\pgfqpoint{1.048890in}{2.065055in}}%
\pgfpathlineto{\pgfqpoint{1.042450in}{2.020865in}}%
\pgfpathlineto{\pgfqpoint{1.038163in}{1.976334in}}%
\pgfpathlineto{\pgfqpoint{1.035902in}{1.929116in}}%
\pgfpathlineto{\pgfqpoint{1.035910in}{1.881827in}}%
\pgfpathlineto{\pgfqpoint{1.038162in}{1.834608in}}%
\pgfpathlineto{\pgfqpoint{1.042652in}{1.787599in}}%
\pgfpathlineto{\pgfqpoint{1.049402in}{1.740946in}}%
\pgfpathlineto{\pgfqpoint{1.057917in}{1.697218in}}%
\pgfpathlineto{\pgfqpoint{1.068544in}{1.654101in}}%
\pgfpathlineto{\pgfqpoint{1.081337in}{1.611761in}}%
\pgfpathlineto{\pgfqpoint{1.095457in}{1.572657in}}%
\pgfpathlineto{\pgfqpoint{1.111594in}{1.534589in}}%
\pgfpathlineto{\pgfqpoint{1.128647in}{1.499876in}}%
\pgfpathlineto{\pgfqpoint{1.147510in}{1.466413in}}%
\pgfpathlineto{\pgfqpoint{1.168160in}{1.434361in}}%
\pgfpathlineto{\pgfqpoint{1.190550in}{1.403877in}}%
\pgfpathlineto{\pgfqpoint{1.214604in}{1.375102in}}%
\pgfpathlineto{\pgfqpoint{1.238576in}{1.349778in}}%
\pgfpathlineto{\pgfqpoint{1.265545in}{1.324611in}}%
\pgfpathlineto{\pgfqpoint{1.293821in}{1.301399in}}%
\pgfpathlineto{\pgfqpoint{1.323258in}{1.280160in}}%
\pgfpathlineto{\pgfqpoint{1.353707in}{1.260878in}}%
\pgfpathlineto{\pgfqpoint{1.387006in}{1.242484in}}%
\pgfpathlineto{\pgfqpoint{1.421123in}{1.226168in}}%
\pgfpathlineto{\pgfqpoint{1.457975in}{1.211038in}}%
\pgfpathlineto{\pgfqpoint{1.497519in}{1.197300in}}%
\pgfpathlineto{\pgfqpoint{1.539695in}{1.185100in}}%
\pgfpathlineto{\pgfqpoint{1.584426in}{1.174503in}}%
\pgfpathlineto{\pgfqpoint{1.635920in}{1.164678in}}%
\pgfpathlineto{\pgfqpoint{1.707057in}{1.153671in}}%
\pgfpathlineto{\pgfqpoint{1.767340in}{1.143747in}}%
\pgfpathlineto{\pgfqpoint{1.794998in}{1.137044in}}%
\pgfpathlineto{\pgfqpoint{1.813643in}{1.130253in}}%
\pgfpathlineto{\pgfqpoint{1.825393in}{1.123787in}}%
\pgfpathlineto{\pgfqpoint{1.834055in}{1.116311in}}%
\pgfpathlineto{\pgfqpoint{1.839207in}{1.108343in}}%
\pgfpathlineto{\pgfqpoint{1.841142in}{1.101245in}}%
\pgfpathlineto{\pgfqpoint{1.841017in}{1.093809in}}%
\pgfpathlineto{\pgfqpoint{1.838182in}{1.084437in}}%
\pgfpathlineto{\pgfqpoint{1.833412in}{1.076129in}}%
\pgfpathlineto{\pgfqpoint{1.825926in}{1.067118in}}%
\pgfpathlineto{\pgfqpoint{1.813875in}{1.056492in}}%
\pgfpathlineto{\pgfqpoint{1.798853in}{1.046460in}}%
\pgfpathlineto{\pgfqpoint{1.781031in}{1.037205in}}%
\pgfpathlineto{\pgfqpoint{1.758448in}{1.028177in}}%
\pgfpathlineto{\pgfqpoint{1.733196in}{1.020637in}}%
\pgfpathlineto{\pgfqpoint{1.705399in}{1.014723in}}%
\pgfpathlineto{\pgfqpoint{1.675164in}{1.010589in}}%
\pgfpathlineto{\pgfqpoint{1.642595in}{1.008402in}}%
\pgfpathlineto{\pgfqpoint{1.607794in}{1.008344in}}%
\pgfpathlineto{\pgfqpoint{1.570872in}{1.010618in}}%
\pgfpathlineto{\pgfqpoint{1.534105in}{1.015122in}}%
\pgfpathlineto{\pgfqpoint{1.495444in}{1.022189in}}%
\pgfpathlineto{\pgfqpoint{1.457153in}{1.031534in}}%
\pgfpathlineto{\pgfqpoint{1.419332in}{1.043115in}}%
\pgfpathlineto{\pgfqpoint{1.382087in}{1.056922in}}%
\pgfpathlineto{\pgfqpoint{1.347543in}{1.072017in}}%
\pgfpathlineto{\pgfqpoint{1.313727in}{1.089133in}}%
\pgfpathlineto{\pgfqpoint{1.280761in}{1.108296in}}%
\pgfpathlineto{\pgfqpoint{1.248778in}{1.129529in}}%
\pgfpathlineto{\pgfqpoint{1.219702in}{1.151410in}}%
\pgfpathlineto{\pgfqpoint{1.191742in}{1.175121in}}%
\pgfpathlineto{\pgfqpoint{1.165018in}{1.200628in}}%
\pgfpathlineto{\pgfqpoint{1.139639in}{1.227874in}}%
\pgfpathlineto{\pgfqpoint{1.115699in}{1.256776in}}%
\pgfpathlineto{\pgfqpoint{1.093273in}{1.287227in}}%
\pgfpathlineto{\pgfqpoint{1.071162in}{1.321139in}}%
\pgfpathlineto{\pgfqpoint{1.050853in}{1.356495in}}%
\pgfpathlineto{\pgfqpoint{1.032349in}{1.393127in}}%
\pgfpathlineto{\pgfqpoint{1.014700in}{1.433116in}}%
\pgfpathlineto{\pgfqpoint{0.999004in}{1.474159in}}%
\pgfpathlineto{\pgfqpoint{0.984484in}{1.518433in}}%
\pgfpathlineto{\pgfqpoint{0.971984in}{1.563508in}}%
\pgfpathlineto{\pgfqpoint{0.960915in}{1.611648in}}%
\pgfpathlineto{\pgfqpoint{0.951499in}{1.662794in}}%
\pgfpathlineto{\pgfqpoint{0.944255in}{1.714401in}}%
\pgfpathlineto{\pgfqpoint{0.938921in}{1.768816in}}%
\pgfpathlineto{\pgfqpoint{0.935845in}{1.823461in}}%
\pgfpathlineto{\pgfqpoint{0.935014in}{1.878210in}}%
\pgfpathlineto{\pgfqpoint{0.936451in}{1.932942in}}%
\pgfpathlineto{\pgfqpoint{0.939997in}{1.985053in}}%
\pgfpathlineto{\pgfqpoint{0.945755in}{2.036904in}}%
\pgfpathlineto{\pgfqpoint{0.953410in}{2.085906in}}%
\pgfpathlineto{\pgfqpoint{0.962766in}{2.131968in}}%
\pgfpathlineto{\pgfqpoint{0.974287in}{2.177381in}}%
\pgfpathlineto{\pgfqpoint{0.987331in}{2.219621in}}%
\pgfpathlineto{\pgfqpoint{1.001662in}{2.258624in}}%
\pgfpathlineto{\pgfqpoint{1.018041in}{2.296556in}}%
\pgfpathlineto{\pgfqpoint{1.035387in}{2.331077in}}%
\pgfpathlineto{\pgfqpoint{1.054631in}{2.364255in}}%
\pgfpathlineto{\pgfqpoint{1.074383in}{2.393966in}}%
\pgfpathlineto{\pgfqpoint{1.095745in}{2.422183in}}%
\pgfpathlineto{\pgfqpoint{1.118633in}{2.448786in}}%
\pgfpathlineto{\pgfqpoint{1.142935in}{2.473693in}}%
\pgfpathlineto{\pgfqpoint{1.168517in}{2.496862in}}%
\pgfpathlineto{\pgfqpoint{1.197050in}{2.519660in}}%
\pgfpathlineto{\pgfqpoint{1.226690in}{2.540526in}}%
\pgfpathlineto{\pgfqpoint{1.259205in}{2.560675in}}%
\pgfpathlineto{\pgfqpoint{1.294574in}{2.579885in}}%
\pgfpathlineto{\pgfqpoint{1.332754in}{2.597987in}}%
\pgfpathlineto{\pgfqpoint{1.373679in}{2.614867in}}%
\pgfpathlineto{\pgfqpoint{1.417280in}{2.630455in}}%
\pgfpathlineto{\pgfqpoint{1.465592in}{2.645324in}}%
\pgfpathlineto{\pgfqpoint{1.518599in}{2.659218in}}%
\pgfpathlineto{\pgfqpoint{1.576268in}{2.671945in}}%
\pgfpathlineto{\pgfqpoint{1.638556in}{2.683360in}}%
\pgfpathlineto{\pgfqpoint{1.705421in}{2.693360in}}%
\pgfpathlineto{\pgfqpoint{1.778986in}{2.702082in}}%
\pgfpathlineto{\pgfqpoint{1.857056in}{2.709096in}}%
\pgfpathlineto{\pgfqpoint{1.939591in}{2.714300in}}%
\pgfpathlineto{\pgfqpoint{2.026557in}{2.717535in}}%
\pgfpathlineto{\pgfqpoint{2.113564in}{2.718547in}}%
\pgfpathlineto{\pgfqpoint{2.198393in}{2.717330in}}%
\pgfpathlineto{\pgfqpoint{2.278824in}{2.713959in}}%
\pgfpathlineto{\pgfqpoint{2.352636in}{2.708632in}}%
\pgfpathlineto{\pgfqpoint{2.417616in}{2.701747in}}%
\pgfpathlineto{\pgfqpoint{2.473730in}{2.693673in}}%
\pgfpathlineto{\pgfqpoint{2.523101in}{2.684417in}}%
\pgfpathlineto{\pgfqpoint{2.565688in}{2.674259in}}%
\pgfpathlineto{\pgfqpoint{2.603563in}{2.662913in}}%
\pgfpathlineto{\pgfqpoint{2.634591in}{2.651378in}}%
\pgfpathlineto{\pgfqpoint{2.660862in}{2.639417in}}%
\pgfpathlineto{\pgfqpoint{2.684329in}{2.626353in}}%
\pgfpathlineto{\pgfqpoint{2.704857in}{2.612302in}}%
\pgfpathlineto{\pgfqpoint{2.722345in}{2.597516in}}%
\pgfpathlineto{\pgfqpoint{2.736786in}{2.582404in}}%
\pgfpathlineto{\pgfqpoint{2.749661in}{2.565545in}}%
\pgfpathlineto{\pgfqpoint{2.760687in}{2.547049in}}%
\pgfpathlineto{\pgfqpoint{2.769704in}{2.527178in}}%
\pgfpathlineto{\pgfqpoint{2.776725in}{2.506277in}}%
\pgfpathlineto{\pgfqpoint{2.782387in}{2.482254in}}%
\pgfpathlineto{\pgfqpoint{2.786421in}{2.455273in}}%
\pgfpathlineto{\pgfqpoint{2.788874in}{2.423043in}}%
\pgfpathlineto{\pgfqpoint{2.789447in}{2.385717in}}%
\pgfpathlineto{\pgfqpoint{2.787874in}{2.338461in}}%
\pgfpathlineto{\pgfqpoint{2.783296in}{2.273958in}}%
\pgfpathlineto{\pgfqpoint{2.773564in}{2.172511in}}%
\pgfpathlineto{\pgfqpoint{2.744488in}{1.878175in}}%
\pgfpathlineto{\pgfqpoint{2.730844in}{1.712132in}}%
\pgfpathlineto{\pgfqpoint{2.718053in}{1.528515in}}%
\pgfpathlineto{\pgfqpoint{2.703691in}{1.287623in}}%
\pgfpathlineto{\pgfqpoint{2.684065in}{0.959820in}}%
\pgfpathlineto{\pgfqpoint{2.674904in}{0.845800in}}%
\pgfpathlineto{\pgfqpoint{2.666418in}{0.766744in}}%
\pgfpathlineto{\pgfqpoint{2.657955in}{0.707799in}}%
\pgfpathlineto{\pgfqpoint{2.649145in}{0.661598in}}%
\pgfpathlineto{\pgfqpoint{2.640173in}{0.625706in}}%
\pgfpathlineto{\pgfqpoint{2.630336in}{0.595378in}}%
\pgfpathlineto{\pgfqpoint{2.620019in}{0.570684in}}%
\pgfpathlineto{\pgfqpoint{2.608682in}{0.549454in}}%
\pgfpathlineto{\pgfqpoint{2.596685in}{0.531765in}}%
\pgfpathlineto{\pgfqpoint{2.582963in}{0.515804in}}%
\pgfpathlineto{\pgfqpoint{2.567673in}{0.501833in}}%
\pgfpathlineto{\pgfqpoint{2.551109in}{0.489912in}}%
\pgfpathlineto{\pgfqpoint{2.531609in}{0.478900in}}%
\pgfpathlineto{\pgfqpoint{2.509274in}{0.469100in}}%
\pgfpathlineto{\pgfqpoint{2.484255in}{0.460598in}}%
\pgfpathlineto{\pgfqpoint{2.454569in}{0.452842in}}%
\pgfpathlineto{\pgfqpoint{2.418129in}{0.445656in}}%
\pgfpathlineto{\pgfqpoint{2.372814in}{0.439074in}}%
\pgfpathlineto{\pgfqpoint{2.316491in}{0.433202in}}%
\pgfpathlineto{\pgfqpoint{2.242678in}{0.427865in}}%
\pgfpathlineto{\pgfqpoint{2.144875in}{0.423186in}}%
\pgfpathlineto{\pgfqpoint{2.012226in}{0.419228in}}%
\pgfpathlineto{\pgfqpoint{1.827344in}{0.416092in}}%
\pgfpathlineto{\pgfqpoint{1.566311in}{0.414030in}}%
\pgfpathlineto{\pgfqpoint{1.237837in}{0.413695in}}%
\pgfpathlineto{\pgfqpoint{0.959400in}{0.415471in}}%
\pgfpathlineto{\pgfqpoint{0.785396in}{0.418626in}}%
\pgfpathlineto{\pgfqpoint{0.681040in}{0.422584in}}%
\pgfpathlineto{\pgfqpoint{0.615904in}{0.427127in}}%
\pgfpathlineto{\pgfqpoint{0.572634in}{0.432262in}}%
\pgfpathlineto{\pgfqpoint{0.542608in}{0.438032in}}%
\pgfpathlineto{\pgfqpoint{0.521569in}{0.444317in}}%
\pgfpathlineto{\pgfqpoint{0.507369in}{0.450581in}}%
\pgfpathlineto{\pgfqpoint{0.495951in}{0.457786in}}%
\pgfpathlineto{\pgfqpoint{0.487377in}{0.465422in}}%
\pgfpathlineto{\pgfqpoint{0.480041in}{0.474591in}}%
\pgfpathlineto{\pgfqpoint{0.473121in}{0.487228in}}%
\pgfpathlineto{\pgfqpoint{0.467368in}{0.503343in}}%
\pgfpathlineto{\pgfqpoint{0.462960in}{0.522597in}}%
\pgfpathlineto{\pgfqpoint{0.459133in}{0.549618in}}%
\pgfpathlineto{\pgfqpoint{0.455930in}{0.589271in}}%
\pgfpathlineto{\pgfqpoint{0.453357in}{0.651428in}}%
\pgfpathlineto{\pgfqpoint{0.451405in}{0.758439in}}%
\pgfpathlineto{\pgfqpoint{0.450035in}{0.977483in}}%
\pgfpathlineto{\pgfqpoint{0.449263in}{1.545020in}}%
\pgfpathlineto{\pgfqpoint{0.449802in}{2.762240in}}%
\pgfpathlineto{\pgfqpoint{0.451339in}{2.864274in}}%
\pgfpathlineto{\pgfqpoint{0.453304in}{2.881521in}}%
\pgfpathlineto{\pgfqpoint{0.455130in}{2.886013in}}%
\pgfpathlineto{\pgfqpoint{0.458531in}{2.888986in}}%
\pgfpathlineto{\pgfqpoint{0.464894in}{2.890540in}}%
\pgfpathlineto{\pgfqpoint{0.482275in}{2.891421in}}%
\pgfpathlineto{\pgfqpoint{0.564936in}{2.891729in}}%
\pgfpathlineto{\pgfqpoint{2.729390in}{2.891760in}}%
\pgfpathlineto{\pgfqpoint{4.789408in}{2.890892in}}%
\pgfpathlineto{\pgfqpoint{4.793630in}{2.889758in}}%
\pgfpathlineto{\pgfqpoint{4.795405in}{2.888364in}}%
\pgfpathlineto{\pgfqpoint{4.797080in}{2.881225in}}%
\pgfpathlineto{\pgfqpoint{4.797985in}{2.858852in}}%
\pgfpathlineto{\pgfqpoint{4.798028in}{2.856363in}}%
\pgfpathlineto{\pgfqpoint{4.798028in}{2.856363in}}%
\pgfusepath{stroke}%
\end{pgfscope}%
\begin{pgfscope}%
\pgfpathrectangle{\pgfqpoint{0.448634in}{0.402556in}}{\pgfqpoint{4.350661in}{2.489204in}} %
\pgfusepath{clip}%
\pgfsetrectcap%
\pgfsetroundjoin%
\pgfsetlinewidth{1.003750pt}%
\definecolor{currentstroke}{rgb}{0.498039,0.498039,0.498039}%
\pgfsetstrokecolor{currentstroke}%
\pgfsetdash{}{0pt}%
\pgfpathmoveto{\pgfqpoint{4.798840in}{2.852369in}}%
\pgfpathlineto{\pgfqpoint{4.797564in}{2.889610in}}%
\pgfpathlineto{\pgfqpoint{4.796215in}{2.891483in}}%
\pgfpathlineto{\pgfqpoint{4.787551in}{2.891760in}}%
\pgfpathlineto{\pgfqpoint{0.452128in}{2.891660in}}%
\pgfpathlineto{\pgfqpoint{0.450530in}{2.890083in}}%
\pgfpathlineto{\pgfqpoint{0.449454in}{2.882764in}}%
\pgfpathlineto{\pgfqpoint{0.448970in}{2.845432in}}%
\pgfpathlineto{\pgfqpoint{0.448743in}{2.494455in}}%
\pgfpathlineto{\pgfqpoint{0.449624in}{0.615108in}}%
\pgfpathlineto{\pgfqpoint{0.451433in}{0.510587in}}%
\pgfpathlineto{\pgfqpoint{0.453994in}{0.473375in}}%
\pgfpathlineto{\pgfqpoint{0.457407in}{0.453869in}}%
\pgfpathlineto{\pgfqpoint{0.461541in}{0.442385in}}%
\pgfpathlineto{\pgfqpoint{0.466739in}{0.434437in}}%
\pgfpathlineto{\pgfqpoint{0.473594in}{0.428350in}}%
\pgfpathlineto{\pgfqpoint{0.483492in}{0.423243in}}%
\pgfpathlineto{\pgfqpoint{0.491853in}{0.420500in}}%
\pgfpathlineto{\pgfqpoint{0.491853in}{0.420500in}}%
\pgfusepath{stroke}%
\end{pgfscope}%
\begin{pgfscope}%
\pgfpathrectangle{\pgfqpoint{0.448634in}{0.402556in}}{\pgfqpoint{4.350661in}{2.489204in}} %
\pgfusepath{clip}%
\pgfsetrectcap%
\pgfsetroundjoin%
\pgfsetlinewidth{1.003750pt}%
\definecolor{currentstroke}{rgb}{0.498039,0.498039,0.498039}%
\pgfsetstrokecolor{currentstroke}%
\pgfsetdash{}{0pt}%
\pgfpathmoveto{\pgfqpoint{2.583205in}{2.736528in}}%
\pgfpathlineto{\pgfqpoint{2.637031in}{2.727670in}}%
\pgfpathlineto{\pgfqpoint{2.681949in}{2.718174in}}%
\pgfpathlineto{\pgfqpoint{2.720074in}{2.707978in}}%
\pgfpathlineto{\pgfqpoint{2.751388in}{2.697502in}}%
\pgfpathlineto{\pgfqpoint{2.777962in}{2.686454in}}%
\pgfpathlineto{\pgfqpoint{2.799777in}{2.675219in}}%
\pgfpathlineto{\pgfqpoint{2.818755in}{2.663072in}}%
\pgfpathlineto{\pgfqpoint{2.834761in}{2.650192in}}%
\pgfpathlineto{\pgfqpoint{2.847732in}{2.636934in}}%
\pgfpathlineto{\pgfqpoint{2.857760in}{2.623836in}}%
\pgfpathlineto{\pgfqpoint{2.866245in}{2.609383in}}%
\pgfpathlineto{\pgfqpoint{2.872978in}{2.593768in}}%
\pgfpathlineto{\pgfqpoint{2.878468in}{2.574889in}}%
\pgfpathlineto{\pgfqpoint{2.881790in}{2.555353in}}%
\pgfpathlineto{\pgfqpoint{2.883376in}{2.533034in}}%
\pgfpathlineto{\pgfqpoint{2.882977in}{2.505666in}}%
\pgfpathlineto{\pgfqpoint{2.880229in}{2.473466in}}%
\pgfpathlineto{\pgfqpoint{2.874373in}{2.431688in}}%
\pgfpathlineto{\pgfqpoint{2.862975in}{2.368300in}}%
\pgfpathlineto{\pgfqpoint{2.816797in}{2.122519in}}%
\pgfpathlineto{\pgfqpoint{2.801089in}{2.022060in}}%
\pgfpathlineto{\pgfqpoint{2.786839in}{1.916276in}}%
\pgfpathlineto{\pgfqpoint{2.774124in}{1.805213in}}%
\pgfpathlineto{\pgfqpoint{2.762480in}{1.683974in}}%
\pgfpathlineto{\pgfqpoint{2.751869in}{1.550108in}}%
\pgfpathlineto{\pgfqpoint{2.742307in}{1.401159in}}%
\pgfpathlineto{\pgfqpoint{2.733735in}{1.232179in}}%
\pgfpathlineto{\pgfqpoint{2.726273in}{1.040702in}}%
\pgfpathlineto{\pgfqpoint{2.718961in}{0.791923in}}%
\pgfpathlineto{\pgfqpoint{2.712034in}{0.570530in}}%
\pgfpathlineto{\pgfqpoint{2.708008in}{0.508477in}}%
\pgfpathlineto{\pgfqpoint{2.703672in}{0.473996in}}%
\pgfpathlineto{\pgfqpoint{2.699331in}{0.454725in}}%
\pgfpathlineto{\pgfqpoint{2.694147in}{0.441042in}}%
\pgfpathlineto{\pgfqpoint{2.687776in}{0.430993in}}%
\pgfpathlineto{\pgfqpoint{2.681170in}{0.424542in}}%
\pgfpathlineto{\pgfqpoint{2.671567in}{0.418751in}}%
\pgfpathlineto{\pgfqpoint{2.659107in}{0.414356in}}%
\pgfpathlineto{\pgfqpoint{2.641970in}{0.410936in}}%
\pgfpathlineto{\pgfqpoint{2.615980in}{0.408204in}}%
\pgfpathlineto{\pgfqpoint{2.570341in}{0.405996in}}%
\pgfpathlineto{\pgfqpoint{2.483340in}{0.404421in}}%
\pgfpathlineto{\pgfqpoint{2.278861in}{0.403454in}}%
\pgfpathlineto{\pgfqpoint{1.543599in}{0.402942in}}%
\pgfpathlineto{\pgfqpoint{0.536423in}{0.403923in}}%
\pgfpathlineto{\pgfqpoint{0.471185in}{0.405627in}}%
\pgfpathlineto{\pgfqpoint{0.458228in}{0.407301in}}%
\pgfpathlineto{\pgfqpoint{0.454130in}{0.408923in}}%
\pgfpathlineto{\pgfqpoint{0.451096in}{0.412347in}}%
\pgfpathlineto{\pgfqpoint{0.449593in}{0.419578in}}%
\pgfpathlineto{\pgfqpoint{0.448903in}{0.441960in}}%
\pgfpathlineto{\pgfqpoint{0.448660in}{0.586333in}}%
\pgfpathlineto{\pgfqpoint{0.448679in}{2.891336in}}%
\pgfpathlineto{\pgfqpoint{0.448679in}{2.891336in}}%
\pgfusepath{stroke}%
\end{pgfscope}%
\begin{pgfscope}%
\pgfpathrectangle{\pgfqpoint{0.448634in}{0.402556in}}{\pgfqpoint{4.350661in}{2.489204in}} %
\pgfusepath{clip}%
\pgfsetrectcap%
\pgfsetroundjoin%
\pgfsetlinewidth{1.003750pt}%
\definecolor{currentstroke}{rgb}{0.498039,0.498039,0.498039}%
\pgfsetstrokecolor{currentstroke}%
\pgfsetdash{}{0pt}%
\pgfpathmoveto{\pgfqpoint{3.427742in}{0.402583in}}%
\pgfpathlineto{\pgfqpoint{2.779498in}{0.403654in}}%
\pgfpathlineto{\pgfqpoint{2.753448in}{0.405400in}}%
\pgfpathlineto{\pgfqpoint{2.742839in}{0.408014in}}%
\pgfpathlineto{\pgfqpoint{2.737219in}{0.411735in}}%
\pgfpathlineto{\pgfqpoint{2.733222in}{0.417586in}}%
\pgfpathlineto{\pgfqpoint{2.730300in}{0.426941in}}%
\pgfpathlineto{\pgfqpoint{2.727904in}{0.444132in}}%
\pgfpathlineto{\pgfqpoint{2.726113in}{0.478916in}}%
\pgfpathlineto{\pgfqpoint{2.724973in}{0.551088in}}%
\pgfpathlineto{\pgfqpoint{2.725338in}{0.687992in}}%
\pgfpathlineto{\pgfqpoint{2.728071in}{0.872166in}}%
\pgfpathlineto{\pgfqpoint{2.733135in}{1.071215in}}%
\pgfpathlineto{\pgfqpoint{2.739997in}{1.250264in}}%
\pgfpathlineto{\pgfqpoint{2.748898in}{1.421715in}}%
\pgfpathlineto{\pgfqpoint{2.758925in}{1.570624in}}%
\pgfpathlineto{\pgfqpoint{2.771671in}{1.724249in}}%
\pgfpathlineto{\pgfqpoint{2.783945in}{1.840393in}}%
\pgfpathlineto{\pgfqpoint{2.800730in}{1.975909in}}%
\pgfpathlineto{\pgfqpoint{2.815815in}{2.073966in}}%
\pgfpathlineto{\pgfqpoint{2.832741in}{2.169091in}}%
\pgfpathlineto{\pgfqpoint{2.852323in}{2.266102in}}%
\pgfpathlineto{\pgfqpoint{2.892461in}{2.459809in}}%
\pgfpathlineto{\pgfqpoint{2.898586in}{2.501533in}}%
\pgfpathlineto{\pgfqpoint{2.900938in}{2.531273in}}%
\pgfpathlineto{\pgfqpoint{2.900867in}{2.556154in}}%
\pgfpathlineto{\pgfqpoint{2.898554in}{2.578385in}}%
\pgfpathlineto{\pgfqpoint{2.894191in}{2.597645in}}%
\pgfpathlineto{\pgfqpoint{2.888330in}{2.613712in}}%
\pgfpathlineto{\pgfqpoint{2.880538in}{2.628665in}}%
\pgfpathlineto{\pgfqpoint{2.871005in}{2.642236in}}%
\pgfpathlineto{\pgfqpoint{2.858390in}{2.655932in}}%
\pgfpathlineto{\pgfqpoint{2.844425in}{2.667798in}}%
\pgfpathlineto{\pgfqpoint{2.827602in}{2.679240in}}%
\pgfpathlineto{\pgfqpoint{2.808000in}{2.690018in}}%
\pgfpathlineto{\pgfqpoint{2.783673in}{2.700833in}}%
\pgfpathlineto{\pgfqpoint{2.754614in}{2.711243in}}%
\pgfpathlineto{\pgfqpoint{2.720868in}{2.720976in}}%
\pgfpathlineto{\pgfqpoint{2.680356in}{2.730325in}}%
\pgfpathlineto{\pgfqpoint{2.633100in}{2.738959in}}%
\pgfpathlineto{\pgfqpoint{2.576976in}{2.746948in}}%
\pgfpathlineto{\pgfqpoint{2.509839in}{2.754161in}}%
\pgfpathlineto{\pgfqpoint{2.433879in}{2.760061in}}%
\pgfpathlineto{\pgfqpoint{2.346957in}{2.764578in}}%
\pgfpathlineto{\pgfqpoint{2.253448in}{2.767222in}}%
\pgfpathlineto{\pgfqpoint{2.149036in}{2.768103in}}%
\pgfpathlineto{\pgfqpoint{2.040277in}{2.766778in}}%
\pgfpathlineto{\pgfqpoint{1.929382in}{2.763186in}}%
\pgfpathlineto{\pgfqpoint{1.827263in}{2.757526in}}%
\pgfpathlineto{\pgfqpoint{1.725264in}{2.749583in}}%
\pgfpathlineto{\pgfqpoint{1.640797in}{2.740547in}}%
\pgfpathlineto{\pgfqpoint{1.565190in}{2.730305in}}%
\pgfpathlineto{\pgfqpoint{1.466154in}{2.714252in}}%
\pgfpathlineto{\pgfqpoint{1.406336in}{2.701132in}}%
\pgfpathlineto{\pgfqpoint{1.353313in}{2.687318in}}%
\pgfpathlineto{\pgfqpoint{1.304982in}{2.672527in}}%
\pgfpathlineto{\pgfqpoint{1.259299in}{2.656222in}}%
\pgfpathlineto{\pgfqpoint{1.216338in}{2.638471in}}%
\pgfpathlineto{\pgfqpoint{1.178254in}{2.620108in}}%
\pgfpathlineto{\pgfqpoint{1.143023in}{2.600567in}}%
\pgfpathlineto{\pgfqpoint{1.110696in}{2.580028in}}%
\pgfpathlineto{\pgfqpoint{1.092562in}{2.566375in}}%
\pgfpathlineto{\pgfqpoint{1.080009in}{2.556524in}}%
\pgfpathlineto{\pgfqpoint{1.051881in}{2.533078in}}%
\pgfpathlineto{\pgfqpoint{1.026743in}{2.509282in}}%
\pgfpathlineto{\pgfqpoint{1.002933in}{2.483759in}}%
\pgfpathlineto{\pgfqpoint{0.980560in}{2.456588in}}%
\pgfpathlineto{\pgfqpoint{0.959701in}{2.427883in}}%
\pgfpathlineto{\pgfqpoint{0.939164in}{2.395737in}}%
\pgfpathlineto{\pgfqpoint{0.919394in}{2.359982in}}%
\pgfpathlineto{\pgfqpoint{0.901446in}{2.322990in}}%
\pgfpathlineto{\pgfqpoint{0.884454in}{2.282630in}}%
\pgfpathlineto{\pgfqpoint{0.868611in}{2.238954in}}%
\pgfpathlineto{\pgfqpoint{0.854057in}{2.192043in}}%
\pgfpathlineto{\pgfqpoint{0.842947in}{2.149092in}}%
\pgfpathlineto{\pgfqpoint{0.831003in}{2.096068in}}%
\pgfpathlineto{\pgfqpoint{0.820102in}{2.037648in}}%
\pgfpathlineto{\pgfqpoint{0.812500in}{1.986111in}}%
\pgfpathlineto{\pgfqpoint{0.796969in}{1.835344in}}%
\pgfpathlineto{\pgfqpoint{0.792025in}{1.753398in}}%
\pgfpathlineto{\pgfqpoint{0.788568in}{1.658893in}}%
\pgfpathlineto{\pgfqpoint{0.785056in}{1.522047in}}%
\pgfpathlineto{\pgfqpoint{0.785056in}{1.522047in}}%
\pgfusepath{stroke}%
\end{pgfscope}%
\begin{pgfscope}%
\pgfpathrectangle{\pgfqpoint{0.448634in}{0.402556in}}{\pgfqpoint{4.350661in}{2.489204in}} %
\pgfusepath{clip}%
\pgfsetrectcap%
\pgfsetroundjoin%
\pgfsetlinewidth{1.003750pt}%
\definecolor{currentstroke}{rgb}{0.498039,0.498039,0.498039}%
\pgfsetstrokecolor{currentstroke}%
\pgfsetdash{}{0pt}%
\pgfpathmoveto{\pgfqpoint{0.456540in}{2.527429in}}%
\pgfpathlineto{\pgfqpoint{0.459326in}{2.691683in}}%
\pgfpathlineto{\pgfqpoint{0.462680in}{2.771240in}}%
\pgfpathlineto{\pgfqpoint{0.466711in}{2.815797in}}%
\pgfpathlineto{\pgfqpoint{0.470964in}{2.840195in}}%
\pgfpathlineto{\pgfqpoint{0.475204in}{2.854305in}}%
\pgfpathlineto{\pgfqpoint{0.480565in}{2.865105in}}%
\pgfpathlineto{\pgfqpoint{0.486490in}{2.872364in}}%
\pgfpathlineto{\pgfqpoint{0.493744in}{2.877826in}}%
\pgfpathlineto{\pgfqpoint{0.503875in}{2.882302in}}%
\pgfpathlineto{\pgfqpoint{0.518778in}{2.885816in}}%
\pgfpathlineto{\pgfqpoint{0.540419in}{2.888282in}}%
\pgfpathlineto{\pgfqpoint{0.579540in}{2.890078in}}%
\pgfpathlineto{\pgfqpoint{0.664371in}{2.891173in}}%
\pgfpathlineto{\pgfqpoint{0.942813in}{2.891663in}}%
\pgfpathlineto{\pgfqpoint{3.759866in}{2.891747in}}%
\pgfpathlineto{\pgfqpoint{4.717007in}{2.890626in}}%
\pgfpathlineto{\pgfqpoint{4.749588in}{2.888754in}}%
\pgfpathlineto{\pgfqpoint{4.762457in}{2.886340in}}%
\pgfpathlineto{\pgfqpoint{4.770592in}{2.882877in}}%
\pgfpathlineto{\pgfqpoint{4.775823in}{2.878455in}}%
\pgfpathlineto{\pgfqpoint{4.779732in}{2.872503in}}%
\pgfpathlineto{\pgfqpoint{4.783093in}{2.863339in}}%
\pgfpathlineto{\pgfqpoint{4.786357in}{2.846336in}}%
\pgfpathlineto{\pgfqpoint{4.788932in}{2.819120in}}%
\pgfpathlineto{\pgfqpoint{4.791141in}{2.769404in}}%
\pgfpathlineto{\pgfqpoint{4.793037in}{2.669861in}}%
\pgfpathlineto{\pgfqpoint{4.794466in}{2.458286in}}%
\pgfpathlineto{\pgfqpoint{4.794957in}{2.012719in}}%
\pgfpathlineto{\pgfqpoint{4.793513in}{1.487501in}}%
\pgfpathlineto{\pgfqpoint{4.790680in}{1.181347in}}%
\pgfpathlineto{\pgfqpoint{4.787114in}{1.007152in}}%
\pgfpathlineto{\pgfqpoint{4.782703in}{0.890272in}}%
\pgfpathlineto{\pgfqpoint{4.777518in}{0.808346in}}%
\pgfpathlineto{\pgfqpoint{4.771671in}{0.748987in}}%
\pgfpathlineto{\pgfqpoint{4.764877in}{0.702342in}}%
\pgfpathlineto{\pgfqpoint{4.757342in}{0.666023in}}%
\pgfpathlineto{\pgfqpoint{4.749267in}{0.637627in}}%
\pgfpathlineto{\pgfqpoint{4.740658in}{0.614778in}}%
\pgfpathlineto{\pgfqpoint{4.731115in}{0.595229in}}%
\pgfpathlineto{\pgfqpoint{4.719669in}{0.577071in}}%
\pgfpathlineto{\pgfqpoint{4.707884in}{0.562434in}}%
\pgfpathlineto{\pgfqpoint{4.694723in}{0.549420in}}%
\pgfpathlineto{\pgfqpoint{4.678574in}{0.536776in}}%
\pgfpathlineto{\pgfqpoint{4.661369in}{0.526103in}}%
\pgfpathlineto{\pgfqpoint{4.641406in}{0.516233in}}%
\pgfpathlineto{\pgfqpoint{4.616700in}{0.506610in}}%
\pgfpathlineto{\pgfqpoint{4.587262in}{0.497704in}}%
\pgfpathlineto{\pgfqpoint{4.553162in}{0.489749in}}%
\pgfpathlineto{\pgfqpoint{4.510168in}{0.482155in}}%
\pgfpathlineto{\pgfqpoint{4.458298in}{0.475392in}}%
\pgfpathlineto{\pgfqpoint{4.395426in}{0.469504in}}%
\pgfpathlineto{\pgfqpoint{4.317239in}{0.464493in}}%
\pgfpathlineto{\pgfqpoint{4.221584in}{0.460670in}}%
\pgfpathlineto{\pgfqpoint{4.108485in}{0.458414in}}%
\pgfpathlineto{\pgfqpoint{3.980142in}{0.458122in}}%
\pgfpathlineto{\pgfqpoint{3.849634in}{0.460057in}}%
\pgfpathlineto{\pgfqpoint{3.730042in}{0.463977in}}%
\pgfpathlineto{\pgfqpoint{3.623568in}{0.469651in}}%
\pgfpathlineto{\pgfqpoint{3.534584in}{0.476526in}}%
\pgfpathlineto{\pgfqpoint{3.458771in}{0.484522in}}%
\pgfpathlineto{\pgfqpoint{3.393988in}{0.493509in}}%
\pgfpathlineto{\pgfqpoint{3.338105in}{0.503465in}}%
\pgfpathlineto{\pgfqpoint{3.289015in}{0.514500in}}%
\pgfpathlineto{\pgfqpoint{3.246755in}{0.526312in}}%
\pgfpathlineto{\pgfqpoint{3.209252in}{0.539175in}}%
\pgfpathlineto{\pgfqpoint{3.176550in}{0.552793in}}%
\pgfpathlineto{\pgfqpoint{3.148638in}{0.566717in}}%
\pgfpathlineto{\pgfqpoint{3.123520in}{0.581571in}}%
\pgfpathlineto{\pgfqpoint{3.099409in}{0.598465in}}%
\pgfpathlineto{\pgfqpoint{3.078276in}{0.615987in}}%
\pgfpathlineto{\pgfqpoint{3.058434in}{0.635384in}}%
\pgfpathlineto{\pgfqpoint{3.040067in}{0.656596in}}%
\pgfpathlineto{\pgfqpoint{3.023299in}{0.679479in}}%
\pgfpathlineto{\pgfqpoint{3.008187in}{0.703826in}}%
\pgfpathlineto{\pgfqpoint{2.993659in}{0.731579in}}%
\pgfpathlineto{\pgfqpoint{2.980047in}{0.762744in}}%
\pgfpathlineto{\pgfqpoint{2.967590in}{0.797246in}}%
\pgfpathlineto{\pgfqpoint{2.956447in}{0.834970in}}%
\pgfpathlineto{\pgfqpoint{2.946695in}{0.875783in}}%
\pgfpathlineto{\pgfqpoint{2.937958in}{0.922004in}}%
\pgfpathlineto{\pgfqpoint{2.930529in}{0.973576in}}%
\pgfpathlineto{\pgfqpoint{2.924641in}{1.030426in}}%
\pgfpathlineto{\pgfqpoint{2.920492in}{1.092470in}}%
\pgfpathlineto{\pgfqpoint{2.918275in}{1.159627in}}%
\pgfpathlineto{\pgfqpoint{2.918197in}{1.231810in}}%
\pgfpathlineto{\pgfqpoint{2.920423in}{1.306439in}}%
\pgfpathlineto{\pgfqpoint{2.924879in}{1.380937in}}%
\pgfpathlineto{\pgfqpoint{2.931562in}{1.455217in}}%
\pgfpathlineto{\pgfqpoint{2.940217in}{1.526717in}}%
\pgfpathlineto{\pgfqpoint{2.950758in}{1.595359in}}%
\pgfpathlineto{\pgfqpoint{2.963137in}{1.661054in}}%
\pgfpathlineto{\pgfqpoint{2.976741in}{1.721301in}}%
\pgfpathlineto{\pgfqpoint{2.991960in}{1.778442in}}%
\pgfpathlineto{\pgfqpoint{3.007970in}{1.830044in}}%
\pgfpathlineto{\pgfqpoint{3.025289in}{1.878409in}}%
\pgfpathlineto{\pgfqpoint{3.043810in}{1.923450in}}%
\pgfpathlineto{\pgfqpoint{3.063393in}{1.965093in}}%
\pgfpathlineto{\pgfqpoint{3.083855in}{2.003287in}}%
\pgfpathlineto{\pgfqpoint{3.104981in}{2.038011in}}%
\pgfpathlineto{\pgfqpoint{3.126519in}{2.069289in}}%
\pgfpathlineto{\pgfqpoint{3.149691in}{2.098998in}}%
\pgfpathlineto{\pgfqpoint{3.172899in}{2.125236in}}%
\pgfpathlineto{\pgfqpoint{3.197519in}{2.149730in}}%
\pgfpathlineto{\pgfqpoint{3.223504in}{2.172301in}}%
\pgfpathlineto{\pgfqpoint{3.248930in}{2.191472in}}%
\pgfpathlineto{\pgfqpoint{3.275422in}{2.208647in}}%
\pgfpathlineto{\pgfqpoint{3.300915in}{2.222640in}}%
\pgfpathlineto{\pgfqpoint{3.327210in}{2.234525in}}%
\pgfpathlineto{\pgfqpoint{3.352150in}{2.243320in}}%
\pgfpathlineto{\pgfqpoint{3.373388in}{2.248672in}}%
\pgfpathlineto{\pgfqpoint{3.392811in}{2.251391in}}%
\pgfpathlineto{\pgfqpoint{3.408027in}{2.251398in}}%
\pgfpathlineto{\pgfqpoint{3.418753in}{2.249446in}}%
\pgfpathlineto{\pgfqpoint{3.426780in}{2.245697in}}%
\pgfpathlineto{\pgfqpoint{3.431542in}{2.240670in}}%
\pgfpathlineto{\pgfqpoint{3.433331in}{2.236160in}}%
\pgfpathlineto{\pgfqpoint{3.433513in}{2.228748in}}%
\pgfpathlineto{\pgfqpoint{3.430759in}{2.219338in}}%
\pgfpathlineto{\pgfqpoint{3.423943in}{2.206624in}}%
\pgfpathlineto{\pgfqpoint{3.411508in}{2.189333in}}%
\pgfpathlineto{\pgfqpoint{3.388871in}{2.162448in}}%
\pgfpathlineto{\pgfqpoint{3.317091in}{2.079155in}}%
\pgfpathlineto{\pgfqpoint{3.286940in}{2.039892in}}%
\pgfpathlineto{\pgfqpoint{3.261227in}{2.002870in}}%
\pgfpathlineto{\pgfqpoint{3.237182in}{1.964409in}}%
\pgfpathlineto{\pgfqpoint{3.214915in}{1.924570in}}%
\pgfpathlineto{\pgfqpoint{3.194480in}{1.883466in}}%
\pgfpathlineto{\pgfqpoint{3.175877in}{1.841239in}}%
\pgfpathlineto{\pgfqpoint{3.159093in}{1.798025in}}%
\pgfpathlineto{\pgfqpoint{3.142652in}{1.749260in}}%
\pgfpathlineto{\pgfqpoint{3.128292in}{1.699642in}}%
\pgfpathlineto{\pgfqpoint{3.116010in}{1.649299in}}%
\pgfpathlineto{\pgfqpoint{3.105195in}{1.595957in}}%
\pgfpathlineto{\pgfqpoint{3.096093in}{1.539666in}}%
\pgfpathlineto{\pgfqpoint{3.089223in}{1.482962in}}%
\pgfpathlineto{\pgfqpoint{3.084532in}{1.425966in}}%
\pgfpathlineto{\pgfqpoint{3.082023in}{1.368791in}}%
\pgfpathlineto{\pgfqpoint{3.081753in}{1.311545in}}%
\pgfpathlineto{\pgfqpoint{3.083680in}{1.256832in}}%
\pgfpathlineto{\pgfqpoint{3.087696in}{1.204766in}}%
\pgfpathlineto{\pgfqpoint{3.093605in}{1.155448in}}%
\pgfpathlineto{\pgfqpoint{3.101784in}{1.106558in}}%
\pgfpathlineto{\pgfqpoint{3.111831in}{1.060688in}}%
\pgfpathlineto{\pgfqpoint{3.123641in}{1.017976in}}%
\pgfpathlineto{\pgfqpoint{3.136183in}{0.980831in}}%
\pgfpathlineto{\pgfqpoint{3.150730in}{0.944658in}}%
\pgfpathlineto{\pgfqpoint{3.167384in}{0.909696in}}%
\pgfpathlineto{\pgfqpoint{3.186205in}{0.876205in}}%
\pgfpathlineto{\pgfqpoint{3.204647in}{0.848482in}}%
\pgfpathlineto{\pgfqpoint{3.224740in}{0.822304in}}%
\pgfpathlineto{\pgfqpoint{3.246424in}{0.797844in}}%
\pgfpathlineto{\pgfqpoint{3.269554in}{0.775185in}}%
\pgfpathlineto{\pgfqpoint{3.293967in}{0.754362in}}%
\pgfpathlineto{\pgfqpoint{3.321344in}{0.734059in}}%
\pgfpathlineto{\pgfqpoint{3.349809in}{0.715817in}}%
\pgfpathlineto{\pgfqpoint{3.381114in}{0.698426in}}%
\pgfpathlineto{\pgfqpoint{3.415206in}{0.682044in}}%
\pgfpathlineto{\pgfqpoint{3.445962in}{0.669599in}}%
\pgfpathlineto{\pgfqpoint{3.483318in}{0.656196in}}%
\pgfpathlineto{\pgfqpoint{3.527433in}{0.642636in}}%
\pgfpathlineto{\pgfqpoint{3.572027in}{0.631331in}}%
\pgfpathlineto{\pgfqpoint{3.636108in}{0.617400in}}%
\pgfpathlineto{\pgfqpoint{3.692109in}{0.608354in}}%
\pgfpathlineto{\pgfqpoint{3.765579in}{0.598919in}}%
\pgfpathlineto{\pgfqpoint{3.832834in}{0.593310in}}%
\pgfpathlineto{\pgfqpoint{3.900201in}{0.589913in}}%
\pgfpathlineto{\pgfqpoint{3.971973in}{0.588408in}}%
\pgfpathlineto{\pgfqpoint{4.041578in}{0.589120in}}%
\pgfpathlineto{\pgfqpoint{4.113314in}{0.592106in}}%
\pgfpathlineto{\pgfqpoint{4.176248in}{0.597051in}}%
\pgfpathlineto{\pgfqpoint{4.236844in}{0.604075in}}%
\pgfpathlineto{\pgfqpoint{4.288550in}{0.612318in}}%
\pgfpathlineto{\pgfqpoint{4.335647in}{0.622019in}}%
\pgfpathlineto{\pgfqpoint{4.378075in}{0.633013in}}%
\pgfpathlineto{\pgfqpoint{4.415783in}{0.645067in}}%
\pgfpathlineto{\pgfqpoint{4.448745in}{0.657841in}}%
\pgfpathlineto{\pgfqpoint{4.480995in}{0.672803in}}%
\pgfpathlineto{\pgfqpoint{4.508339in}{0.688132in}}%
\pgfpathlineto{\pgfqpoint{4.532798in}{0.704360in}}%
\pgfpathlineto{\pgfqpoint{4.554350in}{0.721201in}}%
\pgfpathlineto{\pgfqpoint{4.574720in}{0.739869in}}%
\pgfpathlineto{\pgfqpoint{4.593716in}{0.760344in}}%
\pgfpathlineto{\pgfqpoint{4.611194in}{0.782520in}}%
\pgfpathlineto{\pgfqpoint{4.627061in}{0.806230in}}%
\pgfpathlineto{\pgfqpoint{4.642404in}{0.833404in}}%
\pgfpathlineto{\pgfqpoint{4.655976in}{0.861786in}}%
\pgfpathlineto{\pgfqpoint{4.668730in}{0.893424in}}%
\pgfpathlineto{\pgfqpoint{4.681220in}{0.930592in}}%
\pgfpathlineto{\pgfqpoint{4.692419in}{0.970915in}}%
\pgfpathlineto{\pgfqpoint{4.702828in}{1.016680in}}%
\pgfpathlineto{\pgfqpoint{4.712753in}{1.070248in}}%
\pgfpathlineto{\pgfqpoint{4.721833in}{1.131600in}}%
\pgfpathlineto{\pgfqpoint{4.729597in}{1.198215in}}%
\pgfpathlineto{\pgfqpoint{4.731478in}{1.220504in}}%
\pgfpathlineto{\pgfqpoint{4.738338in}{1.302269in}}%
\pgfpathlineto{\pgfqpoint{4.745179in}{1.409015in}}%
\pgfpathlineto{\pgfqpoint{4.750548in}{1.528337in}}%
\pgfpathlineto{\pgfqpoint{4.754938in}{1.672622in}}%
\pgfpathlineto{\pgfqpoint{4.757774in}{1.836876in}}%
\pgfpathlineto{\pgfqpoint{4.758912in}{2.023560in}}%
\pgfpathlineto{\pgfqpoint{4.757847in}{2.202777in}}%
\pgfpathlineto{\pgfqpoint{4.754776in}{2.357066in}}%
\pgfpathlineto{\pgfqpoint{4.750319in}{2.473945in}}%
\pgfpathlineto{\pgfqpoint{4.744787in}{2.563328in}}%
\pgfpathlineto{\pgfqpoint{4.738474in}{2.630143in}}%
\pgfpathlineto{\pgfqpoint{4.731757in}{2.679324in}}%
\pgfpathlineto{\pgfqpoint{4.723862in}{2.720658in}}%
\pgfpathlineto{\pgfqpoint{4.716270in}{2.749228in}}%
\pgfpathlineto{\pgfqpoint{4.708112in}{2.772293in}}%
\pgfpathlineto{\pgfqpoint{4.698861in}{2.792022in}}%
\pgfpathlineto{\pgfqpoint{4.688818in}{2.808270in}}%
\pgfpathlineto{\pgfqpoint{4.678457in}{2.821022in}}%
\pgfpathlineto{\pgfqpoint{4.666724in}{2.832109in}}%
\pgfpathlineto{\pgfqpoint{4.653865in}{2.841422in}}%
\pgfpathlineto{\pgfqpoint{4.638177in}{2.850017in}}%
\pgfpathlineto{\pgfqpoint{4.619730in}{2.857495in}}%
\pgfpathlineto{\pgfqpoint{4.596552in}{2.864265in}}%
\pgfpathlineto{\pgfqpoint{4.568719in}{2.869960in}}%
\pgfpathlineto{\pgfqpoint{4.532009in}{2.875038in}}%
\pgfpathlineto{\pgfqpoint{4.484296in}{2.879254in}}%
\pgfpathlineto{\pgfqpoint{4.416937in}{2.882874in}}%
\pgfpathlineto{\pgfqpoint{4.316906in}{2.885821in}}%
\pgfpathlineto{\pgfqpoint{4.158120in}{2.888056in}}%
\pgfpathlineto{\pgfqpoint{3.866630in}{2.889694in}}%
\pgfpathlineto{\pgfqpoint{3.227083in}{2.890637in}}%
\pgfpathlineto{\pgfqpoint{1.930586in}{2.890224in}}%
\pgfpathlineto{\pgfqpoint{1.378056in}{2.888262in}}%
\pgfpathlineto{\pgfqpoint{1.138784in}{2.885393in}}%
\pgfpathlineto{\pgfqpoint{1.006129in}{2.881705in}}%
\pgfpathlineto{\pgfqpoint{0.919208in}{2.877192in}}%
\pgfpathlineto{\pgfqpoint{0.858472in}{2.871976in}}%
\pgfpathlineto{\pgfqpoint{0.813089in}{2.866047in}}%
\pgfpathlineto{\pgfqpoint{0.776605in}{2.859167in}}%
\pgfpathlineto{\pgfqpoint{0.746927in}{2.851379in}}%
\pgfpathlineto{\pgfqpoint{0.721999in}{2.842540in}}%
\pgfpathlineto{\pgfqpoint{0.701849in}{2.833183in}}%
\pgfpathlineto{\pgfqpoint{0.684492in}{2.822840in}}%
\pgfpathlineto{\pgfqpoint{0.668179in}{2.810474in}}%
\pgfpathlineto{\pgfqpoint{0.654811in}{2.797738in}}%
\pgfpathlineto{\pgfqpoint{0.641271in}{2.781575in}}%
\pgfpathlineto{\pgfqpoint{0.629412in}{2.763763in}}%
\pgfpathlineto{\pgfqpoint{0.618143in}{2.742484in}}%
\pgfpathlineto{\pgfqpoint{0.607800in}{2.717803in}}%
\pgfpathlineto{\pgfqpoint{0.598548in}{2.689880in}}%
\pgfpathlineto{\pgfqpoint{0.589854in}{2.656489in}}%
\pgfpathlineto{\pgfqpoint{0.581582in}{2.615251in}}%
\pgfpathlineto{\pgfqpoint{0.574184in}{2.566197in}}%
\pgfpathlineto{\pgfqpoint{0.567903in}{2.509402in}}%
\pgfpathlineto{\pgfqpoint{0.562547in}{2.439978in}}%
\pgfpathlineto{\pgfqpoint{0.558567in}{2.357964in}}%
\pgfpathlineto{\pgfqpoint{0.556353in}{2.265901in}}%
\pgfpathlineto{\pgfqpoint{0.556155in}{2.163846in}}%
\pgfpathlineto{\pgfqpoint{0.558147in}{2.059327in}}%
\pgfpathlineto{\pgfqpoint{0.562269in}{1.957381in}}%
\pgfpathlineto{\pgfqpoint{0.568235in}{1.863041in}}%
\pgfpathlineto{\pgfqpoint{0.572343in}{1.813480in}}%
\pgfpathlineto{\pgfqpoint{0.572343in}{1.813480in}}%
\pgfusepath{stroke}%
\end{pgfscope}%
\begin{pgfscope}%
\pgfpathrectangle{\pgfqpoint{0.448634in}{0.402556in}}{\pgfqpoint{4.350661in}{2.489204in}} %
\pgfusepath{clip}%
\pgfsetrectcap%
\pgfsetroundjoin%
\pgfsetlinewidth{1.003750pt}%
\definecolor{currentstroke}{rgb}{0.737255,0.741176,0.133333}%
\pgfsetstrokecolor{currentstroke}%
\pgfsetdash{}{0pt}%
\pgfpathmoveto{\pgfqpoint{0.448634in}{2.896245in}}%
\pgfpathlineto{\pgfqpoint{0.448593in}{0.407043in}}%
\pgfpathlineto{\pgfqpoint{0.448593in}{0.407043in}}%
\pgfusepath{stroke}%
\end{pgfscope}%
\begin{pgfscope}%
\pgfpathrectangle{\pgfqpoint{0.448634in}{0.402556in}}{\pgfqpoint{4.350661in}{2.489204in}} %
\pgfusepath{clip}%
\pgfsetrectcap%
\pgfsetroundjoin%
\pgfsetlinewidth{1.003750pt}%
\definecolor{currentstroke}{rgb}{0.737255,0.741176,0.133333}%
\pgfsetstrokecolor{currentstroke}%
\pgfsetdash{}{0pt}%
\pgfpathmoveto{\pgfqpoint{1.875515in}{2.754162in}}%
\pgfpathlineto{\pgfqpoint{1.975495in}{2.758856in}}%
\pgfpathlineto{\pgfqpoint{2.077708in}{2.761461in}}%
\pgfpathlineto{\pgfqpoint{2.184297in}{2.761935in}}%
\pgfpathlineto{\pgfqpoint{2.284349in}{2.760197in}}%
\pgfpathlineto{\pgfqpoint{2.380002in}{2.756314in}}%
\pgfpathlineto{\pgfqpoint{2.462522in}{2.750805in}}%
\pgfpathlineto{\pgfqpoint{2.536218in}{2.743679in}}%
\pgfpathlineto{\pgfqpoint{2.598884in}{2.735403in}}%
\pgfpathlineto{\pgfqpoint{2.650494in}{2.726418in}}%
\pgfpathlineto{\pgfqpoint{2.695319in}{2.716362in}}%
\pgfpathlineto{\pgfqpoint{2.731207in}{2.706174in}}%
\pgfpathlineto{\pgfqpoint{2.762362in}{2.695094in}}%
\pgfpathlineto{\pgfqpoint{2.788703in}{2.683342in}}%
\pgfpathlineto{\pgfqpoint{2.810194in}{2.671323in}}%
\pgfpathlineto{\pgfqpoint{2.826913in}{2.659683in}}%
\pgfpathlineto{\pgfqpoint{2.842442in}{2.646062in}}%
\pgfpathlineto{\pgfqpoint{2.854807in}{2.632069in}}%
\pgfpathlineto{\pgfqpoint{2.864143in}{2.618319in}}%
\pgfpathlineto{\pgfqpoint{2.871811in}{2.603281in}}%
\pgfpathlineto{\pgfqpoint{2.877666in}{2.587209in}}%
\pgfpathlineto{\pgfqpoint{2.882185in}{2.567993in}}%
\pgfpathlineto{\pgfqpoint{2.884842in}{2.545811in}}%
\pgfpathlineto{\pgfqpoint{2.885423in}{2.520938in}}%
\pgfpathlineto{\pgfqpoint{2.883779in}{2.491135in}}%
\pgfpathlineto{\pgfqpoint{2.879413in}{2.454139in}}%
\pgfpathlineto{\pgfqpoint{2.870917in}{2.402782in}}%
\pgfpathlineto{\pgfqpoint{2.852926in}{2.310461in}}%
\pgfpathlineto{\pgfqpoint{2.826351in}{2.171877in}}%
\pgfpathlineto{\pgfqpoint{2.809494in}{2.071662in}}%
\pgfpathlineto{\pgfqpoint{2.794807in}{1.971000in}}%
\pgfpathlineto{\pgfqpoint{2.781561in}{1.865045in}}%
\pgfpathlineto{\pgfqpoint{2.769287in}{1.748901in}}%
\pgfpathlineto{\pgfqpoint{2.758176in}{1.622592in}}%
\pgfpathlineto{\pgfqpoint{2.748001in}{1.481188in}}%
\pgfpathlineto{\pgfqpoint{2.738868in}{1.322223in}}%
\pgfpathlineto{\pgfqpoint{2.730892in}{1.143235in}}%
\pgfpathlineto{\pgfqpoint{2.723924in}{0.934295in}}%
\pgfpathlineto{\pgfqpoint{2.710732in}{0.511429in}}%
\pgfpathlineto{\pgfqpoint{2.706953in}{0.474351in}}%
\pgfpathlineto{\pgfqpoint{2.702755in}{0.452485in}}%
\pgfpathlineto{\pgfqpoint{2.698007in}{0.438596in}}%
\pgfpathlineto{\pgfqpoint{2.693301in}{0.430242in}}%
\pgfpathlineto{\pgfqpoint{2.687071in}{0.423327in}}%
\pgfpathlineto{\pgfqpoint{2.679613in}{0.418234in}}%
\pgfpathlineto{\pgfqpoint{2.669389in}{0.414032in}}%
\pgfpathlineto{\pgfqpoint{2.654473in}{0.410580in}}%
\pgfpathlineto{\pgfqpoint{2.630675in}{0.407785in}}%
\pgfpathlineto{\pgfqpoint{2.591563in}{0.405732in}}%
\pgfpathlineto{\pgfqpoint{2.513263in}{0.404220in}}%
\pgfpathlineto{\pgfqpoint{2.324012in}{0.403318in}}%
\pgfpathlineto{\pgfqpoint{1.536542in}{0.402847in}}%
\pgfpathlineto{\pgfqpoint{0.518489in}{0.403862in}}%
\pgfpathlineto{\pgfqpoint{0.461960in}{0.405634in}}%
\pgfpathlineto{\pgfqpoint{0.453492in}{0.407690in}}%
\pgfpathlineto{\pgfqpoint{0.450609in}{0.411243in}}%
\pgfpathlineto{\pgfqpoint{0.449348in}{0.418542in}}%
\pgfpathlineto{\pgfqpoint{0.448806in}{0.443421in}}%
\pgfpathlineto{\pgfqpoint{0.448644in}{0.664960in}}%
\pgfpathlineto{\pgfqpoint{0.448661in}{2.890308in}}%
\pgfpathlineto{\pgfqpoint{0.448661in}{2.890308in}}%
\pgfusepath{stroke}%
\end{pgfscope}%
\begin{pgfscope}%
\pgfpathrectangle{\pgfqpoint{0.448634in}{0.402556in}}{\pgfqpoint{4.350661in}{2.489204in}} %
\pgfusepath{clip}%
\pgfsetrectcap%
\pgfsetroundjoin%
\pgfsetlinewidth{1.003750pt}%
\definecolor{currentstroke}{rgb}{0.737255,0.741176,0.133333}%
\pgfsetstrokecolor{currentstroke}%
\pgfsetdash{}{0pt}%
\pgfpathmoveto{\pgfqpoint{3.577928in}{0.462532in}}%
\pgfpathlineto{\pgfqpoint{3.488947in}{0.469464in}}%
\pgfpathlineto{\pgfqpoint{3.413143in}{0.477572in}}%
\pgfpathlineto{\pgfqpoint{3.350539in}{0.486440in}}%
\pgfpathlineto{\pgfqpoint{3.296837in}{0.496234in}}%
\pgfpathlineto{\pgfqpoint{3.249928in}{0.507057in}}%
\pgfpathlineto{\pgfqpoint{3.209844in}{0.518569in}}%
\pgfpathlineto{\pgfqpoint{3.174506in}{0.531023in}}%
\pgfpathlineto{\pgfqpoint{3.143941in}{0.544078in}}%
\pgfpathlineto{\pgfqpoint{3.116156in}{0.558330in}}%
\pgfpathlineto{\pgfqpoint{3.091232in}{0.573603in}}%
\pgfpathlineto{\pgfqpoint{3.069198in}{0.589607in}}%
\pgfpathlineto{\pgfqpoint{3.048324in}{0.607530in}}%
\pgfpathlineto{\pgfqpoint{3.028823in}{0.627373in}}%
\pgfpathlineto{\pgfqpoint{3.010869in}{0.649043in}}%
\pgfpathlineto{\pgfqpoint{2.994572in}{0.672367in}}%
\pgfpathlineto{\pgfqpoint{2.979969in}{0.697116in}}%
\pgfpathlineto{\pgfqpoint{2.966016in}{0.725254in}}%
\pgfpathlineto{\pgfqpoint{2.953036in}{0.756771in}}%
\pgfpathlineto{\pgfqpoint{2.941270in}{0.791589in}}%
\pgfpathlineto{\pgfqpoint{2.930875in}{0.829591in}}%
\pgfpathlineto{\pgfqpoint{2.921937in}{0.870648in}}%
\pgfpathlineto{\pgfqpoint{2.914124in}{0.917084in}}%
\pgfpathlineto{\pgfqpoint{2.907725in}{0.968837in}}%
\pgfpathlineto{\pgfqpoint{2.902951in}{1.025823in}}%
\pgfpathlineto{\pgfqpoint{2.899973in}{1.087956in}}%
\pgfpathlineto{\pgfqpoint{2.898946in}{1.155151in}}%
\pgfpathlineto{\pgfqpoint{2.900062in}{1.227323in}}%
\pgfpathlineto{\pgfqpoint{2.903422in}{1.301896in}}%
\pgfpathlineto{\pgfqpoint{2.909135in}{1.378781in}}%
\pgfpathlineto{\pgfqpoint{2.917106in}{1.455401in}}%
\pgfpathlineto{\pgfqpoint{2.927025in}{1.529206in}}%
\pgfpathlineto{\pgfqpoint{2.938778in}{1.600125in}}%
\pgfpathlineto{\pgfqpoint{2.952285in}{1.668084in}}%
\pgfpathlineto{\pgfqpoint{2.967495in}{1.732995in}}%
\pgfpathlineto{\pgfqpoint{2.983690in}{1.792397in}}%
\pgfpathlineto{\pgfqpoint{3.000591in}{1.846279in}}%
\pgfpathlineto{\pgfqpoint{3.018777in}{1.896928in}}%
\pgfpathlineto{\pgfqpoint{3.037234in}{1.942004in}}%
\pgfpathlineto{\pgfqpoint{3.056656in}{1.983746in}}%
\pgfpathlineto{\pgfqpoint{3.076909in}{2.022085in}}%
\pgfpathlineto{\pgfqpoint{3.097806in}{2.056991in}}%
\pgfpathlineto{\pgfqpoint{3.119108in}{2.088480in}}%
\pgfpathlineto{\pgfqpoint{3.142023in}{2.118449in}}%
\pgfpathlineto{\pgfqpoint{3.166537in}{2.146713in}}%
\pgfpathlineto{\pgfqpoint{3.190927in}{2.171508in}}%
\pgfpathlineto{\pgfqpoint{3.216610in}{2.194529in}}%
\pgfpathlineto{\pgfqpoint{3.243498in}{2.215671in}}%
\pgfpathlineto{\pgfqpoint{3.271492in}{2.234844in}}%
\pgfpathlineto{\pgfqpoint{3.300483in}{2.251964in}}%
\pgfpathlineto{\pgfqpoint{3.330366in}{2.266944in}}%
\pgfpathlineto{\pgfqpoint{3.358969in}{2.278893in}}%
\pgfpathlineto{\pgfqpoint{3.388178in}{2.288736in}}%
\pgfpathlineto{\pgfqpoint{3.415773in}{2.295783in}}%
\pgfpathlineto{\pgfqpoint{3.441584in}{2.300201in}}%
\pgfpathlineto{\pgfqpoint{3.465451in}{2.302034in}}%
\pgfpathlineto{\pgfqpoint{3.485010in}{2.301361in}}%
\pgfpathlineto{\pgfqpoint{3.500056in}{2.298764in}}%
\pgfpathlineto{\pgfqpoint{3.510434in}{2.295088in}}%
\pgfpathlineto{\pgfqpoint{3.518030in}{2.290285in}}%
\pgfpathlineto{\pgfqpoint{3.522601in}{2.285001in}}%
\pgfpathlineto{\pgfqpoint{3.525082in}{2.278160in}}%
\pgfpathlineto{\pgfqpoint{3.524895in}{2.270740in}}%
\pgfpathlineto{\pgfqpoint{3.521760in}{2.261490in}}%
\pgfpathlineto{\pgfqpoint{3.515629in}{2.251230in}}%
\pgfpathlineto{\pgfqpoint{3.505306in}{2.238434in}}%
\pgfpathlineto{\pgfqpoint{3.487435in}{2.220237in}}%
\pgfpathlineto{\pgfqpoint{3.454979in}{2.190959in}}%
\pgfpathlineto{\pgfqpoint{3.381275in}{2.125064in}}%
\pgfpathlineto{\pgfqpoint{3.346676in}{2.090939in}}%
\pgfpathlineto{\pgfqpoint{3.316697in}{2.058388in}}%
\pgfpathlineto{\pgfqpoint{3.289722in}{2.025917in}}%
\pgfpathlineto{\pgfqpoint{3.264337in}{1.991811in}}%
\pgfpathlineto{\pgfqpoint{3.241959in}{1.958128in}}%
\pgfpathlineto{\pgfqpoint{3.220047in}{1.921002in}}%
\pgfpathlineto{\pgfqpoint{3.200012in}{1.882513in}}%
\pgfpathlineto{\pgfqpoint{3.181856in}{1.842822in}}%
\pgfpathlineto{\pgfqpoint{3.164685in}{1.799808in}}%
\pgfpathlineto{\pgfqpoint{3.149480in}{1.755835in}}%
\pgfpathlineto{\pgfqpoint{3.135488in}{1.708701in}}%
\pgfpathlineto{\pgfqpoint{3.122890in}{1.658460in}}%
\pgfpathlineto{\pgfqpoint{3.111842in}{1.605181in}}%
\pgfpathlineto{\pgfqpoint{3.102506in}{1.548940in}}%
\pgfpathlineto{\pgfqpoint{3.095325in}{1.492285in}}%
\pgfpathlineto{\pgfqpoint{3.090066in}{1.432853in}}%
\pgfpathlineto{\pgfqpoint{3.087085in}{1.373215in}}%
\pgfpathlineto{\pgfqpoint{3.086423in}{1.315972in}}%
\pgfpathlineto{\pgfqpoint{3.088025in}{1.258755in}}%
\pgfpathlineto{\pgfqpoint{3.091829in}{1.204172in}}%
\pgfpathlineto{\pgfqpoint{3.097735in}{1.152343in}}%
\pgfpathlineto{\pgfqpoint{3.105184in}{1.105828in}}%
\pgfpathlineto{\pgfqpoint{3.114277in}{1.062253in}}%
\pgfpathlineto{\pgfqpoint{3.124881in}{1.021720in}}%
\pgfpathlineto{\pgfqpoint{3.137541in}{0.981968in}}%
\pgfpathlineto{\pgfqpoint{3.151592in}{0.945537in}}%
\pgfpathlineto{\pgfqpoint{3.165725in}{0.914678in}}%
\pgfpathlineto{\pgfqpoint{3.181712in}{0.885025in}}%
\pgfpathlineto{\pgfqpoint{3.199513in}{0.856758in}}%
\pgfpathlineto{\pgfqpoint{3.217643in}{0.831932in}}%
\pgfpathlineto{\pgfqpoint{3.237237in}{0.808609in}}%
\pgfpathlineto{\pgfqpoint{3.258222in}{0.786926in}}%
\pgfpathlineto{\pgfqpoint{3.282126in}{0.765344in}}%
\pgfpathlineto{\pgfqpoint{3.307230in}{0.745626in}}%
\pgfpathlineto{\pgfqpoint{3.333377in}{0.727769in}}%
\pgfpathlineto{\pgfqpoint{3.362290in}{0.710474in}}%
\pgfpathlineto{\pgfqpoint{3.399754in}{0.690550in}}%
\pgfpathlineto{\pgfqpoint{3.434256in}{0.675337in}}%
\pgfpathlineto{\pgfqpoint{3.473442in}{0.660313in}}%
\pgfpathlineto{\pgfqpoint{3.515272in}{0.646640in}}%
\pgfpathlineto{\pgfqpoint{3.561797in}{0.633823in}}%
\pgfpathlineto{\pgfqpoint{3.610857in}{0.622609in}}%
\pgfpathlineto{\pgfqpoint{3.703389in}{0.607031in}}%
\pgfpathlineto{\pgfqpoint{3.776992in}{0.598751in}}%
\pgfpathlineto{\pgfqpoint{3.844273in}{0.593576in}}%
\pgfpathlineto{\pgfqpoint{3.913827in}{0.590435in}}%
\pgfpathlineto{\pgfqpoint{3.987779in}{0.589241in}}%
\pgfpathlineto{\pgfqpoint{4.059556in}{0.590330in}}%
\pgfpathlineto{\pgfqpoint{4.131278in}{0.593699in}}%
\pgfpathlineto{\pgfqpoint{4.194191in}{0.598984in}}%
\pgfpathlineto{\pgfqpoint{4.256916in}{0.606603in}}%
\pgfpathlineto{\pgfqpoint{4.308549in}{0.615415in}}%
\pgfpathlineto{\pgfqpoint{4.359832in}{0.626540in}}%
\pgfpathlineto{\pgfqpoint{4.397819in}{0.637376in}}%
\pgfpathlineto{\pgfqpoint{4.435320in}{0.650250in}}%
\pgfpathlineto{\pgfqpoint{4.461908in}{0.661261in}}%
\pgfpathlineto{\pgfqpoint{4.491900in}{0.675945in}}%
\pgfpathlineto{\pgfqpoint{4.517061in}{0.690704in}}%
\pgfpathlineto{\pgfqpoint{4.541232in}{0.707487in}}%
\pgfpathlineto{\pgfqpoint{4.562441in}{0.724889in}}%
\pgfpathlineto{\pgfqpoint{4.582363in}{0.744176in}}%
\pgfpathlineto{\pgfqpoint{4.600812in}{0.765297in}}%
\pgfpathlineto{\pgfqpoint{4.617647in}{0.788114in}}%
\pgfpathlineto{\pgfqpoint{4.632793in}{0.812431in}}%
\pgfpathlineto{\pgfqpoint{4.647375in}{0.840148in}}%
\pgfpathlineto{\pgfqpoint{4.661112in}{0.871242in}}%
\pgfpathlineto{\pgfqpoint{4.673813in}{0.905628in}}%
\pgfpathlineto{\pgfqpoint{4.686062in}{0.945550in}}%
\pgfpathlineto{\pgfqpoint{4.697519in}{0.990985in}}%
\pgfpathlineto{\pgfqpoint{4.708006in}{1.041858in}}%
\pgfpathlineto{\pgfqpoint{4.717737in}{1.100547in}}%
\pgfpathlineto{\pgfqpoint{4.726437in}{1.167010in}}%
\pgfpathlineto{\pgfqpoint{4.733977in}{1.241182in}}%
\pgfpathlineto{\pgfqpoint{4.746071in}{1.417360in}}%
\pgfpathlineto{\pgfqpoint{4.750861in}{1.529236in}}%
\pgfpathlineto{\pgfqpoint{4.755153in}{1.673525in}}%
\pgfpathlineto{\pgfqpoint{4.757894in}{1.840271in}}%
\pgfpathlineto{\pgfqpoint{4.758846in}{2.036914in}}%
\pgfpathlineto{\pgfqpoint{4.757561in}{2.218618in}}%
\pgfpathlineto{\pgfqpoint{4.754381in}{2.370413in}}%
\pgfpathlineto{\pgfqpoint{4.749872in}{2.482306in}}%
\pgfpathlineto{\pgfqpoint{4.744260in}{2.569187in}}%
\pgfpathlineto{\pgfqpoint{4.737677in}{2.635967in}}%
\pgfpathlineto{\pgfqpoint{4.730687in}{2.685098in}}%
\pgfpathlineto{\pgfqpoint{4.722983in}{2.723929in}}%
\pgfpathlineto{\pgfqpoint{4.715195in}{2.752430in}}%
\pgfpathlineto{\pgfqpoint{4.706800in}{2.775382in}}%
\pgfpathlineto{\pgfqpoint{4.697255in}{2.794925in}}%
\pgfpathlineto{\pgfqpoint{4.686870in}{2.810886in}}%
\pgfpathlineto{\pgfqpoint{4.676215in}{2.823316in}}%
\pgfpathlineto{\pgfqpoint{4.664230in}{2.834045in}}%
\pgfpathlineto{\pgfqpoint{4.651177in}{2.842999in}}%
\pgfpathlineto{\pgfqpoint{4.635341in}{2.851234in}}%
\pgfpathlineto{\pgfqpoint{4.616802in}{2.858409in}}%
\pgfpathlineto{\pgfqpoint{4.593566in}{2.864920in}}%
\pgfpathlineto{\pgfqpoint{4.563548in}{2.870766in}}%
\pgfpathlineto{\pgfqpoint{4.524644in}{2.875799in}}%
\pgfpathlineto{\pgfqpoint{4.474739in}{2.879854in}}%
\pgfpathlineto{\pgfqpoint{4.403021in}{2.883368in}}%
\pgfpathlineto{\pgfqpoint{4.292109in}{2.886305in}}%
\pgfpathlineto{\pgfqpoint{4.126794in}{2.888302in}}%
\pgfpathlineto{\pgfqpoint{3.765693in}{2.890094in}}%
\pgfpathlineto{\pgfqpoint{2.934717in}{2.890806in}}%
\pgfpathlineto{\pgfqpoint{1.694780in}{2.889876in}}%
\pgfpathlineto{\pgfqpoint{1.292348in}{2.887733in}}%
\pgfpathlineto{\pgfqpoint{1.087886in}{2.884666in}}%
\pgfpathlineto{\pgfqpoint{0.974815in}{2.881033in}}%
\pgfpathlineto{\pgfqpoint{0.898776in}{2.876678in}}%
\pgfpathlineto{\pgfqpoint{0.840239in}{2.871243in}}%
\pgfpathlineto{\pgfqpoint{0.794909in}{2.864807in}}%
\pgfpathlineto{\pgfqpoint{0.764913in}{2.858807in}}%
\pgfpathlineto{\pgfqpoint{0.737401in}{2.851349in}}%
\pgfpathlineto{\pgfqpoint{0.712531in}{2.842303in}}%
\pgfpathlineto{\pgfqpoint{0.692510in}{2.832592in}}%
\pgfpathlineto{\pgfqpoint{0.675353in}{2.821821in}}%
\pgfpathlineto{\pgfqpoint{0.659329in}{2.808972in}}%
\pgfpathlineto{\pgfqpoint{0.646293in}{2.795794in}}%
\pgfpathlineto{\pgfqpoint{0.633187in}{2.779169in}}%
\pgfpathlineto{\pgfqpoint{0.621791in}{2.760965in}}%
\pgfpathlineto{\pgfqpoint{0.611028in}{2.739346in}}%
\pgfpathlineto{\pgfqpoint{0.601198in}{2.714391in}}%
\pgfpathlineto{\pgfqpoint{0.592427in}{2.686265in}}%
\pgfpathlineto{\pgfqpoint{0.584261in}{2.652699in}}%
\pgfpathlineto{\pgfqpoint{0.576045in}{2.608897in}}%
\pgfpathlineto{\pgfqpoint{0.568846in}{2.557282in}}%
\pgfpathlineto{\pgfqpoint{0.562852in}{2.497940in}}%
\pgfpathlineto{\pgfqpoint{0.557742in}{2.423496in}}%
\pgfpathlineto{\pgfqpoint{0.554210in}{2.338963in}}%
\pgfpathlineto{\pgfqpoint{0.552416in}{2.241908in}}%
\pgfpathlineto{\pgfqpoint{0.552652in}{2.137364in}}%
\pgfpathlineto{\pgfqpoint{0.555048in}{2.032856in}}%
\pgfpathlineto{\pgfqpoint{0.559675in}{1.928447in}}%
\pgfpathlineto{\pgfqpoint{0.566208in}{1.831659in}}%
\pgfpathlineto{\pgfqpoint{0.574080in}{1.747510in}}%
\pgfpathlineto{\pgfqpoint{0.583430in}{1.671095in}}%
\pgfpathlineto{\pgfqpoint{0.593449in}{1.607403in}}%
\pgfpathlineto{\pgfqpoint{0.604436in}{1.551554in}}%
\pgfpathlineto{\pgfqpoint{0.616084in}{1.503593in}}%
\pgfpathlineto{\pgfqpoint{0.627961in}{1.463525in}}%
\pgfpathlineto{\pgfqpoint{0.640477in}{1.429052in}}%
\pgfpathlineto{\pgfqpoint{0.652251in}{1.402402in}}%
\pgfpathlineto{\pgfqpoint{0.664862in}{1.379145in}}%
\pgfpathlineto{\pgfqpoint{0.676814in}{1.361414in}}%
\pgfpathlineto{\pgfqpoint{0.687550in}{1.349075in}}%
\pgfpathlineto{\pgfqpoint{0.697984in}{1.340130in}}%
\pgfpathlineto{\pgfqpoint{0.707697in}{1.334575in}}%
\pgfpathlineto{\pgfqpoint{0.716103in}{1.332081in}}%
\pgfpathlineto{\pgfqpoint{0.724776in}{1.332051in}}%
\pgfpathlineto{\pgfqpoint{0.733089in}{1.334868in}}%
\pgfpathlineto{\pgfqpoint{0.740401in}{1.340215in}}%
\pgfpathlineto{\pgfqpoint{0.746494in}{1.347301in}}%
\pgfpathlineto{\pgfqpoint{0.752596in}{1.357587in}}%
\pgfpathlineto{\pgfqpoint{0.759026in}{1.373363in}}%
\pgfpathlineto{\pgfqpoint{0.765376in}{1.397160in}}%
\pgfpathlineto{\pgfqpoint{0.770544in}{1.426431in}}%
\pgfpathlineto{\pgfqpoint{0.774934in}{1.465933in}}%
\pgfpathlineto{\pgfqpoint{0.778489in}{1.520538in}}%
\pgfpathlineto{\pgfqpoint{0.783538in}{1.647353in}}%
\pgfpathlineto{\pgfqpoint{0.789369in}{1.766646in}}%
\pgfpathlineto{\pgfqpoint{0.796122in}{1.858418in}}%
\pgfpathlineto{\pgfqpoint{0.803915in}{1.935063in}}%
\pgfpathlineto{\pgfqpoint{0.813064in}{2.003966in}}%
\pgfpathlineto{\pgfqpoint{0.823714in}{2.067523in}}%
\pgfpathlineto{\pgfqpoint{0.835717in}{2.125659in}}%
\pgfpathlineto{\pgfqpoint{0.849414in}{2.180718in}}%
\pgfpathlineto{\pgfqpoint{0.862593in}{2.225541in}}%
\pgfpathlineto{\pgfqpoint{0.876074in}{2.264941in}}%
\pgfpathlineto{\pgfqpoint{0.892182in}{2.305774in}}%
\pgfpathlineto{\pgfqpoint{0.909244in}{2.343311in}}%
\pgfpathlineto{\pgfqpoint{0.919674in}{2.362230in}}%
\pgfpathlineto{\pgfqpoint{0.948393in}{2.414845in}}%
\pgfpathlineto{\pgfqpoint{0.969944in}{2.446110in}}%
\pgfpathlineto{\pgfqpoint{0.990343in}{2.471975in}}%
\pgfpathlineto{\pgfqpoint{1.013516in}{2.498254in}}%
\pgfpathlineto{\pgfqpoint{1.039625in}{2.524572in}}%
\pgfpathlineto{\pgfqpoint{1.067274in}{2.548752in}}%
\pgfpathlineto{\pgfqpoint{1.080022in}{2.558206in}}%
\pgfpathlineto{\pgfqpoint{1.087468in}{2.563324in}}%
\pgfpathlineto{\pgfqpoint{1.115232in}{2.582932in}}%
\pgfpathlineto{\pgfqpoint{1.145764in}{2.602041in}}%
\pgfpathlineto{\pgfqpoint{1.179083in}{2.620389in}}%
\pgfpathlineto{\pgfqpoint{1.219203in}{2.639632in}}%
\pgfpathlineto{\pgfqpoint{1.258032in}{2.655828in}}%
\pgfpathlineto{\pgfqpoint{1.303680in}{2.672256in}}%
\pgfpathlineto{\pgfqpoint{1.320549in}{2.676952in}}%
\pgfpathlineto{\pgfqpoint{1.331109in}{2.679878in}}%
\pgfpathlineto{\pgfqpoint{1.386022in}{2.695356in}}%
\pgfpathlineto{\pgfqpoint{1.441429in}{2.708326in}}%
\pgfpathlineto{\pgfqpoint{1.492916in}{2.718088in}}%
\pgfpathlineto{\pgfqpoint{1.557513in}{2.728569in}}%
\pgfpathlineto{\pgfqpoint{1.630928in}{2.738822in}}%
\pgfpathlineto{\pgfqpoint{1.711034in}{2.747746in}}%
\pgfpathlineto{\pgfqpoint{1.799972in}{2.755366in}}%
\pgfpathlineto{\pgfqpoint{1.880341in}{2.760324in}}%
\pgfpathlineto{\pgfqpoint{1.989027in}{2.765032in}}%
\pgfpathlineto{\pgfqpoint{2.091249in}{2.767235in}}%
\pgfpathlineto{\pgfqpoint{2.197838in}{2.767481in}}%
\pgfpathlineto{\pgfqpoint{2.302238in}{2.765521in}}%
\pgfpathlineto{\pgfqpoint{2.397887in}{2.761525in}}%
\pgfpathlineto{\pgfqpoint{2.482576in}{2.755807in}}%
\pgfpathlineto{\pgfqpoint{2.556269in}{2.748635in}}%
\pgfpathlineto{\pgfqpoint{2.618932in}{2.740334in}}%
\pgfpathlineto{\pgfqpoint{2.670543in}{2.731348in}}%
\pgfpathlineto{\pgfqpoint{2.713244in}{2.721837in}}%
\pgfpathlineto{\pgfqpoint{2.749158in}{2.711767in}}%
\pgfpathlineto{\pgfqpoint{2.780340in}{2.700790in}}%
\pgfpathlineto{\pgfqpoint{2.806704in}{2.689105in}}%
\pgfpathlineto{\pgfqpoint{2.828204in}{2.677107in}}%
\pgfpathlineto{\pgfqpoint{2.844901in}{2.665427in}}%
\pgfpathlineto{\pgfqpoint{2.858710in}{2.653324in}}%
\pgfpathlineto{\pgfqpoint{2.871113in}{2.639377in}}%
\pgfpathlineto{\pgfqpoint{2.880416in}{2.625599in}}%
\pgfpathlineto{\pgfqpoint{2.887952in}{2.610476in}}%
\pgfpathlineto{\pgfqpoint{2.893557in}{2.594289in}}%
\pgfpathlineto{\pgfqpoint{2.897662in}{2.574954in}}%
\pgfpathlineto{\pgfqpoint{2.899750in}{2.552693in}}%
\pgfpathlineto{\pgfqpoint{2.899665in}{2.527811in}}%
\pgfpathlineto{\pgfqpoint{2.897047in}{2.495597in}}%
\pgfpathlineto{\pgfqpoint{2.892030in}{2.461230in}}%
\pgfpathlineto{\pgfqpoint{2.881852in}{2.407724in}}%
\pgfpathlineto{\pgfqpoint{2.865882in}{2.332756in}}%
\pgfpathlineto{\pgfqpoint{2.837460in}{2.197211in}}%
\pgfpathlineto{\pgfqpoint{2.819970in}{2.102221in}}%
\pgfpathlineto{\pgfqpoint{2.804678in}{2.006735in}}%
\pgfpathlineto{\pgfqpoint{2.793105in}{1.935983in}}%
\pgfpathlineto{\pgfqpoint{2.788318in}{1.899052in}}%
\pgfpathlineto{\pgfqpoint{2.775642in}{1.787983in}}%
\pgfpathlineto{\pgfqpoint{2.764079in}{1.666734in}}%
\pgfpathlineto{\pgfqpoint{2.754354in}{1.540277in}}%
\pgfpathlineto{\pgfqpoint{2.744858in}{1.391322in}}%
\pgfpathlineto{\pgfqpoint{2.736569in}{1.224817in}}%
\pgfpathlineto{\pgfqpoint{2.729619in}{1.038298in}}%
\pgfpathlineto{\pgfqpoint{2.724372in}{0.836763in}}%
\pgfpathlineto{\pgfqpoint{2.714680in}{0.446179in}}%
\pgfpathlineto{\pgfqpoint{2.711841in}{0.429077in}}%
\pgfpathlineto{\pgfqpoint{2.708551in}{0.419888in}}%
\pgfpathlineto{\pgfqpoint{2.704422in}{0.414151in}}%
\pgfpathlineto{\pgfqpoint{2.698880in}{0.410266in}}%
\pgfpathlineto{\pgfqpoint{2.690547in}{0.407461in}}%
\pgfpathlineto{\pgfqpoint{2.675443in}{0.405326in}}%
\pgfpathlineto{\pgfqpoint{2.645017in}{0.403904in}}%
\pgfpathlineto{\pgfqpoint{2.564535in}{0.403052in}}%
\pgfpathlineto{\pgfqpoint{2.218657in}{0.402676in}}%
\pgfpathlineto{\pgfqpoint{0.458817in}{0.403264in}}%
\pgfpathlineto{\pgfqpoint{0.450132in}{0.403603in}}%
\pgfpathlineto{\pgfqpoint{0.450132in}{0.403603in}}%
\pgfusepath{stroke}%
\end{pgfscope}%
\begin{pgfscope}%
\pgfpathrectangle{\pgfqpoint{0.448634in}{0.402556in}}{\pgfqpoint{4.350661in}{2.489204in}} %
\pgfusepath{clip}%
\pgfsetrectcap%
\pgfsetroundjoin%
\pgfsetlinewidth{1.003750pt}%
\definecolor{currentstroke}{rgb}{0.737255,0.741176,0.133333}%
\pgfsetstrokecolor{currentstroke}%
\pgfsetdash{}{0pt}%
\pgfpathmoveto{\pgfqpoint{4.798836in}{2.852368in}}%
\pgfpathlineto{\pgfqpoint{4.797563in}{2.889609in}}%
\pgfpathlineto{\pgfqpoint{4.796215in}{2.891483in}}%
\pgfpathlineto{\pgfqpoint{4.787551in}{2.891760in}}%
\pgfpathlineto{\pgfqpoint{0.452128in}{2.891658in}}%
\pgfpathlineto{\pgfqpoint{0.450531in}{2.890080in}}%
\pgfpathlineto{\pgfqpoint{0.449454in}{2.882761in}}%
\pgfpathlineto{\pgfqpoint{0.448969in}{2.845430in}}%
\pgfpathlineto{\pgfqpoint{0.448743in}{2.491963in}}%
\pgfpathlineto{\pgfqpoint{0.449614in}{0.610126in}}%
\pgfpathlineto{\pgfqpoint{0.451477in}{0.505608in}}%
\pgfpathlineto{\pgfqpoint{0.451700in}{0.500636in}}%
\pgfpathlineto{\pgfqpoint{0.451700in}{0.500636in}}%
\pgfusepath{stroke}%
\end{pgfscope}%
\begin{pgfscope}%
\pgfpathrectangle{\pgfqpoint{0.448634in}{0.402556in}}{\pgfqpoint{4.350661in}{2.489204in}} %
\pgfusepath{clip}%
\pgfsetrectcap%
\pgfsetroundjoin%
\pgfsetlinewidth{1.003750pt}%
\definecolor{currentstroke}{rgb}{0.737255,0.741176,0.133333}%
\pgfsetstrokecolor{currentstroke}%
\pgfsetdash{}{0pt}%
\pgfpathmoveto{\pgfqpoint{0.445604in}{0.402803in}}%
\pgfpathlineto{\pgfqpoint{0.454230in}{0.403544in}}%
\pgfpathlineto{\pgfqpoint{0.475976in}{0.403032in}}%
\pgfpathlineto{\pgfqpoint{0.676106in}{0.402678in}}%
\pgfpathlineto{\pgfqpoint{2.572993in}{0.403112in}}%
\pgfpathlineto{\pgfqpoint{2.664342in}{0.404671in}}%
\pgfpathlineto{\pgfqpoint{2.686011in}{0.406738in}}%
\pgfpathlineto{\pgfqpoint{2.696595in}{0.409527in}}%
\pgfpathlineto{\pgfqpoint{2.702432in}{0.412817in}}%
\pgfpathlineto{\pgfqpoint{2.707100in}{0.417983in}}%
\pgfpathlineto{\pgfqpoint{2.710970in}{0.426857in}}%
\pgfpathlineto{\pgfqpoint{2.713897in}{0.441394in}}%
\pgfpathlineto{\pgfqpoint{2.716174in}{0.466142in}}%
\pgfpathlineto{\pgfqpoint{2.718266in}{0.518357in}}%
\pgfpathlineto{\pgfqpoint{2.720978in}{0.660207in}}%
\pgfpathlineto{\pgfqpoint{2.726860in}{0.941405in}}%
\pgfpathlineto{\pgfqpoint{2.733494in}{1.150358in}}%
\pgfpathlineto{\pgfqpoint{2.741353in}{1.326861in}}%
\pgfpathlineto{\pgfqpoint{2.750588in}{1.485817in}}%
\pgfpathlineto{\pgfqpoint{2.760961in}{1.627203in}}%
\pgfpathlineto{\pgfqpoint{2.773834in}{1.770814in}}%
\pgfpathlineto{\pgfqpoint{2.786777in}{1.886862in}}%
\pgfpathlineto{\pgfqpoint{2.793775in}{1.938475in}}%
\pgfpathlineto{\pgfqpoint{2.795905in}{1.945466in}}%
\pgfpathlineto{\pgfqpoint{2.810622in}{2.046121in}}%
\pgfpathlineto{\pgfqpoint{2.826806in}{2.141414in}}%
\pgfpathlineto{\pgfqpoint{2.845159in}{2.236192in}}%
\pgfpathlineto{\pgfqpoint{2.871559in}{2.359493in}}%
\pgfpathlineto{\pgfqpoint{2.889746in}{2.449205in}}%
\pgfpathlineto{\pgfqpoint{2.896575in}{2.493317in}}%
\pgfpathlineto{\pgfqpoint{2.899467in}{2.528000in}}%
\pgfpathlineto{\pgfqpoint{2.899519in}{2.552881in}}%
\pgfpathlineto{\pgfqpoint{2.897392in}{2.575137in}}%
\pgfpathlineto{\pgfqpoint{2.893247in}{2.594461in}}%
\pgfpathlineto{\pgfqpoint{2.887607in}{2.610632in}}%
\pgfpathlineto{\pgfqpoint{2.880038in}{2.625734in}}%
\pgfpathlineto{\pgfqpoint{2.870708in}{2.639488in}}%
\pgfpathlineto{\pgfqpoint{2.858282in}{2.653408in}}%
\pgfpathlineto{\pgfqpoint{2.844457in}{2.665487in}}%
\pgfpathlineto{\pgfqpoint{2.827748in}{2.677146in}}%
\pgfpathlineto{\pgfqpoint{2.808232in}{2.688124in}}%
\pgfpathlineto{\pgfqpoint{2.783976in}{2.699144in}}%
\pgfpathlineto{\pgfqpoint{2.754968in}{2.709739in}}%
\pgfpathlineto{\pgfqpoint{2.721259in}{2.719635in}}%
\pgfpathlineto{\pgfqpoint{2.680774in}{2.729137in}}%
\pgfpathlineto{\pgfqpoint{2.633536in}{2.737900in}}%
\pgfpathlineto{\pgfqpoint{2.577424in}{2.745990in}}%
\pgfpathlineto{\pgfqpoint{2.510293in}{2.753287in}}%
\pgfpathlineto{\pgfqpoint{2.434337in}{2.759257in}}%
\pgfpathlineto{\pgfqpoint{2.347418in}{2.763839in}}%
\pgfpathlineto{\pgfqpoint{2.249562in}{2.766715in}}%
\pgfpathlineto{\pgfqpoint{2.145150in}{2.767555in}}%
\pgfpathlineto{\pgfqpoint{2.038566in}{2.766228in}}%
\pgfpathlineto{\pgfqpoint{1.927672in}{2.762596in}}%
\pgfpathlineto{\pgfqpoint{1.827727in}{2.757038in}}%
\pgfpathlineto{\pgfqpoint{1.725730in}{2.749065in}}%
\pgfpathlineto{\pgfqpoint{1.641265in}{2.739996in}}%
\pgfpathlineto{\pgfqpoint{1.565663in}{2.729711in}}%
\pgfpathlineto{\pgfqpoint{1.470933in}{2.714259in}}%
\pgfpathlineto{\pgfqpoint{1.413210in}{2.701862in}}%
\pgfpathlineto{\pgfqpoint{1.360146in}{2.688257in}}%
\pgfpathlineto{\pgfqpoint{1.280324in}{2.663970in}}%
\pgfpathlineto{\pgfqpoint{1.228935in}{2.643629in}}%
\pgfpathlineto{\pgfqpoint{1.190623in}{2.625898in}}%
\pgfpathlineto{\pgfqpoint{1.155083in}{2.607106in}}%
\pgfpathlineto{\pgfqpoint{1.122396in}{2.587325in}}%
\pgfpathlineto{\pgfqpoint{1.090766in}{2.565411in}}%
\pgfpathlineto{\pgfqpoint{1.051245in}{2.534717in}}%
\pgfpathlineto{\pgfqpoint{1.024423in}{2.509348in}}%
\pgfpathlineto{\pgfqpoint{1.002292in}{2.485414in}}%
\pgfpathlineto{\pgfqpoint{0.978274in}{2.456597in}}%
\pgfpathlineto{\pgfqpoint{0.956114in}{2.425894in}}%
\pgfpathlineto{\pgfqpoint{0.937078in}{2.395592in}}%
\pgfpathlineto{\pgfqpoint{0.933531in}{2.386577in}}%
\pgfpathlineto{\pgfqpoint{0.909282in}{2.342317in}}%
\pgfpathlineto{\pgfqpoint{0.892275in}{2.304747in}}%
\pgfpathlineto{\pgfqpoint{0.876220in}{2.263887in}}%
\pgfpathlineto{\pgfqpoint{0.862105in}{2.222102in}}%
\pgfpathlineto{\pgfqpoint{0.845303in}{2.162928in}}%
\pgfpathlineto{\pgfqpoint{0.832819in}{2.110066in}}%
\pgfpathlineto{\pgfqpoint{0.821424in}{2.051770in}}%
\pgfpathlineto{\pgfqpoint{0.811329in}{1.988094in}}%
\pgfpathlineto{\pgfqpoint{0.802973in}{1.921573in}}%
\pgfpathlineto{\pgfqpoint{0.795595in}{1.844874in}}%
\pgfpathlineto{\pgfqpoint{0.789549in}{1.758030in}}%
\pgfpathlineto{\pgfqpoint{0.784641in}{1.653636in}}%
\pgfpathlineto{\pgfqpoint{0.776094in}{1.459753in}}%
\pgfpathlineto{\pgfqpoint{0.771437in}{1.417782in}}%
\pgfpathlineto{\pgfqpoint{0.766133in}{1.388545in}}%
\pgfpathlineto{\pgfqpoint{0.759715in}{1.364782in}}%
\pgfpathlineto{\pgfqpoint{0.753003in}{1.349163in}}%
\pgfpathlineto{\pgfqpoint{0.746619in}{1.339107in}}%
\pgfpathlineto{\pgfqpoint{0.740252in}{1.332345in}}%
\pgfpathlineto{\pgfqpoint{0.732684in}{1.327493in}}%
\pgfpathlineto{\pgfqpoint{0.724240in}{1.325238in}}%
\pgfpathlineto{\pgfqpoint{0.715576in}{1.325761in}}%
\pgfpathlineto{\pgfqpoint{0.707248in}{1.328582in}}%
\pgfpathlineto{\pgfqpoint{0.697638in}{1.334375in}}%
\pgfpathlineto{\pgfqpoint{0.687290in}{1.343453in}}%
\pgfpathlineto{\pgfqpoint{0.676605in}{1.355852in}}%
\pgfpathlineto{\pgfqpoint{0.666031in}{1.371651in}}%
\pgfpathlineto{\pgfqpoint{0.654295in}{1.392599in}}%
\pgfpathlineto{\pgfqpoint{0.642194in}{1.419058in}}%
\pgfpathlineto{\pgfqpoint{0.630141in}{1.451054in}}%
\pgfpathlineto{\pgfqpoint{0.618499in}{1.488580in}}%
\pgfpathlineto{\pgfqpoint{0.606358in}{1.536380in}}%
\pgfpathlineto{\pgfqpoint{0.595809in}{1.587236in}}%
\pgfpathlineto{\pgfqpoint{0.585931in}{1.645894in}}%
\pgfpathlineto{\pgfqpoint{0.577349in}{1.709859in}}%
\pgfpathlineto{\pgfqpoint{0.569193in}{1.786455in}}%
\pgfpathlineto{\pgfqpoint{0.562390in}{1.870726in}}%
\pgfpathlineto{\pgfqpoint{0.556971in}{1.965110in}}%
\pgfpathlineto{\pgfqpoint{0.553370in}{2.062099in}}%
\pgfpathlineto{\pgfqpoint{0.551524in}{2.171601in}}%
\pgfpathlineto{\pgfqpoint{0.552086in}{2.278632in}}%
\pgfpathlineto{\pgfqpoint{0.554602in}{2.375664in}}%
\pgfpathlineto{\pgfqpoint{0.559050in}{2.460140in}}%
\pgfpathlineto{\pgfqpoint{0.565052in}{2.531995in}}%
\pgfpathlineto{\pgfqpoint{0.571905in}{2.588702in}}%
\pgfpathlineto{\pgfqpoint{0.580041in}{2.637601in}}%
\pgfpathlineto{\pgfqpoint{0.589164in}{2.678602in}}%
\pgfpathlineto{\pgfqpoint{0.598957in}{2.711590in}}%
\pgfpathlineto{\pgfqpoint{0.609488in}{2.738909in}}%
\pgfpathlineto{\pgfqpoint{0.620157in}{2.760591in}}%
\pgfpathlineto{\pgfqpoint{0.631469in}{2.778862in}}%
\pgfpathlineto{\pgfqpoint{0.644509in}{2.795554in}}%
\pgfpathlineto{\pgfqpoint{0.657498in}{2.808793in}}%
\pgfpathlineto{\pgfqpoint{0.673475in}{2.821718in}}%
\pgfpathlineto{\pgfqpoint{0.690598in}{2.832561in}}%
\pgfpathlineto{\pgfqpoint{0.708569in}{2.841430in}}%
\pgfpathlineto{\pgfqpoint{0.729199in}{2.849304in}}%
\pgfpathlineto{\pgfqpoint{0.756598in}{2.857285in}}%
\pgfpathlineto{\pgfqpoint{0.786521in}{2.863742in}}%
\pgfpathlineto{\pgfqpoint{0.825322in}{2.869736in}}%
\pgfpathlineto{\pgfqpoint{0.885988in}{2.875902in}}%
\pgfpathlineto{\pgfqpoint{0.953316in}{2.880217in}}%
\pgfpathlineto{\pgfqpoint{1.042455in}{2.883573in}}%
\pgfpathlineto{\pgfqpoint{1.179476in}{2.886467in}}%
\pgfpathlineto{\pgfqpoint{1.390475in}{2.888541in}}%
\pgfpathlineto{\pgfqpoint{1.773331in}{2.890080in}}%
\pgfpathlineto{\pgfqpoint{2.667391in}{2.890819in}}%
\pgfpathlineto{\pgfqpoint{3.852945in}{2.889859in}}%
\pgfpathlineto{\pgfqpoint{4.231445in}{2.887281in}}%
\pgfpathlineto{\pgfqpoint{4.381517in}{2.884202in}}%
\pgfpathlineto{\pgfqpoint{4.468465in}{2.880396in}}%
\pgfpathlineto{\pgfqpoint{4.527055in}{2.875752in}}%
\pgfpathlineto{\pgfqpoint{4.568105in}{2.870289in}}%
\pgfpathlineto{\pgfqpoint{4.598080in}{2.864169in}}%
\pgfpathlineto{\pgfqpoint{4.621240in}{2.857316in}}%
\pgfpathlineto{\pgfqpoint{4.639658in}{2.849747in}}%
\pgfpathlineto{\pgfqpoint{4.655304in}{2.841053in}}%
\pgfpathlineto{\pgfqpoint{4.668106in}{2.831638in}}%
\pgfpathlineto{\pgfqpoint{4.679756in}{2.820438in}}%
\pgfpathlineto{\pgfqpoint{4.690016in}{2.807579in}}%
\pgfpathlineto{\pgfqpoint{4.699948in}{2.791244in}}%
\pgfpathlineto{\pgfqpoint{4.709040in}{2.771418in}}%
\pgfpathlineto{\pgfqpoint{4.717034in}{2.748278in}}%
\pgfpathlineto{\pgfqpoint{4.724468in}{2.719652in}}%
\pgfpathlineto{\pgfqpoint{4.732255in}{2.678293in}}%
\pgfpathlineto{\pgfqpoint{4.738601in}{2.631563in}}%
\pgfpathlineto{\pgfqpoint{4.744580in}{2.569714in}}%
\pgfpathlineto{\pgfqpoint{4.749496in}{2.495254in}}%
\pgfpathlineto{\pgfqpoint{4.753843in}{2.395812in}}%
\pgfpathlineto{\pgfqpoint{4.757039in}{2.268917in}}%
\pgfpathlineto{\pgfqpoint{4.758868in}{2.107134in}}%
\pgfpathlineto{\pgfqpoint{4.758782in}{1.917956in}}%
\pgfpathlineto{\pgfqpoint{4.756585in}{1.733773in}}%
\pgfpathlineto{\pgfqpoint{4.752288in}{1.562091in}}%
\pgfpathlineto{\pgfqpoint{4.746248in}{1.415396in}}%
\pgfpathlineto{\pgfqpoint{4.739256in}{1.303671in}}%
\pgfpathlineto{\pgfqpoint{4.730547in}{1.202108in}}%
\pgfpathlineto{\pgfqpoint{4.725729in}{1.157645in}}%
\pgfpathlineto{\pgfqpoint{4.716737in}{1.091232in}}%
\pgfpathlineto{\pgfqpoint{4.706704in}{1.032609in}}%
\pgfpathlineto{\pgfqpoint{4.695890in}{0.981827in}}%
\pgfpathlineto{\pgfqpoint{4.684057in}{0.936518in}}%
\pgfpathlineto{\pgfqpoint{4.671393in}{0.896768in}}%
\pgfpathlineto{\pgfqpoint{4.658249in}{0.862601in}}%
\pgfpathlineto{\pgfqpoint{4.644031in}{0.831792in}}%
\pgfpathlineto{\pgfqpoint{4.628962in}{0.804419in}}%
\pgfpathlineto{\pgfqpoint{4.613311in}{0.780523in}}%
\pgfpathlineto{\pgfqpoint{4.595996in}{0.758181in}}%
\pgfpathlineto{\pgfqpoint{4.577097in}{0.737589in}}%
\pgfpathlineto{\pgfqpoint{4.556775in}{0.718855in}}%
\pgfpathlineto{\pgfqpoint{4.535229in}{0.702003in}}%
\pgfpathlineto{\pgfqpoint{4.510761in}{0.685794in}}%
\pgfpathlineto{\pgfqpoint{4.485367in}{0.671566in}}%
\pgfpathlineto{\pgfqpoint{4.457218in}{0.658281in}}%
\pgfpathlineto{\pgfqpoint{4.426379in}{0.646100in}}%
\pgfpathlineto{\pgfqpoint{4.390858in}{0.634340in}}%
\pgfpathlineto{\pgfqpoint{4.361268in}{0.626109in}}%
\pgfpathlineto{\pgfqpoint{4.322855in}{0.617468in}}%
\pgfpathlineto{\pgfqpoint{4.275616in}{0.608711in}}%
\pgfpathlineto{\pgfqpoint{4.223809in}{0.601346in}}%
\pgfpathlineto{\pgfqpoint{4.169641in}{0.595844in}}%
\pgfpathlineto{\pgfqpoint{4.100157in}{0.591105in}}%
\pgfpathlineto{\pgfqpoint{3.980546in}{0.588453in}}%
\pgfpathlineto{\pgfqpoint{3.908771in}{0.589781in}}%
\pgfpathlineto{\pgfqpoint{3.834879in}{0.593352in}}%
\pgfpathlineto{\pgfqpoint{3.765446in}{0.599007in}}%
\pgfpathlineto{\pgfqpoint{3.702683in}{0.606248in}}%
\pgfpathlineto{\pgfqpoint{3.644510in}{0.615497in}}%
\pgfpathlineto{\pgfqpoint{3.586640in}{0.626925in}}%
\pgfpathlineto{\pgfqpoint{3.537760in}{0.639129in}}%
\pgfpathlineto{\pgfqpoint{3.493567in}{0.652349in}}%
\pgfpathlineto{\pgfqpoint{3.451996in}{0.667019in}}%
\pgfpathlineto{\pgfqpoint{3.411075in}{0.683907in}}%
\pgfpathlineto{\pgfqpoint{3.377068in}{0.700513in}}%
\pgfpathlineto{\pgfqpoint{3.355665in}{0.712738in}}%
\pgfpathlineto{\pgfqpoint{3.325016in}{0.731593in}}%
\pgfpathlineto{\pgfqpoint{3.295495in}{0.752670in}}%
\pgfpathlineto{\pgfqpoint{3.270852in}{0.773134in}}%
\pgfpathlineto{\pgfqpoint{3.247486in}{0.795476in}}%
\pgfpathlineto{\pgfqpoint{3.227108in}{0.817901in}}%
\pgfpathlineto{\pgfqpoint{3.208151in}{0.841906in}}%
\pgfpathlineto{\pgfqpoint{3.190698in}{0.867359in}}%
\pgfpathlineto{\pgfqpoint{3.173641in}{0.896220in}}%
\pgfpathlineto{\pgfqpoint{3.158389in}{0.926376in}}%
\pgfpathlineto{\pgfqpoint{3.144989in}{0.957663in}}%
\pgfpathlineto{\pgfqpoint{3.130964in}{0.996807in}}%
\pgfpathlineto{\pgfqpoint{3.116016in}{1.048816in}}%
\pgfpathlineto{\pgfqpoint{3.106283in}{1.092209in}}%
\pgfpathlineto{\pgfqpoint{3.098211in}{1.138588in}}%
\pgfpathlineto{\pgfqpoint{3.091928in}{1.187845in}}%
\pgfpathlineto{\pgfqpoint{3.087556in}{1.239873in}}%
\pgfpathlineto{\pgfqpoint{3.085195in}{1.294564in}}%
\pgfpathlineto{\pgfqpoint{3.084970in}{1.351811in}}%
\pgfpathlineto{\pgfqpoint{3.086919in}{1.409014in}}%
\pgfpathlineto{\pgfqpoint{3.091165in}{1.468553in}}%
\pgfpathlineto{\pgfqpoint{3.097662in}{1.527824in}}%
\pgfpathlineto{\pgfqpoint{3.106443in}{1.586709in}}%
\pgfpathlineto{\pgfqpoint{3.117112in}{1.642638in}}%
\pgfpathlineto{\pgfqpoint{3.129500in}{1.695529in}}%
\pgfpathlineto{\pgfqpoint{3.143469in}{1.745293in}}%
\pgfpathlineto{\pgfqpoint{3.158831in}{1.791864in}}%
\pgfpathlineto{\pgfqpoint{3.175403in}{1.835185in}}%
\pgfpathlineto{\pgfqpoint{3.194004in}{1.877413in}}%
\pgfpathlineto{\pgfqpoint{3.213551in}{1.916230in}}%
\pgfpathlineto{\pgfqpoint{3.234934in}{1.953758in}}%
\pgfpathlineto{\pgfqpoint{3.258142in}{1.989837in}}%
\pgfpathlineto{\pgfqpoint{3.260829in}{1.993753in}}%
\pgfpathlineto{\pgfqpoint{3.260829in}{1.993753in}}%
\pgfusepath{stroke}%
\end{pgfscope}%
\begin{pgfscope}%
\pgfpathrectangle{\pgfqpoint{0.448634in}{0.402556in}}{\pgfqpoint{4.350661in}{2.489204in}} %
\pgfusepath{clip}%
\pgfsetrectcap%
\pgfsetroundjoin%
\pgfsetlinewidth{1.003750pt}%
\definecolor{currentstroke}{rgb}{0.737255,0.741176,0.133333}%
\pgfsetstrokecolor{currentstroke}%
\pgfsetdash{}{0pt}%
\pgfpathmoveto{\pgfqpoint{2.763303in}{2.149319in}}%
\pgfpathlineto{\pgfqpoint{2.735963in}{1.827242in}}%
\pgfpathlineto{\pgfqpoint{2.722658in}{1.636181in}}%
\pgfpathlineto{\pgfqpoint{2.709151in}{1.405202in}}%
\pgfpathlineto{\pgfqpoint{2.688092in}{1.030110in}}%
\pgfpathlineto{\pgfqpoint{2.679352in}{0.918546in}}%
\pgfpathlineto{\pgfqpoint{2.670349in}{0.832039in}}%
\pgfpathlineto{\pgfqpoint{2.660483in}{0.758225in}}%
\pgfpathlineto{\pgfqpoint{2.650453in}{0.699603in}}%
\pgfpathlineto{\pgfqpoint{2.640893in}{0.656160in}}%
\pgfpathlineto{\pgfqpoint{2.630829in}{0.620651in}}%
\pgfpathlineto{\pgfqpoint{2.620840in}{0.593062in}}%
\pgfpathlineto{\pgfqpoint{2.609738in}{0.568820in}}%
\pgfpathlineto{\pgfqpoint{2.598936in}{0.550148in}}%
\pgfpathlineto{\pgfqpoint{2.586449in}{0.532908in}}%
\pgfpathlineto{\pgfqpoint{2.572305in}{0.517437in}}%
\pgfpathlineto{\pgfqpoint{2.556698in}{0.503931in}}%
\pgfpathlineto{\pgfqpoint{2.539924in}{0.492396in}}%
\pgfpathlineto{\pgfqpoint{2.520294in}{0.481689in}}%
\pgfpathlineto{\pgfqpoint{2.497894in}{0.472082in}}%
\pgfpathlineto{\pgfqpoint{2.470747in}{0.463040in}}%
\pgfpathlineto{\pgfqpoint{2.438901in}{0.454928in}}%
\pgfpathlineto{\pgfqpoint{2.400288in}{0.447521in}}%
\pgfpathlineto{\pgfqpoint{2.352795in}{0.440802in}}%
\pgfpathlineto{\pgfqpoint{2.294295in}{0.434836in}}%
\pgfpathlineto{\pgfqpoint{2.218307in}{0.429424in}}%
\pgfpathlineto{\pgfqpoint{2.118330in}{0.424675in}}%
\pgfpathlineto{\pgfqpoint{1.983506in}{0.420654in}}%
\pgfpathlineto{\pgfqpoint{1.798624in}{0.417503in}}%
\pgfpathlineto{\pgfqpoint{1.544117in}{0.415510in}}%
\pgfpathlineto{\pgfqpoint{1.211292in}{0.415228in}}%
\pgfpathlineto{\pgfqpoint{0.967661in}{0.417104in}}%
\pgfpathlineto{\pgfqpoint{0.787137in}{0.420610in}}%
\pgfpathlineto{\pgfqpoint{0.687136in}{0.424673in}}%
\pgfpathlineto{\pgfqpoint{0.624181in}{0.429235in}}%
\pgfpathlineto{\pgfqpoint{0.580915in}{0.434424in}}%
\pgfpathlineto{\pgfqpoint{0.550875in}{0.440111in}}%
\pgfpathlineto{\pgfqpoint{0.529775in}{0.446131in}}%
\pgfpathlineto{\pgfqpoint{0.513434in}{0.452943in}}%
\pgfpathlineto{\pgfqpoint{0.500003in}{0.461115in}}%
\pgfpathlineto{\pgfqpoint{0.489697in}{0.470248in}}%
\pgfpathlineto{\pgfqpoint{0.482370in}{0.479427in}}%
\pgfpathlineto{\pgfqpoint{0.475334in}{0.491984in}}%
\pgfpathlineto{\pgfqpoint{0.469334in}{0.507981in}}%
\pgfpathlineto{\pgfqpoint{0.464620in}{0.527140in}}%
\pgfpathlineto{\pgfqpoint{0.460442in}{0.554092in}}%
\pgfpathlineto{\pgfqpoint{0.456885in}{0.593704in}}%
\pgfpathlineto{\pgfqpoint{0.454085in}{0.653356in}}%
\pgfpathlineto{\pgfqpoint{0.451927in}{0.752891in}}%
\pgfpathlineto{\pgfqpoint{0.450409in}{0.942062in}}%
\pgfpathlineto{\pgfqpoint{0.449469in}{1.405052in}}%
\pgfpathlineto{\pgfqpoint{0.449544in}{2.579957in}}%
\pgfpathlineto{\pgfqpoint{0.450955in}{2.838825in}}%
\pgfpathlineto{\pgfqpoint{0.452801in}{2.873593in}}%
\pgfpathlineto{\pgfqpoint{0.454976in}{2.883198in}}%
\pgfpathlineto{\pgfqpoint{0.457631in}{2.887092in}}%
\pgfpathlineto{\pgfqpoint{0.461522in}{2.889247in}}%
\pgfpathlineto{\pgfqpoint{0.470113in}{2.890712in}}%
\pgfpathlineto{\pgfqpoint{0.494029in}{2.891493in}}%
\pgfpathlineto{\pgfqpoint{0.624548in}{2.891742in}}%
\pgfpathlineto{\pgfqpoint{4.779428in}{2.891527in}}%
\pgfpathlineto{\pgfqpoint{4.792415in}{2.890342in}}%
\pgfpathlineto{\pgfqpoint{4.795944in}{2.887712in}}%
\pgfpathlineto{\pgfqpoint{4.797235in}{2.880425in}}%
\pgfpathlineto{\pgfqpoint{4.798039in}{2.858045in}}%
\pgfpathlineto{\pgfqpoint{4.798039in}{2.858045in}}%
\pgfusepath{stroke}%
\end{pgfscope}%
\begin{pgfscope}%
\pgfpathrectangle{\pgfqpoint{0.448634in}{0.402556in}}{\pgfqpoint{4.350661in}{2.489204in}} %
\pgfusepath{clip}%
\pgfsetrectcap%
\pgfsetroundjoin%
\pgfsetlinewidth{1.003750pt}%
\definecolor{currentstroke}{rgb}{0.737255,0.741176,0.133333}%
\pgfsetstrokecolor{currentstroke}%
\pgfsetdash{}{0pt}%
\pgfpathmoveto{\pgfqpoint{2.791360in}{1.953249in}}%
\pgfpathlineto{\pgfqpoint{2.778019in}{1.842282in}}%
\pgfpathlineto{\pgfqpoint{2.765771in}{1.721122in}}%
\pgfpathlineto{\pgfqpoint{2.754591in}{1.587317in}}%
\pgfpathlineto{\pgfqpoint{2.744806in}{1.443379in}}%
\pgfpathlineto{\pgfqpoint{2.736049in}{1.281893in}}%
\pgfpathlineto{\pgfqpoint{2.728213in}{1.095420in}}%
\pgfpathlineto{\pgfqpoint{2.720967in}{0.866564in}}%
\pgfpathlineto{\pgfqpoint{2.711437in}{0.563085in}}%
\pgfpathlineto{\pgfqpoint{2.707400in}{0.506026in}}%
\pgfpathlineto{\pgfqpoint{2.703154in}{0.474044in}}%
\pgfpathlineto{\pgfqpoint{2.698672in}{0.454817in}}%
\pgfpathlineto{\pgfqpoint{2.693350in}{0.441204in}}%
\pgfpathlineto{\pgfqpoint{2.686870in}{0.431246in}}%
\pgfpathlineto{\pgfqpoint{2.680217in}{0.424857in}}%
\pgfpathlineto{\pgfqpoint{2.670601in}{0.419092in}}%
\pgfpathlineto{\pgfqpoint{2.658147in}{0.414669in}}%
\pgfpathlineto{\pgfqpoint{2.641019in}{0.411192in}}%
\pgfpathlineto{\pgfqpoint{2.615035in}{0.408391in}}%
\pgfpathlineto{\pgfqpoint{2.571573in}{0.406188in}}%
\pgfpathlineto{\pgfqpoint{2.488924in}{0.404557in}}%
\pgfpathlineto{\pgfqpoint{2.299673in}{0.403553in}}%
\pgfpathlineto{\pgfqpoint{1.620970in}{0.402974in}}%
\pgfpathlineto{\pgfqpoint{0.557235in}{0.403891in}}%
\pgfpathlineto{\pgfqpoint{0.476767in}{0.405666in}}%
\pgfpathlineto{\pgfqpoint{0.459436in}{0.407301in}}%
\pgfpathlineto{\pgfqpoint{0.453333in}{0.409788in}}%
\pgfpathlineto{\pgfqpoint{0.450748in}{0.413678in}}%
\pgfpathlineto{\pgfqpoint{0.449350in}{0.423470in}}%
\pgfpathlineto{\pgfqpoint{0.448804in}{0.458311in}}%
\pgfpathlineto{\pgfqpoint{0.448644in}{0.754526in}}%
\pgfpathlineto{\pgfqpoint{0.448676in}{2.890263in}}%
\pgfpathlineto{\pgfqpoint{0.448676in}{2.890263in}}%
\pgfusepath{stroke}%
\end{pgfscope}%
\begin{pgfscope}%
\pgfpathrectangle{\pgfqpoint{0.448634in}{0.402556in}}{\pgfqpoint{4.350661in}{2.489204in}} %
\pgfusepath{clip}%
\pgfsetrectcap%
\pgfsetroundjoin%
\pgfsetlinewidth{1.003750pt}%
\definecolor{currentstroke}{rgb}{0.737255,0.741176,0.133333}%
\pgfsetstrokecolor{currentstroke}%
\pgfsetdash{}{0pt}%
\pgfpathmoveto{\pgfqpoint{3.431449in}{0.402556in}}%
\pgfpathlineto{\pgfqpoint{0.449071in}{0.402556in}}%
\pgfpathlineto{\pgfqpoint{0.449071in}{0.402556in}}%
\pgfusepath{stroke}%
\end{pgfscope}%
\begin{pgfscope}%
\pgfpathrectangle{\pgfqpoint{0.448634in}{0.402556in}}{\pgfqpoint{4.350661in}{2.489204in}} %
\pgfusepath{clip}%
\pgfsetrectcap%
\pgfsetroundjoin%
\pgfsetlinewidth{1.003750pt}%
\definecolor{currentstroke}{rgb}{0.737255,0.741176,0.133333}%
\pgfsetstrokecolor{currentstroke}%
\pgfsetdash{}{0pt}%
\pgfpathmoveto{\pgfqpoint{0.456907in}{2.376067in}}%
\pgfpathlineto{\pgfqpoint{0.460548in}{2.649843in}}%
\pgfpathlineto{\pgfqpoint{0.464079in}{2.741851in}}%
\pgfpathlineto{\pgfqpoint{0.467958in}{2.791431in}}%
\pgfpathlineto{\pgfqpoint{0.472464in}{2.823366in}}%
\pgfpathlineto{\pgfqpoint{0.477082in}{2.842552in}}%
\pgfpathlineto{\pgfqpoint{0.482433in}{2.856152in}}%
\pgfpathlineto{\pgfqpoint{0.488841in}{2.866172in}}%
\pgfpathlineto{\pgfqpoint{0.495412in}{2.872674in}}%
\pgfpathlineto{\pgfqpoint{0.503013in}{2.877486in}}%
\pgfpathlineto{\pgfqpoint{0.513269in}{2.881590in}}%
\pgfpathlineto{\pgfqpoint{0.528185in}{2.885042in}}%
\pgfpathlineto{\pgfqpoint{0.551983in}{2.887843in}}%
\pgfpathlineto{\pgfqpoint{0.591101in}{2.889732in}}%
\pgfpathlineto{\pgfqpoint{0.678106in}{2.891031in}}%
\pgfpathlineto{\pgfqpoint{0.880410in}{2.891630in}}%
\pgfpathlineto{\pgfqpoint{2.459700in}{2.891752in}}%
\pgfpathlineto{\pgfqpoint{4.715513in}{2.890688in}}%
\pgfpathlineto{\pgfqpoint{4.750267in}{2.888727in}}%
\pgfpathlineto{\pgfqpoint{4.763124in}{2.886238in}}%
\pgfpathlineto{\pgfqpoint{4.771207in}{2.882625in}}%
\pgfpathlineto{\pgfqpoint{4.776343in}{2.878061in}}%
\pgfpathlineto{\pgfqpoint{4.780134in}{2.872009in}}%
\pgfpathlineto{\pgfqpoint{4.783383in}{2.862790in}}%
\pgfpathlineto{\pgfqpoint{4.786545in}{2.845761in}}%
\pgfpathlineto{\pgfqpoint{4.789205in}{2.816054in}}%
\pgfpathlineto{\pgfqpoint{4.791421in}{2.761353in}}%
\pgfpathlineto{\pgfqpoint{4.793143in}{2.656827in}}%
\pgfpathlineto{\pgfqpoint{4.794393in}{2.425336in}}%
\pgfpathlineto{\pgfqpoint{4.794729in}{1.949899in}}%
\pgfpathlineto{\pgfqpoint{4.793198in}{1.511803in}}%
\pgfpathlineto{\pgfqpoint{4.789904in}{1.190720in}}%
\pgfpathlineto{\pgfqpoint{4.786136in}{1.021511in}}%
\pgfpathlineto{\pgfqpoint{4.781664in}{0.907124in}}%
\pgfpathlineto{\pgfqpoint{4.776495in}{0.825197in}}%
\pgfpathlineto{\pgfqpoint{4.770490in}{0.763353in}}%
\pgfpathlineto{\pgfqpoint{4.763822in}{0.716684in}}%
\pgfpathlineto{\pgfqpoint{4.756520in}{0.680301in}}%
\pgfpathlineto{\pgfqpoint{4.748082in}{0.649426in}}%
\pgfpathlineto{\pgfqpoint{4.738873in}{0.624165in}}%
\pgfpathlineto{\pgfqpoint{4.728439in}{0.602339in}}%
\pgfpathlineto{\pgfqpoint{4.717138in}{0.584060in}}%
\pgfpathlineto{\pgfqpoint{4.705535in}{0.569233in}}%
\pgfpathlineto{\pgfqpoint{4.692573in}{0.555960in}}%
\pgfpathlineto{\pgfqpoint{4.676629in}{0.542979in}}%
\pgfpathlineto{\pgfqpoint{4.659586in}{0.531972in}}%
\pgfpathlineto{\pgfqpoint{4.639750in}{0.521772in}}%
\pgfpathlineto{\pgfqpoint{4.615144in}{0.511821in}}%
\pgfpathlineto{\pgfqpoint{4.585780in}{0.502604in}}%
\pgfpathlineto{\pgfqpoint{4.551736in}{0.494339in}}%
\pgfpathlineto{\pgfqpoint{4.510948in}{0.486718in}}%
\pgfpathlineto{\pgfqpoint{4.459131in}{0.479450in}}%
\pgfpathlineto{\pgfqpoint{4.398463in}{0.473285in}}%
\pgfpathlineto{\pgfqpoint{4.324641in}{0.468138in}}%
\pgfpathlineto{\pgfqpoint{4.235518in}{0.464246in}}%
\pgfpathlineto{\pgfqpoint{4.126775in}{0.461742in}}%
\pgfpathlineto{\pgfqpoint{3.994083in}{0.461001in}}%
\pgfpathlineto{\pgfqpoint{3.859219in}{0.462414in}}%
\pgfpathlineto{\pgfqpoint{3.737438in}{0.465817in}}%
\pgfpathlineto{\pgfqpoint{3.646166in}{0.470451in}}%
\pgfpathlineto{\pgfqpoint{3.554992in}{0.477163in}}%
\pgfpathlineto{\pgfqpoint{3.476987in}{0.485057in}}%
\pgfpathlineto{\pgfqpoint{3.410013in}{0.494036in}}%
\pgfpathlineto{\pgfqpoint{3.354088in}{0.503678in}}%
\pgfpathlineto{\pgfqpoint{3.304934in}{0.514340in}}%
\pgfpathlineto{\pgfqpoint{3.262583in}{0.525717in}}%
\pgfpathlineto{\pgfqpoint{3.224948in}{0.538067in}}%
\pgfpathlineto{\pgfqpoint{3.192065in}{0.551101in}}%
\pgfpathlineto{\pgfqpoint{3.161933in}{0.565413in}}%
\pgfpathlineto{\pgfqpoint{3.134647in}{0.580877in}}%
\pgfpathlineto{\pgfqpoint{3.110267in}{0.597259in}}%
\pgfpathlineto{\pgfqpoint{3.088791in}{0.614227in}}%
\pgfpathlineto{\pgfqpoint{3.068502in}{0.633010in}}%
\pgfpathlineto{\pgfqpoint{3.049574in}{0.653568in}}%
\pgfpathlineto{\pgfqpoint{3.032140in}{0.675790in}}%
\pgfpathlineto{\pgfqpoint{3.016272in}{0.699498in}}%
\pgfpathlineto{\pgfqpoint{3.000852in}{0.726615in}}%
\pgfpathlineto{\pgfqpoint{2.986230in}{0.757175in}}%
\pgfpathlineto{\pgfqpoint{2.973526in}{0.788840in}}%
\pgfpathlineto{\pgfqpoint{2.961846in}{0.823698in}}%
\pgfpathlineto{\pgfqpoint{2.950753in}{0.864059in}}%
\pgfpathlineto{\pgfqpoint{2.941187in}{0.907501in}}%
\pgfpathlineto{\pgfqpoint{2.932834in}{0.956353in}}%
\pgfpathlineto{\pgfqpoint{2.926289in}{1.008082in}}%
\pgfpathlineto{\pgfqpoint{2.921371in}{1.065052in}}%
\pgfpathlineto{\pgfqpoint{2.918422in}{1.124693in}}%
\pgfpathlineto{\pgfqpoint{2.917436in}{1.189399in}}%
\pgfpathlineto{\pgfqpoint{2.918599in}{1.256590in}}%
\pgfpathlineto{\pgfqpoint{2.921978in}{1.326177in}}%
\pgfpathlineto{\pgfqpoint{2.927718in}{1.398061in}}%
\pgfpathlineto{\pgfqpoint{2.935724in}{1.469660in}}%
\pgfpathlineto{\pgfqpoint{2.945663in}{1.538420in}}%
\pgfpathlineto{\pgfqpoint{2.957415in}{1.604265in}}%
\pgfpathlineto{\pgfqpoint{2.970882in}{1.667118in}}%
\pgfpathlineto{\pgfqpoint{2.985974in}{1.726900in}}%
\pgfpathlineto{\pgfqpoint{3.002604in}{1.783523in}}%
\pgfpathlineto{\pgfqpoint{3.019857in}{1.834598in}}%
\pgfpathlineto{\pgfqpoint{3.038287in}{1.882424in}}%
\pgfpathlineto{\pgfqpoint{3.057787in}{1.926921in}}%
\pgfpathlineto{\pgfqpoint{3.078241in}{1.968011in}}%
\pgfpathlineto{\pgfqpoint{3.099505in}{2.005627in}}%
\pgfpathlineto{\pgfqpoint{3.121400in}{2.039721in}}%
\pgfpathlineto{\pgfqpoint{3.143708in}{2.070283in}}%
\pgfpathlineto{\pgfqpoint{3.166174in}{2.097353in}}%
\pgfpathlineto{\pgfqpoint{3.188519in}{2.121023in}}%
\pgfpathlineto{\pgfqpoint{3.212190in}{2.142938in}}%
\pgfpathlineto{\pgfqpoint{3.235339in}{2.161514in}}%
\pgfpathlineto{\pgfqpoint{3.259588in}{2.178150in}}%
\pgfpathlineto{\pgfqpoint{3.282924in}{2.191518in}}%
\pgfpathlineto{\pgfqpoint{3.305108in}{2.201760in}}%
\pgfpathlineto{\pgfqpoint{3.325932in}{2.208924in}}%
\pgfpathlineto{\pgfqpoint{3.343031in}{2.212559in}}%
\pgfpathlineto{\pgfqpoint{3.356053in}{2.213358in}}%
\pgfpathlineto{\pgfqpoint{3.366806in}{2.211704in}}%
\pgfpathlineto{\pgfqpoint{3.372791in}{2.208789in}}%
\pgfpathlineto{\pgfqpoint{3.376099in}{2.205586in}}%
\pgfpathlineto{\pgfqpoint{3.378277in}{2.201308in}}%
\pgfpathlineto{\pgfqpoint{3.379000in}{2.193945in}}%
\pgfpathlineto{\pgfqpoint{3.376855in}{2.184330in}}%
\pgfpathlineto{\pgfqpoint{3.370821in}{2.171109in}}%
\pgfpathlineto{\pgfqpoint{3.359450in}{2.152884in}}%
\pgfpathlineto{\pgfqpoint{3.338470in}{2.124293in}}%
\pgfpathlineto{\pgfqpoint{3.271859in}{2.035557in}}%
\pgfpathlineto{\pgfqpoint{3.245525in}{1.995935in}}%
\pgfpathlineto{\pgfqpoint{3.222216in}{1.956885in}}%
\pgfpathlineto{\pgfqpoint{3.200754in}{1.916473in}}%
\pgfpathlineto{\pgfqpoint{3.181225in}{1.874797in}}%
\pgfpathlineto{\pgfqpoint{3.163648in}{1.831998in}}%
\pgfpathlineto{\pgfqpoint{3.147216in}{1.785907in}}%
\pgfpathlineto{\pgfqpoint{3.132827in}{1.738930in}}%
\pgfpathlineto{\pgfqpoint{3.119802in}{1.688831in}}%
\pgfpathlineto{\pgfqpoint{3.108302in}{1.635679in}}%
\pgfpathlineto{\pgfqpoint{3.098473in}{1.579547in}}%
\pgfpathlineto{\pgfqpoint{3.090472in}{1.520517in}}%
\pgfpathlineto{\pgfqpoint{3.084676in}{1.461150in}}%
\pgfpathlineto{\pgfqpoint{3.081034in}{1.401558in}}%
\pgfpathlineto{\pgfqpoint{3.079612in}{1.341844in}}%
\pgfpathlineto{\pgfqpoint{3.080418in}{1.284605in}}%
\pgfpathlineto{\pgfqpoint{3.083308in}{1.229947in}}%
\pgfpathlineto{\pgfqpoint{3.088175in}{1.177976in}}%
\pgfpathlineto{\pgfqpoint{3.094935in}{1.128802in}}%
\pgfpathlineto{\pgfqpoint{3.103469in}{1.082532in}}%
\pgfpathlineto{\pgfqpoint{3.113664in}{1.039278in}}%
\pgfpathlineto{\pgfqpoint{3.125375in}{0.999146in}}%
\pgfpathlineto{\pgfqpoint{3.138421in}{0.962230in}}%
\pgfpathlineto{\pgfqpoint{3.152583in}{0.928599in}}%
\pgfpathlineto{\pgfqpoint{3.168728in}{0.896162in}}%
\pgfpathlineto{\pgfqpoint{3.185650in}{0.867196in}}%
\pgfpathlineto{\pgfqpoint{3.204370in}{0.839719in}}%
\pgfpathlineto{\pgfqpoint{3.223316in}{0.815704in}}%
\pgfpathlineto{\pgfqpoint{3.243673in}{0.793252in}}%
\pgfpathlineto{\pgfqpoint{3.267023in}{0.770890in}}%
\pgfpathlineto{\pgfqpoint{3.291661in}{0.750417in}}%
\pgfpathlineto{\pgfqpoint{3.319263in}{0.730516in}}%
\pgfpathlineto{\pgfqpoint{3.347882in}{0.712594in}}%
\pgfpathlineto{\pgfqpoint{3.381258in}{0.694381in}}%
\pgfpathlineto{\pgfqpoint{3.415479in}{0.678352in}}%
\pgfpathlineto{\pgfqpoint{3.452406in}{0.663464in}}%
\pgfpathlineto{\pgfqpoint{3.494048in}{0.649059in}}%
\pgfpathlineto{\pgfqpoint{3.538300in}{0.636097in}}%
\pgfpathlineto{\pgfqpoint{3.587224in}{0.624130in}}%
\pgfpathlineto{\pgfqpoint{3.638656in}{0.613889in}}%
\pgfpathlineto{\pgfqpoint{3.696818in}{0.604560in}}%
\pgfpathlineto{\pgfqpoint{3.750898in}{0.598016in}}%
\pgfpathlineto{\pgfqpoint{3.818133in}{0.592120in}}%
\pgfpathlineto{\pgfqpoint{3.889831in}{0.588114in}}%
\pgfpathlineto{\pgfqpoint{3.963772in}{0.586231in}}%
\pgfpathlineto{\pgfqpoint{4.035555in}{0.586559in}}%
\pgfpathlineto{\pgfqpoint{4.107304in}{0.589145in}}%
\pgfpathlineto{\pgfqpoint{4.172438in}{0.593720in}}%
\pgfpathlineto{\pgfqpoint{4.233083in}{0.600173in}}%
\pgfpathlineto{\pgfqpoint{4.287011in}{0.608180in}}%
\pgfpathlineto{\pgfqpoint{4.336330in}{0.617793in}}%
\pgfpathlineto{\pgfqpoint{4.376730in}{0.627757in}}%
\pgfpathlineto{\pgfqpoint{4.414569in}{0.639261in}}%
\pgfpathlineto{\pgfqpoint{4.447684in}{0.651507in}}%
\pgfpathlineto{\pgfqpoint{4.478091in}{0.665035in}}%
\pgfpathlineto{\pgfqpoint{4.505699in}{0.679731in}}%
\pgfpathlineto{\pgfqpoint{4.530441in}{0.695388in}}%
\pgfpathlineto{\pgfqpoint{4.552295in}{0.711711in}}%
\pgfpathlineto{\pgfqpoint{4.572982in}{0.729915in}}%
\pgfpathlineto{\pgfqpoint{4.592294in}{0.750000in}}%
\pgfpathlineto{\pgfqpoint{4.610052in}{0.771881in}}%
\pgfpathlineto{\pgfqpoint{4.626142in}{0.795392in}}%
\pgfpathlineto{\pgfqpoint{4.640556in}{0.820288in}}%
\pgfpathlineto{\pgfqpoint{4.654361in}{0.848521in}}%
\pgfpathlineto{\pgfqpoint{4.667312in}{0.880054in}}%
\pgfpathlineto{\pgfqpoint{4.680018in}{0.917126in}}%
\pgfpathlineto{\pgfqpoint{4.692063in}{0.959752in}}%
\pgfpathlineto{\pgfqpoint{4.703171in}{1.007881in}}%
\pgfpathlineto{\pgfqpoint{4.713207in}{1.061421in}}%
\pgfpathlineto{\pgfqpoint{4.722418in}{1.122749in}}%
\pgfpathlineto{\pgfqpoint{4.730584in}{1.191814in}}%
\pgfpathlineto{\pgfqpoint{4.737495in}{1.268569in}}%
\pgfpathlineto{\pgfqpoint{4.744304in}{1.365332in}}%
\pgfpathlineto{\pgfqpoint{4.749906in}{1.477158in}}%
\pgfpathlineto{\pgfqpoint{4.754970in}{1.623905in}}%
\pgfpathlineto{\pgfqpoint{4.758231in}{1.785658in}}%
\pgfpathlineto{\pgfqpoint{4.759766in}{1.962382in}}%
\pgfpathlineto{\pgfqpoint{4.759312in}{2.156537in}}%
\pgfpathlineto{\pgfqpoint{4.756770in}{2.323287in}}%
\pgfpathlineto{\pgfqpoint{4.752920in}{2.445176in}}%
\pgfpathlineto{\pgfqpoint{4.747592in}{2.544554in}}%
\pgfpathlineto{\pgfqpoint{4.741518in}{2.616401in}}%
\pgfpathlineto{\pgfqpoint{4.734822in}{2.670619in}}%
\pgfpathlineto{\pgfqpoint{4.727592in}{2.712112in}}%
\pgfpathlineto{\pgfqpoint{4.720163in}{2.743325in}}%
\pgfpathlineto{\pgfqpoint{4.711857in}{2.768992in}}%
\pgfpathlineto{\pgfqpoint{4.703158in}{2.789049in}}%
\pgfpathlineto{\pgfqpoint{4.693598in}{2.805672in}}%
\pgfpathlineto{\pgfqpoint{4.683667in}{2.818865in}}%
\pgfpathlineto{\pgfqpoint{4.672290in}{2.830425in}}%
\pgfpathlineto{\pgfqpoint{4.659685in}{2.840180in}}%
\pgfpathlineto{\pgfqpoint{4.644184in}{2.849205in}}%
\pgfpathlineto{\pgfqpoint{4.625851in}{2.857040in}}%
\pgfpathlineto{\pgfqpoint{4.604861in}{2.863548in}}%
\pgfpathlineto{\pgfqpoint{4.577088in}{2.869609in}}%
\pgfpathlineto{\pgfqpoint{4.542570in}{2.874687in}}%
\pgfpathlineto{\pgfqpoint{4.499214in}{2.878784in}}%
\pgfpathlineto{\pgfqpoint{4.436216in}{2.882509in}}%
\pgfpathlineto{\pgfqpoint{4.344890in}{2.885485in}}%
\pgfpathlineto{\pgfqpoint{4.205684in}{2.887764in}}%
\pgfpathlineto{\pgfqpoint{3.946825in}{2.889558in}}%
\pgfpathlineto{\pgfqpoint{3.285526in}{2.890736in}}%
\pgfpathlineto{\pgfqpoint{1.934646in}{2.890412in}}%
\pgfpathlineto{\pgfqpoint{1.373413in}{2.888549in}}%
\pgfpathlineto{\pgfqpoint{1.108038in}{2.885478in}}%
\pgfpathlineto{\pgfqpoint{0.971039in}{2.881477in}}%
\pgfpathlineto{\pgfqpoint{0.899337in}{2.877562in}}%
\pgfpathlineto{\pgfqpoint{0.840776in}{2.872460in}}%
\pgfpathlineto{\pgfqpoint{0.793246in}{2.866115in}}%
\pgfpathlineto{\pgfqpoint{0.765343in}{2.860872in}}%
\pgfpathlineto{\pgfqpoint{0.737718in}{2.853979in}}%
\pgfpathlineto{\pgfqpoint{0.710597in}{2.844845in}}%
\pgfpathlineto{\pgfqpoint{0.688420in}{2.834604in}}%
\pgfpathlineto{\pgfqpoint{0.671182in}{2.824000in}}%
\pgfpathlineto{\pgfqpoint{0.655003in}{2.811416in}}%
\pgfpathlineto{\pgfqpoint{0.640117in}{2.796887in}}%
\pgfpathlineto{\pgfqpoint{0.626906in}{2.780372in}}%
\pgfpathlineto{\pgfqpoint{0.615584in}{2.762107in}}%
\pgfpathlineto{\pgfqpoint{0.604963in}{2.740395in}}%
\pgfpathlineto{\pgfqpoint{0.595325in}{2.715343in}}%
\pgfpathlineto{\pgfqpoint{0.586155in}{2.684740in}}%
\pgfpathlineto{\pgfqpoint{0.578356in}{2.651060in}}%
\pgfpathlineto{\pgfqpoint{0.570961in}{2.609604in}}%
\pgfpathlineto{\pgfqpoint{0.563015in}{2.548049in}}%
\pgfpathlineto{\pgfqpoint{0.557596in}{2.486133in}}%
\pgfpathlineto{\pgfqpoint{0.552491in}{2.396721in}}%
\pgfpathlineto{\pgfqpoint{0.549901in}{2.312146in}}%
\pgfpathlineto{\pgfqpoint{0.549281in}{2.125464in}}%
\pgfpathlineto{\pgfqpoint{0.551809in}{2.043373in}}%
\pgfpathlineto{\pgfqpoint{0.555895in}{1.946412in}}%
\pgfpathlineto{\pgfqpoint{0.563546in}{1.829753in}}%
\pgfpathlineto{\pgfqpoint{0.569913in}{1.760440in}}%
\pgfpathlineto{\pgfqpoint{0.579699in}{1.676557in}}%
\pgfpathlineto{\pgfqpoint{0.589273in}{1.612775in}}%
\pgfpathlineto{\pgfqpoint{0.600715in}{1.551944in}}%
\pgfpathlineto{\pgfqpoint{0.611917in}{1.503843in}}%
\pgfpathlineto{\pgfqpoint{0.624045in}{1.461248in}}%
\pgfpathlineto{\pgfqpoint{0.636090in}{1.426554in}}%
\pgfpathlineto{\pgfqpoint{0.648339in}{1.397397in}}%
\pgfpathlineto{\pgfqpoint{0.660415in}{1.373768in}}%
\pgfpathlineto{\pgfqpoint{0.673163in}{1.353614in}}%
\pgfpathlineto{\pgfqpoint{0.684942in}{1.338974in}}%
\pgfpathlineto{\pgfqpoint{0.696737in}{1.327983in}}%
\pgfpathlineto{\pgfqpoint{0.706185in}{1.321848in}}%
\pgfpathlineto{\pgfqpoint{0.714382in}{1.318550in}}%
\pgfpathlineto{\pgfqpoint{0.722995in}{1.317318in}}%
\pgfpathlineto{\pgfqpoint{0.731582in}{1.318706in}}%
\pgfpathlineto{\pgfqpoint{0.739490in}{1.322784in}}%
\pgfpathlineto{\pgfqpoint{0.746267in}{1.328996in}}%
\pgfpathlineto{\pgfqpoint{0.753106in}{1.338647in}}%
\pgfpathlineto{\pgfqpoint{0.759389in}{1.351720in}}%
\pgfpathlineto{\pgfqpoint{0.765505in}{1.370348in}}%
\pgfpathlineto{\pgfqpoint{0.770808in}{1.394480in}}%
\pgfpathlineto{\pgfqpoint{0.775369in}{1.426408in}}%
\pgfpathlineto{\pgfqpoint{0.779226in}{1.470991in}}%
\pgfpathlineto{\pgfqpoint{0.782290in}{1.535611in}}%
\pgfpathlineto{\pgfqpoint{0.792620in}{1.786725in}}%
\pgfpathlineto{\pgfqpoint{0.798686in}{1.866074in}}%
\pgfpathlineto{\pgfqpoint{0.807852in}{1.952557in}}%
\pgfpathlineto{\pgfqpoint{0.818726in}{2.028706in}}%
\pgfpathlineto{\pgfqpoint{0.829282in}{2.087209in}}%
\pgfpathlineto{\pgfqpoint{0.842029in}{2.145136in}}%
\pgfpathlineto{\pgfqpoint{0.855892in}{2.197545in}}%
\pgfpathlineto{\pgfqpoint{0.871354in}{2.246725in}}%
\pgfpathlineto{\pgfqpoint{0.887521in}{2.290246in}}%
\pgfpathlineto{\pgfqpoint{0.904847in}{2.330420in}}%
\pgfpathlineto{\pgfqpoint{0.916897in}{2.354036in}}%
\pgfpathlineto{\pgfqpoint{0.937145in}{2.392321in}}%
\pgfpathlineto{\pgfqpoint{0.940161in}{2.398894in}}%
\pgfpathlineto{\pgfqpoint{0.960832in}{2.430929in}}%
\pgfpathlineto{\pgfqpoint{0.981817in}{2.459513in}}%
\pgfpathlineto{\pgfqpoint{1.007589in}{2.489854in}}%
\pgfpathlineto{\pgfqpoint{1.034876in}{2.518400in}}%
\pgfpathlineto{\pgfqpoint{1.062212in}{2.543042in}}%
\pgfpathlineto{\pgfqpoint{1.080173in}{2.557013in}}%
\pgfpathlineto{\pgfqpoint{1.087645in}{2.562064in}}%
\pgfpathlineto{\pgfqpoint{1.115363in}{2.581757in}}%
\pgfpathlineto{\pgfqpoint{1.145869in}{2.600919in}}%
\pgfpathlineto{\pgfqpoint{1.179170in}{2.619310in}}%
\pgfpathlineto{\pgfqpoint{1.219264in}{2.638622in}}%
\pgfpathlineto{\pgfqpoint{1.258069in}{2.654894in}}%
\pgfpathlineto{\pgfqpoint{1.305786in}{2.672087in}}%
\pgfpathlineto{\pgfqpoint{1.326895in}{2.677933in}}%
\pgfpathlineto{\pgfqpoint{1.386003in}{2.694729in}}%
\pgfpathlineto{\pgfqpoint{1.439267in}{2.707273in}}%
\pgfpathlineto{\pgfqpoint{1.495032in}{2.717973in}}%
\pgfpathlineto{\pgfqpoint{1.555318in}{2.727816in}}%
\pgfpathlineto{\pgfqpoint{1.628722in}{2.738160in}}%
\pgfpathlineto{\pgfqpoint{1.708822in}{2.747164in}}%
\pgfpathlineto{\pgfqpoint{1.797755in}{2.754864in}}%
\pgfpathlineto{\pgfqpoint{1.897677in}{2.760928in}}%
\pgfpathlineto{\pgfqpoint{1.999852in}{2.765065in}}%
\pgfpathlineto{\pgfqpoint{2.152109in}{2.767346in}}%
\pgfpathlineto{\pgfqpoint{2.258694in}{2.766311in}}%
\pgfpathlineto{\pgfqpoint{2.356545in}{2.763216in}}%
\pgfpathlineto{\pgfqpoint{2.445626in}{2.758243in}}%
\pgfpathlineto{\pgfqpoint{2.523728in}{2.751719in}}%
\pgfpathlineto{\pgfqpoint{2.590817in}{2.743940in}}%
\pgfpathlineto{\pgfqpoint{2.646866in}{2.735296in}}%
\pgfpathlineto{\pgfqpoint{2.694007in}{2.725885in}}%
\pgfpathlineto{\pgfqpoint{2.734348in}{2.715613in}}%
\pgfpathlineto{\pgfqpoint{2.767847in}{2.704827in}}%
\pgfpathlineto{\pgfqpoint{2.794517in}{2.694086in}}%
\pgfpathlineto{\pgfqpoint{2.818397in}{2.682045in}}%
\pgfpathlineto{\pgfqpoint{2.837408in}{2.669966in}}%
\pgfpathlineto{\pgfqpoint{2.853415in}{2.657087in}}%
\pgfpathlineto{\pgfqpoint{2.866326in}{2.643756in}}%
\pgfpathlineto{\pgfqpoint{2.876215in}{2.630522in}}%
\pgfpathlineto{\pgfqpoint{2.884445in}{2.615879in}}%
\pgfpathlineto{\pgfqpoint{2.890790in}{2.600055in}}%
\pgfpathlineto{\pgfqpoint{2.895213in}{2.583394in}}%
\pgfpathlineto{\pgfqpoint{2.898120in}{2.563773in}}%
\pgfpathlineto{\pgfqpoint{2.899155in}{2.541412in}}%
\pgfpathlineto{\pgfqpoint{2.898036in}{2.514071in}}%
\pgfpathlineto{\pgfqpoint{2.894143in}{2.479516in}}%
\pgfpathlineto{\pgfqpoint{2.886703in}{2.435531in}}%
\pgfpathlineto{\pgfqpoint{2.871825in}{2.362824in}}%
\pgfpathlineto{\pgfqpoint{2.831166in}{2.166651in}}%
\pgfpathlineto{\pgfqpoint{2.814431in}{2.071482in}}%
\pgfpathlineto{\pgfqpoint{2.799865in}{1.975847in}}%
\pgfpathlineto{\pgfqpoint{2.793937in}{1.939240in}}%
\pgfpathlineto{\pgfqpoint{2.792365in}{1.932031in}}%
\pgfpathlineto{\pgfqpoint{2.779324in}{1.823529in}}%
\pgfpathlineto{\pgfqpoint{2.767356in}{1.704837in}}%
\pgfpathlineto{\pgfqpoint{2.757601in}{1.585883in}}%
\pgfpathlineto{\pgfqpoint{2.747868in}{1.444438in}}%
\pgfpathlineto{\pgfqpoint{2.739452in}{1.290410in}}%
\pgfpathlineto{\pgfqpoint{2.732138in}{1.116368in}}%
\pgfpathlineto{\pgfqpoint{2.726317in}{0.924817in}}%
\pgfpathlineto{\pgfqpoint{2.721265in}{0.685924in}}%
\pgfpathlineto{\pgfqpoint{2.716535in}{0.479400in}}%
\pgfpathlineto{\pgfqpoint{2.713788in}{0.444704in}}%
\pgfpathlineto{\pgfqpoint{2.710464in}{0.427724in}}%
\pgfpathlineto{\pgfqpoint{2.706606in}{0.418839in}}%
\pgfpathlineto{\pgfqpoint{2.702014in}{0.413581in}}%
\pgfpathlineto{\pgfqpoint{2.696250in}{0.410128in}}%
\pgfpathlineto{\pgfqpoint{2.685710in}{0.407136in}}%
\pgfpathlineto{\pgfqpoint{2.668401in}{0.405158in}}%
\pgfpathlineto{\pgfqpoint{2.631442in}{0.403789in}}%
\pgfpathlineto{\pgfqpoint{2.529205in}{0.402988in}}%
\pgfpathlineto{\pgfqpoint{2.052807in}{0.402659in}}%
\pgfpathlineto{\pgfqpoint{0.456120in}{0.403499in}}%
\pgfpathlineto{\pgfqpoint{0.449622in}{0.403674in}}%
\pgfpathlineto{\pgfqpoint{0.449622in}{0.403674in}}%
\pgfusepath{stroke}%
\end{pgfscope}%
\begin{pgfscope}%
\pgfsetrectcap%
\pgfsetmiterjoin%
\pgfsetlinewidth{0.803000pt}%
\definecolor{currentstroke}{rgb}{0.000000,0.000000,0.000000}%
\pgfsetstrokecolor{currentstroke}%
\pgfsetdash{}{0pt}%
\pgfpathmoveto{\pgfqpoint{0.448634in}{0.402556in}}%
\pgfpathlineto{\pgfqpoint{0.448634in}{2.891760in}}%
\pgfusepath{stroke}%
\end{pgfscope}%
\begin{pgfscope}%
\pgfsetrectcap%
\pgfsetmiterjoin%
\pgfsetlinewidth{0.803000pt}%
\definecolor{currentstroke}{rgb}{0.000000,0.000000,0.000000}%
\pgfsetstrokecolor{currentstroke}%
\pgfsetdash{}{0pt}%
\pgfpathmoveto{\pgfqpoint{4.799294in}{0.402556in}}%
\pgfpathlineto{\pgfqpoint{4.799294in}{2.891760in}}%
\pgfusepath{stroke}%
\end{pgfscope}%
\begin{pgfscope}%
\pgfsetrectcap%
\pgfsetmiterjoin%
\pgfsetlinewidth{0.803000pt}%
\definecolor{currentstroke}{rgb}{0.000000,0.000000,0.000000}%
\pgfsetstrokecolor{currentstroke}%
\pgfsetdash{}{0pt}%
\pgfpathmoveto{\pgfqpoint{0.448634in}{0.402556in}}%
\pgfpathlineto{\pgfqpoint{4.799294in}{0.402556in}}%
\pgfusepath{stroke}%
\end{pgfscope}%
\begin{pgfscope}%
\pgfsetrectcap%
\pgfsetmiterjoin%
\pgfsetlinewidth{0.803000pt}%
\definecolor{currentstroke}{rgb}{0.000000,0.000000,0.000000}%
\pgfsetstrokecolor{currentstroke}%
\pgfsetdash{}{0pt}%
\pgfpathmoveto{\pgfqpoint{0.448634in}{2.891760in}}%
\pgfpathlineto{\pgfqpoint{4.799294in}{2.891760in}}%
\pgfusepath{stroke}%
\end{pgfscope}%
\begin{pgfscope}%
\pgfsetbuttcap%
\pgfsetmiterjoin%
\definecolor{currentfill}{rgb}{1.000000,1.000000,1.000000}%
\pgfsetfillcolor{currentfill}%
\pgfsetfillopacity{0.500000}%
\pgfsetlinewidth{1.003750pt}%
\definecolor{currentstroke}{rgb}{0.800000,0.800000,0.800000}%
\pgfsetstrokecolor{currentstroke}%
\pgfsetstrokeopacity{0.500000}%
\pgfsetdash{}{0pt}%
\pgfpathmoveto{\pgfqpoint{3.700085in}{0.761312in}}%
\pgfpathlineto{\pgfqpoint{4.547129in}{0.761312in}}%
\pgfpathquadraticcurveto{\pgfqpoint{4.574907in}{0.761312in}}{\pgfqpoint{4.574907in}{0.789090in}}%
\pgfpathlineto{\pgfqpoint{4.574907in}{2.769646in}}%
\pgfpathquadraticcurveto{\pgfqpoint{4.574907in}{2.797424in}}{\pgfqpoint{4.547129in}{2.797424in}}%
\pgfpathlineto{\pgfqpoint{3.700085in}{2.797424in}}%
\pgfpathquadraticcurveto{\pgfqpoint{3.672307in}{2.797424in}}{\pgfqpoint{3.672307in}{2.769646in}}%
\pgfpathlineto{\pgfqpoint{3.672307in}{0.789090in}}%
\pgfpathquadraticcurveto{\pgfqpoint{3.672307in}{0.761312in}}{\pgfqpoint{3.700085in}{0.761312in}}%
\pgfpathclose%
\pgfusepath{stroke,fill}%
\end{pgfscope}%
\begin{pgfscope}%
\pgfsetrectcap%
\pgfsetroundjoin%
\pgfsetlinewidth{1.003750pt}%
\definecolor{currentstroke}{rgb}{1.000000,0.388235,0.278431}%
\pgfsetstrokecolor{currentstroke}%
\pgfsetdash{}{0pt}%
\pgfpathmoveto{\pgfqpoint{3.727863in}{2.693257in}}%
\pgfpathlineto{\pgfqpoint{3.797307in}{2.693257in}}%
\pgfusepath{stroke}%
\end{pgfscope}%
\begin{pgfscope}%
\pgftext[x=3.908418in,y=2.644646in,left,base]{\rmfamily\fontsize{10.000000}{12.000000}\selectfont \textnormal{Reference}}%
\end{pgfscope}%
\begin{pgfscope}%
\pgfsetrectcap%
\pgfsetroundjoin%
\pgfsetlinewidth{1.003750pt}%
\definecolor{currentstroke}{rgb}{0.121569,0.466667,0.705882}%
\pgfsetstrokecolor{currentstroke}%
\pgfsetdash{}{0pt}%
\pgfpathmoveto{\pgfqpoint{3.727863in}{2.493812in}}%
\pgfpathlineto{\pgfqpoint{3.797307in}{2.493812in}}%
\pgfusepath{stroke}%
\end{pgfscope}%
\begin{pgfscope}%
\pgftext[x=3.908418in,y=2.445201in,left,base]{\rmfamily\fontsize{10.000000}{12.000000}\selectfont \(\displaystyle \textnormal{tol}=10^{{-}10}\)}%
\end{pgfscope}%
\begin{pgfscope}%
\pgfsetrectcap%
\pgfsetroundjoin%
\pgfsetlinewidth{1.003750pt}%
\definecolor{currentstroke}{rgb}{1.000000,0.498039,0.054902}%
\pgfsetstrokecolor{currentstroke}%
\pgfsetdash{}{0pt}%
\pgfpathmoveto{\pgfqpoint{3.727863in}{2.294368in}}%
\pgfpathlineto{\pgfqpoint{3.797307in}{2.294368in}}%
\pgfusepath{stroke}%
\end{pgfscope}%
\begin{pgfscope}%
\pgftext[x=3.908418in,y=2.245757in,left,base]{\rmfamily\fontsize{10.000000}{12.000000}\selectfont \(\displaystyle \textnormal{tol}=10^{{-}9}\)}%
\end{pgfscope}%
\begin{pgfscope}%
\pgfsetrectcap%
\pgfsetroundjoin%
\pgfsetlinewidth{1.003750pt}%
\definecolor{currentstroke}{rgb}{0.172549,0.627451,0.172549}%
\pgfsetstrokecolor{currentstroke}%
\pgfsetdash{}{0pt}%
\pgfpathmoveto{\pgfqpoint{3.727863in}{2.094924in}}%
\pgfpathlineto{\pgfqpoint{3.797307in}{2.094924in}}%
\pgfusepath{stroke}%
\end{pgfscope}%
\begin{pgfscope}%
\pgftext[x=3.908418in,y=2.046312in,left,base]{\rmfamily\fontsize{10.000000}{12.000000}\selectfont \(\displaystyle \textnormal{tol}=10^{{-}8}\)}%
\end{pgfscope}%
\begin{pgfscope}%
\pgfsetrectcap%
\pgfsetroundjoin%
\pgfsetlinewidth{1.003750pt}%
\definecolor{currentstroke}{rgb}{0.839216,0.152941,0.156863}%
\pgfsetstrokecolor{currentstroke}%
\pgfsetdash{}{0pt}%
\pgfpathmoveto{\pgfqpoint{3.727863in}{1.895479in}}%
\pgfpathlineto{\pgfqpoint{3.797307in}{1.895479in}}%
\pgfusepath{stroke}%
\end{pgfscope}%
\begin{pgfscope}%
\pgftext[x=3.908418in,y=1.846868in,left,base]{\rmfamily\fontsize{10.000000}{12.000000}\selectfont \(\displaystyle \textnormal{tol}=10^{{-}7}\)}%
\end{pgfscope}%
\begin{pgfscope}%
\pgfsetrectcap%
\pgfsetroundjoin%
\pgfsetlinewidth{1.003750pt}%
\definecolor{currentstroke}{rgb}{0.580392,0.403922,0.741176}%
\pgfsetstrokecolor{currentstroke}%
\pgfsetdash{}{0pt}%
\pgfpathmoveto{\pgfqpoint{3.727863in}{1.696035in}}%
\pgfpathlineto{\pgfqpoint{3.797307in}{1.696035in}}%
\pgfusepath{stroke}%
\end{pgfscope}%
\begin{pgfscope}%
\pgftext[x=3.908418in,y=1.647424in,left,base]{\rmfamily\fontsize{10.000000}{12.000000}\selectfont \(\displaystyle \textnormal{tol}=10^{{-}6}\)}%
\end{pgfscope}%
\begin{pgfscope}%
\pgfsetrectcap%
\pgfsetroundjoin%
\pgfsetlinewidth{1.003750pt}%
\definecolor{currentstroke}{rgb}{0.549020,0.337255,0.294118}%
\pgfsetstrokecolor{currentstroke}%
\pgfsetdash{}{0pt}%
\pgfpathmoveto{\pgfqpoint{3.727863in}{1.496590in}}%
\pgfpathlineto{\pgfqpoint{3.797307in}{1.496590in}}%
\pgfusepath{stroke}%
\end{pgfscope}%
\begin{pgfscope}%
\pgftext[x=3.908418in,y=1.447979in,left,base]{\rmfamily\fontsize{10.000000}{12.000000}\selectfont \(\displaystyle \textnormal{tol}=10^{{-}5}\)}%
\end{pgfscope}%
\begin{pgfscope}%
\pgfsetrectcap%
\pgfsetroundjoin%
\pgfsetlinewidth{1.003750pt}%
\definecolor{currentstroke}{rgb}{0.890196,0.466667,0.760784}%
\pgfsetstrokecolor{currentstroke}%
\pgfsetdash{}{0pt}%
\pgfpathmoveto{\pgfqpoint{3.727863in}{1.297146in}}%
\pgfpathlineto{\pgfqpoint{3.797307in}{1.297146in}}%
\pgfusepath{stroke}%
\end{pgfscope}%
\begin{pgfscope}%
\pgftext[x=3.908418in,y=1.248535in,left,base]{\rmfamily\fontsize{10.000000}{12.000000}\selectfont \(\displaystyle \textnormal{tol}=10^{{-}4}\)}%
\end{pgfscope}%
\begin{pgfscope}%
\pgfsetrectcap%
\pgfsetroundjoin%
\pgfsetlinewidth{1.003750pt}%
\definecolor{currentstroke}{rgb}{0.498039,0.498039,0.498039}%
\pgfsetstrokecolor{currentstroke}%
\pgfsetdash{}{0pt}%
\pgfpathmoveto{\pgfqpoint{3.727863in}{1.097701in}}%
\pgfpathlineto{\pgfqpoint{3.797307in}{1.097701in}}%
\pgfusepath{stroke}%
\end{pgfscope}%
\begin{pgfscope}%
\pgftext[x=3.908418in,y=1.049090in,left,base]{\rmfamily\fontsize{10.000000}{12.000000}\selectfont \(\displaystyle \textnormal{tol}=10^{{-}3}\)}%
\end{pgfscope}%
\begin{pgfscope}%
\pgfsetrectcap%
\pgfsetroundjoin%
\pgfsetlinewidth{1.003750pt}%
\definecolor{currentstroke}{rgb}{0.737255,0.741176,0.133333}%
\pgfsetstrokecolor{currentstroke}%
\pgfsetdash{}{0pt}%
\pgfpathmoveto{\pgfqpoint{3.727863in}{0.898257in}}%
\pgfpathlineto{\pgfqpoint{3.797307in}{0.898257in}}%
\pgfusepath{stroke}%
\end{pgfscope}%
\begin{pgfscope}%
\pgftext[x=3.908418in,y=0.849646in,left,base]{\rmfamily\fontsize{10.000000}{12.000000}\selectfont \(\displaystyle \textnormal{tol}=10^{{-}2}\)}%
\end{pgfscope}%
\end{pgfpicture}%
\makeatother%
\endgroup%

    %\resizebox{0.9\linewidth}{!}{%% Creator: Matplotlib, PGF backend
%%
%% To include the figure in your LaTeX document, write
%%   \input{<filename>.pgf}
%%
%% Make sure the required packages are loaded in your preamble
%%   \usepackage{pgf}
%%
%% Figures using additional raster images can only be included by \input if
%% they are in the same directory as the main LaTeX file. For loading figures
%% from other directories you can use the `import` package
%%   \usepackage{import}
%% and then include the figures with
%%   \import{<path to file>}{<filename>.pgf}
%%
%% Matplotlib used the following preamble
%%   \usepackage[utf8x]{inputenc}
%%   \usepackage[T1]{fontenc}
%%   \usepackage[]{libertine}\usepackage[libertine]{newtxmath}
%%
\begingroup%
\makeatletter%
\begin{pgfpicture}%
\pgfpathrectangle{\pgfpointorigin}{\pgfqpoint{5.050000in}{3.100000in}}%
\pgfusepath{use as bounding box, clip}%
\begin{pgfscope}%
\pgfsetbuttcap%
\pgfsetmiterjoin%
\definecolor{currentfill}{rgb}{1.000000,1.000000,1.000000}%
\pgfsetfillcolor{currentfill}%
\pgfsetlinewidth{0.000000pt}%
\definecolor{currentstroke}{rgb}{1.000000,1.000000,1.000000}%
\pgfsetstrokecolor{currentstroke}%
\pgfsetdash{}{0pt}%
\pgfpathmoveto{\pgfqpoint{0.000000in}{0.000000in}}%
\pgfpathlineto{\pgfqpoint{5.050000in}{0.000000in}}%
\pgfpathlineto{\pgfqpoint{5.050000in}{3.100000in}}%
\pgfpathlineto{\pgfqpoint{0.000000in}{3.100000in}}%
\pgfpathclose%
\pgfusepath{fill}%
\end{pgfscope}%
\begin{pgfscope}%
\pgfsetbuttcap%
\pgfsetmiterjoin%
\definecolor{currentfill}{rgb}{1.000000,1.000000,1.000000}%
\pgfsetfillcolor{currentfill}%
\pgfsetlinewidth{0.000000pt}%
\definecolor{currentstroke}{rgb}{0.000000,0.000000,0.000000}%
\pgfsetstrokecolor{currentstroke}%
\pgfsetstrokeopacity{0.000000}%
\pgfsetdash{}{0pt}%
\pgfpathmoveto{\pgfqpoint{0.448634in}{0.402556in}}%
\pgfpathlineto{\pgfqpoint{4.799294in}{0.402556in}}%
\pgfpathlineto{\pgfqpoint{4.799294in}{2.891760in}}%
\pgfpathlineto{\pgfqpoint{0.448634in}{2.891760in}}%
\pgfpathclose%
\pgfusepath{fill}%
\end{pgfscope}%
\begin{pgfscope}%
\pgfsetbuttcap%
\pgfsetroundjoin%
\definecolor{currentfill}{rgb}{0.000000,0.000000,0.000000}%
\pgfsetfillcolor{currentfill}%
\pgfsetlinewidth{0.803000pt}%
\definecolor{currentstroke}{rgb}{0.000000,0.000000,0.000000}%
\pgfsetstrokecolor{currentstroke}%
\pgfsetdash{}{0pt}%
\pgfsys@defobject{currentmarker}{\pgfqpoint{0.000000in}{-0.048611in}}{\pgfqpoint{0.000000in}{0.000000in}}{%
\pgfpathmoveto{\pgfqpoint{0.000000in}{0.000000in}}%
\pgfpathlineto{\pgfqpoint{0.000000in}{-0.048611in}}%
\pgfusepath{stroke,fill}%
}%
\begin{pgfscope}%
\pgfsys@transformshift{0.448634in}{0.402556in}%
\pgfsys@useobject{currentmarker}{}%
\end{pgfscope}%
\end{pgfscope}%
\begin{pgfscope}%
\pgftext[x=0.448634in,y=0.305334in,,top]{\rmfamily\fontsize{12.000000}{14.400000}\selectfont \(\displaystyle 0.00\)}%
\end{pgfscope}%
\begin{pgfscope}%
\pgfsetbuttcap%
\pgfsetroundjoin%
\definecolor{currentfill}{rgb}{0.000000,0.000000,0.000000}%
\pgfsetfillcolor{currentfill}%
\pgfsetlinewidth{0.803000pt}%
\definecolor{currentstroke}{rgb}{0.000000,0.000000,0.000000}%
\pgfsetstrokecolor{currentstroke}%
\pgfsetdash{}{0pt}%
\pgfsys@defobject{currentmarker}{\pgfqpoint{0.000000in}{-0.048611in}}{\pgfqpoint{0.000000in}{0.000000in}}{%
\pgfpathmoveto{\pgfqpoint{0.000000in}{0.000000in}}%
\pgfpathlineto{\pgfqpoint{0.000000in}{-0.048611in}}%
\pgfusepath{stroke,fill}%
}%
\begin{pgfscope}%
\pgfsys@transformshift{0.992466in}{0.402556in}%
\pgfsys@useobject{currentmarker}{}%
\end{pgfscope}%
\end{pgfscope}%
\begin{pgfscope}%
\pgftext[x=0.992466in,y=0.305334in,,top]{\rmfamily\fontsize{12.000000}{14.400000}\selectfont \(\displaystyle 0.25\)}%
\end{pgfscope}%
\begin{pgfscope}%
\pgfsetbuttcap%
\pgfsetroundjoin%
\definecolor{currentfill}{rgb}{0.000000,0.000000,0.000000}%
\pgfsetfillcolor{currentfill}%
\pgfsetlinewidth{0.803000pt}%
\definecolor{currentstroke}{rgb}{0.000000,0.000000,0.000000}%
\pgfsetstrokecolor{currentstroke}%
\pgfsetdash{}{0pt}%
\pgfsys@defobject{currentmarker}{\pgfqpoint{0.000000in}{-0.048611in}}{\pgfqpoint{0.000000in}{0.000000in}}{%
\pgfpathmoveto{\pgfqpoint{0.000000in}{0.000000in}}%
\pgfpathlineto{\pgfqpoint{0.000000in}{-0.048611in}}%
\pgfusepath{stroke,fill}%
}%
\begin{pgfscope}%
\pgfsys@transformshift{1.536299in}{0.402556in}%
\pgfsys@useobject{currentmarker}{}%
\end{pgfscope}%
\end{pgfscope}%
\begin{pgfscope}%
\pgftext[x=1.536299in,y=0.305334in,,top]{\rmfamily\fontsize{12.000000}{14.400000}\selectfont \(\displaystyle 0.50\)}%
\end{pgfscope}%
\begin{pgfscope}%
\pgfsetbuttcap%
\pgfsetroundjoin%
\definecolor{currentfill}{rgb}{0.000000,0.000000,0.000000}%
\pgfsetfillcolor{currentfill}%
\pgfsetlinewidth{0.803000pt}%
\definecolor{currentstroke}{rgb}{0.000000,0.000000,0.000000}%
\pgfsetstrokecolor{currentstroke}%
\pgfsetdash{}{0pt}%
\pgfsys@defobject{currentmarker}{\pgfqpoint{0.000000in}{-0.048611in}}{\pgfqpoint{0.000000in}{0.000000in}}{%
\pgfpathmoveto{\pgfqpoint{0.000000in}{0.000000in}}%
\pgfpathlineto{\pgfqpoint{0.000000in}{-0.048611in}}%
\pgfusepath{stroke,fill}%
}%
\begin{pgfscope}%
\pgfsys@transformshift{2.080131in}{0.402556in}%
\pgfsys@useobject{currentmarker}{}%
\end{pgfscope}%
\end{pgfscope}%
\begin{pgfscope}%
\pgftext[x=2.080131in,y=0.305334in,,top]{\rmfamily\fontsize{12.000000}{14.400000}\selectfont \(\displaystyle 0.75\)}%
\end{pgfscope}%
\begin{pgfscope}%
\pgfsetbuttcap%
\pgfsetroundjoin%
\definecolor{currentfill}{rgb}{0.000000,0.000000,0.000000}%
\pgfsetfillcolor{currentfill}%
\pgfsetlinewidth{0.803000pt}%
\definecolor{currentstroke}{rgb}{0.000000,0.000000,0.000000}%
\pgfsetstrokecolor{currentstroke}%
\pgfsetdash{}{0pt}%
\pgfsys@defobject{currentmarker}{\pgfqpoint{0.000000in}{-0.048611in}}{\pgfqpoint{0.000000in}{0.000000in}}{%
\pgfpathmoveto{\pgfqpoint{0.000000in}{0.000000in}}%
\pgfpathlineto{\pgfqpoint{0.000000in}{-0.048611in}}%
\pgfusepath{stroke,fill}%
}%
\begin{pgfscope}%
\pgfsys@transformshift{2.623964in}{0.402556in}%
\pgfsys@useobject{currentmarker}{}%
\end{pgfscope}%
\end{pgfscope}%
\begin{pgfscope}%
\pgftext[x=2.623964in,y=0.305334in,,top]{\rmfamily\fontsize{12.000000}{14.400000}\selectfont \(\displaystyle 1.00\)}%
\end{pgfscope}%
\begin{pgfscope}%
\pgfsetbuttcap%
\pgfsetroundjoin%
\definecolor{currentfill}{rgb}{0.000000,0.000000,0.000000}%
\pgfsetfillcolor{currentfill}%
\pgfsetlinewidth{0.803000pt}%
\definecolor{currentstroke}{rgb}{0.000000,0.000000,0.000000}%
\pgfsetstrokecolor{currentstroke}%
\pgfsetdash{}{0pt}%
\pgfsys@defobject{currentmarker}{\pgfqpoint{0.000000in}{-0.048611in}}{\pgfqpoint{0.000000in}{0.000000in}}{%
\pgfpathmoveto{\pgfqpoint{0.000000in}{0.000000in}}%
\pgfpathlineto{\pgfqpoint{0.000000in}{-0.048611in}}%
\pgfusepath{stroke,fill}%
}%
\begin{pgfscope}%
\pgfsys@transformshift{3.167797in}{0.402556in}%
\pgfsys@useobject{currentmarker}{}%
\end{pgfscope}%
\end{pgfscope}%
\begin{pgfscope}%
\pgftext[x=3.167797in,y=0.305334in,,top]{\rmfamily\fontsize{12.000000}{14.400000}\selectfont \(\displaystyle 1.25\)}%
\end{pgfscope}%
\begin{pgfscope}%
\pgfsetbuttcap%
\pgfsetroundjoin%
\definecolor{currentfill}{rgb}{0.000000,0.000000,0.000000}%
\pgfsetfillcolor{currentfill}%
\pgfsetlinewidth{0.803000pt}%
\definecolor{currentstroke}{rgb}{0.000000,0.000000,0.000000}%
\pgfsetstrokecolor{currentstroke}%
\pgfsetdash{}{0pt}%
\pgfsys@defobject{currentmarker}{\pgfqpoint{0.000000in}{-0.048611in}}{\pgfqpoint{0.000000in}{0.000000in}}{%
\pgfpathmoveto{\pgfqpoint{0.000000in}{0.000000in}}%
\pgfpathlineto{\pgfqpoint{0.000000in}{-0.048611in}}%
\pgfusepath{stroke,fill}%
}%
\begin{pgfscope}%
\pgfsys@transformshift{3.711629in}{0.402556in}%
\pgfsys@useobject{currentmarker}{}%
\end{pgfscope}%
\end{pgfscope}%
\begin{pgfscope}%
\pgftext[x=3.711629in,y=0.305334in,,top]{\rmfamily\fontsize{12.000000}{14.400000}\selectfont \(\displaystyle 1.50\)}%
\end{pgfscope}%
\begin{pgfscope}%
\pgfsetbuttcap%
\pgfsetroundjoin%
\definecolor{currentfill}{rgb}{0.000000,0.000000,0.000000}%
\pgfsetfillcolor{currentfill}%
\pgfsetlinewidth{0.803000pt}%
\definecolor{currentstroke}{rgb}{0.000000,0.000000,0.000000}%
\pgfsetstrokecolor{currentstroke}%
\pgfsetdash{}{0pt}%
\pgfsys@defobject{currentmarker}{\pgfqpoint{0.000000in}{-0.048611in}}{\pgfqpoint{0.000000in}{0.000000in}}{%
\pgfpathmoveto{\pgfqpoint{0.000000in}{0.000000in}}%
\pgfpathlineto{\pgfqpoint{0.000000in}{-0.048611in}}%
\pgfusepath{stroke,fill}%
}%
\begin{pgfscope}%
\pgfsys@transformshift{4.255462in}{0.402556in}%
\pgfsys@useobject{currentmarker}{}%
\end{pgfscope}%
\end{pgfscope}%
\begin{pgfscope}%
\pgftext[x=4.255462in,y=0.305334in,,top]{\rmfamily\fontsize{12.000000}{14.400000}\selectfont \(\displaystyle 1.75\)}%
\end{pgfscope}%
\begin{pgfscope}%
\pgfsetbuttcap%
\pgfsetroundjoin%
\definecolor{currentfill}{rgb}{0.000000,0.000000,0.000000}%
\pgfsetfillcolor{currentfill}%
\pgfsetlinewidth{0.803000pt}%
\definecolor{currentstroke}{rgb}{0.000000,0.000000,0.000000}%
\pgfsetstrokecolor{currentstroke}%
\pgfsetdash{}{0pt}%
\pgfsys@defobject{currentmarker}{\pgfqpoint{0.000000in}{-0.048611in}}{\pgfqpoint{0.000000in}{0.000000in}}{%
\pgfpathmoveto{\pgfqpoint{0.000000in}{0.000000in}}%
\pgfpathlineto{\pgfqpoint{0.000000in}{-0.048611in}}%
\pgfusepath{stroke,fill}%
}%
\begin{pgfscope}%
\pgfsys@transformshift{4.799294in}{0.402556in}%
\pgfsys@useobject{currentmarker}{}%
\end{pgfscope}%
\end{pgfscope}%
\begin{pgfscope}%
\pgftext[x=4.799294in,y=0.305334in,,top]{\rmfamily\fontsize{12.000000}{14.400000}\selectfont \(\displaystyle 2.00\)}%
\end{pgfscope}%
\begin{pgfscope}%
\pgfsetbuttcap%
\pgfsetroundjoin%
\definecolor{currentfill}{rgb}{0.000000,0.000000,0.000000}%
\pgfsetfillcolor{currentfill}%
\pgfsetlinewidth{0.803000pt}%
\definecolor{currentstroke}{rgb}{0.000000,0.000000,0.000000}%
\pgfsetstrokecolor{currentstroke}%
\pgfsetdash{}{0pt}%
\pgfsys@defobject{currentmarker}{\pgfqpoint{-0.048611in}{0.000000in}}{\pgfqpoint{0.000000in}{0.000000in}}{%
\pgfpathmoveto{\pgfqpoint{0.000000in}{0.000000in}}%
\pgfpathlineto{\pgfqpoint{-0.048611in}{0.000000in}}%
\pgfusepath{stroke,fill}%
}%
\begin{pgfscope}%
\pgfsys@transformshift{0.448634in}{0.402556in}%
\pgfsys@useobject{currentmarker}{}%
\end{pgfscope}%
\end{pgfscope}%
\begin{pgfscope}%
\pgftext[x=0.149245in,y=0.345015in,left,base]{\rmfamily\fontsize{12.000000}{14.400000}\selectfont \(\displaystyle 0.0\)}%
\end{pgfscope}%
\begin{pgfscope}%
\pgfsetbuttcap%
\pgfsetroundjoin%
\definecolor{currentfill}{rgb}{0.000000,0.000000,0.000000}%
\pgfsetfillcolor{currentfill}%
\pgfsetlinewidth{0.803000pt}%
\definecolor{currentstroke}{rgb}{0.000000,0.000000,0.000000}%
\pgfsetstrokecolor{currentstroke}%
\pgfsetdash{}{0pt}%
\pgfsys@defobject{currentmarker}{\pgfqpoint{-0.048611in}{0.000000in}}{\pgfqpoint{0.000000in}{0.000000in}}{%
\pgfpathmoveto{\pgfqpoint{0.000000in}{0.000000in}}%
\pgfpathlineto{\pgfqpoint{-0.048611in}{0.000000in}}%
\pgfusepath{stroke,fill}%
}%
\begin{pgfscope}%
\pgfsys@transformshift{0.448634in}{0.900397in}%
\pgfsys@useobject{currentmarker}{}%
\end{pgfscope}%
\end{pgfscope}%
\begin{pgfscope}%
\pgftext[x=0.149245in,y=0.842855in,left,base]{\rmfamily\fontsize{12.000000}{14.400000}\selectfont \(\displaystyle 0.2\)}%
\end{pgfscope}%
\begin{pgfscope}%
\pgfsetbuttcap%
\pgfsetroundjoin%
\definecolor{currentfill}{rgb}{0.000000,0.000000,0.000000}%
\pgfsetfillcolor{currentfill}%
\pgfsetlinewidth{0.803000pt}%
\definecolor{currentstroke}{rgb}{0.000000,0.000000,0.000000}%
\pgfsetstrokecolor{currentstroke}%
\pgfsetdash{}{0pt}%
\pgfsys@defobject{currentmarker}{\pgfqpoint{-0.048611in}{0.000000in}}{\pgfqpoint{0.000000in}{0.000000in}}{%
\pgfpathmoveto{\pgfqpoint{0.000000in}{0.000000in}}%
\pgfpathlineto{\pgfqpoint{-0.048611in}{0.000000in}}%
\pgfusepath{stroke,fill}%
}%
\begin{pgfscope}%
\pgfsys@transformshift{0.448634in}{1.398238in}%
\pgfsys@useobject{currentmarker}{}%
\end{pgfscope}%
\end{pgfscope}%
\begin{pgfscope}%
\pgftext[x=0.149245in,y=1.340696in,left,base]{\rmfamily\fontsize{12.000000}{14.400000}\selectfont \(\displaystyle 0.4\)}%
\end{pgfscope}%
\begin{pgfscope}%
\pgfsetbuttcap%
\pgfsetroundjoin%
\definecolor{currentfill}{rgb}{0.000000,0.000000,0.000000}%
\pgfsetfillcolor{currentfill}%
\pgfsetlinewidth{0.803000pt}%
\definecolor{currentstroke}{rgb}{0.000000,0.000000,0.000000}%
\pgfsetstrokecolor{currentstroke}%
\pgfsetdash{}{0pt}%
\pgfsys@defobject{currentmarker}{\pgfqpoint{-0.048611in}{0.000000in}}{\pgfqpoint{0.000000in}{0.000000in}}{%
\pgfpathmoveto{\pgfqpoint{0.000000in}{0.000000in}}%
\pgfpathlineto{\pgfqpoint{-0.048611in}{0.000000in}}%
\pgfusepath{stroke,fill}%
}%
\begin{pgfscope}%
\pgfsys@transformshift{0.448634in}{1.896079in}%
\pgfsys@useobject{currentmarker}{}%
\end{pgfscope}%
\end{pgfscope}%
\begin{pgfscope}%
\pgftext[x=0.149245in,y=1.838537in,left,base]{\rmfamily\fontsize{12.000000}{14.400000}\selectfont \(\displaystyle 0.6\)}%
\end{pgfscope}%
\begin{pgfscope}%
\pgfsetbuttcap%
\pgfsetroundjoin%
\definecolor{currentfill}{rgb}{0.000000,0.000000,0.000000}%
\pgfsetfillcolor{currentfill}%
\pgfsetlinewidth{0.803000pt}%
\definecolor{currentstroke}{rgb}{0.000000,0.000000,0.000000}%
\pgfsetstrokecolor{currentstroke}%
\pgfsetdash{}{0pt}%
\pgfsys@defobject{currentmarker}{\pgfqpoint{-0.048611in}{0.000000in}}{\pgfqpoint{0.000000in}{0.000000in}}{%
\pgfpathmoveto{\pgfqpoint{0.000000in}{0.000000in}}%
\pgfpathlineto{\pgfqpoint{-0.048611in}{0.000000in}}%
\pgfusepath{stroke,fill}%
}%
\begin{pgfscope}%
\pgfsys@transformshift{0.448634in}{2.393919in}%
\pgfsys@useobject{currentmarker}{}%
\end{pgfscope}%
\end{pgfscope}%
\begin{pgfscope}%
\pgftext[x=0.149245in,y=2.336378in,left,base]{\rmfamily\fontsize{12.000000}{14.400000}\selectfont \(\displaystyle 0.8\)}%
\end{pgfscope}%
\begin{pgfscope}%
\pgfsetbuttcap%
\pgfsetroundjoin%
\definecolor{currentfill}{rgb}{0.000000,0.000000,0.000000}%
\pgfsetfillcolor{currentfill}%
\pgfsetlinewidth{0.803000pt}%
\definecolor{currentstroke}{rgb}{0.000000,0.000000,0.000000}%
\pgfsetstrokecolor{currentstroke}%
\pgfsetdash{}{0pt}%
\pgfsys@defobject{currentmarker}{\pgfqpoint{-0.048611in}{0.000000in}}{\pgfqpoint{0.000000in}{0.000000in}}{%
\pgfpathmoveto{\pgfqpoint{0.000000in}{0.000000in}}%
\pgfpathlineto{\pgfqpoint{-0.048611in}{0.000000in}}%
\pgfusepath{stroke,fill}%
}%
\begin{pgfscope}%
\pgfsys@transformshift{0.448634in}{2.891760in}%
\pgfsys@useobject{currentmarker}{}%
\end{pgfscope}%
\end{pgfscope}%
\begin{pgfscope}%
\pgftext[x=0.149245in,y=2.834219in,left,base]{\rmfamily\fontsize{12.000000}{14.400000}\selectfont \(\displaystyle 1.0\)}%
\end{pgfscope}%
\begin{pgfscope}%
\pgfpathrectangle{\pgfqpoint{0.448634in}{0.402556in}}{\pgfqpoint{4.350661in}{2.489204in}} %
\pgfusepath{clip}%
\pgfsetrectcap%
\pgfsetroundjoin%
\pgfsetlinewidth{1.003750pt}%
\definecolor{currentstroke}{rgb}{1.000000,0.388235,0.278431}%
\pgfsetstrokecolor{currentstroke}%
\pgfsetdash{}{0pt}%
\pgfpathmoveto{\pgfqpoint{1.127319in}{2.572074in}}%
\pgfpathlineto{\pgfqpoint{1.159575in}{2.592758in}}%
\pgfpathlineto{\pgfqpoint{1.192763in}{2.611414in}}%
\pgfpathlineto{\pgfqpoint{1.228726in}{2.629126in}}%
\pgfpathlineto{\pgfqpoint{1.267413in}{2.645758in}}%
\pgfpathlineto{\pgfqpoint{1.310846in}{2.661945in}}%
\pgfpathlineto{\pgfqpoint{1.356920in}{2.676740in}}%
\pgfpathlineto{\pgfqpoint{1.407680in}{2.690702in}}%
\pgfpathlineto{\pgfqpoint{1.463094in}{2.703640in}}%
\pgfpathlineto{\pgfqpoint{1.525273in}{2.715813in}}%
\pgfpathlineto{\pgfqpoint{1.594199in}{2.726937in}}%
\pgfpathlineto{\pgfqpoint{1.669843in}{2.736808in}}%
\pgfpathlineto{\pgfqpoint{1.752172in}{2.745271in}}%
\pgfpathlineto{\pgfqpoint{1.843325in}{2.752344in}}%
\pgfpathlineto{\pgfqpoint{1.941103in}{2.757656in}}%
\pgfpathlineto{\pgfqpoint{2.043301in}{2.760987in}}%
\pgfpathlineto{\pgfqpoint{2.147710in}{2.762199in}}%
\pgfpathlineto{\pgfqpoint{2.249945in}{2.761215in}}%
\pgfpathlineto{\pgfqpoint{2.345620in}{2.758145in}}%
\pgfpathlineto{\pgfqpoint{2.432525in}{2.753210in}}%
\pgfpathlineto{\pgfqpoint{2.508451in}{2.746766in}}%
\pgfpathlineto{\pgfqpoint{2.573368in}{2.739156in}}%
\pgfpathlineto{\pgfqpoint{2.629410in}{2.730451in}}%
\pgfpathlineto{\pgfqpoint{2.676543in}{2.720985in}}%
\pgfpathlineto{\pgfqpoint{2.716874in}{2.710666in}}%
\pgfpathlineto{\pgfqpoint{2.750366in}{2.699848in}}%
\pgfpathlineto{\pgfqpoint{2.779059in}{2.688192in}}%
\pgfpathlineto{\pgfqpoint{2.802882in}{2.676004in}}%
\pgfpathlineto{\pgfqpoint{2.821842in}{2.663819in}}%
\pgfpathlineto{\pgfqpoint{2.837815in}{2.650886in}}%
\pgfpathlineto{\pgfqpoint{2.850736in}{2.637564in}}%
\pgfpathlineto{\pgfqpoint{2.860694in}{2.624398in}}%
\pgfpathlineto{\pgfqpoint{2.869084in}{2.609873in}}%
\pgfpathlineto{\pgfqpoint{2.875698in}{2.594192in}}%
\pgfpathlineto{\pgfqpoint{2.881035in}{2.575255in}}%
\pgfpathlineto{\pgfqpoint{2.884200in}{2.555685in}}%
\pgfpathlineto{\pgfqpoint{2.885619in}{2.533351in}}%
\pgfpathlineto{\pgfqpoint{2.885038in}{2.505987in}}%
\pgfpathlineto{\pgfqpoint{2.882112in}{2.473807in}}%
\pgfpathlineto{\pgfqpoint{2.875657in}{2.429620in}}%
\pgfpathlineto{\pgfqpoint{2.863489in}{2.363873in}}%
\pgfpathlineto{\pgfqpoint{2.821102in}{2.142619in}}%
\pgfpathlineto{\pgfqpoint{2.804859in}{2.042271in}}%
\pgfpathlineto{\pgfqpoint{2.790421in}{1.939040in}}%
\pgfpathlineto{\pgfqpoint{2.777207in}{1.828054in}}%
\pgfpathlineto{\pgfqpoint{2.765338in}{1.709349in}}%
\pgfpathlineto{\pgfqpoint{2.754471in}{1.578010in}}%
\pgfpathlineto{\pgfqpoint{2.744640in}{1.431580in}}%
\pgfpathlineto{\pgfqpoint{2.735914in}{1.267598in}}%
\pgfpathlineto{\pgfqpoint{2.728277in}{1.081114in}}%
\pgfpathlineto{\pgfqpoint{2.721437in}{0.857223in}}%
\pgfpathlineto{\pgfqpoint{2.711961in}{0.541290in}}%
\pgfpathlineto{\pgfqpoint{2.708250in}{0.491694in}}%
\pgfpathlineto{\pgfqpoint{2.703951in}{0.462246in}}%
\pgfpathlineto{\pgfqpoint{2.699504in}{0.445599in}}%
\pgfpathlineto{\pgfqpoint{2.694517in}{0.434563in}}%
\pgfpathlineto{\pgfqpoint{2.688942in}{0.426947in}}%
\pgfpathlineto{\pgfqpoint{2.681980in}{0.421009in}}%
\pgfpathlineto{\pgfqpoint{2.672064in}{0.415948in}}%
\pgfpathlineto{\pgfqpoint{2.659429in}{0.412247in}}%
\pgfpathlineto{\pgfqpoint{2.640044in}{0.409163in}}%
\pgfpathlineto{\pgfqpoint{2.607490in}{0.406692in}}%
\pgfpathlineto{\pgfqpoint{2.548779in}{0.404894in}}%
\pgfpathlineto{\pgfqpoint{2.422615in}{0.403701in}}%
\pgfpathlineto{\pgfqpoint{2.026705in}{0.403016in}}%
\pgfpathlineto{\pgfqpoint{0.623617in}{0.403253in}}%
\pgfpathlineto{\pgfqpoint{0.477880in}{0.404742in}}%
\pgfpathlineto{\pgfqpoint{0.458368in}{0.406382in}}%
\pgfpathlineto{\pgfqpoint{0.452304in}{0.408937in}}%
\pgfpathlineto{\pgfqpoint{0.450213in}{0.413215in}}%
\pgfpathlineto{\pgfqpoint{0.449165in}{0.423080in}}%
\pgfpathlineto{\pgfqpoint{0.448735in}{0.465392in}}%
\pgfpathlineto{\pgfqpoint{0.448637in}{0.983146in}}%
\pgfpathlineto{\pgfqpoint{0.448652in}{2.889876in}}%
\pgfpathlineto{\pgfqpoint{0.448652in}{2.889876in}}%
\pgfusepath{stroke}%
\end{pgfscope}%
\begin{pgfscope}%
\pgfpathrectangle{\pgfqpoint{0.448634in}{0.402556in}}{\pgfqpoint{4.350661in}{2.489204in}} %
\pgfusepath{clip}%
\pgfsetrectcap%
\pgfsetroundjoin%
\pgfsetlinewidth{1.003750pt}%
\definecolor{currentstroke}{rgb}{1.000000,0.388235,0.278431}%
\pgfsetstrokecolor{currentstroke}%
\pgfsetdash{}{0pt}%
\pgfpathmoveto{\pgfqpoint{0.448634in}{2.896245in}}%
\pgfpathlineto{\pgfqpoint{0.448593in}{0.407043in}}%
\pgfpathlineto{\pgfqpoint{0.448593in}{0.407043in}}%
\pgfusepath{stroke}%
\end{pgfscope}%
\begin{pgfscope}%
\pgfpathrectangle{\pgfqpoint{0.448634in}{0.402556in}}{\pgfqpoint{4.350661in}{2.489204in}} %
\pgfusepath{clip}%
\pgfsetrectcap%
\pgfsetroundjoin%
\pgfsetlinewidth{1.003750pt}%
\definecolor{currentstroke}{rgb}{1.000000,0.388235,0.278431}%
\pgfsetstrokecolor{currentstroke}%
\pgfsetdash{}{0pt}%
\pgfpathmoveto{\pgfqpoint{0.576853in}{1.760817in}}%
\pgfpathlineto{\pgfqpoint{0.569394in}{1.840010in}}%
\pgfpathlineto{\pgfqpoint{0.563209in}{1.929338in}}%
\pgfpathlineto{\pgfqpoint{0.558592in}{2.028764in}}%
\pgfpathlineto{\pgfqpoint{0.555985in}{2.133265in}}%
\pgfpathlineto{\pgfqpoint{0.555566in}{2.237808in}}%
\pgfpathlineto{\pgfqpoint{0.557371in}{2.337352in}}%
\pgfpathlineto{\pgfqpoint{0.561096in}{2.424366in}}%
\pgfpathlineto{\pgfqpoint{0.566403in}{2.498791in}}%
\pgfpathlineto{\pgfqpoint{0.572909in}{2.560570in}}%
\pgfpathlineto{\pgfqpoint{0.580458in}{2.612119in}}%
\pgfpathlineto{\pgfqpoint{0.589086in}{2.655816in}}%
\pgfpathlineto{\pgfqpoint{0.598406in}{2.691589in}}%
\pgfpathlineto{\pgfqpoint{0.608613in}{2.721757in}}%
\pgfpathlineto{\pgfqpoint{0.619241in}{2.746278in}}%
\pgfpathlineto{\pgfqpoint{0.630817in}{2.767339in}}%
\pgfpathlineto{\pgfqpoint{0.642975in}{2.784884in}}%
\pgfpathlineto{\pgfqpoint{0.656813in}{2.800712in}}%
\pgfpathlineto{\pgfqpoint{0.672197in}{2.814549in}}%
\pgfpathlineto{\pgfqpoint{0.688853in}{2.826301in}}%
\pgfpathlineto{\pgfqpoint{0.706461in}{2.836076in}}%
\pgfpathlineto{\pgfqpoint{0.726804in}{2.844875in}}%
\pgfpathlineto{\pgfqpoint{0.751866in}{2.853203in}}%
\pgfpathlineto{\pgfqpoint{0.781631in}{2.860547in}}%
\pgfpathlineto{\pgfqpoint{0.818168in}{2.867054in}}%
\pgfpathlineto{\pgfqpoint{0.863581in}{2.872685in}}%
\pgfpathlineto{\pgfqpoint{0.922161in}{2.877518in}}%
\pgfpathlineto{\pgfqpoint{1.000391in}{2.881567in}}%
\pgfpathlineto{\pgfqpoint{1.111294in}{2.884881in}}%
\pgfpathlineto{\pgfqpoint{1.274428in}{2.887367in}}%
\pgfpathlineto{\pgfqpoint{1.552865in}{2.889263in}}%
\pgfpathlineto{\pgfqpoint{2.107573in}{2.890457in}}%
\pgfpathlineto{\pgfqpoint{3.343161in}{2.890573in}}%
\pgfpathlineto{\pgfqpoint{4.043615in}{2.888941in}}%
\pgfpathlineto{\pgfqpoint{4.289417in}{2.886404in}}%
\pgfpathlineto{\pgfqpoint{4.413375in}{2.883093in}}%
\pgfpathlineto{\pgfqpoint{4.489424in}{2.878997in}}%
\pgfpathlineto{\pgfqpoint{4.541451in}{2.874081in}}%
\pgfpathlineto{\pgfqpoint{4.578100in}{2.868470in}}%
\pgfpathlineto{\pgfqpoint{4.605818in}{2.862092in}}%
\pgfpathlineto{\pgfqpoint{4.626725in}{2.855245in}}%
\pgfpathlineto{\pgfqpoint{4.644925in}{2.847018in}}%
\pgfpathlineto{\pgfqpoint{4.660241in}{2.837590in}}%
\pgfpathlineto{\pgfqpoint{4.672623in}{2.827468in}}%
\pgfpathlineto{\pgfqpoint{4.683751in}{2.815592in}}%
\pgfpathlineto{\pgfqpoint{4.693406in}{2.802135in}}%
\pgfpathlineto{\pgfqpoint{4.702740in}{2.785343in}}%
\pgfpathlineto{\pgfqpoint{4.711277in}{2.765194in}}%
\pgfpathlineto{\pgfqpoint{4.719482in}{2.739484in}}%
\pgfpathlineto{\pgfqpoint{4.726293in}{2.710657in}}%
\pgfpathlineto{\pgfqpoint{4.733259in}{2.671643in}}%
\pgfpathlineto{\pgfqpoint{4.739604in}{2.622396in}}%
\pgfpathlineto{\pgfqpoint{4.745236in}{2.560504in}}%
\pgfpathlineto{\pgfqpoint{4.750164in}{2.481052in}}%
\pgfpathlineto{\pgfqpoint{4.754367in}{2.376618in}}%
\pgfpathlineto{\pgfqpoint{4.757443in}{2.242249in}}%
\pgfpathlineto{\pgfqpoint{4.758977in}{2.075483in}}%
\pgfpathlineto{\pgfqpoint{4.758447in}{1.888795in}}%
\pgfpathlineto{\pgfqpoint{4.755756in}{1.707111in}}%
\pgfpathlineto{\pgfqpoint{4.750925in}{1.532957in}}%
\pgfpathlineto{\pgfqpoint{4.744785in}{1.398726in}}%
\pgfpathlineto{\pgfqpoint{4.737575in}{1.289516in}}%
\pgfpathlineto{\pgfqpoint{4.728714in}{1.190470in}}%
\pgfpathlineto{\pgfqpoint{4.719652in}{1.116521in}}%
\pgfpathlineto{\pgfqpoint{4.710036in}{1.055276in}}%
\pgfpathlineto{\pgfqpoint{4.699503in}{1.001861in}}%
\pgfpathlineto{\pgfqpoint{4.689040in}{0.958690in}}%
\pgfpathlineto{\pgfqpoint{4.677219in}{0.918600in}}%
\pgfpathlineto{\pgfqpoint{4.664034in}{0.881749in}}%
\pgfpathlineto{\pgfqpoint{4.650584in}{0.850492in}}%
\pgfpathlineto{\pgfqpoint{4.636303in}{0.822570in}}%
\pgfpathlineto{\pgfqpoint{4.620207in}{0.795974in}}%
\pgfpathlineto{\pgfqpoint{4.603640in}{0.772901in}}%
\pgfpathlineto{\pgfqpoint{4.585488in}{0.751446in}}%
\pgfpathlineto{\pgfqpoint{4.565874in}{0.731749in}}%
\pgfpathlineto{\pgfqpoint{4.544964in}{0.713879in}}%
\pgfpathlineto{\pgfqpoint{4.522958in}{0.697824in}}%
\pgfpathlineto{\pgfqpoint{4.496157in}{0.681290in}}%
\pgfpathlineto{\pgfqpoint{4.470397in}{0.667953in}}%
\pgfpathlineto{\pgfqpoint{4.439961in}{0.654509in}}%
\pgfpathlineto{\pgfqpoint{4.406841in}{0.642281in}}%
\pgfpathlineto{\pgfqpoint{4.369009in}{0.630748in}}%
\pgfpathlineto{\pgfqpoint{4.326489in}{0.620226in}}%
\pgfpathlineto{\pgfqpoint{4.279327in}{0.610949in}}%
\pgfpathlineto{\pgfqpoint{4.227576in}{0.603085in}}%
\pgfpathlineto{\pgfqpoint{4.173450in}{0.597063in}}%
\pgfpathlineto{\pgfqpoint{4.110511in}{0.592203in}}%
\pgfpathlineto{\pgfqpoint{4.047471in}{0.589537in}}%
\pgfpathlineto{\pgfqpoint{3.977867in}{0.588624in}}%
\pgfpathlineto{\pgfqpoint{3.906093in}{0.589934in}}%
\pgfpathlineto{\pgfqpoint{3.834377in}{0.593496in}}%
\pgfpathlineto{\pgfqpoint{3.767120in}{0.599067in}}%
\pgfpathlineto{\pgfqpoint{3.704364in}{0.606392in}}%
\pgfpathlineto{\pgfqpoint{3.678516in}{0.610510in}}%
\pgfpathlineto{\pgfqpoint{3.620438in}{0.620500in}}%
\pgfpathlineto{\pgfqpoint{3.586319in}{0.628207in}}%
\pgfpathlineto{\pgfqpoint{3.495240in}{0.652428in}}%
\pgfpathlineto{\pgfqpoint{3.451528in}{0.667583in}}%
\pgfpathlineto{\pgfqpoint{3.408538in}{0.685220in}}%
\pgfpathlineto{\pgfqpoint{3.374594in}{0.702001in}}%
\pgfpathlineto{\pgfqpoint{3.345407in}{0.718682in}}%
\pgfpathlineto{\pgfqpoint{3.315236in}{0.738520in}}%
\pgfpathlineto{\pgfqpoint{3.288127in}{0.759290in}}%
\pgfpathlineto{\pgfqpoint{3.264004in}{0.780551in}}%
\pgfpathlineto{\pgfqpoint{3.241208in}{0.803648in}}%
\pgfpathlineto{\pgfqpoint{3.219894in}{0.828530in}}%
\pgfpathlineto{\pgfqpoint{3.200189in}{0.855091in}}%
\pgfpathlineto{\pgfqpoint{3.182177in}{0.883182in}}%
\pgfpathlineto{\pgfqpoint{3.165906in}{0.912633in}}%
\pgfpathlineto{\pgfqpoint{3.150351in}{0.945448in}}%
\pgfpathlineto{\pgfqpoint{3.136682in}{0.979345in}}%
\pgfpathlineto{\pgfqpoint{3.124073in}{1.016460in}}%
\pgfpathlineto{\pgfqpoint{3.112834in}{1.056769in}}%
\pgfpathlineto{\pgfqpoint{3.103046in}{1.100146in}}%
\pgfpathlineto{\pgfqpoint{3.095343in}{1.144071in}}%
\pgfpathlineto{\pgfqpoint{3.089208in}{1.190837in}}%
\pgfpathlineto{\pgfqpoint{3.084595in}{1.242838in}}%
\pgfpathlineto{\pgfqpoint{3.082137in}{1.295031in}}%
\pgfpathlineto{\pgfqpoint{3.081687in}{1.349787in}}%
\pgfpathlineto{\pgfqpoint{3.083451in}{1.406998in}}%
\pgfpathlineto{\pgfqpoint{3.087181in}{1.461589in}}%
\pgfpathlineto{\pgfqpoint{3.093485in}{1.520888in}}%
\pgfpathlineto{\pgfqpoint{3.101823in}{1.577334in}}%
\pgfpathlineto{\pgfqpoint{3.111930in}{1.630856in}}%
\pgfpathlineto{\pgfqpoint{3.124690in}{1.686208in}}%
\pgfpathlineto{\pgfqpoint{3.139178in}{1.738395in}}%
\pgfpathlineto{\pgfqpoint{3.155145in}{1.787366in}}%
\pgfpathlineto{\pgfqpoint{3.172353in}{1.833085in}}%
\pgfpathlineto{\pgfqpoint{3.191618in}{1.877716in}}%
\pgfpathlineto{\pgfqpoint{3.214026in}{1.923261in}}%
\pgfpathlineto{\pgfqpoint{3.236214in}{1.963157in}}%
\pgfpathlineto{\pgfqpoint{3.260178in}{2.001684in}}%
\pgfpathlineto{\pgfqpoint{3.285814in}{2.038776in}}%
\pgfpathlineto{\pgfqpoint{3.314415in}{2.076285in}}%
\pgfpathlineto{\pgfqpoint{3.348944in}{2.117711in}}%
\pgfpathlineto{\pgfqpoint{3.417133in}{2.198022in}}%
\pgfpathlineto{\pgfqpoint{3.426053in}{2.212128in}}%
\pgfpathlineto{\pgfqpoint{3.430798in}{2.223297in}}%
\pgfpathlineto{\pgfqpoint{3.432034in}{2.230603in}}%
\pgfpathlineto{\pgfqpoint{3.430773in}{2.237856in}}%
\pgfpathlineto{\pgfqpoint{3.426621in}{2.243526in}}%
\pgfpathlineto{\pgfqpoint{3.420908in}{2.247084in}}%
\pgfpathlineto{\pgfqpoint{3.412501in}{2.249583in}}%
\pgfpathlineto{\pgfqpoint{3.399499in}{2.250689in}}%
\pgfpathlineto{\pgfqpoint{3.384305in}{2.249671in}}%
\pgfpathlineto{\pgfqpoint{3.364985in}{2.246098in}}%
\pgfpathlineto{\pgfqpoint{3.341804in}{2.239342in}}%
\pgfpathlineto{\pgfqpoint{3.317109in}{2.229682in}}%
\pgfpathlineto{\pgfqpoint{3.291104in}{2.216986in}}%
\pgfpathlineto{\pgfqpoint{3.265928in}{2.202261in}}%
\pgfpathlineto{\pgfqpoint{3.239805in}{2.184361in}}%
\pgfpathlineto{\pgfqpoint{3.214775in}{2.164519in}}%
\pgfpathlineto{\pgfqpoint{3.190900in}{2.142893in}}%
\pgfpathlineto{\pgfqpoint{3.166657in}{2.117912in}}%
\pgfpathlineto{\pgfqpoint{3.143835in}{2.091233in}}%
\pgfpathlineto{\pgfqpoint{3.121079in}{2.061107in}}%
\pgfpathlineto{\pgfqpoint{3.099952in}{2.029463in}}%
\pgfpathlineto{\pgfqpoint{3.079251in}{1.994406in}}%
\pgfpathlineto{\pgfqpoint{3.059218in}{1.955915in}}%
\pgfpathlineto{\pgfqpoint{3.040058in}{1.914015in}}%
\pgfpathlineto{\pgfqpoint{3.022809in}{1.871041in}}%
\pgfpathlineto{\pgfqpoint{3.005790in}{1.822536in}}%
\pgfpathlineto{\pgfqpoint{2.990067in}{1.770819in}}%
\pgfpathlineto{\pgfqpoint{2.975708in}{1.715979in}}%
\pgfpathlineto{\pgfqpoint{2.962284in}{1.655680in}}%
\pgfpathlineto{\pgfqpoint{2.950496in}{1.592386in}}%
\pgfpathlineto{\pgfqpoint{2.940383in}{1.526185in}}%
\pgfpathlineto{\pgfqpoint{2.931745in}{1.454681in}}%
\pgfpathlineto{\pgfqpoint{2.925082in}{1.380399in}}%
\pgfpathlineto{\pgfqpoint{2.920647in}{1.305899in}}%
\pgfpathlineto{\pgfqpoint{2.918444in}{1.231270in}}%
\pgfpathlineto{\pgfqpoint{2.918545in}{1.159087in}}%
\pgfpathlineto{\pgfqpoint{2.920787in}{1.091931in}}%
\pgfpathlineto{\pgfqpoint{2.925177in}{1.027412in}}%
\pgfpathlineto{\pgfqpoint{2.931192in}{0.970580in}}%
\pgfpathlineto{\pgfqpoint{2.938760in}{0.919034in}}%
\pgfpathlineto{\pgfqpoint{2.947651in}{0.872852in}}%
\pgfpathlineto{\pgfqpoint{2.958213in}{0.829714in}}%
\pgfpathlineto{\pgfqpoint{2.969670in}{0.792114in}}%
\pgfpathlineto{\pgfqpoint{2.982463in}{0.757773in}}%
\pgfpathlineto{\pgfqpoint{2.996425in}{0.726812in}}%
\pgfpathlineto{\pgfqpoint{3.011299in}{0.699300in}}%
\pgfpathlineto{\pgfqpoint{3.026739in}{0.675225in}}%
\pgfpathlineto{\pgfqpoint{3.043828in}{0.652656in}}%
\pgfpathlineto{\pgfqpoint{3.062495in}{0.631788in}}%
\pgfpathlineto{\pgfqpoint{3.082602in}{0.612753in}}%
\pgfpathlineto{\pgfqpoint{3.103961in}{0.595592in}}%
\pgfpathlineto{\pgfqpoint{3.128268in}{0.579069in}}%
\pgfpathlineto{\pgfqpoint{3.153537in}{0.564554in}}%
\pgfpathlineto{\pgfqpoint{3.181571in}{0.550952in}}%
\pgfpathlineto{\pgfqpoint{3.214371in}{0.537647in}}%
\pgfpathlineto{\pgfqpoint{3.249846in}{0.525712in}}%
\pgfpathlineto{\pgfqpoint{3.290011in}{0.514571in}}%
\pgfpathlineto{\pgfqpoint{3.334820in}{0.504423in}}%
\pgfpathlineto{\pgfqpoint{3.386372in}{0.494999in}}%
\pgfpathlineto{\pgfqpoint{3.446798in}{0.486257in}}%
\pgfpathlineto{\pgfqpoint{3.518243in}{0.478282in}}%
\pgfpathlineto{\pgfqpoint{3.600685in}{0.471409in}}%
\pgfpathlineto{\pgfqpoint{3.696268in}{0.465713in}}%
\pgfpathlineto{\pgfqpoint{3.807144in}{0.461369in}}%
\pgfpathlineto{\pgfqpoint{3.933291in}{0.458719in}}%
\pgfpathlineto{\pgfqpoint{4.063808in}{0.458211in}}%
\pgfpathlineto{\pgfqpoint{4.187792in}{0.459914in}}%
\pgfpathlineto{\pgfqpoint{4.294335in}{0.463521in}}%
\pgfpathlineto{\pgfqpoint{4.381234in}{0.468574in}}%
\pgfpathlineto{\pgfqpoint{4.450636in}{0.474701in}}%
\pgfpathlineto{\pgfqpoint{4.506850in}{0.481799in}}%
\pgfpathlineto{\pgfqpoint{4.552009in}{0.489658in}}%
\pgfpathlineto{\pgfqpoint{4.588239in}{0.498115in}}%
\pgfpathlineto{\pgfqpoint{4.617656in}{0.507110in}}%
\pgfpathlineto{\pgfqpoint{4.642328in}{0.516843in}}%
\pgfpathlineto{\pgfqpoint{4.664194in}{0.527940in}}%
\pgfpathlineto{\pgfqpoint{4.681238in}{0.538945in}}%
\pgfpathlineto{\pgfqpoint{4.697164in}{0.551953in}}%
\pgfpathlineto{\pgfqpoint{4.710076in}{0.565289in}}%
\pgfpathlineto{\pgfqpoint{4.721578in}{0.580218in}}%
\pgfpathlineto{\pgfqpoint{4.731557in}{0.596521in}}%
\pgfpathlineto{\pgfqpoint{4.741000in}{0.616134in}}%
\pgfpathlineto{\pgfqpoint{4.749521in}{0.639027in}}%
\pgfpathlineto{\pgfqpoint{4.757522in}{0.667450in}}%
\pgfpathlineto{\pgfqpoint{4.764572in}{0.701345in}}%
\pgfpathlineto{\pgfqpoint{4.770840in}{0.743043in}}%
\pgfpathlineto{\pgfqpoint{4.776327in}{0.794934in}}%
\pgfpathlineto{\pgfqpoint{4.781278in}{0.864398in}}%
\pgfpathlineto{\pgfqpoint{4.785468in}{0.956371in}}%
\pgfpathlineto{\pgfqpoint{4.789000in}{1.085745in}}%
\pgfpathlineto{\pgfqpoint{4.791852in}{1.277385in}}%
\pgfpathlineto{\pgfqpoint{4.793959in}{1.581057in}}%
\pgfpathlineto{\pgfqpoint{4.794962in}{2.071429in}}%
\pgfpathlineto{\pgfqpoint{4.793967in}{2.559311in}}%
\pgfpathlineto{\pgfqpoint{4.791733in}{2.745981in}}%
\pgfpathlineto{\pgfqpoint{4.788955in}{2.818091in}}%
\pgfpathlineto{\pgfqpoint{4.785731in}{2.850227in}}%
\pgfpathlineto{\pgfqpoint{4.781879in}{2.867057in}}%
\pgfpathlineto{\pgfqpoint{4.777744in}{2.875780in}}%
\pgfpathlineto{\pgfqpoint{4.773097in}{2.880982in}}%
\pgfpathlineto{\pgfqpoint{4.767363in}{2.884504in}}%
\pgfpathlineto{\pgfqpoint{4.756853in}{2.887622in}}%
\pgfpathlineto{\pgfqpoint{4.739548in}{2.889639in}}%
\pgfpathlineto{\pgfqpoint{4.704762in}{2.890882in}}%
\pgfpathlineto{\pgfqpoint{4.602524in}{2.891538in}}%
\pgfpathlineto{\pgfqpoint{3.952100in}{2.891742in}}%
\pgfpathlineto{\pgfqpoint{0.617321in}{2.890753in}}%
\pgfpathlineto{\pgfqpoint{0.549910in}{2.888858in}}%
\pgfpathlineto{\pgfqpoint{0.521735in}{2.886179in}}%
\pgfpathlineto{\pgfqpoint{0.504666in}{2.882389in}}%
\pgfpathlineto{\pgfqpoint{0.494501in}{2.878011in}}%
\pgfpathlineto{\pgfqpoint{0.487180in}{2.872667in}}%
\pgfpathlineto{\pgfqpoint{0.481152in}{2.865519in}}%
\pgfpathlineto{\pgfqpoint{0.475664in}{2.854804in}}%
\pgfpathlineto{\pgfqpoint{0.471318in}{2.840737in}}%
\pgfpathlineto{\pgfqpoint{0.467301in}{2.818823in}}%
\pgfpathlineto{\pgfqpoint{0.463927in}{2.786700in}}%
\pgfpathlineto{\pgfqpoint{0.460918in}{2.734544in}}%
\pgfpathlineto{\pgfqpoint{0.458363in}{2.647473in}}%
\pgfpathlineto{\pgfqpoint{0.456575in}{2.523031in}}%
\pgfpathlineto{\pgfqpoint{0.456575in}{2.523031in}}%
\pgfusepath{stroke}%
\end{pgfscope}%
\begin{pgfscope}%
\pgfpathrectangle{\pgfqpoint{0.448634in}{0.402556in}}{\pgfqpoint{4.350661in}{2.489204in}} %
\pgfusepath{clip}%
\pgfsetrectcap%
\pgfsetroundjoin%
\pgfsetlinewidth{1.003750pt}%
\definecolor{currentstroke}{rgb}{1.000000,0.388235,0.278431}%
\pgfsetstrokecolor{currentstroke}%
\pgfsetdash{}{0pt}%
\pgfpathmoveto{\pgfqpoint{0.456424in}{1.370137in}}%
\pgfpathlineto{\pgfqpoint{0.459610in}{1.118755in}}%
\pgfpathlineto{\pgfqpoint{0.463695in}{0.962007in}}%
\pgfpathlineto{\pgfqpoint{0.468519in}{0.857610in}}%
\pgfpathlineto{\pgfqpoint{0.474082in}{0.783210in}}%
\pgfpathlineto{\pgfqpoint{0.480226in}{0.728906in}}%
\pgfpathlineto{\pgfqpoint{0.486970in}{0.687306in}}%
\pgfpathlineto{\pgfqpoint{0.494537in}{0.653558in}}%
\pgfpathlineto{\pgfqpoint{0.503107in}{0.625355in}}%
\pgfpathlineto{\pgfqpoint{0.512193in}{0.602749in}}%
\pgfpathlineto{\pgfqpoint{0.522200in}{0.583508in}}%
\pgfpathlineto{\pgfqpoint{0.534108in}{0.565743in}}%
\pgfpathlineto{\pgfqpoint{0.546263in}{0.551507in}}%
\pgfpathlineto{\pgfqpoint{0.559728in}{0.538907in}}%
\pgfpathlineto{\pgfqpoint{0.576129in}{0.526693in}}%
\pgfpathlineto{\pgfqpoint{0.595483in}{0.515351in}}%
\pgfpathlineto{\pgfqpoint{0.617681in}{0.505147in}}%
\pgfpathlineto{\pgfqpoint{0.642568in}{0.496153in}}%
\pgfpathlineto{\pgfqpoint{0.672126in}{0.487778in}}%
\pgfpathlineto{\pgfqpoint{0.708443in}{0.479824in}}%
\pgfpathlineto{\pgfqpoint{0.753649in}{0.472325in}}%
\pgfpathlineto{\pgfqpoint{0.807717in}{0.465660in}}%
\pgfpathlineto{\pgfqpoint{0.877116in}{0.459475in}}%
\pgfpathlineto{\pgfqpoint{0.961828in}{0.454230in}}%
\pgfpathlineto{\pgfqpoint{1.068351in}{0.449916in}}%
\pgfpathlineto{\pgfqpoint{1.201018in}{0.446839in}}%
\pgfpathlineto{\pgfqpoint{1.357637in}{0.445481in}}%
\pgfpathlineto{\pgfqpoint{1.525135in}{0.446232in}}%
\pgfpathlineto{\pgfqpoint{1.686088in}{0.449142in}}%
\pgfpathlineto{\pgfqpoint{1.823074in}{0.453747in}}%
\pgfpathlineto{\pgfqpoint{1.938245in}{0.459764in}}%
\pgfpathlineto{\pgfqpoint{2.031582in}{0.466759in}}%
\pgfpathlineto{\pgfqpoint{2.109580in}{0.474745in}}%
\pgfpathlineto{\pgfqpoint{2.174384in}{0.483535in}}%
\pgfpathlineto{\pgfqpoint{2.228139in}{0.492940in}}%
\pgfpathlineto{\pgfqpoint{2.275119in}{0.503356in}}%
\pgfpathlineto{\pgfqpoint{2.315282in}{0.514501in}}%
\pgfpathlineto{\pgfqpoint{2.350698in}{0.526659in}}%
\pgfpathlineto{\pgfqpoint{2.381320in}{0.539536in}}%
\pgfpathlineto{\pgfqpoint{2.407164in}{0.552659in}}%
\pgfpathlineto{\pgfqpoint{2.430226in}{0.566639in}}%
\pgfpathlineto{\pgfqpoint{2.452282in}{0.582602in}}%
\pgfpathlineto{\pgfqpoint{2.471391in}{0.599069in}}%
\pgfpathlineto{\pgfqpoint{2.489240in}{0.617293in}}%
\pgfpathlineto{\pgfqpoint{2.505678in}{0.637180in}}%
\pgfpathlineto{\pgfqpoint{2.520620in}{0.658557in}}%
\pgfpathlineto{\pgfqpoint{2.535213in}{0.683314in}}%
\pgfpathlineto{\pgfqpoint{2.549115in}{0.711484in}}%
\pgfpathlineto{\pgfqpoint{2.562091in}{0.743004in}}%
\pgfpathlineto{\pgfqpoint{2.574020in}{0.777751in}}%
\pgfpathlineto{\pgfqpoint{2.585502in}{0.817970in}}%
\pgfpathlineto{\pgfqpoint{2.596809in}{0.866038in}}%
\pgfpathlineto{\pgfqpoint{2.607562in}{0.921948in}}%
\pgfpathlineto{\pgfqpoint{2.617925in}{0.988098in}}%
\pgfpathlineto{\pgfqpoint{2.627958in}{1.066918in}}%
\pgfpathlineto{\pgfqpoint{2.637941in}{1.163320in}}%
\pgfpathlineto{\pgfqpoint{2.648424in}{1.287199in}}%
\pgfpathlineto{\pgfqpoint{2.660103in}{1.453438in}}%
\pgfpathlineto{\pgfqpoint{2.674773in}{1.696801in}}%
\pgfpathlineto{\pgfqpoint{2.687716in}{1.945279in}}%
\pgfpathlineto{\pgfqpoint{2.692670in}{2.079573in}}%
\pgfpathlineto{\pgfqpoint{2.693829in}{2.166682in}}%
\pgfpathlineto{\pgfqpoint{2.692565in}{2.233870in}}%
\pgfpathlineto{\pgfqpoint{2.689436in}{2.286015in}}%
\pgfpathlineto{\pgfqpoint{2.684859in}{2.327999in}}%
\pgfpathlineto{\pgfqpoint{2.678725in}{2.364664in}}%
\pgfpathlineto{\pgfqpoint{2.671356in}{2.395897in}}%
\pgfpathlineto{\pgfqpoint{2.662489in}{2.423981in}}%
\pgfpathlineto{\pgfqpoint{2.652361in}{2.448778in}}%
\pgfpathlineto{\pgfqpoint{2.641365in}{2.470245in}}%
\pgfpathlineto{\pgfqpoint{2.628643in}{2.490425in}}%
\pgfpathlineto{\pgfqpoint{2.614279in}{2.509106in}}%
\pgfpathlineto{\pgfqpoint{2.598443in}{2.526159in}}%
\pgfpathlineto{\pgfqpoint{2.579590in}{2.543005in}}%
\pgfpathlineto{\pgfqpoint{2.559532in}{2.557923in}}%
\pgfpathlineto{\pgfqpoint{2.536602in}{2.572183in}}%
\pgfpathlineto{\pgfqpoint{2.510850in}{2.585538in}}%
\pgfpathlineto{\pgfqpoint{2.482360in}{2.597837in}}%
\pgfpathlineto{\pgfqpoint{2.449134in}{2.609683in}}%
\pgfpathlineto{\pgfqpoint{2.411184in}{2.620696in}}%
\pgfpathlineto{\pgfqpoint{2.368552in}{2.630606in}}%
\pgfpathlineto{\pgfqpoint{2.321294in}{2.639221in}}%
\pgfpathlineto{\pgfqpoint{2.269467in}{2.646399in}}%
\pgfpathlineto{\pgfqpoint{2.210954in}{2.652193in}}%
\pgfpathlineto{\pgfqpoint{2.147967in}{2.656153in}}%
\pgfpathlineto{\pgfqpoint{2.080556in}{2.658135in}}%
\pgfpathlineto{\pgfqpoint{2.010948in}{2.657971in}}%
\pgfpathlineto{\pgfqpoint{1.939195in}{2.655572in}}%
\pgfpathlineto{\pgfqpoint{1.867527in}{2.650913in}}%
\pgfpathlineto{\pgfqpoint{1.798171in}{2.644140in}}%
\pgfpathlineto{\pgfqpoint{1.733341in}{2.635606in}}%
\pgfpathlineto{\pgfqpoint{1.673075in}{2.625521in}}%
\pgfpathlineto{\pgfqpoint{1.615274in}{2.613610in}}%
\pgfpathlineto{\pgfqpoint{1.562133in}{2.600402in}}%
\pgfpathlineto{\pgfqpoint{1.513681in}{2.586139in}}%
\pgfpathlineto{\pgfqpoint{1.467862in}{2.570344in}}%
\pgfpathlineto{\pgfqpoint{1.426794in}{2.553923in}}%
\pgfpathlineto{\pgfqpoint{1.388447in}{2.536289in}}%
\pgfpathlineto{\pgfqpoint{1.352878in}{2.517566in}}%
\pgfpathlineto{\pgfqpoint{1.320128in}{2.497922in}}%
\pgfpathlineto{\pgfqpoint{1.288379in}{2.476236in}}%
\pgfpathlineto{\pgfqpoint{1.259592in}{2.453861in}}%
\pgfpathlineto{\pgfqpoint{1.232050in}{2.429520in}}%
\pgfpathlineto{\pgfqpoint{1.207527in}{2.404898in}}%
\pgfpathlineto{\pgfqpoint{1.184409in}{2.378557in}}%
\pgfpathlineto{\pgfqpoint{1.162828in}{2.350561in}}%
\pgfpathlineto{\pgfqpoint{1.142891in}{2.321011in}}%
\pgfpathlineto{\pgfqpoint{1.124675in}{2.290041in}}%
\pgfpathlineto{\pgfqpoint{1.108225in}{2.257802in}}%
\pgfpathlineto{\pgfqpoint{1.092639in}{2.222199in}}%
\pgfpathlineto{\pgfqpoint{1.079059in}{2.185535in}}%
\pgfpathlineto{\pgfqpoint{1.067443in}{2.147998in}}%
\pgfpathlineto{\pgfqpoint{1.057187in}{2.107348in}}%
\pgfpathlineto{\pgfqpoint{1.049004in}{2.066086in}}%
\pgfpathlineto{\pgfqpoint{1.042513in}{2.021906in}}%
\pgfpathlineto{\pgfqpoint{1.038177in}{1.977382in}}%
\pgfpathlineto{\pgfqpoint{1.035866in}{1.930167in}}%
\pgfpathlineto{\pgfqpoint{1.035826in}{1.882878in}}%
\pgfpathlineto{\pgfqpoint{1.038031in}{1.835656in}}%
\pgfpathlineto{\pgfqpoint{1.042474in}{1.788641in}}%
\pgfpathlineto{\pgfqpoint{1.049176in}{1.741979in}}%
\pgfpathlineto{\pgfqpoint{1.057644in}{1.698239in}}%
\pgfpathlineto{\pgfqpoint{1.068221in}{1.655105in}}%
\pgfpathlineto{\pgfqpoint{1.080962in}{1.612745in}}%
\pgfpathlineto{\pgfqpoint{1.095031in}{1.573617in}}%
\pgfpathlineto{\pgfqpoint{1.111115in}{1.535520in}}%
\pgfpathlineto{\pgfqpoint{1.128118in}{1.500775in}}%
\pgfpathlineto{\pgfqpoint{1.146930in}{1.467274in}}%
\pgfpathlineto{\pgfqpoint{1.167531in}{1.435181in}}%
\pgfpathlineto{\pgfqpoint{1.189874in}{1.404652in}}%
\pgfpathlineto{\pgfqpoint{1.213884in}{1.375828in}}%
\pgfpathlineto{\pgfqpoint{1.237817in}{1.350457in}}%
\pgfpathlineto{\pgfqpoint{1.264748in}{1.325237in}}%
\pgfpathlineto{\pgfqpoint{1.292991in}{1.301972in}}%
\pgfpathlineto{\pgfqpoint{1.322398in}{1.280678in}}%
\pgfpathlineto{\pgfqpoint{1.352820in}{1.261340in}}%
\pgfpathlineto{\pgfqpoint{1.386095in}{1.242889in}}%
\pgfpathlineto{\pgfqpoint{1.420190in}{1.226516in}}%
\pgfpathlineto{\pgfqpoint{1.457024in}{1.211329in}}%
\pgfpathlineto{\pgfqpoint{1.496554in}{1.197536in}}%
\pgfpathlineto{\pgfqpoint{1.538719in}{1.185287in}}%
\pgfpathlineto{\pgfqpoint{1.583441in}{1.174641in}}%
\pgfpathlineto{\pgfqpoint{1.634929in}{1.164775in}}%
\pgfpathlineto{\pgfqpoint{1.706063in}{1.153745in}}%
\pgfpathlineto{\pgfqpoint{1.768492in}{1.143417in}}%
\pgfpathlineto{\pgfqpoint{1.796122in}{1.136567in}}%
\pgfpathlineto{\pgfqpoint{1.812683in}{1.130481in}}%
\pgfpathlineto{\pgfqpoint{1.824471in}{1.124102in}}%
\pgfpathlineto{\pgfqpoint{1.833209in}{1.116741in}}%
\pgfpathlineto{\pgfqpoint{1.838498in}{1.108890in}}%
\pgfpathlineto{\pgfqpoint{1.840588in}{1.101849in}}%
\pgfpathlineto{\pgfqpoint{1.840619in}{1.094412in}}%
\pgfpathlineto{\pgfqpoint{1.837931in}{1.084986in}}%
\pgfpathlineto{\pgfqpoint{1.833246in}{1.076615in}}%
\pgfpathlineto{\pgfqpoint{1.825819in}{1.067542in}}%
\pgfpathlineto{\pgfqpoint{1.813813in}{1.056850in}}%
\pgfpathlineto{\pgfqpoint{1.798819in}{1.046763in}}%
\pgfpathlineto{\pgfqpoint{1.781016in}{1.037462in}}%
\pgfpathlineto{\pgfqpoint{1.758447in}{1.028391in}}%
\pgfpathlineto{\pgfqpoint{1.733203in}{1.020815in}}%
\pgfpathlineto{\pgfqpoint{1.705410in}{1.014872in}}%
\pgfpathlineto{\pgfqpoint{1.675178in}{1.010714in}}%
\pgfpathlineto{\pgfqpoint{1.642610in}{1.008507in}}%
\pgfpathlineto{\pgfqpoint{1.607809in}{1.008432in}}%
\pgfpathlineto{\pgfqpoint{1.570886in}{1.010691in}}%
\pgfpathlineto{\pgfqpoint{1.534118in}{1.015181in}}%
\pgfpathlineto{\pgfqpoint{1.495454in}{1.022233in}}%
\pgfpathlineto{\pgfqpoint{1.457161in}{1.031563in}}%
\pgfpathlineto{\pgfqpoint{1.419337in}{1.043132in}}%
\pgfpathlineto{\pgfqpoint{1.382089in}{1.056929in}}%
\pgfpathlineto{\pgfqpoint{1.347544in}{1.072019in}}%
\pgfpathlineto{\pgfqpoint{1.313727in}{1.089133in}}%
\pgfpathlineto{\pgfqpoint{1.280762in}{1.108299in}}%
\pgfpathlineto{\pgfqpoint{1.248782in}{1.129536in}}%
\pgfpathlineto{\pgfqpoint{1.219708in}{1.151422in}}%
\pgfpathlineto{\pgfqpoint{1.191752in}{1.175138in}}%
\pgfpathlineto{\pgfqpoint{1.165031in}{1.200649in}}%
\pgfpathlineto{\pgfqpoint{1.139653in}{1.227898in}}%
\pgfpathlineto{\pgfqpoint{1.115714in}{1.256800in}}%
\pgfpathlineto{\pgfqpoint{1.093288in}{1.287251in}}%
\pgfpathlineto{\pgfqpoint{1.071178in}{1.321163in}}%
\pgfpathlineto{\pgfqpoint{1.050868in}{1.356520in}}%
\pgfpathlineto{\pgfqpoint{1.032365in}{1.393152in}}%
\pgfpathlineto{\pgfqpoint{1.014718in}{1.433142in}}%
\pgfpathlineto{\pgfqpoint{0.999024in}{1.474185in}}%
\pgfpathlineto{\pgfqpoint{0.984506in}{1.518461in}}%
\pgfpathlineto{\pgfqpoint{0.972010in}{1.563537in}}%
\pgfpathlineto{\pgfqpoint{0.960944in}{1.611678in}}%
\pgfpathlineto{\pgfqpoint{0.951530in}{1.662824in}}%
\pgfpathlineto{\pgfqpoint{0.944286in}{1.714431in}}%
\pgfpathlineto{\pgfqpoint{0.938950in}{1.768847in}}%
\pgfpathlineto{\pgfqpoint{0.935870in}{1.823491in}}%
\pgfpathlineto{\pgfqpoint{0.935034in}{1.878240in}}%
\pgfpathlineto{\pgfqpoint{0.936466in}{1.932973in}}%
\pgfpathlineto{\pgfqpoint{0.940005in}{1.985084in}}%
\pgfpathlineto{\pgfqpoint{0.945759in}{2.036935in}}%
\pgfpathlineto{\pgfqpoint{0.953410in}{2.085938in}}%
\pgfpathlineto{\pgfqpoint{0.962764in}{2.132000in}}%
\pgfpathlineto{\pgfqpoint{0.974287in}{2.177414in}}%
\pgfpathlineto{\pgfqpoint{0.987332in}{2.219653in}}%
\pgfpathlineto{\pgfqpoint{1.001667in}{2.258654in}}%
\pgfpathlineto{\pgfqpoint{1.018051in}{2.296583in}}%
\pgfpathlineto{\pgfqpoint{1.035401in}{2.331101in}}%
\pgfpathlineto{\pgfqpoint{1.054650in}{2.364275in}}%
\pgfpathlineto{\pgfqpoint{1.074406in}{2.393984in}}%
\pgfpathlineto{\pgfqpoint{1.095771in}{2.422197in}}%
\pgfpathlineto{\pgfqpoint{1.118662in}{2.448797in}}%
\pgfpathlineto{\pgfqpoint{1.142967in}{2.473701in}}%
\pgfpathlineto{\pgfqpoint{1.168550in}{2.496867in}}%
\pgfpathlineto{\pgfqpoint{1.197085in}{2.519662in}}%
\pgfpathlineto{\pgfqpoint{1.226727in}{2.540526in}}%
\pgfpathlineto{\pgfqpoint{1.259242in}{2.560673in}}%
\pgfpathlineto{\pgfqpoint{1.294612in}{2.579881in}}%
\pgfpathlineto{\pgfqpoint{1.332792in}{2.597982in}}%
\pgfpathlineto{\pgfqpoint{1.373719in}{2.614859in}}%
\pgfpathlineto{\pgfqpoint{1.417319in}{2.630445in}}%
\pgfpathlineto{\pgfqpoint{1.465632in}{2.645312in}}%
\pgfpathlineto{\pgfqpoint{1.518640in}{2.659204in}}%
\pgfpathlineto{\pgfqpoint{1.576309in}{2.671929in}}%
\pgfpathlineto{\pgfqpoint{1.638597in}{2.683344in}}%
\pgfpathlineto{\pgfqpoint{1.705462in}{2.693343in}}%
\pgfpathlineto{\pgfqpoint{1.779027in}{2.702064in}}%
\pgfpathlineto{\pgfqpoint{1.857097in}{2.709077in}}%
\pgfpathlineto{\pgfqpoint{1.939633in}{2.714280in}}%
\pgfpathlineto{\pgfqpoint{2.026598in}{2.717513in}}%
\pgfpathlineto{\pgfqpoint{2.113605in}{2.718523in}}%
\pgfpathlineto{\pgfqpoint{2.198435in}{2.717303in}}%
\pgfpathlineto{\pgfqpoint{2.278866in}{2.713929in}}%
\pgfpathlineto{\pgfqpoint{2.352678in}{2.708598in}}%
\pgfpathlineto{\pgfqpoint{2.417657in}{2.701709in}}%
\pgfpathlineto{\pgfqpoint{2.473770in}{2.693630in}}%
\pgfpathlineto{\pgfqpoint{2.523140in}{2.684368in}}%
\pgfpathlineto{\pgfqpoint{2.565726in}{2.674202in}}%
\pgfpathlineto{\pgfqpoint{2.601510in}{2.663544in}}%
\pgfpathlineto{\pgfqpoint{2.632577in}{2.652142in}}%
\pgfpathlineto{\pgfqpoint{2.658899in}{2.640331in}}%
\pgfpathlineto{\pgfqpoint{2.682438in}{2.627436in}}%
\pgfpathlineto{\pgfqpoint{2.703062in}{2.613571in}}%
\pgfpathlineto{\pgfqpoint{2.720674in}{2.598978in}}%
\pgfpathlineto{\pgfqpoint{2.735263in}{2.584053in}}%
\pgfpathlineto{\pgfqpoint{2.748320in}{2.567377in}}%
\pgfpathlineto{\pgfqpoint{2.759553in}{2.549046in}}%
\pgfpathlineto{\pgfqpoint{2.768788in}{2.529306in}}%
\pgfpathlineto{\pgfqpoint{2.776017in}{2.508498in}}%
\pgfpathlineto{\pgfqpoint{2.781884in}{2.484540in}}%
\pgfpathlineto{\pgfqpoint{2.786102in}{2.457597in}}%
\pgfpathlineto{\pgfqpoint{2.788720in}{2.425384in}}%
\pgfpathlineto{\pgfqpoint{2.789427in}{2.388061in}}%
\pgfpathlineto{\pgfqpoint{2.787962in}{2.340801in}}%
\pgfpathlineto{\pgfqpoint{2.783672in}{2.278768in}}%
\pgfpathlineto{\pgfqpoint{2.774289in}{2.179783in}}%
\pgfpathlineto{\pgfqpoint{2.743611in}{1.868119in}}%
\pgfpathlineto{\pgfqpoint{2.730112in}{1.702060in}}%
\pgfpathlineto{\pgfqpoint{2.717287in}{1.515949in}}%
\pgfpathlineto{\pgfqpoint{2.702602in}{1.267597in}}%
\pgfpathlineto{\pgfqpoint{2.684434in}{0.964630in}}%
\pgfpathlineto{\pgfqpoint{2.675374in}{0.850600in}}%
\pgfpathlineto{\pgfqpoint{2.667030in}{0.771523in}}%
\pgfpathlineto{\pgfqpoint{2.658752in}{0.712543in}}%
\pgfpathlineto{\pgfqpoint{2.650176in}{0.666284in}}%
\pgfpathlineto{\pgfqpoint{2.640820in}{0.627931in}}%
\pgfpathlineto{\pgfqpoint{2.631145in}{0.597534in}}%
\pgfpathlineto{\pgfqpoint{2.621004in}{0.572745in}}%
\pgfpathlineto{\pgfqpoint{2.609856in}{0.551383in}}%
\pgfpathlineto{\pgfqpoint{2.598042in}{0.533534in}}%
\pgfpathlineto{\pgfqpoint{2.584496in}{0.517378in}}%
\pgfpathlineto{\pgfqpoint{2.571109in}{0.504669in}}%
\pgfpathlineto{\pgfqpoint{2.554789in}{0.492313in}}%
\pgfpathlineto{\pgfqpoint{2.537457in}{0.481914in}}%
\pgfpathlineto{\pgfqpoint{2.517374in}{0.472367in}}%
\pgfpathlineto{\pgfqpoint{2.492542in}{0.463178in}}%
\pgfpathlineto{\pgfqpoint{2.462979in}{0.454833in}}%
\pgfpathlineto{\pgfqpoint{2.428766in}{0.447542in}}%
\pgfpathlineto{\pgfqpoint{2.385671in}{0.440735in}}%
\pgfpathlineto{\pgfqpoint{2.331557in}{0.434581in}}%
\pgfpathlineto{\pgfqpoint{2.262115in}{0.429077in}}%
\pgfpathlineto{\pgfqpoint{2.170851in}{0.424236in}}%
\pgfpathlineto{\pgfqpoint{2.049086in}{0.420134in}}%
\pgfpathlineto{\pgfqpoint{1.879436in}{0.416783in}}%
\pgfpathlineto{\pgfqpoint{1.640159in}{0.414418in}}%
\pgfpathlineto{\pgfqpoint{1.322562in}{0.413569in}}%
\pgfpathlineto{\pgfqpoint{1.020194in}{0.414850in}}%
\pgfpathlineto{\pgfqpoint{0.822256in}{0.417715in}}%
\pgfpathlineto{\pgfqpoint{0.704835in}{0.421430in}}%
\pgfpathlineto{\pgfqpoint{0.630976in}{0.425829in}}%
\pgfpathlineto{\pgfqpoint{0.583316in}{0.430734in}}%
\pgfpathlineto{\pgfqpoint{0.551033in}{0.436123in}}%
\pgfpathlineto{\pgfqpoint{0.527708in}{0.442189in}}%
\pgfpathlineto{\pgfqpoint{0.511250in}{0.448625in}}%
\pgfpathlineto{\pgfqpoint{0.499549in}{0.455216in}}%
\pgfpathlineto{\pgfqpoint{0.488916in}{0.463841in}}%
\pgfpathlineto{\pgfqpoint{0.481322in}{0.472730in}}%
\pgfpathlineto{\pgfqpoint{0.474078in}{0.485127in}}%
\pgfpathlineto{\pgfqpoint{0.468753in}{0.498748in}}%
\pgfpathlineto{\pgfqpoint{0.463870in}{0.517848in}}%
\pgfpathlineto{\pgfqpoint{0.459679in}{0.544796in}}%
\pgfpathlineto{\pgfqpoint{0.456386in}{0.581938in}}%
\pgfpathlineto{\pgfqpoint{0.453731in}{0.639106in}}%
\pgfpathlineto{\pgfqpoint{0.451681in}{0.736155in}}%
\pgfpathlineto{\pgfqpoint{0.450220in}{0.927815in}}%
\pgfpathlineto{\pgfqpoint{0.449345in}{1.403252in}}%
\pgfpathlineto{\pgfqpoint{0.449543in}{2.682703in}}%
\pgfpathlineto{\pgfqpoint{0.451011in}{2.856932in}}%
\pgfpathlineto{\pgfqpoint{0.452802in}{2.879219in}}%
\pgfpathlineto{\pgfqpoint{0.455188in}{2.886108in}}%
\pgfpathlineto{\pgfqpoint{0.458626in}{2.889028in}}%
\pgfpathlineto{\pgfqpoint{0.464996in}{2.890553in}}%
\pgfpathlineto{\pgfqpoint{0.482377in}{2.891423in}}%
\pgfpathlineto{\pgfqpoint{0.565038in}{2.891729in}}%
\pgfpathlineto{\pgfqpoint{2.733842in}{2.891760in}}%
\pgfpathlineto{\pgfqpoint{4.789510in}{2.890885in}}%
\pgfpathlineto{\pgfqpoint{4.793727in}{2.889730in}}%
\pgfpathlineto{\pgfqpoint{4.795481in}{2.888307in}}%
\pgfpathlineto{\pgfqpoint{4.797106in}{2.881145in}}%
\pgfpathlineto{\pgfqpoint{4.797997in}{2.858771in}}%
\pgfpathlineto{\pgfqpoint{4.798039in}{2.856283in}}%
\pgfpathlineto{\pgfqpoint{4.798039in}{2.856283in}}%
\pgfusepath{stroke}%
\end{pgfscope}%
\begin{pgfscope}%
\pgfpathrectangle{\pgfqpoint{0.448634in}{0.402556in}}{\pgfqpoint{4.350661in}{2.489204in}} %
\pgfusepath{clip}%
\pgfsetrectcap%
\pgfsetroundjoin%
\pgfsetlinewidth{1.003750pt}%
\definecolor{currentstroke}{rgb}{1.000000,0.388235,0.278431}%
\pgfsetstrokecolor{currentstroke}%
\pgfsetdash{}{0pt}%
\pgfpathmoveto{\pgfqpoint{3.428772in}{0.402610in}}%
\pgfpathlineto{\pgfqpoint{2.806632in}{0.403760in}}%
\pgfpathlineto{\pgfqpoint{2.769692in}{0.405578in}}%
\pgfpathlineto{\pgfqpoint{2.754632in}{0.408064in}}%
\pgfpathlineto{\pgfqpoint{2.746391in}{0.411198in}}%
\pgfpathlineto{\pgfqpoint{2.740943in}{0.415265in}}%
\pgfpathlineto{\pgfqpoint{2.736784in}{0.420984in}}%
\pgfpathlineto{\pgfqpoint{2.733281in}{0.430071in}}%
\pgfpathlineto{\pgfqpoint{2.730449in}{0.444636in}}%
\pgfpathlineto{\pgfqpoint{2.728238in}{0.469392in}}%
\pgfpathlineto{\pgfqpoint{2.726470in}{0.519131in}}%
\pgfpathlineto{\pgfqpoint{2.725711in}{0.613715in}}%
\pgfpathlineto{\pgfqpoint{2.726842in}{0.768038in}}%
\pgfpathlineto{\pgfqpoint{2.730556in}{0.962148in}}%
\pgfpathlineto{\pgfqpoint{2.736611in}{1.158670in}}%
\pgfpathlineto{\pgfqpoint{2.744092in}{1.327718in}}%
\pgfpathlineto{\pgfqpoint{2.753201in}{1.484189in}}%
\pgfpathlineto{\pgfqpoint{2.763257in}{1.620609in}}%
\pgfpathlineto{\pgfqpoint{2.776118in}{1.764216in}}%
\pgfpathlineto{\pgfqpoint{2.788914in}{1.877776in}}%
\pgfpathlineto{\pgfqpoint{2.805748in}{2.005740in}}%
\pgfpathlineto{\pgfqpoint{2.821176in}{2.101198in}}%
\pgfpathlineto{\pgfqpoint{2.838359in}{2.193718in}}%
\pgfpathlineto{\pgfqpoint{2.859135in}{2.292966in}}%
\pgfpathlineto{\pgfqpoint{2.887209in}{2.425960in}}%
\pgfpathlineto{\pgfqpoint{2.896991in}{2.479559in}}%
\pgfpathlineto{\pgfqpoint{2.901543in}{2.516523in}}%
\pgfpathlineto{\pgfqpoint{2.902849in}{2.543854in}}%
\pgfpathlineto{\pgfqpoint{2.901957in}{2.566223in}}%
\pgfpathlineto{\pgfqpoint{2.899151in}{2.585863in}}%
\pgfpathlineto{\pgfqpoint{2.894794in}{2.602546in}}%
\pgfpathlineto{\pgfqpoint{2.888484in}{2.618388in}}%
\pgfpathlineto{\pgfqpoint{2.880257in}{2.633033in}}%
\pgfpathlineto{\pgfqpoint{2.870348in}{2.646246in}}%
\pgfpathlineto{\pgfqpoint{2.857400in}{2.659530in}}%
\pgfpathlineto{\pgfqpoint{2.843189in}{2.671010in}}%
\pgfpathlineto{\pgfqpoint{2.824237in}{2.683209in}}%
\pgfpathlineto{\pgfqpoint{2.802413in}{2.694418in}}%
\pgfpathlineto{\pgfqpoint{2.775809in}{2.705369in}}%
\pgfpathlineto{\pgfqpoint{2.744461in}{2.715715in}}%
\pgfpathlineto{\pgfqpoint{2.708436in}{2.725252in}}%
\pgfpathlineto{\pgfqpoint{2.665655in}{2.734289in}}%
\pgfpathlineto{\pgfqpoint{2.613991in}{2.742869in}}%
\pgfpathlineto{\pgfqpoint{2.553459in}{2.750589in}}%
\pgfpathlineto{\pgfqpoint{2.481920in}{2.757365in}}%
\pgfpathlineto{\pgfqpoint{2.399398in}{2.762839in}}%
\pgfpathlineto{\pgfqpoint{2.310269in}{2.766482in}}%
\pgfpathlineto{\pgfqpoint{2.175416in}{2.768725in}}%
\pgfpathlineto{\pgfqpoint{2.066653in}{2.767942in}}%
\pgfpathlineto{\pgfqpoint{1.953570in}{2.764859in}}%
\pgfpathlineto{\pgfqpoint{1.851429in}{2.759759in}}%
\pgfpathlineto{\pgfqpoint{1.745051in}{2.752169in}}%
\pgfpathlineto{\pgfqpoint{1.658373in}{2.743453in}}%
\pgfpathlineto{\pgfqpoint{1.580552in}{2.733461in}}%
\pgfpathlineto{\pgfqpoint{1.490057in}{2.719338in}}%
\pgfpathlineto{\pgfqpoint{1.417231in}{2.704698in}}%
\pgfpathlineto{\pgfqpoint{1.361992in}{2.690818in}}%
\pgfpathlineto{\pgfqpoint{1.311460in}{2.675819in}}%
\pgfpathlineto{\pgfqpoint{1.265667in}{2.659924in}}%
\pgfpathlineto{\pgfqpoint{1.222575in}{2.642586in}}%
\pgfpathlineto{\pgfqpoint{1.184324in}{2.624682in}}%
\pgfpathlineto{\pgfqpoint{1.148892in}{2.605623in}}%
\pgfpathlineto{\pgfqpoint{1.116331in}{2.585573in}}%
\pgfpathlineto{\pgfqpoint{1.092327in}{2.568512in}}%
\pgfpathlineto{\pgfqpoint{1.079760in}{2.558686in}}%
\pgfpathlineto{\pgfqpoint{1.051544in}{2.535379in}}%
\pgfpathlineto{\pgfqpoint{1.026312in}{2.511712in}}%
\pgfpathlineto{\pgfqpoint{1.002399in}{2.486318in}}%
\pgfpathlineto{\pgfqpoint{0.979913in}{2.459269in}}%
\pgfpathlineto{\pgfqpoint{0.958934in}{2.430678in}}%
\pgfpathlineto{\pgfqpoint{0.938264in}{2.398643in}}%
\pgfpathlineto{\pgfqpoint{0.923047in}{2.371385in}}%
\pgfpathlineto{\pgfqpoint{0.904513in}{2.334774in}}%
\pgfpathlineto{\pgfqpoint{0.887854in}{2.297001in}}%
\pgfpathlineto{\pgfqpoint{0.872131in}{2.255971in}}%
\pgfpathlineto{\pgfqpoint{0.857508in}{2.211741in}}%
\pgfpathlineto{\pgfqpoint{0.844762in}{2.166757in}}%
\pgfpathlineto{\pgfqpoint{0.838624in}{2.140306in}}%
\pgfpathlineto{\pgfqpoint{0.826982in}{2.087194in}}%
\pgfpathlineto{\pgfqpoint{0.816322in}{2.028715in}}%
\pgfpathlineto{\pgfqpoint{0.810087in}{1.984495in}}%
\pgfpathlineto{\pgfqpoint{0.808026in}{1.967238in}}%
\pgfpathlineto{\pgfqpoint{0.800076in}{1.898140in}}%
\pgfpathlineto{\pgfqpoint{0.793713in}{1.823823in}}%
\pgfpathlineto{\pgfqpoint{0.788799in}{1.741875in}}%
\pgfpathlineto{\pgfqpoint{0.786199in}{1.677225in}}%
\pgfpathlineto{\pgfqpoint{0.776951in}{1.453481in}}%
\pgfpathlineto{\pgfqpoint{0.773280in}{1.418894in}}%
\pgfpathlineto{\pgfqpoint{0.768298in}{1.389582in}}%
\pgfpathlineto{\pgfqpoint{0.762752in}{1.368108in}}%
\pgfpathlineto{\pgfqpoint{0.756722in}{1.352123in}}%
\pgfpathlineto{\pgfqpoint{0.749752in}{1.339519in}}%
\pgfpathlineto{\pgfqpoint{0.742201in}{1.330599in}}%
\pgfpathlineto{\pgfqpoint{0.734854in}{1.325312in}}%
\pgfpathlineto{\pgfqpoint{0.726558in}{1.322419in}}%
\pgfpathlineto{\pgfqpoint{0.717884in}{1.322223in}}%
\pgfpathlineto{\pgfqpoint{0.709412in}{1.324411in}}%
\pgfpathlineto{\pgfqpoint{0.699548in}{1.329604in}}%
\pgfpathlineto{\pgfqpoint{0.688894in}{1.338203in}}%
\pgfpathlineto{\pgfqpoint{0.677907in}{1.350248in}}%
\pgfpathlineto{\pgfqpoint{0.666886in}{1.365647in}}%
\pgfpathlineto{\pgfqpoint{0.654913in}{1.386417in}}%
\pgfpathlineto{\pgfqpoint{0.642574in}{1.412730in}}%
\pgfpathlineto{\pgfqpoint{0.630328in}{1.444629in}}%
\pgfpathlineto{\pgfqpoint{0.618504in}{1.482081in}}%
\pgfpathlineto{\pgfqpoint{0.608613in}{1.520256in}}%
\pgfpathlineto{\pgfqpoint{0.590203in}{1.612445in}}%
\pgfpathlineto{\pgfqpoint{0.581848in}{1.668884in}}%
\pgfpathlineto{\pgfqpoint{0.573137in}{1.740376in}}%
\pgfpathlineto{\pgfqpoint{0.567062in}{1.807213in}}%
\pgfpathlineto{\pgfqpoint{0.560532in}{1.896510in}}%
\pgfpathlineto{\pgfqpoint{0.555526in}{1.995910in}}%
\pgfpathlineto{\pgfqpoint{0.552564in}{2.097908in}}%
\pgfpathlineto{\pgfqpoint{0.551526in}{2.204935in}}%
\pgfpathlineto{\pgfqpoint{0.552728in}{2.309470in}}%
\pgfpathlineto{\pgfqpoint{0.556011in}{2.403981in}}%
\pgfpathlineto{\pgfqpoint{0.560953in}{2.483430in}}%
\pgfpathlineto{\pgfqpoint{0.567303in}{2.550240in}}%
\pgfpathlineto{\pgfqpoint{0.574928in}{2.606817in}}%
\pgfpathlineto{\pgfqpoint{0.582988in}{2.650657in}}%
\pgfpathlineto{\pgfqpoint{0.592756in}{2.691452in}}%
\pgfpathlineto{\pgfqpoint{0.602650in}{2.721756in}}%
\pgfpathlineto{\pgfqpoint{0.612983in}{2.746441in}}%
\pgfpathlineto{\pgfqpoint{0.624292in}{2.767692in}}%
\pgfpathlineto{\pgfqpoint{0.636231in}{2.785433in}}%
\pgfpathlineto{\pgfqpoint{0.649892in}{2.801461in}}%
\pgfpathlineto{\pgfqpoint{0.663386in}{2.814020in}}%
\pgfpathlineto{\pgfqpoint{0.679842in}{2.826135in}}%
\pgfpathlineto{\pgfqpoint{0.697326in}{2.836197in}}%
\pgfpathlineto{\pgfqpoint{0.715574in}{2.844285in}}%
\pgfpathlineto{\pgfqpoint{0.738439in}{2.852335in}}%
\pgfpathlineto{\pgfqpoint{0.765983in}{2.859639in}}%
\pgfpathlineto{\pgfqpoint{0.800300in}{2.866256in}}%
\pgfpathlineto{\pgfqpoint{0.841340in}{2.871832in}}%
\pgfpathlineto{\pgfqpoint{0.895547in}{2.876803in}}%
\pgfpathlineto{\pgfqpoint{0.969413in}{2.881069in}}%
\pgfpathlineto{\pgfqpoint{1.071608in}{2.884501in}}%
\pgfpathlineto{\pgfqpoint{1.219512in}{2.887074in}}%
\pgfpathlineto{\pgfqpoint{1.471844in}{2.889091in}}%
\pgfpathlineto{\pgfqpoint{1.956941in}{2.890384in}}%
\pgfpathlineto{\pgfqpoint{3.096814in}{2.890781in}}%
\pgfpathlineto{\pgfqpoint{3.995224in}{2.889388in}}%
\pgfpathlineto{\pgfqpoint{4.275833in}{2.887011in}}%
\pgfpathlineto{\pgfqpoint{4.412847in}{2.883743in}}%
\pgfpathlineto{\pgfqpoint{4.491081in}{2.879810in}}%
\pgfpathlineto{\pgfqpoint{4.543127in}{2.875163in}}%
\pgfpathlineto{\pgfqpoint{4.579810in}{2.869841in}}%
\pgfpathlineto{\pgfqpoint{4.607580in}{2.863763in}}%
\pgfpathlineto{\pgfqpoint{4.630623in}{2.856424in}}%
\pgfpathlineto{\pgfqpoint{4.648833in}{2.848228in}}%
\pgfpathlineto{\pgfqpoint{4.664136in}{2.838773in}}%
\pgfpathlineto{\pgfqpoint{4.676470in}{2.828576in}}%
\pgfpathlineto{\pgfqpoint{4.687502in}{2.816585in}}%
\pgfpathlineto{\pgfqpoint{4.697051in}{2.803027in}}%
\pgfpathlineto{\pgfqpoint{4.706194in}{2.786098in}}%
\pgfpathlineto{\pgfqpoint{4.714508in}{2.765827in}}%
\pgfpathlineto{\pgfqpoint{4.722462in}{2.740013in}}%
\pgfpathlineto{\pgfqpoint{4.729577in}{2.708703in}}%
\pgfpathlineto{\pgfqpoint{4.736162in}{2.669601in}}%
\pgfpathlineto{\pgfqpoint{4.742419in}{2.617826in}}%
\pgfpathlineto{\pgfqpoint{4.747859in}{2.553410in}}%
\pgfpathlineto{\pgfqpoint{4.752661in}{2.468958in}}%
\pgfpathlineto{\pgfqpoint{4.756610in}{2.359528in}}%
\pgfpathlineto{\pgfqpoint{4.759416in}{2.217681in}}%
\pgfpathlineto{\pgfqpoint{4.760596in}{2.043444in}}%
\pgfpathlineto{\pgfqpoint{4.759662in}{1.851779in}}%
\pgfpathlineto{\pgfqpoint{4.756587in}{1.667613in}}%
\pgfpathlineto{\pgfqpoint{4.751596in}{1.503428in}}%
\pgfpathlineto{\pgfqpoint{4.745410in}{1.374185in}}%
\pgfpathlineto{\pgfqpoint{4.738113in}{1.267479in}}%
\pgfpathlineto{\pgfqpoint{4.729621in}{1.175896in}}%
\pgfpathlineto{\pgfqpoint{4.720762in}{1.104428in}}%
\pgfpathlineto{\pgfqpoint{4.711045in}{1.043204in}}%
\pgfpathlineto{\pgfqpoint{4.700364in}{0.989829in}}%
\pgfpathlineto{\pgfqpoint{4.689055in}{0.944345in}}%
\pgfpathlineto{\pgfqpoint{4.676881in}{0.904394in}}%
\pgfpathlineto{\pgfqpoint{4.676095in}{0.902073in}}%
\pgfpathlineto{\pgfqpoint{4.676095in}{0.902073in}}%
\pgfusepath{stroke}%
\end{pgfscope}%
\begin{pgfscope}%
\pgfpathrectangle{\pgfqpoint{0.448634in}{0.402556in}}{\pgfqpoint{4.350661in}{2.489204in}} %
\pgfusepath{clip}%
\pgfsetrectcap%
\pgfsetroundjoin%
\pgfsetlinewidth{1.003750pt}%
\definecolor{currentstroke}{rgb}{1.000000,0.388235,0.278431}%
\pgfsetstrokecolor{currentstroke}%
\pgfsetdash{}{0pt}%
\pgfpathmoveto{\pgfqpoint{2.795520in}{1.982745in}}%
\pgfpathlineto{\pgfqpoint{2.781780in}{1.874357in}}%
\pgfpathlineto{\pgfqpoint{2.769351in}{1.758234in}}%
\pgfpathlineto{\pgfqpoint{2.758095in}{1.631942in}}%
\pgfpathlineto{\pgfqpoint{2.747786in}{1.490551in}}%
\pgfpathlineto{\pgfqpoint{2.738644in}{1.334082in}}%
\pgfpathlineto{\pgfqpoint{2.730580in}{1.157591in}}%
\pgfpathlineto{\pgfqpoint{2.723334in}{0.948663in}}%
\pgfpathlineto{\pgfqpoint{2.709783in}{0.530788in}}%
\pgfpathlineto{\pgfqpoint{2.705868in}{0.488716in}}%
\pgfpathlineto{\pgfqpoint{2.701769in}{0.464281in}}%
\pgfpathlineto{\pgfqpoint{2.697021in}{0.447744in}}%
\pgfpathlineto{\pgfqpoint{2.691859in}{0.436812in}}%
\pgfpathlineto{\pgfqpoint{2.686245in}{0.429229in}}%
\pgfpathlineto{\pgfqpoint{2.679348in}{0.423188in}}%
\pgfpathlineto{\pgfqpoint{2.669540in}{0.417856in}}%
\pgfpathlineto{\pgfqpoint{2.656987in}{0.413810in}}%
\pgfpathlineto{\pgfqpoint{2.637654in}{0.410337in}}%
\pgfpathlineto{\pgfqpoint{2.607297in}{0.407617in}}%
\pgfpathlineto{\pgfqpoint{2.555121in}{0.405574in}}%
\pgfpathlineto{\pgfqpoint{2.450714in}{0.404139in}}%
\pgfpathlineto{\pgfqpoint{2.176624in}{0.403275in}}%
\pgfpathlineto{\pgfqpoint{1.130290in}{0.402953in}}%
\pgfpathlineto{\pgfqpoint{0.516849in}{0.404175in}}%
\pgfpathlineto{\pgfqpoint{0.466848in}{0.405970in}}%
\pgfpathlineto{\pgfqpoint{0.456130in}{0.407931in}}%
\pgfpathlineto{\pgfqpoint{0.452340in}{0.410303in}}%
\pgfpathlineto{\pgfqpoint{0.450346in}{0.414662in}}%
\pgfpathlineto{\pgfqpoint{0.449266in}{0.424524in}}%
\pgfpathlineto{\pgfqpoint{0.448771in}{0.464344in}}%
\pgfpathlineto{\pgfqpoint{0.448640in}{0.850171in}}%
\pgfpathlineto{\pgfqpoint{0.448653in}{2.891318in}}%
\pgfpathlineto{\pgfqpoint{0.448653in}{2.891318in}}%
\pgfusepath{stroke}%
\end{pgfscope}%
\begin{pgfscope}%
\pgfpathrectangle{\pgfqpoint{0.448634in}{0.402556in}}{\pgfqpoint{4.350661in}{2.489204in}} %
\pgfusepath{clip}%
\pgfsetrectcap%
\pgfsetroundjoin%
\pgfsetlinewidth{1.003750pt}%
\definecolor{currentstroke}{rgb}{1.000000,0.388235,0.278431}%
\pgfsetstrokecolor{currentstroke}%
\pgfsetdash{}{0pt}%
\pgfpathmoveto{\pgfqpoint{3.428189in}{0.402586in}}%
\pgfpathlineto{\pgfqpoint{2.782121in}{0.403701in}}%
\pgfpathlineto{\pgfqpoint{2.753906in}{0.405674in}}%
\pgfpathlineto{\pgfqpoint{2.743328in}{0.408443in}}%
\pgfpathlineto{\pgfqpoint{2.737717in}{0.412188in}}%
\pgfpathlineto{\pgfqpoint{2.733668in}{0.417995in}}%
\pgfpathlineto{\pgfqpoint{2.730649in}{0.427307in}}%
\pgfpathlineto{\pgfqpoint{2.728388in}{0.442004in}}%
\pgfpathlineto{\pgfqpoint{2.726544in}{0.471794in}}%
\pgfpathlineto{\pgfqpoint{2.725216in}{0.534003in}}%
\pgfpathlineto{\pgfqpoint{2.725169in}{0.655973in}}%
\pgfpathlineto{\pgfqpoint{2.727377in}{0.832687in}}%
\pgfpathlineto{\pgfqpoint{2.732259in}{1.041703in}}%
\pgfpathlineto{\pgfqpoint{2.738851in}{1.223257in}}%
\pgfpathlineto{\pgfqpoint{2.747078in}{1.389766in}}%
\pgfpathlineto{\pgfqpoint{2.756608in}{1.538717in}}%
\pgfpathlineto{\pgfqpoint{2.768955in}{1.694887in}}%
\pgfpathlineto{\pgfqpoint{2.781228in}{1.816044in}}%
\pgfpathlineto{\pgfqpoint{2.794401in}{1.924524in}}%
\pgfpathlineto{\pgfqpoint{2.812737in}{2.054722in}}%
\pgfpathlineto{\pgfqpoint{2.828774in}{2.147512in}}%
\pgfpathlineto{\pgfqpoint{2.847382in}{2.242224in}}%
\pgfpathlineto{\pgfqpoint{2.895818in}{2.479699in}}%
\pgfpathlineto{\pgfqpoint{2.900204in}{2.516689in}}%
\pgfpathlineto{\pgfqpoint{2.901346in}{2.544029in}}%
\pgfpathlineto{\pgfqpoint{2.900291in}{2.566388in}}%
\pgfpathlineto{\pgfqpoint{2.897334in}{2.585999in}}%
\pgfpathlineto{\pgfqpoint{2.892836in}{2.602633in}}%
\pgfpathlineto{\pgfqpoint{2.886394in}{2.618405in}}%
\pgfpathlineto{\pgfqpoint{2.878058in}{2.632969in}}%
\pgfpathlineto{\pgfqpoint{2.868065in}{2.646100in}}%
\pgfpathlineto{\pgfqpoint{2.855050in}{2.659300in}}%
\pgfpathlineto{\pgfqpoint{2.840801in}{2.670717in}}%
\pgfpathlineto{\pgfqpoint{2.821822in}{2.682861in}}%
\pgfpathlineto{\pgfqpoint{2.799980in}{2.694026in}}%
\pgfpathlineto{\pgfqpoint{2.773366in}{2.704944in}}%
\pgfpathlineto{\pgfqpoint{2.742012in}{2.715266in}}%
\pgfpathlineto{\pgfqpoint{2.705983in}{2.724785in}}%
\pgfpathlineto{\pgfqpoint{2.663200in}{2.733810in}}%
\pgfpathlineto{\pgfqpoint{2.611535in}{2.742379in}}%
\pgfpathlineto{\pgfqpoint{2.551002in}{2.750090in}}%
\pgfpathlineto{\pgfqpoint{2.481632in}{2.756682in}}%
\pgfpathlineto{\pgfqpoint{2.399112in}{2.762200in}}%
\pgfpathlineto{\pgfqpoint{2.309985in}{2.765886in}}%
\pgfpathlineto{\pgfqpoint{2.188184in}{2.768096in}}%
\pgfpathlineto{\pgfqpoint{2.081595in}{2.767619in}}%
\pgfpathlineto{\pgfqpoint{1.968506in}{2.764840in}}%
\pgfpathlineto{\pgfqpoint{1.864180in}{2.759918in}}%
\pgfpathlineto{\pgfqpoint{1.757786in}{2.752593in}}%
\pgfpathlineto{\pgfqpoint{1.671087in}{2.744171in}}%
\pgfpathlineto{\pgfqpoint{1.591076in}{2.734193in}}%
\pgfpathlineto{\pgfqpoint{1.502689in}{2.720717in}}%
\pgfpathlineto{\pgfqpoint{1.427655in}{2.706083in}}%
\pgfpathlineto{\pgfqpoint{1.372350in}{2.692544in}}%
\pgfpathlineto{\pgfqpoint{1.321734in}{2.677921in}}%
\pgfpathlineto{\pgfqpoint{1.273765in}{2.661664in}}%
\pgfpathlineto{\pgfqpoint{1.230567in}{2.644672in}}%
\pgfpathlineto{\pgfqpoint{1.192197in}{2.627106in}}%
\pgfpathlineto{\pgfqpoint{1.156620in}{2.608403in}}%
\pgfpathlineto{\pgfqpoint{1.123890in}{2.588716in}}%
\pgfpathlineto{\pgfqpoint{1.095883in}{2.569568in}}%
\pgfpathlineto{\pgfqpoint{1.063936in}{2.543701in}}%
\pgfpathlineto{\pgfqpoint{1.038217in}{2.520732in}}%
\pgfpathlineto{\pgfqpoint{1.013766in}{2.496016in}}%
\pgfpathlineto{\pgfqpoint{0.990704in}{2.469610in}}%
\pgfpathlineto{\pgfqpoint{0.969124in}{2.441612in}}%
\pgfpathlineto{\pgfqpoint{0.949083in}{2.412154in}}%
\pgfpathlineto{\pgfqpoint{0.930604in}{2.381387in}}%
\pgfpathlineto{\pgfqpoint{0.906555in}{2.334052in}}%
\pgfpathlineto{\pgfqpoint{0.889925in}{2.296262in}}%
\pgfpathlineto{\pgfqpoint{0.874241in}{2.255213in}}%
\pgfpathlineto{\pgfqpoint{0.859667in}{2.210961in}}%
\pgfpathlineto{\pgfqpoint{0.846986in}{2.165954in}}%
\pgfpathlineto{\pgfqpoint{0.839633in}{2.134715in}}%
\pgfpathlineto{\pgfqpoint{0.828238in}{2.081532in}}%
\pgfpathlineto{\pgfqpoint{0.817866in}{2.022986in}}%
\pgfpathlineto{\pgfqpoint{0.810784in}{1.971352in}}%
\pgfpathlineto{\pgfqpoint{0.802846in}{1.902252in}}%
\pgfpathlineto{\pgfqpoint{0.796554in}{1.827927in}}%
\pgfpathlineto{\pgfqpoint{0.791696in}{1.743480in}}%
\pgfpathlineto{\pgfqpoint{0.787773in}{1.621595in}}%
\pgfpathlineto{\pgfqpoint{0.785408in}{1.522064in}}%
\pgfpathlineto{\pgfqpoint{0.785408in}{1.522064in}}%
\pgfusepath{stroke}%
\end{pgfscope}%
\begin{pgfscope}%
\pgfpathrectangle{\pgfqpoint{0.448634in}{0.402556in}}{\pgfqpoint{4.350661in}{2.489204in}} %
\pgfusepath{clip}%
\pgfsetrectcap%
\pgfsetroundjoin%
\pgfsetlinewidth{1.003750pt}%
\definecolor{currentstroke}{rgb}{1.000000,0.388235,0.278431}%
\pgfsetstrokecolor{currentstroke}%
\pgfsetdash{}{0pt}%
\pgfpathmoveto{\pgfqpoint{2.028735in}{0.425754in}}%
\pgfpathlineto{\pgfqpoint{1.878677in}{0.421879in}}%
\pgfpathlineto{\pgfqpoint{1.676387in}{0.418997in}}%
\pgfpathlineto{\pgfqpoint{1.413176in}{0.417558in}}%
\pgfpathlineto{\pgfqpoint{1.134735in}{0.418204in}}%
\pgfpathlineto{\pgfqpoint{0.921565in}{0.420769in}}%
\pgfpathlineto{\pgfqpoint{0.782384in}{0.424523in}}%
\pgfpathlineto{\pgfqpoint{0.693283in}{0.428974in}}%
\pgfpathlineto{\pgfqpoint{0.632541in}{0.434091in}}%
\pgfpathlineto{\pgfqpoint{0.591492in}{0.439564in}}%
\pgfpathlineto{\pgfqpoint{0.561503in}{0.445595in}}%
\pgfpathlineto{\pgfqpoint{0.538349in}{0.452466in}}%
\pgfpathlineto{\pgfqpoint{0.522042in}{0.459394in}}%
\pgfpathlineto{\pgfqpoint{0.508540in}{0.467420in}}%
\pgfpathlineto{\pgfqpoint{0.497973in}{0.476161in}}%
\pgfpathlineto{\pgfqpoint{0.488790in}{0.486750in}}%
\pgfpathlineto{\pgfqpoint{0.481284in}{0.498948in}}%
\pgfpathlineto{\pgfqpoint{0.474590in}{0.514580in}}%
\pgfpathlineto{\pgfqpoint{0.469106in}{0.533467in}}%
\pgfpathlineto{\pgfqpoint{0.464439in}{0.557771in}}%
\pgfpathlineto{\pgfqpoint{0.460297in}{0.592289in}}%
\pgfpathlineto{\pgfqpoint{0.456856in}{0.641912in}}%
\pgfpathlineto{\pgfqpoint{0.454122in}{0.716520in}}%
\pgfpathlineto{\pgfqpoint{0.451978in}{0.843444in}}%
\pgfpathlineto{\pgfqpoint{0.450459in}{1.087380in}}%
\pgfpathlineto{\pgfqpoint{0.449596in}{1.657406in}}%
\pgfpathlineto{\pgfqpoint{0.450150in}{2.687936in}}%
\pgfpathlineto{\pgfqpoint{0.451781in}{2.839761in}}%
\pgfpathlineto{\pgfqpoint{0.453975in}{2.872003in}}%
\pgfpathlineto{\pgfqpoint{0.456339in}{2.881553in}}%
\pgfpathlineto{\pgfqpoint{0.458888in}{2.885549in}}%
\pgfpathlineto{\pgfqpoint{0.462554in}{2.888171in}}%
\pgfpathlineto{\pgfqpoint{0.471046in}{2.890205in}}%
\pgfpathlineto{\pgfqpoint{0.490597in}{2.891263in}}%
\pgfpathlineto{\pgfqpoint{0.564556in}{2.891692in}}%
\pgfpathlineto{\pgfqpoint{1.569559in}{2.891759in}}%
\pgfpathlineto{\pgfqpoint{4.784679in}{2.890785in}}%
\pgfpathlineto{\pgfqpoint{4.791005in}{2.889098in}}%
\pgfpathlineto{\pgfqpoint{4.793910in}{2.885555in}}%
\pgfpathlineto{\pgfqpoint{4.795579in}{2.878366in}}%
\pgfpathlineto{\pgfqpoint{4.796850in}{2.858513in}}%
\pgfpathlineto{\pgfqpoint{4.796850in}{2.858513in}}%
\pgfusepath{stroke}%
\end{pgfscope}%
\begin{pgfscope}%
\pgfpathrectangle{\pgfqpoint{0.448634in}{0.402556in}}{\pgfqpoint{4.350661in}{2.489204in}} %
\pgfusepath{clip}%
\pgfsetrectcap%
\pgfsetroundjoin%
\pgfsetlinewidth{1.003750pt}%
\definecolor{currentstroke}{rgb}{0.121569,0.466667,0.705882}%
\pgfsetstrokecolor{currentstroke}%
\pgfsetdash{}{0pt}%
\pgfpathmoveto{\pgfqpoint{0.448634in}{2.896245in}}%
\pgfpathlineto{\pgfqpoint{0.448593in}{0.407043in}}%
\pgfpathlineto{\pgfqpoint{0.448593in}{0.407043in}}%
\pgfusepath{stroke}%
\end{pgfscope}%
\begin{pgfscope}%
\pgfpathrectangle{\pgfqpoint{0.448634in}{0.402556in}}{\pgfqpoint{4.350661in}{2.489204in}} %
\pgfusepath{clip}%
\pgfsetrectcap%
\pgfsetroundjoin%
\pgfsetlinewidth{1.003750pt}%
\definecolor{currentstroke}{rgb}{0.121569,0.466667,0.705882}%
\pgfsetstrokecolor{currentstroke}%
\pgfsetdash{}{0pt}%
\pgfpathmoveto{\pgfqpoint{0.576853in}{1.760817in}}%
\pgfpathlineto{\pgfqpoint{0.569394in}{1.840010in}}%
\pgfpathlineto{\pgfqpoint{0.563209in}{1.929338in}}%
\pgfpathlineto{\pgfqpoint{0.558592in}{2.028764in}}%
\pgfpathlineto{\pgfqpoint{0.555985in}{2.133265in}}%
\pgfpathlineto{\pgfqpoint{0.555566in}{2.237808in}}%
\pgfpathlineto{\pgfqpoint{0.557371in}{2.337352in}}%
\pgfpathlineto{\pgfqpoint{0.561096in}{2.424366in}}%
\pgfpathlineto{\pgfqpoint{0.566403in}{2.498791in}}%
\pgfpathlineto{\pgfqpoint{0.572909in}{2.560570in}}%
\pgfpathlineto{\pgfqpoint{0.580458in}{2.612119in}}%
\pgfpathlineto{\pgfqpoint{0.589086in}{2.655816in}}%
\pgfpathlineto{\pgfqpoint{0.598406in}{2.691589in}}%
\pgfpathlineto{\pgfqpoint{0.608613in}{2.721757in}}%
\pgfpathlineto{\pgfqpoint{0.619241in}{2.746278in}}%
\pgfpathlineto{\pgfqpoint{0.630817in}{2.767339in}}%
\pgfpathlineto{\pgfqpoint{0.642975in}{2.784884in}}%
\pgfpathlineto{\pgfqpoint{0.656813in}{2.800712in}}%
\pgfpathlineto{\pgfqpoint{0.672197in}{2.814549in}}%
\pgfpathlineto{\pgfqpoint{0.688853in}{2.826301in}}%
\pgfpathlineto{\pgfqpoint{0.706461in}{2.836076in}}%
\pgfpathlineto{\pgfqpoint{0.726804in}{2.844875in}}%
\pgfpathlineto{\pgfqpoint{0.751866in}{2.853203in}}%
\pgfpathlineto{\pgfqpoint{0.781631in}{2.860547in}}%
\pgfpathlineto{\pgfqpoint{0.818168in}{2.867054in}}%
\pgfpathlineto{\pgfqpoint{0.863581in}{2.872685in}}%
\pgfpathlineto{\pgfqpoint{0.922161in}{2.877518in}}%
\pgfpathlineto{\pgfqpoint{1.000391in}{2.881567in}}%
\pgfpathlineto{\pgfqpoint{1.111294in}{2.884881in}}%
\pgfpathlineto{\pgfqpoint{1.274428in}{2.887367in}}%
\pgfpathlineto{\pgfqpoint{1.552865in}{2.889263in}}%
\pgfpathlineto{\pgfqpoint{2.107573in}{2.890457in}}%
\pgfpathlineto{\pgfqpoint{3.343161in}{2.890573in}}%
\pgfpathlineto{\pgfqpoint{4.043615in}{2.888941in}}%
\pgfpathlineto{\pgfqpoint{4.289417in}{2.886404in}}%
\pgfpathlineto{\pgfqpoint{4.413375in}{2.883093in}}%
\pgfpathlineto{\pgfqpoint{4.489424in}{2.878997in}}%
\pgfpathlineto{\pgfqpoint{4.541451in}{2.874081in}}%
\pgfpathlineto{\pgfqpoint{4.578100in}{2.868470in}}%
\pgfpathlineto{\pgfqpoint{4.605818in}{2.862092in}}%
\pgfpathlineto{\pgfqpoint{4.626725in}{2.855245in}}%
\pgfpathlineto{\pgfqpoint{4.644925in}{2.847018in}}%
\pgfpathlineto{\pgfqpoint{4.660241in}{2.837590in}}%
\pgfpathlineto{\pgfqpoint{4.672623in}{2.827468in}}%
\pgfpathlineto{\pgfqpoint{4.683751in}{2.815592in}}%
\pgfpathlineto{\pgfqpoint{4.693406in}{2.802135in}}%
\pgfpathlineto{\pgfqpoint{4.702740in}{2.785343in}}%
\pgfpathlineto{\pgfqpoint{4.711277in}{2.765194in}}%
\pgfpathlineto{\pgfqpoint{4.719482in}{2.739484in}}%
\pgfpathlineto{\pgfqpoint{4.726293in}{2.710657in}}%
\pgfpathlineto{\pgfqpoint{4.733259in}{2.671643in}}%
\pgfpathlineto{\pgfqpoint{4.739604in}{2.622396in}}%
\pgfpathlineto{\pgfqpoint{4.745236in}{2.560504in}}%
\pgfpathlineto{\pgfqpoint{4.750164in}{2.481052in}}%
\pgfpathlineto{\pgfqpoint{4.754367in}{2.376618in}}%
\pgfpathlineto{\pgfqpoint{4.757443in}{2.242249in}}%
\pgfpathlineto{\pgfqpoint{4.758977in}{2.075483in}}%
\pgfpathlineto{\pgfqpoint{4.758447in}{1.888795in}}%
\pgfpathlineto{\pgfqpoint{4.755756in}{1.707111in}}%
\pgfpathlineto{\pgfqpoint{4.750925in}{1.532957in}}%
\pgfpathlineto{\pgfqpoint{4.744785in}{1.398726in}}%
\pgfpathlineto{\pgfqpoint{4.737575in}{1.289516in}}%
\pgfpathlineto{\pgfqpoint{4.728714in}{1.190470in}}%
\pgfpathlineto{\pgfqpoint{4.719652in}{1.116521in}}%
\pgfpathlineto{\pgfqpoint{4.710036in}{1.055276in}}%
\pgfpathlineto{\pgfqpoint{4.699503in}{1.001861in}}%
\pgfpathlineto{\pgfqpoint{4.689040in}{0.958690in}}%
\pgfpathlineto{\pgfqpoint{4.677219in}{0.918600in}}%
\pgfpathlineto{\pgfqpoint{4.664034in}{0.881749in}}%
\pgfpathlineto{\pgfqpoint{4.650584in}{0.850492in}}%
\pgfpathlineto{\pgfqpoint{4.636303in}{0.822570in}}%
\pgfpathlineto{\pgfqpoint{4.620207in}{0.795974in}}%
\pgfpathlineto{\pgfqpoint{4.603640in}{0.772901in}}%
\pgfpathlineto{\pgfqpoint{4.585488in}{0.751446in}}%
\pgfpathlineto{\pgfqpoint{4.565874in}{0.731749in}}%
\pgfpathlineto{\pgfqpoint{4.544964in}{0.713879in}}%
\pgfpathlineto{\pgfqpoint{4.522958in}{0.697824in}}%
\pgfpathlineto{\pgfqpoint{4.496157in}{0.681290in}}%
\pgfpathlineto{\pgfqpoint{4.470397in}{0.667953in}}%
\pgfpathlineto{\pgfqpoint{4.439961in}{0.654509in}}%
\pgfpathlineto{\pgfqpoint{4.406841in}{0.642281in}}%
\pgfpathlineto{\pgfqpoint{4.369009in}{0.630748in}}%
\pgfpathlineto{\pgfqpoint{4.326489in}{0.620226in}}%
\pgfpathlineto{\pgfqpoint{4.279327in}{0.610949in}}%
\pgfpathlineto{\pgfqpoint{4.227576in}{0.603085in}}%
\pgfpathlineto{\pgfqpoint{4.173450in}{0.597063in}}%
\pgfpathlineto{\pgfqpoint{4.110511in}{0.592203in}}%
\pgfpathlineto{\pgfqpoint{4.047471in}{0.589537in}}%
\pgfpathlineto{\pgfqpoint{3.977867in}{0.588624in}}%
\pgfpathlineto{\pgfqpoint{3.906093in}{0.589934in}}%
\pgfpathlineto{\pgfqpoint{3.834377in}{0.593496in}}%
\pgfpathlineto{\pgfqpoint{3.767120in}{0.599067in}}%
\pgfpathlineto{\pgfqpoint{3.704364in}{0.606392in}}%
\pgfpathlineto{\pgfqpoint{3.678516in}{0.610510in}}%
\pgfpathlineto{\pgfqpoint{3.620438in}{0.620500in}}%
\pgfpathlineto{\pgfqpoint{3.586319in}{0.628207in}}%
\pgfpathlineto{\pgfqpoint{3.495240in}{0.652428in}}%
\pgfpathlineto{\pgfqpoint{3.451528in}{0.667583in}}%
\pgfpathlineto{\pgfqpoint{3.408538in}{0.685220in}}%
\pgfpathlineto{\pgfqpoint{3.374594in}{0.702001in}}%
\pgfpathlineto{\pgfqpoint{3.345407in}{0.718682in}}%
\pgfpathlineto{\pgfqpoint{3.315236in}{0.738520in}}%
\pgfpathlineto{\pgfqpoint{3.288127in}{0.759290in}}%
\pgfpathlineto{\pgfqpoint{3.264004in}{0.780551in}}%
\pgfpathlineto{\pgfqpoint{3.241208in}{0.803648in}}%
\pgfpathlineto{\pgfqpoint{3.219894in}{0.828530in}}%
\pgfpathlineto{\pgfqpoint{3.200189in}{0.855091in}}%
\pgfpathlineto{\pgfqpoint{3.182177in}{0.883182in}}%
\pgfpathlineto{\pgfqpoint{3.165906in}{0.912633in}}%
\pgfpathlineto{\pgfqpoint{3.150351in}{0.945448in}}%
\pgfpathlineto{\pgfqpoint{3.136682in}{0.979345in}}%
\pgfpathlineto{\pgfqpoint{3.124073in}{1.016460in}}%
\pgfpathlineto{\pgfqpoint{3.112834in}{1.056769in}}%
\pgfpathlineto{\pgfqpoint{3.103046in}{1.100146in}}%
\pgfpathlineto{\pgfqpoint{3.095343in}{1.144071in}}%
\pgfpathlineto{\pgfqpoint{3.089208in}{1.190837in}}%
\pgfpathlineto{\pgfqpoint{3.084595in}{1.242838in}}%
\pgfpathlineto{\pgfqpoint{3.082137in}{1.295031in}}%
\pgfpathlineto{\pgfqpoint{3.081687in}{1.349787in}}%
\pgfpathlineto{\pgfqpoint{3.083451in}{1.406998in}}%
\pgfpathlineto{\pgfqpoint{3.087181in}{1.461589in}}%
\pgfpathlineto{\pgfqpoint{3.093485in}{1.520888in}}%
\pgfpathlineto{\pgfqpoint{3.101823in}{1.577334in}}%
\pgfpathlineto{\pgfqpoint{3.111930in}{1.630856in}}%
\pgfpathlineto{\pgfqpoint{3.124690in}{1.686208in}}%
\pgfpathlineto{\pgfqpoint{3.139178in}{1.738395in}}%
\pgfpathlineto{\pgfqpoint{3.155145in}{1.787366in}}%
\pgfpathlineto{\pgfqpoint{3.172353in}{1.833085in}}%
\pgfpathlineto{\pgfqpoint{3.191618in}{1.877716in}}%
\pgfpathlineto{\pgfqpoint{3.214026in}{1.923261in}}%
\pgfpathlineto{\pgfqpoint{3.236214in}{1.963157in}}%
\pgfpathlineto{\pgfqpoint{3.260178in}{2.001684in}}%
\pgfpathlineto{\pgfqpoint{3.285814in}{2.038776in}}%
\pgfpathlineto{\pgfqpoint{3.314415in}{2.076285in}}%
\pgfpathlineto{\pgfqpoint{3.348944in}{2.117711in}}%
\pgfpathlineto{\pgfqpoint{3.417133in}{2.198022in}}%
\pgfpathlineto{\pgfqpoint{3.426053in}{2.212128in}}%
\pgfpathlineto{\pgfqpoint{3.430798in}{2.223297in}}%
\pgfpathlineto{\pgfqpoint{3.432034in}{2.230603in}}%
\pgfpathlineto{\pgfqpoint{3.430773in}{2.237856in}}%
\pgfpathlineto{\pgfqpoint{3.426621in}{2.243526in}}%
\pgfpathlineto{\pgfqpoint{3.420908in}{2.247084in}}%
\pgfpathlineto{\pgfqpoint{3.412501in}{2.249583in}}%
\pgfpathlineto{\pgfqpoint{3.399499in}{2.250689in}}%
\pgfpathlineto{\pgfqpoint{3.384305in}{2.249671in}}%
\pgfpathlineto{\pgfqpoint{3.364985in}{2.246098in}}%
\pgfpathlineto{\pgfqpoint{3.341804in}{2.239342in}}%
\pgfpathlineto{\pgfqpoint{3.317109in}{2.229682in}}%
\pgfpathlineto{\pgfqpoint{3.291104in}{2.216986in}}%
\pgfpathlineto{\pgfqpoint{3.265928in}{2.202261in}}%
\pgfpathlineto{\pgfqpoint{3.239805in}{2.184361in}}%
\pgfpathlineto{\pgfqpoint{3.214775in}{2.164519in}}%
\pgfpathlineto{\pgfqpoint{3.190900in}{2.142893in}}%
\pgfpathlineto{\pgfqpoint{3.166657in}{2.117912in}}%
\pgfpathlineto{\pgfqpoint{3.143835in}{2.091233in}}%
\pgfpathlineto{\pgfqpoint{3.121079in}{2.061107in}}%
\pgfpathlineto{\pgfqpoint{3.099952in}{2.029463in}}%
\pgfpathlineto{\pgfqpoint{3.079251in}{1.994406in}}%
\pgfpathlineto{\pgfqpoint{3.059218in}{1.955915in}}%
\pgfpathlineto{\pgfqpoint{3.040058in}{1.914015in}}%
\pgfpathlineto{\pgfqpoint{3.022809in}{1.871041in}}%
\pgfpathlineto{\pgfqpoint{3.005790in}{1.822536in}}%
\pgfpathlineto{\pgfqpoint{2.990067in}{1.770819in}}%
\pgfpathlineto{\pgfqpoint{2.975708in}{1.715979in}}%
\pgfpathlineto{\pgfqpoint{2.962284in}{1.655680in}}%
\pgfpathlineto{\pgfqpoint{2.950496in}{1.592386in}}%
\pgfpathlineto{\pgfqpoint{2.940383in}{1.526185in}}%
\pgfpathlineto{\pgfqpoint{2.931745in}{1.454681in}}%
\pgfpathlineto{\pgfqpoint{2.925082in}{1.380399in}}%
\pgfpathlineto{\pgfqpoint{2.920647in}{1.305899in}}%
\pgfpathlineto{\pgfqpoint{2.918444in}{1.231270in}}%
\pgfpathlineto{\pgfqpoint{2.918545in}{1.159087in}}%
\pgfpathlineto{\pgfqpoint{2.920787in}{1.091931in}}%
\pgfpathlineto{\pgfqpoint{2.925177in}{1.027412in}}%
\pgfpathlineto{\pgfqpoint{2.931192in}{0.970580in}}%
\pgfpathlineto{\pgfqpoint{2.938760in}{0.919034in}}%
\pgfpathlineto{\pgfqpoint{2.947651in}{0.872852in}}%
\pgfpathlineto{\pgfqpoint{2.958213in}{0.829714in}}%
\pgfpathlineto{\pgfqpoint{2.969670in}{0.792114in}}%
\pgfpathlineto{\pgfqpoint{2.982463in}{0.757773in}}%
\pgfpathlineto{\pgfqpoint{2.996425in}{0.726812in}}%
\pgfpathlineto{\pgfqpoint{3.011299in}{0.699300in}}%
\pgfpathlineto{\pgfqpoint{3.026739in}{0.675225in}}%
\pgfpathlineto{\pgfqpoint{3.043828in}{0.652656in}}%
\pgfpathlineto{\pgfqpoint{3.062495in}{0.631788in}}%
\pgfpathlineto{\pgfqpoint{3.082602in}{0.612753in}}%
\pgfpathlineto{\pgfqpoint{3.103961in}{0.595592in}}%
\pgfpathlineto{\pgfqpoint{3.128268in}{0.579069in}}%
\pgfpathlineto{\pgfqpoint{3.153537in}{0.564554in}}%
\pgfpathlineto{\pgfqpoint{3.181571in}{0.550952in}}%
\pgfpathlineto{\pgfqpoint{3.214371in}{0.537647in}}%
\pgfpathlineto{\pgfqpoint{3.249846in}{0.525712in}}%
\pgfpathlineto{\pgfqpoint{3.290011in}{0.514571in}}%
\pgfpathlineto{\pgfqpoint{3.334820in}{0.504423in}}%
\pgfpathlineto{\pgfqpoint{3.386372in}{0.494999in}}%
\pgfpathlineto{\pgfqpoint{3.446798in}{0.486257in}}%
\pgfpathlineto{\pgfqpoint{3.518243in}{0.478282in}}%
\pgfpathlineto{\pgfqpoint{3.600685in}{0.471409in}}%
\pgfpathlineto{\pgfqpoint{3.696268in}{0.465713in}}%
\pgfpathlineto{\pgfqpoint{3.807144in}{0.461369in}}%
\pgfpathlineto{\pgfqpoint{3.933291in}{0.458719in}}%
\pgfpathlineto{\pgfqpoint{4.063808in}{0.458211in}}%
\pgfpathlineto{\pgfqpoint{4.187792in}{0.459914in}}%
\pgfpathlineto{\pgfqpoint{4.294335in}{0.463521in}}%
\pgfpathlineto{\pgfqpoint{4.381234in}{0.468574in}}%
\pgfpathlineto{\pgfqpoint{4.450636in}{0.474701in}}%
\pgfpathlineto{\pgfqpoint{4.506850in}{0.481799in}}%
\pgfpathlineto{\pgfqpoint{4.552009in}{0.489658in}}%
\pgfpathlineto{\pgfqpoint{4.588239in}{0.498115in}}%
\pgfpathlineto{\pgfqpoint{4.617656in}{0.507110in}}%
\pgfpathlineto{\pgfqpoint{4.642328in}{0.516843in}}%
\pgfpathlineto{\pgfqpoint{4.664194in}{0.527940in}}%
\pgfpathlineto{\pgfqpoint{4.681238in}{0.538945in}}%
\pgfpathlineto{\pgfqpoint{4.697164in}{0.551953in}}%
\pgfpathlineto{\pgfqpoint{4.710076in}{0.565289in}}%
\pgfpathlineto{\pgfqpoint{4.721578in}{0.580218in}}%
\pgfpathlineto{\pgfqpoint{4.731557in}{0.596521in}}%
\pgfpathlineto{\pgfqpoint{4.741000in}{0.616134in}}%
\pgfpathlineto{\pgfqpoint{4.749521in}{0.639027in}}%
\pgfpathlineto{\pgfqpoint{4.757522in}{0.667450in}}%
\pgfpathlineto{\pgfqpoint{4.764572in}{0.701345in}}%
\pgfpathlineto{\pgfqpoint{4.770840in}{0.743043in}}%
\pgfpathlineto{\pgfqpoint{4.776327in}{0.794934in}}%
\pgfpathlineto{\pgfqpoint{4.781278in}{0.864398in}}%
\pgfpathlineto{\pgfqpoint{4.785468in}{0.956371in}}%
\pgfpathlineto{\pgfqpoint{4.789000in}{1.085745in}}%
\pgfpathlineto{\pgfqpoint{4.791852in}{1.277385in}}%
\pgfpathlineto{\pgfqpoint{4.793959in}{1.581057in}}%
\pgfpathlineto{\pgfqpoint{4.794962in}{2.071429in}}%
\pgfpathlineto{\pgfqpoint{4.793967in}{2.559311in}}%
\pgfpathlineto{\pgfqpoint{4.791733in}{2.745981in}}%
\pgfpathlineto{\pgfqpoint{4.788955in}{2.818091in}}%
\pgfpathlineto{\pgfqpoint{4.785731in}{2.850227in}}%
\pgfpathlineto{\pgfqpoint{4.781879in}{2.867057in}}%
\pgfpathlineto{\pgfqpoint{4.777744in}{2.875780in}}%
\pgfpathlineto{\pgfqpoint{4.773097in}{2.880982in}}%
\pgfpathlineto{\pgfqpoint{4.767363in}{2.884504in}}%
\pgfpathlineto{\pgfqpoint{4.756853in}{2.887622in}}%
\pgfpathlineto{\pgfqpoint{4.739548in}{2.889639in}}%
\pgfpathlineto{\pgfqpoint{4.704762in}{2.890882in}}%
\pgfpathlineto{\pgfqpoint{4.602524in}{2.891538in}}%
\pgfpathlineto{\pgfqpoint{3.952100in}{2.891742in}}%
\pgfpathlineto{\pgfqpoint{0.617321in}{2.890753in}}%
\pgfpathlineto{\pgfqpoint{0.549910in}{2.888858in}}%
\pgfpathlineto{\pgfqpoint{0.521735in}{2.886179in}}%
\pgfpathlineto{\pgfqpoint{0.504666in}{2.882389in}}%
\pgfpathlineto{\pgfqpoint{0.494501in}{2.878011in}}%
\pgfpathlineto{\pgfqpoint{0.487180in}{2.872667in}}%
\pgfpathlineto{\pgfqpoint{0.481152in}{2.865519in}}%
\pgfpathlineto{\pgfqpoint{0.475664in}{2.854804in}}%
\pgfpathlineto{\pgfqpoint{0.471318in}{2.840737in}}%
\pgfpathlineto{\pgfqpoint{0.467301in}{2.818823in}}%
\pgfpathlineto{\pgfqpoint{0.463927in}{2.786700in}}%
\pgfpathlineto{\pgfqpoint{0.460918in}{2.734544in}}%
\pgfpathlineto{\pgfqpoint{0.458363in}{2.647473in}}%
\pgfpathlineto{\pgfqpoint{0.456575in}{2.523031in}}%
\pgfpathlineto{\pgfqpoint{0.456575in}{2.523031in}}%
\pgfusepath{stroke}%
\end{pgfscope}%
\begin{pgfscope}%
\pgfpathrectangle{\pgfqpoint{0.448634in}{0.402556in}}{\pgfqpoint{4.350661in}{2.489204in}} %
\pgfusepath{clip}%
\pgfsetrectcap%
\pgfsetroundjoin%
\pgfsetlinewidth{1.003750pt}%
\definecolor{currentstroke}{rgb}{0.121569,0.466667,0.705882}%
\pgfsetstrokecolor{currentstroke}%
\pgfsetdash{}{0pt}%
\pgfpathmoveto{\pgfqpoint{4.798840in}{2.852369in}}%
\pgfpathlineto{\pgfqpoint{4.797564in}{2.889610in}}%
\pgfpathlineto{\pgfqpoint{4.796215in}{2.891483in}}%
\pgfpathlineto{\pgfqpoint{4.787551in}{2.891760in}}%
\pgfpathlineto{\pgfqpoint{0.452128in}{2.891659in}}%
\pgfpathlineto{\pgfqpoint{0.450530in}{2.890082in}}%
\pgfpathlineto{\pgfqpoint{0.449454in}{2.882763in}}%
\pgfpathlineto{\pgfqpoint{0.448970in}{2.845432in}}%
\pgfpathlineto{\pgfqpoint{0.448743in}{2.494454in}}%
\pgfpathlineto{\pgfqpoint{0.449624in}{0.615107in}}%
\pgfpathlineto{\pgfqpoint{0.451433in}{0.510586in}}%
\pgfpathlineto{\pgfqpoint{0.453993in}{0.473374in}}%
\pgfpathlineto{\pgfqpoint{0.457406in}{0.453868in}}%
\pgfpathlineto{\pgfqpoint{0.461540in}{0.442384in}}%
\pgfpathlineto{\pgfqpoint{0.466739in}{0.434437in}}%
\pgfpathlineto{\pgfqpoint{0.473595in}{0.428350in}}%
\pgfpathlineto{\pgfqpoint{0.483492in}{0.423244in}}%
\pgfpathlineto{\pgfqpoint{0.491854in}{0.420501in}}%
\pgfpathlineto{\pgfqpoint{0.491854in}{0.420501in}}%
\pgfusepath{stroke}%
\end{pgfscope}%
\begin{pgfscope}%
\pgfpathrectangle{\pgfqpoint{0.448634in}{0.402556in}}{\pgfqpoint{4.350661in}{2.489204in}} %
\pgfusepath{clip}%
\pgfsetrectcap%
\pgfsetroundjoin%
\pgfsetlinewidth{1.003750pt}%
\definecolor{currentstroke}{rgb}{0.121569,0.466667,0.705882}%
\pgfsetstrokecolor{currentstroke}%
\pgfsetdash{}{0pt}%
\pgfpathmoveto{\pgfqpoint{0.456424in}{1.370137in}}%
\pgfpathlineto{\pgfqpoint{0.459610in}{1.118755in}}%
\pgfpathlineto{\pgfqpoint{0.463695in}{0.962007in}}%
\pgfpathlineto{\pgfqpoint{0.468519in}{0.857610in}}%
\pgfpathlineto{\pgfqpoint{0.474082in}{0.783210in}}%
\pgfpathlineto{\pgfqpoint{0.480226in}{0.728906in}}%
\pgfpathlineto{\pgfqpoint{0.486970in}{0.687306in}}%
\pgfpathlineto{\pgfqpoint{0.494537in}{0.653558in}}%
\pgfpathlineto{\pgfqpoint{0.503107in}{0.625355in}}%
\pgfpathlineto{\pgfqpoint{0.512193in}{0.602750in}}%
\pgfpathlineto{\pgfqpoint{0.522200in}{0.583508in}}%
\pgfpathlineto{\pgfqpoint{0.534108in}{0.565743in}}%
\pgfpathlineto{\pgfqpoint{0.546263in}{0.551507in}}%
\pgfpathlineto{\pgfqpoint{0.559728in}{0.538907in}}%
\pgfpathlineto{\pgfqpoint{0.576129in}{0.526693in}}%
\pgfpathlineto{\pgfqpoint{0.595483in}{0.515351in}}%
\pgfpathlineto{\pgfqpoint{0.617681in}{0.505147in}}%
\pgfpathlineto{\pgfqpoint{0.642568in}{0.496153in}}%
\pgfpathlineto{\pgfqpoint{0.672126in}{0.487778in}}%
\pgfpathlineto{\pgfqpoint{0.708443in}{0.479824in}}%
\pgfpathlineto{\pgfqpoint{0.753649in}{0.472325in}}%
\pgfpathlineto{\pgfqpoint{0.807717in}{0.465660in}}%
\pgfpathlineto{\pgfqpoint{0.877116in}{0.459475in}}%
\pgfpathlineto{\pgfqpoint{0.961828in}{0.454230in}}%
\pgfpathlineto{\pgfqpoint{1.068351in}{0.449916in}}%
\pgfpathlineto{\pgfqpoint{1.201018in}{0.446839in}}%
\pgfpathlineto{\pgfqpoint{1.357637in}{0.445481in}}%
\pgfpathlineto{\pgfqpoint{1.525135in}{0.446232in}}%
\pgfpathlineto{\pgfqpoint{1.686088in}{0.449142in}}%
\pgfpathlineto{\pgfqpoint{1.823074in}{0.453747in}}%
\pgfpathlineto{\pgfqpoint{1.938245in}{0.459764in}}%
\pgfpathlineto{\pgfqpoint{2.031582in}{0.466759in}}%
\pgfpathlineto{\pgfqpoint{2.109580in}{0.474745in}}%
\pgfpathlineto{\pgfqpoint{2.174384in}{0.483535in}}%
\pgfpathlineto{\pgfqpoint{2.228139in}{0.492940in}}%
\pgfpathlineto{\pgfqpoint{2.275119in}{0.503356in}}%
\pgfpathlineto{\pgfqpoint{2.315282in}{0.514501in}}%
\pgfpathlineto{\pgfqpoint{2.350698in}{0.526659in}}%
\pgfpathlineto{\pgfqpoint{2.381320in}{0.539536in}}%
\pgfpathlineto{\pgfqpoint{2.407164in}{0.552659in}}%
\pgfpathlineto{\pgfqpoint{2.430226in}{0.566639in}}%
\pgfpathlineto{\pgfqpoint{2.452282in}{0.582602in}}%
\pgfpathlineto{\pgfqpoint{2.471391in}{0.599069in}}%
\pgfpathlineto{\pgfqpoint{2.489240in}{0.617293in}}%
\pgfpathlineto{\pgfqpoint{2.505678in}{0.637180in}}%
\pgfpathlineto{\pgfqpoint{2.520620in}{0.658557in}}%
\pgfpathlineto{\pgfqpoint{2.535213in}{0.683314in}}%
\pgfpathlineto{\pgfqpoint{2.549115in}{0.711484in}}%
\pgfpathlineto{\pgfqpoint{2.562091in}{0.743004in}}%
\pgfpathlineto{\pgfqpoint{2.574020in}{0.777751in}}%
\pgfpathlineto{\pgfqpoint{2.585502in}{0.817970in}}%
\pgfpathlineto{\pgfqpoint{2.596809in}{0.866038in}}%
\pgfpathlineto{\pgfqpoint{2.607562in}{0.921948in}}%
\pgfpathlineto{\pgfqpoint{2.617925in}{0.988098in}}%
\pgfpathlineto{\pgfqpoint{2.627958in}{1.066918in}}%
\pgfpathlineto{\pgfqpoint{2.637941in}{1.163320in}}%
\pgfpathlineto{\pgfqpoint{2.648424in}{1.287199in}}%
\pgfpathlineto{\pgfqpoint{2.660103in}{1.453438in}}%
\pgfpathlineto{\pgfqpoint{2.674773in}{1.696801in}}%
\pgfpathlineto{\pgfqpoint{2.687716in}{1.945279in}}%
\pgfpathlineto{\pgfqpoint{2.692670in}{2.079573in}}%
\pgfpathlineto{\pgfqpoint{2.693829in}{2.166682in}}%
\pgfpathlineto{\pgfqpoint{2.692565in}{2.233870in}}%
\pgfpathlineto{\pgfqpoint{2.689436in}{2.286015in}}%
\pgfpathlineto{\pgfqpoint{2.684859in}{2.327999in}}%
\pgfpathlineto{\pgfqpoint{2.678725in}{2.364664in}}%
\pgfpathlineto{\pgfqpoint{2.671356in}{2.395897in}}%
\pgfpathlineto{\pgfqpoint{2.662489in}{2.423981in}}%
\pgfpathlineto{\pgfqpoint{2.652361in}{2.448778in}}%
\pgfpathlineto{\pgfqpoint{2.641365in}{2.470245in}}%
\pgfpathlineto{\pgfqpoint{2.628643in}{2.490425in}}%
\pgfpathlineto{\pgfqpoint{2.614279in}{2.509106in}}%
\pgfpathlineto{\pgfqpoint{2.598443in}{2.526159in}}%
\pgfpathlineto{\pgfqpoint{2.579590in}{2.543005in}}%
\pgfpathlineto{\pgfqpoint{2.559532in}{2.557923in}}%
\pgfpathlineto{\pgfqpoint{2.536602in}{2.572183in}}%
\pgfpathlineto{\pgfqpoint{2.510850in}{2.585538in}}%
\pgfpathlineto{\pgfqpoint{2.482360in}{2.597837in}}%
\pgfpathlineto{\pgfqpoint{2.449134in}{2.609683in}}%
\pgfpathlineto{\pgfqpoint{2.411184in}{2.620696in}}%
\pgfpathlineto{\pgfqpoint{2.368552in}{2.630606in}}%
\pgfpathlineto{\pgfqpoint{2.321294in}{2.639221in}}%
\pgfpathlineto{\pgfqpoint{2.269467in}{2.646399in}}%
\pgfpathlineto{\pgfqpoint{2.210954in}{2.652193in}}%
\pgfpathlineto{\pgfqpoint{2.147967in}{2.656153in}}%
\pgfpathlineto{\pgfqpoint{2.080556in}{2.658135in}}%
\pgfpathlineto{\pgfqpoint{2.010948in}{2.657971in}}%
\pgfpathlineto{\pgfqpoint{1.939195in}{2.655572in}}%
\pgfpathlineto{\pgfqpoint{1.867527in}{2.650913in}}%
\pgfpathlineto{\pgfqpoint{1.798171in}{2.644140in}}%
\pgfpathlineto{\pgfqpoint{1.733341in}{2.635606in}}%
\pgfpathlineto{\pgfqpoint{1.673075in}{2.625521in}}%
\pgfpathlineto{\pgfqpoint{1.615274in}{2.613610in}}%
\pgfpathlineto{\pgfqpoint{1.562133in}{2.600402in}}%
\pgfpathlineto{\pgfqpoint{1.513681in}{2.586139in}}%
\pgfpathlineto{\pgfqpoint{1.467862in}{2.570344in}}%
\pgfpathlineto{\pgfqpoint{1.426794in}{2.553923in}}%
\pgfpathlineto{\pgfqpoint{1.388447in}{2.536289in}}%
\pgfpathlineto{\pgfqpoint{1.352878in}{2.517566in}}%
\pgfpathlineto{\pgfqpoint{1.320128in}{2.497922in}}%
\pgfpathlineto{\pgfqpoint{1.288379in}{2.476236in}}%
\pgfpathlineto{\pgfqpoint{1.259592in}{2.453861in}}%
\pgfpathlineto{\pgfqpoint{1.232050in}{2.429520in}}%
\pgfpathlineto{\pgfqpoint{1.207527in}{2.404898in}}%
\pgfpathlineto{\pgfqpoint{1.184409in}{2.378557in}}%
\pgfpathlineto{\pgfqpoint{1.162828in}{2.350561in}}%
\pgfpathlineto{\pgfqpoint{1.142891in}{2.321011in}}%
\pgfpathlineto{\pgfqpoint{1.124675in}{2.290041in}}%
\pgfpathlineto{\pgfqpoint{1.108225in}{2.257802in}}%
\pgfpathlineto{\pgfqpoint{1.092639in}{2.222199in}}%
\pgfpathlineto{\pgfqpoint{1.079059in}{2.185535in}}%
\pgfpathlineto{\pgfqpoint{1.067443in}{2.147998in}}%
\pgfpathlineto{\pgfqpoint{1.057187in}{2.107348in}}%
\pgfpathlineto{\pgfqpoint{1.049004in}{2.066086in}}%
\pgfpathlineto{\pgfqpoint{1.042513in}{2.021906in}}%
\pgfpathlineto{\pgfqpoint{1.038177in}{1.977382in}}%
\pgfpathlineto{\pgfqpoint{1.035866in}{1.930167in}}%
\pgfpathlineto{\pgfqpoint{1.035826in}{1.882878in}}%
\pgfpathlineto{\pgfqpoint{1.038031in}{1.835656in}}%
\pgfpathlineto{\pgfqpoint{1.042474in}{1.788641in}}%
\pgfpathlineto{\pgfqpoint{1.049176in}{1.741979in}}%
\pgfpathlineto{\pgfqpoint{1.057644in}{1.698239in}}%
\pgfpathlineto{\pgfqpoint{1.068221in}{1.655105in}}%
\pgfpathlineto{\pgfqpoint{1.080962in}{1.612745in}}%
\pgfpathlineto{\pgfqpoint{1.095031in}{1.573617in}}%
\pgfpathlineto{\pgfqpoint{1.111115in}{1.535520in}}%
\pgfpathlineto{\pgfqpoint{1.128118in}{1.500775in}}%
\pgfpathlineto{\pgfqpoint{1.146930in}{1.467274in}}%
\pgfpathlineto{\pgfqpoint{1.167531in}{1.435181in}}%
\pgfpathlineto{\pgfqpoint{1.189874in}{1.404652in}}%
\pgfpathlineto{\pgfqpoint{1.213884in}{1.375828in}}%
\pgfpathlineto{\pgfqpoint{1.237817in}{1.350457in}}%
\pgfpathlineto{\pgfqpoint{1.264748in}{1.325237in}}%
\pgfpathlineto{\pgfqpoint{1.292991in}{1.301972in}}%
\pgfpathlineto{\pgfqpoint{1.322398in}{1.280678in}}%
\pgfpathlineto{\pgfqpoint{1.352820in}{1.261340in}}%
\pgfpathlineto{\pgfqpoint{1.386095in}{1.242889in}}%
\pgfpathlineto{\pgfqpoint{1.420190in}{1.226516in}}%
\pgfpathlineto{\pgfqpoint{1.457024in}{1.211329in}}%
\pgfpathlineto{\pgfqpoint{1.496554in}{1.197536in}}%
\pgfpathlineto{\pgfqpoint{1.538719in}{1.185287in}}%
\pgfpathlineto{\pgfqpoint{1.583441in}{1.174641in}}%
\pgfpathlineto{\pgfqpoint{1.634929in}{1.164775in}}%
\pgfpathlineto{\pgfqpoint{1.706063in}{1.153745in}}%
\pgfpathlineto{\pgfqpoint{1.768492in}{1.143417in}}%
\pgfpathlineto{\pgfqpoint{1.796122in}{1.136567in}}%
\pgfpathlineto{\pgfqpoint{1.812683in}{1.130481in}}%
\pgfpathlineto{\pgfqpoint{1.824471in}{1.124102in}}%
\pgfpathlineto{\pgfqpoint{1.833209in}{1.116741in}}%
\pgfpathlineto{\pgfqpoint{1.838498in}{1.108890in}}%
\pgfpathlineto{\pgfqpoint{1.840588in}{1.101849in}}%
\pgfpathlineto{\pgfqpoint{1.840619in}{1.094412in}}%
\pgfpathlineto{\pgfqpoint{1.837931in}{1.084986in}}%
\pgfpathlineto{\pgfqpoint{1.833246in}{1.076615in}}%
\pgfpathlineto{\pgfqpoint{1.825819in}{1.067542in}}%
\pgfpathlineto{\pgfqpoint{1.813813in}{1.056850in}}%
\pgfpathlineto{\pgfqpoint{1.798819in}{1.046763in}}%
\pgfpathlineto{\pgfqpoint{1.781016in}{1.037462in}}%
\pgfpathlineto{\pgfqpoint{1.758447in}{1.028391in}}%
\pgfpathlineto{\pgfqpoint{1.733203in}{1.020815in}}%
\pgfpathlineto{\pgfqpoint{1.705410in}{1.014872in}}%
\pgfpathlineto{\pgfqpoint{1.675178in}{1.010714in}}%
\pgfpathlineto{\pgfqpoint{1.642610in}{1.008507in}}%
\pgfpathlineto{\pgfqpoint{1.607809in}{1.008432in}}%
\pgfpathlineto{\pgfqpoint{1.570886in}{1.010691in}}%
\pgfpathlineto{\pgfqpoint{1.534118in}{1.015181in}}%
\pgfpathlineto{\pgfqpoint{1.495454in}{1.022233in}}%
\pgfpathlineto{\pgfqpoint{1.457161in}{1.031563in}}%
\pgfpathlineto{\pgfqpoint{1.419337in}{1.043132in}}%
\pgfpathlineto{\pgfqpoint{1.382089in}{1.056929in}}%
\pgfpathlineto{\pgfqpoint{1.347544in}{1.072019in}}%
\pgfpathlineto{\pgfqpoint{1.313727in}{1.089133in}}%
\pgfpathlineto{\pgfqpoint{1.280762in}{1.108299in}}%
\pgfpathlineto{\pgfqpoint{1.248782in}{1.129536in}}%
\pgfpathlineto{\pgfqpoint{1.219708in}{1.151422in}}%
\pgfpathlineto{\pgfqpoint{1.191752in}{1.175138in}}%
\pgfpathlineto{\pgfqpoint{1.165031in}{1.200649in}}%
\pgfpathlineto{\pgfqpoint{1.139653in}{1.227898in}}%
\pgfpathlineto{\pgfqpoint{1.115714in}{1.256800in}}%
\pgfpathlineto{\pgfqpoint{1.093288in}{1.287251in}}%
\pgfpathlineto{\pgfqpoint{1.071178in}{1.321163in}}%
\pgfpathlineto{\pgfqpoint{1.050868in}{1.356520in}}%
\pgfpathlineto{\pgfqpoint{1.032365in}{1.393152in}}%
\pgfpathlineto{\pgfqpoint{1.014718in}{1.433142in}}%
\pgfpathlineto{\pgfqpoint{0.999024in}{1.474185in}}%
\pgfpathlineto{\pgfqpoint{0.984506in}{1.518461in}}%
\pgfpathlineto{\pgfqpoint{0.972010in}{1.563537in}}%
\pgfpathlineto{\pgfqpoint{0.960944in}{1.611678in}}%
\pgfpathlineto{\pgfqpoint{0.951530in}{1.662824in}}%
\pgfpathlineto{\pgfqpoint{0.944286in}{1.714431in}}%
\pgfpathlineto{\pgfqpoint{0.938950in}{1.768847in}}%
\pgfpathlineto{\pgfqpoint{0.935870in}{1.823491in}}%
\pgfpathlineto{\pgfqpoint{0.935034in}{1.878240in}}%
\pgfpathlineto{\pgfqpoint{0.936466in}{1.932973in}}%
\pgfpathlineto{\pgfqpoint{0.940005in}{1.985084in}}%
\pgfpathlineto{\pgfqpoint{0.945759in}{2.036935in}}%
\pgfpathlineto{\pgfqpoint{0.953410in}{2.085938in}}%
\pgfpathlineto{\pgfqpoint{0.962764in}{2.132000in}}%
\pgfpathlineto{\pgfqpoint{0.974287in}{2.177414in}}%
\pgfpathlineto{\pgfqpoint{0.987332in}{2.219653in}}%
\pgfpathlineto{\pgfqpoint{1.001667in}{2.258654in}}%
\pgfpathlineto{\pgfqpoint{1.018051in}{2.296583in}}%
\pgfpathlineto{\pgfqpoint{1.035401in}{2.331101in}}%
\pgfpathlineto{\pgfqpoint{1.054650in}{2.364275in}}%
\pgfpathlineto{\pgfqpoint{1.074406in}{2.393984in}}%
\pgfpathlineto{\pgfqpoint{1.095771in}{2.422197in}}%
\pgfpathlineto{\pgfqpoint{1.118662in}{2.448797in}}%
\pgfpathlineto{\pgfqpoint{1.142967in}{2.473701in}}%
\pgfpathlineto{\pgfqpoint{1.168550in}{2.496867in}}%
\pgfpathlineto{\pgfqpoint{1.197085in}{2.519662in}}%
\pgfpathlineto{\pgfqpoint{1.226727in}{2.540526in}}%
\pgfpathlineto{\pgfqpoint{1.259242in}{2.560673in}}%
\pgfpathlineto{\pgfqpoint{1.294612in}{2.579881in}}%
\pgfpathlineto{\pgfqpoint{1.332792in}{2.597982in}}%
\pgfpathlineto{\pgfqpoint{1.373719in}{2.614859in}}%
\pgfpathlineto{\pgfqpoint{1.417319in}{2.630445in}}%
\pgfpathlineto{\pgfqpoint{1.465632in}{2.645312in}}%
\pgfpathlineto{\pgfqpoint{1.518640in}{2.659204in}}%
\pgfpathlineto{\pgfqpoint{1.576309in}{2.671929in}}%
\pgfpathlineto{\pgfqpoint{1.638597in}{2.683344in}}%
\pgfpathlineto{\pgfqpoint{1.705462in}{2.693343in}}%
\pgfpathlineto{\pgfqpoint{1.779027in}{2.702064in}}%
\pgfpathlineto{\pgfqpoint{1.857097in}{2.709077in}}%
\pgfpathlineto{\pgfqpoint{1.939633in}{2.714280in}}%
\pgfpathlineto{\pgfqpoint{2.026598in}{2.717513in}}%
\pgfpathlineto{\pgfqpoint{2.113605in}{2.718523in}}%
\pgfpathlineto{\pgfqpoint{2.198435in}{2.717303in}}%
\pgfpathlineto{\pgfqpoint{2.278866in}{2.713929in}}%
\pgfpathlineto{\pgfqpoint{2.352678in}{2.708598in}}%
\pgfpathlineto{\pgfqpoint{2.417657in}{2.701709in}}%
\pgfpathlineto{\pgfqpoint{2.473770in}{2.693630in}}%
\pgfpathlineto{\pgfqpoint{2.523140in}{2.684368in}}%
\pgfpathlineto{\pgfqpoint{2.565726in}{2.674202in}}%
\pgfpathlineto{\pgfqpoint{2.601510in}{2.663544in}}%
\pgfpathlineto{\pgfqpoint{2.632577in}{2.652142in}}%
\pgfpathlineto{\pgfqpoint{2.658899in}{2.640331in}}%
\pgfpathlineto{\pgfqpoint{2.682438in}{2.627436in}}%
\pgfpathlineto{\pgfqpoint{2.703062in}{2.613571in}}%
\pgfpathlineto{\pgfqpoint{2.720674in}{2.598978in}}%
\pgfpathlineto{\pgfqpoint{2.735263in}{2.584053in}}%
\pgfpathlineto{\pgfqpoint{2.748320in}{2.567377in}}%
\pgfpathlineto{\pgfqpoint{2.759553in}{2.549046in}}%
\pgfpathlineto{\pgfqpoint{2.768788in}{2.529306in}}%
\pgfpathlineto{\pgfqpoint{2.776017in}{2.508498in}}%
\pgfpathlineto{\pgfqpoint{2.781884in}{2.484540in}}%
\pgfpathlineto{\pgfqpoint{2.786102in}{2.457597in}}%
\pgfpathlineto{\pgfqpoint{2.788720in}{2.425384in}}%
\pgfpathlineto{\pgfqpoint{2.789427in}{2.388061in}}%
\pgfpathlineto{\pgfqpoint{2.787962in}{2.340801in}}%
\pgfpathlineto{\pgfqpoint{2.783672in}{2.278768in}}%
\pgfpathlineto{\pgfqpoint{2.774289in}{2.179783in}}%
\pgfpathlineto{\pgfqpoint{2.743611in}{1.868119in}}%
\pgfpathlineto{\pgfqpoint{2.730112in}{1.702060in}}%
\pgfpathlineto{\pgfqpoint{2.717287in}{1.515949in}}%
\pgfpathlineto{\pgfqpoint{2.702602in}{1.267597in}}%
\pgfpathlineto{\pgfqpoint{2.684434in}{0.964630in}}%
\pgfpathlineto{\pgfqpoint{2.675374in}{0.850600in}}%
\pgfpathlineto{\pgfqpoint{2.667030in}{0.771523in}}%
\pgfpathlineto{\pgfqpoint{2.658752in}{0.712543in}}%
\pgfpathlineto{\pgfqpoint{2.650176in}{0.666284in}}%
\pgfpathlineto{\pgfqpoint{2.640820in}{0.627931in}}%
\pgfpathlineto{\pgfqpoint{2.631145in}{0.597534in}}%
\pgfpathlineto{\pgfqpoint{2.621004in}{0.572745in}}%
\pgfpathlineto{\pgfqpoint{2.609856in}{0.551383in}}%
\pgfpathlineto{\pgfqpoint{2.598042in}{0.533534in}}%
\pgfpathlineto{\pgfqpoint{2.584496in}{0.517378in}}%
\pgfpathlineto{\pgfqpoint{2.571109in}{0.504669in}}%
\pgfpathlineto{\pgfqpoint{2.554789in}{0.492313in}}%
\pgfpathlineto{\pgfqpoint{2.537457in}{0.481914in}}%
\pgfpathlineto{\pgfqpoint{2.517374in}{0.472367in}}%
\pgfpathlineto{\pgfqpoint{2.492542in}{0.463178in}}%
\pgfpathlineto{\pgfqpoint{2.462979in}{0.454833in}}%
\pgfpathlineto{\pgfqpoint{2.428766in}{0.447542in}}%
\pgfpathlineto{\pgfqpoint{2.385671in}{0.440735in}}%
\pgfpathlineto{\pgfqpoint{2.331557in}{0.434581in}}%
\pgfpathlineto{\pgfqpoint{2.262115in}{0.429077in}}%
\pgfpathlineto{\pgfqpoint{2.170851in}{0.424236in}}%
\pgfpathlineto{\pgfqpoint{2.049086in}{0.420134in}}%
\pgfpathlineto{\pgfqpoint{1.879436in}{0.416783in}}%
\pgfpathlineto{\pgfqpoint{1.640159in}{0.414418in}}%
\pgfpathlineto{\pgfqpoint{1.322562in}{0.413569in}}%
\pgfpathlineto{\pgfqpoint{1.020194in}{0.414850in}}%
\pgfpathlineto{\pgfqpoint{0.822256in}{0.417715in}}%
\pgfpathlineto{\pgfqpoint{0.704835in}{0.421430in}}%
\pgfpathlineto{\pgfqpoint{0.630976in}{0.425829in}}%
\pgfpathlineto{\pgfqpoint{0.583316in}{0.430734in}}%
\pgfpathlineto{\pgfqpoint{0.551033in}{0.436123in}}%
\pgfpathlineto{\pgfqpoint{0.527708in}{0.442189in}}%
\pgfpathlineto{\pgfqpoint{0.511250in}{0.448625in}}%
\pgfpathlineto{\pgfqpoint{0.499549in}{0.455216in}}%
\pgfpathlineto{\pgfqpoint{0.488916in}{0.463841in}}%
\pgfpathlineto{\pgfqpoint{0.481322in}{0.472730in}}%
\pgfpathlineto{\pgfqpoint{0.474078in}{0.485127in}}%
\pgfpathlineto{\pgfqpoint{0.468753in}{0.498748in}}%
\pgfpathlineto{\pgfqpoint{0.463870in}{0.517848in}}%
\pgfpathlineto{\pgfqpoint{0.459679in}{0.544796in}}%
\pgfpathlineto{\pgfqpoint{0.456386in}{0.581938in}}%
\pgfpathlineto{\pgfqpoint{0.453731in}{0.639106in}}%
\pgfpathlineto{\pgfqpoint{0.451681in}{0.736155in}}%
\pgfpathlineto{\pgfqpoint{0.450220in}{0.927815in}}%
\pgfpathlineto{\pgfqpoint{0.449345in}{1.403252in}}%
\pgfpathlineto{\pgfqpoint{0.449543in}{2.682703in}}%
\pgfpathlineto{\pgfqpoint{0.451011in}{2.856932in}}%
\pgfpathlineto{\pgfqpoint{0.452802in}{2.879219in}}%
\pgfpathlineto{\pgfqpoint{0.455188in}{2.886108in}}%
\pgfpathlineto{\pgfqpoint{0.458626in}{2.889028in}}%
\pgfpathlineto{\pgfqpoint{0.464996in}{2.890553in}}%
\pgfpathlineto{\pgfqpoint{0.482377in}{2.891423in}}%
\pgfpathlineto{\pgfqpoint{0.565038in}{2.891729in}}%
\pgfpathlineto{\pgfqpoint{2.733842in}{2.891760in}}%
\pgfpathlineto{\pgfqpoint{4.789510in}{2.890885in}}%
\pgfpathlineto{\pgfqpoint{4.793727in}{2.889730in}}%
\pgfpathlineto{\pgfqpoint{4.795481in}{2.888307in}}%
\pgfpathlineto{\pgfqpoint{4.797106in}{2.881145in}}%
\pgfpathlineto{\pgfqpoint{4.797997in}{2.858771in}}%
\pgfpathlineto{\pgfqpoint{4.798039in}{2.856283in}}%
\pgfpathlineto{\pgfqpoint{4.798039in}{2.856283in}}%
\pgfusepath{stroke}%
\end{pgfscope}%
\begin{pgfscope}%
\pgfpathrectangle{\pgfqpoint{0.448634in}{0.402556in}}{\pgfqpoint{4.350661in}{2.489204in}} %
\pgfusepath{clip}%
\pgfsetrectcap%
\pgfsetroundjoin%
\pgfsetlinewidth{1.003750pt}%
\definecolor{currentstroke}{rgb}{0.121569,0.466667,0.705882}%
\pgfsetstrokecolor{currentstroke}%
\pgfsetdash{}{0pt}%
\pgfpathmoveto{\pgfqpoint{3.428772in}{0.402610in}}%
\pgfpathlineto{\pgfqpoint{2.806632in}{0.403760in}}%
\pgfpathlineto{\pgfqpoint{2.769692in}{0.405578in}}%
\pgfpathlineto{\pgfqpoint{2.754632in}{0.408064in}}%
\pgfpathlineto{\pgfqpoint{2.746391in}{0.411198in}}%
\pgfpathlineto{\pgfqpoint{2.740943in}{0.415265in}}%
\pgfpathlineto{\pgfqpoint{2.736784in}{0.420984in}}%
\pgfpathlineto{\pgfqpoint{2.733281in}{0.430071in}}%
\pgfpathlineto{\pgfqpoint{2.730449in}{0.444636in}}%
\pgfpathlineto{\pgfqpoint{2.728238in}{0.469392in}}%
\pgfpathlineto{\pgfqpoint{2.726470in}{0.519131in}}%
\pgfpathlineto{\pgfqpoint{2.725711in}{0.613715in}}%
\pgfpathlineto{\pgfqpoint{2.726842in}{0.768038in}}%
\pgfpathlineto{\pgfqpoint{2.730556in}{0.962148in}}%
\pgfpathlineto{\pgfqpoint{2.736611in}{1.158670in}}%
\pgfpathlineto{\pgfqpoint{2.744092in}{1.327718in}}%
\pgfpathlineto{\pgfqpoint{2.753201in}{1.484189in}}%
\pgfpathlineto{\pgfqpoint{2.763257in}{1.620609in}}%
\pgfpathlineto{\pgfqpoint{2.776118in}{1.764216in}}%
\pgfpathlineto{\pgfqpoint{2.788914in}{1.877776in}}%
\pgfpathlineto{\pgfqpoint{2.805748in}{2.005740in}}%
\pgfpathlineto{\pgfqpoint{2.821176in}{2.101198in}}%
\pgfpathlineto{\pgfqpoint{2.838359in}{2.193718in}}%
\pgfpathlineto{\pgfqpoint{2.859135in}{2.292966in}}%
\pgfpathlineto{\pgfqpoint{2.887209in}{2.425960in}}%
\pgfpathlineto{\pgfqpoint{2.896991in}{2.479559in}}%
\pgfpathlineto{\pgfqpoint{2.901543in}{2.516523in}}%
\pgfpathlineto{\pgfqpoint{2.902849in}{2.543854in}}%
\pgfpathlineto{\pgfqpoint{2.901957in}{2.566223in}}%
\pgfpathlineto{\pgfqpoint{2.899151in}{2.585863in}}%
\pgfpathlineto{\pgfqpoint{2.894794in}{2.602546in}}%
\pgfpathlineto{\pgfqpoint{2.888484in}{2.618388in}}%
\pgfpathlineto{\pgfqpoint{2.880257in}{2.633033in}}%
\pgfpathlineto{\pgfqpoint{2.870348in}{2.646246in}}%
\pgfpathlineto{\pgfqpoint{2.857400in}{2.659530in}}%
\pgfpathlineto{\pgfqpoint{2.843189in}{2.671010in}}%
\pgfpathlineto{\pgfqpoint{2.824237in}{2.683209in}}%
\pgfpathlineto{\pgfqpoint{2.802413in}{2.694418in}}%
\pgfpathlineto{\pgfqpoint{2.775809in}{2.705369in}}%
\pgfpathlineto{\pgfqpoint{2.744461in}{2.715715in}}%
\pgfpathlineto{\pgfqpoint{2.708436in}{2.725252in}}%
\pgfpathlineto{\pgfqpoint{2.665655in}{2.734289in}}%
\pgfpathlineto{\pgfqpoint{2.613991in}{2.742869in}}%
\pgfpathlineto{\pgfqpoint{2.553459in}{2.750589in}}%
\pgfpathlineto{\pgfqpoint{2.481920in}{2.757365in}}%
\pgfpathlineto{\pgfqpoint{2.399398in}{2.762839in}}%
\pgfpathlineto{\pgfqpoint{2.310269in}{2.766482in}}%
\pgfpathlineto{\pgfqpoint{2.175416in}{2.768725in}}%
\pgfpathlineto{\pgfqpoint{2.066653in}{2.767942in}}%
\pgfpathlineto{\pgfqpoint{1.953570in}{2.764859in}}%
\pgfpathlineto{\pgfqpoint{1.851429in}{2.759759in}}%
\pgfpathlineto{\pgfqpoint{1.745051in}{2.752169in}}%
\pgfpathlineto{\pgfqpoint{1.658373in}{2.743453in}}%
\pgfpathlineto{\pgfqpoint{1.580552in}{2.733461in}}%
\pgfpathlineto{\pgfqpoint{1.490057in}{2.719338in}}%
\pgfpathlineto{\pgfqpoint{1.417231in}{2.704698in}}%
\pgfpathlineto{\pgfqpoint{1.361992in}{2.690818in}}%
\pgfpathlineto{\pgfqpoint{1.311460in}{2.675819in}}%
\pgfpathlineto{\pgfqpoint{1.265667in}{2.659924in}}%
\pgfpathlineto{\pgfqpoint{1.222575in}{2.642586in}}%
\pgfpathlineto{\pgfqpoint{1.184324in}{2.624682in}}%
\pgfpathlineto{\pgfqpoint{1.148892in}{2.605623in}}%
\pgfpathlineto{\pgfqpoint{1.116331in}{2.585573in}}%
\pgfpathlineto{\pgfqpoint{1.092327in}{2.568512in}}%
\pgfpathlineto{\pgfqpoint{1.079760in}{2.558686in}}%
\pgfpathlineto{\pgfqpoint{1.051544in}{2.535379in}}%
\pgfpathlineto{\pgfqpoint{1.026312in}{2.511712in}}%
\pgfpathlineto{\pgfqpoint{1.002399in}{2.486318in}}%
\pgfpathlineto{\pgfqpoint{0.979913in}{2.459269in}}%
\pgfpathlineto{\pgfqpoint{0.958934in}{2.430678in}}%
\pgfpathlineto{\pgfqpoint{0.938264in}{2.398643in}}%
\pgfpathlineto{\pgfqpoint{0.923047in}{2.371385in}}%
\pgfpathlineto{\pgfqpoint{0.904513in}{2.334774in}}%
\pgfpathlineto{\pgfqpoint{0.887854in}{2.297001in}}%
\pgfpathlineto{\pgfqpoint{0.872131in}{2.255971in}}%
\pgfpathlineto{\pgfqpoint{0.857508in}{2.211741in}}%
\pgfpathlineto{\pgfqpoint{0.844762in}{2.166757in}}%
\pgfpathlineto{\pgfqpoint{0.838624in}{2.140306in}}%
\pgfpathlineto{\pgfqpoint{0.826982in}{2.087194in}}%
\pgfpathlineto{\pgfqpoint{0.816322in}{2.028715in}}%
\pgfpathlineto{\pgfqpoint{0.810087in}{1.984495in}}%
\pgfpathlineto{\pgfqpoint{0.808026in}{1.967238in}}%
\pgfpathlineto{\pgfqpoint{0.800076in}{1.898140in}}%
\pgfpathlineto{\pgfqpoint{0.793713in}{1.823823in}}%
\pgfpathlineto{\pgfqpoint{0.788799in}{1.741875in}}%
\pgfpathlineto{\pgfqpoint{0.786199in}{1.677225in}}%
\pgfpathlineto{\pgfqpoint{0.776951in}{1.453481in}}%
\pgfpathlineto{\pgfqpoint{0.773280in}{1.418894in}}%
\pgfpathlineto{\pgfqpoint{0.768298in}{1.389582in}}%
\pgfpathlineto{\pgfqpoint{0.762752in}{1.368108in}}%
\pgfpathlineto{\pgfqpoint{0.756722in}{1.352123in}}%
\pgfpathlineto{\pgfqpoint{0.749752in}{1.339519in}}%
\pgfpathlineto{\pgfqpoint{0.742201in}{1.330599in}}%
\pgfpathlineto{\pgfqpoint{0.734854in}{1.325312in}}%
\pgfpathlineto{\pgfqpoint{0.726558in}{1.322419in}}%
\pgfpathlineto{\pgfqpoint{0.717884in}{1.322223in}}%
\pgfpathlineto{\pgfqpoint{0.709412in}{1.324411in}}%
\pgfpathlineto{\pgfqpoint{0.699548in}{1.329604in}}%
\pgfpathlineto{\pgfqpoint{0.688894in}{1.338203in}}%
\pgfpathlineto{\pgfqpoint{0.677907in}{1.350248in}}%
\pgfpathlineto{\pgfqpoint{0.666886in}{1.365647in}}%
\pgfpathlineto{\pgfqpoint{0.654913in}{1.386417in}}%
\pgfpathlineto{\pgfqpoint{0.642574in}{1.412730in}}%
\pgfpathlineto{\pgfqpoint{0.630328in}{1.444629in}}%
\pgfpathlineto{\pgfqpoint{0.618504in}{1.482081in}}%
\pgfpathlineto{\pgfqpoint{0.608613in}{1.520256in}}%
\pgfpathlineto{\pgfqpoint{0.590203in}{1.612445in}}%
\pgfpathlineto{\pgfqpoint{0.581848in}{1.668884in}}%
\pgfpathlineto{\pgfqpoint{0.573137in}{1.740376in}}%
\pgfpathlineto{\pgfqpoint{0.567062in}{1.807213in}}%
\pgfpathlineto{\pgfqpoint{0.560532in}{1.896510in}}%
\pgfpathlineto{\pgfqpoint{0.555526in}{1.995910in}}%
\pgfpathlineto{\pgfqpoint{0.552564in}{2.097908in}}%
\pgfpathlineto{\pgfqpoint{0.551526in}{2.204935in}}%
\pgfpathlineto{\pgfqpoint{0.552728in}{2.309470in}}%
\pgfpathlineto{\pgfqpoint{0.556011in}{2.403981in}}%
\pgfpathlineto{\pgfqpoint{0.560953in}{2.483430in}}%
\pgfpathlineto{\pgfqpoint{0.567303in}{2.550240in}}%
\pgfpathlineto{\pgfqpoint{0.574928in}{2.606817in}}%
\pgfpathlineto{\pgfqpoint{0.582988in}{2.650657in}}%
\pgfpathlineto{\pgfqpoint{0.592756in}{2.691452in}}%
\pgfpathlineto{\pgfqpoint{0.602650in}{2.721756in}}%
\pgfpathlineto{\pgfqpoint{0.612983in}{2.746441in}}%
\pgfpathlineto{\pgfqpoint{0.624292in}{2.767692in}}%
\pgfpathlineto{\pgfqpoint{0.636231in}{2.785433in}}%
\pgfpathlineto{\pgfqpoint{0.649892in}{2.801461in}}%
\pgfpathlineto{\pgfqpoint{0.663386in}{2.814020in}}%
\pgfpathlineto{\pgfqpoint{0.679842in}{2.826135in}}%
\pgfpathlineto{\pgfqpoint{0.697326in}{2.836197in}}%
\pgfpathlineto{\pgfqpoint{0.715574in}{2.844285in}}%
\pgfpathlineto{\pgfqpoint{0.738439in}{2.852335in}}%
\pgfpathlineto{\pgfqpoint{0.765983in}{2.859639in}}%
\pgfpathlineto{\pgfqpoint{0.800300in}{2.866256in}}%
\pgfpathlineto{\pgfqpoint{0.841340in}{2.871832in}}%
\pgfpathlineto{\pgfqpoint{0.895547in}{2.876803in}}%
\pgfpathlineto{\pgfqpoint{0.969413in}{2.881069in}}%
\pgfpathlineto{\pgfqpoint{1.071608in}{2.884501in}}%
\pgfpathlineto{\pgfqpoint{1.219512in}{2.887074in}}%
\pgfpathlineto{\pgfqpoint{1.471844in}{2.889091in}}%
\pgfpathlineto{\pgfqpoint{1.956941in}{2.890384in}}%
\pgfpathlineto{\pgfqpoint{3.096814in}{2.890781in}}%
\pgfpathlineto{\pgfqpoint{3.995224in}{2.889388in}}%
\pgfpathlineto{\pgfqpoint{4.275833in}{2.887011in}}%
\pgfpathlineto{\pgfqpoint{4.412847in}{2.883743in}}%
\pgfpathlineto{\pgfqpoint{4.491081in}{2.879810in}}%
\pgfpathlineto{\pgfqpoint{4.543127in}{2.875163in}}%
\pgfpathlineto{\pgfqpoint{4.579810in}{2.869841in}}%
\pgfpathlineto{\pgfqpoint{4.607580in}{2.863763in}}%
\pgfpathlineto{\pgfqpoint{4.630623in}{2.856424in}}%
\pgfpathlineto{\pgfqpoint{4.648833in}{2.848228in}}%
\pgfpathlineto{\pgfqpoint{4.664136in}{2.838773in}}%
\pgfpathlineto{\pgfqpoint{4.676470in}{2.828576in}}%
\pgfpathlineto{\pgfqpoint{4.687502in}{2.816585in}}%
\pgfpathlineto{\pgfqpoint{4.697051in}{2.803027in}}%
\pgfpathlineto{\pgfqpoint{4.706194in}{2.786098in}}%
\pgfpathlineto{\pgfqpoint{4.714508in}{2.765827in}}%
\pgfpathlineto{\pgfqpoint{4.722462in}{2.740013in}}%
\pgfpathlineto{\pgfqpoint{4.729577in}{2.708703in}}%
\pgfpathlineto{\pgfqpoint{4.736162in}{2.669601in}}%
\pgfpathlineto{\pgfqpoint{4.742419in}{2.617826in}}%
\pgfpathlineto{\pgfqpoint{4.747859in}{2.553410in}}%
\pgfpathlineto{\pgfqpoint{4.752661in}{2.468958in}}%
\pgfpathlineto{\pgfqpoint{4.756610in}{2.359528in}}%
\pgfpathlineto{\pgfqpoint{4.759416in}{2.217681in}}%
\pgfpathlineto{\pgfqpoint{4.760596in}{2.043444in}}%
\pgfpathlineto{\pgfqpoint{4.759662in}{1.851779in}}%
\pgfpathlineto{\pgfqpoint{4.756587in}{1.667613in}}%
\pgfpathlineto{\pgfqpoint{4.751596in}{1.503428in}}%
\pgfpathlineto{\pgfqpoint{4.745410in}{1.374185in}}%
\pgfpathlineto{\pgfqpoint{4.738113in}{1.267479in}}%
\pgfpathlineto{\pgfqpoint{4.729621in}{1.175896in}}%
\pgfpathlineto{\pgfqpoint{4.720762in}{1.104428in}}%
\pgfpathlineto{\pgfqpoint{4.711045in}{1.043204in}}%
\pgfpathlineto{\pgfqpoint{4.700364in}{0.989829in}}%
\pgfpathlineto{\pgfqpoint{4.689055in}{0.944345in}}%
\pgfpathlineto{\pgfqpoint{4.676881in}{0.904394in}}%
\pgfpathlineto{\pgfqpoint{4.676095in}{0.902073in}}%
\pgfpathlineto{\pgfqpoint{4.676095in}{0.902073in}}%
\pgfusepath{stroke}%
\end{pgfscope}%
\begin{pgfscope}%
\pgfpathrectangle{\pgfqpoint{0.448634in}{0.402556in}}{\pgfqpoint{4.350661in}{2.489204in}} %
\pgfusepath{clip}%
\pgfsetrectcap%
\pgfsetroundjoin%
\pgfsetlinewidth{1.003750pt}%
\definecolor{currentstroke}{rgb}{0.121569,0.466667,0.705882}%
\pgfsetstrokecolor{currentstroke}%
\pgfsetdash{}{0pt}%
\pgfpathmoveto{\pgfqpoint{2.795520in}{1.982745in}}%
\pgfpathlineto{\pgfqpoint{2.781780in}{1.874357in}}%
\pgfpathlineto{\pgfqpoint{2.769351in}{1.758234in}}%
\pgfpathlineto{\pgfqpoint{2.758095in}{1.631942in}}%
\pgfpathlineto{\pgfqpoint{2.747786in}{1.490551in}}%
\pgfpathlineto{\pgfqpoint{2.738644in}{1.334082in}}%
\pgfpathlineto{\pgfqpoint{2.730580in}{1.157591in}}%
\pgfpathlineto{\pgfqpoint{2.723334in}{0.948663in}}%
\pgfpathlineto{\pgfqpoint{2.709783in}{0.530788in}}%
\pgfpathlineto{\pgfqpoint{2.705868in}{0.488716in}}%
\pgfpathlineto{\pgfqpoint{2.701769in}{0.464281in}}%
\pgfpathlineto{\pgfqpoint{2.697021in}{0.447744in}}%
\pgfpathlineto{\pgfqpoint{2.691859in}{0.436812in}}%
\pgfpathlineto{\pgfqpoint{2.686245in}{0.429229in}}%
\pgfpathlineto{\pgfqpoint{2.679348in}{0.423188in}}%
\pgfpathlineto{\pgfqpoint{2.669540in}{0.417856in}}%
\pgfpathlineto{\pgfqpoint{2.656987in}{0.413810in}}%
\pgfpathlineto{\pgfqpoint{2.637654in}{0.410337in}}%
\pgfpathlineto{\pgfqpoint{2.607297in}{0.407617in}}%
\pgfpathlineto{\pgfqpoint{2.555121in}{0.405574in}}%
\pgfpathlineto{\pgfqpoint{2.450714in}{0.404139in}}%
\pgfpathlineto{\pgfqpoint{2.176624in}{0.403275in}}%
\pgfpathlineto{\pgfqpoint{1.130290in}{0.402953in}}%
\pgfpathlineto{\pgfqpoint{0.516849in}{0.404175in}}%
\pgfpathlineto{\pgfqpoint{0.466848in}{0.405970in}}%
\pgfpathlineto{\pgfqpoint{0.456130in}{0.407931in}}%
\pgfpathlineto{\pgfqpoint{0.452340in}{0.410303in}}%
\pgfpathlineto{\pgfqpoint{0.450346in}{0.414662in}}%
\pgfpathlineto{\pgfqpoint{0.449266in}{0.424524in}}%
\pgfpathlineto{\pgfqpoint{0.448771in}{0.464344in}}%
\pgfpathlineto{\pgfqpoint{0.448640in}{0.850171in}}%
\pgfpathlineto{\pgfqpoint{0.448679in}{2.891318in}}%
\pgfpathlineto{\pgfqpoint{0.448679in}{2.891318in}}%
\pgfusepath{stroke}%
\end{pgfscope}%
\begin{pgfscope}%
\pgfpathrectangle{\pgfqpoint{0.448634in}{0.402556in}}{\pgfqpoint{4.350661in}{2.489204in}} %
\pgfusepath{clip}%
\pgfsetrectcap%
\pgfsetroundjoin%
\pgfsetlinewidth{1.003750pt}%
\definecolor{currentstroke}{rgb}{0.121569,0.466667,0.705882}%
\pgfsetstrokecolor{currentstroke}%
\pgfsetdash{}{0pt}%
\pgfpathmoveto{\pgfqpoint{3.428189in}{0.402586in}}%
\pgfpathlineto{\pgfqpoint{2.782121in}{0.403701in}}%
\pgfpathlineto{\pgfqpoint{2.753906in}{0.405674in}}%
\pgfpathlineto{\pgfqpoint{2.743328in}{0.408443in}}%
\pgfpathlineto{\pgfqpoint{2.737717in}{0.412188in}}%
\pgfpathlineto{\pgfqpoint{2.733668in}{0.417995in}}%
\pgfpathlineto{\pgfqpoint{2.730649in}{0.427307in}}%
\pgfpathlineto{\pgfqpoint{2.728388in}{0.442004in}}%
\pgfpathlineto{\pgfqpoint{2.726544in}{0.471794in}}%
\pgfpathlineto{\pgfqpoint{2.725216in}{0.534003in}}%
\pgfpathlineto{\pgfqpoint{2.725169in}{0.655973in}}%
\pgfpathlineto{\pgfqpoint{2.727377in}{0.832687in}}%
\pgfpathlineto{\pgfqpoint{2.732259in}{1.041703in}}%
\pgfpathlineto{\pgfqpoint{2.738851in}{1.223257in}}%
\pgfpathlineto{\pgfqpoint{2.747078in}{1.389766in}}%
\pgfpathlineto{\pgfqpoint{2.756608in}{1.538717in}}%
\pgfpathlineto{\pgfqpoint{2.768955in}{1.694887in}}%
\pgfpathlineto{\pgfqpoint{2.781228in}{1.816044in}}%
\pgfpathlineto{\pgfqpoint{2.794401in}{1.924524in}}%
\pgfpathlineto{\pgfqpoint{2.812737in}{2.054722in}}%
\pgfpathlineto{\pgfqpoint{2.828774in}{2.147512in}}%
\pgfpathlineto{\pgfqpoint{2.847382in}{2.242224in}}%
\pgfpathlineto{\pgfqpoint{2.895818in}{2.479699in}}%
\pgfpathlineto{\pgfqpoint{2.900204in}{2.516689in}}%
\pgfpathlineto{\pgfqpoint{2.901346in}{2.544029in}}%
\pgfpathlineto{\pgfqpoint{2.900291in}{2.566388in}}%
\pgfpathlineto{\pgfqpoint{2.897334in}{2.585999in}}%
\pgfpathlineto{\pgfqpoint{2.892836in}{2.602633in}}%
\pgfpathlineto{\pgfqpoint{2.886394in}{2.618405in}}%
\pgfpathlineto{\pgfqpoint{2.878058in}{2.632969in}}%
\pgfpathlineto{\pgfqpoint{2.868065in}{2.646100in}}%
\pgfpathlineto{\pgfqpoint{2.855050in}{2.659300in}}%
\pgfpathlineto{\pgfqpoint{2.840801in}{2.670717in}}%
\pgfpathlineto{\pgfqpoint{2.821822in}{2.682861in}}%
\pgfpathlineto{\pgfqpoint{2.799980in}{2.694026in}}%
\pgfpathlineto{\pgfqpoint{2.773366in}{2.704944in}}%
\pgfpathlineto{\pgfqpoint{2.742012in}{2.715266in}}%
\pgfpathlineto{\pgfqpoint{2.705983in}{2.724785in}}%
\pgfpathlineto{\pgfqpoint{2.663200in}{2.733810in}}%
\pgfpathlineto{\pgfqpoint{2.611535in}{2.742379in}}%
\pgfpathlineto{\pgfqpoint{2.551002in}{2.750090in}}%
\pgfpathlineto{\pgfqpoint{2.481632in}{2.756682in}}%
\pgfpathlineto{\pgfqpoint{2.399112in}{2.762200in}}%
\pgfpathlineto{\pgfqpoint{2.309985in}{2.765886in}}%
\pgfpathlineto{\pgfqpoint{2.188184in}{2.768096in}}%
\pgfpathlineto{\pgfqpoint{2.081595in}{2.767619in}}%
\pgfpathlineto{\pgfqpoint{1.968506in}{2.764840in}}%
\pgfpathlineto{\pgfqpoint{1.864180in}{2.759918in}}%
\pgfpathlineto{\pgfqpoint{1.757786in}{2.752593in}}%
\pgfpathlineto{\pgfqpoint{1.671087in}{2.744171in}}%
\pgfpathlineto{\pgfqpoint{1.591076in}{2.734193in}}%
\pgfpathlineto{\pgfqpoint{1.502689in}{2.720717in}}%
\pgfpathlineto{\pgfqpoint{1.427655in}{2.706083in}}%
\pgfpathlineto{\pgfqpoint{1.372350in}{2.692544in}}%
\pgfpathlineto{\pgfqpoint{1.321734in}{2.677921in}}%
\pgfpathlineto{\pgfqpoint{1.273765in}{2.661664in}}%
\pgfpathlineto{\pgfqpoint{1.230567in}{2.644672in}}%
\pgfpathlineto{\pgfqpoint{1.192197in}{2.627106in}}%
\pgfpathlineto{\pgfqpoint{1.156620in}{2.608403in}}%
\pgfpathlineto{\pgfqpoint{1.123890in}{2.588716in}}%
\pgfpathlineto{\pgfqpoint{1.095883in}{2.569568in}}%
\pgfpathlineto{\pgfqpoint{1.063936in}{2.543701in}}%
\pgfpathlineto{\pgfqpoint{1.038217in}{2.520732in}}%
\pgfpathlineto{\pgfqpoint{1.013766in}{2.496016in}}%
\pgfpathlineto{\pgfqpoint{0.990704in}{2.469610in}}%
\pgfpathlineto{\pgfqpoint{0.969124in}{2.441612in}}%
\pgfpathlineto{\pgfqpoint{0.949083in}{2.412154in}}%
\pgfpathlineto{\pgfqpoint{0.930604in}{2.381387in}}%
\pgfpathlineto{\pgfqpoint{0.906555in}{2.334052in}}%
\pgfpathlineto{\pgfqpoint{0.889925in}{2.296262in}}%
\pgfpathlineto{\pgfqpoint{0.874241in}{2.255213in}}%
\pgfpathlineto{\pgfqpoint{0.859667in}{2.210961in}}%
\pgfpathlineto{\pgfqpoint{0.846986in}{2.165954in}}%
\pgfpathlineto{\pgfqpoint{0.839633in}{2.134715in}}%
\pgfpathlineto{\pgfqpoint{0.828238in}{2.081532in}}%
\pgfpathlineto{\pgfqpoint{0.817866in}{2.022986in}}%
\pgfpathlineto{\pgfqpoint{0.810784in}{1.971352in}}%
\pgfpathlineto{\pgfqpoint{0.802846in}{1.902252in}}%
\pgfpathlineto{\pgfqpoint{0.796554in}{1.827927in}}%
\pgfpathlineto{\pgfqpoint{0.791696in}{1.743480in}}%
\pgfpathlineto{\pgfqpoint{0.787773in}{1.621595in}}%
\pgfpathlineto{\pgfqpoint{0.785408in}{1.522064in}}%
\pgfpathlineto{\pgfqpoint{0.785408in}{1.522064in}}%
\pgfusepath{stroke}%
\end{pgfscope}%
\begin{pgfscope}%
\pgfpathrectangle{\pgfqpoint{0.448634in}{0.402556in}}{\pgfqpoint{4.350661in}{2.489204in}} %
\pgfusepath{clip}%
\pgfsetrectcap%
\pgfsetroundjoin%
\pgfsetlinewidth{1.003750pt}%
\definecolor{currentstroke}{rgb}{1.000000,0.498039,0.054902}%
\pgfsetstrokecolor{currentstroke}%
\pgfsetdash{}{0pt}%
\pgfpathmoveto{\pgfqpoint{0.448634in}{2.896245in}}%
\pgfpathlineto{\pgfqpoint{0.448593in}{0.407043in}}%
\pgfpathlineto{\pgfqpoint{0.448593in}{0.407043in}}%
\pgfusepath{stroke}%
\end{pgfscope}%
\begin{pgfscope}%
\pgfpathrectangle{\pgfqpoint{0.448634in}{0.402556in}}{\pgfqpoint{4.350661in}{2.489204in}} %
\pgfusepath{clip}%
\pgfsetrectcap%
\pgfsetroundjoin%
\pgfsetlinewidth{1.003750pt}%
\definecolor{currentstroke}{rgb}{1.000000,0.498039,0.054902}%
\pgfsetstrokecolor{currentstroke}%
\pgfsetdash{}{0pt}%
\pgfpathmoveto{\pgfqpoint{0.576853in}{1.760817in}}%
\pgfpathlineto{\pgfqpoint{0.569394in}{1.840010in}}%
\pgfpathlineto{\pgfqpoint{0.563209in}{1.929338in}}%
\pgfpathlineto{\pgfqpoint{0.558592in}{2.028764in}}%
\pgfpathlineto{\pgfqpoint{0.555985in}{2.133265in}}%
\pgfpathlineto{\pgfqpoint{0.555566in}{2.237808in}}%
\pgfpathlineto{\pgfqpoint{0.557371in}{2.337352in}}%
\pgfpathlineto{\pgfqpoint{0.561096in}{2.424366in}}%
\pgfpathlineto{\pgfqpoint{0.566403in}{2.498791in}}%
\pgfpathlineto{\pgfqpoint{0.572909in}{2.560570in}}%
\pgfpathlineto{\pgfqpoint{0.580458in}{2.612119in}}%
\pgfpathlineto{\pgfqpoint{0.589086in}{2.655816in}}%
\pgfpathlineto{\pgfqpoint{0.598406in}{2.691589in}}%
\pgfpathlineto{\pgfqpoint{0.608613in}{2.721757in}}%
\pgfpathlineto{\pgfqpoint{0.619241in}{2.746278in}}%
\pgfpathlineto{\pgfqpoint{0.630817in}{2.767339in}}%
\pgfpathlineto{\pgfqpoint{0.642975in}{2.784884in}}%
\pgfpathlineto{\pgfqpoint{0.656813in}{2.800712in}}%
\pgfpathlineto{\pgfqpoint{0.672197in}{2.814549in}}%
\pgfpathlineto{\pgfqpoint{0.688853in}{2.826301in}}%
\pgfpathlineto{\pgfqpoint{0.706461in}{2.836076in}}%
\pgfpathlineto{\pgfqpoint{0.726804in}{2.844875in}}%
\pgfpathlineto{\pgfqpoint{0.751866in}{2.853203in}}%
\pgfpathlineto{\pgfqpoint{0.781631in}{2.860547in}}%
\pgfpathlineto{\pgfqpoint{0.818168in}{2.867054in}}%
\pgfpathlineto{\pgfqpoint{0.863581in}{2.872685in}}%
\pgfpathlineto{\pgfqpoint{0.922161in}{2.877518in}}%
\pgfpathlineto{\pgfqpoint{1.000391in}{2.881567in}}%
\pgfpathlineto{\pgfqpoint{1.111294in}{2.884881in}}%
\pgfpathlineto{\pgfqpoint{1.274428in}{2.887367in}}%
\pgfpathlineto{\pgfqpoint{1.552865in}{2.889263in}}%
\pgfpathlineto{\pgfqpoint{2.107573in}{2.890457in}}%
\pgfpathlineto{\pgfqpoint{3.343161in}{2.890573in}}%
\pgfpathlineto{\pgfqpoint{4.043615in}{2.888941in}}%
\pgfpathlineto{\pgfqpoint{4.289417in}{2.886404in}}%
\pgfpathlineto{\pgfqpoint{4.413375in}{2.883093in}}%
\pgfpathlineto{\pgfqpoint{4.489424in}{2.878997in}}%
\pgfpathlineto{\pgfqpoint{4.541451in}{2.874081in}}%
\pgfpathlineto{\pgfqpoint{4.578100in}{2.868470in}}%
\pgfpathlineto{\pgfqpoint{4.605818in}{2.862092in}}%
\pgfpathlineto{\pgfqpoint{4.626725in}{2.855245in}}%
\pgfpathlineto{\pgfqpoint{4.644925in}{2.847018in}}%
\pgfpathlineto{\pgfqpoint{4.660241in}{2.837590in}}%
\pgfpathlineto{\pgfqpoint{4.672623in}{2.827468in}}%
\pgfpathlineto{\pgfqpoint{4.683751in}{2.815592in}}%
\pgfpathlineto{\pgfqpoint{4.693406in}{2.802135in}}%
\pgfpathlineto{\pgfqpoint{4.702740in}{2.785343in}}%
\pgfpathlineto{\pgfqpoint{4.711277in}{2.765194in}}%
\pgfpathlineto{\pgfqpoint{4.719482in}{2.739484in}}%
\pgfpathlineto{\pgfqpoint{4.726293in}{2.710657in}}%
\pgfpathlineto{\pgfqpoint{4.733259in}{2.671643in}}%
\pgfpathlineto{\pgfqpoint{4.739604in}{2.622396in}}%
\pgfpathlineto{\pgfqpoint{4.745236in}{2.560504in}}%
\pgfpathlineto{\pgfqpoint{4.750164in}{2.481052in}}%
\pgfpathlineto{\pgfqpoint{4.754367in}{2.376618in}}%
\pgfpathlineto{\pgfqpoint{4.757443in}{2.242249in}}%
\pgfpathlineto{\pgfqpoint{4.758977in}{2.075483in}}%
\pgfpathlineto{\pgfqpoint{4.758447in}{1.888795in}}%
\pgfpathlineto{\pgfqpoint{4.755756in}{1.707111in}}%
\pgfpathlineto{\pgfqpoint{4.750925in}{1.532957in}}%
\pgfpathlineto{\pgfqpoint{4.744785in}{1.398726in}}%
\pgfpathlineto{\pgfqpoint{4.737575in}{1.289516in}}%
\pgfpathlineto{\pgfqpoint{4.728714in}{1.190470in}}%
\pgfpathlineto{\pgfqpoint{4.719652in}{1.116521in}}%
\pgfpathlineto{\pgfqpoint{4.710036in}{1.055276in}}%
\pgfpathlineto{\pgfqpoint{4.699503in}{1.001861in}}%
\pgfpathlineto{\pgfqpoint{4.689040in}{0.958690in}}%
\pgfpathlineto{\pgfqpoint{4.677219in}{0.918600in}}%
\pgfpathlineto{\pgfqpoint{4.664034in}{0.881749in}}%
\pgfpathlineto{\pgfqpoint{4.650584in}{0.850492in}}%
\pgfpathlineto{\pgfqpoint{4.636303in}{0.822570in}}%
\pgfpathlineto{\pgfqpoint{4.620207in}{0.795974in}}%
\pgfpathlineto{\pgfqpoint{4.603640in}{0.772901in}}%
\pgfpathlineto{\pgfqpoint{4.585488in}{0.751446in}}%
\pgfpathlineto{\pgfqpoint{4.565874in}{0.731749in}}%
\pgfpathlineto{\pgfqpoint{4.544964in}{0.713879in}}%
\pgfpathlineto{\pgfqpoint{4.522958in}{0.697824in}}%
\pgfpathlineto{\pgfqpoint{4.496157in}{0.681290in}}%
\pgfpathlineto{\pgfqpoint{4.470397in}{0.667953in}}%
\pgfpathlineto{\pgfqpoint{4.439961in}{0.654509in}}%
\pgfpathlineto{\pgfqpoint{4.406841in}{0.642281in}}%
\pgfpathlineto{\pgfqpoint{4.369009in}{0.630748in}}%
\pgfpathlineto{\pgfqpoint{4.326489in}{0.620226in}}%
\pgfpathlineto{\pgfqpoint{4.279327in}{0.610949in}}%
\pgfpathlineto{\pgfqpoint{4.227576in}{0.603085in}}%
\pgfpathlineto{\pgfqpoint{4.173450in}{0.597063in}}%
\pgfpathlineto{\pgfqpoint{4.110511in}{0.592203in}}%
\pgfpathlineto{\pgfqpoint{4.047471in}{0.589537in}}%
\pgfpathlineto{\pgfqpoint{3.977867in}{0.588624in}}%
\pgfpathlineto{\pgfqpoint{3.906093in}{0.589934in}}%
\pgfpathlineto{\pgfqpoint{3.834377in}{0.593496in}}%
\pgfpathlineto{\pgfqpoint{3.767120in}{0.599067in}}%
\pgfpathlineto{\pgfqpoint{3.704364in}{0.606392in}}%
\pgfpathlineto{\pgfqpoint{3.678516in}{0.610510in}}%
\pgfpathlineto{\pgfqpoint{3.620438in}{0.620500in}}%
\pgfpathlineto{\pgfqpoint{3.586319in}{0.628207in}}%
\pgfpathlineto{\pgfqpoint{3.495240in}{0.652428in}}%
\pgfpathlineto{\pgfqpoint{3.451528in}{0.667583in}}%
\pgfpathlineto{\pgfqpoint{3.408538in}{0.685220in}}%
\pgfpathlineto{\pgfqpoint{3.374594in}{0.702001in}}%
\pgfpathlineto{\pgfqpoint{3.345407in}{0.718682in}}%
\pgfpathlineto{\pgfqpoint{3.315236in}{0.738520in}}%
\pgfpathlineto{\pgfqpoint{3.288127in}{0.759290in}}%
\pgfpathlineto{\pgfqpoint{3.264004in}{0.780551in}}%
\pgfpathlineto{\pgfqpoint{3.241208in}{0.803648in}}%
\pgfpathlineto{\pgfqpoint{3.219894in}{0.828530in}}%
\pgfpathlineto{\pgfqpoint{3.200189in}{0.855091in}}%
\pgfpathlineto{\pgfqpoint{3.182177in}{0.883182in}}%
\pgfpathlineto{\pgfqpoint{3.165906in}{0.912633in}}%
\pgfpathlineto{\pgfqpoint{3.150351in}{0.945448in}}%
\pgfpathlineto{\pgfqpoint{3.136682in}{0.979345in}}%
\pgfpathlineto{\pgfqpoint{3.124073in}{1.016460in}}%
\pgfpathlineto{\pgfqpoint{3.112834in}{1.056769in}}%
\pgfpathlineto{\pgfqpoint{3.103046in}{1.100146in}}%
\pgfpathlineto{\pgfqpoint{3.095343in}{1.144071in}}%
\pgfpathlineto{\pgfqpoint{3.089208in}{1.190837in}}%
\pgfpathlineto{\pgfqpoint{3.084595in}{1.242838in}}%
\pgfpathlineto{\pgfqpoint{3.082137in}{1.295031in}}%
\pgfpathlineto{\pgfqpoint{3.081687in}{1.349787in}}%
\pgfpathlineto{\pgfqpoint{3.083451in}{1.406998in}}%
\pgfpathlineto{\pgfqpoint{3.087181in}{1.461589in}}%
\pgfpathlineto{\pgfqpoint{3.093485in}{1.520888in}}%
\pgfpathlineto{\pgfqpoint{3.101823in}{1.577334in}}%
\pgfpathlineto{\pgfqpoint{3.111930in}{1.630856in}}%
\pgfpathlineto{\pgfqpoint{3.124690in}{1.686208in}}%
\pgfpathlineto{\pgfqpoint{3.139178in}{1.738395in}}%
\pgfpathlineto{\pgfqpoint{3.155145in}{1.787366in}}%
\pgfpathlineto{\pgfqpoint{3.172353in}{1.833085in}}%
\pgfpathlineto{\pgfqpoint{3.191618in}{1.877716in}}%
\pgfpathlineto{\pgfqpoint{3.214026in}{1.923261in}}%
\pgfpathlineto{\pgfqpoint{3.236214in}{1.963157in}}%
\pgfpathlineto{\pgfqpoint{3.260178in}{2.001684in}}%
\pgfpathlineto{\pgfqpoint{3.285814in}{2.038776in}}%
\pgfpathlineto{\pgfqpoint{3.314415in}{2.076285in}}%
\pgfpathlineto{\pgfqpoint{3.348944in}{2.117711in}}%
\pgfpathlineto{\pgfqpoint{3.417133in}{2.198022in}}%
\pgfpathlineto{\pgfqpoint{3.426053in}{2.212128in}}%
\pgfpathlineto{\pgfqpoint{3.430798in}{2.223297in}}%
\pgfpathlineto{\pgfqpoint{3.432034in}{2.230603in}}%
\pgfpathlineto{\pgfqpoint{3.430773in}{2.237856in}}%
\pgfpathlineto{\pgfqpoint{3.426621in}{2.243526in}}%
\pgfpathlineto{\pgfqpoint{3.420908in}{2.247084in}}%
\pgfpathlineto{\pgfqpoint{3.412501in}{2.249583in}}%
\pgfpathlineto{\pgfqpoint{3.399499in}{2.250689in}}%
\pgfpathlineto{\pgfqpoint{3.384305in}{2.249671in}}%
\pgfpathlineto{\pgfqpoint{3.364985in}{2.246098in}}%
\pgfpathlineto{\pgfqpoint{3.341804in}{2.239342in}}%
\pgfpathlineto{\pgfqpoint{3.317109in}{2.229682in}}%
\pgfpathlineto{\pgfqpoint{3.291104in}{2.216986in}}%
\pgfpathlineto{\pgfqpoint{3.265928in}{2.202261in}}%
\pgfpathlineto{\pgfqpoint{3.239805in}{2.184361in}}%
\pgfpathlineto{\pgfqpoint{3.214775in}{2.164519in}}%
\pgfpathlineto{\pgfqpoint{3.190900in}{2.142893in}}%
\pgfpathlineto{\pgfqpoint{3.166657in}{2.117912in}}%
\pgfpathlineto{\pgfqpoint{3.143835in}{2.091233in}}%
\pgfpathlineto{\pgfqpoint{3.121079in}{2.061107in}}%
\pgfpathlineto{\pgfqpoint{3.099952in}{2.029463in}}%
\pgfpathlineto{\pgfqpoint{3.079251in}{1.994406in}}%
\pgfpathlineto{\pgfqpoint{3.059218in}{1.955915in}}%
\pgfpathlineto{\pgfqpoint{3.040058in}{1.914015in}}%
\pgfpathlineto{\pgfqpoint{3.022809in}{1.871041in}}%
\pgfpathlineto{\pgfqpoint{3.005790in}{1.822536in}}%
\pgfpathlineto{\pgfqpoint{2.990067in}{1.770819in}}%
\pgfpathlineto{\pgfqpoint{2.975708in}{1.715979in}}%
\pgfpathlineto{\pgfqpoint{2.962284in}{1.655680in}}%
\pgfpathlineto{\pgfqpoint{2.950496in}{1.592386in}}%
\pgfpathlineto{\pgfqpoint{2.940383in}{1.526185in}}%
\pgfpathlineto{\pgfqpoint{2.931745in}{1.454681in}}%
\pgfpathlineto{\pgfqpoint{2.925082in}{1.380399in}}%
\pgfpathlineto{\pgfqpoint{2.920647in}{1.305899in}}%
\pgfpathlineto{\pgfqpoint{2.918444in}{1.231270in}}%
\pgfpathlineto{\pgfqpoint{2.918545in}{1.159087in}}%
\pgfpathlineto{\pgfqpoint{2.920787in}{1.091931in}}%
\pgfpathlineto{\pgfqpoint{2.925177in}{1.027412in}}%
\pgfpathlineto{\pgfqpoint{2.931192in}{0.970580in}}%
\pgfpathlineto{\pgfqpoint{2.938760in}{0.919034in}}%
\pgfpathlineto{\pgfqpoint{2.947651in}{0.872852in}}%
\pgfpathlineto{\pgfqpoint{2.958213in}{0.829714in}}%
\pgfpathlineto{\pgfqpoint{2.969670in}{0.792114in}}%
\pgfpathlineto{\pgfqpoint{2.982463in}{0.757773in}}%
\pgfpathlineto{\pgfqpoint{2.996425in}{0.726812in}}%
\pgfpathlineto{\pgfqpoint{3.011299in}{0.699300in}}%
\pgfpathlineto{\pgfqpoint{3.026739in}{0.675225in}}%
\pgfpathlineto{\pgfqpoint{3.043828in}{0.652656in}}%
\pgfpathlineto{\pgfqpoint{3.062495in}{0.631788in}}%
\pgfpathlineto{\pgfqpoint{3.082602in}{0.612753in}}%
\pgfpathlineto{\pgfqpoint{3.103961in}{0.595592in}}%
\pgfpathlineto{\pgfqpoint{3.128268in}{0.579069in}}%
\pgfpathlineto{\pgfqpoint{3.153537in}{0.564554in}}%
\pgfpathlineto{\pgfqpoint{3.181571in}{0.550952in}}%
\pgfpathlineto{\pgfqpoint{3.214371in}{0.537647in}}%
\pgfpathlineto{\pgfqpoint{3.249846in}{0.525712in}}%
\pgfpathlineto{\pgfqpoint{3.290011in}{0.514571in}}%
\pgfpathlineto{\pgfqpoint{3.334820in}{0.504423in}}%
\pgfpathlineto{\pgfqpoint{3.386372in}{0.494999in}}%
\pgfpathlineto{\pgfqpoint{3.446798in}{0.486257in}}%
\pgfpathlineto{\pgfqpoint{3.518243in}{0.478282in}}%
\pgfpathlineto{\pgfqpoint{3.600685in}{0.471409in}}%
\pgfpathlineto{\pgfqpoint{3.696268in}{0.465713in}}%
\pgfpathlineto{\pgfqpoint{3.807144in}{0.461369in}}%
\pgfpathlineto{\pgfqpoint{3.933291in}{0.458719in}}%
\pgfpathlineto{\pgfqpoint{4.063808in}{0.458211in}}%
\pgfpathlineto{\pgfqpoint{4.187792in}{0.459914in}}%
\pgfpathlineto{\pgfqpoint{4.294335in}{0.463521in}}%
\pgfpathlineto{\pgfqpoint{4.381234in}{0.468574in}}%
\pgfpathlineto{\pgfqpoint{4.450636in}{0.474701in}}%
\pgfpathlineto{\pgfqpoint{4.506850in}{0.481799in}}%
\pgfpathlineto{\pgfqpoint{4.552009in}{0.489658in}}%
\pgfpathlineto{\pgfqpoint{4.588239in}{0.498115in}}%
\pgfpathlineto{\pgfqpoint{4.617656in}{0.507110in}}%
\pgfpathlineto{\pgfqpoint{4.642328in}{0.516843in}}%
\pgfpathlineto{\pgfqpoint{4.664194in}{0.527940in}}%
\pgfpathlineto{\pgfqpoint{4.681238in}{0.538945in}}%
\pgfpathlineto{\pgfqpoint{4.697164in}{0.551953in}}%
\pgfpathlineto{\pgfqpoint{4.710076in}{0.565289in}}%
\pgfpathlineto{\pgfqpoint{4.721578in}{0.580218in}}%
\pgfpathlineto{\pgfqpoint{4.731557in}{0.596521in}}%
\pgfpathlineto{\pgfqpoint{4.741000in}{0.616134in}}%
\pgfpathlineto{\pgfqpoint{4.749521in}{0.639027in}}%
\pgfpathlineto{\pgfqpoint{4.757522in}{0.667450in}}%
\pgfpathlineto{\pgfqpoint{4.764572in}{0.701345in}}%
\pgfpathlineto{\pgfqpoint{4.770840in}{0.743043in}}%
\pgfpathlineto{\pgfqpoint{4.776327in}{0.794934in}}%
\pgfpathlineto{\pgfqpoint{4.781278in}{0.864398in}}%
\pgfpathlineto{\pgfqpoint{4.785468in}{0.956371in}}%
\pgfpathlineto{\pgfqpoint{4.789000in}{1.085745in}}%
\pgfpathlineto{\pgfqpoint{4.791852in}{1.277385in}}%
\pgfpathlineto{\pgfqpoint{4.793959in}{1.581057in}}%
\pgfpathlineto{\pgfqpoint{4.794962in}{2.071429in}}%
\pgfpathlineto{\pgfqpoint{4.793967in}{2.559311in}}%
\pgfpathlineto{\pgfqpoint{4.791733in}{2.745981in}}%
\pgfpathlineto{\pgfqpoint{4.788955in}{2.818091in}}%
\pgfpathlineto{\pgfqpoint{4.785731in}{2.850227in}}%
\pgfpathlineto{\pgfqpoint{4.781879in}{2.867057in}}%
\pgfpathlineto{\pgfqpoint{4.777744in}{2.875780in}}%
\pgfpathlineto{\pgfqpoint{4.773097in}{2.880982in}}%
\pgfpathlineto{\pgfqpoint{4.767363in}{2.884504in}}%
\pgfpathlineto{\pgfqpoint{4.756853in}{2.887622in}}%
\pgfpathlineto{\pgfqpoint{4.739548in}{2.889639in}}%
\pgfpathlineto{\pgfqpoint{4.704762in}{2.890882in}}%
\pgfpathlineto{\pgfqpoint{4.602524in}{2.891538in}}%
\pgfpathlineto{\pgfqpoint{3.952100in}{2.891742in}}%
\pgfpathlineto{\pgfqpoint{0.617321in}{2.890753in}}%
\pgfpathlineto{\pgfqpoint{0.549910in}{2.888858in}}%
\pgfpathlineto{\pgfqpoint{0.521735in}{2.886179in}}%
\pgfpathlineto{\pgfqpoint{0.504666in}{2.882389in}}%
\pgfpathlineto{\pgfqpoint{0.494501in}{2.878011in}}%
\pgfpathlineto{\pgfqpoint{0.487180in}{2.872667in}}%
\pgfpathlineto{\pgfqpoint{0.481152in}{2.865519in}}%
\pgfpathlineto{\pgfqpoint{0.475664in}{2.854804in}}%
\pgfpathlineto{\pgfqpoint{0.471318in}{2.840737in}}%
\pgfpathlineto{\pgfqpoint{0.467301in}{2.818823in}}%
\pgfpathlineto{\pgfqpoint{0.463927in}{2.786700in}}%
\pgfpathlineto{\pgfqpoint{0.460918in}{2.734544in}}%
\pgfpathlineto{\pgfqpoint{0.458363in}{2.647473in}}%
\pgfpathlineto{\pgfqpoint{0.456575in}{2.523031in}}%
\pgfpathlineto{\pgfqpoint{0.456575in}{2.523031in}}%
\pgfusepath{stroke}%
\end{pgfscope}%
\begin{pgfscope}%
\pgfpathrectangle{\pgfqpoint{0.448634in}{0.402556in}}{\pgfqpoint{4.350661in}{2.489204in}} %
\pgfusepath{clip}%
\pgfsetrectcap%
\pgfsetroundjoin%
\pgfsetlinewidth{1.003750pt}%
\definecolor{currentstroke}{rgb}{1.000000,0.498039,0.054902}%
\pgfsetstrokecolor{currentstroke}%
\pgfsetdash{}{0pt}%
\pgfpathmoveto{\pgfqpoint{4.798840in}{2.852369in}}%
\pgfpathlineto{\pgfqpoint{4.797564in}{2.889610in}}%
\pgfpathlineto{\pgfqpoint{4.796215in}{2.891483in}}%
\pgfpathlineto{\pgfqpoint{4.787551in}{2.891760in}}%
\pgfpathlineto{\pgfqpoint{0.452128in}{2.891659in}}%
\pgfpathlineto{\pgfqpoint{0.450530in}{2.890082in}}%
\pgfpathlineto{\pgfqpoint{0.449454in}{2.882763in}}%
\pgfpathlineto{\pgfqpoint{0.448970in}{2.845432in}}%
\pgfpathlineto{\pgfqpoint{0.448743in}{2.494454in}}%
\pgfpathlineto{\pgfqpoint{0.449624in}{0.615107in}}%
\pgfpathlineto{\pgfqpoint{0.451433in}{0.510586in}}%
\pgfpathlineto{\pgfqpoint{0.453993in}{0.473374in}}%
\pgfpathlineto{\pgfqpoint{0.457406in}{0.453868in}}%
\pgfpathlineto{\pgfqpoint{0.461540in}{0.442384in}}%
\pgfpathlineto{\pgfqpoint{0.466739in}{0.434437in}}%
\pgfpathlineto{\pgfqpoint{0.473595in}{0.428350in}}%
\pgfpathlineto{\pgfqpoint{0.483492in}{0.423244in}}%
\pgfpathlineto{\pgfqpoint{0.491854in}{0.420501in}}%
\pgfpathlineto{\pgfqpoint{0.491854in}{0.420501in}}%
\pgfusepath{stroke}%
\end{pgfscope}%
\begin{pgfscope}%
\pgfpathrectangle{\pgfqpoint{0.448634in}{0.402556in}}{\pgfqpoint{4.350661in}{2.489204in}} %
\pgfusepath{clip}%
\pgfsetrectcap%
\pgfsetroundjoin%
\pgfsetlinewidth{1.003750pt}%
\definecolor{currentstroke}{rgb}{1.000000,0.498039,0.054902}%
\pgfsetstrokecolor{currentstroke}%
\pgfsetdash{}{0pt}%
\pgfpathmoveto{\pgfqpoint{0.456424in}{1.370137in}}%
\pgfpathlineto{\pgfqpoint{0.459610in}{1.118755in}}%
\pgfpathlineto{\pgfqpoint{0.463695in}{0.962007in}}%
\pgfpathlineto{\pgfqpoint{0.468519in}{0.857610in}}%
\pgfpathlineto{\pgfqpoint{0.474082in}{0.783210in}}%
\pgfpathlineto{\pgfqpoint{0.480226in}{0.728906in}}%
\pgfpathlineto{\pgfqpoint{0.486970in}{0.687306in}}%
\pgfpathlineto{\pgfqpoint{0.494537in}{0.653558in}}%
\pgfpathlineto{\pgfqpoint{0.503107in}{0.625355in}}%
\pgfpathlineto{\pgfqpoint{0.512193in}{0.602749in}}%
\pgfpathlineto{\pgfqpoint{0.522200in}{0.583508in}}%
\pgfpathlineto{\pgfqpoint{0.534108in}{0.565743in}}%
\pgfpathlineto{\pgfqpoint{0.546263in}{0.551507in}}%
\pgfpathlineto{\pgfqpoint{0.559728in}{0.538907in}}%
\pgfpathlineto{\pgfqpoint{0.576129in}{0.526693in}}%
\pgfpathlineto{\pgfqpoint{0.595483in}{0.515351in}}%
\pgfpathlineto{\pgfqpoint{0.617681in}{0.505147in}}%
\pgfpathlineto{\pgfqpoint{0.642568in}{0.496153in}}%
\pgfpathlineto{\pgfqpoint{0.672126in}{0.487778in}}%
\pgfpathlineto{\pgfqpoint{0.708443in}{0.479824in}}%
\pgfpathlineto{\pgfqpoint{0.753649in}{0.472325in}}%
\pgfpathlineto{\pgfqpoint{0.807717in}{0.465660in}}%
\pgfpathlineto{\pgfqpoint{0.877116in}{0.459475in}}%
\pgfpathlineto{\pgfqpoint{0.961828in}{0.454230in}}%
\pgfpathlineto{\pgfqpoint{1.068351in}{0.449916in}}%
\pgfpathlineto{\pgfqpoint{1.201018in}{0.446839in}}%
\pgfpathlineto{\pgfqpoint{1.357637in}{0.445481in}}%
\pgfpathlineto{\pgfqpoint{1.525135in}{0.446232in}}%
\pgfpathlineto{\pgfqpoint{1.686088in}{0.449142in}}%
\pgfpathlineto{\pgfqpoint{1.823074in}{0.453747in}}%
\pgfpathlineto{\pgfqpoint{1.938245in}{0.459764in}}%
\pgfpathlineto{\pgfqpoint{2.031582in}{0.466759in}}%
\pgfpathlineto{\pgfqpoint{2.109580in}{0.474745in}}%
\pgfpathlineto{\pgfqpoint{2.174384in}{0.483535in}}%
\pgfpathlineto{\pgfqpoint{2.228139in}{0.492940in}}%
\pgfpathlineto{\pgfqpoint{2.275119in}{0.503356in}}%
\pgfpathlineto{\pgfqpoint{2.315282in}{0.514501in}}%
\pgfpathlineto{\pgfqpoint{2.350698in}{0.526659in}}%
\pgfpathlineto{\pgfqpoint{2.381320in}{0.539536in}}%
\pgfpathlineto{\pgfqpoint{2.407164in}{0.552659in}}%
\pgfpathlineto{\pgfqpoint{2.430226in}{0.566639in}}%
\pgfpathlineto{\pgfqpoint{2.452282in}{0.582602in}}%
\pgfpathlineto{\pgfqpoint{2.471391in}{0.599069in}}%
\pgfpathlineto{\pgfqpoint{2.489240in}{0.617293in}}%
\pgfpathlineto{\pgfqpoint{2.505678in}{0.637180in}}%
\pgfpathlineto{\pgfqpoint{2.520620in}{0.658557in}}%
\pgfpathlineto{\pgfqpoint{2.535213in}{0.683314in}}%
\pgfpathlineto{\pgfqpoint{2.549115in}{0.711484in}}%
\pgfpathlineto{\pgfqpoint{2.562091in}{0.743004in}}%
\pgfpathlineto{\pgfqpoint{2.574020in}{0.777751in}}%
\pgfpathlineto{\pgfqpoint{2.585502in}{0.817970in}}%
\pgfpathlineto{\pgfqpoint{2.596809in}{0.866038in}}%
\pgfpathlineto{\pgfqpoint{2.607562in}{0.921948in}}%
\pgfpathlineto{\pgfqpoint{2.617925in}{0.988098in}}%
\pgfpathlineto{\pgfqpoint{2.627958in}{1.066918in}}%
\pgfpathlineto{\pgfqpoint{2.637941in}{1.163320in}}%
\pgfpathlineto{\pgfqpoint{2.648424in}{1.287199in}}%
\pgfpathlineto{\pgfqpoint{2.660103in}{1.453438in}}%
\pgfpathlineto{\pgfqpoint{2.674773in}{1.696801in}}%
\pgfpathlineto{\pgfqpoint{2.687716in}{1.945279in}}%
\pgfpathlineto{\pgfqpoint{2.692670in}{2.079573in}}%
\pgfpathlineto{\pgfqpoint{2.693829in}{2.166682in}}%
\pgfpathlineto{\pgfqpoint{2.692565in}{2.233870in}}%
\pgfpathlineto{\pgfqpoint{2.689436in}{2.286015in}}%
\pgfpathlineto{\pgfqpoint{2.684859in}{2.327999in}}%
\pgfpathlineto{\pgfqpoint{2.678725in}{2.364664in}}%
\pgfpathlineto{\pgfqpoint{2.671356in}{2.395897in}}%
\pgfpathlineto{\pgfqpoint{2.662489in}{2.423981in}}%
\pgfpathlineto{\pgfqpoint{2.652361in}{2.448778in}}%
\pgfpathlineto{\pgfqpoint{2.641365in}{2.470245in}}%
\pgfpathlineto{\pgfqpoint{2.628643in}{2.490425in}}%
\pgfpathlineto{\pgfqpoint{2.614279in}{2.509106in}}%
\pgfpathlineto{\pgfqpoint{2.598443in}{2.526159in}}%
\pgfpathlineto{\pgfqpoint{2.579590in}{2.543005in}}%
\pgfpathlineto{\pgfqpoint{2.559532in}{2.557923in}}%
\pgfpathlineto{\pgfqpoint{2.536602in}{2.572183in}}%
\pgfpathlineto{\pgfqpoint{2.510850in}{2.585538in}}%
\pgfpathlineto{\pgfqpoint{2.482360in}{2.597837in}}%
\pgfpathlineto{\pgfqpoint{2.449134in}{2.609683in}}%
\pgfpathlineto{\pgfqpoint{2.411184in}{2.620696in}}%
\pgfpathlineto{\pgfqpoint{2.368552in}{2.630606in}}%
\pgfpathlineto{\pgfqpoint{2.321294in}{2.639221in}}%
\pgfpathlineto{\pgfqpoint{2.269467in}{2.646399in}}%
\pgfpathlineto{\pgfqpoint{2.210954in}{2.652193in}}%
\pgfpathlineto{\pgfqpoint{2.147967in}{2.656153in}}%
\pgfpathlineto{\pgfqpoint{2.080556in}{2.658135in}}%
\pgfpathlineto{\pgfqpoint{2.010948in}{2.657971in}}%
\pgfpathlineto{\pgfqpoint{1.939195in}{2.655572in}}%
\pgfpathlineto{\pgfqpoint{1.867527in}{2.650913in}}%
\pgfpathlineto{\pgfqpoint{1.798171in}{2.644140in}}%
\pgfpathlineto{\pgfqpoint{1.733341in}{2.635606in}}%
\pgfpathlineto{\pgfqpoint{1.673075in}{2.625521in}}%
\pgfpathlineto{\pgfqpoint{1.615274in}{2.613610in}}%
\pgfpathlineto{\pgfqpoint{1.562133in}{2.600402in}}%
\pgfpathlineto{\pgfqpoint{1.513681in}{2.586139in}}%
\pgfpathlineto{\pgfqpoint{1.467862in}{2.570344in}}%
\pgfpathlineto{\pgfqpoint{1.426794in}{2.553923in}}%
\pgfpathlineto{\pgfqpoint{1.388447in}{2.536289in}}%
\pgfpathlineto{\pgfqpoint{1.352878in}{2.517566in}}%
\pgfpathlineto{\pgfqpoint{1.320128in}{2.497922in}}%
\pgfpathlineto{\pgfqpoint{1.288379in}{2.476236in}}%
\pgfpathlineto{\pgfqpoint{1.259592in}{2.453861in}}%
\pgfpathlineto{\pgfqpoint{1.232050in}{2.429520in}}%
\pgfpathlineto{\pgfqpoint{1.207527in}{2.404898in}}%
\pgfpathlineto{\pgfqpoint{1.184409in}{2.378557in}}%
\pgfpathlineto{\pgfqpoint{1.162828in}{2.350561in}}%
\pgfpathlineto{\pgfqpoint{1.142891in}{2.321011in}}%
\pgfpathlineto{\pgfqpoint{1.124675in}{2.290041in}}%
\pgfpathlineto{\pgfqpoint{1.108225in}{2.257802in}}%
\pgfpathlineto{\pgfqpoint{1.092639in}{2.222199in}}%
\pgfpathlineto{\pgfqpoint{1.079059in}{2.185535in}}%
\pgfpathlineto{\pgfqpoint{1.067443in}{2.147998in}}%
\pgfpathlineto{\pgfqpoint{1.057187in}{2.107348in}}%
\pgfpathlineto{\pgfqpoint{1.049004in}{2.066086in}}%
\pgfpathlineto{\pgfqpoint{1.042513in}{2.021906in}}%
\pgfpathlineto{\pgfqpoint{1.038177in}{1.977382in}}%
\pgfpathlineto{\pgfqpoint{1.035866in}{1.930167in}}%
\pgfpathlineto{\pgfqpoint{1.035826in}{1.882878in}}%
\pgfpathlineto{\pgfqpoint{1.038031in}{1.835656in}}%
\pgfpathlineto{\pgfqpoint{1.042474in}{1.788641in}}%
\pgfpathlineto{\pgfqpoint{1.049176in}{1.741979in}}%
\pgfpathlineto{\pgfqpoint{1.057644in}{1.698239in}}%
\pgfpathlineto{\pgfqpoint{1.068221in}{1.655105in}}%
\pgfpathlineto{\pgfqpoint{1.080962in}{1.612745in}}%
\pgfpathlineto{\pgfqpoint{1.095031in}{1.573617in}}%
\pgfpathlineto{\pgfqpoint{1.111115in}{1.535520in}}%
\pgfpathlineto{\pgfqpoint{1.128118in}{1.500775in}}%
\pgfpathlineto{\pgfqpoint{1.146930in}{1.467274in}}%
\pgfpathlineto{\pgfqpoint{1.167531in}{1.435181in}}%
\pgfpathlineto{\pgfqpoint{1.189874in}{1.404652in}}%
\pgfpathlineto{\pgfqpoint{1.213884in}{1.375828in}}%
\pgfpathlineto{\pgfqpoint{1.237817in}{1.350457in}}%
\pgfpathlineto{\pgfqpoint{1.264748in}{1.325237in}}%
\pgfpathlineto{\pgfqpoint{1.292991in}{1.301972in}}%
\pgfpathlineto{\pgfqpoint{1.322398in}{1.280678in}}%
\pgfpathlineto{\pgfqpoint{1.352820in}{1.261340in}}%
\pgfpathlineto{\pgfqpoint{1.386095in}{1.242889in}}%
\pgfpathlineto{\pgfqpoint{1.420190in}{1.226516in}}%
\pgfpathlineto{\pgfqpoint{1.457024in}{1.211329in}}%
\pgfpathlineto{\pgfqpoint{1.496554in}{1.197536in}}%
\pgfpathlineto{\pgfqpoint{1.538719in}{1.185287in}}%
\pgfpathlineto{\pgfqpoint{1.583441in}{1.174641in}}%
\pgfpathlineto{\pgfqpoint{1.634929in}{1.164775in}}%
\pgfpathlineto{\pgfqpoint{1.706063in}{1.153745in}}%
\pgfpathlineto{\pgfqpoint{1.768492in}{1.143417in}}%
\pgfpathlineto{\pgfqpoint{1.796122in}{1.136567in}}%
\pgfpathlineto{\pgfqpoint{1.812683in}{1.130481in}}%
\pgfpathlineto{\pgfqpoint{1.824471in}{1.124102in}}%
\pgfpathlineto{\pgfqpoint{1.833209in}{1.116741in}}%
\pgfpathlineto{\pgfqpoint{1.838498in}{1.108890in}}%
\pgfpathlineto{\pgfqpoint{1.840588in}{1.101849in}}%
\pgfpathlineto{\pgfqpoint{1.840619in}{1.094412in}}%
\pgfpathlineto{\pgfqpoint{1.837931in}{1.084986in}}%
\pgfpathlineto{\pgfqpoint{1.833246in}{1.076615in}}%
\pgfpathlineto{\pgfqpoint{1.825819in}{1.067542in}}%
\pgfpathlineto{\pgfqpoint{1.813813in}{1.056850in}}%
\pgfpathlineto{\pgfqpoint{1.798819in}{1.046763in}}%
\pgfpathlineto{\pgfqpoint{1.781016in}{1.037462in}}%
\pgfpathlineto{\pgfqpoint{1.758447in}{1.028391in}}%
\pgfpathlineto{\pgfqpoint{1.733203in}{1.020815in}}%
\pgfpathlineto{\pgfqpoint{1.705410in}{1.014872in}}%
\pgfpathlineto{\pgfqpoint{1.675178in}{1.010714in}}%
\pgfpathlineto{\pgfqpoint{1.642610in}{1.008507in}}%
\pgfpathlineto{\pgfqpoint{1.607809in}{1.008432in}}%
\pgfpathlineto{\pgfqpoint{1.570886in}{1.010691in}}%
\pgfpathlineto{\pgfqpoint{1.534118in}{1.015181in}}%
\pgfpathlineto{\pgfqpoint{1.495454in}{1.022233in}}%
\pgfpathlineto{\pgfqpoint{1.457161in}{1.031563in}}%
\pgfpathlineto{\pgfqpoint{1.419337in}{1.043132in}}%
\pgfpathlineto{\pgfqpoint{1.382089in}{1.056929in}}%
\pgfpathlineto{\pgfqpoint{1.347544in}{1.072019in}}%
\pgfpathlineto{\pgfqpoint{1.313727in}{1.089133in}}%
\pgfpathlineto{\pgfqpoint{1.280762in}{1.108299in}}%
\pgfpathlineto{\pgfqpoint{1.248782in}{1.129536in}}%
\pgfpathlineto{\pgfqpoint{1.219708in}{1.151422in}}%
\pgfpathlineto{\pgfqpoint{1.191752in}{1.175138in}}%
\pgfpathlineto{\pgfqpoint{1.165031in}{1.200649in}}%
\pgfpathlineto{\pgfqpoint{1.139653in}{1.227898in}}%
\pgfpathlineto{\pgfqpoint{1.115714in}{1.256800in}}%
\pgfpathlineto{\pgfqpoint{1.093288in}{1.287251in}}%
\pgfpathlineto{\pgfqpoint{1.071178in}{1.321163in}}%
\pgfpathlineto{\pgfqpoint{1.050868in}{1.356520in}}%
\pgfpathlineto{\pgfqpoint{1.032365in}{1.393152in}}%
\pgfpathlineto{\pgfqpoint{1.014718in}{1.433142in}}%
\pgfpathlineto{\pgfqpoint{0.999024in}{1.474185in}}%
\pgfpathlineto{\pgfqpoint{0.984506in}{1.518461in}}%
\pgfpathlineto{\pgfqpoint{0.972010in}{1.563537in}}%
\pgfpathlineto{\pgfqpoint{0.960944in}{1.611678in}}%
\pgfpathlineto{\pgfqpoint{0.951530in}{1.662824in}}%
\pgfpathlineto{\pgfqpoint{0.944286in}{1.714431in}}%
\pgfpathlineto{\pgfqpoint{0.938950in}{1.768847in}}%
\pgfpathlineto{\pgfqpoint{0.935870in}{1.823491in}}%
\pgfpathlineto{\pgfqpoint{0.935034in}{1.878240in}}%
\pgfpathlineto{\pgfqpoint{0.936466in}{1.932973in}}%
\pgfpathlineto{\pgfqpoint{0.940005in}{1.985084in}}%
\pgfpathlineto{\pgfqpoint{0.945759in}{2.036935in}}%
\pgfpathlineto{\pgfqpoint{0.953410in}{2.085938in}}%
\pgfpathlineto{\pgfqpoint{0.962764in}{2.132000in}}%
\pgfpathlineto{\pgfqpoint{0.974287in}{2.177414in}}%
\pgfpathlineto{\pgfqpoint{0.987332in}{2.219653in}}%
\pgfpathlineto{\pgfqpoint{1.001667in}{2.258654in}}%
\pgfpathlineto{\pgfqpoint{1.018051in}{2.296583in}}%
\pgfpathlineto{\pgfqpoint{1.035401in}{2.331101in}}%
\pgfpathlineto{\pgfqpoint{1.054650in}{2.364275in}}%
\pgfpathlineto{\pgfqpoint{1.074406in}{2.393984in}}%
\pgfpathlineto{\pgfqpoint{1.095771in}{2.422197in}}%
\pgfpathlineto{\pgfqpoint{1.118662in}{2.448797in}}%
\pgfpathlineto{\pgfqpoint{1.142967in}{2.473701in}}%
\pgfpathlineto{\pgfqpoint{1.168550in}{2.496867in}}%
\pgfpathlineto{\pgfqpoint{1.197085in}{2.519662in}}%
\pgfpathlineto{\pgfqpoint{1.226727in}{2.540526in}}%
\pgfpathlineto{\pgfqpoint{1.259242in}{2.560673in}}%
\pgfpathlineto{\pgfqpoint{1.294612in}{2.579881in}}%
\pgfpathlineto{\pgfqpoint{1.332792in}{2.597982in}}%
\pgfpathlineto{\pgfqpoint{1.373719in}{2.614859in}}%
\pgfpathlineto{\pgfqpoint{1.417319in}{2.630445in}}%
\pgfpathlineto{\pgfqpoint{1.465632in}{2.645312in}}%
\pgfpathlineto{\pgfqpoint{1.518640in}{2.659204in}}%
\pgfpathlineto{\pgfqpoint{1.576309in}{2.671929in}}%
\pgfpathlineto{\pgfqpoint{1.638597in}{2.683344in}}%
\pgfpathlineto{\pgfqpoint{1.705462in}{2.693343in}}%
\pgfpathlineto{\pgfqpoint{1.779027in}{2.702064in}}%
\pgfpathlineto{\pgfqpoint{1.857097in}{2.709077in}}%
\pgfpathlineto{\pgfqpoint{1.939633in}{2.714280in}}%
\pgfpathlineto{\pgfqpoint{2.026598in}{2.717513in}}%
\pgfpathlineto{\pgfqpoint{2.113605in}{2.718523in}}%
\pgfpathlineto{\pgfqpoint{2.198435in}{2.717303in}}%
\pgfpathlineto{\pgfqpoint{2.278866in}{2.713929in}}%
\pgfpathlineto{\pgfqpoint{2.352678in}{2.708598in}}%
\pgfpathlineto{\pgfqpoint{2.417657in}{2.701709in}}%
\pgfpathlineto{\pgfqpoint{2.473770in}{2.693630in}}%
\pgfpathlineto{\pgfqpoint{2.523140in}{2.684368in}}%
\pgfpathlineto{\pgfqpoint{2.565726in}{2.674202in}}%
\pgfpathlineto{\pgfqpoint{2.601510in}{2.663544in}}%
\pgfpathlineto{\pgfqpoint{2.632577in}{2.652142in}}%
\pgfpathlineto{\pgfqpoint{2.658899in}{2.640331in}}%
\pgfpathlineto{\pgfqpoint{2.682438in}{2.627436in}}%
\pgfpathlineto{\pgfqpoint{2.703062in}{2.613571in}}%
\pgfpathlineto{\pgfqpoint{2.720674in}{2.598978in}}%
\pgfpathlineto{\pgfqpoint{2.735263in}{2.584053in}}%
\pgfpathlineto{\pgfqpoint{2.748320in}{2.567377in}}%
\pgfpathlineto{\pgfqpoint{2.759553in}{2.549046in}}%
\pgfpathlineto{\pgfqpoint{2.768788in}{2.529306in}}%
\pgfpathlineto{\pgfqpoint{2.776017in}{2.508498in}}%
\pgfpathlineto{\pgfqpoint{2.781884in}{2.484540in}}%
\pgfpathlineto{\pgfqpoint{2.786102in}{2.457597in}}%
\pgfpathlineto{\pgfqpoint{2.788720in}{2.425384in}}%
\pgfpathlineto{\pgfqpoint{2.789427in}{2.388061in}}%
\pgfpathlineto{\pgfqpoint{2.787962in}{2.340801in}}%
\pgfpathlineto{\pgfqpoint{2.783672in}{2.278768in}}%
\pgfpathlineto{\pgfqpoint{2.774289in}{2.179783in}}%
\pgfpathlineto{\pgfqpoint{2.743611in}{1.868119in}}%
\pgfpathlineto{\pgfqpoint{2.730112in}{1.702060in}}%
\pgfpathlineto{\pgfqpoint{2.717287in}{1.515949in}}%
\pgfpathlineto{\pgfqpoint{2.702602in}{1.267597in}}%
\pgfpathlineto{\pgfqpoint{2.684434in}{0.964630in}}%
\pgfpathlineto{\pgfqpoint{2.675374in}{0.850600in}}%
\pgfpathlineto{\pgfqpoint{2.667030in}{0.771523in}}%
\pgfpathlineto{\pgfqpoint{2.658752in}{0.712543in}}%
\pgfpathlineto{\pgfqpoint{2.650176in}{0.666284in}}%
\pgfpathlineto{\pgfqpoint{2.640820in}{0.627931in}}%
\pgfpathlineto{\pgfqpoint{2.631145in}{0.597534in}}%
\pgfpathlineto{\pgfqpoint{2.621004in}{0.572745in}}%
\pgfpathlineto{\pgfqpoint{2.609856in}{0.551383in}}%
\pgfpathlineto{\pgfqpoint{2.598042in}{0.533534in}}%
\pgfpathlineto{\pgfqpoint{2.584496in}{0.517378in}}%
\pgfpathlineto{\pgfqpoint{2.571109in}{0.504669in}}%
\pgfpathlineto{\pgfqpoint{2.554789in}{0.492313in}}%
\pgfpathlineto{\pgfqpoint{2.537457in}{0.481914in}}%
\pgfpathlineto{\pgfqpoint{2.517374in}{0.472367in}}%
\pgfpathlineto{\pgfqpoint{2.492542in}{0.463178in}}%
\pgfpathlineto{\pgfqpoint{2.462979in}{0.454833in}}%
\pgfpathlineto{\pgfqpoint{2.428766in}{0.447542in}}%
\pgfpathlineto{\pgfqpoint{2.385671in}{0.440735in}}%
\pgfpathlineto{\pgfqpoint{2.331557in}{0.434581in}}%
\pgfpathlineto{\pgfqpoint{2.262115in}{0.429077in}}%
\pgfpathlineto{\pgfqpoint{2.170851in}{0.424236in}}%
\pgfpathlineto{\pgfqpoint{2.049086in}{0.420134in}}%
\pgfpathlineto{\pgfqpoint{1.879436in}{0.416783in}}%
\pgfpathlineto{\pgfqpoint{1.640159in}{0.414418in}}%
\pgfpathlineto{\pgfqpoint{1.322562in}{0.413569in}}%
\pgfpathlineto{\pgfqpoint{1.020194in}{0.414850in}}%
\pgfpathlineto{\pgfqpoint{0.822256in}{0.417715in}}%
\pgfpathlineto{\pgfqpoint{0.704835in}{0.421430in}}%
\pgfpathlineto{\pgfqpoint{0.630976in}{0.425829in}}%
\pgfpathlineto{\pgfqpoint{0.583316in}{0.430734in}}%
\pgfpathlineto{\pgfqpoint{0.551033in}{0.436123in}}%
\pgfpathlineto{\pgfqpoint{0.527708in}{0.442189in}}%
\pgfpathlineto{\pgfqpoint{0.511250in}{0.448625in}}%
\pgfpathlineto{\pgfqpoint{0.499549in}{0.455216in}}%
\pgfpathlineto{\pgfqpoint{0.488916in}{0.463841in}}%
\pgfpathlineto{\pgfqpoint{0.481322in}{0.472730in}}%
\pgfpathlineto{\pgfqpoint{0.474078in}{0.485127in}}%
\pgfpathlineto{\pgfqpoint{0.468753in}{0.498748in}}%
\pgfpathlineto{\pgfqpoint{0.463870in}{0.517848in}}%
\pgfpathlineto{\pgfqpoint{0.459679in}{0.544796in}}%
\pgfpathlineto{\pgfqpoint{0.456386in}{0.581938in}}%
\pgfpathlineto{\pgfqpoint{0.453731in}{0.639106in}}%
\pgfpathlineto{\pgfqpoint{0.451681in}{0.736155in}}%
\pgfpathlineto{\pgfqpoint{0.450220in}{0.927815in}}%
\pgfpathlineto{\pgfqpoint{0.449345in}{1.403252in}}%
\pgfpathlineto{\pgfqpoint{0.449543in}{2.682703in}}%
\pgfpathlineto{\pgfqpoint{0.451011in}{2.856932in}}%
\pgfpathlineto{\pgfqpoint{0.452802in}{2.879219in}}%
\pgfpathlineto{\pgfqpoint{0.455188in}{2.886108in}}%
\pgfpathlineto{\pgfqpoint{0.458626in}{2.889028in}}%
\pgfpathlineto{\pgfqpoint{0.464996in}{2.890553in}}%
\pgfpathlineto{\pgfqpoint{0.482377in}{2.891423in}}%
\pgfpathlineto{\pgfqpoint{0.565038in}{2.891729in}}%
\pgfpathlineto{\pgfqpoint{2.733842in}{2.891760in}}%
\pgfpathlineto{\pgfqpoint{4.789510in}{2.890885in}}%
\pgfpathlineto{\pgfqpoint{4.793727in}{2.889730in}}%
\pgfpathlineto{\pgfqpoint{4.795481in}{2.888307in}}%
\pgfpathlineto{\pgfqpoint{4.797106in}{2.881145in}}%
\pgfpathlineto{\pgfqpoint{4.797997in}{2.858771in}}%
\pgfpathlineto{\pgfqpoint{4.798039in}{2.856283in}}%
\pgfpathlineto{\pgfqpoint{4.798039in}{2.856283in}}%
\pgfusepath{stroke}%
\end{pgfscope}%
\begin{pgfscope}%
\pgfpathrectangle{\pgfqpoint{0.448634in}{0.402556in}}{\pgfqpoint{4.350661in}{2.489204in}} %
\pgfusepath{clip}%
\pgfsetrectcap%
\pgfsetroundjoin%
\pgfsetlinewidth{1.003750pt}%
\definecolor{currentstroke}{rgb}{1.000000,0.498039,0.054902}%
\pgfsetstrokecolor{currentstroke}%
\pgfsetdash{}{0pt}%
\pgfpathmoveto{\pgfqpoint{3.428772in}{0.402610in}}%
\pgfpathlineto{\pgfqpoint{2.806632in}{0.403760in}}%
\pgfpathlineto{\pgfqpoint{2.769692in}{0.405578in}}%
\pgfpathlineto{\pgfqpoint{2.754632in}{0.408064in}}%
\pgfpathlineto{\pgfqpoint{2.746391in}{0.411198in}}%
\pgfpathlineto{\pgfqpoint{2.740943in}{0.415265in}}%
\pgfpathlineto{\pgfqpoint{2.736784in}{0.420984in}}%
\pgfpathlineto{\pgfqpoint{2.733281in}{0.430071in}}%
\pgfpathlineto{\pgfqpoint{2.730449in}{0.444636in}}%
\pgfpathlineto{\pgfqpoint{2.728238in}{0.469392in}}%
\pgfpathlineto{\pgfqpoint{2.726470in}{0.519131in}}%
\pgfpathlineto{\pgfqpoint{2.725711in}{0.613715in}}%
\pgfpathlineto{\pgfqpoint{2.726842in}{0.768038in}}%
\pgfpathlineto{\pgfqpoint{2.730556in}{0.962148in}}%
\pgfpathlineto{\pgfqpoint{2.736611in}{1.158670in}}%
\pgfpathlineto{\pgfqpoint{2.744092in}{1.327718in}}%
\pgfpathlineto{\pgfqpoint{2.753201in}{1.484189in}}%
\pgfpathlineto{\pgfqpoint{2.763257in}{1.620609in}}%
\pgfpathlineto{\pgfqpoint{2.776118in}{1.764216in}}%
\pgfpathlineto{\pgfqpoint{2.788914in}{1.877776in}}%
\pgfpathlineto{\pgfqpoint{2.805748in}{2.005740in}}%
\pgfpathlineto{\pgfqpoint{2.821176in}{2.101198in}}%
\pgfpathlineto{\pgfqpoint{2.838359in}{2.193718in}}%
\pgfpathlineto{\pgfqpoint{2.859135in}{2.292966in}}%
\pgfpathlineto{\pgfqpoint{2.887209in}{2.425960in}}%
\pgfpathlineto{\pgfqpoint{2.896991in}{2.479559in}}%
\pgfpathlineto{\pgfqpoint{2.901543in}{2.516523in}}%
\pgfpathlineto{\pgfqpoint{2.902849in}{2.543854in}}%
\pgfpathlineto{\pgfqpoint{2.901957in}{2.566223in}}%
\pgfpathlineto{\pgfqpoint{2.899151in}{2.585863in}}%
\pgfpathlineto{\pgfqpoint{2.894794in}{2.602546in}}%
\pgfpathlineto{\pgfqpoint{2.888484in}{2.618388in}}%
\pgfpathlineto{\pgfqpoint{2.880257in}{2.633033in}}%
\pgfpathlineto{\pgfqpoint{2.870348in}{2.646246in}}%
\pgfpathlineto{\pgfqpoint{2.857400in}{2.659530in}}%
\pgfpathlineto{\pgfqpoint{2.843189in}{2.671010in}}%
\pgfpathlineto{\pgfqpoint{2.824237in}{2.683209in}}%
\pgfpathlineto{\pgfqpoint{2.802413in}{2.694418in}}%
\pgfpathlineto{\pgfqpoint{2.775809in}{2.705369in}}%
\pgfpathlineto{\pgfqpoint{2.744461in}{2.715715in}}%
\pgfpathlineto{\pgfqpoint{2.708436in}{2.725252in}}%
\pgfpathlineto{\pgfqpoint{2.665655in}{2.734289in}}%
\pgfpathlineto{\pgfqpoint{2.613991in}{2.742869in}}%
\pgfpathlineto{\pgfqpoint{2.553459in}{2.750589in}}%
\pgfpathlineto{\pgfqpoint{2.481920in}{2.757365in}}%
\pgfpathlineto{\pgfqpoint{2.399398in}{2.762839in}}%
\pgfpathlineto{\pgfqpoint{2.310269in}{2.766482in}}%
\pgfpathlineto{\pgfqpoint{2.175416in}{2.768725in}}%
\pgfpathlineto{\pgfqpoint{2.066653in}{2.767942in}}%
\pgfpathlineto{\pgfqpoint{1.953570in}{2.764859in}}%
\pgfpathlineto{\pgfqpoint{1.851429in}{2.759759in}}%
\pgfpathlineto{\pgfqpoint{1.745051in}{2.752169in}}%
\pgfpathlineto{\pgfqpoint{1.658373in}{2.743453in}}%
\pgfpathlineto{\pgfqpoint{1.580552in}{2.733461in}}%
\pgfpathlineto{\pgfqpoint{1.490057in}{2.719338in}}%
\pgfpathlineto{\pgfqpoint{1.417231in}{2.704698in}}%
\pgfpathlineto{\pgfqpoint{1.361992in}{2.690818in}}%
\pgfpathlineto{\pgfqpoint{1.311460in}{2.675819in}}%
\pgfpathlineto{\pgfqpoint{1.265667in}{2.659924in}}%
\pgfpathlineto{\pgfqpoint{1.222575in}{2.642586in}}%
\pgfpathlineto{\pgfqpoint{1.184324in}{2.624682in}}%
\pgfpathlineto{\pgfqpoint{1.148892in}{2.605623in}}%
\pgfpathlineto{\pgfqpoint{1.116331in}{2.585573in}}%
\pgfpathlineto{\pgfqpoint{1.092327in}{2.568512in}}%
\pgfpathlineto{\pgfqpoint{1.079760in}{2.558686in}}%
\pgfpathlineto{\pgfqpoint{1.051544in}{2.535379in}}%
\pgfpathlineto{\pgfqpoint{1.026312in}{2.511712in}}%
\pgfpathlineto{\pgfqpoint{1.002399in}{2.486318in}}%
\pgfpathlineto{\pgfqpoint{0.979913in}{2.459269in}}%
\pgfpathlineto{\pgfqpoint{0.958934in}{2.430678in}}%
\pgfpathlineto{\pgfqpoint{0.938264in}{2.398643in}}%
\pgfpathlineto{\pgfqpoint{0.923047in}{2.371385in}}%
\pgfpathlineto{\pgfqpoint{0.904513in}{2.334774in}}%
\pgfpathlineto{\pgfqpoint{0.887854in}{2.297001in}}%
\pgfpathlineto{\pgfqpoint{0.872131in}{2.255971in}}%
\pgfpathlineto{\pgfqpoint{0.857508in}{2.211741in}}%
\pgfpathlineto{\pgfqpoint{0.844762in}{2.166757in}}%
\pgfpathlineto{\pgfqpoint{0.838624in}{2.140306in}}%
\pgfpathlineto{\pgfqpoint{0.826982in}{2.087194in}}%
\pgfpathlineto{\pgfqpoint{0.816322in}{2.028715in}}%
\pgfpathlineto{\pgfqpoint{0.810087in}{1.984495in}}%
\pgfpathlineto{\pgfqpoint{0.808026in}{1.967238in}}%
\pgfpathlineto{\pgfqpoint{0.800076in}{1.898140in}}%
\pgfpathlineto{\pgfqpoint{0.793713in}{1.823823in}}%
\pgfpathlineto{\pgfqpoint{0.788799in}{1.741875in}}%
\pgfpathlineto{\pgfqpoint{0.786199in}{1.677225in}}%
\pgfpathlineto{\pgfqpoint{0.776951in}{1.453481in}}%
\pgfpathlineto{\pgfqpoint{0.773280in}{1.418894in}}%
\pgfpathlineto{\pgfqpoint{0.768298in}{1.389582in}}%
\pgfpathlineto{\pgfqpoint{0.762752in}{1.368108in}}%
\pgfpathlineto{\pgfqpoint{0.756722in}{1.352123in}}%
\pgfpathlineto{\pgfqpoint{0.749752in}{1.339519in}}%
\pgfpathlineto{\pgfqpoint{0.742201in}{1.330599in}}%
\pgfpathlineto{\pgfqpoint{0.734854in}{1.325312in}}%
\pgfpathlineto{\pgfqpoint{0.726558in}{1.322419in}}%
\pgfpathlineto{\pgfqpoint{0.717884in}{1.322223in}}%
\pgfpathlineto{\pgfqpoint{0.709412in}{1.324411in}}%
\pgfpathlineto{\pgfqpoint{0.699548in}{1.329604in}}%
\pgfpathlineto{\pgfqpoint{0.688894in}{1.338203in}}%
\pgfpathlineto{\pgfqpoint{0.677907in}{1.350248in}}%
\pgfpathlineto{\pgfqpoint{0.666886in}{1.365647in}}%
\pgfpathlineto{\pgfqpoint{0.654913in}{1.386417in}}%
\pgfpathlineto{\pgfqpoint{0.642574in}{1.412730in}}%
\pgfpathlineto{\pgfqpoint{0.630328in}{1.444629in}}%
\pgfpathlineto{\pgfqpoint{0.618504in}{1.482081in}}%
\pgfpathlineto{\pgfqpoint{0.608613in}{1.520256in}}%
\pgfpathlineto{\pgfqpoint{0.590203in}{1.612445in}}%
\pgfpathlineto{\pgfqpoint{0.581848in}{1.668884in}}%
\pgfpathlineto{\pgfqpoint{0.573137in}{1.740376in}}%
\pgfpathlineto{\pgfqpoint{0.567062in}{1.807213in}}%
\pgfpathlineto{\pgfqpoint{0.560532in}{1.896510in}}%
\pgfpathlineto{\pgfqpoint{0.555526in}{1.995910in}}%
\pgfpathlineto{\pgfqpoint{0.552564in}{2.097908in}}%
\pgfpathlineto{\pgfqpoint{0.551526in}{2.204935in}}%
\pgfpathlineto{\pgfqpoint{0.552728in}{2.309470in}}%
\pgfpathlineto{\pgfqpoint{0.556011in}{2.403981in}}%
\pgfpathlineto{\pgfqpoint{0.560953in}{2.483430in}}%
\pgfpathlineto{\pgfqpoint{0.567303in}{2.550240in}}%
\pgfpathlineto{\pgfqpoint{0.574928in}{2.606817in}}%
\pgfpathlineto{\pgfqpoint{0.582988in}{2.650657in}}%
\pgfpathlineto{\pgfqpoint{0.592756in}{2.691452in}}%
\pgfpathlineto{\pgfqpoint{0.602650in}{2.721756in}}%
\pgfpathlineto{\pgfqpoint{0.612983in}{2.746441in}}%
\pgfpathlineto{\pgfqpoint{0.624292in}{2.767692in}}%
\pgfpathlineto{\pgfqpoint{0.636231in}{2.785433in}}%
\pgfpathlineto{\pgfqpoint{0.649892in}{2.801461in}}%
\pgfpathlineto{\pgfqpoint{0.663386in}{2.814020in}}%
\pgfpathlineto{\pgfqpoint{0.679842in}{2.826135in}}%
\pgfpathlineto{\pgfqpoint{0.697326in}{2.836197in}}%
\pgfpathlineto{\pgfqpoint{0.715574in}{2.844285in}}%
\pgfpathlineto{\pgfqpoint{0.738439in}{2.852335in}}%
\pgfpathlineto{\pgfqpoint{0.765983in}{2.859639in}}%
\pgfpathlineto{\pgfqpoint{0.800300in}{2.866256in}}%
\pgfpathlineto{\pgfqpoint{0.841340in}{2.871832in}}%
\pgfpathlineto{\pgfqpoint{0.895547in}{2.876803in}}%
\pgfpathlineto{\pgfqpoint{0.969413in}{2.881069in}}%
\pgfpathlineto{\pgfqpoint{1.071608in}{2.884501in}}%
\pgfpathlineto{\pgfqpoint{1.219512in}{2.887074in}}%
\pgfpathlineto{\pgfqpoint{1.471844in}{2.889091in}}%
\pgfpathlineto{\pgfqpoint{1.956941in}{2.890384in}}%
\pgfpathlineto{\pgfqpoint{3.096814in}{2.890781in}}%
\pgfpathlineto{\pgfqpoint{3.995224in}{2.889388in}}%
\pgfpathlineto{\pgfqpoint{4.275833in}{2.887011in}}%
\pgfpathlineto{\pgfqpoint{4.412847in}{2.883743in}}%
\pgfpathlineto{\pgfqpoint{4.491081in}{2.879810in}}%
\pgfpathlineto{\pgfqpoint{4.543127in}{2.875163in}}%
\pgfpathlineto{\pgfqpoint{4.579810in}{2.869841in}}%
\pgfpathlineto{\pgfqpoint{4.607580in}{2.863763in}}%
\pgfpathlineto{\pgfqpoint{4.630623in}{2.856424in}}%
\pgfpathlineto{\pgfqpoint{4.648833in}{2.848228in}}%
\pgfpathlineto{\pgfqpoint{4.664136in}{2.838773in}}%
\pgfpathlineto{\pgfqpoint{4.676470in}{2.828576in}}%
\pgfpathlineto{\pgfqpoint{4.687502in}{2.816585in}}%
\pgfpathlineto{\pgfqpoint{4.697051in}{2.803027in}}%
\pgfpathlineto{\pgfqpoint{4.706194in}{2.786098in}}%
\pgfpathlineto{\pgfqpoint{4.714508in}{2.765827in}}%
\pgfpathlineto{\pgfqpoint{4.722462in}{2.740013in}}%
\pgfpathlineto{\pgfqpoint{4.729577in}{2.708703in}}%
\pgfpathlineto{\pgfqpoint{4.736162in}{2.669601in}}%
\pgfpathlineto{\pgfqpoint{4.742419in}{2.617826in}}%
\pgfpathlineto{\pgfqpoint{4.747859in}{2.553410in}}%
\pgfpathlineto{\pgfqpoint{4.752661in}{2.468958in}}%
\pgfpathlineto{\pgfqpoint{4.756610in}{2.359528in}}%
\pgfpathlineto{\pgfqpoint{4.759416in}{2.217681in}}%
\pgfpathlineto{\pgfqpoint{4.760596in}{2.043444in}}%
\pgfpathlineto{\pgfqpoint{4.759662in}{1.851779in}}%
\pgfpathlineto{\pgfqpoint{4.756587in}{1.667613in}}%
\pgfpathlineto{\pgfqpoint{4.751596in}{1.503428in}}%
\pgfpathlineto{\pgfqpoint{4.745410in}{1.374185in}}%
\pgfpathlineto{\pgfqpoint{4.738113in}{1.267479in}}%
\pgfpathlineto{\pgfqpoint{4.729621in}{1.175896in}}%
\pgfpathlineto{\pgfqpoint{4.720762in}{1.104428in}}%
\pgfpathlineto{\pgfqpoint{4.711045in}{1.043204in}}%
\pgfpathlineto{\pgfqpoint{4.700364in}{0.989829in}}%
\pgfpathlineto{\pgfqpoint{4.689055in}{0.944345in}}%
\pgfpathlineto{\pgfqpoint{4.676881in}{0.904394in}}%
\pgfpathlineto{\pgfqpoint{4.676095in}{0.902073in}}%
\pgfpathlineto{\pgfqpoint{4.676095in}{0.902073in}}%
\pgfusepath{stroke}%
\end{pgfscope}%
\begin{pgfscope}%
\pgfpathrectangle{\pgfqpoint{0.448634in}{0.402556in}}{\pgfqpoint{4.350661in}{2.489204in}} %
\pgfusepath{clip}%
\pgfsetrectcap%
\pgfsetroundjoin%
\pgfsetlinewidth{1.003750pt}%
\definecolor{currentstroke}{rgb}{1.000000,0.498039,0.054902}%
\pgfsetstrokecolor{currentstroke}%
\pgfsetdash{}{0pt}%
\pgfpathmoveto{\pgfqpoint{2.795520in}{1.982745in}}%
\pgfpathlineto{\pgfqpoint{2.781780in}{1.874357in}}%
\pgfpathlineto{\pgfqpoint{2.769351in}{1.758234in}}%
\pgfpathlineto{\pgfqpoint{2.758095in}{1.631942in}}%
\pgfpathlineto{\pgfqpoint{2.747786in}{1.490551in}}%
\pgfpathlineto{\pgfqpoint{2.738644in}{1.334082in}}%
\pgfpathlineto{\pgfqpoint{2.730580in}{1.157591in}}%
\pgfpathlineto{\pgfqpoint{2.723334in}{0.948663in}}%
\pgfpathlineto{\pgfqpoint{2.709783in}{0.530788in}}%
\pgfpathlineto{\pgfqpoint{2.705868in}{0.488716in}}%
\pgfpathlineto{\pgfqpoint{2.701769in}{0.464281in}}%
\pgfpathlineto{\pgfqpoint{2.697021in}{0.447744in}}%
\pgfpathlineto{\pgfqpoint{2.691859in}{0.436812in}}%
\pgfpathlineto{\pgfqpoint{2.686245in}{0.429229in}}%
\pgfpathlineto{\pgfqpoint{2.679348in}{0.423188in}}%
\pgfpathlineto{\pgfqpoint{2.669540in}{0.417856in}}%
\pgfpathlineto{\pgfqpoint{2.656987in}{0.413810in}}%
\pgfpathlineto{\pgfqpoint{2.637654in}{0.410337in}}%
\pgfpathlineto{\pgfqpoint{2.607297in}{0.407617in}}%
\pgfpathlineto{\pgfqpoint{2.555121in}{0.405574in}}%
\pgfpathlineto{\pgfqpoint{2.450714in}{0.404139in}}%
\pgfpathlineto{\pgfqpoint{2.176624in}{0.403275in}}%
\pgfpathlineto{\pgfqpoint{1.130290in}{0.402953in}}%
\pgfpathlineto{\pgfqpoint{0.516849in}{0.404175in}}%
\pgfpathlineto{\pgfqpoint{0.466848in}{0.405970in}}%
\pgfpathlineto{\pgfqpoint{0.456130in}{0.407931in}}%
\pgfpathlineto{\pgfqpoint{0.452340in}{0.410303in}}%
\pgfpathlineto{\pgfqpoint{0.450346in}{0.414662in}}%
\pgfpathlineto{\pgfqpoint{0.449266in}{0.424524in}}%
\pgfpathlineto{\pgfqpoint{0.448771in}{0.464344in}}%
\pgfpathlineto{\pgfqpoint{0.448640in}{0.850171in}}%
\pgfpathlineto{\pgfqpoint{0.448679in}{2.891318in}}%
\pgfpathlineto{\pgfqpoint{0.448679in}{2.891318in}}%
\pgfusepath{stroke}%
\end{pgfscope}%
\begin{pgfscope}%
\pgfpathrectangle{\pgfqpoint{0.448634in}{0.402556in}}{\pgfqpoint{4.350661in}{2.489204in}} %
\pgfusepath{clip}%
\pgfsetrectcap%
\pgfsetroundjoin%
\pgfsetlinewidth{1.003750pt}%
\definecolor{currentstroke}{rgb}{1.000000,0.498039,0.054902}%
\pgfsetstrokecolor{currentstroke}%
\pgfsetdash{}{0pt}%
\pgfpathmoveto{\pgfqpoint{3.428189in}{0.402586in}}%
\pgfpathlineto{\pgfqpoint{2.782121in}{0.403701in}}%
\pgfpathlineto{\pgfqpoint{2.753906in}{0.405674in}}%
\pgfpathlineto{\pgfqpoint{2.743328in}{0.408443in}}%
\pgfpathlineto{\pgfqpoint{2.737717in}{0.412188in}}%
\pgfpathlineto{\pgfqpoint{2.733668in}{0.417995in}}%
\pgfpathlineto{\pgfqpoint{2.730649in}{0.427307in}}%
\pgfpathlineto{\pgfqpoint{2.728388in}{0.442004in}}%
\pgfpathlineto{\pgfqpoint{2.726544in}{0.471794in}}%
\pgfpathlineto{\pgfqpoint{2.725216in}{0.534003in}}%
\pgfpathlineto{\pgfqpoint{2.725169in}{0.655973in}}%
\pgfpathlineto{\pgfqpoint{2.727377in}{0.832687in}}%
\pgfpathlineto{\pgfqpoint{2.732259in}{1.041703in}}%
\pgfpathlineto{\pgfqpoint{2.738851in}{1.223257in}}%
\pgfpathlineto{\pgfqpoint{2.747078in}{1.389766in}}%
\pgfpathlineto{\pgfqpoint{2.756608in}{1.538717in}}%
\pgfpathlineto{\pgfqpoint{2.768955in}{1.694887in}}%
\pgfpathlineto{\pgfqpoint{2.781228in}{1.816044in}}%
\pgfpathlineto{\pgfqpoint{2.794401in}{1.924524in}}%
\pgfpathlineto{\pgfqpoint{2.812737in}{2.054722in}}%
\pgfpathlineto{\pgfqpoint{2.828774in}{2.147512in}}%
\pgfpathlineto{\pgfqpoint{2.847382in}{2.242224in}}%
\pgfpathlineto{\pgfqpoint{2.895818in}{2.479699in}}%
\pgfpathlineto{\pgfqpoint{2.900204in}{2.516689in}}%
\pgfpathlineto{\pgfqpoint{2.901346in}{2.544029in}}%
\pgfpathlineto{\pgfqpoint{2.900291in}{2.566388in}}%
\pgfpathlineto{\pgfqpoint{2.897334in}{2.585999in}}%
\pgfpathlineto{\pgfqpoint{2.892836in}{2.602633in}}%
\pgfpathlineto{\pgfqpoint{2.886394in}{2.618405in}}%
\pgfpathlineto{\pgfqpoint{2.878058in}{2.632969in}}%
\pgfpathlineto{\pgfqpoint{2.868065in}{2.646100in}}%
\pgfpathlineto{\pgfqpoint{2.855050in}{2.659300in}}%
\pgfpathlineto{\pgfqpoint{2.840801in}{2.670717in}}%
\pgfpathlineto{\pgfqpoint{2.821822in}{2.682861in}}%
\pgfpathlineto{\pgfqpoint{2.799980in}{2.694026in}}%
\pgfpathlineto{\pgfqpoint{2.773366in}{2.704944in}}%
\pgfpathlineto{\pgfqpoint{2.742012in}{2.715266in}}%
\pgfpathlineto{\pgfqpoint{2.705983in}{2.724785in}}%
\pgfpathlineto{\pgfqpoint{2.663200in}{2.733810in}}%
\pgfpathlineto{\pgfqpoint{2.611535in}{2.742379in}}%
\pgfpathlineto{\pgfqpoint{2.551002in}{2.750090in}}%
\pgfpathlineto{\pgfqpoint{2.481632in}{2.756682in}}%
\pgfpathlineto{\pgfqpoint{2.399112in}{2.762200in}}%
\pgfpathlineto{\pgfqpoint{2.309985in}{2.765886in}}%
\pgfpathlineto{\pgfqpoint{2.188184in}{2.768096in}}%
\pgfpathlineto{\pgfqpoint{2.081595in}{2.767619in}}%
\pgfpathlineto{\pgfqpoint{1.968506in}{2.764840in}}%
\pgfpathlineto{\pgfqpoint{1.864180in}{2.759918in}}%
\pgfpathlineto{\pgfqpoint{1.757786in}{2.752593in}}%
\pgfpathlineto{\pgfqpoint{1.671087in}{2.744171in}}%
\pgfpathlineto{\pgfqpoint{1.591076in}{2.734193in}}%
\pgfpathlineto{\pgfqpoint{1.502689in}{2.720717in}}%
\pgfpathlineto{\pgfqpoint{1.427655in}{2.706083in}}%
\pgfpathlineto{\pgfqpoint{1.372350in}{2.692544in}}%
\pgfpathlineto{\pgfqpoint{1.321734in}{2.677921in}}%
\pgfpathlineto{\pgfqpoint{1.273765in}{2.661664in}}%
\pgfpathlineto{\pgfqpoint{1.230567in}{2.644672in}}%
\pgfpathlineto{\pgfqpoint{1.192197in}{2.627106in}}%
\pgfpathlineto{\pgfqpoint{1.156620in}{2.608403in}}%
\pgfpathlineto{\pgfqpoint{1.123890in}{2.588716in}}%
\pgfpathlineto{\pgfqpoint{1.095883in}{2.569568in}}%
\pgfpathlineto{\pgfqpoint{1.063936in}{2.543701in}}%
\pgfpathlineto{\pgfqpoint{1.038217in}{2.520732in}}%
\pgfpathlineto{\pgfqpoint{1.013766in}{2.496016in}}%
\pgfpathlineto{\pgfqpoint{0.990704in}{2.469610in}}%
\pgfpathlineto{\pgfqpoint{0.969124in}{2.441612in}}%
\pgfpathlineto{\pgfqpoint{0.949083in}{2.412154in}}%
\pgfpathlineto{\pgfqpoint{0.930604in}{2.381387in}}%
\pgfpathlineto{\pgfqpoint{0.906555in}{2.334052in}}%
\pgfpathlineto{\pgfqpoint{0.889925in}{2.296262in}}%
\pgfpathlineto{\pgfqpoint{0.874241in}{2.255213in}}%
\pgfpathlineto{\pgfqpoint{0.859667in}{2.210961in}}%
\pgfpathlineto{\pgfqpoint{0.846986in}{2.165954in}}%
\pgfpathlineto{\pgfqpoint{0.839633in}{2.134715in}}%
\pgfpathlineto{\pgfqpoint{0.828238in}{2.081532in}}%
\pgfpathlineto{\pgfqpoint{0.817866in}{2.022986in}}%
\pgfpathlineto{\pgfqpoint{0.810784in}{1.971352in}}%
\pgfpathlineto{\pgfqpoint{0.802846in}{1.902252in}}%
\pgfpathlineto{\pgfqpoint{0.796554in}{1.827927in}}%
\pgfpathlineto{\pgfqpoint{0.791696in}{1.743480in}}%
\pgfpathlineto{\pgfqpoint{0.787773in}{1.621595in}}%
\pgfpathlineto{\pgfqpoint{0.785408in}{1.522064in}}%
\pgfpathlineto{\pgfqpoint{0.785408in}{1.522064in}}%
\pgfusepath{stroke}%
\end{pgfscope}%
\begin{pgfscope}%
\pgfpathrectangle{\pgfqpoint{0.448634in}{0.402556in}}{\pgfqpoint{4.350661in}{2.489204in}} %
\pgfusepath{clip}%
\pgfsetrectcap%
\pgfsetroundjoin%
\pgfsetlinewidth{1.003750pt}%
\definecolor{currentstroke}{rgb}{0.172549,0.627451,0.172549}%
\pgfsetstrokecolor{currentstroke}%
\pgfsetdash{}{0pt}%
\pgfpathmoveto{\pgfqpoint{0.448634in}{2.896245in}}%
\pgfpathlineto{\pgfqpoint{0.448593in}{0.407043in}}%
\pgfpathlineto{\pgfqpoint{0.448593in}{0.407043in}}%
\pgfusepath{stroke}%
\end{pgfscope}%
\begin{pgfscope}%
\pgfpathrectangle{\pgfqpoint{0.448634in}{0.402556in}}{\pgfqpoint{4.350661in}{2.489204in}} %
\pgfusepath{clip}%
\pgfsetrectcap%
\pgfsetroundjoin%
\pgfsetlinewidth{1.003750pt}%
\definecolor{currentstroke}{rgb}{0.172549,0.627451,0.172549}%
\pgfsetstrokecolor{currentstroke}%
\pgfsetdash{}{0pt}%
\pgfpathmoveto{\pgfqpoint{0.576853in}{1.760817in}}%
\pgfpathlineto{\pgfqpoint{0.569394in}{1.840010in}}%
\pgfpathlineto{\pgfqpoint{0.563209in}{1.929338in}}%
\pgfpathlineto{\pgfqpoint{0.558592in}{2.028764in}}%
\pgfpathlineto{\pgfqpoint{0.555985in}{2.133265in}}%
\pgfpathlineto{\pgfqpoint{0.555566in}{2.237808in}}%
\pgfpathlineto{\pgfqpoint{0.557371in}{2.337352in}}%
\pgfpathlineto{\pgfqpoint{0.561096in}{2.424366in}}%
\pgfpathlineto{\pgfqpoint{0.566403in}{2.498791in}}%
\pgfpathlineto{\pgfqpoint{0.572909in}{2.560570in}}%
\pgfpathlineto{\pgfqpoint{0.580458in}{2.612119in}}%
\pgfpathlineto{\pgfqpoint{0.589086in}{2.655816in}}%
\pgfpathlineto{\pgfqpoint{0.598406in}{2.691589in}}%
\pgfpathlineto{\pgfqpoint{0.608613in}{2.721757in}}%
\pgfpathlineto{\pgfqpoint{0.619241in}{2.746278in}}%
\pgfpathlineto{\pgfqpoint{0.630817in}{2.767339in}}%
\pgfpathlineto{\pgfqpoint{0.642975in}{2.784884in}}%
\pgfpathlineto{\pgfqpoint{0.656813in}{2.800712in}}%
\pgfpathlineto{\pgfqpoint{0.672197in}{2.814549in}}%
\pgfpathlineto{\pgfqpoint{0.688853in}{2.826301in}}%
\pgfpathlineto{\pgfqpoint{0.706461in}{2.836076in}}%
\pgfpathlineto{\pgfqpoint{0.726804in}{2.844875in}}%
\pgfpathlineto{\pgfqpoint{0.751866in}{2.853203in}}%
\pgfpathlineto{\pgfqpoint{0.781631in}{2.860547in}}%
\pgfpathlineto{\pgfqpoint{0.818168in}{2.867054in}}%
\pgfpathlineto{\pgfqpoint{0.863581in}{2.872685in}}%
\pgfpathlineto{\pgfqpoint{0.922161in}{2.877518in}}%
\pgfpathlineto{\pgfqpoint{1.000391in}{2.881567in}}%
\pgfpathlineto{\pgfqpoint{1.111294in}{2.884881in}}%
\pgfpathlineto{\pgfqpoint{1.274428in}{2.887367in}}%
\pgfpathlineto{\pgfqpoint{1.552865in}{2.889263in}}%
\pgfpathlineto{\pgfqpoint{2.107573in}{2.890457in}}%
\pgfpathlineto{\pgfqpoint{3.343161in}{2.890573in}}%
\pgfpathlineto{\pgfqpoint{4.043615in}{2.888941in}}%
\pgfpathlineto{\pgfqpoint{4.289417in}{2.886404in}}%
\pgfpathlineto{\pgfqpoint{4.413375in}{2.883093in}}%
\pgfpathlineto{\pgfqpoint{4.489424in}{2.878997in}}%
\pgfpathlineto{\pgfqpoint{4.541451in}{2.874081in}}%
\pgfpathlineto{\pgfqpoint{4.578100in}{2.868470in}}%
\pgfpathlineto{\pgfqpoint{4.605818in}{2.862092in}}%
\pgfpathlineto{\pgfqpoint{4.626725in}{2.855245in}}%
\pgfpathlineto{\pgfqpoint{4.644925in}{2.847018in}}%
\pgfpathlineto{\pgfqpoint{4.660241in}{2.837590in}}%
\pgfpathlineto{\pgfqpoint{4.672623in}{2.827468in}}%
\pgfpathlineto{\pgfqpoint{4.683751in}{2.815592in}}%
\pgfpathlineto{\pgfqpoint{4.693406in}{2.802135in}}%
\pgfpathlineto{\pgfqpoint{4.702740in}{2.785343in}}%
\pgfpathlineto{\pgfqpoint{4.711277in}{2.765194in}}%
\pgfpathlineto{\pgfqpoint{4.719482in}{2.739484in}}%
\pgfpathlineto{\pgfqpoint{4.726293in}{2.710657in}}%
\pgfpathlineto{\pgfqpoint{4.733259in}{2.671643in}}%
\pgfpathlineto{\pgfqpoint{4.739604in}{2.622396in}}%
\pgfpathlineto{\pgfqpoint{4.745236in}{2.560504in}}%
\pgfpathlineto{\pgfqpoint{4.750164in}{2.481052in}}%
\pgfpathlineto{\pgfqpoint{4.754367in}{2.376618in}}%
\pgfpathlineto{\pgfqpoint{4.757443in}{2.242249in}}%
\pgfpathlineto{\pgfqpoint{4.758977in}{2.075483in}}%
\pgfpathlineto{\pgfqpoint{4.758447in}{1.888795in}}%
\pgfpathlineto{\pgfqpoint{4.755756in}{1.707111in}}%
\pgfpathlineto{\pgfqpoint{4.750925in}{1.532957in}}%
\pgfpathlineto{\pgfqpoint{4.744785in}{1.398726in}}%
\pgfpathlineto{\pgfqpoint{4.737575in}{1.289516in}}%
\pgfpathlineto{\pgfqpoint{4.728714in}{1.190470in}}%
\pgfpathlineto{\pgfqpoint{4.719652in}{1.116521in}}%
\pgfpathlineto{\pgfqpoint{4.710036in}{1.055276in}}%
\pgfpathlineto{\pgfqpoint{4.699503in}{1.001861in}}%
\pgfpathlineto{\pgfqpoint{4.689040in}{0.958690in}}%
\pgfpathlineto{\pgfqpoint{4.677220in}{0.918600in}}%
\pgfpathlineto{\pgfqpoint{4.664034in}{0.881749in}}%
\pgfpathlineto{\pgfqpoint{4.650584in}{0.850492in}}%
\pgfpathlineto{\pgfqpoint{4.636303in}{0.822570in}}%
\pgfpathlineto{\pgfqpoint{4.620207in}{0.795974in}}%
\pgfpathlineto{\pgfqpoint{4.603640in}{0.772901in}}%
\pgfpathlineto{\pgfqpoint{4.585488in}{0.751446in}}%
\pgfpathlineto{\pgfqpoint{4.565874in}{0.731749in}}%
\pgfpathlineto{\pgfqpoint{4.544964in}{0.713879in}}%
\pgfpathlineto{\pgfqpoint{4.522958in}{0.697824in}}%
\pgfpathlineto{\pgfqpoint{4.496157in}{0.681290in}}%
\pgfpathlineto{\pgfqpoint{4.470397in}{0.667953in}}%
\pgfpathlineto{\pgfqpoint{4.439961in}{0.654509in}}%
\pgfpathlineto{\pgfqpoint{4.406841in}{0.642281in}}%
\pgfpathlineto{\pgfqpoint{4.369009in}{0.630748in}}%
\pgfpathlineto{\pgfqpoint{4.326489in}{0.620226in}}%
\pgfpathlineto{\pgfqpoint{4.279327in}{0.610949in}}%
\pgfpathlineto{\pgfqpoint{4.227576in}{0.603085in}}%
\pgfpathlineto{\pgfqpoint{4.173450in}{0.597063in}}%
\pgfpathlineto{\pgfqpoint{4.110511in}{0.592203in}}%
\pgfpathlineto{\pgfqpoint{4.047471in}{0.589537in}}%
\pgfpathlineto{\pgfqpoint{3.977867in}{0.588624in}}%
\pgfpathlineto{\pgfqpoint{3.906093in}{0.589934in}}%
\pgfpathlineto{\pgfqpoint{3.834377in}{0.593496in}}%
\pgfpathlineto{\pgfqpoint{3.767120in}{0.599067in}}%
\pgfpathlineto{\pgfqpoint{3.704364in}{0.606392in}}%
\pgfpathlineto{\pgfqpoint{3.678516in}{0.610510in}}%
\pgfpathlineto{\pgfqpoint{3.620438in}{0.620500in}}%
\pgfpathlineto{\pgfqpoint{3.586319in}{0.628207in}}%
\pgfpathlineto{\pgfqpoint{3.495240in}{0.652428in}}%
\pgfpathlineto{\pgfqpoint{3.451528in}{0.667583in}}%
\pgfpathlineto{\pgfqpoint{3.408538in}{0.685220in}}%
\pgfpathlineto{\pgfqpoint{3.374594in}{0.702001in}}%
\pgfpathlineto{\pgfqpoint{3.345407in}{0.718682in}}%
\pgfpathlineto{\pgfqpoint{3.315236in}{0.738520in}}%
\pgfpathlineto{\pgfqpoint{3.288127in}{0.759290in}}%
\pgfpathlineto{\pgfqpoint{3.264004in}{0.780551in}}%
\pgfpathlineto{\pgfqpoint{3.241208in}{0.803648in}}%
\pgfpathlineto{\pgfqpoint{3.219894in}{0.828530in}}%
\pgfpathlineto{\pgfqpoint{3.200189in}{0.855091in}}%
\pgfpathlineto{\pgfqpoint{3.182177in}{0.883182in}}%
\pgfpathlineto{\pgfqpoint{3.165906in}{0.912633in}}%
\pgfpathlineto{\pgfqpoint{3.150351in}{0.945448in}}%
\pgfpathlineto{\pgfqpoint{3.136682in}{0.979345in}}%
\pgfpathlineto{\pgfqpoint{3.124073in}{1.016460in}}%
\pgfpathlineto{\pgfqpoint{3.112834in}{1.056769in}}%
\pgfpathlineto{\pgfqpoint{3.103046in}{1.100146in}}%
\pgfpathlineto{\pgfqpoint{3.095343in}{1.144071in}}%
\pgfpathlineto{\pgfqpoint{3.089208in}{1.190837in}}%
\pgfpathlineto{\pgfqpoint{3.084595in}{1.242838in}}%
\pgfpathlineto{\pgfqpoint{3.082137in}{1.295031in}}%
\pgfpathlineto{\pgfqpoint{3.081687in}{1.349787in}}%
\pgfpathlineto{\pgfqpoint{3.083451in}{1.406998in}}%
\pgfpathlineto{\pgfqpoint{3.087181in}{1.461589in}}%
\pgfpathlineto{\pgfqpoint{3.093485in}{1.520888in}}%
\pgfpathlineto{\pgfqpoint{3.101823in}{1.577334in}}%
\pgfpathlineto{\pgfqpoint{3.111930in}{1.630856in}}%
\pgfpathlineto{\pgfqpoint{3.124690in}{1.686208in}}%
\pgfpathlineto{\pgfqpoint{3.139178in}{1.738395in}}%
\pgfpathlineto{\pgfqpoint{3.155145in}{1.787366in}}%
\pgfpathlineto{\pgfqpoint{3.172353in}{1.833085in}}%
\pgfpathlineto{\pgfqpoint{3.191618in}{1.877716in}}%
\pgfpathlineto{\pgfqpoint{3.214026in}{1.923261in}}%
\pgfpathlineto{\pgfqpoint{3.236214in}{1.963157in}}%
\pgfpathlineto{\pgfqpoint{3.260178in}{2.001684in}}%
\pgfpathlineto{\pgfqpoint{3.285814in}{2.038776in}}%
\pgfpathlineto{\pgfqpoint{3.314415in}{2.076285in}}%
\pgfpathlineto{\pgfqpoint{3.348944in}{2.117711in}}%
\pgfpathlineto{\pgfqpoint{3.417133in}{2.198022in}}%
\pgfpathlineto{\pgfqpoint{3.426053in}{2.212128in}}%
\pgfpathlineto{\pgfqpoint{3.430798in}{2.223297in}}%
\pgfpathlineto{\pgfqpoint{3.432034in}{2.230603in}}%
\pgfpathlineto{\pgfqpoint{3.430773in}{2.237856in}}%
\pgfpathlineto{\pgfqpoint{3.426621in}{2.243526in}}%
\pgfpathlineto{\pgfqpoint{3.420908in}{2.247084in}}%
\pgfpathlineto{\pgfqpoint{3.412501in}{2.249583in}}%
\pgfpathlineto{\pgfqpoint{3.399499in}{2.250689in}}%
\pgfpathlineto{\pgfqpoint{3.384305in}{2.249671in}}%
\pgfpathlineto{\pgfqpoint{3.364985in}{2.246098in}}%
\pgfpathlineto{\pgfqpoint{3.341804in}{2.239342in}}%
\pgfpathlineto{\pgfqpoint{3.317109in}{2.229682in}}%
\pgfpathlineto{\pgfqpoint{3.291104in}{2.216986in}}%
\pgfpathlineto{\pgfqpoint{3.265928in}{2.202261in}}%
\pgfpathlineto{\pgfqpoint{3.239805in}{2.184361in}}%
\pgfpathlineto{\pgfqpoint{3.214775in}{2.164519in}}%
\pgfpathlineto{\pgfqpoint{3.190900in}{2.142893in}}%
\pgfpathlineto{\pgfqpoint{3.166657in}{2.117912in}}%
\pgfpathlineto{\pgfqpoint{3.143835in}{2.091233in}}%
\pgfpathlineto{\pgfqpoint{3.121079in}{2.061107in}}%
\pgfpathlineto{\pgfqpoint{3.099952in}{2.029463in}}%
\pgfpathlineto{\pgfqpoint{3.079251in}{1.994406in}}%
\pgfpathlineto{\pgfqpoint{3.059218in}{1.955915in}}%
\pgfpathlineto{\pgfqpoint{3.040058in}{1.914015in}}%
\pgfpathlineto{\pgfqpoint{3.022809in}{1.871041in}}%
\pgfpathlineto{\pgfqpoint{3.005790in}{1.822536in}}%
\pgfpathlineto{\pgfqpoint{2.990067in}{1.770819in}}%
\pgfpathlineto{\pgfqpoint{2.975708in}{1.715979in}}%
\pgfpathlineto{\pgfqpoint{2.962284in}{1.655680in}}%
\pgfpathlineto{\pgfqpoint{2.950496in}{1.592386in}}%
\pgfpathlineto{\pgfqpoint{2.940383in}{1.526185in}}%
\pgfpathlineto{\pgfqpoint{2.931745in}{1.454681in}}%
\pgfpathlineto{\pgfqpoint{2.925082in}{1.380399in}}%
\pgfpathlineto{\pgfqpoint{2.920647in}{1.305899in}}%
\pgfpathlineto{\pgfqpoint{2.918444in}{1.231270in}}%
\pgfpathlineto{\pgfqpoint{2.918545in}{1.159087in}}%
\pgfpathlineto{\pgfqpoint{2.920787in}{1.091931in}}%
\pgfpathlineto{\pgfqpoint{2.925177in}{1.027412in}}%
\pgfpathlineto{\pgfqpoint{2.931192in}{0.970580in}}%
\pgfpathlineto{\pgfqpoint{2.938760in}{0.919034in}}%
\pgfpathlineto{\pgfqpoint{2.947651in}{0.872852in}}%
\pgfpathlineto{\pgfqpoint{2.958213in}{0.829714in}}%
\pgfpathlineto{\pgfqpoint{2.969670in}{0.792114in}}%
\pgfpathlineto{\pgfqpoint{2.982463in}{0.757773in}}%
\pgfpathlineto{\pgfqpoint{2.996425in}{0.726812in}}%
\pgfpathlineto{\pgfqpoint{3.011299in}{0.699300in}}%
\pgfpathlineto{\pgfqpoint{3.026739in}{0.675225in}}%
\pgfpathlineto{\pgfqpoint{3.043828in}{0.652656in}}%
\pgfpathlineto{\pgfqpoint{3.062495in}{0.631788in}}%
\pgfpathlineto{\pgfqpoint{3.082602in}{0.612753in}}%
\pgfpathlineto{\pgfqpoint{3.103961in}{0.595592in}}%
\pgfpathlineto{\pgfqpoint{3.128268in}{0.579069in}}%
\pgfpathlineto{\pgfqpoint{3.153537in}{0.564554in}}%
\pgfpathlineto{\pgfqpoint{3.181571in}{0.550952in}}%
\pgfpathlineto{\pgfqpoint{3.214371in}{0.537647in}}%
\pgfpathlineto{\pgfqpoint{3.249846in}{0.525712in}}%
\pgfpathlineto{\pgfqpoint{3.290011in}{0.514571in}}%
\pgfpathlineto{\pgfqpoint{3.334820in}{0.504423in}}%
\pgfpathlineto{\pgfqpoint{3.386372in}{0.494999in}}%
\pgfpathlineto{\pgfqpoint{3.446798in}{0.486257in}}%
\pgfpathlineto{\pgfqpoint{3.518243in}{0.478282in}}%
\pgfpathlineto{\pgfqpoint{3.600685in}{0.471409in}}%
\pgfpathlineto{\pgfqpoint{3.696268in}{0.465713in}}%
\pgfpathlineto{\pgfqpoint{3.807144in}{0.461369in}}%
\pgfpathlineto{\pgfqpoint{3.933291in}{0.458719in}}%
\pgfpathlineto{\pgfqpoint{4.063808in}{0.458211in}}%
\pgfpathlineto{\pgfqpoint{4.187792in}{0.459914in}}%
\pgfpathlineto{\pgfqpoint{4.294335in}{0.463521in}}%
\pgfpathlineto{\pgfqpoint{4.381234in}{0.468574in}}%
\pgfpathlineto{\pgfqpoint{4.450636in}{0.474701in}}%
\pgfpathlineto{\pgfqpoint{4.506850in}{0.481799in}}%
\pgfpathlineto{\pgfqpoint{4.552009in}{0.489658in}}%
\pgfpathlineto{\pgfqpoint{4.588239in}{0.498115in}}%
\pgfpathlineto{\pgfqpoint{4.617656in}{0.507110in}}%
\pgfpathlineto{\pgfqpoint{4.642328in}{0.516843in}}%
\pgfpathlineto{\pgfqpoint{4.664194in}{0.527940in}}%
\pgfpathlineto{\pgfqpoint{4.681238in}{0.538945in}}%
\pgfpathlineto{\pgfqpoint{4.697164in}{0.551953in}}%
\pgfpathlineto{\pgfqpoint{4.710076in}{0.565289in}}%
\pgfpathlineto{\pgfqpoint{4.721578in}{0.580218in}}%
\pgfpathlineto{\pgfqpoint{4.731557in}{0.596521in}}%
\pgfpathlineto{\pgfqpoint{4.741000in}{0.616134in}}%
\pgfpathlineto{\pgfqpoint{4.749521in}{0.639027in}}%
\pgfpathlineto{\pgfqpoint{4.757522in}{0.667450in}}%
\pgfpathlineto{\pgfqpoint{4.764572in}{0.701345in}}%
\pgfpathlineto{\pgfqpoint{4.770840in}{0.743043in}}%
\pgfpathlineto{\pgfqpoint{4.776327in}{0.794934in}}%
\pgfpathlineto{\pgfqpoint{4.781278in}{0.864398in}}%
\pgfpathlineto{\pgfqpoint{4.785468in}{0.956371in}}%
\pgfpathlineto{\pgfqpoint{4.789000in}{1.085745in}}%
\pgfpathlineto{\pgfqpoint{4.791852in}{1.277385in}}%
\pgfpathlineto{\pgfqpoint{4.793959in}{1.581057in}}%
\pgfpathlineto{\pgfqpoint{4.794962in}{2.071429in}}%
\pgfpathlineto{\pgfqpoint{4.793967in}{2.559311in}}%
\pgfpathlineto{\pgfqpoint{4.791733in}{2.745981in}}%
\pgfpathlineto{\pgfqpoint{4.788955in}{2.818091in}}%
\pgfpathlineto{\pgfqpoint{4.785731in}{2.850227in}}%
\pgfpathlineto{\pgfqpoint{4.781879in}{2.867057in}}%
\pgfpathlineto{\pgfqpoint{4.777744in}{2.875780in}}%
\pgfpathlineto{\pgfqpoint{4.773097in}{2.880982in}}%
\pgfpathlineto{\pgfqpoint{4.767363in}{2.884504in}}%
\pgfpathlineto{\pgfqpoint{4.756853in}{2.887622in}}%
\pgfpathlineto{\pgfqpoint{4.739548in}{2.889639in}}%
\pgfpathlineto{\pgfqpoint{4.704762in}{2.890882in}}%
\pgfpathlineto{\pgfqpoint{4.602524in}{2.891538in}}%
\pgfpathlineto{\pgfqpoint{3.952100in}{2.891742in}}%
\pgfpathlineto{\pgfqpoint{0.617321in}{2.890753in}}%
\pgfpathlineto{\pgfqpoint{0.549910in}{2.888858in}}%
\pgfpathlineto{\pgfqpoint{0.521735in}{2.886179in}}%
\pgfpathlineto{\pgfqpoint{0.504666in}{2.882389in}}%
\pgfpathlineto{\pgfqpoint{0.494501in}{2.878011in}}%
\pgfpathlineto{\pgfqpoint{0.487180in}{2.872667in}}%
\pgfpathlineto{\pgfqpoint{0.481152in}{2.865519in}}%
\pgfpathlineto{\pgfqpoint{0.475664in}{2.854804in}}%
\pgfpathlineto{\pgfqpoint{0.471318in}{2.840737in}}%
\pgfpathlineto{\pgfqpoint{0.467301in}{2.818823in}}%
\pgfpathlineto{\pgfqpoint{0.463927in}{2.786700in}}%
\pgfpathlineto{\pgfqpoint{0.460918in}{2.734544in}}%
\pgfpathlineto{\pgfqpoint{0.458363in}{2.647474in}}%
\pgfpathlineto{\pgfqpoint{0.456575in}{2.523031in}}%
\pgfpathlineto{\pgfqpoint{0.456575in}{2.523031in}}%
\pgfusepath{stroke}%
\end{pgfscope}%
\begin{pgfscope}%
\pgfpathrectangle{\pgfqpoint{0.448634in}{0.402556in}}{\pgfqpoint{4.350661in}{2.489204in}} %
\pgfusepath{clip}%
\pgfsetrectcap%
\pgfsetroundjoin%
\pgfsetlinewidth{1.003750pt}%
\definecolor{currentstroke}{rgb}{0.172549,0.627451,0.172549}%
\pgfsetstrokecolor{currentstroke}%
\pgfsetdash{}{0pt}%
\pgfpathmoveto{\pgfqpoint{4.798840in}{2.852369in}}%
\pgfpathlineto{\pgfqpoint{4.797564in}{2.889610in}}%
\pgfpathlineto{\pgfqpoint{4.796215in}{2.891483in}}%
\pgfpathlineto{\pgfqpoint{4.787551in}{2.891760in}}%
\pgfpathlineto{\pgfqpoint{0.452128in}{2.891659in}}%
\pgfpathlineto{\pgfqpoint{0.450530in}{2.890082in}}%
\pgfpathlineto{\pgfqpoint{0.449454in}{2.882763in}}%
\pgfpathlineto{\pgfqpoint{0.448970in}{2.845432in}}%
\pgfpathlineto{\pgfqpoint{0.448743in}{2.494454in}}%
\pgfpathlineto{\pgfqpoint{0.449624in}{0.615107in}}%
\pgfpathlineto{\pgfqpoint{0.451433in}{0.510586in}}%
\pgfpathlineto{\pgfqpoint{0.453993in}{0.473374in}}%
\pgfpathlineto{\pgfqpoint{0.457406in}{0.453868in}}%
\pgfpathlineto{\pgfqpoint{0.461540in}{0.442384in}}%
\pgfpathlineto{\pgfqpoint{0.466739in}{0.434437in}}%
\pgfpathlineto{\pgfqpoint{0.473595in}{0.428350in}}%
\pgfpathlineto{\pgfqpoint{0.483492in}{0.423244in}}%
\pgfpathlineto{\pgfqpoint{0.491854in}{0.420501in}}%
\pgfpathlineto{\pgfqpoint{0.491854in}{0.420501in}}%
\pgfusepath{stroke}%
\end{pgfscope}%
\begin{pgfscope}%
\pgfpathrectangle{\pgfqpoint{0.448634in}{0.402556in}}{\pgfqpoint{4.350661in}{2.489204in}} %
\pgfusepath{clip}%
\pgfsetrectcap%
\pgfsetroundjoin%
\pgfsetlinewidth{1.003750pt}%
\definecolor{currentstroke}{rgb}{0.172549,0.627451,0.172549}%
\pgfsetstrokecolor{currentstroke}%
\pgfsetdash{}{0pt}%
\pgfpathmoveto{\pgfqpoint{0.456424in}{1.370137in}}%
\pgfpathlineto{\pgfqpoint{0.459610in}{1.118755in}}%
\pgfpathlineto{\pgfqpoint{0.463695in}{0.962007in}}%
\pgfpathlineto{\pgfqpoint{0.468519in}{0.857610in}}%
\pgfpathlineto{\pgfqpoint{0.474082in}{0.783210in}}%
\pgfpathlineto{\pgfqpoint{0.480226in}{0.728906in}}%
\pgfpathlineto{\pgfqpoint{0.486970in}{0.687306in}}%
\pgfpathlineto{\pgfqpoint{0.494537in}{0.653559in}}%
\pgfpathlineto{\pgfqpoint{0.503107in}{0.625355in}}%
\pgfpathlineto{\pgfqpoint{0.512193in}{0.602750in}}%
\pgfpathlineto{\pgfqpoint{0.522200in}{0.583508in}}%
\pgfpathlineto{\pgfqpoint{0.534108in}{0.565743in}}%
\pgfpathlineto{\pgfqpoint{0.546263in}{0.551507in}}%
\pgfpathlineto{\pgfqpoint{0.559728in}{0.538907in}}%
\pgfpathlineto{\pgfqpoint{0.576129in}{0.526693in}}%
\pgfpathlineto{\pgfqpoint{0.595483in}{0.515351in}}%
\pgfpathlineto{\pgfqpoint{0.617681in}{0.505147in}}%
\pgfpathlineto{\pgfqpoint{0.642568in}{0.496153in}}%
\pgfpathlineto{\pgfqpoint{0.672125in}{0.487778in}}%
\pgfpathlineto{\pgfqpoint{0.708443in}{0.479824in}}%
\pgfpathlineto{\pgfqpoint{0.753649in}{0.472325in}}%
\pgfpathlineto{\pgfqpoint{0.807717in}{0.465660in}}%
\pgfpathlineto{\pgfqpoint{0.877116in}{0.459475in}}%
\pgfpathlineto{\pgfqpoint{0.961828in}{0.454230in}}%
\pgfpathlineto{\pgfqpoint{1.068351in}{0.449916in}}%
\pgfpathlineto{\pgfqpoint{1.201018in}{0.446839in}}%
\pgfpathlineto{\pgfqpoint{1.357637in}{0.445481in}}%
\pgfpathlineto{\pgfqpoint{1.525135in}{0.446232in}}%
\pgfpathlineto{\pgfqpoint{1.686088in}{0.449142in}}%
\pgfpathlineto{\pgfqpoint{1.823074in}{0.453747in}}%
\pgfpathlineto{\pgfqpoint{1.938245in}{0.459764in}}%
\pgfpathlineto{\pgfqpoint{2.031582in}{0.466759in}}%
\pgfpathlineto{\pgfqpoint{2.109580in}{0.474745in}}%
\pgfpathlineto{\pgfqpoint{2.174384in}{0.483535in}}%
\pgfpathlineto{\pgfqpoint{2.228139in}{0.492940in}}%
\pgfpathlineto{\pgfqpoint{2.275119in}{0.503356in}}%
\pgfpathlineto{\pgfqpoint{2.315282in}{0.514501in}}%
\pgfpathlineto{\pgfqpoint{2.350698in}{0.526659in}}%
\pgfpathlineto{\pgfqpoint{2.381320in}{0.539536in}}%
\pgfpathlineto{\pgfqpoint{2.407164in}{0.552659in}}%
\pgfpathlineto{\pgfqpoint{2.430226in}{0.566639in}}%
\pgfpathlineto{\pgfqpoint{2.452282in}{0.582602in}}%
\pgfpathlineto{\pgfqpoint{2.471391in}{0.599069in}}%
\pgfpathlineto{\pgfqpoint{2.489240in}{0.617293in}}%
\pgfpathlineto{\pgfqpoint{2.505678in}{0.637180in}}%
\pgfpathlineto{\pgfqpoint{2.520620in}{0.658557in}}%
\pgfpathlineto{\pgfqpoint{2.535213in}{0.683314in}}%
\pgfpathlineto{\pgfqpoint{2.549115in}{0.711484in}}%
\pgfpathlineto{\pgfqpoint{2.562091in}{0.743004in}}%
\pgfpathlineto{\pgfqpoint{2.574020in}{0.777751in}}%
\pgfpathlineto{\pgfqpoint{2.585502in}{0.817970in}}%
\pgfpathlineto{\pgfqpoint{2.596809in}{0.866038in}}%
\pgfpathlineto{\pgfqpoint{2.607562in}{0.921948in}}%
\pgfpathlineto{\pgfqpoint{2.617925in}{0.988098in}}%
\pgfpathlineto{\pgfqpoint{2.627958in}{1.066918in}}%
\pgfpathlineto{\pgfqpoint{2.637941in}{1.163320in}}%
\pgfpathlineto{\pgfqpoint{2.648424in}{1.287199in}}%
\pgfpathlineto{\pgfqpoint{2.660103in}{1.453438in}}%
\pgfpathlineto{\pgfqpoint{2.674773in}{1.696801in}}%
\pgfpathlineto{\pgfqpoint{2.687716in}{1.945279in}}%
\pgfpathlineto{\pgfqpoint{2.692670in}{2.079573in}}%
\pgfpathlineto{\pgfqpoint{2.693829in}{2.166682in}}%
\pgfpathlineto{\pgfqpoint{2.692565in}{2.233870in}}%
\pgfpathlineto{\pgfqpoint{2.689436in}{2.286014in}}%
\pgfpathlineto{\pgfqpoint{2.684859in}{2.327999in}}%
\pgfpathlineto{\pgfqpoint{2.678725in}{2.364664in}}%
\pgfpathlineto{\pgfqpoint{2.671356in}{2.395897in}}%
\pgfpathlineto{\pgfqpoint{2.662489in}{2.423981in}}%
\pgfpathlineto{\pgfqpoint{2.652361in}{2.448778in}}%
\pgfpathlineto{\pgfqpoint{2.641365in}{2.470245in}}%
\pgfpathlineto{\pgfqpoint{2.628643in}{2.490425in}}%
\pgfpathlineto{\pgfqpoint{2.614279in}{2.509105in}}%
\pgfpathlineto{\pgfqpoint{2.598443in}{2.526159in}}%
\pgfpathlineto{\pgfqpoint{2.579590in}{2.543005in}}%
\pgfpathlineto{\pgfqpoint{2.559532in}{2.557923in}}%
\pgfpathlineto{\pgfqpoint{2.536602in}{2.572183in}}%
\pgfpathlineto{\pgfqpoint{2.510850in}{2.585538in}}%
\pgfpathlineto{\pgfqpoint{2.482360in}{2.597837in}}%
\pgfpathlineto{\pgfqpoint{2.449134in}{2.609683in}}%
\pgfpathlineto{\pgfqpoint{2.411184in}{2.620696in}}%
\pgfpathlineto{\pgfqpoint{2.368552in}{2.630606in}}%
\pgfpathlineto{\pgfqpoint{2.321294in}{2.639221in}}%
\pgfpathlineto{\pgfqpoint{2.269467in}{2.646399in}}%
\pgfpathlineto{\pgfqpoint{2.210954in}{2.652193in}}%
\pgfpathlineto{\pgfqpoint{2.147967in}{2.656153in}}%
\pgfpathlineto{\pgfqpoint{2.080556in}{2.658135in}}%
\pgfpathlineto{\pgfqpoint{2.010948in}{2.657971in}}%
\pgfpathlineto{\pgfqpoint{1.939195in}{2.655572in}}%
\pgfpathlineto{\pgfqpoint{1.867527in}{2.650913in}}%
\pgfpathlineto{\pgfqpoint{1.798171in}{2.644140in}}%
\pgfpathlineto{\pgfqpoint{1.733341in}{2.635606in}}%
\pgfpathlineto{\pgfqpoint{1.673075in}{2.625521in}}%
\pgfpathlineto{\pgfqpoint{1.615274in}{2.613610in}}%
\pgfpathlineto{\pgfqpoint{1.562133in}{2.600402in}}%
\pgfpathlineto{\pgfqpoint{1.513681in}{2.586139in}}%
\pgfpathlineto{\pgfqpoint{1.467862in}{2.570344in}}%
\pgfpathlineto{\pgfqpoint{1.426794in}{2.553923in}}%
\pgfpathlineto{\pgfqpoint{1.388447in}{2.536289in}}%
\pgfpathlineto{\pgfqpoint{1.352878in}{2.517566in}}%
\pgfpathlineto{\pgfqpoint{1.320128in}{2.497922in}}%
\pgfpathlineto{\pgfqpoint{1.288379in}{2.476236in}}%
\pgfpathlineto{\pgfqpoint{1.259592in}{2.453861in}}%
\pgfpathlineto{\pgfqpoint{1.232050in}{2.429520in}}%
\pgfpathlineto{\pgfqpoint{1.207527in}{2.404898in}}%
\pgfpathlineto{\pgfqpoint{1.184409in}{2.378557in}}%
\pgfpathlineto{\pgfqpoint{1.162828in}{2.350561in}}%
\pgfpathlineto{\pgfqpoint{1.142891in}{2.321012in}}%
\pgfpathlineto{\pgfqpoint{1.124675in}{2.290041in}}%
\pgfpathlineto{\pgfqpoint{1.108225in}{2.257802in}}%
\pgfpathlineto{\pgfqpoint{1.092639in}{2.222199in}}%
\pgfpathlineto{\pgfqpoint{1.079059in}{2.185535in}}%
\pgfpathlineto{\pgfqpoint{1.067443in}{2.147998in}}%
\pgfpathlineto{\pgfqpoint{1.057187in}{2.107348in}}%
\pgfpathlineto{\pgfqpoint{1.049004in}{2.066086in}}%
\pgfpathlineto{\pgfqpoint{1.042513in}{2.021906in}}%
\pgfpathlineto{\pgfqpoint{1.038177in}{1.977382in}}%
\pgfpathlineto{\pgfqpoint{1.035866in}{1.930167in}}%
\pgfpathlineto{\pgfqpoint{1.035826in}{1.882878in}}%
\pgfpathlineto{\pgfqpoint{1.038031in}{1.835656in}}%
\pgfpathlineto{\pgfqpoint{1.042474in}{1.788641in}}%
\pgfpathlineto{\pgfqpoint{1.049176in}{1.741979in}}%
\pgfpathlineto{\pgfqpoint{1.057644in}{1.698239in}}%
\pgfpathlineto{\pgfqpoint{1.068221in}{1.655105in}}%
\pgfpathlineto{\pgfqpoint{1.080962in}{1.612745in}}%
\pgfpathlineto{\pgfqpoint{1.095031in}{1.573617in}}%
\pgfpathlineto{\pgfqpoint{1.111115in}{1.535520in}}%
\pgfpathlineto{\pgfqpoint{1.128117in}{1.500775in}}%
\pgfpathlineto{\pgfqpoint{1.146930in}{1.467274in}}%
\pgfpathlineto{\pgfqpoint{1.167531in}{1.435181in}}%
\pgfpathlineto{\pgfqpoint{1.189874in}{1.404652in}}%
\pgfpathlineto{\pgfqpoint{1.213884in}{1.375828in}}%
\pgfpathlineto{\pgfqpoint{1.237817in}{1.350457in}}%
\pgfpathlineto{\pgfqpoint{1.264748in}{1.325237in}}%
\pgfpathlineto{\pgfqpoint{1.292991in}{1.301972in}}%
\pgfpathlineto{\pgfqpoint{1.322398in}{1.280678in}}%
\pgfpathlineto{\pgfqpoint{1.352820in}{1.261340in}}%
\pgfpathlineto{\pgfqpoint{1.386095in}{1.242889in}}%
\pgfpathlineto{\pgfqpoint{1.420190in}{1.226516in}}%
\pgfpathlineto{\pgfqpoint{1.457024in}{1.211329in}}%
\pgfpathlineto{\pgfqpoint{1.496554in}{1.197536in}}%
\pgfpathlineto{\pgfqpoint{1.538719in}{1.185287in}}%
\pgfpathlineto{\pgfqpoint{1.583441in}{1.174641in}}%
\pgfpathlineto{\pgfqpoint{1.634929in}{1.164775in}}%
\pgfpathlineto{\pgfqpoint{1.706063in}{1.153745in}}%
\pgfpathlineto{\pgfqpoint{1.768492in}{1.143417in}}%
\pgfpathlineto{\pgfqpoint{1.796122in}{1.136567in}}%
\pgfpathlineto{\pgfqpoint{1.812683in}{1.130481in}}%
\pgfpathlineto{\pgfqpoint{1.824471in}{1.124102in}}%
\pgfpathlineto{\pgfqpoint{1.833209in}{1.116741in}}%
\pgfpathlineto{\pgfqpoint{1.838498in}{1.108890in}}%
\pgfpathlineto{\pgfqpoint{1.840588in}{1.101849in}}%
\pgfpathlineto{\pgfqpoint{1.840619in}{1.094412in}}%
\pgfpathlineto{\pgfqpoint{1.837931in}{1.084986in}}%
\pgfpathlineto{\pgfqpoint{1.833246in}{1.076615in}}%
\pgfpathlineto{\pgfqpoint{1.825819in}{1.067542in}}%
\pgfpathlineto{\pgfqpoint{1.813813in}{1.056850in}}%
\pgfpathlineto{\pgfqpoint{1.798819in}{1.046763in}}%
\pgfpathlineto{\pgfqpoint{1.781016in}{1.037462in}}%
\pgfpathlineto{\pgfqpoint{1.758447in}{1.028391in}}%
\pgfpathlineto{\pgfqpoint{1.733203in}{1.020815in}}%
\pgfpathlineto{\pgfqpoint{1.705410in}{1.014872in}}%
\pgfpathlineto{\pgfqpoint{1.675178in}{1.010714in}}%
\pgfpathlineto{\pgfqpoint{1.642610in}{1.008507in}}%
\pgfpathlineto{\pgfqpoint{1.607809in}{1.008432in}}%
\pgfpathlineto{\pgfqpoint{1.570886in}{1.010691in}}%
\pgfpathlineto{\pgfqpoint{1.534118in}{1.015181in}}%
\pgfpathlineto{\pgfqpoint{1.495454in}{1.022233in}}%
\pgfpathlineto{\pgfqpoint{1.457161in}{1.031563in}}%
\pgfpathlineto{\pgfqpoint{1.419337in}{1.043132in}}%
\pgfpathlineto{\pgfqpoint{1.382089in}{1.056929in}}%
\pgfpathlineto{\pgfqpoint{1.347544in}{1.072019in}}%
\pgfpathlineto{\pgfqpoint{1.313727in}{1.089133in}}%
\pgfpathlineto{\pgfqpoint{1.280762in}{1.108299in}}%
\pgfpathlineto{\pgfqpoint{1.248782in}{1.129536in}}%
\pgfpathlineto{\pgfqpoint{1.219708in}{1.151422in}}%
\pgfpathlineto{\pgfqpoint{1.191752in}{1.175138in}}%
\pgfpathlineto{\pgfqpoint{1.165031in}{1.200649in}}%
\pgfpathlineto{\pgfqpoint{1.139653in}{1.227898in}}%
\pgfpathlineto{\pgfqpoint{1.115714in}{1.256800in}}%
\pgfpathlineto{\pgfqpoint{1.093288in}{1.287251in}}%
\pgfpathlineto{\pgfqpoint{1.071178in}{1.321163in}}%
\pgfpathlineto{\pgfqpoint{1.050868in}{1.356520in}}%
\pgfpathlineto{\pgfqpoint{1.032365in}{1.393152in}}%
\pgfpathlineto{\pgfqpoint{1.014718in}{1.433142in}}%
\pgfpathlineto{\pgfqpoint{0.999024in}{1.474185in}}%
\pgfpathlineto{\pgfqpoint{0.984506in}{1.518461in}}%
\pgfpathlineto{\pgfqpoint{0.972009in}{1.563537in}}%
\pgfpathlineto{\pgfqpoint{0.960944in}{1.611678in}}%
\pgfpathlineto{\pgfqpoint{0.951530in}{1.662824in}}%
\pgfpathlineto{\pgfqpoint{0.944286in}{1.714431in}}%
\pgfpathlineto{\pgfqpoint{0.938950in}{1.768847in}}%
\pgfpathlineto{\pgfqpoint{0.935870in}{1.823491in}}%
\pgfpathlineto{\pgfqpoint{0.935034in}{1.878240in}}%
\pgfpathlineto{\pgfqpoint{0.936466in}{1.932973in}}%
\pgfpathlineto{\pgfqpoint{0.940005in}{1.985084in}}%
\pgfpathlineto{\pgfqpoint{0.945759in}{2.036935in}}%
\pgfpathlineto{\pgfqpoint{0.953410in}{2.085938in}}%
\pgfpathlineto{\pgfqpoint{0.962764in}{2.132000in}}%
\pgfpathlineto{\pgfqpoint{0.974287in}{2.177414in}}%
\pgfpathlineto{\pgfqpoint{0.987332in}{2.219653in}}%
\pgfpathlineto{\pgfqpoint{1.001667in}{2.258654in}}%
\pgfpathlineto{\pgfqpoint{1.018051in}{2.296583in}}%
\pgfpathlineto{\pgfqpoint{1.035401in}{2.331101in}}%
\pgfpathlineto{\pgfqpoint{1.054650in}{2.364275in}}%
\pgfpathlineto{\pgfqpoint{1.074406in}{2.393984in}}%
\pgfpathlineto{\pgfqpoint{1.095771in}{2.422197in}}%
\pgfpathlineto{\pgfqpoint{1.118662in}{2.448797in}}%
\pgfpathlineto{\pgfqpoint{1.142967in}{2.473701in}}%
\pgfpathlineto{\pgfqpoint{1.168550in}{2.496867in}}%
\pgfpathlineto{\pgfqpoint{1.197085in}{2.519662in}}%
\pgfpathlineto{\pgfqpoint{1.226727in}{2.540526in}}%
\pgfpathlineto{\pgfqpoint{1.259242in}{2.560673in}}%
\pgfpathlineto{\pgfqpoint{1.294612in}{2.579881in}}%
\pgfpathlineto{\pgfqpoint{1.332792in}{2.597982in}}%
\pgfpathlineto{\pgfqpoint{1.373719in}{2.614859in}}%
\pgfpathlineto{\pgfqpoint{1.417319in}{2.630445in}}%
\pgfpathlineto{\pgfqpoint{1.465632in}{2.645312in}}%
\pgfpathlineto{\pgfqpoint{1.518640in}{2.659204in}}%
\pgfpathlineto{\pgfqpoint{1.576309in}{2.671929in}}%
\pgfpathlineto{\pgfqpoint{1.638597in}{2.683344in}}%
\pgfpathlineto{\pgfqpoint{1.705462in}{2.693343in}}%
\pgfpathlineto{\pgfqpoint{1.779027in}{2.702064in}}%
\pgfpathlineto{\pgfqpoint{1.857097in}{2.709077in}}%
\pgfpathlineto{\pgfqpoint{1.939633in}{2.714280in}}%
\pgfpathlineto{\pgfqpoint{2.026598in}{2.717513in}}%
\pgfpathlineto{\pgfqpoint{2.113605in}{2.718523in}}%
\pgfpathlineto{\pgfqpoint{2.198435in}{2.717303in}}%
\pgfpathlineto{\pgfqpoint{2.278866in}{2.713929in}}%
\pgfpathlineto{\pgfqpoint{2.352678in}{2.708598in}}%
\pgfpathlineto{\pgfqpoint{2.417657in}{2.701709in}}%
\pgfpathlineto{\pgfqpoint{2.473770in}{2.693630in}}%
\pgfpathlineto{\pgfqpoint{2.523140in}{2.684368in}}%
\pgfpathlineto{\pgfqpoint{2.565726in}{2.674202in}}%
\pgfpathlineto{\pgfqpoint{2.601510in}{2.663544in}}%
\pgfpathlineto{\pgfqpoint{2.632577in}{2.652142in}}%
\pgfpathlineto{\pgfqpoint{2.658899in}{2.640331in}}%
\pgfpathlineto{\pgfqpoint{2.682438in}{2.627436in}}%
\pgfpathlineto{\pgfqpoint{2.703062in}{2.613571in}}%
\pgfpathlineto{\pgfqpoint{2.720674in}{2.598978in}}%
\pgfpathlineto{\pgfqpoint{2.735263in}{2.584053in}}%
\pgfpathlineto{\pgfqpoint{2.748320in}{2.567377in}}%
\pgfpathlineto{\pgfqpoint{2.759553in}{2.549046in}}%
\pgfpathlineto{\pgfqpoint{2.768788in}{2.529306in}}%
\pgfpathlineto{\pgfqpoint{2.776017in}{2.508498in}}%
\pgfpathlineto{\pgfqpoint{2.781884in}{2.484540in}}%
\pgfpathlineto{\pgfqpoint{2.786102in}{2.457597in}}%
\pgfpathlineto{\pgfqpoint{2.788720in}{2.425384in}}%
\pgfpathlineto{\pgfqpoint{2.789427in}{2.388061in}}%
\pgfpathlineto{\pgfqpoint{2.787962in}{2.340801in}}%
\pgfpathlineto{\pgfqpoint{2.783672in}{2.278768in}}%
\pgfpathlineto{\pgfqpoint{2.774289in}{2.179783in}}%
\pgfpathlineto{\pgfqpoint{2.743611in}{1.868119in}}%
\pgfpathlineto{\pgfqpoint{2.730112in}{1.702060in}}%
\pgfpathlineto{\pgfqpoint{2.717287in}{1.515949in}}%
\pgfpathlineto{\pgfqpoint{2.702602in}{1.267597in}}%
\pgfpathlineto{\pgfqpoint{2.684434in}{0.964630in}}%
\pgfpathlineto{\pgfqpoint{2.675374in}{0.850600in}}%
\pgfpathlineto{\pgfqpoint{2.667030in}{0.771523in}}%
\pgfpathlineto{\pgfqpoint{2.658752in}{0.712543in}}%
\pgfpathlineto{\pgfqpoint{2.650176in}{0.666284in}}%
\pgfpathlineto{\pgfqpoint{2.640820in}{0.627931in}}%
\pgfpathlineto{\pgfqpoint{2.631145in}{0.597534in}}%
\pgfpathlineto{\pgfqpoint{2.621004in}{0.572745in}}%
\pgfpathlineto{\pgfqpoint{2.609856in}{0.551383in}}%
\pgfpathlineto{\pgfqpoint{2.598042in}{0.533534in}}%
\pgfpathlineto{\pgfqpoint{2.584496in}{0.517378in}}%
\pgfpathlineto{\pgfqpoint{2.571109in}{0.504669in}}%
\pgfpathlineto{\pgfqpoint{2.554789in}{0.492313in}}%
\pgfpathlineto{\pgfqpoint{2.537457in}{0.481914in}}%
\pgfpathlineto{\pgfqpoint{2.517374in}{0.472367in}}%
\pgfpathlineto{\pgfqpoint{2.492542in}{0.463178in}}%
\pgfpathlineto{\pgfqpoint{2.462979in}{0.454833in}}%
\pgfpathlineto{\pgfqpoint{2.428766in}{0.447542in}}%
\pgfpathlineto{\pgfqpoint{2.385671in}{0.440735in}}%
\pgfpathlineto{\pgfqpoint{2.331557in}{0.434581in}}%
\pgfpathlineto{\pgfqpoint{2.262115in}{0.429077in}}%
\pgfpathlineto{\pgfqpoint{2.170851in}{0.424236in}}%
\pgfpathlineto{\pgfqpoint{2.049086in}{0.420134in}}%
\pgfpathlineto{\pgfqpoint{1.879436in}{0.416783in}}%
\pgfpathlineto{\pgfqpoint{1.640159in}{0.414418in}}%
\pgfpathlineto{\pgfqpoint{1.322562in}{0.413569in}}%
\pgfpathlineto{\pgfqpoint{1.020194in}{0.414850in}}%
\pgfpathlineto{\pgfqpoint{0.822256in}{0.417715in}}%
\pgfpathlineto{\pgfqpoint{0.704835in}{0.421430in}}%
\pgfpathlineto{\pgfqpoint{0.630977in}{0.425829in}}%
\pgfpathlineto{\pgfqpoint{0.583316in}{0.430734in}}%
\pgfpathlineto{\pgfqpoint{0.551033in}{0.436123in}}%
\pgfpathlineto{\pgfqpoint{0.527708in}{0.442189in}}%
\pgfpathlineto{\pgfqpoint{0.511250in}{0.448625in}}%
\pgfpathlineto{\pgfqpoint{0.499549in}{0.455216in}}%
\pgfpathlineto{\pgfqpoint{0.488916in}{0.463841in}}%
\pgfpathlineto{\pgfqpoint{0.481322in}{0.472730in}}%
\pgfpathlineto{\pgfqpoint{0.474078in}{0.485127in}}%
\pgfpathlineto{\pgfqpoint{0.468753in}{0.498748in}}%
\pgfpathlineto{\pgfqpoint{0.463870in}{0.517848in}}%
\pgfpathlineto{\pgfqpoint{0.459679in}{0.544796in}}%
\pgfpathlineto{\pgfqpoint{0.456386in}{0.581938in}}%
\pgfpathlineto{\pgfqpoint{0.453731in}{0.639106in}}%
\pgfpathlineto{\pgfqpoint{0.451681in}{0.736155in}}%
\pgfpathlineto{\pgfqpoint{0.450220in}{0.927815in}}%
\pgfpathlineto{\pgfqpoint{0.449345in}{1.403252in}}%
\pgfpathlineto{\pgfqpoint{0.449543in}{2.682703in}}%
\pgfpathlineto{\pgfqpoint{0.451011in}{2.856932in}}%
\pgfpathlineto{\pgfqpoint{0.452802in}{2.879219in}}%
\pgfpathlineto{\pgfqpoint{0.455188in}{2.886107in}}%
\pgfpathlineto{\pgfqpoint{0.458626in}{2.889028in}}%
\pgfpathlineto{\pgfqpoint{0.464996in}{2.890553in}}%
\pgfpathlineto{\pgfqpoint{0.482376in}{2.891423in}}%
\pgfpathlineto{\pgfqpoint{0.565038in}{2.891729in}}%
\pgfpathlineto{\pgfqpoint{2.733842in}{2.891760in}}%
\pgfpathlineto{\pgfqpoint{4.789510in}{2.890885in}}%
\pgfpathlineto{\pgfqpoint{4.793727in}{2.889730in}}%
\pgfpathlineto{\pgfqpoint{4.795481in}{2.888307in}}%
\pgfpathlineto{\pgfqpoint{4.797106in}{2.881145in}}%
\pgfpathlineto{\pgfqpoint{4.797997in}{2.858771in}}%
\pgfpathlineto{\pgfqpoint{4.798039in}{2.856283in}}%
\pgfpathlineto{\pgfqpoint{4.798039in}{2.856283in}}%
\pgfusepath{stroke}%
\end{pgfscope}%
\begin{pgfscope}%
\pgfpathrectangle{\pgfqpoint{0.448634in}{0.402556in}}{\pgfqpoint{4.350661in}{2.489204in}} %
\pgfusepath{clip}%
\pgfsetrectcap%
\pgfsetroundjoin%
\pgfsetlinewidth{1.003750pt}%
\definecolor{currentstroke}{rgb}{0.172549,0.627451,0.172549}%
\pgfsetstrokecolor{currentstroke}%
\pgfsetdash{}{0pt}%
\pgfpathmoveto{\pgfqpoint{3.428772in}{0.402610in}}%
\pgfpathlineto{\pgfqpoint{2.806632in}{0.403760in}}%
\pgfpathlineto{\pgfqpoint{2.769692in}{0.405578in}}%
\pgfpathlineto{\pgfqpoint{2.754632in}{0.408064in}}%
\pgfpathlineto{\pgfqpoint{2.746391in}{0.411198in}}%
\pgfpathlineto{\pgfqpoint{2.740943in}{0.415265in}}%
\pgfpathlineto{\pgfqpoint{2.736784in}{0.420984in}}%
\pgfpathlineto{\pgfqpoint{2.733281in}{0.430071in}}%
\pgfpathlineto{\pgfqpoint{2.730449in}{0.444636in}}%
\pgfpathlineto{\pgfqpoint{2.728238in}{0.469392in}}%
\pgfpathlineto{\pgfqpoint{2.726470in}{0.519131in}}%
\pgfpathlineto{\pgfqpoint{2.725711in}{0.613715in}}%
\pgfpathlineto{\pgfqpoint{2.726842in}{0.768038in}}%
\pgfpathlineto{\pgfqpoint{2.730556in}{0.962148in}}%
\pgfpathlineto{\pgfqpoint{2.736611in}{1.158670in}}%
\pgfpathlineto{\pgfqpoint{2.744092in}{1.327718in}}%
\pgfpathlineto{\pgfqpoint{2.753201in}{1.484189in}}%
\pgfpathlineto{\pgfqpoint{2.763257in}{1.620609in}}%
\pgfpathlineto{\pgfqpoint{2.776118in}{1.764216in}}%
\pgfpathlineto{\pgfqpoint{2.788914in}{1.877776in}}%
\pgfpathlineto{\pgfqpoint{2.805748in}{2.005740in}}%
\pgfpathlineto{\pgfqpoint{2.821176in}{2.101198in}}%
\pgfpathlineto{\pgfqpoint{2.838359in}{2.193718in}}%
\pgfpathlineto{\pgfqpoint{2.859135in}{2.292966in}}%
\pgfpathlineto{\pgfqpoint{2.887209in}{2.425960in}}%
\pgfpathlineto{\pgfqpoint{2.896991in}{2.479560in}}%
\pgfpathlineto{\pgfqpoint{2.901543in}{2.516523in}}%
\pgfpathlineto{\pgfqpoint{2.902849in}{2.543854in}}%
\pgfpathlineto{\pgfqpoint{2.901957in}{2.566223in}}%
\pgfpathlineto{\pgfqpoint{2.899151in}{2.585863in}}%
\pgfpathlineto{\pgfqpoint{2.894794in}{2.602546in}}%
\pgfpathlineto{\pgfqpoint{2.888484in}{2.618388in}}%
\pgfpathlineto{\pgfqpoint{2.880257in}{2.633033in}}%
\pgfpathlineto{\pgfqpoint{2.870348in}{2.646246in}}%
\pgfpathlineto{\pgfqpoint{2.857400in}{2.659530in}}%
\pgfpathlineto{\pgfqpoint{2.843189in}{2.671010in}}%
\pgfpathlineto{\pgfqpoint{2.824237in}{2.683209in}}%
\pgfpathlineto{\pgfqpoint{2.802413in}{2.694418in}}%
\pgfpathlineto{\pgfqpoint{2.775809in}{2.705369in}}%
\pgfpathlineto{\pgfqpoint{2.744461in}{2.715715in}}%
\pgfpathlineto{\pgfqpoint{2.708436in}{2.725252in}}%
\pgfpathlineto{\pgfqpoint{2.665655in}{2.734289in}}%
\pgfpathlineto{\pgfqpoint{2.613991in}{2.742869in}}%
\pgfpathlineto{\pgfqpoint{2.553459in}{2.750589in}}%
\pgfpathlineto{\pgfqpoint{2.481920in}{2.757365in}}%
\pgfpathlineto{\pgfqpoint{2.399398in}{2.762839in}}%
\pgfpathlineto{\pgfqpoint{2.310269in}{2.766482in}}%
\pgfpathlineto{\pgfqpoint{2.175416in}{2.768725in}}%
\pgfpathlineto{\pgfqpoint{2.066653in}{2.767942in}}%
\pgfpathlineto{\pgfqpoint{1.953570in}{2.764859in}}%
\pgfpathlineto{\pgfqpoint{1.851429in}{2.759759in}}%
\pgfpathlineto{\pgfqpoint{1.745051in}{2.752169in}}%
\pgfpathlineto{\pgfqpoint{1.658373in}{2.743453in}}%
\pgfpathlineto{\pgfqpoint{1.580552in}{2.733461in}}%
\pgfpathlineto{\pgfqpoint{1.490057in}{2.719338in}}%
\pgfpathlineto{\pgfqpoint{1.417231in}{2.704698in}}%
\pgfpathlineto{\pgfqpoint{1.361992in}{2.690818in}}%
\pgfpathlineto{\pgfqpoint{1.311460in}{2.675819in}}%
\pgfpathlineto{\pgfqpoint{1.265667in}{2.659924in}}%
\pgfpathlineto{\pgfqpoint{1.222575in}{2.642586in}}%
\pgfpathlineto{\pgfqpoint{1.184324in}{2.624682in}}%
\pgfpathlineto{\pgfqpoint{1.148892in}{2.605623in}}%
\pgfpathlineto{\pgfqpoint{1.116331in}{2.585573in}}%
\pgfpathlineto{\pgfqpoint{1.092327in}{2.568512in}}%
\pgfpathlineto{\pgfqpoint{1.079760in}{2.558686in}}%
\pgfpathlineto{\pgfqpoint{1.051544in}{2.535379in}}%
\pgfpathlineto{\pgfqpoint{1.026312in}{2.511712in}}%
\pgfpathlineto{\pgfqpoint{1.002399in}{2.486318in}}%
\pgfpathlineto{\pgfqpoint{0.979913in}{2.459269in}}%
\pgfpathlineto{\pgfqpoint{0.958934in}{2.430678in}}%
\pgfpathlineto{\pgfqpoint{0.938264in}{2.398643in}}%
\pgfpathlineto{\pgfqpoint{0.923047in}{2.371385in}}%
\pgfpathlineto{\pgfqpoint{0.904513in}{2.334774in}}%
\pgfpathlineto{\pgfqpoint{0.887854in}{2.297001in}}%
\pgfpathlineto{\pgfqpoint{0.872131in}{2.255971in}}%
\pgfpathlineto{\pgfqpoint{0.857508in}{2.211741in}}%
\pgfpathlineto{\pgfqpoint{0.844762in}{2.166757in}}%
\pgfpathlineto{\pgfqpoint{0.838624in}{2.140306in}}%
\pgfpathlineto{\pgfqpoint{0.826982in}{2.087194in}}%
\pgfpathlineto{\pgfqpoint{0.816322in}{2.028715in}}%
\pgfpathlineto{\pgfqpoint{0.810087in}{1.984495in}}%
\pgfpathlineto{\pgfqpoint{0.808026in}{1.967238in}}%
\pgfpathlineto{\pgfqpoint{0.800076in}{1.898140in}}%
\pgfpathlineto{\pgfqpoint{0.793713in}{1.823823in}}%
\pgfpathlineto{\pgfqpoint{0.788799in}{1.741875in}}%
\pgfpathlineto{\pgfqpoint{0.786199in}{1.677225in}}%
\pgfpathlineto{\pgfqpoint{0.776951in}{1.453481in}}%
\pgfpathlineto{\pgfqpoint{0.773280in}{1.418894in}}%
\pgfpathlineto{\pgfqpoint{0.768298in}{1.389582in}}%
\pgfpathlineto{\pgfqpoint{0.762752in}{1.368108in}}%
\pgfpathlineto{\pgfqpoint{0.756722in}{1.352123in}}%
\pgfpathlineto{\pgfqpoint{0.749752in}{1.339519in}}%
\pgfpathlineto{\pgfqpoint{0.742201in}{1.330599in}}%
\pgfpathlineto{\pgfqpoint{0.734854in}{1.325312in}}%
\pgfpathlineto{\pgfqpoint{0.726558in}{1.322419in}}%
\pgfpathlineto{\pgfqpoint{0.717884in}{1.322223in}}%
\pgfpathlineto{\pgfqpoint{0.709412in}{1.324411in}}%
\pgfpathlineto{\pgfqpoint{0.699548in}{1.329604in}}%
\pgfpathlineto{\pgfqpoint{0.688894in}{1.338203in}}%
\pgfpathlineto{\pgfqpoint{0.677907in}{1.350248in}}%
\pgfpathlineto{\pgfqpoint{0.666886in}{1.365647in}}%
\pgfpathlineto{\pgfqpoint{0.654913in}{1.386417in}}%
\pgfpathlineto{\pgfqpoint{0.642574in}{1.412730in}}%
\pgfpathlineto{\pgfqpoint{0.630328in}{1.444629in}}%
\pgfpathlineto{\pgfqpoint{0.618504in}{1.482081in}}%
\pgfpathlineto{\pgfqpoint{0.608613in}{1.520256in}}%
\pgfpathlineto{\pgfqpoint{0.590203in}{1.612445in}}%
\pgfpathlineto{\pgfqpoint{0.581848in}{1.668884in}}%
\pgfpathlineto{\pgfqpoint{0.573137in}{1.740376in}}%
\pgfpathlineto{\pgfqpoint{0.567062in}{1.807213in}}%
\pgfpathlineto{\pgfqpoint{0.560532in}{1.896510in}}%
\pgfpathlineto{\pgfqpoint{0.555526in}{1.995910in}}%
\pgfpathlineto{\pgfqpoint{0.552564in}{2.097908in}}%
\pgfpathlineto{\pgfqpoint{0.551526in}{2.204935in}}%
\pgfpathlineto{\pgfqpoint{0.552728in}{2.309470in}}%
\pgfpathlineto{\pgfqpoint{0.556011in}{2.403981in}}%
\pgfpathlineto{\pgfqpoint{0.560953in}{2.483430in}}%
\pgfpathlineto{\pgfqpoint{0.567303in}{2.550240in}}%
\pgfpathlineto{\pgfqpoint{0.574928in}{2.606817in}}%
\pgfpathlineto{\pgfqpoint{0.582988in}{2.650657in}}%
\pgfpathlineto{\pgfqpoint{0.592756in}{2.691452in}}%
\pgfpathlineto{\pgfqpoint{0.602650in}{2.721756in}}%
\pgfpathlineto{\pgfqpoint{0.612983in}{2.746441in}}%
\pgfpathlineto{\pgfqpoint{0.624292in}{2.767692in}}%
\pgfpathlineto{\pgfqpoint{0.636231in}{2.785433in}}%
\pgfpathlineto{\pgfqpoint{0.649892in}{2.801461in}}%
\pgfpathlineto{\pgfqpoint{0.663386in}{2.814020in}}%
\pgfpathlineto{\pgfqpoint{0.679842in}{2.826135in}}%
\pgfpathlineto{\pgfqpoint{0.697326in}{2.836197in}}%
\pgfpathlineto{\pgfqpoint{0.715574in}{2.844285in}}%
\pgfpathlineto{\pgfqpoint{0.738439in}{2.852335in}}%
\pgfpathlineto{\pgfqpoint{0.765983in}{2.859639in}}%
\pgfpathlineto{\pgfqpoint{0.800300in}{2.866256in}}%
\pgfpathlineto{\pgfqpoint{0.841340in}{2.871832in}}%
\pgfpathlineto{\pgfqpoint{0.895547in}{2.876803in}}%
\pgfpathlineto{\pgfqpoint{0.969413in}{2.881069in}}%
\pgfpathlineto{\pgfqpoint{1.071608in}{2.884501in}}%
\pgfpathlineto{\pgfqpoint{1.219512in}{2.887074in}}%
\pgfpathlineto{\pgfqpoint{1.471844in}{2.889091in}}%
\pgfpathlineto{\pgfqpoint{1.956941in}{2.890384in}}%
\pgfpathlineto{\pgfqpoint{3.096814in}{2.890781in}}%
\pgfpathlineto{\pgfqpoint{3.995224in}{2.889388in}}%
\pgfpathlineto{\pgfqpoint{4.275833in}{2.887011in}}%
\pgfpathlineto{\pgfqpoint{4.412847in}{2.883743in}}%
\pgfpathlineto{\pgfqpoint{4.491081in}{2.879810in}}%
\pgfpathlineto{\pgfqpoint{4.543127in}{2.875163in}}%
\pgfpathlineto{\pgfqpoint{4.579810in}{2.869841in}}%
\pgfpathlineto{\pgfqpoint{4.607580in}{2.863763in}}%
\pgfpathlineto{\pgfqpoint{4.630623in}{2.856424in}}%
\pgfpathlineto{\pgfqpoint{4.648833in}{2.848228in}}%
\pgfpathlineto{\pgfqpoint{4.664136in}{2.838773in}}%
\pgfpathlineto{\pgfqpoint{4.676470in}{2.828576in}}%
\pgfpathlineto{\pgfqpoint{4.687502in}{2.816585in}}%
\pgfpathlineto{\pgfqpoint{4.697051in}{2.803027in}}%
\pgfpathlineto{\pgfqpoint{4.706194in}{2.786098in}}%
\pgfpathlineto{\pgfqpoint{4.714508in}{2.765827in}}%
\pgfpathlineto{\pgfqpoint{4.722462in}{2.740013in}}%
\pgfpathlineto{\pgfqpoint{4.729577in}{2.708703in}}%
\pgfpathlineto{\pgfqpoint{4.736162in}{2.669601in}}%
\pgfpathlineto{\pgfqpoint{4.742419in}{2.617826in}}%
\pgfpathlineto{\pgfqpoint{4.747859in}{2.553410in}}%
\pgfpathlineto{\pgfqpoint{4.752661in}{2.468958in}}%
\pgfpathlineto{\pgfqpoint{4.756610in}{2.359528in}}%
\pgfpathlineto{\pgfqpoint{4.759416in}{2.217681in}}%
\pgfpathlineto{\pgfqpoint{4.760596in}{2.043444in}}%
\pgfpathlineto{\pgfqpoint{4.759662in}{1.851779in}}%
\pgfpathlineto{\pgfqpoint{4.756587in}{1.667613in}}%
\pgfpathlineto{\pgfqpoint{4.751596in}{1.503428in}}%
\pgfpathlineto{\pgfqpoint{4.745410in}{1.374185in}}%
\pgfpathlineto{\pgfqpoint{4.738113in}{1.267479in}}%
\pgfpathlineto{\pgfqpoint{4.729621in}{1.175896in}}%
\pgfpathlineto{\pgfqpoint{4.720762in}{1.104428in}}%
\pgfpathlineto{\pgfqpoint{4.711045in}{1.043204in}}%
\pgfpathlineto{\pgfqpoint{4.700364in}{0.989829in}}%
\pgfpathlineto{\pgfqpoint{4.689055in}{0.944345in}}%
\pgfpathlineto{\pgfqpoint{4.676881in}{0.904394in}}%
\pgfpathlineto{\pgfqpoint{4.676095in}{0.902073in}}%
\pgfpathlineto{\pgfqpoint{4.676095in}{0.902073in}}%
\pgfusepath{stroke}%
\end{pgfscope}%
\begin{pgfscope}%
\pgfpathrectangle{\pgfqpoint{0.448634in}{0.402556in}}{\pgfqpoint{4.350661in}{2.489204in}} %
\pgfusepath{clip}%
\pgfsetrectcap%
\pgfsetroundjoin%
\pgfsetlinewidth{1.003750pt}%
\definecolor{currentstroke}{rgb}{0.172549,0.627451,0.172549}%
\pgfsetstrokecolor{currentstroke}%
\pgfsetdash{}{0pt}%
\pgfpathmoveto{\pgfqpoint{2.795520in}{1.982745in}}%
\pgfpathlineto{\pgfqpoint{2.781780in}{1.874357in}}%
\pgfpathlineto{\pgfqpoint{2.769351in}{1.758234in}}%
\pgfpathlineto{\pgfqpoint{2.758095in}{1.631942in}}%
\pgfpathlineto{\pgfqpoint{2.747786in}{1.490551in}}%
\pgfpathlineto{\pgfqpoint{2.738644in}{1.334082in}}%
\pgfpathlineto{\pgfqpoint{2.730580in}{1.157591in}}%
\pgfpathlineto{\pgfqpoint{2.723334in}{0.948663in}}%
\pgfpathlineto{\pgfqpoint{2.709783in}{0.530788in}}%
\pgfpathlineto{\pgfqpoint{2.705868in}{0.488716in}}%
\pgfpathlineto{\pgfqpoint{2.701769in}{0.464281in}}%
\pgfpathlineto{\pgfqpoint{2.697021in}{0.447744in}}%
\pgfpathlineto{\pgfqpoint{2.691859in}{0.436812in}}%
\pgfpathlineto{\pgfqpoint{2.686245in}{0.429229in}}%
\pgfpathlineto{\pgfqpoint{2.679348in}{0.423188in}}%
\pgfpathlineto{\pgfqpoint{2.669540in}{0.417856in}}%
\pgfpathlineto{\pgfqpoint{2.656987in}{0.413810in}}%
\pgfpathlineto{\pgfqpoint{2.637654in}{0.410337in}}%
\pgfpathlineto{\pgfqpoint{2.607297in}{0.407617in}}%
\pgfpathlineto{\pgfqpoint{2.555121in}{0.405574in}}%
\pgfpathlineto{\pgfqpoint{2.450714in}{0.404139in}}%
\pgfpathlineto{\pgfqpoint{2.176624in}{0.403275in}}%
\pgfpathlineto{\pgfqpoint{1.130290in}{0.402953in}}%
\pgfpathlineto{\pgfqpoint{0.516849in}{0.404175in}}%
\pgfpathlineto{\pgfqpoint{0.466848in}{0.405970in}}%
\pgfpathlineto{\pgfqpoint{0.456130in}{0.407931in}}%
\pgfpathlineto{\pgfqpoint{0.452340in}{0.410303in}}%
\pgfpathlineto{\pgfqpoint{0.450346in}{0.414662in}}%
\pgfpathlineto{\pgfqpoint{0.449266in}{0.424524in}}%
\pgfpathlineto{\pgfqpoint{0.448771in}{0.464344in}}%
\pgfpathlineto{\pgfqpoint{0.448640in}{0.850171in}}%
\pgfpathlineto{\pgfqpoint{0.448679in}{2.891318in}}%
\pgfpathlineto{\pgfqpoint{0.448679in}{2.891318in}}%
\pgfusepath{stroke}%
\end{pgfscope}%
\begin{pgfscope}%
\pgfpathrectangle{\pgfqpoint{0.448634in}{0.402556in}}{\pgfqpoint{4.350661in}{2.489204in}} %
\pgfusepath{clip}%
\pgfsetrectcap%
\pgfsetroundjoin%
\pgfsetlinewidth{1.003750pt}%
\definecolor{currentstroke}{rgb}{0.172549,0.627451,0.172549}%
\pgfsetstrokecolor{currentstroke}%
\pgfsetdash{}{0pt}%
\pgfpathmoveto{\pgfqpoint{3.428189in}{0.402586in}}%
\pgfpathlineto{\pgfqpoint{2.782121in}{0.403701in}}%
\pgfpathlineto{\pgfqpoint{2.753906in}{0.405674in}}%
\pgfpathlineto{\pgfqpoint{2.743328in}{0.408444in}}%
\pgfpathlineto{\pgfqpoint{2.737717in}{0.412188in}}%
\pgfpathlineto{\pgfqpoint{2.733668in}{0.417995in}}%
\pgfpathlineto{\pgfqpoint{2.730649in}{0.427307in}}%
\pgfpathlineto{\pgfqpoint{2.728388in}{0.442004in}}%
\pgfpathlineto{\pgfqpoint{2.726544in}{0.471794in}}%
\pgfpathlineto{\pgfqpoint{2.725216in}{0.534003in}}%
\pgfpathlineto{\pgfqpoint{2.725169in}{0.655973in}}%
\pgfpathlineto{\pgfqpoint{2.727377in}{0.832687in}}%
\pgfpathlineto{\pgfqpoint{2.732259in}{1.041703in}}%
\pgfpathlineto{\pgfqpoint{2.738851in}{1.223257in}}%
\pgfpathlineto{\pgfqpoint{2.747078in}{1.389766in}}%
\pgfpathlineto{\pgfqpoint{2.756608in}{1.538717in}}%
\pgfpathlineto{\pgfqpoint{2.768955in}{1.694887in}}%
\pgfpathlineto{\pgfqpoint{2.781228in}{1.816044in}}%
\pgfpathlineto{\pgfqpoint{2.794401in}{1.924524in}}%
\pgfpathlineto{\pgfqpoint{2.812737in}{2.054722in}}%
\pgfpathlineto{\pgfqpoint{2.828774in}{2.147512in}}%
\pgfpathlineto{\pgfqpoint{2.847382in}{2.242224in}}%
\pgfpathlineto{\pgfqpoint{2.895818in}{2.479699in}}%
\pgfpathlineto{\pgfqpoint{2.900204in}{2.516689in}}%
\pgfpathlineto{\pgfqpoint{2.901346in}{2.544029in}}%
\pgfpathlineto{\pgfqpoint{2.900291in}{2.566388in}}%
\pgfpathlineto{\pgfqpoint{2.897334in}{2.585999in}}%
\pgfpathlineto{\pgfqpoint{2.892836in}{2.602633in}}%
\pgfpathlineto{\pgfqpoint{2.886394in}{2.618406in}}%
\pgfpathlineto{\pgfqpoint{2.878058in}{2.632970in}}%
\pgfpathlineto{\pgfqpoint{2.868065in}{2.646100in}}%
\pgfpathlineto{\pgfqpoint{2.855050in}{2.659300in}}%
\pgfpathlineto{\pgfqpoint{2.840801in}{2.670717in}}%
\pgfpathlineto{\pgfqpoint{2.821822in}{2.682861in}}%
\pgfpathlineto{\pgfqpoint{2.799980in}{2.694026in}}%
\pgfpathlineto{\pgfqpoint{2.773366in}{2.704944in}}%
\pgfpathlineto{\pgfqpoint{2.742012in}{2.715266in}}%
\pgfpathlineto{\pgfqpoint{2.705983in}{2.724785in}}%
\pgfpathlineto{\pgfqpoint{2.663200in}{2.733810in}}%
\pgfpathlineto{\pgfqpoint{2.611535in}{2.742379in}}%
\pgfpathlineto{\pgfqpoint{2.551002in}{2.750090in}}%
\pgfpathlineto{\pgfqpoint{2.481632in}{2.756682in}}%
\pgfpathlineto{\pgfqpoint{2.399112in}{2.762200in}}%
\pgfpathlineto{\pgfqpoint{2.309985in}{2.765886in}}%
\pgfpathlineto{\pgfqpoint{2.188184in}{2.768096in}}%
\pgfpathlineto{\pgfqpoint{2.081595in}{2.767619in}}%
\pgfpathlineto{\pgfqpoint{1.968506in}{2.764840in}}%
\pgfpathlineto{\pgfqpoint{1.864180in}{2.759918in}}%
\pgfpathlineto{\pgfqpoint{1.757786in}{2.752593in}}%
\pgfpathlineto{\pgfqpoint{1.671087in}{2.744171in}}%
\pgfpathlineto{\pgfqpoint{1.591076in}{2.734193in}}%
\pgfpathlineto{\pgfqpoint{1.502689in}{2.720717in}}%
\pgfpathlineto{\pgfqpoint{1.427655in}{2.706083in}}%
\pgfpathlineto{\pgfqpoint{1.372350in}{2.692544in}}%
\pgfpathlineto{\pgfqpoint{1.321734in}{2.677921in}}%
\pgfpathlineto{\pgfqpoint{1.273765in}{2.661664in}}%
\pgfpathlineto{\pgfqpoint{1.230567in}{2.644672in}}%
\pgfpathlineto{\pgfqpoint{1.192197in}{2.627106in}}%
\pgfpathlineto{\pgfqpoint{1.156620in}{2.608403in}}%
\pgfpathlineto{\pgfqpoint{1.123890in}{2.588716in}}%
\pgfpathlineto{\pgfqpoint{1.095883in}{2.569568in}}%
\pgfpathlineto{\pgfqpoint{1.063936in}{2.543701in}}%
\pgfpathlineto{\pgfqpoint{1.038217in}{2.520732in}}%
\pgfpathlineto{\pgfqpoint{1.013766in}{2.496016in}}%
\pgfpathlineto{\pgfqpoint{0.990704in}{2.469610in}}%
\pgfpathlineto{\pgfqpoint{0.969124in}{2.441612in}}%
\pgfpathlineto{\pgfqpoint{0.949083in}{2.412154in}}%
\pgfpathlineto{\pgfqpoint{0.930604in}{2.381387in}}%
\pgfpathlineto{\pgfqpoint{0.906555in}{2.334052in}}%
\pgfpathlineto{\pgfqpoint{0.889925in}{2.296262in}}%
\pgfpathlineto{\pgfqpoint{0.874241in}{2.255213in}}%
\pgfpathlineto{\pgfqpoint{0.859667in}{2.210961in}}%
\pgfpathlineto{\pgfqpoint{0.846986in}{2.165954in}}%
\pgfpathlineto{\pgfqpoint{0.839633in}{2.134715in}}%
\pgfpathlineto{\pgfqpoint{0.828238in}{2.081532in}}%
\pgfpathlineto{\pgfqpoint{0.817866in}{2.022986in}}%
\pgfpathlineto{\pgfqpoint{0.810784in}{1.971352in}}%
\pgfpathlineto{\pgfqpoint{0.802845in}{1.902252in}}%
\pgfpathlineto{\pgfqpoint{0.796554in}{1.827927in}}%
\pgfpathlineto{\pgfqpoint{0.791696in}{1.743480in}}%
\pgfpathlineto{\pgfqpoint{0.787773in}{1.621595in}}%
\pgfpathlineto{\pgfqpoint{0.785407in}{1.522064in}}%
\pgfpathlineto{\pgfqpoint{0.785407in}{1.522064in}}%
\pgfusepath{stroke}%
\end{pgfscope}%
\begin{pgfscope}%
\pgfpathrectangle{\pgfqpoint{0.448634in}{0.402556in}}{\pgfqpoint{4.350661in}{2.489204in}} %
\pgfusepath{clip}%
\pgfsetrectcap%
\pgfsetroundjoin%
\pgfsetlinewidth{1.003750pt}%
\definecolor{currentstroke}{rgb}{0.839216,0.152941,0.156863}%
\pgfsetstrokecolor{currentstroke}%
\pgfsetdash{}{0pt}%
\pgfpathmoveto{\pgfqpoint{1.127319in}{2.572074in}}%
\pgfpathlineto{\pgfqpoint{1.159575in}{2.592758in}}%
\pgfpathlineto{\pgfqpoint{1.192763in}{2.611414in}}%
\pgfpathlineto{\pgfqpoint{1.228726in}{2.629126in}}%
\pgfpathlineto{\pgfqpoint{1.267413in}{2.645758in}}%
\pgfpathlineto{\pgfqpoint{1.310846in}{2.661945in}}%
\pgfpathlineto{\pgfqpoint{1.356920in}{2.676740in}}%
\pgfpathlineto{\pgfqpoint{1.407680in}{2.690702in}}%
\pgfpathlineto{\pgfqpoint{1.463094in}{2.703640in}}%
\pgfpathlineto{\pgfqpoint{1.525273in}{2.715813in}}%
\pgfpathlineto{\pgfqpoint{1.594199in}{2.726937in}}%
\pgfpathlineto{\pgfqpoint{1.669843in}{2.736808in}}%
\pgfpathlineto{\pgfqpoint{1.752172in}{2.745271in}}%
\pgfpathlineto{\pgfqpoint{1.843325in}{2.752344in}}%
\pgfpathlineto{\pgfqpoint{1.941103in}{2.757656in}}%
\pgfpathlineto{\pgfqpoint{2.043301in}{2.760987in}}%
\pgfpathlineto{\pgfqpoint{2.147710in}{2.762199in}}%
\pgfpathlineto{\pgfqpoint{2.249945in}{2.761215in}}%
\pgfpathlineto{\pgfqpoint{2.345620in}{2.758145in}}%
\pgfpathlineto{\pgfqpoint{2.432525in}{2.753210in}}%
\pgfpathlineto{\pgfqpoint{2.508451in}{2.746766in}}%
\pgfpathlineto{\pgfqpoint{2.573368in}{2.739156in}}%
\pgfpathlineto{\pgfqpoint{2.629410in}{2.730451in}}%
\pgfpathlineto{\pgfqpoint{2.676543in}{2.720985in}}%
\pgfpathlineto{\pgfqpoint{2.716874in}{2.710666in}}%
\pgfpathlineto{\pgfqpoint{2.750366in}{2.699848in}}%
\pgfpathlineto{\pgfqpoint{2.779059in}{2.688192in}}%
\pgfpathlineto{\pgfqpoint{2.802882in}{2.676004in}}%
\pgfpathlineto{\pgfqpoint{2.821842in}{2.663819in}}%
\pgfpathlineto{\pgfqpoint{2.837815in}{2.650886in}}%
\pgfpathlineto{\pgfqpoint{2.850736in}{2.637564in}}%
\pgfpathlineto{\pgfqpoint{2.860694in}{2.624398in}}%
\pgfpathlineto{\pgfqpoint{2.869084in}{2.609873in}}%
\pgfpathlineto{\pgfqpoint{2.875698in}{2.594192in}}%
\pgfpathlineto{\pgfqpoint{2.881035in}{2.575255in}}%
\pgfpathlineto{\pgfqpoint{2.884200in}{2.555685in}}%
\pgfpathlineto{\pgfqpoint{2.885619in}{2.533351in}}%
\pgfpathlineto{\pgfqpoint{2.885038in}{2.505987in}}%
\pgfpathlineto{\pgfqpoint{2.882112in}{2.473807in}}%
\pgfpathlineto{\pgfqpoint{2.875657in}{2.429620in}}%
\pgfpathlineto{\pgfqpoint{2.863489in}{2.363873in}}%
\pgfpathlineto{\pgfqpoint{2.821102in}{2.142619in}}%
\pgfpathlineto{\pgfqpoint{2.804859in}{2.042271in}}%
\pgfpathlineto{\pgfqpoint{2.790421in}{1.939040in}}%
\pgfpathlineto{\pgfqpoint{2.777207in}{1.828054in}}%
\pgfpathlineto{\pgfqpoint{2.765338in}{1.709349in}}%
\pgfpathlineto{\pgfqpoint{2.754471in}{1.578010in}}%
\pgfpathlineto{\pgfqpoint{2.744640in}{1.431580in}}%
\pgfpathlineto{\pgfqpoint{2.735914in}{1.267598in}}%
\pgfpathlineto{\pgfqpoint{2.728277in}{1.081114in}}%
\pgfpathlineto{\pgfqpoint{2.721437in}{0.857223in}}%
\pgfpathlineto{\pgfqpoint{2.711961in}{0.541290in}}%
\pgfpathlineto{\pgfqpoint{2.708250in}{0.491694in}}%
\pgfpathlineto{\pgfqpoint{2.703951in}{0.462246in}}%
\pgfpathlineto{\pgfqpoint{2.699504in}{0.445599in}}%
\pgfpathlineto{\pgfqpoint{2.694517in}{0.434563in}}%
\pgfpathlineto{\pgfqpoint{2.688942in}{0.426947in}}%
\pgfpathlineto{\pgfqpoint{2.681980in}{0.421009in}}%
\pgfpathlineto{\pgfqpoint{2.672064in}{0.415948in}}%
\pgfpathlineto{\pgfqpoint{2.659429in}{0.412247in}}%
\pgfpathlineto{\pgfqpoint{2.640044in}{0.409163in}}%
\pgfpathlineto{\pgfqpoint{2.607490in}{0.406692in}}%
\pgfpathlineto{\pgfqpoint{2.548779in}{0.404894in}}%
\pgfpathlineto{\pgfqpoint{2.422615in}{0.403701in}}%
\pgfpathlineto{\pgfqpoint{2.026705in}{0.403016in}}%
\pgfpathlineto{\pgfqpoint{0.623617in}{0.403253in}}%
\pgfpathlineto{\pgfqpoint{0.477880in}{0.404742in}}%
\pgfpathlineto{\pgfqpoint{0.458368in}{0.406382in}}%
\pgfpathlineto{\pgfqpoint{0.452304in}{0.408937in}}%
\pgfpathlineto{\pgfqpoint{0.450213in}{0.413215in}}%
\pgfpathlineto{\pgfqpoint{0.449165in}{0.423080in}}%
\pgfpathlineto{\pgfqpoint{0.448735in}{0.465392in}}%
\pgfpathlineto{\pgfqpoint{0.448637in}{0.983146in}}%
\pgfpathlineto{\pgfqpoint{0.448652in}{2.889876in}}%
\pgfpathlineto{\pgfqpoint{0.448652in}{2.889876in}}%
\pgfusepath{stroke}%
\end{pgfscope}%
\begin{pgfscope}%
\pgfpathrectangle{\pgfqpoint{0.448634in}{0.402556in}}{\pgfqpoint{4.350661in}{2.489204in}} %
\pgfusepath{clip}%
\pgfsetrectcap%
\pgfsetroundjoin%
\pgfsetlinewidth{1.003750pt}%
\definecolor{currentstroke}{rgb}{0.839216,0.152941,0.156863}%
\pgfsetstrokecolor{currentstroke}%
\pgfsetdash{}{0pt}%
\pgfpathmoveto{\pgfqpoint{0.448634in}{2.896245in}}%
\pgfpathlineto{\pgfqpoint{0.448593in}{0.407043in}}%
\pgfpathlineto{\pgfqpoint{0.448593in}{0.407043in}}%
\pgfusepath{stroke}%
\end{pgfscope}%
\begin{pgfscope}%
\pgfpathrectangle{\pgfqpoint{0.448634in}{0.402556in}}{\pgfqpoint{4.350661in}{2.489204in}} %
\pgfusepath{clip}%
\pgfsetrectcap%
\pgfsetroundjoin%
\pgfsetlinewidth{1.003750pt}%
\definecolor{currentstroke}{rgb}{0.839216,0.152941,0.156863}%
\pgfsetstrokecolor{currentstroke}%
\pgfsetdash{}{0pt}%
\pgfpathmoveto{\pgfqpoint{0.576853in}{1.760817in}}%
\pgfpathlineto{\pgfqpoint{0.569394in}{1.840010in}}%
\pgfpathlineto{\pgfqpoint{0.563208in}{1.929339in}}%
\pgfpathlineto{\pgfqpoint{0.558592in}{2.028764in}}%
\pgfpathlineto{\pgfqpoint{0.555985in}{2.133265in}}%
\pgfpathlineto{\pgfqpoint{0.555565in}{2.237808in}}%
\pgfpathlineto{\pgfqpoint{0.557371in}{2.337352in}}%
\pgfpathlineto{\pgfqpoint{0.561096in}{2.424366in}}%
\pgfpathlineto{\pgfqpoint{0.566403in}{2.498791in}}%
\pgfpathlineto{\pgfqpoint{0.572909in}{2.560570in}}%
\pgfpathlineto{\pgfqpoint{0.580458in}{2.612119in}}%
\pgfpathlineto{\pgfqpoint{0.589086in}{2.655816in}}%
\pgfpathlineto{\pgfqpoint{0.598406in}{2.691590in}}%
\pgfpathlineto{\pgfqpoint{0.608613in}{2.721757in}}%
\pgfpathlineto{\pgfqpoint{0.619241in}{2.746278in}}%
\pgfpathlineto{\pgfqpoint{0.630817in}{2.767339in}}%
\pgfpathlineto{\pgfqpoint{0.642975in}{2.784884in}}%
\pgfpathlineto{\pgfqpoint{0.656813in}{2.800713in}}%
\pgfpathlineto{\pgfqpoint{0.672197in}{2.814549in}}%
\pgfpathlineto{\pgfqpoint{0.688853in}{2.826301in}}%
\pgfpathlineto{\pgfqpoint{0.706461in}{2.836076in}}%
\pgfpathlineto{\pgfqpoint{0.726804in}{2.844876in}}%
\pgfpathlineto{\pgfqpoint{0.751866in}{2.853203in}}%
\pgfpathlineto{\pgfqpoint{0.781632in}{2.860547in}}%
\pgfpathlineto{\pgfqpoint{0.818168in}{2.867054in}}%
\pgfpathlineto{\pgfqpoint{0.863581in}{2.872685in}}%
\pgfpathlineto{\pgfqpoint{0.922161in}{2.877518in}}%
\pgfpathlineto{\pgfqpoint{1.000391in}{2.881567in}}%
\pgfpathlineto{\pgfqpoint{1.111294in}{2.884881in}}%
\pgfpathlineto{\pgfqpoint{1.274428in}{2.887367in}}%
\pgfpathlineto{\pgfqpoint{1.552865in}{2.889263in}}%
\pgfpathlineto{\pgfqpoint{2.107573in}{2.890457in}}%
\pgfpathlineto{\pgfqpoint{3.343161in}{2.890573in}}%
\pgfpathlineto{\pgfqpoint{4.043615in}{2.888941in}}%
\pgfpathlineto{\pgfqpoint{4.289417in}{2.886404in}}%
\pgfpathlineto{\pgfqpoint{4.413375in}{2.883093in}}%
\pgfpathlineto{\pgfqpoint{4.489425in}{2.878997in}}%
\pgfpathlineto{\pgfqpoint{4.541451in}{2.874081in}}%
\pgfpathlineto{\pgfqpoint{4.578100in}{2.868470in}}%
\pgfpathlineto{\pgfqpoint{4.605819in}{2.862092in}}%
\pgfpathlineto{\pgfqpoint{4.626726in}{2.855245in}}%
\pgfpathlineto{\pgfqpoint{4.644925in}{2.847018in}}%
\pgfpathlineto{\pgfqpoint{4.660241in}{2.837590in}}%
\pgfpathlineto{\pgfqpoint{4.672623in}{2.827468in}}%
\pgfpathlineto{\pgfqpoint{4.683751in}{2.815592in}}%
\pgfpathlineto{\pgfqpoint{4.693406in}{2.802135in}}%
\pgfpathlineto{\pgfqpoint{4.702740in}{2.785343in}}%
\pgfpathlineto{\pgfqpoint{4.711277in}{2.765194in}}%
\pgfpathlineto{\pgfqpoint{4.719483in}{2.739484in}}%
\pgfpathlineto{\pgfqpoint{4.726294in}{2.710657in}}%
\pgfpathlineto{\pgfqpoint{4.733260in}{2.671642in}}%
\pgfpathlineto{\pgfqpoint{4.739604in}{2.622396in}}%
\pgfpathlineto{\pgfqpoint{4.745236in}{2.560504in}}%
\pgfpathlineto{\pgfqpoint{4.750164in}{2.481051in}}%
\pgfpathlineto{\pgfqpoint{4.754367in}{2.376618in}}%
\pgfpathlineto{\pgfqpoint{4.757443in}{2.242249in}}%
\pgfpathlineto{\pgfqpoint{4.758977in}{2.075483in}}%
\pgfpathlineto{\pgfqpoint{4.758447in}{1.888795in}}%
\pgfpathlineto{\pgfqpoint{4.755756in}{1.707110in}}%
\pgfpathlineto{\pgfqpoint{4.750925in}{1.532957in}}%
\pgfpathlineto{\pgfqpoint{4.744785in}{1.398726in}}%
\pgfpathlineto{\pgfqpoint{4.737575in}{1.289515in}}%
\pgfpathlineto{\pgfqpoint{4.728714in}{1.190469in}}%
\pgfpathlineto{\pgfqpoint{4.719652in}{1.116521in}}%
\pgfpathlineto{\pgfqpoint{4.710036in}{1.055276in}}%
\pgfpathlineto{\pgfqpoint{4.699503in}{1.001861in}}%
\pgfpathlineto{\pgfqpoint{4.689040in}{0.958690in}}%
\pgfpathlineto{\pgfqpoint{4.677219in}{0.918600in}}%
\pgfpathlineto{\pgfqpoint{4.664034in}{0.881749in}}%
\pgfpathlineto{\pgfqpoint{4.650584in}{0.850491in}}%
\pgfpathlineto{\pgfqpoint{4.636303in}{0.822569in}}%
\pgfpathlineto{\pgfqpoint{4.620207in}{0.795974in}}%
\pgfpathlineto{\pgfqpoint{4.603640in}{0.772901in}}%
\pgfpathlineto{\pgfqpoint{4.585488in}{0.751446in}}%
\pgfpathlineto{\pgfqpoint{4.565874in}{0.731749in}}%
\pgfpathlineto{\pgfqpoint{4.544964in}{0.713879in}}%
\pgfpathlineto{\pgfqpoint{4.522958in}{0.697824in}}%
\pgfpathlineto{\pgfqpoint{4.496157in}{0.681290in}}%
\pgfpathlineto{\pgfqpoint{4.470397in}{0.667953in}}%
\pgfpathlineto{\pgfqpoint{4.439961in}{0.654509in}}%
\pgfpathlineto{\pgfqpoint{4.406841in}{0.642281in}}%
\pgfpathlineto{\pgfqpoint{4.369009in}{0.630748in}}%
\pgfpathlineto{\pgfqpoint{4.326489in}{0.620226in}}%
\pgfpathlineto{\pgfqpoint{4.279327in}{0.610949in}}%
\pgfpathlineto{\pgfqpoint{4.227576in}{0.603085in}}%
\pgfpathlineto{\pgfqpoint{4.173450in}{0.597063in}}%
\pgfpathlineto{\pgfqpoint{4.110511in}{0.592203in}}%
\pgfpathlineto{\pgfqpoint{4.047471in}{0.589537in}}%
\pgfpathlineto{\pgfqpoint{3.977867in}{0.588624in}}%
\pgfpathlineto{\pgfqpoint{3.906093in}{0.589934in}}%
\pgfpathlineto{\pgfqpoint{3.834377in}{0.593496in}}%
\pgfpathlineto{\pgfqpoint{3.767120in}{0.599067in}}%
\pgfpathlineto{\pgfqpoint{3.704364in}{0.606392in}}%
\pgfpathlineto{\pgfqpoint{3.678516in}{0.610510in}}%
\pgfpathlineto{\pgfqpoint{3.620438in}{0.620500in}}%
\pgfpathlineto{\pgfqpoint{3.586319in}{0.628207in}}%
\pgfpathlineto{\pgfqpoint{3.495240in}{0.652428in}}%
\pgfpathlineto{\pgfqpoint{3.451528in}{0.667583in}}%
\pgfpathlineto{\pgfqpoint{3.408538in}{0.685220in}}%
\pgfpathlineto{\pgfqpoint{3.374594in}{0.702001in}}%
\pgfpathlineto{\pgfqpoint{3.345407in}{0.718682in}}%
\pgfpathlineto{\pgfqpoint{3.315236in}{0.738520in}}%
\pgfpathlineto{\pgfqpoint{3.288127in}{0.759290in}}%
\pgfpathlineto{\pgfqpoint{3.264004in}{0.780551in}}%
\pgfpathlineto{\pgfqpoint{3.241208in}{0.803648in}}%
\pgfpathlineto{\pgfqpoint{3.219894in}{0.828530in}}%
\pgfpathlineto{\pgfqpoint{3.200189in}{0.855091in}}%
\pgfpathlineto{\pgfqpoint{3.182177in}{0.883183in}}%
\pgfpathlineto{\pgfqpoint{3.165906in}{0.912633in}}%
\pgfpathlineto{\pgfqpoint{3.150351in}{0.945448in}}%
\pgfpathlineto{\pgfqpoint{3.136682in}{0.979345in}}%
\pgfpathlineto{\pgfqpoint{3.124073in}{1.016460in}}%
\pgfpathlineto{\pgfqpoint{3.112834in}{1.056769in}}%
\pgfpathlineto{\pgfqpoint{3.103046in}{1.100146in}}%
\pgfpathlineto{\pgfqpoint{3.095343in}{1.144071in}}%
\pgfpathlineto{\pgfqpoint{3.089208in}{1.190837in}}%
\pgfpathlineto{\pgfqpoint{3.084595in}{1.242838in}}%
\pgfpathlineto{\pgfqpoint{3.082137in}{1.295031in}}%
\pgfpathlineto{\pgfqpoint{3.081687in}{1.349787in}}%
\pgfpathlineto{\pgfqpoint{3.083451in}{1.406998in}}%
\pgfpathlineto{\pgfqpoint{3.087181in}{1.461589in}}%
\pgfpathlineto{\pgfqpoint{3.093485in}{1.520888in}}%
\pgfpathlineto{\pgfqpoint{3.101823in}{1.577334in}}%
\pgfpathlineto{\pgfqpoint{3.111930in}{1.630856in}}%
\pgfpathlineto{\pgfqpoint{3.124690in}{1.686208in}}%
\pgfpathlineto{\pgfqpoint{3.139178in}{1.738395in}}%
\pgfpathlineto{\pgfqpoint{3.155145in}{1.787366in}}%
\pgfpathlineto{\pgfqpoint{3.172353in}{1.833085in}}%
\pgfpathlineto{\pgfqpoint{3.191618in}{1.877716in}}%
\pgfpathlineto{\pgfqpoint{3.214026in}{1.923261in}}%
\pgfpathlineto{\pgfqpoint{3.236214in}{1.963157in}}%
\pgfpathlineto{\pgfqpoint{3.260178in}{2.001684in}}%
\pgfpathlineto{\pgfqpoint{3.285814in}{2.038776in}}%
\pgfpathlineto{\pgfqpoint{3.314415in}{2.076285in}}%
\pgfpathlineto{\pgfqpoint{3.348944in}{2.117711in}}%
\pgfpathlineto{\pgfqpoint{3.417133in}{2.198022in}}%
\pgfpathlineto{\pgfqpoint{3.426053in}{2.212128in}}%
\pgfpathlineto{\pgfqpoint{3.430798in}{2.223297in}}%
\pgfpathlineto{\pgfqpoint{3.432034in}{2.230603in}}%
\pgfpathlineto{\pgfqpoint{3.430773in}{2.237856in}}%
\pgfpathlineto{\pgfqpoint{3.426621in}{2.243526in}}%
\pgfpathlineto{\pgfqpoint{3.420908in}{2.247084in}}%
\pgfpathlineto{\pgfqpoint{3.412501in}{2.249583in}}%
\pgfpathlineto{\pgfqpoint{3.399499in}{2.250689in}}%
\pgfpathlineto{\pgfqpoint{3.384305in}{2.249671in}}%
\pgfpathlineto{\pgfqpoint{3.364985in}{2.246098in}}%
\pgfpathlineto{\pgfqpoint{3.341804in}{2.239342in}}%
\pgfpathlineto{\pgfqpoint{3.317109in}{2.229682in}}%
\pgfpathlineto{\pgfqpoint{3.291104in}{2.216986in}}%
\pgfpathlineto{\pgfqpoint{3.265928in}{2.202261in}}%
\pgfpathlineto{\pgfqpoint{3.239805in}{2.184361in}}%
\pgfpathlineto{\pgfqpoint{3.214775in}{2.164519in}}%
\pgfpathlineto{\pgfqpoint{3.190900in}{2.142893in}}%
\pgfpathlineto{\pgfqpoint{3.166657in}{2.117912in}}%
\pgfpathlineto{\pgfqpoint{3.143835in}{2.091233in}}%
\pgfpathlineto{\pgfqpoint{3.121079in}{2.061107in}}%
\pgfpathlineto{\pgfqpoint{3.099952in}{2.029463in}}%
\pgfpathlineto{\pgfqpoint{3.079251in}{1.994406in}}%
\pgfpathlineto{\pgfqpoint{3.059218in}{1.955915in}}%
\pgfpathlineto{\pgfqpoint{3.040058in}{1.914015in}}%
\pgfpathlineto{\pgfqpoint{3.022809in}{1.871041in}}%
\pgfpathlineto{\pgfqpoint{3.005790in}{1.822536in}}%
\pgfpathlineto{\pgfqpoint{2.990067in}{1.770819in}}%
\pgfpathlineto{\pgfqpoint{2.975708in}{1.715979in}}%
\pgfpathlineto{\pgfqpoint{2.962284in}{1.655680in}}%
\pgfpathlineto{\pgfqpoint{2.950496in}{1.592386in}}%
\pgfpathlineto{\pgfqpoint{2.940383in}{1.526185in}}%
\pgfpathlineto{\pgfqpoint{2.931745in}{1.454681in}}%
\pgfpathlineto{\pgfqpoint{2.925082in}{1.380399in}}%
\pgfpathlineto{\pgfqpoint{2.920647in}{1.305899in}}%
\pgfpathlineto{\pgfqpoint{2.918444in}{1.231270in}}%
\pgfpathlineto{\pgfqpoint{2.918545in}{1.159087in}}%
\pgfpathlineto{\pgfqpoint{2.920787in}{1.091931in}}%
\pgfpathlineto{\pgfqpoint{2.925177in}{1.027412in}}%
\pgfpathlineto{\pgfqpoint{2.931192in}{0.970580in}}%
\pgfpathlineto{\pgfqpoint{2.938760in}{0.919034in}}%
\pgfpathlineto{\pgfqpoint{2.947651in}{0.872852in}}%
\pgfpathlineto{\pgfqpoint{2.958213in}{0.829714in}}%
\pgfpathlineto{\pgfqpoint{2.969670in}{0.792114in}}%
\pgfpathlineto{\pgfqpoint{2.982463in}{0.757773in}}%
\pgfpathlineto{\pgfqpoint{2.996425in}{0.726812in}}%
\pgfpathlineto{\pgfqpoint{3.011299in}{0.699300in}}%
\pgfpathlineto{\pgfqpoint{3.026739in}{0.675225in}}%
\pgfpathlineto{\pgfqpoint{3.043828in}{0.652656in}}%
\pgfpathlineto{\pgfqpoint{3.062495in}{0.631788in}}%
\pgfpathlineto{\pgfqpoint{3.082602in}{0.612753in}}%
\pgfpathlineto{\pgfqpoint{3.103961in}{0.595592in}}%
\pgfpathlineto{\pgfqpoint{3.128268in}{0.579069in}}%
\pgfpathlineto{\pgfqpoint{3.153537in}{0.564554in}}%
\pgfpathlineto{\pgfqpoint{3.181571in}{0.550952in}}%
\pgfpathlineto{\pgfqpoint{3.214371in}{0.537647in}}%
\pgfpathlineto{\pgfqpoint{3.249846in}{0.525712in}}%
\pgfpathlineto{\pgfqpoint{3.290011in}{0.514571in}}%
\pgfpathlineto{\pgfqpoint{3.334820in}{0.504423in}}%
\pgfpathlineto{\pgfqpoint{3.386372in}{0.494999in}}%
\pgfpathlineto{\pgfqpoint{3.446798in}{0.486257in}}%
\pgfpathlineto{\pgfqpoint{3.518243in}{0.478282in}}%
\pgfpathlineto{\pgfqpoint{3.600685in}{0.471409in}}%
\pgfpathlineto{\pgfqpoint{3.696268in}{0.465713in}}%
\pgfpathlineto{\pgfqpoint{3.807144in}{0.461369in}}%
\pgfpathlineto{\pgfqpoint{3.933291in}{0.458719in}}%
\pgfpathlineto{\pgfqpoint{4.063808in}{0.458211in}}%
\pgfpathlineto{\pgfqpoint{4.187792in}{0.459914in}}%
\pgfpathlineto{\pgfqpoint{4.294335in}{0.463521in}}%
\pgfpathlineto{\pgfqpoint{4.381234in}{0.468574in}}%
\pgfpathlineto{\pgfqpoint{4.450636in}{0.474701in}}%
\pgfpathlineto{\pgfqpoint{4.506850in}{0.481799in}}%
\pgfpathlineto{\pgfqpoint{4.552009in}{0.489658in}}%
\pgfpathlineto{\pgfqpoint{4.588239in}{0.498115in}}%
\pgfpathlineto{\pgfqpoint{4.617656in}{0.507110in}}%
\pgfpathlineto{\pgfqpoint{4.642328in}{0.516843in}}%
\pgfpathlineto{\pgfqpoint{4.664194in}{0.527940in}}%
\pgfpathlineto{\pgfqpoint{4.681238in}{0.538945in}}%
\pgfpathlineto{\pgfqpoint{4.697164in}{0.551953in}}%
\pgfpathlineto{\pgfqpoint{4.710076in}{0.565289in}}%
\pgfpathlineto{\pgfqpoint{4.721578in}{0.580218in}}%
\pgfpathlineto{\pgfqpoint{4.731557in}{0.596521in}}%
\pgfpathlineto{\pgfqpoint{4.741000in}{0.616134in}}%
\pgfpathlineto{\pgfqpoint{4.749521in}{0.639027in}}%
\pgfpathlineto{\pgfqpoint{4.757522in}{0.667450in}}%
\pgfpathlineto{\pgfqpoint{4.764572in}{0.701345in}}%
\pgfpathlineto{\pgfqpoint{4.770840in}{0.743043in}}%
\pgfpathlineto{\pgfqpoint{4.776327in}{0.794934in}}%
\pgfpathlineto{\pgfqpoint{4.781278in}{0.864398in}}%
\pgfpathlineto{\pgfqpoint{4.785468in}{0.956371in}}%
\pgfpathlineto{\pgfqpoint{4.789000in}{1.085745in}}%
\pgfpathlineto{\pgfqpoint{4.791852in}{1.277385in}}%
\pgfpathlineto{\pgfqpoint{4.793959in}{1.581057in}}%
\pgfpathlineto{\pgfqpoint{4.794962in}{2.071429in}}%
\pgfpathlineto{\pgfqpoint{4.793967in}{2.559311in}}%
\pgfpathlineto{\pgfqpoint{4.791733in}{2.745981in}}%
\pgfpathlineto{\pgfqpoint{4.788955in}{2.818091in}}%
\pgfpathlineto{\pgfqpoint{4.785731in}{2.850227in}}%
\pgfpathlineto{\pgfqpoint{4.781879in}{2.867057in}}%
\pgfpathlineto{\pgfqpoint{4.777744in}{2.875780in}}%
\pgfpathlineto{\pgfqpoint{4.773097in}{2.880982in}}%
\pgfpathlineto{\pgfqpoint{4.767363in}{2.884504in}}%
\pgfpathlineto{\pgfqpoint{4.756853in}{2.887622in}}%
\pgfpathlineto{\pgfqpoint{4.739548in}{2.889639in}}%
\pgfpathlineto{\pgfqpoint{4.704762in}{2.890882in}}%
\pgfpathlineto{\pgfqpoint{4.602524in}{2.891538in}}%
\pgfpathlineto{\pgfqpoint{3.952100in}{2.891742in}}%
\pgfpathlineto{\pgfqpoint{0.617321in}{2.890753in}}%
\pgfpathlineto{\pgfqpoint{0.549910in}{2.888858in}}%
\pgfpathlineto{\pgfqpoint{0.521735in}{2.886179in}}%
\pgfpathlineto{\pgfqpoint{0.504666in}{2.882389in}}%
\pgfpathlineto{\pgfqpoint{0.494501in}{2.878011in}}%
\pgfpathlineto{\pgfqpoint{0.487180in}{2.872667in}}%
\pgfpathlineto{\pgfqpoint{0.481152in}{2.865519in}}%
\pgfpathlineto{\pgfqpoint{0.475664in}{2.854804in}}%
\pgfpathlineto{\pgfqpoint{0.471318in}{2.840737in}}%
\pgfpathlineto{\pgfqpoint{0.467301in}{2.818823in}}%
\pgfpathlineto{\pgfqpoint{0.463927in}{2.786700in}}%
\pgfpathlineto{\pgfqpoint{0.460918in}{2.734544in}}%
\pgfpathlineto{\pgfqpoint{0.458363in}{2.647473in}}%
\pgfpathlineto{\pgfqpoint{0.456575in}{2.523031in}}%
\pgfpathlineto{\pgfqpoint{0.456575in}{2.523031in}}%
\pgfusepath{stroke}%
\end{pgfscope}%
\begin{pgfscope}%
\pgfpathrectangle{\pgfqpoint{0.448634in}{0.402556in}}{\pgfqpoint{4.350661in}{2.489204in}} %
\pgfusepath{clip}%
\pgfsetrectcap%
\pgfsetroundjoin%
\pgfsetlinewidth{1.003750pt}%
\definecolor{currentstroke}{rgb}{0.839216,0.152941,0.156863}%
\pgfsetstrokecolor{currentstroke}%
\pgfsetdash{}{0pt}%
\pgfpathmoveto{\pgfqpoint{0.456424in}{1.370137in}}%
\pgfpathlineto{\pgfqpoint{0.459610in}{1.118755in}}%
\pgfpathlineto{\pgfqpoint{0.463695in}{0.962007in}}%
\pgfpathlineto{\pgfqpoint{0.468519in}{0.857610in}}%
\pgfpathlineto{\pgfqpoint{0.474082in}{0.783210in}}%
\pgfpathlineto{\pgfqpoint{0.480226in}{0.728906in}}%
\pgfpathlineto{\pgfqpoint{0.486970in}{0.687306in}}%
\pgfpathlineto{\pgfqpoint{0.494537in}{0.653558in}}%
\pgfpathlineto{\pgfqpoint{0.503107in}{0.625355in}}%
\pgfpathlineto{\pgfqpoint{0.512193in}{0.602749in}}%
\pgfpathlineto{\pgfqpoint{0.522200in}{0.583508in}}%
\pgfpathlineto{\pgfqpoint{0.534108in}{0.565743in}}%
\pgfpathlineto{\pgfqpoint{0.546263in}{0.551507in}}%
\pgfpathlineto{\pgfqpoint{0.559728in}{0.538907in}}%
\pgfpathlineto{\pgfqpoint{0.576130in}{0.526693in}}%
\pgfpathlineto{\pgfqpoint{0.595483in}{0.515351in}}%
\pgfpathlineto{\pgfqpoint{0.617681in}{0.505147in}}%
\pgfpathlineto{\pgfqpoint{0.642568in}{0.496153in}}%
\pgfpathlineto{\pgfqpoint{0.672126in}{0.487778in}}%
\pgfpathlineto{\pgfqpoint{0.708443in}{0.479824in}}%
\pgfpathlineto{\pgfqpoint{0.753649in}{0.472325in}}%
\pgfpathlineto{\pgfqpoint{0.807718in}{0.465660in}}%
\pgfpathlineto{\pgfqpoint{0.877116in}{0.459475in}}%
\pgfpathlineto{\pgfqpoint{0.961828in}{0.454230in}}%
\pgfpathlineto{\pgfqpoint{1.068351in}{0.449916in}}%
\pgfpathlineto{\pgfqpoint{1.201018in}{0.446839in}}%
\pgfpathlineto{\pgfqpoint{1.357637in}{0.445481in}}%
\pgfpathlineto{\pgfqpoint{1.525135in}{0.446232in}}%
\pgfpathlineto{\pgfqpoint{1.686088in}{0.449142in}}%
\pgfpathlineto{\pgfqpoint{1.823074in}{0.453747in}}%
\pgfpathlineto{\pgfqpoint{1.938245in}{0.459764in}}%
\pgfpathlineto{\pgfqpoint{2.031582in}{0.466759in}}%
\pgfpathlineto{\pgfqpoint{2.109580in}{0.474745in}}%
\pgfpathlineto{\pgfqpoint{2.174384in}{0.483535in}}%
\pgfpathlineto{\pgfqpoint{2.228139in}{0.492940in}}%
\pgfpathlineto{\pgfqpoint{2.275119in}{0.503356in}}%
\pgfpathlineto{\pgfqpoint{2.315282in}{0.514501in}}%
\pgfpathlineto{\pgfqpoint{2.350698in}{0.526659in}}%
\pgfpathlineto{\pgfqpoint{2.381320in}{0.539536in}}%
\pgfpathlineto{\pgfqpoint{2.407164in}{0.552659in}}%
\pgfpathlineto{\pgfqpoint{2.430226in}{0.566639in}}%
\pgfpathlineto{\pgfqpoint{2.452282in}{0.582602in}}%
\pgfpathlineto{\pgfqpoint{2.471391in}{0.599069in}}%
\pgfpathlineto{\pgfqpoint{2.489240in}{0.617293in}}%
\pgfpathlineto{\pgfqpoint{2.505678in}{0.637180in}}%
\pgfpathlineto{\pgfqpoint{2.520620in}{0.658557in}}%
\pgfpathlineto{\pgfqpoint{2.535213in}{0.683314in}}%
\pgfpathlineto{\pgfqpoint{2.549115in}{0.711484in}}%
\pgfpathlineto{\pgfqpoint{2.562091in}{0.743004in}}%
\pgfpathlineto{\pgfqpoint{2.574020in}{0.777751in}}%
\pgfpathlineto{\pgfqpoint{2.585502in}{0.817970in}}%
\pgfpathlineto{\pgfqpoint{2.596809in}{0.866038in}}%
\pgfpathlineto{\pgfqpoint{2.607562in}{0.921948in}}%
\pgfpathlineto{\pgfqpoint{2.617925in}{0.988098in}}%
\pgfpathlineto{\pgfqpoint{2.627958in}{1.066918in}}%
\pgfpathlineto{\pgfqpoint{2.637941in}{1.163320in}}%
\pgfpathlineto{\pgfqpoint{2.648424in}{1.287199in}}%
\pgfpathlineto{\pgfqpoint{2.660103in}{1.453438in}}%
\pgfpathlineto{\pgfqpoint{2.674773in}{1.696801in}}%
\pgfpathlineto{\pgfqpoint{2.687716in}{1.945279in}}%
\pgfpathlineto{\pgfqpoint{2.692670in}{2.079573in}}%
\pgfpathlineto{\pgfqpoint{2.693829in}{2.166682in}}%
\pgfpathlineto{\pgfqpoint{2.692565in}{2.233870in}}%
\pgfpathlineto{\pgfqpoint{2.689436in}{2.286015in}}%
\pgfpathlineto{\pgfqpoint{2.684859in}{2.327999in}}%
\pgfpathlineto{\pgfqpoint{2.678725in}{2.364664in}}%
\pgfpathlineto{\pgfqpoint{2.671356in}{2.395897in}}%
\pgfpathlineto{\pgfqpoint{2.662489in}{2.423981in}}%
\pgfpathlineto{\pgfqpoint{2.652361in}{2.448778in}}%
\pgfpathlineto{\pgfqpoint{2.641365in}{2.470245in}}%
\pgfpathlineto{\pgfqpoint{2.628643in}{2.490425in}}%
\pgfpathlineto{\pgfqpoint{2.614279in}{2.509106in}}%
\pgfpathlineto{\pgfqpoint{2.598443in}{2.526159in}}%
\pgfpathlineto{\pgfqpoint{2.579590in}{2.543005in}}%
\pgfpathlineto{\pgfqpoint{2.559532in}{2.557923in}}%
\pgfpathlineto{\pgfqpoint{2.536602in}{2.572183in}}%
\pgfpathlineto{\pgfqpoint{2.510850in}{2.585538in}}%
\pgfpathlineto{\pgfqpoint{2.482360in}{2.597837in}}%
\pgfpathlineto{\pgfqpoint{2.449134in}{2.609683in}}%
\pgfpathlineto{\pgfqpoint{2.411184in}{2.620696in}}%
\pgfpathlineto{\pgfqpoint{2.368552in}{2.630606in}}%
\pgfpathlineto{\pgfqpoint{2.321294in}{2.639221in}}%
\pgfpathlineto{\pgfqpoint{2.269467in}{2.646399in}}%
\pgfpathlineto{\pgfqpoint{2.210954in}{2.652193in}}%
\pgfpathlineto{\pgfqpoint{2.147967in}{2.656153in}}%
\pgfpathlineto{\pgfqpoint{2.080556in}{2.658135in}}%
\pgfpathlineto{\pgfqpoint{2.010948in}{2.657971in}}%
\pgfpathlineto{\pgfqpoint{1.939195in}{2.655572in}}%
\pgfpathlineto{\pgfqpoint{1.867527in}{2.650913in}}%
\pgfpathlineto{\pgfqpoint{1.798171in}{2.644140in}}%
\pgfpathlineto{\pgfqpoint{1.733341in}{2.635606in}}%
\pgfpathlineto{\pgfqpoint{1.673075in}{2.625521in}}%
\pgfpathlineto{\pgfqpoint{1.615274in}{2.613610in}}%
\pgfpathlineto{\pgfqpoint{1.562133in}{2.600402in}}%
\pgfpathlineto{\pgfqpoint{1.513681in}{2.586139in}}%
\pgfpathlineto{\pgfqpoint{1.467862in}{2.570344in}}%
\pgfpathlineto{\pgfqpoint{1.426794in}{2.553923in}}%
\pgfpathlineto{\pgfqpoint{1.388447in}{2.536289in}}%
\pgfpathlineto{\pgfqpoint{1.352878in}{2.517566in}}%
\pgfpathlineto{\pgfqpoint{1.320128in}{2.497922in}}%
\pgfpathlineto{\pgfqpoint{1.288379in}{2.476236in}}%
\pgfpathlineto{\pgfqpoint{1.259592in}{2.453861in}}%
\pgfpathlineto{\pgfqpoint{1.232050in}{2.429520in}}%
\pgfpathlineto{\pgfqpoint{1.207527in}{2.404898in}}%
\pgfpathlineto{\pgfqpoint{1.184409in}{2.378557in}}%
\pgfpathlineto{\pgfqpoint{1.162828in}{2.350561in}}%
\pgfpathlineto{\pgfqpoint{1.142891in}{2.321011in}}%
\pgfpathlineto{\pgfqpoint{1.124675in}{2.290041in}}%
\pgfpathlineto{\pgfqpoint{1.108225in}{2.257802in}}%
\pgfpathlineto{\pgfqpoint{1.092639in}{2.222199in}}%
\pgfpathlineto{\pgfqpoint{1.079059in}{2.185535in}}%
\pgfpathlineto{\pgfqpoint{1.067443in}{2.147998in}}%
\pgfpathlineto{\pgfqpoint{1.057187in}{2.107347in}}%
\pgfpathlineto{\pgfqpoint{1.049004in}{2.066086in}}%
\pgfpathlineto{\pgfqpoint{1.042513in}{2.021906in}}%
\pgfpathlineto{\pgfqpoint{1.038177in}{1.977382in}}%
\pgfpathlineto{\pgfqpoint{1.035866in}{1.930167in}}%
\pgfpathlineto{\pgfqpoint{1.035826in}{1.882878in}}%
\pgfpathlineto{\pgfqpoint{1.038031in}{1.835656in}}%
\pgfpathlineto{\pgfqpoint{1.042474in}{1.788641in}}%
\pgfpathlineto{\pgfqpoint{1.049176in}{1.741979in}}%
\pgfpathlineto{\pgfqpoint{1.057644in}{1.698239in}}%
\pgfpathlineto{\pgfqpoint{1.068221in}{1.655105in}}%
\pgfpathlineto{\pgfqpoint{1.080962in}{1.612745in}}%
\pgfpathlineto{\pgfqpoint{1.095031in}{1.573617in}}%
\pgfpathlineto{\pgfqpoint{1.111115in}{1.535520in}}%
\pgfpathlineto{\pgfqpoint{1.128118in}{1.500775in}}%
\pgfpathlineto{\pgfqpoint{1.146930in}{1.467274in}}%
\pgfpathlineto{\pgfqpoint{1.167531in}{1.435181in}}%
\pgfpathlineto{\pgfqpoint{1.189874in}{1.404652in}}%
\pgfpathlineto{\pgfqpoint{1.213884in}{1.375828in}}%
\pgfpathlineto{\pgfqpoint{1.237817in}{1.350457in}}%
\pgfpathlineto{\pgfqpoint{1.264748in}{1.325237in}}%
\pgfpathlineto{\pgfqpoint{1.292991in}{1.301972in}}%
\pgfpathlineto{\pgfqpoint{1.322398in}{1.280678in}}%
\pgfpathlineto{\pgfqpoint{1.352820in}{1.261340in}}%
\pgfpathlineto{\pgfqpoint{1.386095in}{1.242889in}}%
\pgfpathlineto{\pgfqpoint{1.420190in}{1.226516in}}%
\pgfpathlineto{\pgfqpoint{1.457024in}{1.211329in}}%
\pgfpathlineto{\pgfqpoint{1.496554in}{1.197536in}}%
\pgfpathlineto{\pgfqpoint{1.538719in}{1.185287in}}%
\pgfpathlineto{\pgfqpoint{1.583441in}{1.174641in}}%
\pgfpathlineto{\pgfqpoint{1.634929in}{1.164775in}}%
\pgfpathlineto{\pgfqpoint{1.706063in}{1.153745in}}%
\pgfpathlineto{\pgfqpoint{1.768492in}{1.143417in}}%
\pgfpathlineto{\pgfqpoint{1.796122in}{1.136567in}}%
\pgfpathlineto{\pgfqpoint{1.812683in}{1.130481in}}%
\pgfpathlineto{\pgfqpoint{1.824471in}{1.124102in}}%
\pgfpathlineto{\pgfqpoint{1.833209in}{1.116741in}}%
\pgfpathlineto{\pgfqpoint{1.838498in}{1.108890in}}%
\pgfpathlineto{\pgfqpoint{1.840588in}{1.101849in}}%
\pgfpathlineto{\pgfqpoint{1.840619in}{1.094412in}}%
\pgfpathlineto{\pgfqpoint{1.837931in}{1.084986in}}%
\pgfpathlineto{\pgfqpoint{1.833246in}{1.076615in}}%
\pgfpathlineto{\pgfqpoint{1.825819in}{1.067542in}}%
\pgfpathlineto{\pgfqpoint{1.813813in}{1.056850in}}%
\pgfpathlineto{\pgfqpoint{1.798819in}{1.046763in}}%
\pgfpathlineto{\pgfqpoint{1.781016in}{1.037462in}}%
\pgfpathlineto{\pgfqpoint{1.758447in}{1.028391in}}%
\pgfpathlineto{\pgfqpoint{1.733203in}{1.020815in}}%
\pgfpathlineto{\pgfqpoint{1.705410in}{1.014872in}}%
\pgfpathlineto{\pgfqpoint{1.675178in}{1.010714in}}%
\pgfpathlineto{\pgfqpoint{1.642610in}{1.008507in}}%
\pgfpathlineto{\pgfqpoint{1.607809in}{1.008432in}}%
\pgfpathlineto{\pgfqpoint{1.570886in}{1.010691in}}%
\pgfpathlineto{\pgfqpoint{1.534118in}{1.015181in}}%
\pgfpathlineto{\pgfqpoint{1.495454in}{1.022233in}}%
\pgfpathlineto{\pgfqpoint{1.457161in}{1.031563in}}%
\pgfpathlineto{\pgfqpoint{1.419337in}{1.043132in}}%
\pgfpathlineto{\pgfqpoint{1.382089in}{1.056929in}}%
\pgfpathlineto{\pgfqpoint{1.347544in}{1.072019in}}%
\pgfpathlineto{\pgfqpoint{1.313727in}{1.089133in}}%
\pgfpathlineto{\pgfqpoint{1.280762in}{1.108299in}}%
\pgfpathlineto{\pgfqpoint{1.248782in}{1.129536in}}%
\pgfpathlineto{\pgfqpoint{1.219708in}{1.151422in}}%
\pgfpathlineto{\pgfqpoint{1.191752in}{1.175138in}}%
\pgfpathlineto{\pgfqpoint{1.165031in}{1.200649in}}%
\pgfpathlineto{\pgfqpoint{1.139653in}{1.227898in}}%
\pgfpathlineto{\pgfqpoint{1.115714in}{1.256800in}}%
\pgfpathlineto{\pgfqpoint{1.093288in}{1.287251in}}%
\pgfpathlineto{\pgfqpoint{1.071178in}{1.321163in}}%
\pgfpathlineto{\pgfqpoint{1.050868in}{1.356520in}}%
\pgfpathlineto{\pgfqpoint{1.032365in}{1.393152in}}%
\pgfpathlineto{\pgfqpoint{1.014718in}{1.433142in}}%
\pgfpathlineto{\pgfqpoint{0.999024in}{1.474185in}}%
\pgfpathlineto{\pgfqpoint{0.984506in}{1.518461in}}%
\pgfpathlineto{\pgfqpoint{0.972010in}{1.563537in}}%
\pgfpathlineto{\pgfqpoint{0.960944in}{1.611678in}}%
\pgfpathlineto{\pgfqpoint{0.951530in}{1.662824in}}%
\pgfpathlineto{\pgfqpoint{0.944286in}{1.714431in}}%
\pgfpathlineto{\pgfqpoint{0.938950in}{1.768847in}}%
\pgfpathlineto{\pgfqpoint{0.935870in}{1.823491in}}%
\pgfpathlineto{\pgfqpoint{0.935034in}{1.878240in}}%
\pgfpathlineto{\pgfqpoint{0.936466in}{1.932973in}}%
\pgfpathlineto{\pgfqpoint{0.940005in}{1.985084in}}%
\pgfpathlineto{\pgfqpoint{0.945759in}{2.036935in}}%
\pgfpathlineto{\pgfqpoint{0.953410in}{2.085938in}}%
\pgfpathlineto{\pgfqpoint{0.962764in}{2.132000in}}%
\pgfpathlineto{\pgfqpoint{0.974287in}{2.177414in}}%
\pgfpathlineto{\pgfqpoint{0.987332in}{2.219653in}}%
\pgfpathlineto{\pgfqpoint{1.001667in}{2.258654in}}%
\pgfpathlineto{\pgfqpoint{1.018051in}{2.296583in}}%
\pgfpathlineto{\pgfqpoint{1.035401in}{2.331101in}}%
\pgfpathlineto{\pgfqpoint{1.054650in}{2.364275in}}%
\pgfpathlineto{\pgfqpoint{1.074406in}{2.393984in}}%
\pgfpathlineto{\pgfqpoint{1.095771in}{2.422197in}}%
\pgfpathlineto{\pgfqpoint{1.118662in}{2.448797in}}%
\pgfpathlineto{\pgfqpoint{1.142967in}{2.473701in}}%
\pgfpathlineto{\pgfqpoint{1.168550in}{2.496867in}}%
\pgfpathlineto{\pgfqpoint{1.197085in}{2.519662in}}%
\pgfpathlineto{\pgfqpoint{1.226727in}{2.540526in}}%
\pgfpathlineto{\pgfqpoint{1.259242in}{2.560673in}}%
\pgfpathlineto{\pgfqpoint{1.294612in}{2.579881in}}%
\pgfpathlineto{\pgfqpoint{1.332792in}{2.597982in}}%
\pgfpathlineto{\pgfqpoint{1.373719in}{2.614859in}}%
\pgfpathlineto{\pgfqpoint{1.417319in}{2.630445in}}%
\pgfpathlineto{\pgfqpoint{1.465632in}{2.645312in}}%
\pgfpathlineto{\pgfqpoint{1.518640in}{2.659204in}}%
\pgfpathlineto{\pgfqpoint{1.576309in}{2.671929in}}%
\pgfpathlineto{\pgfqpoint{1.638597in}{2.683344in}}%
\pgfpathlineto{\pgfqpoint{1.705462in}{2.693343in}}%
\pgfpathlineto{\pgfqpoint{1.779027in}{2.702064in}}%
\pgfpathlineto{\pgfqpoint{1.857097in}{2.709077in}}%
\pgfpathlineto{\pgfqpoint{1.939633in}{2.714280in}}%
\pgfpathlineto{\pgfqpoint{2.026598in}{2.717513in}}%
\pgfpathlineto{\pgfqpoint{2.113605in}{2.718523in}}%
\pgfpathlineto{\pgfqpoint{2.198435in}{2.717303in}}%
\pgfpathlineto{\pgfqpoint{2.278866in}{2.713929in}}%
\pgfpathlineto{\pgfqpoint{2.352678in}{2.708598in}}%
\pgfpathlineto{\pgfqpoint{2.417657in}{2.701709in}}%
\pgfpathlineto{\pgfqpoint{2.473770in}{2.693630in}}%
\pgfpathlineto{\pgfqpoint{2.523140in}{2.684368in}}%
\pgfpathlineto{\pgfqpoint{2.565726in}{2.674202in}}%
\pgfpathlineto{\pgfqpoint{2.601510in}{2.663544in}}%
\pgfpathlineto{\pgfqpoint{2.632577in}{2.652142in}}%
\pgfpathlineto{\pgfqpoint{2.658899in}{2.640331in}}%
\pgfpathlineto{\pgfqpoint{2.682438in}{2.627436in}}%
\pgfpathlineto{\pgfqpoint{2.703062in}{2.613571in}}%
\pgfpathlineto{\pgfqpoint{2.720674in}{2.598978in}}%
\pgfpathlineto{\pgfqpoint{2.735263in}{2.584053in}}%
\pgfpathlineto{\pgfqpoint{2.748320in}{2.567377in}}%
\pgfpathlineto{\pgfqpoint{2.759553in}{2.549046in}}%
\pgfpathlineto{\pgfqpoint{2.768788in}{2.529306in}}%
\pgfpathlineto{\pgfqpoint{2.776017in}{2.508498in}}%
\pgfpathlineto{\pgfqpoint{2.781884in}{2.484540in}}%
\pgfpathlineto{\pgfqpoint{2.786102in}{2.457597in}}%
\pgfpathlineto{\pgfqpoint{2.788720in}{2.425384in}}%
\pgfpathlineto{\pgfqpoint{2.789427in}{2.388061in}}%
\pgfpathlineto{\pgfqpoint{2.787962in}{2.340801in}}%
\pgfpathlineto{\pgfqpoint{2.783672in}{2.278768in}}%
\pgfpathlineto{\pgfqpoint{2.774289in}{2.179783in}}%
\pgfpathlineto{\pgfqpoint{2.743611in}{1.868119in}}%
\pgfpathlineto{\pgfqpoint{2.730112in}{1.702060in}}%
\pgfpathlineto{\pgfqpoint{2.717287in}{1.515949in}}%
\pgfpathlineto{\pgfqpoint{2.702602in}{1.267597in}}%
\pgfpathlineto{\pgfqpoint{2.684434in}{0.964630in}}%
\pgfpathlineto{\pgfqpoint{2.675374in}{0.850600in}}%
\pgfpathlineto{\pgfqpoint{2.667030in}{0.771523in}}%
\pgfpathlineto{\pgfqpoint{2.658752in}{0.712543in}}%
\pgfpathlineto{\pgfqpoint{2.650176in}{0.666284in}}%
\pgfpathlineto{\pgfqpoint{2.640820in}{0.627931in}}%
\pgfpathlineto{\pgfqpoint{2.631145in}{0.597534in}}%
\pgfpathlineto{\pgfqpoint{2.621004in}{0.572745in}}%
\pgfpathlineto{\pgfqpoint{2.609856in}{0.551383in}}%
\pgfpathlineto{\pgfqpoint{2.598042in}{0.533534in}}%
\pgfpathlineto{\pgfqpoint{2.584496in}{0.517378in}}%
\pgfpathlineto{\pgfqpoint{2.571109in}{0.504669in}}%
\pgfpathlineto{\pgfqpoint{2.554789in}{0.492313in}}%
\pgfpathlineto{\pgfqpoint{2.537457in}{0.481914in}}%
\pgfpathlineto{\pgfqpoint{2.517374in}{0.472367in}}%
\pgfpathlineto{\pgfqpoint{2.492542in}{0.463178in}}%
\pgfpathlineto{\pgfqpoint{2.462979in}{0.454833in}}%
\pgfpathlineto{\pgfqpoint{2.428766in}{0.447542in}}%
\pgfpathlineto{\pgfqpoint{2.385671in}{0.440735in}}%
\pgfpathlineto{\pgfqpoint{2.331557in}{0.434581in}}%
\pgfpathlineto{\pgfqpoint{2.262115in}{0.429077in}}%
\pgfpathlineto{\pgfqpoint{2.170851in}{0.424236in}}%
\pgfpathlineto{\pgfqpoint{2.049086in}{0.420134in}}%
\pgfpathlineto{\pgfqpoint{1.879436in}{0.416783in}}%
\pgfpathlineto{\pgfqpoint{1.640159in}{0.414418in}}%
\pgfpathlineto{\pgfqpoint{1.322562in}{0.413569in}}%
\pgfpathlineto{\pgfqpoint{1.020194in}{0.414850in}}%
\pgfpathlineto{\pgfqpoint{0.822256in}{0.417715in}}%
\pgfpathlineto{\pgfqpoint{0.704835in}{0.421430in}}%
\pgfpathlineto{\pgfqpoint{0.630976in}{0.425829in}}%
\pgfpathlineto{\pgfqpoint{0.583316in}{0.430734in}}%
\pgfpathlineto{\pgfqpoint{0.551033in}{0.436123in}}%
\pgfpathlineto{\pgfqpoint{0.527708in}{0.442189in}}%
\pgfpathlineto{\pgfqpoint{0.511250in}{0.448625in}}%
\pgfpathlineto{\pgfqpoint{0.499549in}{0.455216in}}%
\pgfpathlineto{\pgfqpoint{0.488916in}{0.463841in}}%
\pgfpathlineto{\pgfqpoint{0.481322in}{0.472730in}}%
\pgfpathlineto{\pgfqpoint{0.474078in}{0.485127in}}%
\pgfpathlineto{\pgfqpoint{0.468753in}{0.498748in}}%
\pgfpathlineto{\pgfqpoint{0.463870in}{0.517848in}}%
\pgfpathlineto{\pgfqpoint{0.459679in}{0.544796in}}%
\pgfpathlineto{\pgfqpoint{0.456386in}{0.581938in}}%
\pgfpathlineto{\pgfqpoint{0.453731in}{0.639106in}}%
\pgfpathlineto{\pgfqpoint{0.451681in}{0.736155in}}%
\pgfpathlineto{\pgfqpoint{0.450220in}{0.927815in}}%
\pgfpathlineto{\pgfqpoint{0.449345in}{1.403252in}}%
\pgfpathlineto{\pgfqpoint{0.449543in}{2.682703in}}%
\pgfpathlineto{\pgfqpoint{0.451011in}{2.856932in}}%
\pgfpathlineto{\pgfqpoint{0.452802in}{2.879219in}}%
\pgfpathlineto{\pgfqpoint{0.455188in}{2.886108in}}%
\pgfpathlineto{\pgfqpoint{0.458626in}{2.889028in}}%
\pgfpathlineto{\pgfqpoint{0.464996in}{2.890553in}}%
\pgfpathlineto{\pgfqpoint{0.482376in}{2.891423in}}%
\pgfpathlineto{\pgfqpoint{0.565038in}{2.891729in}}%
\pgfpathlineto{\pgfqpoint{2.733842in}{2.891760in}}%
\pgfpathlineto{\pgfqpoint{4.789510in}{2.890885in}}%
\pgfpathlineto{\pgfqpoint{4.793727in}{2.889730in}}%
\pgfpathlineto{\pgfqpoint{4.795481in}{2.888307in}}%
\pgfpathlineto{\pgfqpoint{4.797106in}{2.881145in}}%
\pgfpathlineto{\pgfqpoint{4.797997in}{2.858771in}}%
\pgfpathlineto{\pgfqpoint{4.798039in}{2.856283in}}%
\pgfpathlineto{\pgfqpoint{4.798039in}{2.856283in}}%
\pgfusepath{stroke}%
\end{pgfscope}%
\begin{pgfscope}%
\pgfpathrectangle{\pgfqpoint{0.448634in}{0.402556in}}{\pgfqpoint{4.350661in}{2.489204in}} %
\pgfusepath{clip}%
\pgfsetrectcap%
\pgfsetroundjoin%
\pgfsetlinewidth{1.003750pt}%
\definecolor{currentstroke}{rgb}{0.839216,0.152941,0.156863}%
\pgfsetstrokecolor{currentstroke}%
\pgfsetdash{}{0pt}%
\pgfpathmoveto{\pgfqpoint{3.428772in}{0.402610in}}%
\pgfpathlineto{\pgfqpoint{2.806632in}{0.403760in}}%
\pgfpathlineto{\pgfqpoint{2.769692in}{0.405578in}}%
\pgfpathlineto{\pgfqpoint{2.754632in}{0.408064in}}%
\pgfpathlineto{\pgfqpoint{2.746391in}{0.411198in}}%
\pgfpathlineto{\pgfqpoint{2.740943in}{0.415265in}}%
\pgfpathlineto{\pgfqpoint{2.736785in}{0.420985in}}%
\pgfpathlineto{\pgfqpoint{2.733281in}{0.430071in}}%
\pgfpathlineto{\pgfqpoint{2.730449in}{0.444637in}}%
\pgfpathlineto{\pgfqpoint{2.728238in}{0.469392in}}%
\pgfpathlineto{\pgfqpoint{2.726470in}{0.519131in}}%
\pgfpathlineto{\pgfqpoint{2.725711in}{0.613715in}}%
\pgfpathlineto{\pgfqpoint{2.726842in}{0.768039in}}%
\pgfpathlineto{\pgfqpoint{2.730557in}{0.962149in}}%
\pgfpathlineto{\pgfqpoint{2.736611in}{1.158671in}}%
\pgfpathlineto{\pgfqpoint{2.744092in}{1.327719in}}%
\pgfpathlineto{\pgfqpoint{2.753202in}{1.484190in}}%
\pgfpathlineto{\pgfqpoint{2.763257in}{1.620610in}}%
\pgfpathlineto{\pgfqpoint{2.776118in}{1.764216in}}%
\pgfpathlineto{\pgfqpoint{2.788914in}{1.877777in}}%
\pgfpathlineto{\pgfqpoint{2.805748in}{2.005741in}}%
\pgfpathlineto{\pgfqpoint{2.821176in}{2.101198in}}%
\pgfpathlineto{\pgfqpoint{2.838360in}{2.193719in}}%
\pgfpathlineto{\pgfqpoint{2.859135in}{2.292966in}}%
\pgfpathlineto{\pgfqpoint{2.887209in}{2.425960in}}%
\pgfpathlineto{\pgfqpoint{2.896992in}{2.479560in}}%
\pgfpathlineto{\pgfqpoint{2.901543in}{2.516524in}}%
\pgfpathlineto{\pgfqpoint{2.902849in}{2.543855in}}%
\pgfpathlineto{\pgfqpoint{2.901958in}{2.566223in}}%
\pgfpathlineto{\pgfqpoint{2.899152in}{2.585863in}}%
\pgfpathlineto{\pgfqpoint{2.894794in}{2.602546in}}%
\pgfpathlineto{\pgfqpoint{2.888484in}{2.618388in}}%
\pgfpathlineto{\pgfqpoint{2.880257in}{2.633033in}}%
\pgfpathlineto{\pgfqpoint{2.870348in}{2.646246in}}%
\pgfpathlineto{\pgfqpoint{2.857400in}{2.659531in}}%
\pgfpathlineto{\pgfqpoint{2.843189in}{2.671010in}}%
\pgfpathlineto{\pgfqpoint{2.824238in}{2.683209in}}%
\pgfpathlineto{\pgfqpoint{2.802413in}{2.694419in}}%
\pgfpathlineto{\pgfqpoint{2.775809in}{2.705369in}}%
\pgfpathlineto{\pgfqpoint{2.744461in}{2.715715in}}%
\pgfpathlineto{\pgfqpoint{2.708436in}{2.725252in}}%
\pgfpathlineto{\pgfqpoint{2.665655in}{2.734289in}}%
\pgfpathlineto{\pgfqpoint{2.613991in}{2.742869in}}%
\pgfpathlineto{\pgfqpoint{2.553459in}{2.750589in}}%
\pgfpathlineto{\pgfqpoint{2.481920in}{2.757365in}}%
\pgfpathlineto{\pgfqpoint{2.399398in}{2.762839in}}%
\pgfpathlineto{\pgfqpoint{2.310269in}{2.766482in}}%
\pgfpathlineto{\pgfqpoint{2.175416in}{2.768725in}}%
\pgfpathlineto{\pgfqpoint{2.066653in}{2.767942in}}%
\pgfpathlineto{\pgfqpoint{1.953571in}{2.764859in}}%
\pgfpathlineto{\pgfqpoint{1.851429in}{2.759759in}}%
\pgfpathlineto{\pgfqpoint{1.745051in}{2.752169in}}%
\pgfpathlineto{\pgfqpoint{1.658374in}{2.743454in}}%
\pgfpathlineto{\pgfqpoint{1.580552in}{2.733461in}}%
\pgfpathlineto{\pgfqpoint{1.490058in}{2.719338in}}%
\pgfpathlineto{\pgfqpoint{1.417231in}{2.704698in}}%
\pgfpathlineto{\pgfqpoint{1.361992in}{2.690818in}}%
\pgfpathlineto{\pgfqpoint{1.311460in}{2.675819in}}%
\pgfpathlineto{\pgfqpoint{1.265667in}{2.659924in}}%
\pgfpathlineto{\pgfqpoint{1.222575in}{2.642586in}}%
\pgfpathlineto{\pgfqpoint{1.184324in}{2.624682in}}%
\pgfpathlineto{\pgfqpoint{1.148892in}{2.605623in}}%
\pgfpathlineto{\pgfqpoint{1.116332in}{2.585573in}}%
\pgfpathlineto{\pgfqpoint{1.092327in}{2.568512in}}%
\pgfpathlineto{\pgfqpoint{1.079760in}{2.558686in}}%
\pgfpathlineto{\pgfqpoint{1.051544in}{2.535379in}}%
\pgfpathlineto{\pgfqpoint{1.026312in}{2.511712in}}%
\pgfpathlineto{\pgfqpoint{1.002399in}{2.486318in}}%
\pgfpathlineto{\pgfqpoint{0.979913in}{2.459269in}}%
\pgfpathlineto{\pgfqpoint{0.958934in}{2.430678in}}%
\pgfpathlineto{\pgfqpoint{0.938264in}{2.398644in}}%
\pgfpathlineto{\pgfqpoint{0.923047in}{2.371385in}}%
\pgfpathlineto{\pgfqpoint{0.904513in}{2.334774in}}%
\pgfpathlineto{\pgfqpoint{0.887854in}{2.297001in}}%
\pgfpathlineto{\pgfqpoint{0.872131in}{2.255972in}}%
\pgfpathlineto{\pgfqpoint{0.857508in}{2.211741in}}%
\pgfpathlineto{\pgfqpoint{0.844762in}{2.166757in}}%
\pgfpathlineto{\pgfqpoint{0.838624in}{2.140306in}}%
\pgfpathlineto{\pgfqpoint{0.826982in}{2.087194in}}%
\pgfpathlineto{\pgfqpoint{0.816322in}{2.028716in}}%
\pgfpathlineto{\pgfqpoint{0.810087in}{1.984495in}}%
\pgfpathlineto{\pgfqpoint{0.808026in}{1.967238in}}%
\pgfpathlineto{\pgfqpoint{0.800076in}{1.898141in}}%
\pgfpathlineto{\pgfqpoint{0.793713in}{1.823823in}}%
\pgfpathlineto{\pgfqpoint{0.788799in}{1.741875in}}%
\pgfpathlineto{\pgfqpoint{0.786199in}{1.677225in}}%
\pgfpathlineto{\pgfqpoint{0.776951in}{1.453481in}}%
\pgfpathlineto{\pgfqpoint{0.773280in}{1.418894in}}%
\pgfpathlineto{\pgfqpoint{0.768298in}{1.389582in}}%
\pgfpathlineto{\pgfqpoint{0.762752in}{1.368108in}}%
\pgfpathlineto{\pgfqpoint{0.756722in}{1.352123in}}%
\pgfpathlineto{\pgfqpoint{0.749752in}{1.339519in}}%
\pgfpathlineto{\pgfqpoint{0.742201in}{1.330599in}}%
\pgfpathlineto{\pgfqpoint{0.734854in}{1.325312in}}%
\pgfpathlineto{\pgfqpoint{0.726558in}{1.322419in}}%
\pgfpathlineto{\pgfqpoint{0.717884in}{1.322223in}}%
\pgfpathlineto{\pgfqpoint{0.709412in}{1.324411in}}%
\pgfpathlineto{\pgfqpoint{0.699548in}{1.329604in}}%
\pgfpathlineto{\pgfqpoint{0.688894in}{1.338203in}}%
\pgfpathlineto{\pgfqpoint{0.677907in}{1.350248in}}%
\pgfpathlineto{\pgfqpoint{0.666886in}{1.365647in}}%
\pgfpathlineto{\pgfqpoint{0.654913in}{1.386417in}}%
\pgfpathlineto{\pgfqpoint{0.642574in}{1.412730in}}%
\pgfpathlineto{\pgfqpoint{0.630328in}{1.444629in}}%
\pgfpathlineto{\pgfqpoint{0.618504in}{1.482081in}}%
\pgfpathlineto{\pgfqpoint{0.608613in}{1.520256in}}%
\pgfpathlineto{\pgfqpoint{0.590203in}{1.612445in}}%
\pgfpathlineto{\pgfqpoint{0.581848in}{1.668884in}}%
\pgfpathlineto{\pgfqpoint{0.573137in}{1.740376in}}%
\pgfpathlineto{\pgfqpoint{0.567062in}{1.807213in}}%
\pgfpathlineto{\pgfqpoint{0.560532in}{1.896510in}}%
\pgfpathlineto{\pgfqpoint{0.555526in}{1.995910in}}%
\pgfpathlineto{\pgfqpoint{0.552564in}{2.097908in}}%
\pgfpathlineto{\pgfqpoint{0.551526in}{2.204935in}}%
\pgfpathlineto{\pgfqpoint{0.552728in}{2.309470in}}%
\pgfpathlineto{\pgfqpoint{0.556011in}{2.403981in}}%
\pgfpathlineto{\pgfqpoint{0.560953in}{2.483430in}}%
\pgfpathlineto{\pgfqpoint{0.567303in}{2.550240in}}%
\pgfpathlineto{\pgfqpoint{0.574928in}{2.606817in}}%
\pgfpathlineto{\pgfqpoint{0.582988in}{2.650657in}}%
\pgfpathlineto{\pgfqpoint{0.592756in}{2.691452in}}%
\pgfpathlineto{\pgfqpoint{0.602650in}{2.721756in}}%
\pgfpathlineto{\pgfqpoint{0.612983in}{2.746441in}}%
\pgfpathlineto{\pgfqpoint{0.624292in}{2.767692in}}%
\pgfpathlineto{\pgfqpoint{0.636231in}{2.785433in}}%
\pgfpathlineto{\pgfqpoint{0.649892in}{2.801461in}}%
\pgfpathlineto{\pgfqpoint{0.663386in}{2.814020in}}%
\pgfpathlineto{\pgfqpoint{0.679842in}{2.826135in}}%
\pgfpathlineto{\pgfqpoint{0.697326in}{2.836197in}}%
\pgfpathlineto{\pgfqpoint{0.715574in}{2.844285in}}%
\pgfpathlineto{\pgfqpoint{0.738439in}{2.852335in}}%
\pgfpathlineto{\pgfqpoint{0.765983in}{2.859639in}}%
\pgfpathlineto{\pgfqpoint{0.800300in}{2.866256in}}%
\pgfpathlineto{\pgfqpoint{0.841340in}{2.871832in}}%
\pgfpathlineto{\pgfqpoint{0.895547in}{2.876803in}}%
\pgfpathlineto{\pgfqpoint{0.969413in}{2.881069in}}%
\pgfpathlineto{\pgfqpoint{1.071608in}{2.884501in}}%
\pgfpathlineto{\pgfqpoint{1.219512in}{2.887074in}}%
\pgfpathlineto{\pgfqpoint{1.471844in}{2.889091in}}%
\pgfpathlineto{\pgfqpoint{1.956941in}{2.890384in}}%
\pgfpathlineto{\pgfqpoint{3.096814in}{2.890781in}}%
\pgfpathlineto{\pgfqpoint{3.995224in}{2.889388in}}%
\pgfpathlineto{\pgfqpoint{4.275833in}{2.887011in}}%
\pgfpathlineto{\pgfqpoint{4.412847in}{2.883743in}}%
\pgfpathlineto{\pgfqpoint{4.491081in}{2.879810in}}%
\pgfpathlineto{\pgfqpoint{4.543127in}{2.875163in}}%
\pgfpathlineto{\pgfqpoint{4.579810in}{2.869841in}}%
\pgfpathlineto{\pgfqpoint{4.607580in}{2.863763in}}%
\pgfpathlineto{\pgfqpoint{4.630623in}{2.856424in}}%
\pgfpathlineto{\pgfqpoint{4.648833in}{2.848228in}}%
\pgfpathlineto{\pgfqpoint{4.664136in}{2.838773in}}%
\pgfpathlineto{\pgfqpoint{4.676470in}{2.828576in}}%
\pgfpathlineto{\pgfqpoint{4.687502in}{2.816585in}}%
\pgfpathlineto{\pgfqpoint{4.697051in}{2.803027in}}%
\pgfpathlineto{\pgfqpoint{4.706194in}{2.786098in}}%
\pgfpathlineto{\pgfqpoint{4.714508in}{2.765827in}}%
\pgfpathlineto{\pgfqpoint{4.722462in}{2.740013in}}%
\pgfpathlineto{\pgfqpoint{4.729577in}{2.708703in}}%
\pgfpathlineto{\pgfqpoint{4.736162in}{2.669601in}}%
\pgfpathlineto{\pgfqpoint{4.742419in}{2.617826in}}%
\pgfpathlineto{\pgfqpoint{4.747859in}{2.553410in}}%
\pgfpathlineto{\pgfqpoint{4.752661in}{2.468958in}}%
\pgfpathlineto{\pgfqpoint{4.756610in}{2.359528in}}%
\pgfpathlineto{\pgfqpoint{4.759416in}{2.217681in}}%
\pgfpathlineto{\pgfqpoint{4.760596in}{2.043444in}}%
\pgfpathlineto{\pgfqpoint{4.759662in}{1.851779in}}%
\pgfpathlineto{\pgfqpoint{4.756587in}{1.667613in}}%
\pgfpathlineto{\pgfqpoint{4.751596in}{1.503428in}}%
\pgfpathlineto{\pgfqpoint{4.745410in}{1.374185in}}%
\pgfpathlineto{\pgfqpoint{4.738113in}{1.267479in}}%
\pgfpathlineto{\pgfqpoint{4.729621in}{1.175896in}}%
\pgfpathlineto{\pgfqpoint{4.720762in}{1.104428in}}%
\pgfpathlineto{\pgfqpoint{4.711045in}{1.043204in}}%
\pgfpathlineto{\pgfqpoint{4.700364in}{0.989829in}}%
\pgfpathlineto{\pgfqpoint{4.689055in}{0.944345in}}%
\pgfpathlineto{\pgfqpoint{4.676881in}{0.904394in}}%
\pgfpathlineto{\pgfqpoint{4.676095in}{0.902073in}}%
\pgfpathlineto{\pgfqpoint{4.676095in}{0.902073in}}%
\pgfusepath{stroke}%
\end{pgfscope}%
\begin{pgfscope}%
\pgfpathrectangle{\pgfqpoint{0.448634in}{0.402556in}}{\pgfqpoint{4.350661in}{2.489204in}} %
\pgfusepath{clip}%
\pgfsetrectcap%
\pgfsetroundjoin%
\pgfsetlinewidth{1.003750pt}%
\definecolor{currentstroke}{rgb}{0.839216,0.152941,0.156863}%
\pgfsetstrokecolor{currentstroke}%
\pgfsetdash{}{0pt}%
\pgfpathmoveto{\pgfqpoint{2.795520in}{1.982745in}}%
\pgfpathlineto{\pgfqpoint{2.781780in}{1.874357in}}%
\pgfpathlineto{\pgfqpoint{2.769351in}{1.758234in}}%
\pgfpathlineto{\pgfqpoint{2.758095in}{1.631942in}}%
\pgfpathlineto{\pgfqpoint{2.747786in}{1.490551in}}%
\pgfpathlineto{\pgfqpoint{2.738644in}{1.334082in}}%
\pgfpathlineto{\pgfqpoint{2.730580in}{1.157591in}}%
\pgfpathlineto{\pgfqpoint{2.723334in}{0.948663in}}%
\pgfpathlineto{\pgfqpoint{2.709783in}{0.530788in}}%
\pgfpathlineto{\pgfqpoint{2.705868in}{0.488716in}}%
\pgfpathlineto{\pgfqpoint{2.701769in}{0.464281in}}%
\pgfpathlineto{\pgfqpoint{2.697021in}{0.447744in}}%
\pgfpathlineto{\pgfqpoint{2.691859in}{0.436812in}}%
\pgfpathlineto{\pgfqpoint{2.686245in}{0.429229in}}%
\pgfpathlineto{\pgfqpoint{2.679348in}{0.423188in}}%
\pgfpathlineto{\pgfqpoint{2.669540in}{0.417856in}}%
\pgfpathlineto{\pgfqpoint{2.656987in}{0.413810in}}%
\pgfpathlineto{\pgfqpoint{2.637654in}{0.410337in}}%
\pgfpathlineto{\pgfqpoint{2.607297in}{0.407617in}}%
\pgfpathlineto{\pgfqpoint{2.555121in}{0.405574in}}%
\pgfpathlineto{\pgfqpoint{2.450714in}{0.404139in}}%
\pgfpathlineto{\pgfqpoint{2.176624in}{0.403275in}}%
\pgfpathlineto{\pgfqpoint{1.130290in}{0.402953in}}%
\pgfpathlineto{\pgfqpoint{0.516849in}{0.404175in}}%
\pgfpathlineto{\pgfqpoint{0.466848in}{0.405970in}}%
\pgfpathlineto{\pgfqpoint{0.456130in}{0.407931in}}%
\pgfpathlineto{\pgfqpoint{0.452340in}{0.410303in}}%
\pgfpathlineto{\pgfqpoint{0.450346in}{0.414662in}}%
\pgfpathlineto{\pgfqpoint{0.449266in}{0.424524in}}%
\pgfpathlineto{\pgfqpoint{0.448771in}{0.464344in}}%
\pgfpathlineto{\pgfqpoint{0.448640in}{0.850171in}}%
\pgfpathlineto{\pgfqpoint{0.448653in}{2.891318in}}%
\pgfpathlineto{\pgfqpoint{0.448653in}{2.891318in}}%
\pgfusepath{stroke}%
\end{pgfscope}%
\begin{pgfscope}%
\pgfpathrectangle{\pgfqpoint{0.448634in}{0.402556in}}{\pgfqpoint{4.350661in}{2.489204in}} %
\pgfusepath{clip}%
\pgfsetrectcap%
\pgfsetroundjoin%
\pgfsetlinewidth{1.003750pt}%
\definecolor{currentstroke}{rgb}{0.839216,0.152941,0.156863}%
\pgfsetstrokecolor{currentstroke}%
\pgfsetdash{}{0pt}%
\pgfpathmoveto{\pgfqpoint{3.428190in}{0.402586in}}%
\pgfpathlineto{\pgfqpoint{2.782122in}{0.403702in}}%
\pgfpathlineto{\pgfqpoint{2.753907in}{0.405674in}}%
\pgfpathlineto{\pgfqpoint{2.743329in}{0.408444in}}%
\pgfpathlineto{\pgfqpoint{2.737718in}{0.412189in}}%
\pgfpathlineto{\pgfqpoint{2.733668in}{0.417995in}}%
\pgfpathlineto{\pgfqpoint{2.730649in}{0.427308in}}%
\pgfpathlineto{\pgfqpoint{2.728388in}{0.442005in}}%
\pgfpathlineto{\pgfqpoint{2.726544in}{0.471795in}}%
\pgfpathlineto{\pgfqpoint{2.725216in}{0.534004in}}%
\pgfpathlineto{\pgfqpoint{2.725169in}{0.655973in}}%
\pgfpathlineto{\pgfqpoint{2.727377in}{0.832687in}}%
\pgfpathlineto{\pgfqpoint{2.732259in}{1.041703in}}%
\pgfpathlineto{\pgfqpoint{2.738851in}{1.223257in}}%
\pgfpathlineto{\pgfqpoint{2.747078in}{1.389766in}}%
\pgfpathlineto{\pgfqpoint{2.756608in}{1.538718in}}%
\pgfpathlineto{\pgfqpoint{2.768955in}{1.694887in}}%
\pgfpathlineto{\pgfqpoint{2.781228in}{1.816045in}}%
\pgfpathlineto{\pgfqpoint{2.794401in}{1.924525in}}%
\pgfpathlineto{\pgfqpoint{2.812737in}{2.054723in}}%
\pgfpathlineto{\pgfqpoint{2.828774in}{2.147513in}}%
\pgfpathlineto{\pgfqpoint{2.847382in}{2.242225in}}%
\pgfpathlineto{\pgfqpoint{2.895818in}{2.479700in}}%
\pgfpathlineto{\pgfqpoint{2.900204in}{2.516690in}}%
\pgfpathlineto{\pgfqpoint{2.901346in}{2.544030in}}%
\pgfpathlineto{\pgfqpoint{2.900292in}{2.566389in}}%
\pgfpathlineto{\pgfqpoint{2.897335in}{2.586000in}}%
\pgfpathlineto{\pgfqpoint{2.892836in}{2.602634in}}%
\pgfpathlineto{\pgfqpoint{2.886394in}{2.618406in}}%
\pgfpathlineto{\pgfqpoint{2.878058in}{2.632970in}}%
\pgfpathlineto{\pgfqpoint{2.868065in}{2.646101in}}%
\pgfpathlineto{\pgfqpoint{2.855050in}{2.659301in}}%
\pgfpathlineto{\pgfqpoint{2.840801in}{2.670717in}}%
\pgfpathlineto{\pgfqpoint{2.821822in}{2.682861in}}%
\pgfpathlineto{\pgfqpoint{2.799980in}{2.694026in}}%
\pgfpathlineto{\pgfqpoint{2.773366in}{2.704944in}}%
\pgfpathlineto{\pgfqpoint{2.742012in}{2.715266in}}%
\pgfpathlineto{\pgfqpoint{2.705983in}{2.724786in}}%
\pgfpathlineto{\pgfqpoint{2.663200in}{2.733811in}}%
\pgfpathlineto{\pgfqpoint{2.611535in}{2.742379in}}%
\pgfpathlineto{\pgfqpoint{2.551002in}{2.750090in}}%
\pgfpathlineto{\pgfqpoint{2.481632in}{2.756682in}}%
\pgfpathlineto{\pgfqpoint{2.399112in}{2.762200in}}%
\pgfpathlineto{\pgfqpoint{2.309985in}{2.765886in}}%
\pgfpathlineto{\pgfqpoint{2.188184in}{2.768097in}}%
\pgfpathlineto{\pgfqpoint{2.081595in}{2.767619in}}%
\pgfpathlineto{\pgfqpoint{1.968506in}{2.764840in}}%
\pgfpathlineto{\pgfqpoint{1.864180in}{2.759918in}}%
\pgfpathlineto{\pgfqpoint{1.757786in}{2.752593in}}%
\pgfpathlineto{\pgfqpoint{1.671087in}{2.744171in}}%
\pgfpathlineto{\pgfqpoint{1.591076in}{2.734193in}}%
\pgfpathlineto{\pgfqpoint{1.502689in}{2.720717in}}%
\pgfpathlineto{\pgfqpoint{1.427655in}{2.706083in}}%
\pgfpathlineto{\pgfqpoint{1.372350in}{2.692544in}}%
\pgfpathlineto{\pgfqpoint{1.321734in}{2.677921in}}%
\pgfpathlineto{\pgfqpoint{1.273765in}{2.661664in}}%
\pgfpathlineto{\pgfqpoint{1.230567in}{2.644672in}}%
\pgfpathlineto{\pgfqpoint{1.192197in}{2.627106in}}%
\pgfpathlineto{\pgfqpoint{1.156620in}{2.608403in}}%
\pgfpathlineto{\pgfqpoint{1.123890in}{2.588716in}}%
\pgfpathlineto{\pgfqpoint{1.095883in}{2.569568in}}%
\pgfpathlineto{\pgfqpoint{1.063936in}{2.543701in}}%
\pgfpathlineto{\pgfqpoint{1.038217in}{2.520732in}}%
\pgfpathlineto{\pgfqpoint{1.013766in}{2.496016in}}%
\pgfpathlineto{\pgfqpoint{0.990704in}{2.469610in}}%
\pgfpathlineto{\pgfqpoint{0.969124in}{2.441612in}}%
\pgfpathlineto{\pgfqpoint{0.949082in}{2.412154in}}%
\pgfpathlineto{\pgfqpoint{0.930604in}{2.381387in}}%
\pgfpathlineto{\pgfqpoint{0.906555in}{2.334052in}}%
\pgfpathlineto{\pgfqpoint{0.889925in}{2.296262in}}%
\pgfpathlineto{\pgfqpoint{0.874241in}{2.255213in}}%
\pgfpathlineto{\pgfqpoint{0.859667in}{2.210961in}}%
\pgfpathlineto{\pgfqpoint{0.846985in}{2.165954in}}%
\pgfpathlineto{\pgfqpoint{0.839633in}{2.134715in}}%
\pgfpathlineto{\pgfqpoint{0.828238in}{2.081532in}}%
\pgfpathlineto{\pgfqpoint{0.817866in}{2.022986in}}%
\pgfpathlineto{\pgfqpoint{0.810784in}{1.971352in}}%
\pgfpathlineto{\pgfqpoint{0.802845in}{1.902253in}}%
\pgfpathlineto{\pgfqpoint{0.796554in}{1.827927in}}%
\pgfpathlineto{\pgfqpoint{0.791696in}{1.743480in}}%
\pgfpathlineto{\pgfqpoint{0.787773in}{1.621595in}}%
\pgfpathlineto{\pgfqpoint{0.785407in}{1.522064in}}%
\pgfpathlineto{\pgfqpoint{0.785407in}{1.522064in}}%
\pgfusepath{stroke}%
\end{pgfscope}%
\begin{pgfscope}%
\pgfpathrectangle{\pgfqpoint{0.448634in}{0.402556in}}{\pgfqpoint{4.350661in}{2.489204in}} %
\pgfusepath{clip}%
\pgfsetrectcap%
\pgfsetroundjoin%
\pgfsetlinewidth{1.003750pt}%
\definecolor{currentstroke}{rgb}{0.839216,0.152941,0.156863}%
\pgfsetstrokecolor{currentstroke}%
\pgfsetdash{}{0pt}%
\pgfpathmoveto{\pgfqpoint{2.028735in}{0.425754in}}%
\pgfpathlineto{\pgfqpoint{1.878677in}{0.421879in}}%
\pgfpathlineto{\pgfqpoint{1.676387in}{0.418997in}}%
\pgfpathlineto{\pgfqpoint{1.413176in}{0.417558in}}%
\pgfpathlineto{\pgfqpoint{1.134735in}{0.418204in}}%
\pgfpathlineto{\pgfqpoint{0.921565in}{0.420769in}}%
\pgfpathlineto{\pgfqpoint{0.782384in}{0.424523in}}%
\pgfpathlineto{\pgfqpoint{0.693283in}{0.428974in}}%
\pgfpathlineto{\pgfqpoint{0.632541in}{0.434091in}}%
\pgfpathlineto{\pgfqpoint{0.591492in}{0.439564in}}%
\pgfpathlineto{\pgfqpoint{0.561503in}{0.445595in}}%
\pgfpathlineto{\pgfqpoint{0.538349in}{0.452466in}}%
\pgfpathlineto{\pgfqpoint{0.522042in}{0.459394in}}%
\pgfpathlineto{\pgfqpoint{0.508540in}{0.467420in}}%
\pgfpathlineto{\pgfqpoint{0.497973in}{0.476161in}}%
\pgfpathlineto{\pgfqpoint{0.488790in}{0.486750in}}%
\pgfpathlineto{\pgfqpoint{0.481284in}{0.498948in}}%
\pgfpathlineto{\pgfqpoint{0.474590in}{0.514580in}}%
\pgfpathlineto{\pgfqpoint{0.469106in}{0.533467in}}%
\pgfpathlineto{\pgfqpoint{0.464439in}{0.557771in}}%
\pgfpathlineto{\pgfqpoint{0.460297in}{0.592289in}}%
\pgfpathlineto{\pgfqpoint{0.456856in}{0.641912in}}%
\pgfpathlineto{\pgfqpoint{0.454122in}{0.716520in}}%
\pgfpathlineto{\pgfqpoint{0.451978in}{0.843444in}}%
\pgfpathlineto{\pgfqpoint{0.450459in}{1.087380in}}%
\pgfpathlineto{\pgfqpoint{0.449596in}{1.657406in}}%
\pgfpathlineto{\pgfqpoint{0.450150in}{2.687936in}}%
\pgfpathlineto{\pgfqpoint{0.451781in}{2.839761in}}%
\pgfpathlineto{\pgfqpoint{0.453975in}{2.872003in}}%
\pgfpathlineto{\pgfqpoint{0.456339in}{2.881553in}}%
\pgfpathlineto{\pgfqpoint{0.458888in}{2.885549in}}%
\pgfpathlineto{\pgfqpoint{0.462554in}{2.888171in}}%
\pgfpathlineto{\pgfqpoint{0.471046in}{2.890205in}}%
\pgfpathlineto{\pgfqpoint{0.490597in}{2.891263in}}%
\pgfpathlineto{\pgfqpoint{0.564556in}{2.891692in}}%
\pgfpathlineto{\pgfqpoint{1.569559in}{2.891759in}}%
\pgfpathlineto{\pgfqpoint{4.784679in}{2.890785in}}%
\pgfpathlineto{\pgfqpoint{4.791005in}{2.889098in}}%
\pgfpathlineto{\pgfqpoint{4.793910in}{2.885555in}}%
\pgfpathlineto{\pgfqpoint{4.795579in}{2.878366in}}%
\pgfpathlineto{\pgfqpoint{4.796850in}{2.858513in}}%
\pgfpathlineto{\pgfqpoint{4.796850in}{2.858513in}}%
\pgfusepath{stroke}%
\end{pgfscope}%
\begin{pgfscope}%
\pgfpathrectangle{\pgfqpoint{0.448634in}{0.402556in}}{\pgfqpoint{4.350661in}{2.489204in}} %
\pgfusepath{clip}%
\pgfsetrectcap%
\pgfsetroundjoin%
\pgfsetlinewidth{1.003750pt}%
\definecolor{currentstroke}{rgb}{0.580392,0.403922,0.741176}%
\pgfsetstrokecolor{currentstroke}%
\pgfsetdash{}{0pt}%
\pgfpathmoveto{\pgfqpoint{0.448634in}{2.896245in}}%
\pgfpathlineto{\pgfqpoint{0.448593in}{0.407043in}}%
\pgfpathlineto{\pgfqpoint{0.448593in}{0.407043in}}%
\pgfusepath{stroke}%
\end{pgfscope}%
\begin{pgfscope}%
\pgfpathrectangle{\pgfqpoint{0.448634in}{0.402556in}}{\pgfqpoint{4.350661in}{2.489204in}} %
\pgfusepath{clip}%
\pgfsetrectcap%
\pgfsetroundjoin%
\pgfsetlinewidth{1.003750pt}%
\definecolor{currentstroke}{rgb}{0.580392,0.403922,0.741176}%
\pgfsetstrokecolor{currentstroke}%
\pgfsetdash{}{0pt}%
\pgfpathmoveto{\pgfqpoint{0.576852in}{1.760819in}}%
\pgfpathlineto{\pgfqpoint{0.569393in}{1.840012in}}%
\pgfpathlineto{\pgfqpoint{0.563208in}{1.929341in}}%
\pgfpathlineto{\pgfqpoint{0.558592in}{2.028766in}}%
\pgfpathlineto{\pgfqpoint{0.555985in}{2.133267in}}%
\pgfpathlineto{\pgfqpoint{0.555565in}{2.237810in}}%
\pgfpathlineto{\pgfqpoint{0.557371in}{2.337354in}}%
\pgfpathlineto{\pgfqpoint{0.561096in}{2.424369in}}%
\pgfpathlineto{\pgfqpoint{0.566403in}{2.498793in}}%
\pgfpathlineto{\pgfqpoint{0.572908in}{2.560572in}}%
\pgfpathlineto{\pgfqpoint{0.580458in}{2.612121in}}%
\pgfpathlineto{\pgfqpoint{0.589086in}{2.655818in}}%
\pgfpathlineto{\pgfqpoint{0.598406in}{2.691592in}}%
\pgfpathlineto{\pgfqpoint{0.608613in}{2.721759in}}%
\pgfpathlineto{\pgfqpoint{0.619241in}{2.746280in}}%
\pgfpathlineto{\pgfqpoint{0.630817in}{2.767341in}}%
\pgfpathlineto{\pgfqpoint{0.642976in}{2.784886in}}%
\pgfpathlineto{\pgfqpoint{0.656814in}{2.800714in}}%
\pgfpathlineto{\pgfqpoint{0.672198in}{2.814550in}}%
\pgfpathlineto{\pgfqpoint{0.688854in}{2.826302in}}%
\pgfpathlineto{\pgfqpoint{0.706462in}{2.836077in}}%
\pgfpathlineto{\pgfqpoint{0.726805in}{2.844876in}}%
\pgfpathlineto{\pgfqpoint{0.751867in}{2.853203in}}%
\pgfpathlineto{\pgfqpoint{0.781633in}{2.860548in}}%
\pgfpathlineto{\pgfqpoint{0.818169in}{2.867054in}}%
\pgfpathlineto{\pgfqpoint{0.863582in}{2.872685in}}%
\pgfpathlineto{\pgfqpoint{0.922162in}{2.877518in}}%
\pgfpathlineto{\pgfqpoint{1.000392in}{2.881567in}}%
\pgfpathlineto{\pgfqpoint{1.111295in}{2.884881in}}%
\pgfpathlineto{\pgfqpoint{1.274430in}{2.887367in}}%
\pgfpathlineto{\pgfqpoint{1.552867in}{2.889263in}}%
\pgfpathlineto{\pgfqpoint{2.107575in}{2.890457in}}%
\pgfpathlineto{\pgfqpoint{3.343162in}{2.890573in}}%
\pgfpathlineto{\pgfqpoint{4.043617in}{2.888941in}}%
\pgfpathlineto{\pgfqpoint{4.289418in}{2.886404in}}%
\pgfpathlineto{\pgfqpoint{4.413376in}{2.883093in}}%
\pgfpathlineto{\pgfqpoint{4.489426in}{2.878997in}}%
\pgfpathlineto{\pgfqpoint{4.541453in}{2.874081in}}%
\pgfpathlineto{\pgfqpoint{4.578102in}{2.868470in}}%
\pgfpathlineto{\pgfqpoint{4.605820in}{2.862092in}}%
\pgfpathlineto{\pgfqpoint{4.626727in}{2.855245in}}%
\pgfpathlineto{\pgfqpoint{4.644927in}{2.847018in}}%
\pgfpathlineto{\pgfqpoint{4.660242in}{2.837589in}}%
\pgfpathlineto{\pgfqpoint{4.672625in}{2.827468in}}%
\pgfpathlineto{\pgfqpoint{4.683752in}{2.815592in}}%
\pgfpathlineto{\pgfqpoint{4.693407in}{2.802134in}}%
\pgfpathlineto{\pgfqpoint{4.702741in}{2.785342in}}%
\pgfpathlineto{\pgfqpoint{4.711278in}{2.765193in}}%
\pgfpathlineto{\pgfqpoint{4.719483in}{2.739483in}}%
\pgfpathlineto{\pgfqpoint{4.726294in}{2.710656in}}%
\pgfpathlineto{\pgfqpoint{4.733260in}{2.671642in}}%
\pgfpathlineto{\pgfqpoint{4.739604in}{2.622395in}}%
\pgfpathlineto{\pgfqpoint{4.745236in}{2.560503in}}%
\pgfpathlineto{\pgfqpoint{4.750164in}{2.481051in}}%
\pgfpathlineto{\pgfqpoint{4.754367in}{2.376617in}}%
\pgfpathlineto{\pgfqpoint{4.757443in}{2.242248in}}%
\pgfpathlineto{\pgfqpoint{4.758977in}{2.075482in}}%
\pgfpathlineto{\pgfqpoint{4.758447in}{1.888794in}}%
\pgfpathlineto{\pgfqpoint{4.755756in}{1.707109in}}%
\pgfpathlineto{\pgfqpoint{4.750925in}{1.532956in}}%
\pgfpathlineto{\pgfqpoint{4.744786in}{1.398725in}}%
\pgfpathlineto{\pgfqpoint{4.737575in}{1.289514in}}%
\pgfpathlineto{\pgfqpoint{4.728714in}{1.190468in}}%
\pgfpathlineto{\pgfqpoint{4.719653in}{1.116520in}}%
\pgfpathlineto{\pgfqpoint{4.710036in}{1.055275in}}%
\pgfpathlineto{\pgfqpoint{4.699504in}{1.001860in}}%
\pgfpathlineto{\pgfqpoint{4.689041in}{0.958689in}}%
\pgfpathlineto{\pgfqpoint{4.677220in}{0.918599in}}%
\pgfpathlineto{\pgfqpoint{4.664034in}{0.881748in}}%
\pgfpathlineto{\pgfqpoint{4.650584in}{0.850490in}}%
\pgfpathlineto{\pgfqpoint{4.636303in}{0.822568in}}%
\pgfpathlineto{\pgfqpoint{4.620207in}{0.795973in}}%
\pgfpathlineto{\pgfqpoint{4.603640in}{0.772900in}}%
\pgfpathlineto{\pgfqpoint{4.585488in}{0.751445in}}%
\pgfpathlineto{\pgfqpoint{4.565874in}{0.731748in}}%
\pgfpathlineto{\pgfqpoint{4.544964in}{0.713878in}}%
\pgfpathlineto{\pgfqpoint{4.522958in}{0.697823in}}%
\pgfpathlineto{\pgfqpoint{4.496157in}{0.681289in}}%
\pgfpathlineto{\pgfqpoint{4.470397in}{0.667952in}}%
\pgfpathlineto{\pgfqpoint{4.439961in}{0.654509in}}%
\pgfpathlineto{\pgfqpoint{4.406841in}{0.642281in}}%
\pgfpathlineto{\pgfqpoint{4.369009in}{0.630748in}}%
\pgfpathlineto{\pgfqpoint{4.326489in}{0.620225in}}%
\pgfpathlineto{\pgfqpoint{4.279327in}{0.610948in}}%
\pgfpathlineto{\pgfqpoint{4.227576in}{0.603084in}}%
\pgfpathlineto{\pgfqpoint{4.173450in}{0.597062in}}%
\pgfpathlineto{\pgfqpoint{4.110511in}{0.592202in}}%
\pgfpathlineto{\pgfqpoint{4.047471in}{0.589536in}}%
\pgfpathlineto{\pgfqpoint{3.977867in}{0.588623in}}%
\pgfpathlineto{\pgfqpoint{3.906093in}{0.589933in}}%
\pgfpathlineto{\pgfqpoint{3.834377in}{0.593496in}}%
\pgfpathlineto{\pgfqpoint{3.767120in}{0.599067in}}%
\pgfpathlineto{\pgfqpoint{3.704364in}{0.606392in}}%
\pgfpathlineto{\pgfqpoint{3.678516in}{0.610510in}}%
\pgfpathlineto{\pgfqpoint{3.620438in}{0.620500in}}%
\pgfpathlineto{\pgfqpoint{3.586319in}{0.628207in}}%
\pgfpathlineto{\pgfqpoint{3.495241in}{0.652428in}}%
\pgfpathlineto{\pgfqpoint{3.451528in}{0.667583in}}%
\pgfpathlineto{\pgfqpoint{3.408538in}{0.685220in}}%
\pgfpathlineto{\pgfqpoint{3.374594in}{0.702001in}}%
\pgfpathlineto{\pgfqpoint{3.345407in}{0.718682in}}%
\pgfpathlineto{\pgfqpoint{3.315236in}{0.738520in}}%
\pgfpathlineto{\pgfqpoint{3.288127in}{0.759290in}}%
\pgfpathlineto{\pgfqpoint{3.264004in}{0.780551in}}%
\pgfpathlineto{\pgfqpoint{3.241208in}{0.803648in}}%
\pgfpathlineto{\pgfqpoint{3.219894in}{0.828530in}}%
\pgfpathlineto{\pgfqpoint{3.200189in}{0.855091in}}%
\pgfpathlineto{\pgfqpoint{3.182177in}{0.883182in}}%
\pgfpathlineto{\pgfqpoint{3.165906in}{0.912633in}}%
\pgfpathlineto{\pgfqpoint{3.150351in}{0.945448in}}%
\pgfpathlineto{\pgfqpoint{3.136682in}{0.979345in}}%
\pgfpathlineto{\pgfqpoint{3.124073in}{1.016460in}}%
\pgfpathlineto{\pgfqpoint{3.112834in}{1.056769in}}%
\pgfpathlineto{\pgfqpoint{3.103046in}{1.100146in}}%
\pgfpathlineto{\pgfqpoint{3.095343in}{1.144071in}}%
\pgfpathlineto{\pgfqpoint{3.089208in}{1.190837in}}%
\pgfpathlineto{\pgfqpoint{3.084595in}{1.242838in}}%
\pgfpathlineto{\pgfqpoint{3.082137in}{1.295031in}}%
\pgfpathlineto{\pgfqpoint{3.081687in}{1.349787in}}%
\pgfpathlineto{\pgfqpoint{3.083451in}{1.406998in}}%
\pgfpathlineto{\pgfqpoint{3.087181in}{1.461589in}}%
\pgfpathlineto{\pgfqpoint{3.093485in}{1.520887in}}%
\pgfpathlineto{\pgfqpoint{3.101823in}{1.577334in}}%
\pgfpathlineto{\pgfqpoint{3.111930in}{1.630856in}}%
\pgfpathlineto{\pgfqpoint{3.124690in}{1.686208in}}%
\pgfpathlineto{\pgfqpoint{3.139178in}{1.738395in}}%
\pgfpathlineto{\pgfqpoint{3.155145in}{1.787366in}}%
\pgfpathlineto{\pgfqpoint{3.172353in}{1.833084in}}%
\pgfpathlineto{\pgfqpoint{3.191618in}{1.877716in}}%
\pgfpathlineto{\pgfqpoint{3.214026in}{1.923261in}}%
\pgfpathlineto{\pgfqpoint{3.236214in}{1.963157in}}%
\pgfpathlineto{\pgfqpoint{3.260178in}{2.001684in}}%
\pgfpathlineto{\pgfqpoint{3.285814in}{2.038776in}}%
\pgfpathlineto{\pgfqpoint{3.314415in}{2.076285in}}%
\pgfpathlineto{\pgfqpoint{3.348944in}{2.117711in}}%
\pgfpathlineto{\pgfqpoint{3.417133in}{2.198022in}}%
\pgfpathlineto{\pgfqpoint{3.426053in}{2.212128in}}%
\pgfpathlineto{\pgfqpoint{3.430798in}{2.223297in}}%
\pgfpathlineto{\pgfqpoint{3.432034in}{2.230603in}}%
\pgfpathlineto{\pgfqpoint{3.430773in}{2.237856in}}%
\pgfpathlineto{\pgfqpoint{3.426621in}{2.243526in}}%
\pgfpathlineto{\pgfqpoint{3.420908in}{2.247084in}}%
\pgfpathlineto{\pgfqpoint{3.412501in}{2.249583in}}%
\pgfpathlineto{\pgfqpoint{3.399499in}{2.250689in}}%
\pgfpathlineto{\pgfqpoint{3.384305in}{2.249671in}}%
\pgfpathlineto{\pgfqpoint{3.364985in}{2.246098in}}%
\pgfpathlineto{\pgfqpoint{3.341804in}{2.239342in}}%
\pgfpathlineto{\pgfqpoint{3.317109in}{2.229682in}}%
\pgfpathlineto{\pgfqpoint{3.291104in}{2.216986in}}%
\pgfpathlineto{\pgfqpoint{3.265928in}{2.202261in}}%
\pgfpathlineto{\pgfqpoint{3.239805in}{2.184361in}}%
\pgfpathlineto{\pgfqpoint{3.214775in}{2.164519in}}%
\pgfpathlineto{\pgfqpoint{3.190900in}{2.142893in}}%
\pgfpathlineto{\pgfqpoint{3.166657in}{2.117912in}}%
\pgfpathlineto{\pgfqpoint{3.143835in}{2.091233in}}%
\pgfpathlineto{\pgfqpoint{3.121079in}{2.061107in}}%
\pgfpathlineto{\pgfqpoint{3.099952in}{2.029463in}}%
\pgfpathlineto{\pgfqpoint{3.079251in}{1.994406in}}%
\pgfpathlineto{\pgfqpoint{3.059218in}{1.955915in}}%
\pgfpathlineto{\pgfqpoint{3.040058in}{1.914015in}}%
\pgfpathlineto{\pgfqpoint{3.022809in}{1.871041in}}%
\pgfpathlineto{\pgfqpoint{3.005790in}{1.822536in}}%
\pgfpathlineto{\pgfqpoint{2.990067in}{1.770819in}}%
\pgfpathlineto{\pgfqpoint{2.975708in}{1.715979in}}%
\pgfpathlineto{\pgfqpoint{2.962284in}{1.655680in}}%
\pgfpathlineto{\pgfqpoint{2.950496in}{1.592386in}}%
\pgfpathlineto{\pgfqpoint{2.940383in}{1.526185in}}%
\pgfpathlineto{\pgfqpoint{2.931745in}{1.454681in}}%
\pgfpathlineto{\pgfqpoint{2.925082in}{1.380399in}}%
\pgfpathlineto{\pgfqpoint{2.920647in}{1.305899in}}%
\pgfpathlineto{\pgfqpoint{2.918444in}{1.231270in}}%
\pgfpathlineto{\pgfqpoint{2.918545in}{1.159087in}}%
\pgfpathlineto{\pgfqpoint{2.920787in}{1.091931in}}%
\pgfpathlineto{\pgfqpoint{2.925177in}{1.027412in}}%
\pgfpathlineto{\pgfqpoint{2.931192in}{0.970580in}}%
\pgfpathlineto{\pgfqpoint{2.938760in}{0.919034in}}%
\pgfpathlineto{\pgfqpoint{2.947651in}{0.872852in}}%
\pgfpathlineto{\pgfqpoint{2.958213in}{0.829714in}}%
\pgfpathlineto{\pgfqpoint{2.969670in}{0.792114in}}%
\pgfpathlineto{\pgfqpoint{2.982463in}{0.757773in}}%
\pgfpathlineto{\pgfqpoint{2.996425in}{0.726812in}}%
\pgfpathlineto{\pgfqpoint{3.011299in}{0.699300in}}%
\pgfpathlineto{\pgfqpoint{3.026739in}{0.675225in}}%
\pgfpathlineto{\pgfqpoint{3.043828in}{0.652656in}}%
\pgfpathlineto{\pgfqpoint{3.062495in}{0.631788in}}%
\pgfpathlineto{\pgfqpoint{3.082602in}{0.612753in}}%
\pgfpathlineto{\pgfqpoint{3.103961in}{0.595592in}}%
\pgfpathlineto{\pgfqpoint{3.128268in}{0.579069in}}%
\pgfpathlineto{\pgfqpoint{3.153537in}{0.564554in}}%
\pgfpathlineto{\pgfqpoint{3.181571in}{0.550952in}}%
\pgfpathlineto{\pgfqpoint{3.214371in}{0.537647in}}%
\pgfpathlineto{\pgfqpoint{3.249846in}{0.525712in}}%
\pgfpathlineto{\pgfqpoint{3.290011in}{0.514571in}}%
\pgfpathlineto{\pgfqpoint{3.334820in}{0.504423in}}%
\pgfpathlineto{\pgfqpoint{3.386372in}{0.494999in}}%
\pgfpathlineto{\pgfqpoint{3.446798in}{0.486257in}}%
\pgfpathlineto{\pgfqpoint{3.518243in}{0.478282in}}%
\pgfpathlineto{\pgfqpoint{3.600685in}{0.471409in}}%
\pgfpathlineto{\pgfqpoint{3.696268in}{0.465713in}}%
\pgfpathlineto{\pgfqpoint{3.807144in}{0.461369in}}%
\pgfpathlineto{\pgfqpoint{3.933291in}{0.458719in}}%
\pgfpathlineto{\pgfqpoint{4.063808in}{0.458211in}}%
\pgfpathlineto{\pgfqpoint{4.187792in}{0.459914in}}%
\pgfpathlineto{\pgfqpoint{4.294335in}{0.463521in}}%
\pgfpathlineto{\pgfqpoint{4.381234in}{0.468574in}}%
\pgfpathlineto{\pgfqpoint{4.450636in}{0.474701in}}%
\pgfpathlineto{\pgfqpoint{4.506850in}{0.481799in}}%
\pgfpathlineto{\pgfqpoint{4.552009in}{0.489658in}}%
\pgfpathlineto{\pgfqpoint{4.588239in}{0.498115in}}%
\pgfpathlineto{\pgfqpoint{4.617656in}{0.507110in}}%
\pgfpathlineto{\pgfqpoint{4.642328in}{0.516843in}}%
\pgfpathlineto{\pgfqpoint{4.664194in}{0.527940in}}%
\pgfpathlineto{\pgfqpoint{4.681238in}{0.538945in}}%
\pgfpathlineto{\pgfqpoint{4.697164in}{0.551953in}}%
\pgfpathlineto{\pgfqpoint{4.710076in}{0.565289in}}%
\pgfpathlineto{\pgfqpoint{4.721578in}{0.580218in}}%
\pgfpathlineto{\pgfqpoint{4.731557in}{0.596521in}}%
\pgfpathlineto{\pgfqpoint{4.741000in}{0.616134in}}%
\pgfpathlineto{\pgfqpoint{4.749521in}{0.639027in}}%
\pgfpathlineto{\pgfqpoint{4.757522in}{0.667450in}}%
\pgfpathlineto{\pgfqpoint{4.764572in}{0.701345in}}%
\pgfpathlineto{\pgfqpoint{4.770840in}{0.743043in}}%
\pgfpathlineto{\pgfqpoint{4.776327in}{0.794934in}}%
\pgfpathlineto{\pgfqpoint{4.781278in}{0.864398in}}%
\pgfpathlineto{\pgfqpoint{4.785468in}{0.956371in}}%
\pgfpathlineto{\pgfqpoint{4.789000in}{1.085745in}}%
\pgfpathlineto{\pgfqpoint{4.791852in}{1.277385in}}%
\pgfpathlineto{\pgfqpoint{4.793959in}{1.581058in}}%
\pgfpathlineto{\pgfqpoint{4.794962in}{2.071429in}}%
\pgfpathlineto{\pgfqpoint{4.793967in}{2.559311in}}%
\pgfpathlineto{\pgfqpoint{4.791733in}{2.745981in}}%
\pgfpathlineto{\pgfqpoint{4.788955in}{2.818091in}}%
\pgfpathlineto{\pgfqpoint{4.785731in}{2.850227in}}%
\pgfpathlineto{\pgfqpoint{4.781879in}{2.867057in}}%
\pgfpathlineto{\pgfqpoint{4.777744in}{2.875780in}}%
\pgfpathlineto{\pgfqpoint{4.773097in}{2.880982in}}%
\pgfpathlineto{\pgfqpoint{4.767363in}{2.884504in}}%
\pgfpathlineto{\pgfqpoint{4.756853in}{2.887622in}}%
\pgfpathlineto{\pgfqpoint{4.739548in}{2.889639in}}%
\pgfpathlineto{\pgfqpoint{4.704762in}{2.890882in}}%
\pgfpathlineto{\pgfqpoint{4.602524in}{2.891538in}}%
\pgfpathlineto{\pgfqpoint{3.952100in}{2.891742in}}%
\pgfpathlineto{\pgfqpoint{0.617321in}{2.890753in}}%
\pgfpathlineto{\pgfqpoint{0.549910in}{2.888858in}}%
\pgfpathlineto{\pgfqpoint{0.521735in}{2.886179in}}%
\pgfpathlineto{\pgfqpoint{0.504666in}{2.882389in}}%
\pgfpathlineto{\pgfqpoint{0.494501in}{2.878011in}}%
\pgfpathlineto{\pgfqpoint{0.487180in}{2.872667in}}%
\pgfpathlineto{\pgfqpoint{0.481152in}{2.865519in}}%
\pgfpathlineto{\pgfqpoint{0.475664in}{2.854804in}}%
\pgfpathlineto{\pgfqpoint{0.471318in}{2.840737in}}%
\pgfpathlineto{\pgfqpoint{0.467301in}{2.818823in}}%
\pgfpathlineto{\pgfqpoint{0.463927in}{2.786700in}}%
\pgfpathlineto{\pgfqpoint{0.460918in}{2.734545in}}%
\pgfpathlineto{\pgfqpoint{0.458363in}{2.647474in}}%
\pgfpathlineto{\pgfqpoint{0.456575in}{2.523031in}}%
\pgfpathlineto{\pgfqpoint{0.456575in}{2.523031in}}%
\pgfusepath{stroke}%
\end{pgfscope}%
\begin{pgfscope}%
\pgfpathrectangle{\pgfqpoint{0.448634in}{0.402556in}}{\pgfqpoint{4.350661in}{2.489204in}} %
\pgfusepath{clip}%
\pgfsetrectcap%
\pgfsetroundjoin%
\pgfsetlinewidth{1.003750pt}%
\definecolor{currentstroke}{rgb}{0.580392,0.403922,0.741176}%
\pgfsetstrokecolor{currentstroke}%
\pgfsetdash{}{0pt}%
\pgfpathmoveto{\pgfqpoint{4.798840in}{2.852369in}}%
\pgfpathlineto{\pgfqpoint{4.797564in}{2.889610in}}%
\pgfpathlineto{\pgfqpoint{4.796215in}{2.891483in}}%
\pgfpathlineto{\pgfqpoint{4.787551in}{2.891760in}}%
\pgfpathlineto{\pgfqpoint{0.452128in}{2.891664in}}%
\pgfpathlineto{\pgfqpoint{0.450530in}{2.890087in}}%
\pgfpathlineto{\pgfqpoint{0.449453in}{2.882768in}}%
\pgfpathlineto{\pgfqpoint{0.448969in}{2.845437in}}%
\pgfpathlineto{\pgfqpoint{0.448742in}{2.491970in}}%
\pgfpathlineto{\pgfqpoint{0.449622in}{0.615112in}}%
\pgfpathlineto{\pgfqpoint{0.451429in}{0.510591in}}%
\pgfpathlineto{\pgfqpoint{0.453985in}{0.473379in}}%
\pgfpathlineto{\pgfqpoint{0.457393in}{0.453872in}}%
\pgfpathlineto{\pgfqpoint{0.461521in}{0.442385in}}%
\pgfpathlineto{\pgfqpoint{0.466715in}{0.434434in}}%
\pgfpathlineto{\pgfqpoint{0.473568in}{0.428343in}}%
\pgfpathlineto{\pgfqpoint{0.483465in}{0.423234in}}%
\pgfpathlineto{\pgfqpoint{0.489716in}{0.421095in}}%
\pgfpathlineto{\pgfqpoint{0.489716in}{0.421095in}}%
\pgfusepath{stroke}%
\end{pgfscope}%
\begin{pgfscope}%
\pgfpathrectangle{\pgfqpoint{0.448634in}{0.402556in}}{\pgfqpoint{4.350661in}{2.489204in}} %
\pgfusepath{clip}%
\pgfsetrectcap%
\pgfsetroundjoin%
\pgfsetlinewidth{1.003750pt}%
\definecolor{currentstroke}{rgb}{0.580392,0.403922,0.741176}%
\pgfsetstrokecolor{currentstroke}%
\pgfsetdash{}{0pt}%
\pgfpathmoveto{\pgfqpoint{0.456424in}{1.370141in}}%
\pgfpathlineto{\pgfqpoint{0.459610in}{1.118760in}}%
\pgfpathlineto{\pgfqpoint{0.463694in}{0.962012in}}%
\pgfpathlineto{\pgfqpoint{0.468518in}{0.857614in}}%
\pgfpathlineto{\pgfqpoint{0.474082in}{0.783215in}}%
\pgfpathlineto{\pgfqpoint{0.480225in}{0.728911in}}%
\pgfpathlineto{\pgfqpoint{0.486970in}{0.687310in}}%
\pgfpathlineto{\pgfqpoint{0.494536in}{0.653563in}}%
\pgfpathlineto{\pgfqpoint{0.503106in}{0.625359in}}%
\pgfpathlineto{\pgfqpoint{0.512191in}{0.602753in}}%
\pgfpathlineto{\pgfqpoint{0.522198in}{0.583511in}}%
\pgfpathlineto{\pgfqpoint{0.534105in}{0.565746in}}%
\pgfpathlineto{\pgfqpoint{0.546261in}{0.551510in}}%
\pgfpathlineto{\pgfqpoint{0.559725in}{0.538910in}}%
\pgfpathlineto{\pgfqpoint{0.576126in}{0.526696in}}%
\pgfpathlineto{\pgfqpoint{0.595480in}{0.515353in}}%
\pgfpathlineto{\pgfqpoint{0.617677in}{0.505148in}}%
\pgfpathlineto{\pgfqpoint{0.642564in}{0.496154in}}%
\pgfpathlineto{\pgfqpoint{0.672122in}{0.487779in}}%
\pgfpathlineto{\pgfqpoint{0.708439in}{0.479824in}}%
\pgfpathlineto{\pgfqpoint{0.753646in}{0.472326in}}%
\pgfpathlineto{\pgfqpoint{0.807714in}{0.465661in}}%
\pgfpathlineto{\pgfqpoint{0.877112in}{0.459475in}}%
\pgfpathlineto{\pgfqpoint{0.961824in}{0.454230in}}%
\pgfpathlineto{\pgfqpoint{1.068348in}{0.449916in}}%
\pgfpathlineto{\pgfqpoint{1.201014in}{0.446839in}}%
\pgfpathlineto{\pgfqpoint{1.357633in}{0.445481in}}%
\pgfpathlineto{\pgfqpoint{1.525131in}{0.446232in}}%
\pgfpathlineto{\pgfqpoint{1.686084in}{0.449142in}}%
\pgfpathlineto{\pgfqpoint{1.823070in}{0.453747in}}%
\pgfpathlineto{\pgfqpoint{1.938241in}{0.459764in}}%
\pgfpathlineto{\pgfqpoint{2.031578in}{0.466759in}}%
\pgfpathlineto{\pgfqpoint{2.109576in}{0.474745in}}%
\pgfpathlineto{\pgfqpoint{2.174380in}{0.483534in}}%
\pgfpathlineto{\pgfqpoint{2.228135in}{0.492939in}}%
\pgfpathlineto{\pgfqpoint{2.275115in}{0.503356in}}%
\pgfpathlineto{\pgfqpoint{2.315278in}{0.514500in}}%
\pgfpathlineto{\pgfqpoint{2.350694in}{0.526658in}}%
\pgfpathlineto{\pgfqpoint{2.381317in}{0.539534in}}%
\pgfpathlineto{\pgfqpoint{2.407161in}{0.552657in}}%
\pgfpathlineto{\pgfqpoint{2.430223in}{0.566637in}}%
\pgfpathlineto{\pgfqpoint{2.452278in}{0.582599in}}%
\pgfpathlineto{\pgfqpoint{2.471388in}{0.599066in}}%
\pgfpathlineto{\pgfqpoint{2.489238in}{0.617290in}}%
\pgfpathlineto{\pgfqpoint{2.505675in}{0.637177in}}%
\pgfpathlineto{\pgfqpoint{2.520618in}{0.658554in}}%
\pgfpathlineto{\pgfqpoint{2.535211in}{0.683310in}}%
\pgfpathlineto{\pgfqpoint{2.549114in}{0.711480in}}%
\pgfpathlineto{\pgfqpoint{2.562089in}{0.743000in}}%
\pgfpathlineto{\pgfqpoint{2.574019in}{0.777747in}}%
\pgfpathlineto{\pgfqpoint{2.585501in}{0.817966in}}%
\pgfpathlineto{\pgfqpoint{2.596808in}{0.866034in}}%
\pgfpathlineto{\pgfqpoint{2.607561in}{0.921943in}}%
\pgfpathlineto{\pgfqpoint{2.617924in}{0.988094in}}%
\pgfpathlineto{\pgfqpoint{2.627958in}{1.066914in}}%
\pgfpathlineto{\pgfqpoint{2.637941in}{1.163316in}}%
\pgfpathlineto{\pgfqpoint{2.648424in}{1.287195in}}%
\pgfpathlineto{\pgfqpoint{2.660103in}{1.453434in}}%
\pgfpathlineto{\pgfqpoint{2.674773in}{1.696797in}}%
\pgfpathlineto{\pgfqpoint{2.687716in}{1.945274in}}%
\pgfpathlineto{\pgfqpoint{2.692670in}{2.079569in}}%
\pgfpathlineto{\pgfqpoint{2.693829in}{2.166678in}}%
\pgfpathlineto{\pgfqpoint{2.692565in}{2.233866in}}%
\pgfpathlineto{\pgfqpoint{2.689437in}{2.286010in}}%
\pgfpathlineto{\pgfqpoint{2.684859in}{2.327995in}}%
\pgfpathlineto{\pgfqpoint{2.678725in}{2.364660in}}%
\pgfpathlineto{\pgfqpoint{2.671357in}{2.395893in}}%
\pgfpathlineto{\pgfqpoint{2.662490in}{2.423977in}}%
\pgfpathlineto{\pgfqpoint{2.652363in}{2.448774in}}%
\pgfpathlineto{\pgfqpoint{2.641367in}{2.470242in}}%
\pgfpathlineto{\pgfqpoint{2.628645in}{2.490421in}}%
\pgfpathlineto{\pgfqpoint{2.614281in}{2.509102in}}%
\pgfpathlineto{\pgfqpoint{2.598446in}{2.526156in}}%
\pgfpathlineto{\pgfqpoint{2.579592in}{2.543003in}}%
\pgfpathlineto{\pgfqpoint{2.559535in}{2.557921in}}%
\pgfpathlineto{\pgfqpoint{2.536605in}{2.572181in}}%
\pgfpathlineto{\pgfqpoint{2.510853in}{2.585537in}}%
\pgfpathlineto{\pgfqpoint{2.482363in}{2.597836in}}%
\pgfpathlineto{\pgfqpoint{2.449138in}{2.609682in}}%
\pgfpathlineto{\pgfqpoint{2.411187in}{2.620695in}}%
\pgfpathlineto{\pgfqpoint{2.368555in}{2.630605in}}%
\pgfpathlineto{\pgfqpoint{2.321297in}{2.639220in}}%
\pgfpathlineto{\pgfqpoint{2.269470in}{2.646398in}}%
\pgfpathlineto{\pgfqpoint{2.210958in}{2.652193in}}%
\pgfpathlineto{\pgfqpoint{2.147970in}{2.656153in}}%
\pgfpathlineto{\pgfqpoint{2.080560in}{2.658135in}}%
\pgfpathlineto{\pgfqpoint{2.010951in}{2.657971in}}%
\pgfpathlineto{\pgfqpoint{1.939198in}{2.655572in}}%
\pgfpathlineto{\pgfqpoint{1.867530in}{2.650913in}}%
\pgfpathlineto{\pgfqpoint{1.798174in}{2.644141in}}%
\pgfpathlineto{\pgfqpoint{1.733344in}{2.635606in}}%
\pgfpathlineto{\pgfqpoint{1.673079in}{2.625522in}}%
\pgfpathlineto{\pgfqpoint{1.615277in}{2.613611in}}%
\pgfpathlineto{\pgfqpoint{1.562136in}{2.600402in}}%
\pgfpathlineto{\pgfqpoint{1.513684in}{2.586140in}}%
\pgfpathlineto{\pgfqpoint{1.467865in}{2.570345in}}%
\pgfpathlineto{\pgfqpoint{1.426797in}{2.553924in}}%
\pgfpathlineto{\pgfqpoint{1.388450in}{2.536290in}}%
\pgfpathlineto{\pgfqpoint{1.352881in}{2.517568in}}%
\pgfpathlineto{\pgfqpoint{1.320131in}{2.497924in}}%
\pgfpathlineto{\pgfqpoint{1.288382in}{2.476238in}}%
\pgfpathlineto{\pgfqpoint{1.259595in}{2.453863in}}%
\pgfpathlineto{\pgfqpoint{1.232053in}{2.429522in}}%
\pgfpathlineto{\pgfqpoint{1.207530in}{2.404901in}}%
\pgfpathlineto{\pgfqpoint{1.184411in}{2.378560in}}%
\pgfpathlineto{\pgfqpoint{1.162830in}{2.350564in}}%
\pgfpathlineto{\pgfqpoint{1.142893in}{2.321015in}}%
\pgfpathlineto{\pgfqpoint{1.124677in}{2.290044in}}%
\pgfpathlineto{\pgfqpoint{1.108227in}{2.257805in}}%
\pgfpathlineto{\pgfqpoint{1.092641in}{2.222202in}}%
\pgfpathlineto{\pgfqpoint{1.079061in}{2.185539in}}%
\pgfpathlineto{\pgfqpoint{1.067444in}{2.148001in}}%
\pgfpathlineto{\pgfqpoint{1.057188in}{2.107351in}}%
\pgfpathlineto{\pgfqpoint{1.049005in}{2.066090in}}%
\pgfpathlineto{\pgfqpoint{1.042514in}{2.021910in}}%
\pgfpathlineto{\pgfqpoint{1.038177in}{1.977385in}}%
\pgfpathlineto{\pgfqpoint{1.035866in}{1.930171in}}%
\pgfpathlineto{\pgfqpoint{1.035826in}{1.882882in}}%
\pgfpathlineto{\pgfqpoint{1.038031in}{1.835660in}}%
\pgfpathlineto{\pgfqpoint{1.042474in}{1.788645in}}%
\pgfpathlineto{\pgfqpoint{1.049175in}{1.741982in}}%
\pgfpathlineto{\pgfqpoint{1.057643in}{1.698243in}}%
\pgfpathlineto{\pgfqpoint{1.068221in}{1.655109in}}%
\pgfpathlineto{\pgfqpoint{1.080962in}{1.612748in}}%
\pgfpathlineto{\pgfqpoint{1.095030in}{1.573620in}}%
\pgfpathlineto{\pgfqpoint{1.111114in}{1.535523in}}%
\pgfpathlineto{\pgfqpoint{1.128116in}{1.500778in}}%
\pgfpathlineto{\pgfqpoint{1.146928in}{1.467277in}}%
\pgfpathlineto{\pgfqpoint{1.167529in}{1.435184in}}%
\pgfpathlineto{\pgfqpoint{1.189872in}{1.404655in}}%
\pgfpathlineto{\pgfqpoint{1.213882in}{1.375830in}}%
\pgfpathlineto{\pgfqpoint{1.237815in}{1.350459in}}%
\pgfpathlineto{\pgfqpoint{1.264746in}{1.325240in}}%
\pgfpathlineto{\pgfqpoint{1.292988in}{1.301974in}}%
\pgfpathlineto{\pgfqpoint{1.322395in}{1.280680in}}%
\pgfpathlineto{\pgfqpoint{1.352817in}{1.261342in}}%
\pgfpathlineto{\pgfqpoint{1.386092in}{1.242890in}}%
\pgfpathlineto{\pgfqpoint{1.420188in}{1.226517in}}%
\pgfpathlineto{\pgfqpoint{1.457021in}{1.211330in}}%
\pgfpathlineto{\pgfqpoint{1.496551in}{1.197537in}}%
\pgfpathlineto{\pgfqpoint{1.538717in}{1.185287in}}%
\pgfpathlineto{\pgfqpoint{1.583438in}{1.174642in}}%
\pgfpathlineto{\pgfqpoint{1.634926in}{1.164775in}}%
\pgfpathlineto{\pgfqpoint{1.706060in}{1.153745in}}%
\pgfpathlineto{\pgfqpoint{1.768489in}{1.143417in}}%
\pgfpathlineto{\pgfqpoint{1.796119in}{1.136568in}}%
\pgfpathlineto{\pgfqpoint{1.812680in}{1.130482in}}%
\pgfpathlineto{\pgfqpoint{1.824468in}{1.124103in}}%
\pgfpathlineto{\pgfqpoint{1.833207in}{1.116743in}}%
\pgfpathlineto{\pgfqpoint{1.838496in}{1.108891in}}%
\pgfpathlineto{\pgfqpoint{1.840587in}{1.101851in}}%
\pgfpathlineto{\pgfqpoint{1.840618in}{1.094414in}}%
\pgfpathlineto{\pgfqpoint{1.837930in}{1.084987in}}%
\pgfpathlineto{\pgfqpoint{1.833246in}{1.076617in}}%
\pgfpathlineto{\pgfqpoint{1.825818in}{1.067544in}}%
\pgfpathlineto{\pgfqpoint{1.813813in}{1.056851in}}%
\pgfpathlineto{\pgfqpoint{1.798819in}{1.046764in}}%
\pgfpathlineto{\pgfqpoint{1.781016in}{1.037463in}}%
\pgfpathlineto{\pgfqpoint{1.758447in}{1.028392in}}%
\pgfpathlineto{\pgfqpoint{1.733203in}{1.020816in}}%
\pgfpathlineto{\pgfqpoint{1.705410in}{1.014872in}}%
\pgfpathlineto{\pgfqpoint{1.675178in}{1.010714in}}%
\pgfpathlineto{\pgfqpoint{1.642610in}{1.008508in}}%
\pgfpathlineto{\pgfqpoint{1.607809in}{1.008433in}}%
\pgfpathlineto{\pgfqpoint{1.570886in}{1.010692in}}%
\pgfpathlineto{\pgfqpoint{1.534118in}{1.015182in}}%
\pgfpathlineto{\pgfqpoint{1.495455in}{1.022233in}}%
\pgfpathlineto{\pgfqpoint{1.457161in}{1.031564in}}%
\pgfpathlineto{\pgfqpoint{1.419338in}{1.043132in}}%
\pgfpathlineto{\pgfqpoint{1.382089in}{1.056929in}}%
\pgfpathlineto{\pgfqpoint{1.347544in}{1.072019in}}%
\pgfpathlineto{\pgfqpoint{1.313727in}{1.089133in}}%
\pgfpathlineto{\pgfqpoint{1.280762in}{1.108299in}}%
\pgfpathlineto{\pgfqpoint{1.248782in}{1.129536in}}%
\pgfpathlineto{\pgfqpoint{1.219708in}{1.151422in}}%
\pgfpathlineto{\pgfqpoint{1.191752in}{1.175138in}}%
\pgfpathlineto{\pgfqpoint{1.165031in}{1.200649in}}%
\pgfpathlineto{\pgfqpoint{1.139653in}{1.227897in}}%
\pgfpathlineto{\pgfqpoint{1.115714in}{1.256800in}}%
\pgfpathlineto{\pgfqpoint{1.093288in}{1.287251in}}%
\pgfpathlineto{\pgfqpoint{1.071177in}{1.321163in}}%
\pgfpathlineto{\pgfqpoint{1.050868in}{1.356519in}}%
\pgfpathlineto{\pgfqpoint{1.032365in}{1.393151in}}%
\pgfpathlineto{\pgfqpoint{1.014718in}{1.433142in}}%
\pgfpathlineto{\pgfqpoint{0.999024in}{1.474185in}}%
\pgfpathlineto{\pgfqpoint{0.984506in}{1.518461in}}%
\pgfpathlineto{\pgfqpoint{0.972009in}{1.563537in}}%
\pgfpathlineto{\pgfqpoint{0.960943in}{1.611678in}}%
\pgfpathlineto{\pgfqpoint{0.951530in}{1.662824in}}%
\pgfpathlineto{\pgfqpoint{0.944286in}{1.714431in}}%
\pgfpathlineto{\pgfqpoint{0.938950in}{1.768847in}}%
\pgfpathlineto{\pgfqpoint{0.935870in}{1.823491in}}%
\pgfpathlineto{\pgfqpoint{0.935034in}{1.878240in}}%
\pgfpathlineto{\pgfqpoint{0.936465in}{1.932972in}}%
\pgfpathlineto{\pgfqpoint{0.940005in}{1.985084in}}%
\pgfpathlineto{\pgfqpoint{0.945758in}{2.036935in}}%
\pgfpathlineto{\pgfqpoint{0.953409in}{2.085938in}}%
\pgfpathlineto{\pgfqpoint{0.962764in}{2.132000in}}%
\pgfpathlineto{\pgfqpoint{0.974286in}{2.177413in}}%
\pgfpathlineto{\pgfqpoint{0.987332in}{2.219652in}}%
\pgfpathlineto{\pgfqpoint{1.001667in}{2.258654in}}%
\pgfpathlineto{\pgfqpoint{1.018050in}{2.296583in}}%
\pgfpathlineto{\pgfqpoint{1.035401in}{2.331101in}}%
\pgfpathlineto{\pgfqpoint{1.054650in}{2.364275in}}%
\pgfpathlineto{\pgfqpoint{1.074406in}{2.393983in}}%
\pgfpathlineto{\pgfqpoint{1.095771in}{2.422197in}}%
\pgfpathlineto{\pgfqpoint{1.118662in}{2.448796in}}%
\pgfpathlineto{\pgfqpoint{1.142966in}{2.473700in}}%
\pgfpathlineto{\pgfqpoint{1.168550in}{2.496867in}}%
\pgfpathlineto{\pgfqpoint{1.197084in}{2.519662in}}%
\pgfpathlineto{\pgfqpoint{1.226726in}{2.540526in}}%
\pgfpathlineto{\pgfqpoint{1.259241in}{2.560673in}}%
\pgfpathlineto{\pgfqpoint{1.294612in}{2.579881in}}%
\pgfpathlineto{\pgfqpoint{1.332792in}{2.597981in}}%
\pgfpathlineto{\pgfqpoint{1.373718in}{2.614859in}}%
\pgfpathlineto{\pgfqpoint{1.417319in}{2.630445in}}%
\pgfpathlineto{\pgfqpoint{1.465632in}{2.645312in}}%
\pgfpathlineto{\pgfqpoint{1.518640in}{2.659204in}}%
\pgfpathlineto{\pgfqpoint{1.576309in}{2.671929in}}%
\pgfpathlineto{\pgfqpoint{1.638597in}{2.683344in}}%
\pgfpathlineto{\pgfqpoint{1.705462in}{2.693343in}}%
\pgfpathlineto{\pgfqpoint{1.779027in}{2.702064in}}%
\pgfpathlineto{\pgfqpoint{1.857097in}{2.709076in}}%
\pgfpathlineto{\pgfqpoint{1.939633in}{2.714280in}}%
\pgfpathlineto{\pgfqpoint{2.026598in}{2.717513in}}%
\pgfpathlineto{\pgfqpoint{2.113605in}{2.718523in}}%
\pgfpathlineto{\pgfqpoint{2.198434in}{2.717303in}}%
\pgfpathlineto{\pgfqpoint{2.278865in}{2.713929in}}%
\pgfpathlineto{\pgfqpoint{2.352677in}{2.708598in}}%
\pgfpathlineto{\pgfqpoint{2.417656in}{2.701709in}}%
\pgfpathlineto{\pgfqpoint{2.473770in}{2.693630in}}%
\pgfpathlineto{\pgfqpoint{2.523140in}{2.684368in}}%
\pgfpathlineto{\pgfqpoint{2.565726in}{2.674202in}}%
\pgfpathlineto{\pgfqpoint{2.601510in}{2.663544in}}%
\pgfpathlineto{\pgfqpoint{2.632576in}{2.652142in}}%
\pgfpathlineto{\pgfqpoint{2.658899in}{2.640331in}}%
\pgfpathlineto{\pgfqpoint{2.682438in}{2.627436in}}%
\pgfpathlineto{\pgfqpoint{2.703062in}{2.613571in}}%
\pgfpathlineto{\pgfqpoint{2.720674in}{2.598978in}}%
\pgfpathlineto{\pgfqpoint{2.735262in}{2.584053in}}%
\pgfpathlineto{\pgfqpoint{2.748319in}{2.567377in}}%
\pgfpathlineto{\pgfqpoint{2.759553in}{2.549046in}}%
\pgfpathlineto{\pgfqpoint{2.768787in}{2.529306in}}%
\pgfpathlineto{\pgfqpoint{2.776016in}{2.508498in}}%
\pgfpathlineto{\pgfqpoint{2.781884in}{2.484540in}}%
\pgfpathlineto{\pgfqpoint{2.786102in}{2.457597in}}%
\pgfpathlineto{\pgfqpoint{2.788720in}{2.425384in}}%
\pgfpathlineto{\pgfqpoint{2.789427in}{2.388061in}}%
\pgfpathlineto{\pgfqpoint{2.787962in}{2.340801in}}%
\pgfpathlineto{\pgfqpoint{2.783672in}{2.278768in}}%
\pgfpathlineto{\pgfqpoint{2.774288in}{2.179783in}}%
\pgfpathlineto{\pgfqpoint{2.743611in}{1.868119in}}%
\pgfpathlineto{\pgfqpoint{2.730111in}{1.702060in}}%
\pgfpathlineto{\pgfqpoint{2.717287in}{1.515949in}}%
\pgfpathlineto{\pgfqpoint{2.702602in}{1.267597in}}%
\pgfpathlineto{\pgfqpoint{2.684434in}{0.964630in}}%
\pgfpathlineto{\pgfqpoint{2.675374in}{0.850600in}}%
\pgfpathlineto{\pgfqpoint{2.667030in}{0.771523in}}%
\pgfpathlineto{\pgfqpoint{2.658752in}{0.712543in}}%
\pgfpathlineto{\pgfqpoint{2.650176in}{0.666284in}}%
\pgfpathlineto{\pgfqpoint{2.640820in}{0.627931in}}%
\pgfpathlineto{\pgfqpoint{2.631144in}{0.597534in}}%
\pgfpathlineto{\pgfqpoint{2.621003in}{0.572745in}}%
\pgfpathlineto{\pgfqpoint{2.609856in}{0.551383in}}%
\pgfpathlineto{\pgfqpoint{2.598042in}{0.533534in}}%
\pgfpathlineto{\pgfqpoint{2.584495in}{0.517378in}}%
\pgfpathlineto{\pgfqpoint{2.571108in}{0.504669in}}%
\pgfpathlineto{\pgfqpoint{2.554789in}{0.492313in}}%
\pgfpathlineto{\pgfqpoint{2.537456in}{0.481914in}}%
\pgfpathlineto{\pgfqpoint{2.517373in}{0.472367in}}%
\pgfpathlineto{\pgfqpoint{2.492542in}{0.463178in}}%
\pgfpathlineto{\pgfqpoint{2.462979in}{0.454833in}}%
\pgfpathlineto{\pgfqpoint{2.428766in}{0.447542in}}%
\pgfpathlineto{\pgfqpoint{2.385671in}{0.440735in}}%
\pgfpathlineto{\pgfqpoint{2.331557in}{0.434581in}}%
\pgfpathlineto{\pgfqpoint{2.262115in}{0.429077in}}%
\pgfpathlineto{\pgfqpoint{2.170850in}{0.424236in}}%
\pgfpathlineto{\pgfqpoint{2.049086in}{0.420134in}}%
\pgfpathlineto{\pgfqpoint{1.879436in}{0.416783in}}%
\pgfpathlineto{\pgfqpoint{1.640159in}{0.414418in}}%
\pgfpathlineto{\pgfqpoint{1.322562in}{0.413569in}}%
\pgfpathlineto{\pgfqpoint{1.020194in}{0.414850in}}%
\pgfpathlineto{\pgfqpoint{0.822256in}{0.417715in}}%
\pgfpathlineto{\pgfqpoint{0.704834in}{0.421430in}}%
\pgfpathlineto{\pgfqpoint{0.630976in}{0.425829in}}%
\pgfpathlineto{\pgfqpoint{0.583315in}{0.430734in}}%
\pgfpathlineto{\pgfqpoint{0.551033in}{0.436124in}}%
\pgfpathlineto{\pgfqpoint{0.527708in}{0.442189in}}%
\pgfpathlineto{\pgfqpoint{0.511250in}{0.448626in}}%
\pgfpathlineto{\pgfqpoint{0.499548in}{0.455216in}}%
\pgfpathlineto{\pgfqpoint{0.488916in}{0.463842in}}%
\pgfpathlineto{\pgfqpoint{0.481322in}{0.472731in}}%
\pgfpathlineto{\pgfqpoint{0.474078in}{0.485127in}}%
\pgfpathlineto{\pgfqpoint{0.468753in}{0.498749in}}%
\pgfpathlineto{\pgfqpoint{0.463869in}{0.517849in}}%
\pgfpathlineto{\pgfqpoint{0.459679in}{0.544797in}}%
\pgfpathlineto{\pgfqpoint{0.456386in}{0.581939in}}%
\pgfpathlineto{\pgfqpoint{0.453731in}{0.639107in}}%
\pgfpathlineto{\pgfqpoint{0.451681in}{0.736156in}}%
\pgfpathlineto{\pgfqpoint{0.450220in}{0.927816in}}%
\pgfpathlineto{\pgfqpoint{0.449345in}{1.403253in}}%
\pgfpathlineto{\pgfqpoint{0.449543in}{2.682703in}}%
\pgfpathlineto{\pgfqpoint{0.451011in}{2.856933in}}%
\pgfpathlineto{\pgfqpoint{0.452802in}{2.879220in}}%
\pgfpathlineto{\pgfqpoint{0.455188in}{2.886108in}}%
\pgfpathlineto{\pgfqpoint{0.458626in}{2.889029in}}%
\pgfpathlineto{\pgfqpoint{0.464996in}{2.890553in}}%
\pgfpathlineto{\pgfqpoint{0.482377in}{2.891423in}}%
\pgfpathlineto{\pgfqpoint{0.565039in}{2.891729in}}%
\pgfpathlineto{\pgfqpoint{2.733843in}{2.891760in}}%
\pgfpathlineto{\pgfqpoint{4.789510in}{2.890885in}}%
\pgfpathlineto{\pgfqpoint{4.793727in}{2.889729in}}%
\pgfpathlineto{\pgfqpoint{4.795481in}{2.888304in}}%
\pgfpathlineto{\pgfqpoint{4.797105in}{2.881142in}}%
\pgfpathlineto{\pgfqpoint{4.797996in}{2.858769in}}%
\pgfpathlineto{\pgfqpoint{4.798039in}{2.856280in}}%
\pgfpathlineto{\pgfqpoint{4.798039in}{2.856280in}}%
\pgfusepath{stroke}%
\end{pgfscope}%
\begin{pgfscope}%
\pgfpathrectangle{\pgfqpoint{0.448634in}{0.402556in}}{\pgfqpoint{4.350661in}{2.489204in}} %
\pgfusepath{clip}%
\pgfsetrectcap%
\pgfsetroundjoin%
\pgfsetlinewidth{1.003750pt}%
\definecolor{currentstroke}{rgb}{0.580392,0.403922,0.741176}%
\pgfsetstrokecolor{currentstroke}%
\pgfsetdash{}{0pt}%
\pgfpathmoveto{\pgfqpoint{3.428775in}{0.402610in}}%
\pgfpathlineto{\pgfqpoint{2.806635in}{0.403761in}}%
\pgfpathlineto{\pgfqpoint{2.769695in}{0.405580in}}%
\pgfpathlineto{\pgfqpoint{2.754636in}{0.408067in}}%
\pgfpathlineto{\pgfqpoint{2.746395in}{0.411202in}}%
\pgfpathlineto{\pgfqpoint{2.740947in}{0.415270in}}%
\pgfpathlineto{\pgfqpoint{2.736788in}{0.420989in}}%
\pgfpathlineto{\pgfqpoint{2.733285in}{0.430076in}}%
\pgfpathlineto{\pgfqpoint{2.730452in}{0.444641in}}%
\pgfpathlineto{\pgfqpoint{2.728240in}{0.469397in}}%
\pgfpathlineto{\pgfqpoint{2.726472in}{0.519136in}}%
\pgfpathlineto{\pgfqpoint{2.725712in}{0.613719in}}%
\pgfpathlineto{\pgfqpoint{2.726843in}{0.768043in}}%
\pgfpathlineto{\pgfqpoint{2.730557in}{0.962153in}}%
\pgfpathlineto{\pgfqpoint{2.736612in}{1.158675in}}%
\pgfpathlineto{\pgfqpoint{2.744093in}{1.327723in}}%
\pgfpathlineto{\pgfqpoint{2.753202in}{1.484194in}}%
\pgfpathlineto{\pgfqpoint{2.763258in}{1.620614in}}%
\pgfpathlineto{\pgfqpoint{2.776119in}{1.764220in}}%
\pgfpathlineto{\pgfqpoint{2.788916in}{1.877781in}}%
\pgfpathlineto{\pgfqpoint{2.805750in}{2.005745in}}%
\pgfpathlineto{\pgfqpoint{2.821178in}{2.101202in}}%
\pgfpathlineto{\pgfqpoint{2.838362in}{2.193723in}}%
\pgfpathlineto{\pgfqpoint{2.859138in}{2.292970in}}%
\pgfpathlineto{\pgfqpoint{2.887212in}{2.425964in}}%
\pgfpathlineto{\pgfqpoint{2.896994in}{2.479564in}}%
\pgfpathlineto{\pgfqpoint{2.901546in}{2.516527in}}%
\pgfpathlineto{\pgfqpoint{2.902852in}{2.543858in}}%
\pgfpathlineto{\pgfqpoint{2.901960in}{2.566227in}}%
\pgfpathlineto{\pgfqpoint{2.899154in}{2.585867in}}%
\pgfpathlineto{\pgfqpoint{2.894796in}{2.602550in}}%
\pgfpathlineto{\pgfqpoint{2.888486in}{2.618392in}}%
\pgfpathlineto{\pgfqpoint{2.880259in}{2.633036in}}%
\pgfpathlineto{\pgfqpoint{2.870350in}{2.646250in}}%
\pgfpathlineto{\pgfqpoint{2.857401in}{2.659534in}}%
\pgfpathlineto{\pgfqpoint{2.843191in}{2.671013in}}%
\pgfpathlineto{\pgfqpoint{2.824239in}{2.683212in}}%
\pgfpathlineto{\pgfqpoint{2.802414in}{2.694421in}}%
\pgfpathlineto{\pgfqpoint{2.775810in}{2.705371in}}%
\pgfpathlineto{\pgfqpoint{2.744462in}{2.715717in}}%
\pgfpathlineto{\pgfqpoint{2.708437in}{2.725254in}}%
\pgfpathlineto{\pgfqpoint{2.665656in}{2.734291in}}%
\pgfpathlineto{\pgfqpoint{2.613992in}{2.742871in}}%
\pgfpathlineto{\pgfqpoint{2.553460in}{2.750590in}}%
\pgfpathlineto{\pgfqpoint{2.481921in}{2.757367in}}%
\pgfpathlineto{\pgfqpoint{2.399399in}{2.762840in}}%
\pgfpathlineto{\pgfqpoint{2.310270in}{2.766483in}}%
\pgfpathlineto{\pgfqpoint{2.175417in}{2.768726in}}%
\pgfpathlineto{\pgfqpoint{2.066654in}{2.767943in}}%
\pgfpathlineto{\pgfqpoint{1.953571in}{2.764861in}}%
\pgfpathlineto{\pgfqpoint{1.851430in}{2.759760in}}%
\pgfpathlineto{\pgfqpoint{1.745052in}{2.752170in}}%
\pgfpathlineto{\pgfqpoint{1.658374in}{2.743455in}}%
\pgfpathlineto{\pgfqpoint{1.580553in}{2.733463in}}%
\pgfpathlineto{\pgfqpoint{1.490058in}{2.719340in}}%
\pgfpathlineto{\pgfqpoint{1.417232in}{2.704698in}}%
\pgfpathlineto{\pgfqpoint{1.361993in}{2.690818in}}%
\pgfpathlineto{\pgfqpoint{1.311461in}{2.675819in}}%
\pgfpathlineto{\pgfqpoint{1.265668in}{2.659923in}}%
\pgfpathlineto{\pgfqpoint{1.222576in}{2.642586in}}%
\pgfpathlineto{\pgfqpoint{1.184325in}{2.624682in}}%
\pgfpathlineto{\pgfqpoint{1.148893in}{2.605623in}}%
\pgfpathlineto{\pgfqpoint{1.116332in}{2.585573in}}%
\pgfpathlineto{\pgfqpoint{1.092328in}{2.568512in}}%
\pgfpathlineto{\pgfqpoint{1.079761in}{2.558686in}}%
\pgfpathlineto{\pgfqpoint{1.051545in}{2.535379in}}%
\pgfpathlineto{\pgfqpoint{1.026313in}{2.511713in}}%
\pgfpathlineto{\pgfqpoint{1.002399in}{2.486318in}}%
\pgfpathlineto{\pgfqpoint{0.979913in}{2.459270in}}%
\pgfpathlineto{\pgfqpoint{0.958935in}{2.430679in}}%
\pgfpathlineto{\pgfqpoint{0.938265in}{2.398644in}}%
\pgfpathlineto{\pgfqpoint{0.923048in}{2.371386in}}%
\pgfpathlineto{\pgfqpoint{0.904514in}{2.334775in}}%
\pgfpathlineto{\pgfqpoint{0.887854in}{2.297002in}}%
\pgfpathlineto{\pgfqpoint{0.872132in}{2.255972in}}%
\pgfpathlineto{\pgfqpoint{0.857508in}{2.211742in}}%
\pgfpathlineto{\pgfqpoint{0.844762in}{2.166758in}}%
\pgfpathlineto{\pgfqpoint{0.838624in}{2.140307in}}%
\pgfpathlineto{\pgfqpoint{0.826982in}{2.087195in}}%
\pgfpathlineto{\pgfqpoint{0.816322in}{2.028716in}}%
\pgfpathlineto{\pgfqpoint{0.810087in}{1.984496in}}%
\pgfpathlineto{\pgfqpoint{0.808026in}{1.967239in}}%
\pgfpathlineto{\pgfqpoint{0.800076in}{1.898141in}}%
\pgfpathlineto{\pgfqpoint{0.793713in}{1.823824in}}%
\pgfpathlineto{\pgfqpoint{0.788798in}{1.741876in}}%
\pgfpathlineto{\pgfqpoint{0.786199in}{1.677226in}}%
\pgfpathlineto{\pgfqpoint{0.776951in}{1.453482in}}%
\pgfpathlineto{\pgfqpoint{0.773280in}{1.418895in}}%
\pgfpathlineto{\pgfqpoint{0.768298in}{1.389583in}}%
\pgfpathlineto{\pgfqpoint{0.762752in}{1.368109in}}%
\pgfpathlineto{\pgfqpoint{0.756722in}{1.352124in}}%
\pgfpathlineto{\pgfqpoint{0.749752in}{1.339520in}}%
\pgfpathlineto{\pgfqpoint{0.742201in}{1.330600in}}%
\pgfpathlineto{\pgfqpoint{0.734854in}{1.325312in}}%
\pgfpathlineto{\pgfqpoint{0.726558in}{1.322419in}}%
\pgfpathlineto{\pgfqpoint{0.717884in}{1.322223in}}%
\pgfpathlineto{\pgfqpoint{0.709413in}{1.324411in}}%
\pgfpathlineto{\pgfqpoint{0.699548in}{1.329605in}}%
\pgfpathlineto{\pgfqpoint{0.688894in}{1.338203in}}%
\pgfpathlineto{\pgfqpoint{0.677907in}{1.350248in}}%
\pgfpathlineto{\pgfqpoint{0.666886in}{1.365647in}}%
\pgfpathlineto{\pgfqpoint{0.654913in}{1.386417in}}%
\pgfpathlineto{\pgfqpoint{0.642574in}{1.412730in}}%
\pgfpathlineto{\pgfqpoint{0.630328in}{1.444629in}}%
\pgfpathlineto{\pgfqpoint{0.618505in}{1.482080in}}%
\pgfpathlineto{\pgfqpoint{0.608613in}{1.520256in}}%
\pgfpathlineto{\pgfqpoint{0.590203in}{1.612445in}}%
\pgfpathlineto{\pgfqpoint{0.581848in}{1.668884in}}%
\pgfpathlineto{\pgfqpoint{0.573138in}{1.740376in}}%
\pgfpathlineto{\pgfqpoint{0.567062in}{1.807213in}}%
\pgfpathlineto{\pgfqpoint{0.560532in}{1.896509in}}%
\pgfpathlineto{\pgfqpoint{0.555526in}{1.995910in}}%
\pgfpathlineto{\pgfqpoint{0.552564in}{2.097908in}}%
\pgfpathlineto{\pgfqpoint{0.551526in}{2.204935in}}%
\pgfpathlineto{\pgfqpoint{0.552728in}{2.309470in}}%
\pgfpathlineto{\pgfqpoint{0.556011in}{2.403981in}}%
\pgfpathlineto{\pgfqpoint{0.560953in}{2.483430in}}%
\pgfpathlineto{\pgfqpoint{0.567303in}{2.550240in}}%
\pgfpathlineto{\pgfqpoint{0.574928in}{2.606817in}}%
\pgfpathlineto{\pgfqpoint{0.582988in}{2.650657in}}%
\pgfpathlineto{\pgfqpoint{0.592756in}{2.691452in}}%
\pgfpathlineto{\pgfqpoint{0.602650in}{2.721756in}}%
\pgfpathlineto{\pgfqpoint{0.612983in}{2.746441in}}%
\pgfpathlineto{\pgfqpoint{0.624292in}{2.767692in}}%
\pgfpathlineto{\pgfqpoint{0.636231in}{2.785432in}}%
\pgfpathlineto{\pgfqpoint{0.649892in}{2.801461in}}%
\pgfpathlineto{\pgfqpoint{0.663386in}{2.814020in}}%
\pgfpathlineto{\pgfqpoint{0.679842in}{2.826135in}}%
\pgfpathlineto{\pgfqpoint{0.697326in}{2.836197in}}%
\pgfpathlineto{\pgfqpoint{0.715574in}{2.844285in}}%
\pgfpathlineto{\pgfqpoint{0.738439in}{2.852335in}}%
\pgfpathlineto{\pgfqpoint{0.765983in}{2.859639in}}%
\pgfpathlineto{\pgfqpoint{0.800300in}{2.866256in}}%
\pgfpathlineto{\pgfqpoint{0.841340in}{2.871832in}}%
\pgfpathlineto{\pgfqpoint{0.895547in}{2.876803in}}%
\pgfpathlineto{\pgfqpoint{0.969413in}{2.881069in}}%
\pgfpathlineto{\pgfqpoint{1.071608in}{2.884501in}}%
\pgfpathlineto{\pgfqpoint{1.219512in}{2.887074in}}%
\pgfpathlineto{\pgfqpoint{1.471844in}{2.889091in}}%
\pgfpathlineto{\pgfqpoint{1.956941in}{2.890384in}}%
\pgfpathlineto{\pgfqpoint{3.096814in}{2.890781in}}%
\pgfpathlineto{\pgfqpoint{3.995224in}{2.889388in}}%
\pgfpathlineto{\pgfqpoint{4.275833in}{2.887011in}}%
\pgfpathlineto{\pgfqpoint{4.412847in}{2.883743in}}%
\pgfpathlineto{\pgfqpoint{4.491081in}{2.879810in}}%
\pgfpathlineto{\pgfqpoint{4.543127in}{2.875163in}}%
\pgfpathlineto{\pgfqpoint{4.579810in}{2.869841in}}%
\pgfpathlineto{\pgfqpoint{4.607580in}{2.863763in}}%
\pgfpathlineto{\pgfqpoint{4.630623in}{2.856424in}}%
\pgfpathlineto{\pgfqpoint{4.648833in}{2.848228in}}%
\pgfpathlineto{\pgfqpoint{4.664136in}{2.838773in}}%
\pgfpathlineto{\pgfqpoint{4.676470in}{2.828576in}}%
\pgfpathlineto{\pgfqpoint{4.687502in}{2.816585in}}%
\pgfpathlineto{\pgfqpoint{4.697051in}{2.803027in}}%
\pgfpathlineto{\pgfqpoint{4.706194in}{2.786098in}}%
\pgfpathlineto{\pgfqpoint{4.714508in}{2.765827in}}%
\pgfpathlineto{\pgfqpoint{4.722462in}{2.740013in}}%
\pgfpathlineto{\pgfqpoint{4.729577in}{2.708703in}}%
\pgfpathlineto{\pgfqpoint{4.736162in}{2.669601in}}%
\pgfpathlineto{\pgfqpoint{4.742419in}{2.617826in}}%
\pgfpathlineto{\pgfqpoint{4.747859in}{2.553410in}}%
\pgfpathlineto{\pgfqpoint{4.752661in}{2.468958in}}%
\pgfpathlineto{\pgfqpoint{4.756610in}{2.359528in}}%
\pgfpathlineto{\pgfqpoint{4.759416in}{2.217681in}}%
\pgfpathlineto{\pgfqpoint{4.760596in}{2.043444in}}%
\pgfpathlineto{\pgfqpoint{4.759662in}{1.851779in}}%
\pgfpathlineto{\pgfqpoint{4.756587in}{1.667613in}}%
\pgfpathlineto{\pgfqpoint{4.751596in}{1.503428in}}%
\pgfpathlineto{\pgfqpoint{4.745410in}{1.374185in}}%
\pgfpathlineto{\pgfqpoint{4.738113in}{1.267479in}}%
\pgfpathlineto{\pgfqpoint{4.729621in}{1.175896in}}%
\pgfpathlineto{\pgfqpoint{4.720762in}{1.104428in}}%
\pgfpathlineto{\pgfqpoint{4.711045in}{1.043204in}}%
\pgfpathlineto{\pgfqpoint{4.700364in}{0.989829in}}%
\pgfpathlineto{\pgfqpoint{4.689055in}{0.944345in}}%
\pgfpathlineto{\pgfqpoint{4.676881in}{0.904394in}}%
\pgfpathlineto{\pgfqpoint{4.676095in}{0.902073in}}%
\pgfpathlineto{\pgfqpoint{4.676095in}{0.902073in}}%
\pgfusepath{stroke}%
\end{pgfscope}%
\begin{pgfscope}%
\pgfpathrectangle{\pgfqpoint{0.448634in}{0.402556in}}{\pgfqpoint{4.350661in}{2.489204in}} %
\pgfusepath{clip}%
\pgfsetrectcap%
\pgfsetroundjoin%
\pgfsetlinewidth{1.003750pt}%
\definecolor{currentstroke}{rgb}{0.580392,0.403922,0.741176}%
\pgfsetstrokecolor{currentstroke}%
\pgfsetdash{}{0pt}%
\pgfpathmoveto{\pgfqpoint{2.795520in}{1.982745in}}%
\pgfpathlineto{\pgfqpoint{2.781780in}{1.874357in}}%
\pgfpathlineto{\pgfqpoint{2.769351in}{1.758234in}}%
\pgfpathlineto{\pgfqpoint{2.758095in}{1.631942in}}%
\pgfpathlineto{\pgfqpoint{2.747786in}{1.490551in}}%
\pgfpathlineto{\pgfqpoint{2.738644in}{1.334082in}}%
\pgfpathlineto{\pgfqpoint{2.730580in}{1.157591in}}%
\pgfpathlineto{\pgfqpoint{2.723334in}{0.948663in}}%
\pgfpathlineto{\pgfqpoint{2.709783in}{0.530788in}}%
\pgfpathlineto{\pgfqpoint{2.705868in}{0.488716in}}%
\pgfpathlineto{\pgfqpoint{2.701769in}{0.464281in}}%
\pgfpathlineto{\pgfqpoint{2.697021in}{0.447744in}}%
\pgfpathlineto{\pgfqpoint{2.691859in}{0.436812in}}%
\pgfpathlineto{\pgfqpoint{2.686245in}{0.429229in}}%
\pgfpathlineto{\pgfqpoint{2.679348in}{0.423188in}}%
\pgfpathlineto{\pgfqpoint{2.669540in}{0.417856in}}%
\pgfpathlineto{\pgfqpoint{2.656987in}{0.413810in}}%
\pgfpathlineto{\pgfqpoint{2.637654in}{0.410337in}}%
\pgfpathlineto{\pgfqpoint{2.607297in}{0.407617in}}%
\pgfpathlineto{\pgfqpoint{2.555121in}{0.405574in}}%
\pgfpathlineto{\pgfqpoint{2.450714in}{0.404139in}}%
\pgfpathlineto{\pgfqpoint{2.176624in}{0.403275in}}%
\pgfpathlineto{\pgfqpoint{1.130290in}{0.402953in}}%
\pgfpathlineto{\pgfqpoint{0.516849in}{0.404175in}}%
\pgfpathlineto{\pgfqpoint{0.466848in}{0.405970in}}%
\pgfpathlineto{\pgfqpoint{0.456130in}{0.407931in}}%
\pgfpathlineto{\pgfqpoint{0.452340in}{0.410303in}}%
\pgfpathlineto{\pgfqpoint{0.450346in}{0.414662in}}%
\pgfpathlineto{\pgfqpoint{0.449266in}{0.424524in}}%
\pgfpathlineto{\pgfqpoint{0.448771in}{0.464344in}}%
\pgfpathlineto{\pgfqpoint{0.448640in}{0.850171in}}%
\pgfpathlineto{\pgfqpoint{0.448679in}{2.891318in}}%
\pgfpathlineto{\pgfqpoint{0.448679in}{2.891318in}}%
\pgfusepath{stroke}%
\end{pgfscope}%
\begin{pgfscope}%
\pgfpathrectangle{\pgfqpoint{0.448634in}{0.402556in}}{\pgfqpoint{4.350661in}{2.489204in}} %
\pgfusepath{clip}%
\pgfsetrectcap%
\pgfsetroundjoin%
\pgfsetlinewidth{1.003750pt}%
\definecolor{currentstroke}{rgb}{0.580392,0.403922,0.741176}%
\pgfsetstrokecolor{currentstroke}%
\pgfsetdash{}{0pt}%
\pgfpathmoveto{\pgfqpoint{3.428198in}{0.402586in}}%
\pgfpathlineto{\pgfqpoint{2.782130in}{0.403703in}}%
\pgfpathlineto{\pgfqpoint{2.753915in}{0.405679in}}%
\pgfpathlineto{\pgfqpoint{2.743337in}{0.408451in}}%
\pgfpathlineto{\pgfqpoint{2.737726in}{0.412196in}}%
\pgfpathlineto{\pgfqpoint{2.733676in}{0.418002in}}%
\pgfpathlineto{\pgfqpoint{2.730655in}{0.427314in}}%
\pgfpathlineto{\pgfqpoint{2.728393in}{0.442010in}}%
\pgfpathlineto{\pgfqpoint{2.726547in}{0.471800in}}%
\pgfpathlineto{\pgfqpoint{2.725218in}{0.534009in}}%
\pgfpathlineto{\pgfqpoint{2.725171in}{0.655978in}}%
\pgfpathlineto{\pgfqpoint{2.727378in}{0.832692in}}%
\pgfpathlineto{\pgfqpoint{2.732260in}{1.041709in}}%
\pgfpathlineto{\pgfqpoint{2.738852in}{1.223262in}}%
\pgfpathlineto{\pgfqpoint{2.747079in}{1.389771in}}%
\pgfpathlineto{\pgfqpoint{2.756610in}{1.538723in}}%
\pgfpathlineto{\pgfqpoint{2.768956in}{1.694893in}}%
\pgfpathlineto{\pgfqpoint{2.781229in}{1.816050in}}%
\pgfpathlineto{\pgfqpoint{2.794403in}{1.924530in}}%
\pgfpathlineto{\pgfqpoint{2.812739in}{2.054728in}}%
\pgfpathlineto{\pgfqpoint{2.828776in}{2.147518in}}%
\pgfpathlineto{\pgfqpoint{2.847385in}{2.242230in}}%
\pgfpathlineto{\pgfqpoint{2.895821in}{2.479705in}}%
\pgfpathlineto{\pgfqpoint{2.900207in}{2.516694in}}%
\pgfpathlineto{\pgfqpoint{2.901349in}{2.544034in}}%
\pgfpathlineto{\pgfqpoint{2.900294in}{2.566394in}}%
\pgfpathlineto{\pgfqpoint{2.897337in}{2.586004in}}%
\pgfpathlineto{\pgfqpoint{2.892838in}{2.602638in}}%
\pgfpathlineto{\pgfqpoint{2.886396in}{2.618410in}}%
\pgfpathlineto{\pgfqpoint{2.878059in}{2.632974in}}%
\pgfpathlineto{\pgfqpoint{2.868066in}{2.646105in}}%
\pgfpathlineto{\pgfqpoint{2.855051in}{2.659304in}}%
\pgfpathlineto{\pgfqpoint{2.840801in}{2.670720in}}%
\pgfpathlineto{\pgfqpoint{2.821823in}{2.682864in}}%
\pgfpathlineto{\pgfqpoint{2.799981in}{2.694029in}}%
\pgfpathlineto{\pgfqpoint{2.773366in}{2.704947in}}%
\pgfpathlineto{\pgfqpoint{2.742012in}{2.715268in}}%
\pgfpathlineto{\pgfqpoint{2.705983in}{2.724787in}}%
\pgfpathlineto{\pgfqpoint{2.663200in}{2.733812in}}%
\pgfpathlineto{\pgfqpoint{2.611535in}{2.742381in}}%
\pgfpathlineto{\pgfqpoint{2.551002in}{2.750092in}}%
\pgfpathlineto{\pgfqpoint{2.481632in}{2.756684in}}%
\pgfpathlineto{\pgfqpoint{2.399112in}{2.762202in}}%
\pgfpathlineto{\pgfqpoint{2.309985in}{2.765887in}}%
\pgfpathlineto{\pgfqpoint{2.188184in}{2.768098in}}%
\pgfpathlineto{\pgfqpoint{2.081595in}{2.767620in}}%
\pgfpathlineto{\pgfqpoint{1.968506in}{2.764841in}}%
\pgfpathlineto{\pgfqpoint{1.864180in}{2.759919in}}%
\pgfpathlineto{\pgfqpoint{1.757786in}{2.752594in}}%
\pgfpathlineto{\pgfqpoint{1.671087in}{2.744172in}}%
\pgfpathlineto{\pgfqpoint{1.591075in}{2.734194in}}%
\pgfpathlineto{\pgfqpoint{1.502689in}{2.720719in}}%
\pgfpathlineto{\pgfqpoint{1.427655in}{2.706083in}}%
\pgfpathlineto{\pgfqpoint{1.372350in}{2.692544in}}%
\pgfpathlineto{\pgfqpoint{1.321734in}{2.677921in}}%
\pgfpathlineto{\pgfqpoint{1.273765in}{2.661664in}}%
\pgfpathlineto{\pgfqpoint{1.230567in}{2.644672in}}%
\pgfpathlineto{\pgfqpoint{1.192197in}{2.627106in}}%
\pgfpathlineto{\pgfqpoint{1.156620in}{2.608403in}}%
\pgfpathlineto{\pgfqpoint{1.123890in}{2.588717in}}%
\pgfpathlineto{\pgfqpoint{1.095883in}{2.569568in}}%
\pgfpathlineto{\pgfqpoint{1.063936in}{2.543701in}}%
\pgfpathlineto{\pgfqpoint{1.038216in}{2.520733in}}%
\pgfpathlineto{\pgfqpoint{1.013766in}{2.496017in}}%
\pgfpathlineto{\pgfqpoint{0.990704in}{2.469611in}}%
\pgfpathlineto{\pgfqpoint{0.969124in}{2.441613in}}%
\pgfpathlineto{\pgfqpoint{0.949082in}{2.412155in}}%
\pgfpathlineto{\pgfqpoint{0.930603in}{2.381387in}}%
\pgfpathlineto{\pgfqpoint{0.906555in}{2.334053in}}%
\pgfpathlineto{\pgfqpoint{0.889924in}{2.296263in}}%
\pgfpathlineto{\pgfqpoint{0.874240in}{2.255214in}}%
\pgfpathlineto{\pgfqpoint{0.859667in}{2.210962in}}%
\pgfpathlineto{\pgfqpoint{0.846985in}{2.165955in}}%
\pgfpathlineto{\pgfqpoint{0.839632in}{2.134716in}}%
\pgfpathlineto{\pgfqpoint{0.828237in}{2.081533in}}%
\pgfpathlineto{\pgfqpoint{0.817866in}{2.022987in}}%
\pgfpathlineto{\pgfqpoint{0.810783in}{1.971353in}}%
\pgfpathlineto{\pgfqpoint{0.802845in}{1.902253in}}%
\pgfpathlineto{\pgfqpoint{0.796553in}{1.827928in}}%
\pgfpathlineto{\pgfqpoint{0.791695in}{1.743481in}}%
\pgfpathlineto{\pgfqpoint{0.787772in}{1.621596in}}%
\pgfpathlineto{\pgfqpoint{0.785406in}{1.522065in}}%
\pgfpathlineto{\pgfqpoint{0.785406in}{1.522065in}}%
\pgfusepath{stroke}%
\end{pgfscope}%
\begin{pgfscope}%
\pgfpathrectangle{\pgfqpoint{0.448634in}{0.402556in}}{\pgfqpoint{4.350661in}{2.489204in}} %
\pgfusepath{clip}%
\pgfsetrectcap%
\pgfsetroundjoin%
\pgfsetlinewidth{1.003750pt}%
\definecolor{currentstroke}{rgb}{0.549020,0.337255,0.294118}%
\pgfsetstrokecolor{currentstroke}%
\pgfsetdash{}{0pt}%
\pgfpathmoveto{\pgfqpoint{1.127319in}{2.572073in}}%
\pgfpathlineto{\pgfqpoint{1.159575in}{2.592758in}}%
\pgfpathlineto{\pgfqpoint{1.192763in}{2.611414in}}%
\pgfpathlineto{\pgfqpoint{1.228725in}{2.629126in}}%
\pgfpathlineto{\pgfqpoint{1.267413in}{2.645758in}}%
\pgfpathlineto{\pgfqpoint{1.310846in}{2.661945in}}%
\pgfpathlineto{\pgfqpoint{1.356920in}{2.676740in}}%
\pgfpathlineto{\pgfqpoint{1.407679in}{2.690702in}}%
\pgfpathlineto{\pgfqpoint{1.463094in}{2.703640in}}%
\pgfpathlineto{\pgfqpoint{1.525272in}{2.715812in}}%
\pgfpathlineto{\pgfqpoint{1.594198in}{2.726937in}}%
\pgfpathlineto{\pgfqpoint{1.669843in}{2.736808in}}%
\pgfpathlineto{\pgfqpoint{1.752172in}{2.745271in}}%
\pgfpathlineto{\pgfqpoint{1.843325in}{2.752344in}}%
\pgfpathlineto{\pgfqpoint{1.941103in}{2.757656in}}%
\pgfpathlineto{\pgfqpoint{2.043301in}{2.760987in}}%
\pgfpathlineto{\pgfqpoint{2.147710in}{2.762199in}}%
\pgfpathlineto{\pgfqpoint{2.249945in}{2.761215in}}%
\pgfpathlineto{\pgfqpoint{2.345620in}{2.758145in}}%
\pgfpathlineto{\pgfqpoint{2.432525in}{2.753210in}}%
\pgfpathlineto{\pgfqpoint{2.508450in}{2.746766in}}%
\pgfpathlineto{\pgfqpoint{2.573368in}{2.739156in}}%
\pgfpathlineto{\pgfqpoint{2.629410in}{2.730451in}}%
\pgfpathlineto{\pgfqpoint{2.676543in}{2.720985in}}%
\pgfpathlineto{\pgfqpoint{2.716874in}{2.710666in}}%
\pgfpathlineto{\pgfqpoint{2.750365in}{2.699848in}}%
\pgfpathlineto{\pgfqpoint{2.779059in}{2.688192in}}%
\pgfpathlineto{\pgfqpoint{2.802882in}{2.676004in}}%
\pgfpathlineto{\pgfqpoint{2.821841in}{2.663819in}}%
\pgfpathlineto{\pgfqpoint{2.837815in}{2.650886in}}%
\pgfpathlineto{\pgfqpoint{2.850735in}{2.637564in}}%
\pgfpathlineto{\pgfqpoint{2.860694in}{2.624398in}}%
\pgfpathlineto{\pgfqpoint{2.869084in}{2.609873in}}%
\pgfpathlineto{\pgfqpoint{2.875698in}{2.594192in}}%
\pgfpathlineto{\pgfqpoint{2.881035in}{2.575255in}}%
\pgfpathlineto{\pgfqpoint{2.884200in}{2.555685in}}%
\pgfpathlineto{\pgfqpoint{2.885619in}{2.533351in}}%
\pgfpathlineto{\pgfqpoint{2.885038in}{2.505987in}}%
\pgfpathlineto{\pgfqpoint{2.882112in}{2.473807in}}%
\pgfpathlineto{\pgfqpoint{2.875657in}{2.429620in}}%
\pgfpathlineto{\pgfqpoint{2.863489in}{2.363873in}}%
\pgfpathlineto{\pgfqpoint{2.821102in}{2.142619in}}%
\pgfpathlineto{\pgfqpoint{2.804859in}{2.042271in}}%
\pgfpathlineto{\pgfqpoint{2.790421in}{1.939040in}}%
\pgfpathlineto{\pgfqpoint{2.777207in}{1.828054in}}%
\pgfpathlineto{\pgfqpoint{2.765338in}{1.709349in}}%
\pgfpathlineto{\pgfqpoint{2.754471in}{1.578010in}}%
\pgfpathlineto{\pgfqpoint{2.744640in}{1.431580in}}%
\pgfpathlineto{\pgfqpoint{2.735914in}{1.267598in}}%
\pgfpathlineto{\pgfqpoint{2.728277in}{1.081114in}}%
\pgfpathlineto{\pgfqpoint{2.721436in}{0.857223in}}%
\pgfpathlineto{\pgfqpoint{2.711961in}{0.541290in}}%
\pgfpathlineto{\pgfqpoint{2.708250in}{0.491694in}}%
\pgfpathlineto{\pgfqpoint{2.703951in}{0.462246in}}%
\pgfpathlineto{\pgfqpoint{2.699504in}{0.445599in}}%
\pgfpathlineto{\pgfqpoint{2.694517in}{0.434563in}}%
\pgfpathlineto{\pgfqpoint{2.688941in}{0.426947in}}%
\pgfpathlineto{\pgfqpoint{2.681980in}{0.421009in}}%
\pgfpathlineto{\pgfqpoint{2.672063in}{0.415948in}}%
\pgfpathlineto{\pgfqpoint{2.659429in}{0.412247in}}%
\pgfpathlineto{\pgfqpoint{2.640044in}{0.409163in}}%
\pgfpathlineto{\pgfqpoint{2.607489in}{0.406692in}}%
\pgfpathlineto{\pgfqpoint{2.548778in}{0.404894in}}%
\pgfpathlineto{\pgfqpoint{2.422614in}{0.403701in}}%
\pgfpathlineto{\pgfqpoint{2.026705in}{0.403016in}}%
\pgfpathlineto{\pgfqpoint{0.623617in}{0.403253in}}%
\pgfpathlineto{\pgfqpoint{0.477879in}{0.404742in}}%
\pgfpathlineto{\pgfqpoint{0.458367in}{0.406382in}}%
\pgfpathlineto{\pgfqpoint{0.452303in}{0.408938in}}%
\pgfpathlineto{\pgfqpoint{0.450213in}{0.413215in}}%
\pgfpathlineto{\pgfqpoint{0.449165in}{0.423081in}}%
\pgfpathlineto{\pgfqpoint{0.448735in}{0.465392in}}%
\pgfpathlineto{\pgfqpoint{0.448637in}{0.983147in}}%
\pgfpathlineto{\pgfqpoint{0.448652in}{2.889877in}}%
\pgfpathlineto{\pgfqpoint{0.448652in}{2.889877in}}%
\pgfusepath{stroke}%
\end{pgfscope}%
\begin{pgfscope}%
\pgfpathrectangle{\pgfqpoint{0.448634in}{0.402556in}}{\pgfqpoint{4.350661in}{2.489204in}} %
\pgfusepath{clip}%
\pgfsetrectcap%
\pgfsetroundjoin%
\pgfsetlinewidth{1.003750pt}%
\definecolor{currentstroke}{rgb}{0.549020,0.337255,0.294118}%
\pgfsetstrokecolor{currentstroke}%
\pgfsetdash{}{0pt}%
\pgfpathmoveto{\pgfqpoint{0.448634in}{2.896245in}}%
\pgfpathlineto{\pgfqpoint{0.448593in}{0.407043in}}%
\pgfpathlineto{\pgfqpoint{0.448593in}{0.407043in}}%
\pgfusepath{stroke}%
\end{pgfscope}%
\begin{pgfscope}%
\pgfpathrectangle{\pgfqpoint{0.448634in}{0.402556in}}{\pgfqpoint{4.350661in}{2.489204in}} %
\pgfusepath{clip}%
\pgfsetrectcap%
\pgfsetroundjoin%
\pgfsetlinewidth{1.003750pt}%
\definecolor{currentstroke}{rgb}{0.549020,0.337255,0.294118}%
\pgfsetstrokecolor{currentstroke}%
\pgfsetdash{}{0pt}%
\pgfpathmoveto{\pgfqpoint{0.576842in}{1.760843in}}%
\pgfpathlineto{\pgfqpoint{0.569384in}{1.840036in}}%
\pgfpathlineto{\pgfqpoint{0.563200in}{1.929365in}}%
\pgfpathlineto{\pgfqpoint{0.558585in}{2.028790in}}%
\pgfpathlineto{\pgfqpoint{0.555979in}{2.133292in}}%
\pgfpathlineto{\pgfqpoint{0.555560in}{2.237835in}}%
\pgfpathlineto{\pgfqpoint{0.557366in}{2.337378in}}%
\pgfpathlineto{\pgfqpoint{0.561092in}{2.424393in}}%
\pgfpathlineto{\pgfqpoint{0.566399in}{2.498818in}}%
\pgfpathlineto{\pgfqpoint{0.572906in}{2.560596in}}%
\pgfpathlineto{\pgfqpoint{0.580456in}{2.612145in}}%
\pgfpathlineto{\pgfqpoint{0.589085in}{2.655842in}}%
\pgfpathlineto{\pgfqpoint{0.598406in}{2.691615in}}%
\pgfpathlineto{\pgfqpoint{0.608615in}{2.721782in}}%
\pgfpathlineto{\pgfqpoint{0.619244in}{2.746302in}}%
\pgfpathlineto{\pgfqpoint{0.630822in}{2.767362in}}%
\pgfpathlineto{\pgfqpoint{0.642982in}{2.784905in}}%
\pgfpathlineto{\pgfqpoint{0.656822in}{2.800731in}}%
\pgfpathlineto{\pgfqpoint{0.672207in}{2.814565in}}%
\pgfpathlineto{\pgfqpoint{0.688865in}{2.826316in}}%
\pgfpathlineto{\pgfqpoint{0.706474in}{2.836088in}}%
\pgfpathlineto{\pgfqpoint{0.726817in}{2.844885in}}%
\pgfpathlineto{\pgfqpoint{0.751880in}{2.853211in}}%
\pgfpathlineto{\pgfqpoint{0.781646in}{2.860554in}}%
\pgfpathlineto{\pgfqpoint{0.818182in}{2.867058in}}%
\pgfpathlineto{\pgfqpoint{0.863595in}{2.872688in}}%
\pgfpathlineto{\pgfqpoint{0.922175in}{2.877520in}}%
\pgfpathlineto{\pgfqpoint{1.000405in}{2.881568in}}%
\pgfpathlineto{\pgfqpoint{1.111308in}{2.884882in}}%
\pgfpathlineto{\pgfqpoint{1.274443in}{2.887368in}}%
\pgfpathlineto{\pgfqpoint{1.552880in}{2.889263in}}%
\pgfpathlineto{\pgfqpoint{2.107588in}{2.890457in}}%
\pgfpathlineto{\pgfqpoint{3.343175in}{2.890573in}}%
\pgfpathlineto{\pgfqpoint{4.043630in}{2.888941in}}%
\pgfpathlineto{\pgfqpoint{4.289431in}{2.886404in}}%
\pgfpathlineto{\pgfqpoint{4.413389in}{2.883093in}}%
\pgfpathlineto{\pgfqpoint{4.489439in}{2.878997in}}%
\pgfpathlineto{\pgfqpoint{4.541466in}{2.874080in}}%
\pgfpathlineto{\pgfqpoint{4.578115in}{2.868469in}}%
\pgfpathlineto{\pgfqpoint{4.605833in}{2.862091in}}%
\pgfpathlineto{\pgfqpoint{4.626740in}{2.855243in}}%
\pgfpathlineto{\pgfqpoint{4.644939in}{2.847015in}}%
\pgfpathlineto{\pgfqpoint{4.660254in}{2.837585in}}%
\pgfpathlineto{\pgfqpoint{4.672636in}{2.827463in}}%
\pgfpathlineto{\pgfqpoint{4.683762in}{2.815585in}}%
\pgfpathlineto{\pgfqpoint{4.693416in}{2.802126in}}%
\pgfpathlineto{\pgfqpoint{4.702749in}{2.785334in}}%
\pgfpathlineto{\pgfqpoint{4.711285in}{2.765184in}}%
\pgfpathlineto{\pgfqpoint{4.719489in}{2.739473in}}%
\pgfpathlineto{\pgfqpoint{4.726299in}{2.710647in}}%
\pgfpathlineto{\pgfqpoint{4.733264in}{2.671632in}}%
\pgfpathlineto{\pgfqpoint{4.739608in}{2.622385in}}%
\pgfpathlineto{\pgfqpoint{4.745239in}{2.560493in}}%
\pgfpathlineto{\pgfqpoint{4.750167in}{2.481040in}}%
\pgfpathlineto{\pgfqpoint{4.754369in}{2.376607in}}%
\pgfpathlineto{\pgfqpoint{4.757445in}{2.242238in}}%
\pgfpathlineto{\pgfqpoint{4.758979in}{2.075472in}}%
\pgfpathlineto{\pgfqpoint{4.758448in}{1.888784in}}%
\pgfpathlineto{\pgfqpoint{4.755757in}{1.707099in}}%
\pgfpathlineto{\pgfqpoint{4.750927in}{1.532946in}}%
\pgfpathlineto{\pgfqpoint{4.744787in}{1.398715in}}%
\pgfpathlineto{\pgfqpoint{4.737576in}{1.289504in}}%
\pgfpathlineto{\pgfqpoint{4.728716in}{1.190458in}}%
\pgfpathlineto{\pgfqpoint{4.719654in}{1.116510in}}%
\pgfpathlineto{\pgfqpoint{4.710038in}{1.055265in}}%
\pgfpathlineto{\pgfqpoint{4.699505in}{1.001850in}}%
\pgfpathlineto{\pgfqpoint{4.689042in}{0.958679in}}%
\pgfpathlineto{\pgfqpoint{4.677221in}{0.918589in}}%
\pgfpathlineto{\pgfqpoint{4.664036in}{0.881738in}}%
\pgfpathlineto{\pgfqpoint{4.650585in}{0.850480in}}%
\pgfpathlineto{\pgfqpoint{4.636304in}{0.822559in}}%
\pgfpathlineto{\pgfqpoint{4.620208in}{0.795964in}}%
\pgfpathlineto{\pgfqpoint{4.603641in}{0.772891in}}%
\pgfpathlineto{\pgfqpoint{4.585489in}{0.751436in}}%
\pgfpathlineto{\pgfqpoint{4.565874in}{0.731738in}}%
\pgfpathlineto{\pgfqpoint{4.544964in}{0.713869in}}%
\pgfpathlineto{\pgfqpoint{4.522958in}{0.697814in}}%
\pgfpathlineto{\pgfqpoint{4.496157in}{0.681281in}}%
\pgfpathlineto{\pgfqpoint{4.470398in}{0.667943in}}%
\pgfpathlineto{\pgfqpoint{4.439961in}{0.654500in}}%
\pgfpathlineto{\pgfqpoint{4.406841in}{0.642273in}}%
\pgfpathlineto{\pgfqpoint{4.369009in}{0.630740in}}%
\pgfpathlineto{\pgfqpoint{4.326489in}{0.620218in}}%
\pgfpathlineto{\pgfqpoint{4.279327in}{0.610941in}}%
\pgfpathlineto{\pgfqpoint{4.227576in}{0.603077in}}%
\pgfpathlineto{\pgfqpoint{4.173450in}{0.597055in}}%
\pgfpathlineto{\pgfqpoint{4.110511in}{0.592195in}}%
\pgfpathlineto{\pgfqpoint{4.047471in}{0.589529in}}%
\pgfpathlineto{\pgfqpoint{3.977867in}{0.588616in}}%
\pgfpathlineto{\pgfqpoint{3.906092in}{0.589925in}}%
\pgfpathlineto{\pgfqpoint{3.834376in}{0.593488in}}%
\pgfpathlineto{\pgfqpoint{3.767119in}{0.599059in}}%
\pgfpathlineto{\pgfqpoint{3.704364in}{0.606384in}}%
\pgfpathlineto{\pgfqpoint{3.678517in}{0.610507in}}%
\pgfpathlineto{\pgfqpoint{3.620439in}{0.620497in}}%
\pgfpathlineto{\pgfqpoint{3.586320in}{0.628204in}}%
\pgfpathlineto{\pgfqpoint{3.495242in}{0.652427in}}%
\pgfpathlineto{\pgfqpoint{3.451529in}{0.667582in}}%
\pgfpathlineto{\pgfqpoint{3.408539in}{0.685219in}}%
\pgfpathlineto{\pgfqpoint{3.374595in}{0.702000in}}%
\pgfpathlineto{\pgfqpoint{3.345408in}{0.718681in}}%
\pgfpathlineto{\pgfqpoint{3.315237in}{0.738519in}}%
\pgfpathlineto{\pgfqpoint{3.288128in}{0.759288in}}%
\pgfpathlineto{\pgfqpoint{3.264005in}{0.780549in}}%
\pgfpathlineto{\pgfqpoint{3.241209in}{0.803647in}}%
\pgfpathlineto{\pgfqpoint{3.219895in}{0.828528in}}%
\pgfpathlineto{\pgfqpoint{3.200190in}{0.855090in}}%
\pgfpathlineto{\pgfqpoint{3.182178in}{0.883181in}}%
\pgfpathlineto{\pgfqpoint{3.165907in}{0.912631in}}%
\pgfpathlineto{\pgfqpoint{3.150352in}{0.945446in}}%
\pgfpathlineto{\pgfqpoint{3.136683in}{0.979343in}}%
\pgfpathlineto{\pgfqpoint{3.124073in}{1.016458in}}%
\pgfpathlineto{\pgfqpoint{3.112834in}{1.056767in}}%
\pgfpathlineto{\pgfqpoint{3.103046in}{1.100144in}}%
\pgfpathlineto{\pgfqpoint{3.095344in}{1.144069in}}%
\pgfpathlineto{\pgfqpoint{3.089209in}{1.190836in}}%
\pgfpathlineto{\pgfqpoint{3.084596in}{1.242837in}}%
\pgfpathlineto{\pgfqpoint{3.082137in}{1.295030in}}%
\pgfpathlineto{\pgfqpoint{3.081687in}{1.349785in}}%
\pgfpathlineto{\pgfqpoint{3.083451in}{1.406997in}}%
\pgfpathlineto{\pgfqpoint{3.087182in}{1.461588in}}%
\pgfpathlineto{\pgfqpoint{3.093486in}{1.520886in}}%
\pgfpathlineto{\pgfqpoint{3.101824in}{1.577332in}}%
\pgfpathlineto{\pgfqpoint{3.111929in}{1.630855in}}%
\pgfpathlineto{\pgfqpoint{3.124690in}{1.686207in}}%
\pgfpathlineto{\pgfqpoint{3.139177in}{1.738394in}}%
\pgfpathlineto{\pgfqpoint{3.155144in}{1.787365in}}%
\pgfpathlineto{\pgfqpoint{3.172352in}{1.833083in}}%
\pgfpathlineto{\pgfqpoint{3.191617in}{1.877715in}}%
\pgfpathlineto{\pgfqpoint{3.214025in}{1.923260in}}%
\pgfpathlineto{\pgfqpoint{3.236213in}{1.963156in}}%
\pgfpathlineto{\pgfqpoint{3.260177in}{2.001683in}}%
\pgfpathlineto{\pgfqpoint{3.285813in}{2.038775in}}%
\pgfpathlineto{\pgfqpoint{3.314414in}{2.076284in}}%
\pgfpathlineto{\pgfqpoint{3.348943in}{2.117710in}}%
\pgfpathlineto{\pgfqpoint{3.417132in}{2.198021in}}%
\pgfpathlineto{\pgfqpoint{3.426053in}{2.212127in}}%
\pgfpathlineto{\pgfqpoint{3.430798in}{2.223296in}}%
\pgfpathlineto{\pgfqpoint{3.432034in}{2.230602in}}%
\pgfpathlineto{\pgfqpoint{3.430773in}{2.237855in}}%
\pgfpathlineto{\pgfqpoint{3.426622in}{2.243525in}}%
\pgfpathlineto{\pgfqpoint{3.420909in}{2.247083in}}%
\pgfpathlineto{\pgfqpoint{3.412501in}{2.249582in}}%
\pgfpathlineto{\pgfqpoint{3.399499in}{2.250688in}}%
\pgfpathlineto{\pgfqpoint{3.384306in}{2.249670in}}%
\pgfpathlineto{\pgfqpoint{3.364986in}{2.246097in}}%
\pgfpathlineto{\pgfqpoint{3.341804in}{2.239342in}}%
\pgfpathlineto{\pgfqpoint{3.317110in}{2.229681in}}%
\pgfpathlineto{\pgfqpoint{3.291105in}{2.216985in}}%
\pgfpathlineto{\pgfqpoint{3.265928in}{2.202261in}}%
\pgfpathlineto{\pgfqpoint{3.239805in}{2.184361in}}%
\pgfpathlineto{\pgfqpoint{3.214776in}{2.164518in}}%
\pgfpathlineto{\pgfqpoint{3.190901in}{2.142893in}}%
\pgfpathlineto{\pgfqpoint{3.166657in}{2.117912in}}%
\pgfpathlineto{\pgfqpoint{3.143836in}{2.091233in}}%
\pgfpathlineto{\pgfqpoint{3.121079in}{2.061107in}}%
\pgfpathlineto{\pgfqpoint{3.099952in}{2.029463in}}%
\pgfpathlineto{\pgfqpoint{3.079251in}{1.994405in}}%
\pgfpathlineto{\pgfqpoint{3.059218in}{1.955915in}}%
\pgfpathlineto{\pgfqpoint{3.040059in}{1.914015in}}%
\pgfpathlineto{\pgfqpoint{3.022809in}{1.871041in}}%
\pgfpathlineto{\pgfqpoint{3.005790in}{1.822536in}}%
\pgfpathlineto{\pgfqpoint{2.990067in}{1.770819in}}%
\pgfpathlineto{\pgfqpoint{2.975708in}{1.715979in}}%
\pgfpathlineto{\pgfqpoint{2.962284in}{1.655680in}}%
\pgfpathlineto{\pgfqpoint{2.950496in}{1.592386in}}%
\pgfpathlineto{\pgfqpoint{2.940383in}{1.526185in}}%
\pgfpathlineto{\pgfqpoint{2.931745in}{1.454681in}}%
\pgfpathlineto{\pgfqpoint{2.925082in}{1.380399in}}%
\pgfpathlineto{\pgfqpoint{2.920647in}{1.305899in}}%
\pgfpathlineto{\pgfqpoint{2.918444in}{1.231270in}}%
\pgfpathlineto{\pgfqpoint{2.918545in}{1.159087in}}%
\pgfpathlineto{\pgfqpoint{2.920787in}{1.091931in}}%
\pgfpathlineto{\pgfqpoint{2.925177in}{1.027412in}}%
\pgfpathlineto{\pgfqpoint{2.931192in}{0.970580in}}%
\pgfpathlineto{\pgfqpoint{2.938760in}{0.919034in}}%
\pgfpathlineto{\pgfqpoint{2.947651in}{0.872852in}}%
\pgfpathlineto{\pgfqpoint{2.958213in}{0.829714in}}%
\pgfpathlineto{\pgfqpoint{2.969670in}{0.792114in}}%
\pgfpathlineto{\pgfqpoint{2.982463in}{0.757774in}}%
\pgfpathlineto{\pgfqpoint{2.996425in}{0.726812in}}%
\pgfpathlineto{\pgfqpoint{3.011299in}{0.699300in}}%
\pgfpathlineto{\pgfqpoint{3.026739in}{0.675225in}}%
\pgfpathlineto{\pgfqpoint{3.043828in}{0.652656in}}%
\pgfpathlineto{\pgfqpoint{3.062495in}{0.631788in}}%
\pgfpathlineto{\pgfqpoint{3.082602in}{0.612753in}}%
\pgfpathlineto{\pgfqpoint{3.103961in}{0.595592in}}%
\pgfpathlineto{\pgfqpoint{3.128268in}{0.579069in}}%
\pgfpathlineto{\pgfqpoint{3.153537in}{0.564554in}}%
\pgfpathlineto{\pgfqpoint{3.181571in}{0.550952in}}%
\pgfpathlineto{\pgfqpoint{3.214371in}{0.537647in}}%
\pgfpathlineto{\pgfqpoint{3.249846in}{0.525712in}}%
\pgfpathlineto{\pgfqpoint{3.290011in}{0.514571in}}%
\pgfpathlineto{\pgfqpoint{3.334821in}{0.504423in}}%
\pgfpathlineto{\pgfqpoint{3.386372in}{0.494999in}}%
\pgfpathlineto{\pgfqpoint{3.446798in}{0.486257in}}%
\pgfpathlineto{\pgfqpoint{3.518243in}{0.478282in}}%
\pgfpathlineto{\pgfqpoint{3.600685in}{0.471409in}}%
\pgfpathlineto{\pgfqpoint{3.696269in}{0.465713in}}%
\pgfpathlineto{\pgfqpoint{3.807144in}{0.461369in}}%
\pgfpathlineto{\pgfqpoint{3.933291in}{0.458719in}}%
\pgfpathlineto{\pgfqpoint{4.063809in}{0.458211in}}%
\pgfpathlineto{\pgfqpoint{4.187792in}{0.459914in}}%
\pgfpathlineto{\pgfqpoint{4.294335in}{0.463522in}}%
\pgfpathlineto{\pgfqpoint{4.381234in}{0.468575in}}%
\pgfpathlineto{\pgfqpoint{4.450636in}{0.474702in}}%
\pgfpathlineto{\pgfqpoint{4.506850in}{0.481800in}}%
\pgfpathlineto{\pgfqpoint{4.552009in}{0.489659in}}%
\pgfpathlineto{\pgfqpoint{4.588239in}{0.498116in}}%
\pgfpathlineto{\pgfqpoint{4.617656in}{0.507111in}}%
\pgfpathlineto{\pgfqpoint{4.642329in}{0.516843in}}%
\pgfpathlineto{\pgfqpoint{4.664194in}{0.527941in}}%
\pgfpathlineto{\pgfqpoint{4.681238in}{0.538946in}}%
\pgfpathlineto{\pgfqpoint{4.697164in}{0.551955in}}%
\pgfpathlineto{\pgfqpoint{4.710076in}{0.565291in}}%
\pgfpathlineto{\pgfqpoint{4.721578in}{0.580220in}}%
\pgfpathlineto{\pgfqpoint{4.731557in}{0.596522in}}%
\pgfpathlineto{\pgfqpoint{4.741000in}{0.616135in}}%
\pgfpathlineto{\pgfqpoint{4.749521in}{0.639028in}}%
\pgfpathlineto{\pgfqpoint{4.757522in}{0.667451in}}%
\pgfpathlineto{\pgfqpoint{4.764572in}{0.701346in}}%
\pgfpathlineto{\pgfqpoint{4.770840in}{0.743044in}}%
\pgfpathlineto{\pgfqpoint{4.776327in}{0.794935in}}%
\pgfpathlineto{\pgfqpoint{4.781278in}{0.864399in}}%
\pgfpathlineto{\pgfqpoint{4.785468in}{0.956372in}}%
\pgfpathlineto{\pgfqpoint{4.789000in}{1.085746in}}%
\pgfpathlineto{\pgfqpoint{4.791852in}{1.277386in}}%
\pgfpathlineto{\pgfqpoint{4.793959in}{1.581059in}}%
\pgfpathlineto{\pgfqpoint{4.794962in}{2.071430in}}%
\pgfpathlineto{\pgfqpoint{4.793967in}{2.559312in}}%
\pgfpathlineto{\pgfqpoint{4.791733in}{2.745982in}}%
\pgfpathlineto{\pgfqpoint{4.788955in}{2.818092in}}%
\pgfpathlineto{\pgfqpoint{4.785731in}{2.850228in}}%
\pgfpathlineto{\pgfqpoint{4.781878in}{2.867058in}}%
\pgfpathlineto{\pgfqpoint{4.777743in}{2.875781in}}%
\pgfpathlineto{\pgfqpoint{4.773096in}{2.880983in}}%
\pgfpathlineto{\pgfqpoint{4.767362in}{2.884504in}}%
\pgfpathlineto{\pgfqpoint{4.756852in}{2.887622in}}%
\pgfpathlineto{\pgfqpoint{4.739547in}{2.889639in}}%
\pgfpathlineto{\pgfqpoint{4.704761in}{2.890882in}}%
\pgfpathlineto{\pgfqpoint{4.602523in}{2.891538in}}%
\pgfpathlineto{\pgfqpoint{3.952099in}{2.891742in}}%
\pgfpathlineto{\pgfqpoint{0.617319in}{2.890753in}}%
\pgfpathlineto{\pgfqpoint{0.549909in}{2.888858in}}%
\pgfpathlineto{\pgfqpoint{0.521734in}{2.886179in}}%
\pgfpathlineto{\pgfqpoint{0.504665in}{2.882389in}}%
\pgfpathlineto{\pgfqpoint{0.494500in}{2.878010in}}%
\pgfpathlineto{\pgfqpoint{0.487179in}{2.872665in}}%
\pgfpathlineto{\pgfqpoint{0.481151in}{2.865517in}}%
\pgfpathlineto{\pgfqpoint{0.475663in}{2.854802in}}%
\pgfpathlineto{\pgfqpoint{0.471318in}{2.840735in}}%
\pgfpathlineto{\pgfqpoint{0.467301in}{2.818821in}}%
\pgfpathlineto{\pgfqpoint{0.463927in}{2.786698in}}%
\pgfpathlineto{\pgfqpoint{0.460918in}{2.734542in}}%
\pgfpathlineto{\pgfqpoint{0.458363in}{2.647471in}}%
\pgfpathlineto{\pgfqpoint{0.456575in}{2.523029in}}%
\pgfpathlineto{\pgfqpoint{0.456575in}{2.523029in}}%
\pgfusepath{stroke}%
\end{pgfscope}%
\begin{pgfscope}%
\pgfpathrectangle{\pgfqpoint{0.448634in}{0.402556in}}{\pgfqpoint{4.350661in}{2.489204in}} %
\pgfusepath{clip}%
\pgfsetrectcap%
\pgfsetroundjoin%
\pgfsetlinewidth{1.003750pt}%
\definecolor{currentstroke}{rgb}{0.549020,0.337255,0.294118}%
\pgfsetstrokecolor{currentstroke}%
\pgfsetdash{}{0pt}%
\pgfpathmoveto{\pgfqpoint{0.456424in}{1.370135in}}%
\pgfpathlineto{\pgfqpoint{0.459610in}{1.118754in}}%
\pgfpathlineto{\pgfqpoint{0.463695in}{0.962006in}}%
\pgfpathlineto{\pgfqpoint{0.468519in}{0.857608in}}%
\pgfpathlineto{\pgfqpoint{0.474082in}{0.783209in}}%
\pgfpathlineto{\pgfqpoint{0.480226in}{0.728905in}}%
\pgfpathlineto{\pgfqpoint{0.486971in}{0.687305in}}%
\pgfpathlineto{\pgfqpoint{0.494537in}{0.653557in}}%
\pgfpathlineto{\pgfqpoint{0.503108in}{0.625354in}}%
\pgfpathlineto{\pgfqpoint{0.512193in}{0.602748in}}%
\pgfpathlineto{\pgfqpoint{0.522201in}{0.583506in}}%
\pgfpathlineto{\pgfqpoint{0.534109in}{0.565742in}}%
\pgfpathlineto{\pgfqpoint{0.546264in}{0.551506in}}%
\pgfpathlineto{\pgfqpoint{0.559729in}{0.538906in}}%
\pgfpathlineto{\pgfqpoint{0.576131in}{0.526693in}}%
\pgfpathlineto{\pgfqpoint{0.595484in}{0.515350in}}%
\pgfpathlineto{\pgfqpoint{0.617682in}{0.505146in}}%
\pgfpathlineto{\pgfqpoint{0.642569in}{0.496153in}}%
\pgfpathlineto{\pgfqpoint{0.672127in}{0.487777in}}%
\pgfpathlineto{\pgfqpoint{0.708444in}{0.479823in}}%
\pgfpathlineto{\pgfqpoint{0.753651in}{0.472325in}}%
\pgfpathlineto{\pgfqpoint{0.807719in}{0.465660in}}%
\pgfpathlineto{\pgfqpoint{0.877117in}{0.459475in}}%
\pgfpathlineto{\pgfqpoint{0.961829in}{0.454230in}}%
\pgfpathlineto{\pgfqpoint{1.068353in}{0.449916in}}%
\pgfpathlineto{\pgfqpoint{1.201020in}{0.446839in}}%
\pgfpathlineto{\pgfqpoint{1.357638in}{0.445481in}}%
\pgfpathlineto{\pgfqpoint{1.525136in}{0.446232in}}%
\pgfpathlineto{\pgfqpoint{1.686089in}{0.449142in}}%
\pgfpathlineto{\pgfqpoint{1.823075in}{0.453747in}}%
\pgfpathlineto{\pgfqpoint{1.938246in}{0.459765in}}%
\pgfpathlineto{\pgfqpoint{2.031583in}{0.466759in}}%
\pgfpathlineto{\pgfqpoint{2.109581in}{0.474745in}}%
\pgfpathlineto{\pgfqpoint{2.174385in}{0.483535in}}%
\pgfpathlineto{\pgfqpoint{2.228140in}{0.492940in}}%
\pgfpathlineto{\pgfqpoint{2.275120in}{0.503357in}}%
\pgfpathlineto{\pgfqpoint{2.315283in}{0.514502in}}%
\pgfpathlineto{\pgfqpoint{2.350699in}{0.526660in}}%
\pgfpathlineto{\pgfqpoint{2.381322in}{0.539536in}}%
\pgfpathlineto{\pgfqpoint{2.407165in}{0.552659in}}%
\pgfpathlineto{\pgfqpoint{2.430227in}{0.566640in}}%
\pgfpathlineto{\pgfqpoint{2.452283in}{0.582603in}}%
\pgfpathlineto{\pgfqpoint{2.471392in}{0.599070in}}%
\pgfpathlineto{\pgfqpoint{2.489241in}{0.617294in}}%
\pgfpathlineto{\pgfqpoint{2.505678in}{0.637182in}}%
\pgfpathlineto{\pgfqpoint{2.520621in}{0.658558in}}%
\pgfpathlineto{\pgfqpoint{2.535214in}{0.683315in}}%
\pgfpathlineto{\pgfqpoint{2.549116in}{0.711486in}}%
\pgfpathlineto{\pgfqpoint{2.562091in}{0.743006in}}%
\pgfpathlineto{\pgfqpoint{2.574020in}{0.777753in}}%
\pgfpathlineto{\pgfqpoint{2.585503in}{0.817972in}}%
\pgfpathlineto{\pgfqpoint{2.596809in}{0.866040in}}%
\pgfpathlineto{\pgfqpoint{2.607562in}{0.921949in}}%
\pgfpathlineto{\pgfqpoint{2.617925in}{0.988100in}}%
\pgfpathlineto{\pgfqpoint{2.627958in}{1.066920in}}%
\pgfpathlineto{\pgfqpoint{2.637941in}{1.163322in}}%
\pgfpathlineto{\pgfqpoint{2.648424in}{1.287201in}}%
\pgfpathlineto{\pgfqpoint{2.660103in}{1.453440in}}%
\pgfpathlineto{\pgfqpoint{2.674773in}{1.696803in}}%
\pgfpathlineto{\pgfqpoint{2.687717in}{1.945280in}}%
\pgfpathlineto{\pgfqpoint{2.692671in}{2.079575in}}%
\pgfpathlineto{\pgfqpoint{2.693829in}{2.166684in}}%
\pgfpathlineto{\pgfqpoint{2.692565in}{2.233871in}}%
\pgfpathlineto{\pgfqpoint{2.689436in}{2.286016in}}%
\pgfpathlineto{\pgfqpoint{2.684859in}{2.328001in}}%
\pgfpathlineto{\pgfqpoint{2.678725in}{2.364665in}}%
\pgfpathlineto{\pgfqpoint{2.671356in}{2.395899in}}%
\pgfpathlineto{\pgfqpoint{2.662489in}{2.423983in}}%
\pgfpathlineto{\pgfqpoint{2.652361in}{2.448780in}}%
\pgfpathlineto{\pgfqpoint{2.641364in}{2.470247in}}%
\pgfpathlineto{\pgfqpoint{2.628642in}{2.490426in}}%
\pgfpathlineto{\pgfqpoint{2.614278in}{2.509107in}}%
\pgfpathlineto{\pgfqpoint{2.598442in}{2.526160in}}%
\pgfpathlineto{\pgfqpoint{2.579589in}{2.543006in}}%
\pgfpathlineto{\pgfqpoint{2.559531in}{2.557924in}}%
\pgfpathlineto{\pgfqpoint{2.536601in}{2.572184in}}%
\pgfpathlineto{\pgfqpoint{2.510848in}{2.585539in}}%
\pgfpathlineto{\pgfqpoint{2.482359in}{2.597838in}}%
\pgfpathlineto{\pgfqpoint{2.449133in}{2.609684in}}%
\pgfpathlineto{\pgfqpoint{2.411183in}{2.620697in}}%
\pgfpathlineto{\pgfqpoint{2.368551in}{2.630607in}}%
\pgfpathlineto{\pgfqpoint{2.321293in}{2.639221in}}%
\pgfpathlineto{\pgfqpoint{2.269466in}{2.646399in}}%
\pgfpathlineto{\pgfqpoint{2.210953in}{2.652193in}}%
\pgfpathlineto{\pgfqpoint{2.147966in}{2.656153in}}%
\pgfpathlineto{\pgfqpoint{2.080555in}{2.658135in}}%
\pgfpathlineto{\pgfqpoint{2.010947in}{2.657971in}}%
\pgfpathlineto{\pgfqpoint{1.939194in}{2.655572in}}%
\pgfpathlineto{\pgfqpoint{1.867526in}{2.650913in}}%
\pgfpathlineto{\pgfqpoint{1.798170in}{2.644140in}}%
\pgfpathlineto{\pgfqpoint{1.733340in}{2.635606in}}%
\pgfpathlineto{\pgfqpoint{1.673074in}{2.625521in}}%
\pgfpathlineto{\pgfqpoint{1.615273in}{2.613610in}}%
\pgfpathlineto{\pgfqpoint{1.562132in}{2.600401in}}%
\pgfpathlineto{\pgfqpoint{1.513680in}{2.586139in}}%
\pgfpathlineto{\pgfqpoint{1.467860in}{2.570343in}}%
\pgfpathlineto{\pgfqpoint{1.426792in}{2.553922in}}%
\pgfpathlineto{\pgfqpoint{1.388446in}{2.536288in}}%
\pgfpathlineto{\pgfqpoint{1.352877in}{2.517565in}}%
\pgfpathlineto{\pgfqpoint{1.320127in}{2.497921in}}%
\pgfpathlineto{\pgfqpoint{1.288378in}{2.476235in}}%
\pgfpathlineto{\pgfqpoint{1.259591in}{2.453860in}}%
\pgfpathlineto{\pgfqpoint{1.232050in}{2.429519in}}%
\pgfpathlineto{\pgfqpoint{1.207526in}{2.404897in}}%
\pgfpathlineto{\pgfqpoint{1.184408in}{2.378556in}}%
\pgfpathlineto{\pgfqpoint{1.162827in}{2.350560in}}%
\pgfpathlineto{\pgfqpoint{1.142890in}{2.321010in}}%
\pgfpathlineto{\pgfqpoint{1.124674in}{2.290040in}}%
\pgfpathlineto{\pgfqpoint{1.108225in}{2.257801in}}%
\pgfpathlineto{\pgfqpoint{1.092639in}{2.222198in}}%
\pgfpathlineto{\pgfqpoint{1.079059in}{2.185534in}}%
\pgfpathlineto{\pgfqpoint{1.067443in}{2.147996in}}%
\pgfpathlineto{\pgfqpoint{1.057187in}{2.107346in}}%
\pgfpathlineto{\pgfqpoint{1.049004in}{2.066084in}}%
\pgfpathlineto{\pgfqpoint{1.042513in}{2.021905in}}%
\pgfpathlineto{\pgfqpoint{1.038177in}{1.977380in}}%
\pgfpathlineto{\pgfqpoint{1.035866in}{1.930165in}}%
\pgfpathlineto{\pgfqpoint{1.035826in}{1.882876in}}%
\pgfpathlineto{\pgfqpoint{1.038031in}{1.835655in}}%
\pgfpathlineto{\pgfqpoint{1.042475in}{1.788640in}}%
\pgfpathlineto{\pgfqpoint{1.049176in}{1.741977in}}%
\pgfpathlineto{\pgfqpoint{1.057644in}{1.698237in}}%
\pgfpathlineto{\pgfqpoint{1.068222in}{1.655104in}}%
\pgfpathlineto{\pgfqpoint{1.080963in}{1.612743in}}%
\pgfpathlineto{\pgfqpoint{1.095031in}{1.573615in}}%
\pgfpathlineto{\pgfqpoint{1.111116in}{1.535518in}}%
\pgfpathlineto{\pgfqpoint{1.128119in}{1.500773in}}%
\pgfpathlineto{\pgfqpoint{1.146931in}{1.467272in}}%
\pgfpathlineto{\pgfqpoint{1.167532in}{1.435180in}}%
\pgfpathlineto{\pgfqpoint{1.189875in}{1.404651in}}%
\pgfpathlineto{\pgfqpoint{1.213885in}{1.375827in}}%
\pgfpathlineto{\pgfqpoint{1.237819in}{1.350456in}}%
\pgfpathlineto{\pgfqpoint{1.264750in}{1.325237in}}%
\pgfpathlineto{\pgfqpoint{1.292992in}{1.301971in}}%
\pgfpathlineto{\pgfqpoint{1.322399in}{1.280677in}}%
\pgfpathlineto{\pgfqpoint{1.352822in}{1.261340in}}%
\pgfpathlineto{\pgfqpoint{1.386096in}{1.242889in}}%
\pgfpathlineto{\pgfqpoint{1.420192in}{1.226516in}}%
\pgfpathlineto{\pgfqpoint{1.457026in}{1.211329in}}%
\pgfpathlineto{\pgfqpoint{1.496556in}{1.197536in}}%
\pgfpathlineto{\pgfqpoint{1.538721in}{1.185287in}}%
\pgfpathlineto{\pgfqpoint{1.583443in}{1.174641in}}%
\pgfpathlineto{\pgfqpoint{1.634931in}{1.164775in}}%
\pgfpathlineto{\pgfqpoint{1.706065in}{1.153745in}}%
\pgfpathlineto{\pgfqpoint{1.768494in}{1.143417in}}%
\pgfpathlineto{\pgfqpoint{1.796124in}{1.136567in}}%
\pgfpathlineto{\pgfqpoint{1.812685in}{1.130481in}}%
\pgfpathlineto{\pgfqpoint{1.824472in}{1.124102in}}%
\pgfpathlineto{\pgfqpoint{1.833211in}{1.116741in}}%
\pgfpathlineto{\pgfqpoint{1.838500in}{1.108889in}}%
\pgfpathlineto{\pgfqpoint{1.840590in}{1.101848in}}%
\pgfpathlineto{\pgfqpoint{1.840620in}{1.094411in}}%
\pgfpathlineto{\pgfqpoint{1.837931in}{1.084985in}}%
\pgfpathlineto{\pgfqpoint{1.833247in}{1.076614in}}%
\pgfpathlineto{\pgfqpoint{1.825819in}{1.067542in}}%
\pgfpathlineto{\pgfqpoint{1.813813in}{1.056849in}}%
\pgfpathlineto{\pgfqpoint{1.798819in}{1.046763in}}%
\pgfpathlineto{\pgfqpoint{1.781016in}{1.037462in}}%
\pgfpathlineto{\pgfqpoint{1.758447in}{1.028391in}}%
\pgfpathlineto{\pgfqpoint{1.733203in}{1.020815in}}%
\pgfpathlineto{\pgfqpoint{1.705410in}{1.014872in}}%
\pgfpathlineto{\pgfqpoint{1.675178in}{1.010714in}}%
\pgfpathlineto{\pgfqpoint{1.642610in}{1.008507in}}%
\pgfpathlineto{\pgfqpoint{1.607809in}{1.008432in}}%
\pgfpathlineto{\pgfqpoint{1.570886in}{1.010691in}}%
\pgfpathlineto{\pgfqpoint{1.534118in}{1.015181in}}%
\pgfpathlineto{\pgfqpoint{1.495455in}{1.022233in}}%
\pgfpathlineto{\pgfqpoint{1.457161in}{1.031564in}}%
\pgfpathlineto{\pgfqpoint{1.419338in}{1.043132in}}%
\pgfpathlineto{\pgfqpoint{1.382089in}{1.056929in}}%
\pgfpathlineto{\pgfqpoint{1.347544in}{1.072019in}}%
\pgfpathlineto{\pgfqpoint{1.313727in}{1.089133in}}%
\pgfpathlineto{\pgfqpoint{1.280762in}{1.108299in}}%
\pgfpathlineto{\pgfqpoint{1.248782in}{1.129536in}}%
\pgfpathlineto{\pgfqpoint{1.219708in}{1.151422in}}%
\pgfpathlineto{\pgfqpoint{1.191752in}{1.175138in}}%
\pgfpathlineto{\pgfqpoint{1.165031in}{1.200649in}}%
\pgfpathlineto{\pgfqpoint{1.139653in}{1.227898in}}%
\pgfpathlineto{\pgfqpoint{1.115714in}{1.256800in}}%
\pgfpathlineto{\pgfqpoint{1.093288in}{1.287251in}}%
\pgfpathlineto{\pgfqpoint{1.071178in}{1.321163in}}%
\pgfpathlineto{\pgfqpoint{1.050868in}{1.356519in}}%
\pgfpathlineto{\pgfqpoint{1.032365in}{1.393151in}}%
\pgfpathlineto{\pgfqpoint{1.014717in}{1.433142in}}%
\pgfpathlineto{\pgfqpoint{0.999023in}{1.474185in}}%
\pgfpathlineto{\pgfqpoint{0.984506in}{1.518461in}}%
\pgfpathlineto{\pgfqpoint{0.972009in}{1.563537in}}%
\pgfpathlineto{\pgfqpoint{0.960943in}{1.611678in}}%
\pgfpathlineto{\pgfqpoint{0.951530in}{1.662824in}}%
\pgfpathlineto{\pgfqpoint{0.944286in}{1.714431in}}%
\pgfpathlineto{\pgfqpoint{0.938949in}{1.768847in}}%
\pgfpathlineto{\pgfqpoint{0.935870in}{1.823491in}}%
\pgfpathlineto{\pgfqpoint{0.935034in}{1.878240in}}%
\pgfpathlineto{\pgfqpoint{0.936465in}{1.932972in}}%
\pgfpathlineto{\pgfqpoint{0.940005in}{1.985084in}}%
\pgfpathlineto{\pgfqpoint{0.945758in}{2.036935in}}%
\pgfpathlineto{\pgfqpoint{0.953409in}{2.085938in}}%
\pgfpathlineto{\pgfqpoint{0.962764in}{2.132000in}}%
\pgfpathlineto{\pgfqpoint{0.974286in}{2.177413in}}%
\pgfpathlineto{\pgfqpoint{0.987332in}{2.219652in}}%
\pgfpathlineto{\pgfqpoint{1.001667in}{2.258654in}}%
\pgfpathlineto{\pgfqpoint{1.018050in}{2.296583in}}%
\pgfpathlineto{\pgfqpoint{1.035401in}{2.331101in}}%
\pgfpathlineto{\pgfqpoint{1.054649in}{2.364275in}}%
\pgfpathlineto{\pgfqpoint{1.074406in}{2.393983in}}%
\pgfpathlineto{\pgfqpoint{1.095770in}{2.422197in}}%
\pgfpathlineto{\pgfqpoint{1.118661in}{2.448797in}}%
\pgfpathlineto{\pgfqpoint{1.142966in}{2.473701in}}%
\pgfpathlineto{\pgfqpoint{1.168550in}{2.496867in}}%
\pgfpathlineto{\pgfqpoint{1.197084in}{2.519662in}}%
\pgfpathlineto{\pgfqpoint{1.226726in}{2.540526in}}%
\pgfpathlineto{\pgfqpoint{1.259241in}{2.560673in}}%
\pgfpathlineto{\pgfqpoint{1.294611in}{2.579881in}}%
\pgfpathlineto{\pgfqpoint{1.332791in}{2.597982in}}%
\pgfpathlineto{\pgfqpoint{1.373718in}{2.614859in}}%
\pgfpathlineto{\pgfqpoint{1.417318in}{2.630445in}}%
\pgfpathlineto{\pgfqpoint{1.465631in}{2.645312in}}%
\pgfpathlineto{\pgfqpoint{1.518639in}{2.659205in}}%
\pgfpathlineto{\pgfqpoint{1.576308in}{2.671930in}}%
\pgfpathlineto{\pgfqpoint{1.638596in}{2.683344in}}%
\pgfpathlineto{\pgfqpoint{1.705461in}{2.693343in}}%
\pgfpathlineto{\pgfqpoint{1.779026in}{2.702064in}}%
\pgfpathlineto{\pgfqpoint{1.857096in}{2.709077in}}%
\pgfpathlineto{\pgfqpoint{1.939632in}{2.714280in}}%
\pgfpathlineto{\pgfqpoint{2.026597in}{2.717514in}}%
\pgfpathlineto{\pgfqpoint{2.113604in}{2.718524in}}%
\pgfpathlineto{\pgfqpoint{2.198434in}{2.717303in}}%
\pgfpathlineto{\pgfqpoint{2.278865in}{2.713929in}}%
\pgfpathlineto{\pgfqpoint{2.352677in}{2.708598in}}%
\pgfpathlineto{\pgfqpoint{2.417656in}{2.701709in}}%
\pgfpathlineto{\pgfqpoint{2.473769in}{2.693631in}}%
\pgfpathlineto{\pgfqpoint{2.523139in}{2.684368in}}%
\pgfpathlineto{\pgfqpoint{2.565725in}{2.674203in}}%
\pgfpathlineto{\pgfqpoint{2.601509in}{2.663545in}}%
\pgfpathlineto{\pgfqpoint{2.632576in}{2.652143in}}%
\pgfpathlineto{\pgfqpoint{2.658898in}{2.640332in}}%
\pgfpathlineto{\pgfqpoint{2.682437in}{2.627437in}}%
\pgfpathlineto{\pgfqpoint{2.703062in}{2.613573in}}%
\pgfpathlineto{\pgfqpoint{2.720674in}{2.598980in}}%
\pgfpathlineto{\pgfqpoint{2.735262in}{2.584055in}}%
\pgfpathlineto{\pgfqpoint{2.748319in}{2.567379in}}%
\pgfpathlineto{\pgfqpoint{2.759553in}{2.549048in}}%
\pgfpathlineto{\pgfqpoint{2.768788in}{2.529308in}}%
\pgfpathlineto{\pgfqpoint{2.776017in}{2.508500in}}%
\pgfpathlineto{\pgfqpoint{2.781884in}{2.484542in}}%
\pgfpathlineto{\pgfqpoint{2.786103in}{2.457599in}}%
\pgfpathlineto{\pgfqpoint{2.788721in}{2.425387in}}%
\pgfpathlineto{\pgfqpoint{2.789428in}{2.388064in}}%
\pgfpathlineto{\pgfqpoint{2.787963in}{2.340804in}}%
\pgfpathlineto{\pgfqpoint{2.783673in}{2.278771in}}%
\pgfpathlineto{\pgfqpoint{2.774289in}{2.179785in}}%
\pgfpathlineto{\pgfqpoint{2.743611in}{1.868121in}}%
\pgfpathlineto{\pgfqpoint{2.730112in}{1.702063in}}%
\pgfpathlineto{\pgfqpoint{2.717287in}{1.515952in}}%
\pgfpathlineto{\pgfqpoint{2.702602in}{1.267600in}}%
\pgfpathlineto{\pgfqpoint{2.684434in}{0.964632in}}%
\pgfpathlineto{\pgfqpoint{2.675375in}{0.850602in}}%
\pgfpathlineto{\pgfqpoint{2.667030in}{0.771526in}}%
\pgfpathlineto{\pgfqpoint{2.658753in}{0.712545in}}%
\pgfpathlineto{\pgfqpoint{2.650177in}{0.666286in}}%
\pgfpathlineto{\pgfqpoint{2.640821in}{0.627933in}}%
\pgfpathlineto{\pgfqpoint{2.631146in}{0.597536in}}%
\pgfpathlineto{\pgfqpoint{2.621005in}{0.572747in}}%
\pgfpathlineto{\pgfqpoint{2.609858in}{0.551385in}}%
\pgfpathlineto{\pgfqpoint{2.598044in}{0.533535in}}%
\pgfpathlineto{\pgfqpoint{2.584497in}{0.517380in}}%
\pgfpathlineto{\pgfqpoint{2.571111in}{0.504670in}}%
\pgfpathlineto{\pgfqpoint{2.554791in}{0.492314in}}%
\pgfpathlineto{\pgfqpoint{2.537459in}{0.481915in}}%
\pgfpathlineto{\pgfqpoint{2.517376in}{0.472367in}}%
\pgfpathlineto{\pgfqpoint{2.492544in}{0.463179in}}%
\pgfpathlineto{\pgfqpoint{2.462982in}{0.454833in}}%
\pgfpathlineto{\pgfqpoint{2.428768in}{0.447542in}}%
\pgfpathlineto{\pgfqpoint{2.385674in}{0.440735in}}%
\pgfpathlineto{\pgfqpoint{2.331559in}{0.434582in}}%
\pgfpathlineto{\pgfqpoint{2.262117in}{0.429077in}}%
\pgfpathlineto{\pgfqpoint{2.170853in}{0.424236in}}%
\pgfpathlineto{\pgfqpoint{2.049088in}{0.420134in}}%
\pgfpathlineto{\pgfqpoint{1.879438in}{0.416783in}}%
\pgfpathlineto{\pgfqpoint{1.640162in}{0.414418in}}%
\pgfpathlineto{\pgfqpoint{1.322565in}{0.413569in}}%
\pgfpathlineto{\pgfqpoint{1.020196in}{0.414850in}}%
\pgfpathlineto{\pgfqpoint{0.822258in}{0.417714in}}%
\pgfpathlineto{\pgfqpoint{0.704837in}{0.421430in}}%
\pgfpathlineto{\pgfqpoint{0.630979in}{0.425828in}}%
\pgfpathlineto{\pgfqpoint{0.583318in}{0.430734in}}%
\pgfpathlineto{\pgfqpoint{0.551036in}{0.436123in}}%
\pgfpathlineto{\pgfqpoint{0.527710in}{0.442188in}}%
\pgfpathlineto{\pgfqpoint{0.511252in}{0.448624in}}%
\pgfpathlineto{\pgfqpoint{0.499551in}{0.455214in}}%
\pgfpathlineto{\pgfqpoint{0.488918in}{0.463840in}}%
\pgfpathlineto{\pgfqpoint{0.481324in}{0.472728in}}%
\pgfpathlineto{\pgfqpoint{0.474079in}{0.485124in}}%
\pgfpathlineto{\pgfqpoint{0.468754in}{0.498746in}}%
\pgfpathlineto{\pgfqpoint{0.463870in}{0.517846in}}%
\pgfpathlineto{\pgfqpoint{0.459679in}{0.544794in}}%
\pgfpathlineto{\pgfqpoint{0.456386in}{0.581936in}}%
\pgfpathlineto{\pgfqpoint{0.453731in}{0.639103in}}%
\pgfpathlineto{\pgfqpoint{0.451681in}{0.736152in}}%
\pgfpathlineto{\pgfqpoint{0.450220in}{0.927813in}}%
\pgfpathlineto{\pgfqpoint{0.449345in}{1.403249in}}%
\pgfpathlineto{\pgfqpoint{0.449542in}{2.682700in}}%
\pgfpathlineto{\pgfqpoint{0.451010in}{2.856929in}}%
\pgfpathlineto{\pgfqpoint{0.452802in}{2.879217in}}%
\pgfpathlineto{\pgfqpoint{0.455186in}{2.886105in}}%
\pgfpathlineto{\pgfqpoint{0.458623in}{2.889027in}}%
\pgfpathlineto{\pgfqpoint{0.464993in}{2.890553in}}%
\pgfpathlineto{\pgfqpoint{0.482374in}{2.891422in}}%
\pgfpathlineto{\pgfqpoint{0.565036in}{2.891729in}}%
\pgfpathlineto{\pgfqpoint{2.733840in}{2.891760in}}%
\pgfpathlineto{\pgfqpoint{4.789507in}{2.890886in}}%
\pgfpathlineto{\pgfqpoint{4.793724in}{2.889732in}}%
\pgfpathlineto{\pgfqpoint{4.795479in}{2.888309in}}%
\pgfpathlineto{\pgfqpoint{4.797105in}{2.881148in}}%
\pgfpathlineto{\pgfqpoint{4.797997in}{2.858774in}}%
\pgfpathlineto{\pgfqpoint{4.798039in}{2.856286in}}%
\pgfpathlineto{\pgfqpoint{4.798039in}{2.856286in}}%
\pgfusepath{stroke}%
\end{pgfscope}%
\begin{pgfscope}%
\pgfpathrectangle{\pgfqpoint{0.448634in}{0.402556in}}{\pgfqpoint{4.350661in}{2.489204in}} %
\pgfusepath{clip}%
\pgfsetrectcap%
\pgfsetroundjoin%
\pgfsetlinewidth{1.003750pt}%
\definecolor{currentstroke}{rgb}{0.549020,0.337255,0.294118}%
\pgfsetstrokecolor{currentstroke}%
\pgfsetdash{}{0pt}%
\pgfpathmoveto{\pgfqpoint{3.428768in}{0.402610in}}%
\pgfpathlineto{\pgfqpoint{2.806627in}{0.403757in}}%
\pgfpathlineto{\pgfqpoint{2.769687in}{0.405570in}}%
\pgfpathlineto{\pgfqpoint{2.754627in}{0.408050in}}%
\pgfpathlineto{\pgfqpoint{2.746384in}{0.411178in}}%
\pgfpathlineto{\pgfqpoint{2.740933in}{0.415240in}}%
\pgfpathlineto{\pgfqpoint{2.736772in}{0.420957in}}%
\pgfpathlineto{\pgfqpoint{2.733268in}{0.430044in}}%
\pgfpathlineto{\pgfqpoint{2.730437in}{0.444609in}}%
\pgfpathlineto{\pgfqpoint{2.728229in}{0.469365in}}%
\pgfpathlineto{\pgfqpoint{2.726463in}{0.519104in}}%
\pgfpathlineto{\pgfqpoint{2.725706in}{0.613688in}}%
\pgfpathlineto{\pgfqpoint{2.726838in}{0.768012in}}%
\pgfpathlineto{\pgfqpoint{2.730553in}{0.962122in}}%
\pgfpathlineto{\pgfqpoint{2.736608in}{1.158644in}}%
\pgfpathlineto{\pgfqpoint{2.744088in}{1.327692in}}%
\pgfpathlineto{\pgfqpoint{2.753197in}{1.484163in}}%
\pgfpathlineto{\pgfqpoint{2.763252in}{1.620583in}}%
\pgfpathlineto{\pgfqpoint{2.776113in}{1.764189in}}%
\pgfpathlineto{\pgfqpoint{2.788908in}{1.877750in}}%
\pgfpathlineto{\pgfqpoint{2.805740in}{2.005714in}}%
\pgfpathlineto{\pgfqpoint{2.821167in}{2.101172in}}%
\pgfpathlineto{\pgfqpoint{2.838349in}{2.193693in}}%
\pgfpathlineto{\pgfqpoint{2.859124in}{2.292941in}}%
\pgfpathlineto{\pgfqpoint{2.887197in}{2.425935in}}%
\pgfpathlineto{\pgfqpoint{2.896979in}{2.479535in}}%
\pgfpathlineto{\pgfqpoint{2.901531in}{2.516499in}}%
\pgfpathlineto{\pgfqpoint{2.902838in}{2.543829in}}%
\pgfpathlineto{\pgfqpoint{2.901948in}{2.566198in}}%
\pgfpathlineto{\pgfqpoint{2.899143in}{2.585839in}}%
\pgfpathlineto{\pgfqpoint{2.894787in}{2.602522in}}%
\pgfpathlineto{\pgfqpoint{2.888479in}{2.618365in}}%
\pgfpathlineto{\pgfqpoint{2.880255in}{2.633011in}}%
\pgfpathlineto{\pgfqpoint{2.870347in}{2.646226in}}%
\pgfpathlineto{\pgfqpoint{2.857400in}{2.659513in}}%
\pgfpathlineto{\pgfqpoint{2.843191in}{2.670994in}}%
\pgfpathlineto{\pgfqpoint{2.824240in}{2.683195in}}%
\pgfpathlineto{\pgfqpoint{2.802416in}{2.694406in}}%
\pgfpathlineto{\pgfqpoint{2.775812in}{2.705357in}}%
\pgfpathlineto{\pgfqpoint{2.744465in}{2.715705in}}%
\pgfpathlineto{\pgfqpoint{2.708440in}{2.725243in}}%
\pgfpathlineto{\pgfqpoint{2.665659in}{2.734281in}}%
\pgfpathlineto{\pgfqpoint{2.613996in}{2.742862in}}%
\pgfpathlineto{\pgfqpoint{2.553464in}{2.750582in}}%
\pgfpathlineto{\pgfqpoint{2.481924in}{2.757360in}}%
\pgfpathlineto{\pgfqpoint{2.399402in}{2.762834in}}%
\pgfpathlineto{\pgfqpoint{2.310274in}{2.766477in}}%
\pgfpathlineto{\pgfqpoint{2.175421in}{2.768721in}}%
\pgfpathlineto{\pgfqpoint{2.066658in}{2.767937in}}%
\pgfpathlineto{\pgfqpoint{1.953575in}{2.764855in}}%
\pgfpathlineto{\pgfqpoint{1.851433in}{2.759754in}}%
\pgfpathlineto{\pgfqpoint{1.745055in}{2.752164in}}%
\pgfpathlineto{\pgfqpoint{1.658378in}{2.743449in}}%
\pgfpathlineto{\pgfqpoint{1.580556in}{2.733456in}}%
\pgfpathlineto{\pgfqpoint{1.490062in}{2.719333in}}%
\pgfpathlineto{\pgfqpoint{1.417237in}{2.704685in}}%
\pgfpathlineto{\pgfqpoint{1.361997in}{2.690804in}}%
\pgfpathlineto{\pgfqpoint{1.311466in}{2.675803in}}%
\pgfpathlineto{\pgfqpoint{1.265673in}{2.659907in}}%
\pgfpathlineto{\pgfqpoint{1.222581in}{2.642570in}}%
\pgfpathlineto{\pgfqpoint{1.184330in}{2.624665in}}%
\pgfpathlineto{\pgfqpoint{1.148899in}{2.605605in}}%
\pgfpathlineto{\pgfqpoint{1.116339in}{2.585554in}}%
\pgfpathlineto{\pgfqpoint{1.092330in}{2.568498in}}%
\pgfpathlineto{\pgfqpoint{1.079759in}{2.558679in}}%
\pgfpathlineto{\pgfqpoint{1.051543in}{2.535372in}}%
\pgfpathlineto{\pgfqpoint{1.026311in}{2.511705in}}%
\pgfpathlineto{\pgfqpoint{1.002398in}{2.486310in}}%
\pgfpathlineto{\pgfqpoint{0.979913in}{2.459261in}}%
\pgfpathlineto{\pgfqpoint{0.958935in}{2.430669in}}%
\pgfpathlineto{\pgfqpoint{0.938265in}{2.398634in}}%
\pgfpathlineto{\pgfqpoint{0.923046in}{2.371377in}}%
\pgfpathlineto{\pgfqpoint{0.904513in}{2.334766in}}%
\pgfpathlineto{\pgfqpoint{0.887854in}{2.296992in}}%
\pgfpathlineto{\pgfqpoint{0.872132in}{2.255963in}}%
\pgfpathlineto{\pgfqpoint{0.857509in}{2.211732in}}%
\pgfpathlineto{\pgfqpoint{0.844763in}{2.166748in}}%
\pgfpathlineto{\pgfqpoint{0.838623in}{2.140298in}}%
\pgfpathlineto{\pgfqpoint{0.826981in}{2.087185in}}%
\pgfpathlineto{\pgfqpoint{0.816321in}{2.028707in}}%
\pgfpathlineto{\pgfqpoint{0.810088in}{1.984486in}}%
\pgfpathlineto{\pgfqpoint{0.808025in}{1.967229in}}%
\pgfpathlineto{\pgfqpoint{0.800076in}{1.898131in}}%
\pgfpathlineto{\pgfqpoint{0.793713in}{1.823814in}}%
\pgfpathlineto{\pgfqpoint{0.788799in}{1.741866in}}%
\pgfpathlineto{\pgfqpoint{0.786199in}{1.677216in}}%
\pgfpathlineto{\pgfqpoint{0.776950in}{1.453472in}}%
\pgfpathlineto{\pgfqpoint{0.773279in}{1.418885in}}%
\pgfpathlineto{\pgfqpoint{0.768298in}{1.389573in}}%
\pgfpathlineto{\pgfqpoint{0.762751in}{1.368099in}}%
\pgfpathlineto{\pgfqpoint{0.756720in}{1.352114in}}%
\pgfpathlineto{\pgfqpoint{0.749750in}{1.339511in}}%
\pgfpathlineto{\pgfqpoint{0.742198in}{1.330592in}}%
\pgfpathlineto{\pgfqpoint{0.734851in}{1.325306in}}%
\pgfpathlineto{\pgfqpoint{0.726554in}{1.322414in}}%
\pgfpathlineto{\pgfqpoint{0.717880in}{1.322220in}}%
\pgfpathlineto{\pgfqpoint{0.709409in}{1.324409in}}%
\pgfpathlineto{\pgfqpoint{0.699545in}{1.329603in}}%
\pgfpathlineto{\pgfqpoint{0.688891in}{1.338202in}}%
\pgfpathlineto{\pgfqpoint{0.677905in}{1.350248in}}%
\pgfpathlineto{\pgfqpoint{0.666884in}{1.365647in}}%
\pgfpathlineto{\pgfqpoint{0.654911in}{1.386417in}}%
\pgfpathlineto{\pgfqpoint{0.642572in}{1.412730in}}%
\pgfpathlineto{\pgfqpoint{0.630327in}{1.444629in}}%
\pgfpathlineto{\pgfqpoint{0.618503in}{1.482081in}}%
\pgfpathlineto{\pgfqpoint{0.608612in}{1.520256in}}%
\pgfpathlineto{\pgfqpoint{0.590202in}{1.612446in}}%
\pgfpathlineto{\pgfqpoint{0.581847in}{1.668885in}}%
\pgfpathlineto{\pgfqpoint{0.573137in}{1.740377in}}%
\pgfpathlineto{\pgfqpoint{0.567061in}{1.807214in}}%
\pgfpathlineto{\pgfqpoint{0.560531in}{1.896510in}}%
\pgfpathlineto{\pgfqpoint{0.555525in}{1.995911in}}%
\pgfpathlineto{\pgfqpoint{0.552564in}{2.097909in}}%
\pgfpathlineto{\pgfqpoint{0.551525in}{2.204936in}}%
\pgfpathlineto{\pgfqpoint{0.552727in}{2.309470in}}%
\pgfpathlineto{\pgfqpoint{0.556011in}{2.403982in}}%
\pgfpathlineto{\pgfqpoint{0.560952in}{2.483431in}}%
\pgfpathlineto{\pgfqpoint{0.567303in}{2.550241in}}%
\pgfpathlineto{\pgfqpoint{0.574927in}{2.606818in}}%
\pgfpathlineto{\pgfqpoint{0.582987in}{2.650657in}}%
\pgfpathlineto{\pgfqpoint{0.592755in}{2.691453in}}%
\pgfpathlineto{\pgfqpoint{0.602650in}{2.721757in}}%
\pgfpathlineto{\pgfqpoint{0.612983in}{2.746442in}}%
\pgfpathlineto{\pgfqpoint{0.624292in}{2.767693in}}%
\pgfpathlineto{\pgfqpoint{0.636231in}{2.785433in}}%
\pgfpathlineto{\pgfqpoint{0.649892in}{2.801462in}}%
\pgfpathlineto{\pgfqpoint{0.663386in}{2.814021in}}%
\pgfpathlineto{\pgfqpoint{0.679842in}{2.826135in}}%
\pgfpathlineto{\pgfqpoint{0.697326in}{2.836198in}}%
\pgfpathlineto{\pgfqpoint{0.715574in}{2.844285in}}%
\pgfpathlineto{\pgfqpoint{0.738439in}{2.852335in}}%
\pgfpathlineto{\pgfqpoint{0.765984in}{2.859640in}}%
\pgfpathlineto{\pgfqpoint{0.800301in}{2.866257in}}%
\pgfpathlineto{\pgfqpoint{0.841340in}{2.871832in}}%
\pgfpathlineto{\pgfqpoint{0.895547in}{2.876803in}}%
\pgfpathlineto{\pgfqpoint{0.969413in}{2.881069in}}%
\pgfpathlineto{\pgfqpoint{1.071608in}{2.884501in}}%
\pgfpathlineto{\pgfqpoint{1.219512in}{2.887074in}}%
\pgfpathlineto{\pgfqpoint{1.471844in}{2.889091in}}%
\pgfpathlineto{\pgfqpoint{1.956941in}{2.890384in}}%
\pgfpathlineto{\pgfqpoint{3.096814in}{2.890781in}}%
\pgfpathlineto{\pgfqpoint{3.995224in}{2.889388in}}%
\pgfpathlineto{\pgfqpoint{4.275833in}{2.887011in}}%
\pgfpathlineto{\pgfqpoint{4.412848in}{2.883743in}}%
\pgfpathlineto{\pgfqpoint{4.491081in}{2.879810in}}%
\pgfpathlineto{\pgfqpoint{4.543127in}{2.875163in}}%
\pgfpathlineto{\pgfqpoint{4.579810in}{2.869841in}}%
\pgfpathlineto{\pgfqpoint{4.607580in}{2.863763in}}%
\pgfpathlineto{\pgfqpoint{4.630623in}{2.856424in}}%
\pgfpathlineto{\pgfqpoint{4.648834in}{2.848228in}}%
\pgfpathlineto{\pgfqpoint{4.664136in}{2.838773in}}%
\pgfpathlineto{\pgfqpoint{4.676470in}{2.828576in}}%
\pgfpathlineto{\pgfqpoint{4.687502in}{2.816585in}}%
\pgfpathlineto{\pgfqpoint{4.697051in}{2.803027in}}%
\pgfpathlineto{\pgfqpoint{4.706194in}{2.786098in}}%
\pgfpathlineto{\pgfqpoint{4.714508in}{2.765827in}}%
\pgfpathlineto{\pgfqpoint{4.722462in}{2.740013in}}%
\pgfpathlineto{\pgfqpoint{4.729577in}{2.708703in}}%
\pgfpathlineto{\pgfqpoint{4.736162in}{2.669601in}}%
\pgfpathlineto{\pgfqpoint{4.742419in}{2.617826in}}%
\pgfpathlineto{\pgfqpoint{4.747859in}{2.553410in}}%
\pgfpathlineto{\pgfqpoint{4.752661in}{2.468958in}}%
\pgfpathlineto{\pgfqpoint{4.756610in}{2.359528in}}%
\pgfpathlineto{\pgfqpoint{4.759416in}{2.217681in}}%
\pgfpathlineto{\pgfqpoint{4.760596in}{2.043444in}}%
\pgfpathlineto{\pgfqpoint{4.759662in}{1.851779in}}%
\pgfpathlineto{\pgfqpoint{4.756587in}{1.667613in}}%
\pgfpathlineto{\pgfqpoint{4.751596in}{1.503428in}}%
\pgfpathlineto{\pgfqpoint{4.745410in}{1.374185in}}%
\pgfpathlineto{\pgfqpoint{4.738113in}{1.267479in}}%
\pgfpathlineto{\pgfqpoint{4.729621in}{1.175896in}}%
\pgfpathlineto{\pgfqpoint{4.720762in}{1.104428in}}%
\pgfpathlineto{\pgfqpoint{4.711045in}{1.043204in}}%
\pgfpathlineto{\pgfqpoint{4.700364in}{0.989829in}}%
\pgfpathlineto{\pgfqpoint{4.689055in}{0.944345in}}%
\pgfpathlineto{\pgfqpoint{4.676881in}{0.904394in}}%
\pgfpathlineto{\pgfqpoint{4.676095in}{0.902073in}}%
\pgfpathlineto{\pgfqpoint{4.676095in}{0.902073in}}%
\pgfusepath{stroke}%
\end{pgfscope}%
\begin{pgfscope}%
\pgfpathrectangle{\pgfqpoint{0.448634in}{0.402556in}}{\pgfqpoint{4.350661in}{2.489204in}} %
\pgfusepath{clip}%
\pgfsetrectcap%
\pgfsetroundjoin%
\pgfsetlinewidth{1.003750pt}%
\definecolor{currentstroke}{rgb}{0.549020,0.337255,0.294118}%
\pgfsetstrokecolor{currentstroke}%
\pgfsetdash{}{0pt}%
\pgfpathmoveto{\pgfqpoint{2.795520in}{1.982745in}}%
\pgfpathlineto{\pgfqpoint{2.781780in}{1.874357in}}%
\pgfpathlineto{\pgfqpoint{2.769351in}{1.758234in}}%
\pgfpathlineto{\pgfqpoint{2.758095in}{1.631942in}}%
\pgfpathlineto{\pgfqpoint{2.747786in}{1.490551in}}%
\pgfpathlineto{\pgfqpoint{2.738644in}{1.334082in}}%
\pgfpathlineto{\pgfqpoint{2.730580in}{1.157591in}}%
\pgfpathlineto{\pgfqpoint{2.723334in}{0.948663in}}%
\pgfpathlineto{\pgfqpoint{2.709783in}{0.530788in}}%
\pgfpathlineto{\pgfqpoint{2.705868in}{0.488716in}}%
\pgfpathlineto{\pgfqpoint{2.701769in}{0.464281in}}%
\pgfpathlineto{\pgfqpoint{2.697021in}{0.447744in}}%
\pgfpathlineto{\pgfqpoint{2.691859in}{0.436812in}}%
\pgfpathlineto{\pgfqpoint{2.686245in}{0.429229in}}%
\pgfpathlineto{\pgfqpoint{2.679348in}{0.423188in}}%
\pgfpathlineto{\pgfqpoint{2.669540in}{0.417856in}}%
\pgfpathlineto{\pgfqpoint{2.656987in}{0.413810in}}%
\pgfpathlineto{\pgfqpoint{2.637654in}{0.410337in}}%
\pgfpathlineto{\pgfqpoint{2.607297in}{0.407617in}}%
\pgfpathlineto{\pgfqpoint{2.555121in}{0.405574in}}%
\pgfpathlineto{\pgfqpoint{2.450714in}{0.404139in}}%
\pgfpathlineto{\pgfqpoint{2.176624in}{0.403275in}}%
\pgfpathlineto{\pgfqpoint{1.130290in}{0.402953in}}%
\pgfpathlineto{\pgfqpoint{0.516849in}{0.404175in}}%
\pgfpathlineto{\pgfqpoint{0.466848in}{0.405970in}}%
\pgfpathlineto{\pgfqpoint{0.456129in}{0.407931in}}%
\pgfpathlineto{\pgfqpoint{0.452340in}{0.410303in}}%
\pgfpathlineto{\pgfqpoint{0.450346in}{0.414662in}}%
\pgfpathlineto{\pgfqpoint{0.449266in}{0.424524in}}%
\pgfpathlineto{\pgfqpoint{0.448771in}{0.464345in}}%
\pgfpathlineto{\pgfqpoint{0.448640in}{0.850171in}}%
\pgfpathlineto{\pgfqpoint{0.448653in}{2.891318in}}%
\pgfpathlineto{\pgfqpoint{0.448653in}{2.891318in}}%
\pgfusepath{stroke}%
\end{pgfscope}%
\begin{pgfscope}%
\pgfpathrectangle{\pgfqpoint{0.448634in}{0.402556in}}{\pgfqpoint{4.350661in}{2.489204in}} %
\pgfusepath{clip}%
\pgfsetrectcap%
\pgfsetroundjoin%
\pgfsetlinewidth{1.003750pt}%
\definecolor{currentstroke}{rgb}{0.549020,0.337255,0.294118}%
\pgfsetstrokecolor{currentstroke}%
\pgfsetdash{}{0pt}%
\pgfpathmoveto{\pgfqpoint{3.428229in}{0.402586in}}%
\pgfpathlineto{\pgfqpoint{2.782161in}{0.403711in}}%
\pgfpathlineto{\pgfqpoint{2.753947in}{0.405697in}}%
\pgfpathlineto{\pgfqpoint{2.743371in}{0.408479in}}%
\pgfpathlineto{\pgfqpoint{2.737761in}{0.412225in}}%
\pgfpathlineto{\pgfqpoint{2.733707in}{0.418028in}}%
\pgfpathlineto{\pgfqpoint{2.730679in}{0.427337in}}%
\pgfpathlineto{\pgfqpoint{2.728410in}{0.442032in}}%
\pgfpathlineto{\pgfqpoint{2.726559in}{0.471822in}}%
\pgfpathlineto{\pgfqpoint{2.725226in}{0.534030in}}%
\pgfpathlineto{\pgfqpoint{2.725176in}{0.656000in}}%
\pgfpathlineto{\pgfqpoint{2.727382in}{0.832714in}}%
\pgfpathlineto{\pgfqpoint{2.732264in}{1.041730in}}%
\pgfpathlineto{\pgfqpoint{2.738856in}{1.223284in}}%
\pgfpathlineto{\pgfqpoint{2.747083in}{1.389793in}}%
\pgfpathlineto{\pgfqpoint{2.756614in}{1.538744in}}%
\pgfpathlineto{\pgfqpoint{2.768961in}{1.694914in}}%
\pgfpathlineto{\pgfqpoint{2.781235in}{1.816071in}}%
\pgfpathlineto{\pgfqpoint{2.794410in}{1.924551in}}%
\pgfpathlineto{\pgfqpoint{2.812747in}{2.054749in}}%
\pgfpathlineto{\pgfqpoint{2.828786in}{2.147538in}}%
\pgfpathlineto{\pgfqpoint{2.847396in}{2.242250in}}%
\pgfpathlineto{\pgfqpoint{2.895833in}{2.479724in}}%
\pgfpathlineto{\pgfqpoint{2.900218in}{2.516714in}}%
\pgfpathlineto{\pgfqpoint{2.901360in}{2.544054in}}%
\pgfpathlineto{\pgfqpoint{2.900304in}{2.566413in}}%
\pgfpathlineto{\pgfqpoint{2.897346in}{2.586024in}}%
\pgfpathlineto{\pgfqpoint{2.892846in}{2.602657in}}%
\pgfpathlineto{\pgfqpoint{2.886402in}{2.618429in}}%
\pgfpathlineto{\pgfqpoint{2.878064in}{2.632991in}}%
\pgfpathlineto{\pgfqpoint{2.868069in}{2.646120in}}%
\pgfpathlineto{\pgfqpoint{2.855054in}{2.659319in}}%
\pgfpathlineto{\pgfqpoint{2.840803in}{2.670733in}}%
\pgfpathlineto{\pgfqpoint{2.821824in}{2.682876in}}%
\pgfpathlineto{\pgfqpoint{2.799981in}{2.694039in}}%
\pgfpathlineto{\pgfqpoint{2.773366in}{2.704956in}}%
\pgfpathlineto{\pgfqpoint{2.742012in}{2.715277in}}%
\pgfpathlineto{\pgfqpoint{2.705983in}{2.724795in}}%
\pgfpathlineto{\pgfqpoint{2.663200in}{2.733819in}}%
\pgfpathlineto{\pgfqpoint{2.611535in}{2.742387in}}%
\pgfpathlineto{\pgfqpoint{2.551002in}{2.750097in}}%
\pgfpathlineto{\pgfqpoint{2.481632in}{2.756689in}}%
\pgfpathlineto{\pgfqpoint{2.399112in}{2.762206in}}%
\pgfpathlineto{\pgfqpoint{2.309984in}{2.765891in}}%
\pgfpathlineto{\pgfqpoint{2.188184in}{2.768102in}}%
\pgfpathlineto{\pgfqpoint{2.081595in}{2.767624in}}%
\pgfpathlineto{\pgfqpoint{1.968505in}{2.764845in}}%
\pgfpathlineto{\pgfqpoint{1.864180in}{2.759924in}}%
\pgfpathlineto{\pgfqpoint{1.757786in}{2.752599in}}%
\pgfpathlineto{\pgfqpoint{1.671086in}{2.744177in}}%
\pgfpathlineto{\pgfqpoint{1.591075in}{2.734200in}}%
\pgfpathlineto{\pgfqpoint{1.502689in}{2.720726in}}%
\pgfpathlineto{\pgfqpoint{1.427655in}{2.706083in}}%
\pgfpathlineto{\pgfqpoint{1.372350in}{2.692544in}}%
\pgfpathlineto{\pgfqpoint{1.321734in}{2.677921in}}%
\pgfpathlineto{\pgfqpoint{1.273765in}{2.661664in}}%
\pgfpathlineto{\pgfqpoint{1.230567in}{2.644673in}}%
\pgfpathlineto{\pgfqpoint{1.192196in}{2.627107in}}%
\pgfpathlineto{\pgfqpoint{1.156620in}{2.608404in}}%
\pgfpathlineto{\pgfqpoint{1.123890in}{2.588718in}}%
\pgfpathlineto{\pgfqpoint{1.095882in}{2.569570in}}%
\pgfpathlineto{\pgfqpoint{1.026625in}{2.509436in}}%
\pgfpathlineto{\pgfqpoint{1.002809in}{2.483922in}}%
\pgfpathlineto{\pgfqpoint{0.980428in}{2.456759in}}%
\pgfpathlineto{\pgfqpoint{0.959560in}{2.428062in}}%
\pgfpathlineto{\pgfqpoint{0.939013in}{2.395924in}}%
\pgfpathlineto{\pgfqpoint{0.923905in}{2.368586in}}%
\pgfpathlineto{\pgfqpoint{0.905505in}{2.331886in}}%
\pgfpathlineto{\pgfqpoint{0.888981in}{2.294035in}}%
\pgfpathlineto{\pgfqpoint{0.873402in}{2.252935in}}%
\pgfpathlineto{\pgfqpoint{0.858929in}{2.208639in}}%
\pgfpathlineto{\pgfqpoint{0.846473in}{2.163558in}}%
\pgfpathlineto{\pgfqpoint{0.841904in}{2.144348in}}%
\pgfpathlineto{\pgfqpoint{0.830146in}{2.091269in}}%
\pgfpathlineto{\pgfqpoint{0.819426in}{2.032806in}}%
\pgfpathlineto{\pgfqpoint{0.812048in}{1.981229in}}%
\pgfpathlineto{\pgfqpoint{0.800927in}{1.882487in}}%
\pgfpathlineto{\pgfqpoint{0.795032in}{1.805620in}}%
\pgfpathlineto{\pgfqpoint{0.790757in}{1.721132in}}%
\pgfpathlineto{\pgfqpoint{0.787501in}{1.611673in}}%
\pgfpathlineto{\pgfqpoint{0.785354in}{1.522095in}}%
\pgfpathlineto{\pgfqpoint{0.785354in}{1.522095in}}%
\pgfusepath{stroke}%
\end{pgfscope}%
\begin{pgfscope}%
\pgfpathrectangle{\pgfqpoint{0.448634in}{0.402556in}}{\pgfqpoint{4.350661in}{2.489204in}} %
\pgfusepath{clip}%
\pgfsetrectcap%
\pgfsetroundjoin%
\pgfsetlinewidth{1.003750pt}%
\definecolor{currentstroke}{rgb}{0.549020,0.337255,0.294118}%
\pgfsetstrokecolor{currentstroke}%
\pgfsetdash{}{0pt}%
\pgfpathmoveto{\pgfqpoint{2.028735in}{0.425754in}}%
\pgfpathlineto{\pgfqpoint{1.878677in}{0.421879in}}%
\pgfpathlineto{\pgfqpoint{1.676387in}{0.418997in}}%
\pgfpathlineto{\pgfqpoint{1.413176in}{0.417558in}}%
\pgfpathlineto{\pgfqpoint{1.134735in}{0.418204in}}%
\pgfpathlineto{\pgfqpoint{0.921565in}{0.420769in}}%
\pgfpathlineto{\pgfqpoint{0.782384in}{0.424523in}}%
\pgfpathlineto{\pgfqpoint{0.693283in}{0.428974in}}%
\pgfpathlineto{\pgfqpoint{0.632541in}{0.434091in}}%
\pgfpathlineto{\pgfqpoint{0.591492in}{0.439564in}}%
\pgfpathlineto{\pgfqpoint{0.561503in}{0.445595in}}%
\pgfpathlineto{\pgfqpoint{0.538349in}{0.452466in}}%
\pgfpathlineto{\pgfqpoint{0.522042in}{0.459394in}}%
\pgfpathlineto{\pgfqpoint{0.508540in}{0.467420in}}%
\pgfpathlineto{\pgfqpoint{0.497973in}{0.476161in}}%
\pgfpathlineto{\pgfqpoint{0.488790in}{0.486749in}}%
\pgfpathlineto{\pgfqpoint{0.481284in}{0.498948in}}%
\pgfpathlineto{\pgfqpoint{0.474590in}{0.514580in}}%
\pgfpathlineto{\pgfqpoint{0.469106in}{0.533467in}}%
\pgfpathlineto{\pgfqpoint{0.464439in}{0.557771in}}%
\pgfpathlineto{\pgfqpoint{0.460297in}{0.592289in}}%
\pgfpathlineto{\pgfqpoint{0.456855in}{0.641912in}}%
\pgfpathlineto{\pgfqpoint{0.454122in}{0.716520in}}%
\pgfpathlineto{\pgfqpoint{0.451978in}{0.843444in}}%
\pgfpathlineto{\pgfqpoint{0.450459in}{1.087379in}}%
\pgfpathlineto{\pgfqpoint{0.449596in}{1.657406in}}%
\pgfpathlineto{\pgfqpoint{0.450150in}{2.687936in}}%
\pgfpathlineto{\pgfqpoint{0.451781in}{2.839761in}}%
\pgfpathlineto{\pgfqpoint{0.453975in}{2.872003in}}%
\pgfpathlineto{\pgfqpoint{0.456339in}{2.881553in}}%
\pgfpathlineto{\pgfqpoint{0.458888in}{2.885549in}}%
\pgfpathlineto{\pgfqpoint{0.462554in}{2.888171in}}%
\pgfpathlineto{\pgfqpoint{0.471046in}{2.890205in}}%
\pgfpathlineto{\pgfqpoint{0.490597in}{2.891263in}}%
\pgfpathlineto{\pgfqpoint{0.564556in}{2.891692in}}%
\pgfpathlineto{\pgfqpoint{1.569559in}{2.891759in}}%
\pgfpathlineto{\pgfqpoint{4.784679in}{2.890785in}}%
\pgfpathlineto{\pgfqpoint{4.791004in}{2.889098in}}%
\pgfpathlineto{\pgfqpoint{4.793910in}{2.885555in}}%
\pgfpathlineto{\pgfqpoint{4.795579in}{2.878366in}}%
\pgfpathlineto{\pgfqpoint{4.796850in}{2.858514in}}%
\pgfpathlineto{\pgfqpoint{4.796850in}{2.858514in}}%
\pgfusepath{stroke}%
\end{pgfscope}%
\begin{pgfscope}%
\pgfpathrectangle{\pgfqpoint{0.448634in}{0.402556in}}{\pgfqpoint{4.350661in}{2.489204in}} %
\pgfusepath{clip}%
\pgfsetrectcap%
\pgfsetroundjoin%
\pgfsetlinewidth{1.003750pt}%
\definecolor{currentstroke}{rgb}{0.890196,0.466667,0.760784}%
\pgfsetstrokecolor{currentstroke}%
\pgfsetdash{}{0pt}%
\pgfpathmoveto{\pgfqpoint{0.448634in}{2.896245in}}%
\pgfpathlineto{\pgfqpoint{0.448593in}{0.407043in}}%
\pgfpathlineto{\pgfqpoint{0.448593in}{0.407043in}}%
\pgfusepath{stroke}%
\end{pgfscope}%
\begin{pgfscope}%
\pgfpathrectangle{\pgfqpoint{0.448634in}{0.402556in}}{\pgfqpoint{4.350661in}{2.489204in}} %
\pgfusepath{clip}%
\pgfsetrectcap%
\pgfsetroundjoin%
\pgfsetlinewidth{1.003750pt}%
\definecolor{currentstroke}{rgb}{0.890196,0.466667,0.760784}%
\pgfsetstrokecolor{currentstroke}%
\pgfsetdash{}{0pt}%
\pgfpathmoveto{\pgfqpoint{0.581265in}{1.721678in}}%
\pgfpathlineto{\pgfqpoint{0.573199in}{1.795779in}}%
\pgfpathlineto{\pgfqpoint{0.566265in}{1.880036in}}%
\pgfpathlineto{\pgfqpoint{0.560758in}{1.974413in}}%
\pgfpathlineto{\pgfqpoint{0.557054in}{2.076380in}}%
\pgfpathlineto{\pgfqpoint{0.555430in}{2.183397in}}%
\pgfpathlineto{\pgfqpoint{0.556065in}{2.285449in}}%
\pgfpathlineto{\pgfqpoint{0.558835in}{2.379982in}}%
\pgfpathlineto{\pgfqpoint{0.563404in}{2.461956in}}%
\pgfpathlineto{\pgfqpoint{0.569455in}{2.531304in}}%
\pgfpathlineto{\pgfqpoint{0.576540in}{2.587974in}}%
\pgfpathlineto{\pgfqpoint{0.584450in}{2.634388in}}%
\pgfpathlineto{\pgfqpoint{0.593140in}{2.672947in}}%
\pgfpathlineto{\pgfqpoint{0.602842in}{2.705971in}}%
\pgfpathlineto{\pgfqpoint{0.613219in}{2.733369in}}%
\pgfpathlineto{\pgfqpoint{0.623685in}{2.755179in}}%
\pgfpathlineto{\pgfqpoint{0.634752in}{2.773648in}}%
\pgfpathlineto{\pgfqpoint{0.647498in}{2.790636in}}%
\pgfpathlineto{\pgfqpoint{0.661897in}{2.805796in}}%
\pgfpathlineto{\pgfqpoint{0.677759in}{2.818907in}}%
\pgfpathlineto{\pgfqpoint{0.694783in}{2.829950in}}%
\pgfpathlineto{\pgfqpoint{0.714689in}{2.839960in}}%
\pgfpathlineto{\pgfqpoint{0.737349in}{2.848734in}}%
\pgfpathlineto{\pgfqpoint{0.764749in}{2.856713in}}%
\pgfpathlineto{\pgfqpoint{0.796820in}{2.863566in}}%
\pgfpathlineto{\pgfqpoint{0.835627in}{2.869508in}}%
\pgfpathlineto{\pgfqpoint{0.885445in}{2.874764in}}%
\pgfpathlineto{\pgfqpoint{0.950585in}{2.879248in}}%
\pgfpathlineto{\pgfqpoint{1.039713in}{2.882988in}}%
\pgfpathlineto{\pgfqpoint{1.168029in}{2.885982in}}%
\pgfpathlineto{\pgfqpoint{1.370325in}{2.888239in}}%
\pgfpathlineto{\pgfqpoint{1.733602in}{2.889852in}}%
\pgfpathlineto{\pgfqpoint{2.523247in}{2.890712in}}%
\pgfpathlineto{\pgfqpoint{3.769711in}{2.889981in}}%
\pgfpathlineto{\pgfqpoint{4.176493in}{2.887912in}}%
\pgfpathlineto{\pgfqpoint{4.354850in}{2.884989in}}%
\pgfpathlineto{\pgfqpoint{4.452685in}{2.881304in}}%
\pgfpathlineto{\pgfqpoint{4.515646in}{2.876851in}}%
\pgfpathlineto{\pgfqpoint{4.554602in}{2.872363in}}%
\pgfpathlineto{\pgfqpoint{4.588992in}{2.866278in}}%
\pgfpathlineto{\pgfqpoint{4.614424in}{2.859579in}}%
\pgfpathlineto{\pgfqpoint{4.635088in}{2.851836in}}%
\pgfpathlineto{\pgfqpoint{4.650960in}{2.843692in}}%
\pgfpathlineto{\pgfqpoint{4.665848in}{2.833410in}}%
\pgfpathlineto{\pgfqpoint{4.677713in}{2.822508in}}%
\pgfpathlineto{\pgfqpoint{4.688223in}{2.809917in}}%
\pgfpathlineto{\pgfqpoint{4.698420in}{2.793796in}}%
\pgfpathlineto{\pgfqpoint{4.706877in}{2.776403in}}%
\pgfpathlineto{\pgfqpoint{4.715364in}{2.753495in}}%
\pgfpathlineto{\pgfqpoint{4.723239in}{2.725026in}}%
\pgfpathlineto{\pgfqpoint{4.730107in}{2.691080in}}%
\pgfpathlineto{\pgfqpoint{4.736746in}{2.646929in}}%
\pgfpathlineto{\pgfqpoint{4.742604in}{2.592582in}}%
\pgfpathlineto{\pgfqpoint{4.747970in}{2.520661in}}%
\pgfpathlineto{\pgfqpoint{4.752552in}{2.428712in}}%
\pgfpathlineto{\pgfqpoint{4.756182in}{2.309304in}}%
\pgfpathlineto{\pgfqpoint{4.758511in}{2.157488in}}%
\pgfpathlineto{\pgfqpoint{4.758999in}{1.978267in}}%
\pgfpathlineto{\pgfqpoint{4.757315in}{1.791589in}}%
\pgfpathlineto{\pgfqpoint{4.753521in}{1.614910in}}%
\pgfpathlineto{\pgfqpoint{4.747937in}{1.460716in}}%
\pgfpathlineto{\pgfqpoint{4.741166in}{1.338995in}}%
\pgfpathlineto{\pgfqpoint{4.733254in}{1.237343in}}%
\pgfpathlineto{\pgfqpoint{4.724197in}{1.150846in}}%
\pgfpathlineto{\pgfqpoint{4.714951in}{1.084480in}}%
\pgfpathlineto{\pgfqpoint{4.704590in}{1.025933in}}%
\pgfpathlineto{\pgfqpoint{4.693365in}{0.975268in}}%
\pgfpathlineto{\pgfqpoint{4.681655in}{0.932520in}}%
\pgfpathlineto{\pgfqpoint{4.669262in}{0.895310in}}%
\pgfpathlineto{\pgfqpoint{4.655633in}{0.861393in}}%
\pgfpathlineto{\pgfqpoint{4.640912in}{0.830896in}}%
\pgfpathlineto{\pgfqpoint{4.625396in}{0.803852in}}%
\pgfpathlineto{\pgfqpoint{4.609363in}{0.780288in}}%
\pgfpathlineto{\pgfqpoint{4.591725in}{0.758280in}}%
\pgfpathlineto{\pgfqpoint{4.572578in}{0.737989in}}%
\pgfpathlineto{\pgfqpoint{4.552076in}{0.719512in}}%
\pgfpathlineto{\pgfqpoint{4.530410in}{0.702862in}}%
\pgfpathlineto{\pgfqpoint{4.505851in}{0.686832in}}%
\pgfpathlineto{\pgfqpoint{4.478416in}{0.671722in}}%
\pgfpathlineto{\pgfqpoint{4.439963in}{0.654407in}}%
\pgfpathlineto{\pgfqpoint{4.406842in}{0.642183in}}%
\pgfpathlineto{\pgfqpoint{4.369008in}{0.630654in}}%
\pgfpathlineto{\pgfqpoint{4.326488in}{0.620136in}}%
\pgfpathlineto{\pgfqpoint{4.279325in}{0.610862in}}%
\pgfpathlineto{\pgfqpoint{4.227574in}{0.603000in}}%
\pgfpathlineto{\pgfqpoint{4.173448in}{0.596978in}}%
\pgfpathlineto{\pgfqpoint{4.110509in}{0.592119in}}%
\pgfpathlineto{\pgfqpoint{4.047469in}{0.589450in}}%
\pgfpathlineto{\pgfqpoint{3.977865in}{0.588536in}}%
\pgfpathlineto{\pgfqpoint{3.906091in}{0.589843in}}%
\pgfpathlineto{\pgfqpoint{3.834375in}{0.593403in}}%
\pgfpathlineto{\pgfqpoint{3.767117in}{0.598970in}}%
\pgfpathlineto{\pgfqpoint{3.704362in}{0.606293in}}%
\pgfpathlineto{\pgfqpoint{3.682839in}{0.609829in}}%
\pgfpathlineto{\pgfqpoint{3.624737in}{0.619642in}}%
\pgfpathlineto{\pgfqpoint{3.588418in}{0.627483in}}%
\pgfpathlineto{\pgfqpoint{3.575673in}{0.630667in}}%
\pgfpathlineto{\pgfqpoint{3.522592in}{0.644152in}}%
\pgfpathlineto{\pgfqpoint{3.476421in}{0.658549in}}%
\pgfpathlineto{\pgfqpoint{3.437148in}{0.673272in}}%
\pgfpathlineto{\pgfqpoint{3.394483in}{0.691888in}}%
\pgfpathlineto{\pgfqpoint{3.360887in}{0.709559in}}%
\pgfpathlineto{\pgfqpoint{3.330177in}{0.728291in}}%
\pgfpathlineto{\pgfqpoint{3.302452in}{0.747967in}}%
\pgfpathlineto{\pgfqpoint{3.277652in}{0.768184in}}%
\pgfpathlineto{\pgfqpoint{3.254080in}{0.790239in}}%
\pgfpathlineto{\pgfqpoint{3.231904in}{0.814113in}}%
\pgfpathlineto{\pgfqpoint{3.211272in}{0.839736in}}%
\pgfpathlineto{\pgfqpoint{3.192276in}{0.866965in}}%
\pgfpathlineto{\pgfqpoint{3.175016in}{0.895668in}}%
\pgfpathlineto{\pgfqpoint{3.157408in}{0.930012in}}%
\pgfpathlineto{\pgfqpoint{3.142851in}{0.963422in}}%
\pgfpathlineto{\pgfqpoint{3.129375in}{1.000135in}}%
\pgfpathlineto{\pgfqpoint{3.117913in}{1.037734in}}%
\pgfpathlineto{\pgfqpoint{3.107163in}{1.080812in}}%
\pgfpathlineto{\pgfqpoint{3.098545in}{1.124513in}}%
\pgfpathlineto{\pgfqpoint{3.091608in}{1.171130in}}%
\pgfpathlineto{\pgfqpoint{3.086343in}{1.220544in}}%
\pgfpathlineto{\pgfqpoint{3.082974in}{1.272670in}}%
\pgfpathlineto{\pgfqpoint{3.081645in}{1.327406in}}%
\pgfpathlineto{\pgfqpoint{3.082529in}{1.384644in}}%
\pgfpathlineto{\pgfqpoint{3.085619in}{1.441782in}}%
\pgfpathlineto{\pgfqpoint{3.091167in}{1.501179in}}%
\pgfpathlineto{\pgfqpoint{3.099087in}{1.560222in}}%
\pgfpathlineto{\pgfqpoint{3.108986in}{1.616337in}}%
\pgfpathlineto{\pgfqpoint{3.120552in}{1.669472in}}%
\pgfpathlineto{\pgfqpoint{3.134326in}{1.721911in}}%
\pgfpathlineto{\pgfqpoint{3.149590in}{1.771175in}}%
\pgfpathlineto{\pgfqpoint{3.166110in}{1.817225in}}%
\pgfpathlineto{\pgfqpoint{3.184656in}{1.862253in}}%
\pgfpathlineto{\pgfqpoint{3.204162in}{1.903944in}}%
\pgfpathlineto{\pgfqpoint{3.225489in}{1.944450in}}%
\pgfpathlineto{\pgfqpoint{3.248626in}{1.983634in}}%
\pgfpathlineto{\pgfqpoint{3.273489in}{2.021408in}}%
\pgfpathlineto{\pgfqpoint{3.301375in}{2.059615in}}%
\pgfpathlineto{\pgfqpoint{3.332268in}{2.098116in}}%
\pgfpathlineto{\pgfqpoint{3.375168in}{2.147590in}}%
\pgfpathlineto{\pgfqpoint{3.410081in}{2.188587in}}%
\pgfpathlineto{\pgfqpoint{3.422453in}{2.205936in}}%
\pgfpathlineto{\pgfqpoint{3.429196in}{2.218699in}}%
\pgfpathlineto{\pgfqpoint{3.431848in}{2.228147in}}%
\pgfpathlineto{\pgfqpoint{3.431550in}{2.235551in}}%
\pgfpathlineto{\pgfqpoint{3.429667in}{2.240011in}}%
\pgfpathlineto{\pgfqpoint{3.424820in}{2.244932in}}%
\pgfpathlineto{\pgfqpoint{3.416758in}{2.248582in}}%
\pgfpathlineto{\pgfqpoint{3.406018in}{2.250440in}}%
\pgfpathlineto{\pgfqpoint{3.390803in}{2.250343in}}%
\pgfpathlineto{\pgfqpoint{3.373534in}{2.247944in}}%
\pgfpathlineto{\pgfqpoint{3.352277in}{2.242695in}}%
\pgfpathlineto{\pgfqpoint{3.329375in}{2.234783in}}%
\pgfpathlineto{\pgfqpoint{3.305006in}{2.224092in}}%
\pgfpathlineto{\pgfqpoint{3.279376in}{2.210431in}}%
\pgfpathlineto{\pgfqpoint{3.252732in}{2.193567in}}%
\pgfpathlineto{\pgfqpoint{3.227148in}{2.174674in}}%
\pgfpathlineto{\pgfqpoint{3.202689in}{2.153921in}}%
\pgfpathlineto{\pgfqpoint{3.177794in}{2.129793in}}%
\pgfpathlineto{\pgfqpoint{3.154306in}{2.103884in}}%
\pgfpathlineto{\pgfqpoint{3.132252in}{2.076373in}}%
\pgfpathlineto{\pgfqpoint{3.110313in}{2.045462in}}%
\pgfpathlineto{\pgfqpoint{3.088772in}{2.011074in}}%
\pgfpathlineto{\pgfqpoint{3.068998in}{1.975322in}}%
\pgfpathlineto{\pgfqpoint{3.049897in}{1.936214in}}%
\pgfpathlineto{\pgfqpoint{3.031658in}{1.893780in}}%
\pgfpathlineto{\pgfqpoint{3.014449in}{1.848061in}}%
\pgfpathlineto{\pgfqpoint{2.998389in}{1.799129in}}%
\pgfpathlineto{\pgfqpoint{2.983576in}{1.747061in}}%
\pgfpathlineto{\pgfqpoint{2.969537in}{1.689525in}}%
\pgfpathlineto{\pgfqpoint{2.957039in}{1.628965in}}%
\pgfpathlineto{\pgfqpoint{2.945744in}{1.563015in}}%
\pgfpathlineto{\pgfqpoint{2.936236in}{1.494175in}}%
\pgfpathlineto{\pgfqpoint{2.928581in}{1.422525in}}%
\pgfpathlineto{\pgfqpoint{2.922885in}{1.348138in}}%
\pgfpathlineto{\pgfqpoint{2.919413in}{1.273571in}}%
\pgfpathlineto{\pgfqpoint{2.918210in}{1.201401in}}%
\pgfpathlineto{\pgfqpoint{2.919192in}{1.131716in}}%
\pgfpathlineto{\pgfqpoint{2.922359in}{1.064610in}}%
\pgfpathlineto{\pgfqpoint{2.927525in}{1.002666in}}%
\pgfpathlineto{\pgfqpoint{2.934171in}{0.948439in}}%
\pgfpathlineto{\pgfqpoint{2.942707in}{0.897092in}}%
\pgfpathlineto{\pgfqpoint{2.952649in}{0.851193in}}%
\pgfpathlineto{\pgfqpoint{2.963679in}{0.810809in}}%
\pgfpathlineto{\pgfqpoint{2.975377in}{0.775960in}}%
\pgfpathlineto{\pgfqpoint{2.988197in}{0.744357in}}%
\pgfpathlineto{\pgfqpoint{3.001924in}{0.716074in}}%
\pgfpathlineto{\pgfqpoint{3.017533in}{0.689102in}}%
\pgfpathlineto{\pgfqpoint{3.033664in}{0.665627in}}%
\pgfpathlineto{\pgfqpoint{3.051423in}{0.643747in}}%
\pgfpathlineto{\pgfqpoint{3.070711in}{0.623632in}}%
\pgfpathlineto{\pgfqpoint{3.091365in}{0.605377in}}%
\pgfpathlineto{\pgfqpoint{3.113183in}{0.588991in}}%
\pgfpathlineto{\pgfqpoint{3.137889in}{0.573260in}}%
\pgfpathlineto{\pgfqpoint{3.165464in}{0.558484in}}%
\pgfpathlineto{\pgfqpoint{3.195840in}{0.544862in}}%
\pgfpathlineto{\pgfqpoint{3.228917in}{0.532485in}}%
\pgfpathlineto{\pgfqpoint{3.266707in}{0.520770in}}%
\pgfpathlineto{\pgfqpoint{3.309175in}{0.509977in}}%
\pgfpathlineto{\pgfqpoint{3.358416in}{0.499850in}}%
\pgfpathlineto{\pgfqpoint{3.414401in}{0.490680in}}%
\pgfpathlineto{\pgfqpoint{3.479253in}{0.482363in}}%
\pgfpathlineto{\pgfqpoint{3.555111in}{0.474946in}}%
\pgfpathlineto{\pgfqpoint{3.644124in}{0.468561in}}%
\pgfpathlineto{\pgfqpoint{3.746267in}{0.463491in}}%
\pgfpathlineto{\pgfqpoint{3.863692in}{0.459910in}}%
\pgfpathlineto{\pgfqpoint{3.992026in}{0.458217in}}%
\pgfpathlineto{\pgfqpoint{4.120369in}{0.458704in}}%
\pgfpathlineto{\pgfqpoint{4.237813in}{0.461322in}}%
\pgfpathlineto{\pgfqpoint{4.335629in}{0.465634in}}%
\pgfpathlineto{\pgfqpoint{4.415958in}{0.471353in}}%
\pgfpathlineto{\pgfqpoint{4.478775in}{0.477957in}}%
\pgfpathlineto{\pgfqpoint{4.528397in}{0.485252in}}%
\pgfpathlineto{\pgfqpoint{4.569112in}{0.493359in}}%
\pgfpathlineto{\pgfqpoint{4.603015in}{0.502343in}}%
\pgfpathlineto{\pgfqpoint{4.630078in}{0.511708in}}%
\pgfpathlineto{\pgfqpoint{4.652383in}{0.521599in}}%
\pgfpathlineto{\pgfqpoint{4.671887in}{0.532602in}}%
\pgfpathlineto{\pgfqpoint{4.688479in}{0.544473in}}%
\pgfpathlineto{\pgfqpoint{4.702160in}{0.556766in}}%
\pgfpathlineto{\pgfqpoint{4.714565in}{0.570717in}}%
\pgfpathlineto{\pgfqpoint{4.725501in}{0.586194in}}%
\pgfpathlineto{\pgfqpoint{4.735978in}{0.605104in}}%
\pgfpathlineto{\pgfqpoint{4.745497in}{0.627473in}}%
\pgfpathlineto{\pgfqpoint{4.753787in}{0.653148in}}%
\pgfpathlineto{\pgfqpoint{4.761305in}{0.684335in}}%
\pgfpathlineto{\pgfqpoint{4.767778in}{0.720925in}}%
\pgfpathlineto{\pgfqpoint{4.773698in}{0.767727in}}%
\pgfpathlineto{\pgfqpoint{4.778881in}{0.827169in}}%
\pgfpathlineto{\pgfqpoint{4.783445in}{0.906649in}}%
\pgfpathlineto{\pgfqpoint{4.787336in}{1.016082in}}%
\pgfpathlineto{\pgfqpoint{4.790532in}{1.172858in}}%
\pgfpathlineto{\pgfqpoint{4.792964in}{1.404336in}}%
\pgfpathlineto{\pgfqpoint{4.794628in}{1.790158in}}%
\pgfpathlineto{\pgfqpoint{4.794757in}{2.345250in}}%
\pgfpathlineto{\pgfqpoint{4.792993in}{2.671327in}}%
\pgfpathlineto{\pgfqpoint{4.790462in}{2.788280in}}%
\pgfpathlineto{\pgfqpoint{4.787557in}{2.835450in}}%
\pgfpathlineto{\pgfqpoint{4.784407in}{2.857547in}}%
\pgfpathlineto{\pgfqpoint{4.781030in}{2.869359in}}%
\pgfpathlineto{\pgfqpoint{4.776344in}{2.877698in}}%
\pgfpathlineto{\pgfqpoint{4.771269in}{2.882350in}}%
\pgfpathlineto{\pgfqpoint{4.763220in}{2.886057in}}%
\pgfpathlineto{\pgfqpoint{4.750376in}{2.888628in}}%
\pgfpathlineto{\pgfqpoint{4.728672in}{2.890224in}}%
\pgfpathlineto{\pgfqpoint{4.676473in}{2.891225in}}%
\pgfpathlineto{\pgfqpoint{4.485045in}{2.891665in}}%
\pgfpathlineto{\pgfqpoint{2.250981in}{2.891751in}}%
\pgfpathlineto{\pgfqpoint{0.608609in}{2.890634in}}%
\pgfpathlineto{\pgfqpoint{0.545554in}{2.888587in}}%
\pgfpathlineto{\pgfqpoint{0.519569in}{2.885838in}}%
\pgfpathlineto{\pgfqpoint{0.504655in}{2.882384in}}%
\pgfpathlineto{\pgfqpoint{0.494491in}{2.878003in}}%
\pgfpathlineto{\pgfqpoint{0.487172in}{2.872655in}}%
\pgfpathlineto{\pgfqpoint{0.481146in}{2.865505in}}%
\pgfpathlineto{\pgfqpoint{0.475661in}{2.854789in}}%
\pgfpathlineto{\pgfqpoint{0.471317in}{2.840721in}}%
\pgfpathlineto{\pgfqpoint{0.467301in}{2.818806in}}%
\pgfpathlineto{\pgfqpoint{0.463928in}{2.786684in}}%
\pgfpathlineto{\pgfqpoint{0.460919in}{2.734528in}}%
\pgfpathlineto{\pgfqpoint{0.458363in}{2.647457in}}%
\pgfpathlineto{\pgfqpoint{0.456575in}{2.523014in}}%
\pgfpathlineto{\pgfqpoint{0.456575in}{2.523014in}}%
\pgfusepath{stroke}%
\end{pgfscope}%
\begin{pgfscope}%
\pgfpathrectangle{\pgfqpoint{0.448634in}{0.402556in}}{\pgfqpoint{4.350661in}{2.489204in}} %
\pgfusepath{clip}%
\pgfsetrectcap%
\pgfsetroundjoin%
\pgfsetlinewidth{1.003750pt}%
\definecolor{currentstroke}{rgb}{0.890196,0.466667,0.760784}%
\pgfsetstrokecolor{currentstroke}%
\pgfsetdash{}{0pt}%
\pgfpathmoveto{\pgfqpoint{4.798840in}{2.852369in}}%
\pgfpathlineto{\pgfqpoint{4.797564in}{2.889610in}}%
\pgfpathlineto{\pgfqpoint{4.796215in}{2.891483in}}%
\pgfpathlineto{\pgfqpoint{4.787551in}{2.891760in}}%
\pgfpathlineto{\pgfqpoint{0.452129in}{2.891653in}}%
\pgfpathlineto{\pgfqpoint{0.450532in}{2.890074in}}%
\pgfpathlineto{\pgfqpoint{0.449456in}{2.882754in}}%
\pgfpathlineto{\pgfqpoint{0.448970in}{2.845423in}}%
\pgfpathlineto{\pgfqpoint{0.448743in}{2.494445in}}%
\pgfpathlineto{\pgfqpoint{0.449609in}{0.617587in}}%
\pgfpathlineto{\pgfqpoint{0.451440in}{0.510578in}}%
\pgfpathlineto{\pgfqpoint{0.454007in}{0.473366in}}%
\pgfpathlineto{\pgfqpoint{0.457429in}{0.453862in}}%
\pgfpathlineto{\pgfqpoint{0.461572in}{0.442383in}}%
\pgfpathlineto{\pgfqpoint{0.466779in}{0.434443in}}%
\pgfpathlineto{\pgfqpoint{0.473639in}{0.428362in}}%
\pgfpathlineto{\pgfqpoint{0.483539in}{0.423260in}}%
\pgfpathlineto{\pgfqpoint{0.491900in}{0.420518in}}%
\pgfpathlineto{\pgfqpoint{0.491900in}{0.420518in}}%
\pgfusepath{stroke}%
\end{pgfscope}%
\begin{pgfscope}%
\pgfpathrectangle{\pgfqpoint{0.448634in}{0.402556in}}{\pgfqpoint{4.350661in}{2.489204in}} %
\pgfusepath{clip}%
\pgfsetrectcap%
\pgfsetroundjoin%
\pgfsetlinewidth{1.003750pt}%
\definecolor{currentstroke}{rgb}{0.890196,0.466667,0.760784}%
\pgfsetstrokecolor{currentstroke}%
\pgfsetdash{}{0pt}%
\pgfpathmoveto{\pgfqpoint{0.456425in}{1.370076in}}%
\pgfpathlineto{\pgfqpoint{0.459613in}{1.118694in}}%
\pgfpathlineto{\pgfqpoint{0.463699in}{0.961947in}}%
\pgfpathlineto{\pgfqpoint{0.468525in}{0.857549in}}%
\pgfpathlineto{\pgfqpoint{0.474091in}{0.783150in}}%
\pgfpathlineto{\pgfqpoint{0.480237in}{0.728846in}}%
\pgfpathlineto{\pgfqpoint{0.486986in}{0.687247in}}%
\pgfpathlineto{\pgfqpoint{0.494557in}{0.653501in}}%
\pgfpathlineto{\pgfqpoint{0.503133in}{0.625299in}}%
\pgfpathlineto{\pgfqpoint{0.512224in}{0.602697in}}%
\pgfpathlineto{\pgfqpoint{0.522237in}{0.583459in}}%
\pgfpathlineto{\pgfqpoint{0.534150in}{0.565699in}}%
\pgfpathlineto{\pgfqpoint{0.546311in}{0.551469in}}%
\pgfpathlineto{\pgfqpoint{0.559779in}{0.538874in}}%
\pgfpathlineto{\pgfqpoint{0.576184in}{0.526666in}}%
\pgfpathlineto{\pgfqpoint{0.595540in}{0.515329in}}%
\pgfpathlineto{\pgfqpoint{0.617739in}{0.505130in}}%
\pgfpathlineto{\pgfqpoint{0.642628in}{0.496141in}}%
\pgfpathlineto{\pgfqpoint{0.672186in}{0.487769in}}%
\pgfpathlineto{\pgfqpoint{0.708504in}{0.479817in}}%
\pgfpathlineto{\pgfqpoint{0.753711in}{0.472321in}}%
\pgfpathlineto{\pgfqpoint{0.807779in}{0.465658in}}%
\pgfpathlineto{\pgfqpoint{0.877177in}{0.459474in}}%
\pgfpathlineto{\pgfqpoint{0.961890in}{0.454230in}}%
\pgfpathlineto{\pgfqpoint{1.068413in}{0.449917in}}%
\pgfpathlineto{\pgfqpoint{1.201080in}{0.446841in}}%
\pgfpathlineto{\pgfqpoint{1.357698in}{0.445484in}}%
\pgfpathlineto{\pgfqpoint{1.525197in}{0.446236in}}%
\pgfpathlineto{\pgfqpoint{1.686150in}{0.449147in}}%
\pgfpathlineto{\pgfqpoint{1.823135in}{0.453753in}}%
\pgfpathlineto{\pgfqpoint{1.938306in}{0.459773in}}%
\pgfpathlineto{\pgfqpoint{2.031644in}{0.466769in}}%
\pgfpathlineto{\pgfqpoint{2.109641in}{0.474758in}}%
\pgfpathlineto{\pgfqpoint{2.174445in}{0.483550in}}%
\pgfpathlineto{\pgfqpoint{2.228200in}{0.492959in}}%
\pgfpathlineto{\pgfqpoint{2.275179in}{0.503379in}}%
\pgfpathlineto{\pgfqpoint{2.315341in}{0.514529in}}%
\pgfpathlineto{\pgfqpoint{2.350756in}{0.526691in}}%
\pgfpathlineto{\pgfqpoint{2.381376in}{0.539573in}}%
\pgfpathlineto{\pgfqpoint{2.407218in}{0.552701in}}%
\pgfpathlineto{\pgfqpoint{2.430278in}{0.566687in}}%
\pgfpathlineto{\pgfqpoint{2.452330in}{0.582655in}}%
\pgfpathlineto{\pgfqpoint{2.471436in}{0.599128in}}%
\pgfpathlineto{\pgfqpoint{2.489281in}{0.617357in}}%
\pgfpathlineto{\pgfqpoint{2.505714in}{0.637249in}}%
\pgfpathlineto{\pgfqpoint{2.520653in}{0.658629in}}%
\pgfpathlineto{\pgfqpoint{2.535242in}{0.683389in}}%
\pgfpathlineto{\pgfqpoint{2.549139in}{0.711563in}}%
\pgfpathlineto{\pgfqpoint{2.562111in}{0.743084in}}%
\pgfpathlineto{\pgfqpoint{2.574037in}{0.777833in}}%
\pgfpathlineto{\pgfqpoint{2.585516in}{0.818054in}}%
\pgfpathlineto{\pgfqpoint{2.596819in}{0.866123in}}%
\pgfpathlineto{\pgfqpoint{2.607570in}{0.922033in}}%
\pgfpathlineto{\pgfqpoint{2.617930in}{0.988184in}}%
\pgfpathlineto{\pgfqpoint{2.627961in}{1.067004in}}%
\pgfpathlineto{\pgfqpoint{2.637943in}{1.163406in}}%
\pgfpathlineto{\pgfqpoint{2.648425in}{1.287285in}}%
\pgfpathlineto{\pgfqpoint{2.660105in}{1.453524in}}%
\pgfpathlineto{\pgfqpoint{2.674776in}{1.696887in}}%
\pgfpathlineto{\pgfqpoint{2.687720in}{1.945365in}}%
\pgfpathlineto{\pgfqpoint{2.692672in}{2.079659in}}%
\pgfpathlineto{\pgfqpoint{2.693829in}{2.166768in}}%
\pgfpathlineto{\pgfqpoint{2.692561in}{2.233956in}}%
\pgfpathlineto{\pgfqpoint{2.689428in}{2.286100in}}%
\pgfpathlineto{\pgfqpoint{2.684846in}{2.328084in}}%
\pgfpathlineto{\pgfqpoint{2.678706in}{2.364747in}}%
\pgfpathlineto{\pgfqpoint{2.671332in}{2.395979in}}%
\pgfpathlineto{\pgfqpoint{2.662458in}{2.424060in}}%
\pgfpathlineto{\pgfqpoint{2.652324in}{2.448854in}}%
\pgfpathlineto{\pgfqpoint{2.641321in}{2.470317in}}%
\pgfpathlineto{\pgfqpoint{2.628593in}{2.490491in}}%
\pgfpathlineto{\pgfqpoint{2.614223in}{2.509166in}}%
\pgfpathlineto{\pgfqpoint{2.598383in}{2.526214in}}%
\pgfpathlineto{\pgfqpoint{2.579525in}{2.543054in}}%
\pgfpathlineto{\pgfqpoint{2.559464in}{2.557966in}}%
\pgfpathlineto{\pgfqpoint{2.536531in}{2.572220in}}%
\pgfpathlineto{\pgfqpoint{2.510777in}{2.585570in}}%
\pgfpathlineto{\pgfqpoint{2.482285in}{2.597864in}}%
\pgfpathlineto{\pgfqpoint{2.449059in}{2.609705in}}%
\pgfpathlineto{\pgfqpoint{2.411108in}{2.620714in}}%
\pgfpathlineto{\pgfqpoint{2.368475in}{2.630620in}}%
\pgfpathlineto{\pgfqpoint{2.321216in}{2.639231in}}%
\pgfpathlineto{\pgfqpoint{2.269389in}{2.646406in}}%
\pgfpathlineto{\pgfqpoint{2.210876in}{2.652197in}}%
\pgfpathlineto{\pgfqpoint{2.147889in}{2.656155in}}%
\pgfpathlineto{\pgfqpoint{2.080478in}{2.658134in}}%
\pgfpathlineto{\pgfqpoint{2.010870in}{2.657968in}}%
\pgfpathlineto{\pgfqpoint{1.939117in}{2.655567in}}%
\pgfpathlineto{\pgfqpoint{1.867449in}{2.650905in}}%
\pgfpathlineto{\pgfqpoint{1.798093in}{2.644130in}}%
\pgfpathlineto{\pgfqpoint{1.733263in}{2.635592in}}%
\pgfpathlineto{\pgfqpoint{1.672998in}{2.625505in}}%
\pgfpathlineto{\pgfqpoint{1.615197in}{2.613591in}}%
\pgfpathlineto{\pgfqpoint{1.562057in}{2.600379in}}%
\pgfpathlineto{\pgfqpoint{1.513606in}{2.586112in}}%
\pgfpathlineto{\pgfqpoint{1.467787in}{2.570313in}}%
\pgfpathlineto{\pgfqpoint{1.426720in}{2.553888in}}%
\pgfpathlineto{\pgfqpoint{1.388375in}{2.536250in}}%
\pgfpathlineto{\pgfqpoint{1.352808in}{2.517523in}}%
\pgfpathlineto{\pgfqpoint{1.320060in}{2.497874in}}%
\pgfpathlineto{\pgfqpoint{1.288314in}{2.476183in}}%
\pgfpathlineto{\pgfqpoint{1.259529in}{2.453804in}}%
\pgfpathlineto{\pgfqpoint{1.231991in}{2.429458in}}%
\pgfpathlineto{\pgfqpoint{1.207471in}{2.404832in}}%
\pgfpathlineto{\pgfqpoint{1.184356in}{2.378487in}}%
\pgfpathlineto{\pgfqpoint{1.162779in}{2.350487in}}%
\pgfpathlineto{\pgfqpoint{1.142846in}{2.320934in}}%
\pgfpathlineto{\pgfqpoint{1.124634in}{2.289961in}}%
\pgfpathlineto{\pgfqpoint{1.108188in}{2.257719in}}%
\pgfpathlineto{\pgfqpoint{1.092606in}{2.222114in}}%
\pgfpathlineto{\pgfqpoint{1.079031in}{2.185448in}}%
\pgfpathlineto{\pgfqpoint{1.067418in}{2.147909in}}%
\pgfpathlineto{\pgfqpoint{1.057166in}{2.107257in}}%
\pgfpathlineto{\pgfqpoint{1.048987in}{2.065995in}}%
\pgfpathlineto{\pgfqpoint{1.042499in}{2.021814in}}%
\pgfpathlineto{\pgfqpoint{1.038167in}{1.977289in}}%
\pgfpathlineto{\pgfqpoint{1.035860in}{1.930074in}}%
\pgfpathlineto{\pgfqpoint{1.035825in}{1.882785in}}%
\pgfpathlineto{\pgfqpoint{1.038034in}{1.835564in}}%
\pgfpathlineto{\pgfqpoint{1.042481in}{1.788550in}}%
\pgfpathlineto{\pgfqpoint{1.049187in}{1.741888in}}%
\pgfpathlineto{\pgfqpoint{1.057660in}{1.698149in}}%
\pgfpathlineto{\pgfqpoint{1.068242in}{1.655017in}}%
\pgfpathlineto{\pgfqpoint{1.080988in}{1.612658in}}%
\pgfpathlineto{\pgfqpoint{1.095061in}{1.573532in}}%
\pgfpathlineto{\pgfqpoint{1.111151in}{1.535438in}}%
\pgfpathlineto{\pgfqpoint{1.128158in}{1.500696in}}%
\pgfpathlineto{\pgfqpoint{1.146974in}{1.467198in}}%
\pgfpathlineto{\pgfqpoint{1.167580in}{1.435109in}}%
\pgfpathlineto{\pgfqpoint{1.189927in}{1.404584in}}%
\pgfpathlineto{\pgfqpoint{1.213941in}{1.375764in}}%
\pgfpathlineto{\pgfqpoint{1.237877in}{1.350397in}}%
\pgfpathlineto{\pgfqpoint{1.264812in}{1.325183in}}%
\pgfpathlineto{\pgfqpoint{1.293057in}{1.301922in}}%
\pgfpathlineto{\pgfqpoint{1.322467in}{1.280633in}}%
\pgfpathlineto{\pgfqpoint{1.352892in}{1.261301in}}%
\pgfpathlineto{\pgfqpoint{1.386169in}{1.242855in}}%
\pgfpathlineto{\pgfqpoint{1.420267in}{1.226488in}}%
\pgfpathlineto{\pgfqpoint{1.457102in}{1.211307in}}%
\pgfpathlineto{\pgfqpoint{1.496634in}{1.197521in}}%
\pgfpathlineto{\pgfqpoint{1.538801in}{1.185277in}}%
\pgfpathlineto{\pgfqpoint{1.583523in}{1.174638in}}%
\pgfpathlineto{\pgfqpoint{1.635012in}{1.164779in}}%
\pgfpathlineto{\pgfqpoint{1.706147in}{1.153754in}}%
\pgfpathlineto{\pgfqpoint{1.768576in}{1.143424in}}%
\pgfpathlineto{\pgfqpoint{1.796205in}{1.136568in}}%
\pgfpathlineto{\pgfqpoint{1.812764in}{1.130474in}}%
\pgfpathlineto{\pgfqpoint{1.824548in}{1.124087in}}%
\pgfpathlineto{\pgfqpoint{1.833279in}{1.116714in}}%
\pgfpathlineto{\pgfqpoint{1.838555in}{1.108852in}}%
\pgfpathlineto{\pgfqpoint{1.840632in}{1.101806in}}%
\pgfpathlineto{\pgfqpoint{1.840649in}{1.094369in}}%
\pgfpathlineto{\pgfqpoint{1.837948in}{1.084947in}}%
\pgfpathlineto{\pgfqpoint{1.833257in}{1.076581in}}%
\pgfpathlineto{\pgfqpoint{1.825825in}{1.067513in}}%
\pgfpathlineto{\pgfqpoint{1.813816in}{1.056825in}}%
\pgfpathlineto{\pgfqpoint{1.798821in}{1.046741in}}%
\pgfpathlineto{\pgfqpoint{1.781016in}{1.037443in}}%
\pgfpathlineto{\pgfqpoint{1.758447in}{1.028374in}}%
\pgfpathlineto{\pgfqpoint{1.733202in}{1.020800in}}%
\pgfpathlineto{\pgfqpoint{1.705409in}{1.014858in}}%
\pgfpathlineto{\pgfqpoint{1.675177in}{1.010702in}}%
\pgfpathlineto{\pgfqpoint{1.642609in}{1.008497in}}%
\pgfpathlineto{\pgfqpoint{1.607808in}{1.008425in}}%
\pgfpathlineto{\pgfqpoint{1.570885in}{1.010686in}}%
\pgfpathlineto{\pgfqpoint{1.534117in}{1.015177in}}%
\pgfpathlineto{\pgfqpoint{1.495454in}{1.022230in}}%
\pgfpathlineto{\pgfqpoint{1.457161in}{1.031561in}}%
\pgfpathlineto{\pgfqpoint{1.419337in}{1.043130in}}%
\pgfpathlineto{\pgfqpoint{1.382089in}{1.056928in}}%
\pgfpathlineto{\pgfqpoint{1.347544in}{1.072018in}}%
\pgfpathlineto{\pgfqpoint{1.313727in}{1.089133in}}%
\pgfpathlineto{\pgfqpoint{1.280762in}{1.108299in}}%
\pgfpathlineto{\pgfqpoint{1.248782in}{1.129537in}}%
\pgfpathlineto{\pgfqpoint{1.219709in}{1.151423in}}%
\pgfpathlineto{\pgfqpoint{1.191753in}{1.175139in}}%
\pgfpathlineto{\pgfqpoint{1.165031in}{1.200651in}}%
\pgfpathlineto{\pgfqpoint{1.139654in}{1.227900in}}%
\pgfpathlineto{\pgfqpoint{1.115716in}{1.256803in}}%
\pgfpathlineto{\pgfqpoint{1.093291in}{1.287255in}}%
\pgfpathlineto{\pgfqpoint{1.071181in}{1.321167in}}%
\pgfpathlineto{\pgfqpoint{1.050873in}{1.356525in}}%
\pgfpathlineto{\pgfqpoint{1.032371in}{1.393158in}}%
\pgfpathlineto{\pgfqpoint{1.014725in}{1.433149in}}%
\pgfpathlineto{\pgfqpoint{0.999033in}{1.474193in}}%
\pgfpathlineto{\pgfqpoint{0.984516in}{1.518469in}}%
\pgfpathlineto{\pgfqpoint{0.972020in}{1.563545in}}%
\pgfpathlineto{\pgfqpoint{0.960955in}{1.611687in}}%
\pgfpathlineto{\pgfqpoint{0.951542in}{1.662833in}}%
\pgfpathlineto{\pgfqpoint{0.944298in}{1.714440in}}%
\pgfpathlineto{\pgfqpoint{0.938962in}{1.768856in}}%
\pgfpathlineto{\pgfqpoint{0.935883in}{1.823500in}}%
\pgfpathlineto{\pgfqpoint{0.935047in}{1.878249in}}%
\pgfpathlineto{\pgfqpoint{0.936479in}{1.932981in}}%
\pgfpathlineto{\pgfqpoint{0.940019in}{1.985093in}}%
\pgfpathlineto{\pgfqpoint{0.945773in}{2.036944in}}%
\pgfpathlineto{\pgfqpoint{0.953424in}{2.085947in}}%
\pgfpathlineto{\pgfqpoint{0.962780in}{2.132009in}}%
\pgfpathlineto{\pgfqpoint{0.974303in}{2.177422in}}%
\pgfpathlineto{\pgfqpoint{0.987350in}{2.219660in}}%
\pgfpathlineto{\pgfqpoint{1.001686in}{2.258661in}}%
\pgfpathlineto{\pgfqpoint{1.018070in}{2.296590in}}%
\pgfpathlineto{\pgfqpoint{1.035422in}{2.331107in}}%
\pgfpathlineto{\pgfqpoint{1.054671in}{2.364281in}}%
\pgfpathlineto{\pgfqpoint{1.074428in}{2.393988in}}%
\pgfpathlineto{\pgfqpoint{1.095794in}{2.422201in}}%
\pgfpathlineto{\pgfqpoint{1.118685in}{2.448800in}}%
\pgfpathlineto{\pgfqpoint{1.142990in}{2.473703in}}%
\pgfpathlineto{\pgfqpoint{1.168575in}{2.496869in}}%
\pgfpathlineto{\pgfqpoint{1.197109in}{2.519663in}}%
\pgfpathlineto{\pgfqpoint{1.226751in}{2.540527in}}%
\pgfpathlineto{\pgfqpoint{1.259267in}{2.560673in}}%
\pgfpathlineto{\pgfqpoint{1.294638in}{2.579881in}}%
\pgfpathlineto{\pgfqpoint{1.332818in}{2.597981in}}%
\pgfpathlineto{\pgfqpoint{1.373744in}{2.614858in}}%
\pgfpathlineto{\pgfqpoint{1.417345in}{2.630443in}}%
\pgfpathlineto{\pgfqpoint{1.465658in}{2.645309in}}%
\pgfpathlineto{\pgfqpoint{1.518666in}{2.659201in}}%
\pgfpathlineto{\pgfqpoint{1.576335in}{2.671926in}}%
\pgfpathlineto{\pgfqpoint{1.638624in}{2.683340in}}%
\pgfpathlineto{\pgfqpoint{1.705488in}{2.693338in}}%
\pgfpathlineto{\pgfqpoint{1.779054in}{2.702059in}}%
\pgfpathlineto{\pgfqpoint{1.857124in}{2.709071in}}%
\pgfpathlineto{\pgfqpoint{1.939659in}{2.714274in}}%
\pgfpathlineto{\pgfqpoint{2.026625in}{2.717507in}}%
\pgfpathlineto{\pgfqpoint{2.113632in}{2.718516in}}%
\pgfpathlineto{\pgfqpoint{2.198461in}{2.717294in}}%
\pgfpathlineto{\pgfqpoint{2.278892in}{2.713919in}}%
\pgfpathlineto{\pgfqpoint{2.352704in}{2.708586in}}%
\pgfpathlineto{\pgfqpoint{2.417683in}{2.701696in}}%
\pgfpathlineto{\pgfqpoint{2.473796in}{2.693615in}}%
\pgfpathlineto{\pgfqpoint{2.523166in}{2.684350in}}%
\pgfpathlineto{\pgfqpoint{2.565751in}{2.674182in}}%
\pgfpathlineto{\pgfqpoint{2.601534in}{2.663520in}}%
\pgfpathlineto{\pgfqpoint{2.632600in}{2.652116in}}%
\pgfpathlineto{\pgfqpoint{2.658921in}{2.640301in}}%
\pgfpathlineto{\pgfqpoint{2.682458in}{2.627402in}}%
\pgfpathlineto{\pgfqpoint{2.703081in}{2.613533in}}%
\pgfpathlineto{\pgfqpoint{2.720690in}{2.598936in}}%
\pgfpathlineto{\pgfqpoint{2.735275in}{2.584006in}}%
\pgfpathlineto{\pgfqpoint{2.748328in}{2.567326in}}%
\pgfpathlineto{\pgfqpoint{2.759558in}{2.548992in}}%
\pgfpathlineto{\pgfqpoint{2.768788in}{2.529250in}}%
\pgfpathlineto{\pgfqpoint{2.776013in}{2.508440in}}%
\pgfpathlineto{\pgfqpoint{2.781876in}{2.484480in}}%
\pgfpathlineto{\pgfqpoint{2.786091in}{2.457536in}}%
\pgfpathlineto{\pgfqpoint{2.788706in}{2.425324in}}%
\pgfpathlineto{\pgfqpoint{2.789412in}{2.388001in}}%
\pgfpathlineto{\pgfqpoint{2.787946in}{2.340740in}}%
\pgfpathlineto{\pgfqpoint{2.783655in}{2.278708in}}%
\pgfpathlineto{\pgfqpoint{2.774273in}{2.179722in}}%
\pgfpathlineto{\pgfqpoint{2.743600in}{1.868057in}}%
\pgfpathlineto{\pgfqpoint{2.730102in}{1.701999in}}%
\pgfpathlineto{\pgfqpoint{2.717277in}{1.515888in}}%
\pgfpathlineto{\pgfqpoint{2.702592in}{1.267536in}}%
\pgfpathlineto{\pgfqpoint{2.684421in}{0.964569in}}%
\pgfpathlineto{\pgfqpoint{2.675359in}{0.850539in}}%
\pgfpathlineto{\pgfqpoint{2.667011in}{0.771463in}}%
\pgfpathlineto{\pgfqpoint{2.658730in}{0.712483in}}%
\pgfpathlineto{\pgfqpoint{2.650149in}{0.666225in}}%
\pgfpathlineto{\pgfqpoint{2.640788in}{0.627874in}}%
\pgfpathlineto{\pgfqpoint{2.631107in}{0.597479in}}%
\pgfpathlineto{\pgfqpoint{2.620960in}{0.572693in}}%
\pgfpathlineto{\pgfqpoint{2.609806in}{0.551336in}}%
\pgfpathlineto{\pgfqpoint{2.597986in}{0.533492in}}%
\pgfpathlineto{\pgfqpoint{2.584434in}{0.517342in}}%
\pgfpathlineto{\pgfqpoint{2.571043in}{0.504639in}}%
\pgfpathlineto{\pgfqpoint{2.554719in}{0.492290in}}%
\pgfpathlineto{\pgfqpoint{2.537384in}{0.481896in}}%
\pgfpathlineto{\pgfqpoint{2.517300in}{0.472354in}}%
\pgfpathlineto{\pgfqpoint{2.492467in}{0.463170in}}%
\pgfpathlineto{\pgfqpoint{2.462903in}{0.454828in}}%
\pgfpathlineto{\pgfqpoint{2.428689in}{0.447540in}}%
\pgfpathlineto{\pgfqpoint{2.385594in}{0.440735in}}%
\pgfpathlineto{\pgfqpoint{2.331480in}{0.434582in}}%
\pgfpathlineto{\pgfqpoint{2.262038in}{0.429078in}}%
\pgfpathlineto{\pgfqpoint{2.170774in}{0.424238in}}%
\pgfpathlineto{\pgfqpoint{2.049009in}{0.420136in}}%
\pgfpathlineto{\pgfqpoint{1.879359in}{0.416786in}}%
\pgfpathlineto{\pgfqpoint{1.640082in}{0.414420in}}%
\pgfpathlineto{\pgfqpoint{1.322485in}{0.413572in}}%
\pgfpathlineto{\pgfqpoint{1.020117in}{0.414853in}}%
\pgfpathlineto{\pgfqpoint{0.822179in}{0.417720in}}%
\pgfpathlineto{\pgfqpoint{0.704758in}{0.421438in}}%
\pgfpathlineto{\pgfqpoint{0.630900in}{0.425840in}}%
\pgfpathlineto{\pgfqpoint{0.583239in}{0.430751in}}%
\pgfpathlineto{\pgfqpoint{0.550958in}{0.436148in}}%
\pgfpathlineto{\pgfqpoint{0.527634in}{0.442223in}}%
\pgfpathlineto{\pgfqpoint{0.511180in}{0.448670in}}%
\pgfpathlineto{\pgfqpoint{0.499484in}{0.455272in}}%
\pgfpathlineto{\pgfqpoint{0.488860in}{0.463912in}}%
\pgfpathlineto{\pgfqpoint{0.481276in}{0.472812in}}%
\pgfpathlineto{\pgfqpoint{0.474044in}{0.485218in}}%
\pgfpathlineto{\pgfqpoint{0.468730in}{0.498845in}}%
\pgfpathlineto{\pgfqpoint{0.463855in}{0.517948in}}%
\pgfpathlineto{\pgfqpoint{0.459671in}{0.544898in}}%
\pgfpathlineto{\pgfqpoint{0.456382in}{0.582040in}}%
\pgfpathlineto{\pgfqpoint{0.453730in}{0.639208in}}%
\pgfpathlineto{\pgfqpoint{0.451648in}{0.738746in}}%
\pgfpathlineto{\pgfqpoint{0.450199in}{0.932896in}}%
\pgfpathlineto{\pgfqpoint{0.449337in}{1.415800in}}%
\pgfpathlineto{\pgfqpoint{0.449566in}{2.692761in}}%
\pgfpathlineto{\pgfqpoint{0.451015in}{2.857034in}}%
\pgfpathlineto{\pgfqpoint{0.452823in}{2.879319in}}%
\pgfpathlineto{\pgfqpoint{0.455242in}{2.886190in}}%
\pgfpathlineto{\pgfqpoint{0.458712in}{2.889064in}}%
\pgfpathlineto{\pgfqpoint{0.465087in}{2.890564in}}%
\pgfpathlineto{\pgfqpoint{0.482469in}{2.891424in}}%
\pgfpathlineto{\pgfqpoint{0.567306in}{2.891731in}}%
\pgfpathlineto{\pgfqpoint{2.860104in}{2.891760in}}%
\pgfpathlineto{\pgfqpoint{4.789601in}{2.890872in}}%
\pgfpathlineto{\pgfqpoint{4.793810in}{2.889686in}}%
\pgfpathlineto{\pgfqpoint{4.795539in}{2.888226in}}%
\pgfpathlineto{\pgfqpoint{4.797116in}{2.881043in}}%
\pgfpathlineto{\pgfqpoint{4.798041in}{2.856180in}}%
\pgfpathlineto{\pgfqpoint{4.798041in}{2.856180in}}%
\pgfusepath{stroke}%
\end{pgfscope}%
\begin{pgfscope}%
\pgfpathrectangle{\pgfqpoint{0.448634in}{0.402556in}}{\pgfqpoint{4.350661in}{2.489204in}} %
\pgfusepath{clip}%
\pgfsetrectcap%
\pgfsetroundjoin%
\pgfsetlinewidth{1.003750pt}%
\definecolor{currentstroke}{rgb}{0.890196,0.466667,0.760784}%
\pgfsetstrokecolor{currentstroke}%
\pgfsetdash{}{0pt}%
\pgfpathmoveto{\pgfqpoint{3.430047in}{0.402666in}}%
\pgfpathlineto{\pgfqpoint{2.847062in}{0.403860in}}%
\pgfpathlineto{\pgfqpoint{2.792711in}{0.405833in}}%
\pgfpathlineto{\pgfqpoint{2.771087in}{0.408445in}}%
\pgfpathlineto{\pgfqpoint{2.758426in}{0.411997in}}%
\pgfpathlineto{\pgfqpoint{2.750610in}{0.416321in}}%
\pgfpathlineto{\pgfqpoint{2.744166in}{0.422951in}}%
\pgfpathlineto{\pgfqpoint{2.739632in}{0.431419in}}%
\pgfpathlineto{\pgfqpoint{2.736051in}{0.443154in}}%
\pgfpathlineto{\pgfqpoint{2.732802in}{0.462705in}}%
\pgfpathlineto{\pgfqpoint{2.730155in}{0.494916in}}%
\pgfpathlineto{\pgfqpoint{2.728235in}{0.549631in}}%
\pgfpathlineto{\pgfqpoint{2.727366in}{0.644213in}}%
\pgfpathlineto{\pgfqpoint{2.728341in}{0.791071in}}%
\pgfpathlineto{\pgfqpoint{2.731829in}{0.975227in}}%
\pgfpathlineto{\pgfqpoint{2.737684in}{1.164285in}}%
\pgfpathlineto{\pgfqpoint{2.745203in}{1.333331in}}%
\pgfpathlineto{\pgfqpoint{2.754068in}{1.484831in}}%
\pgfpathlineto{\pgfqpoint{2.764160in}{1.621247in}}%
\pgfpathlineto{\pgfqpoint{2.776679in}{1.759894in}}%
\pgfpathlineto{\pgfqpoint{2.789167in}{1.870991in}}%
\pgfpathlineto{\pgfqpoint{2.805060in}{1.991582in}}%
\pgfpathlineto{\pgfqpoint{2.820332in}{2.087072in}}%
\pgfpathlineto{\pgfqpoint{2.837403in}{2.179620in}}%
\pgfpathlineto{\pgfqpoint{2.857133in}{2.274034in}}%
\pgfpathlineto{\pgfqpoint{2.901327in}{2.479444in}}%
\pgfpathlineto{\pgfqpoint{2.906431in}{2.516314in}}%
\pgfpathlineto{\pgfqpoint{2.908301in}{2.546096in}}%
\pgfpathlineto{\pgfqpoint{2.907676in}{2.568477in}}%
\pgfpathlineto{\pgfqpoint{2.905105in}{2.588159in}}%
\pgfpathlineto{\pgfqpoint{2.900926in}{2.604901in}}%
\pgfpathlineto{\pgfqpoint{2.894753in}{2.620813in}}%
\pgfpathlineto{\pgfqpoint{2.886614in}{2.635521in}}%
\pgfpathlineto{\pgfqpoint{2.876746in}{2.648775in}}%
\pgfpathlineto{\pgfqpoint{2.863810in}{2.662073in}}%
\pgfpathlineto{\pgfqpoint{2.849597in}{2.673548in}}%
\pgfpathlineto{\pgfqpoint{2.832571in}{2.684593in}}%
\pgfpathlineto{\pgfqpoint{2.810800in}{2.695935in}}%
\pgfpathlineto{\pgfqpoint{2.784228in}{2.706989in}}%
\pgfpathlineto{\pgfqpoint{2.752898in}{2.717401in}}%
\pgfpathlineto{\pgfqpoint{2.716881in}{2.726980in}}%
\pgfpathlineto{\pgfqpoint{2.674105in}{2.736045in}}%
\pgfpathlineto{\pgfqpoint{2.622443in}{2.744642in}}%
\pgfpathlineto{\pgfqpoint{2.561912in}{2.752377in}}%
\pgfpathlineto{\pgfqpoint{2.490375in}{2.759176in}}%
\pgfpathlineto{\pgfqpoint{2.407855in}{2.764686in}}%
\pgfpathlineto{\pgfqpoint{2.314380in}{2.768610in}}%
\pgfpathlineto{\pgfqpoint{2.096865in}{2.770441in}}%
\pgfpathlineto{\pgfqpoint{1.983769in}{2.767993in}}%
\pgfpathlineto{\pgfqpoint{1.879430in}{2.763471in}}%
\pgfpathlineto{\pgfqpoint{1.770839in}{2.756462in}}%
\pgfpathlineto{\pgfqpoint{1.681938in}{2.748298in}}%
\pgfpathlineto{\pgfqpoint{1.601880in}{2.738824in}}%
\pgfpathlineto{\pgfqpoint{1.517732in}{2.726543in}}%
\pgfpathlineto{\pgfqpoint{1.481301in}{2.719532in}}%
\pgfpathlineto{\pgfqpoint{1.475248in}{2.716927in}}%
\pgfpathlineto{\pgfqpoint{1.415363in}{2.704217in}}%
\pgfpathlineto{\pgfqpoint{1.360139in}{2.690257in}}%
\pgfpathlineto{\pgfqpoint{1.309627in}{2.675170in}}%
\pgfpathlineto{\pgfqpoint{1.263858in}{2.659182in}}%
\pgfpathlineto{\pgfqpoint{1.220795in}{2.641754in}}%
\pgfpathlineto{\pgfqpoint{1.182581in}{2.623746in}}%
\pgfpathlineto{\pgfqpoint{1.147195in}{2.604578in}}%
\pgfpathlineto{\pgfqpoint{1.114687in}{2.584416in}}%
\pgfpathlineto{\pgfqpoint{1.092532in}{2.568654in}}%
\pgfpathlineto{\pgfqpoint{1.079965in}{2.558827in}}%
\pgfpathlineto{\pgfqpoint{1.051740in}{2.535533in}}%
\pgfpathlineto{\pgfqpoint{1.026500in}{2.511878in}}%
\pgfpathlineto{\pgfqpoint{1.002577in}{2.486496in}}%
\pgfpathlineto{\pgfqpoint{0.980081in}{2.459458in}}%
\pgfpathlineto{\pgfqpoint{0.959092in}{2.430877in}}%
\pgfpathlineto{\pgfqpoint{0.938410in}{2.398852in}}%
\pgfpathlineto{\pgfqpoint{0.922028in}{2.369491in}}%
\pgfpathlineto{\pgfqpoint{0.903596in}{2.332813in}}%
\pgfpathlineto{\pgfqpoint{0.887032in}{2.294985in}}%
\pgfpathlineto{\pgfqpoint{0.871404in}{2.253908in}}%
\pgfpathlineto{\pgfqpoint{0.856871in}{2.209639in}}%
\pgfpathlineto{\pgfqpoint{0.844318in}{2.164591in}}%
\pgfpathlineto{\pgfqpoint{0.840367in}{2.147772in}}%
\pgfpathlineto{\pgfqpoint{0.828437in}{2.094743in}}%
\pgfpathlineto{\pgfqpoint{0.817505in}{2.036331in}}%
\pgfpathlineto{\pgfqpoint{0.810175in}{1.987265in}}%
\pgfpathlineto{\pgfqpoint{0.807069in}{1.960134in}}%
\pgfpathlineto{\pgfqpoint{0.799330in}{1.891004in}}%
\pgfpathlineto{\pgfqpoint{0.793154in}{1.816666in}}%
\pgfpathlineto{\pgfqpoint{0.787893in}{1.719783in}}%
\pgfpathlineto{\pgfqpoint{0.783944in}{1.605372in}}%
\pgfpathlineto{\pgfqpoint{0.779255in}{1.486016in}}%
\pgfpathlineto{\pgfqpoint{0.775089in}{1.433968in}}%
\pgfpathlineto{\pgfqpoint{0.770204in}{1.399579in}}%
\pgfpathlineto{\pgfqpoint{0.764173in}{1.373096in}}%
\pgfpathlineto{\pgfqpoint{0.757729in}{1.354615in}}%
\pgfpathlineto{\pgfqpoint{0.751103in}{1.341768in}}%
\pgfpathlineto{\pgfqpoint{0.743923in}{1.332450in}}%
\pgfpathlineto{\pgfqpoint{0.736880in}{1.326642in}}%
\pgfpathlineto{\pgfqpoint{0.728790in}{1.323060in}}%
\pgfpathlineto{\pgfqpoint{0.720152in}{1.322159in}}%
\pgfpathlineto{\pgfqpoint{0.711585in}{1.323773in}}%
\pgfpathlineto{\pgfqpoint{0.701534in}{1.328467in}}%
\pgfpathlineto{\pgfqpoint{0.690660in}{1.336692in}}%
\pgfpathlineto{\pgfqpoint{0.679459in}{1.348474in}}%
\pgfpathlineto{\pgfqpoint{0.668244in}{1.363687in}}%
\pgfpathlineto{\pgfqpoint{0.656083in}{1.384313in}}%
\pgfpathlineto{\pgfqpoint{0.643575in}{1.410521in}}%
\pgfpathlineto{\pgfqpoint{0.631996in}{1.440036in}}%
\pgfpathlineto{\pgfqpoint{0.620606in}{1.475019in}}%
\pgfpathlineto{\pgfqpoint{0.609684in}{1.515443in}}%
\pgfpathlineto{\pgfqpoint{0.594921in}{1.585615in}}%
\pgfpathlineto{\pgfqpoint{0.585196in}{1.644307in}}%
\pgfpathlineto{\pgfqpoint{0.577101in}{1.705839in}}%
\pgfpathlineto{\pgfqpoint{0.570356in}{1.770084in}}%
\pgfpathlineto{\pgfqpoint{0.567731in}{1.799801in}}%
\pgfpathlineto{\pgfqpoint{0.561180in}{1.886597in}}%
\pgfpathlineto{\pgfqpoint{0.556058in}{1.983496in}}%
\pgfpathlineto{\pgfqpoint{0.552853in}{2.085484in}}%
\pgfpathlineto{\pgfqpoint{0.551559in}{2.192507in}}%
\pgfpathlineto{\pgfqpoint{0.552485in}{2.297046in}}%
\pgfpathlineto{\pgfqpoint{0.555572in}{2.394057in}}%
\pgfpathlineto{\pgfqpoint{0.560558in}{2.478492in}}%
\pgfpathlineto{\pgfqpoint{0.566735in}{2.545324in}}%
\pgfpathlineto{\pgfqpoint{0.574150in}{2.601937in}}%
\pgfpathlineto{\pgfqpoint{0.582466in}{2.648258in}}%
\pgfpathlineto{\pgfqpoint{0.592075in}{2.689105in}}%
\pgfpathlineto{\pgfqpoint{0.601800in}{2.719481in}}%
\pgfpathlineto{\pgfqpoint{0.611952in}{2.744265in}}%
\pgfpathlineto{\pgfqpoint{0.623066in}{2.765650in}}%
\pgfpathlineto{\pgfqpoint{0.634815in}{2.783557in}}%
\pgfpathlineto{\pgfqpoint{0.648288in}{2.799792in}}%
\pgfpathlineto{\pgfqpoint{0.661634in}{2.812557in}}%
\pgfpathlineto{\pgfqpoint{0.677956in}{2.824907in}}%
\pgfpathlineto{\pgfqpoint{0.695342in}{2.835186in}}%
\pgfpathlineto{\pgfqpoint{0.713526in}{2.843458in}}%
\pgfpathlineto{\pgfqpoint{0.736341in}{2.851689in}}%
\pgfpathlineto{\pgfqpoint{0.763852in}{2.859154in}}%
\pgfpathlineto{\pgfqpoint{0.795996in}{2.865547in}}%
\pgfpathlineto{\pgfqpoint{0.837013in}{2.871337in}}%
\pgfpathlineto{\pgfqpoint{0.889037in}{2.876312in}}%
\pgfpathlineto{\pgfqpoint{0.958547in}{2.880567in}}%
\pgfpathlineto{\pgfqpoint{1.054212in}{2.884048in}}%
\pgfpathlineto{\pgfqpoint{1.193412in}{2.886754in}}%
\pgfpathlineto{\pgfqpoint{2.102690in}{2.890547in}}%
\pgfpathlineto{\pgfqpoint{3.383959in}{2.890630in}}%
\pgfpathlineto{\pgfqpoint{4.069187in}{2.889001in}}%
\pgfpathlineto{\pgfqpoint{4.308461in}{2.886463in}}%
\pgfpathlineto{\pgfqpoint{4.428068in}{2.883158in}}%
\pgfpathlineto{\pgfqpoint{4.501938in}{2.879023in}}%
\pgfpathlineto{\pgfqpoint{4.551781in}{2.874108in}}%
\pgfpathlineto{\pgfqpoint{4.586251in}{2.868631in}}%
\pgfpathlineto{\pgfqpoint{4.613923in}{2.862001in}}%
\pgfpathlineto{\pgfqpoint{4.634737in}{2.854798in}}%
\pgfpathlineto{\pgfqpoint{4.650803in}{2.847168in}}%
\pgfpathlineto{\pgfqpoint{4.665969in}{2.837430in}}%
\pgfpathlineto{\pgfqpoint{4.678133in}{2.826970in}}%
\pgfpathlineto{\pgfqpoint{4.688961in}{2.814737in}}%
\pgfpathlineto{\pgfqpoint{4.698296in}{2.800985in}}%
\pgfpathlineto{\pgfqpoint{4.707215in}{2.783900in}}%
\pgfpathlineto{\pgfqpoint{4.715323in}{2.763519in}}%
\pgfpathlineto{\pgfqpoint{4.723089in}{2.737630in}}%
\pgfpathlineto{\pgfqpoint{4.730052in}{2.706274in}}%
\pgfpathlineto{\pgfqpoint{4.736513in}{2.667144in}}%
\pgfpathlineto{\pgfqpoint{4.742669in}{2.615353in}}%
\pgfpathlineto{\pgfqpoint{4.748204in}{2.548447in}}%
\pgfpathlineto{\pgfqpoint{4.752889in}{2.463986in}}%
\pgfpathlineto{\pgfqpoint{4.756814in}{2.352064in}}%
\pgfpathlineto{\pgfqpoint{4.759513in}{2.210214in}}%
\pgfpathlineto{\pgfqpoint{4.760603in}{2.033487in}}%
\pgfpathlineto{\pgfqpoint{4.759552in}{1.841823in}}%
\pgfpathlineto{\pgfqpoint{4.756349in}{1.657660in}}%
\pgfpathlineto{\pgfqpoint{4.751296in}{1.495968in}}%
\pgfpathlineto{\pgfqpoint{4.744971in}{1.366734in}}%
\pgfpathlineto{\pgfqpoint{4.737707in}{1.262522in}}%
\pgfpathlineto{\pgfqpoint{4.729075in}{1.170957in}}%
\pgfpathlineto{\pgfqpoint{4.720060in}{1.099515in}}%
\pgfpathlineto{\pgfqpoint{4.710161in}{1.038329in}}%
\pgfpathlineto{\pgfqpoint{4.699271in}{0.985009in}}%
\pgfpathlineto{\pgfqpoint{4.687728in}{0.939603in}}%
\pgfpathlineto{\pgfqpoint{4.676092in}{0.902073in}}%
\pgfpathlineto{\pgfqpoint{4.676092in}{0.902073in}}%
\pgfusepath{stroke}%
\end{pgfscope}%
\begin{pgfscope}%
\pgfpathrectangle{\pgfqpoint{0.448634in}{0.402556in}}{\pgfqpoint{4.350661in}{2.489204in}} %
\pgfusepath{clip}%
\pgfsetrectcap%
\pgfsetroundjoin%
\pgfsetlinewidth{1.003750pt}%
\definecolor{currentstroke}{rgb}{0.890196,0.466667,0.760784}%
\pgfsetstrokecolor{currentstroke}%
\pgfsetdash{}{0pt}%
\pgfpathmoveto{\pgfqpoint{2.795521in}{1.982745in}}%
\pgfpathlineto{\pgfqpoint{2.781780in}{1.874357in}}%
\pgfpathlineto{\pgfqpoint{2.769352in}{1.758234in}}%
\pgfpathlineto{\pgfqpoint{2.758095in}{1.631942in}}%
\pgfpathlineto{\pgfqpoint{2.747786in}{1.490551in}}%
\pgfpathlineto{\pgfqpoint{2.738644in}{1.334082in}}%
\pgfpathlineto{\pgfqpoint{2.730580in}{1.157591in}}%
\pgfpathlineto{\pgfqpoint{2.723334in}{0.948663in}}%
\pgfpathlineto{\pgfqpoint{2.709783in}{0.530788in}}%
\pgfpathlineto{\pgfqpoint{2.705868in}{0.488716in}}%
\pgfpathlineto{\pgfqpoint{2.701769in}{0.464281in}}%
\pgfpathlineto{\pgfqpoint{2.697021in}{0.447744in}}%
\pgfpathlineto{\pgfqpoint{2.691859in}{0.436812in}}%
\pgfpathlineto{\pgfqpoint{2.686245in}{0.429229in}}%
\pgfpathlineto{\pgfqpoint{2.679348in}{0.423188in}}%
\pgfpathlineto{\pgfqpoint{2.669540in}{0.417856in}}%
\pgfpathlineto{\pgfqpoint{2.656987in}{0.413810in}}%
\pgfpathlineto{\pgfqpoint{2.637654in}{0.410337in}}%
\pgfpathlineto{\pgfqpoint{2.607297in}{0.407617in}}%
\pgfpathlineto{\pgfqpoint{2.555121in}{0.405574in}}%
\pgfpathlineto{\pgfqpoint{2.450714in}{0.404139in}}%
\pgfpathlineto{\pgfqpoint{2.176624in}{0.403275in}}%
\pgfpathlineto{\pgfqpoint{1.130290in}{0.402953in}}%
\pgfpathlineto{\pgfqpoint{0.516849in}{0.404175in}}%
\pgfpathlineto{\pgfqpoint{0.466848in}{0.405970in}}%
\pgfpathlineto{\pgfqpoint{0.456130in}{0.407931in}}%
\pgfpathlineto{\pgfqpoint{0.452340in}{0.410303in}}%
\pgfpathlineto{\pgfqpoint{0.450346in}{0.414662in}}%
\pgfpathlineto{\pgfqpoint{0.449266in}{0.424523in}}%
\pgfpathlineto{\pgfqpoint{0.448771in}{0.464344in}}%
\pgfpathlineto{\pgfqpoint{0.448640in}{0.850171in}}%
\pgfpathlineto{\pgfqpoint{0.448679in}{2.891318in}}%
\pgfpathlineto{\pgfqpoint{0.448679in}{2.891318in}}%
\pgfusepath{stroke}%
\end{pgfscope}%
\begin{pgfscope}%
\pgfpathrectangle{\pgfqpoint{0.448634in}{0.402556in}}{\pgfqpoint{4.350661in}{2.489204in}} %
\pgfusepath{clip}%
\pgfsetrectcap%
\pgfsetroundjoin%
\pgfsetlinewidth{1.003750pt}%
\definecolor{currentstroke}{rgb}{0.890196,0.466667,0.760784}%
\pgfsetstrokecolor{currentstroke}%
\pgfsetdash{}{0pt}%
\pgfpathmoveto{\pgfqpoint{3.429565in}{0.402653in}}%
\pgfpathlineto{\pgfqpoint{2.840053in}{0.403823in}}%
\pgfpathlineto{\pgfqpoint{2.787880in}{0.405806in}}%
\pgfpathlineto{\pgfqpoint{2.766286in}{0.408695in}}%
\pgfpathlineto{\pgfqpoint{2.755802in}{0.411949in}}%
\pgfpathlineto{\pgfqpoint{2.748124in}{0.416574in}}%
\pgfpathlineto{\pgfqpoint{2.743356in}{0.421646in}}%
\pgfpathlineto{\pgfqpoint{2.738772in}{0.430076in}}%
\pgfpathlineto{\pgfqpoint{2.735224in}{0.441823in}}%
\pgfpathlineto{\pgfqpoint{2.732078in}{0.461396in}}%
\pgfpathlineto{\pgfqpoint{2.729567in}{0.493622in}}%
\pgfpathlineto{\pgfqpoint{2.727736in}{0.550831in}}%
\pgfpathlineto{\pgfqpoint{2.727026in}{0.650394in}}%
\pgfpathlineto{\pgfqpoint{2.728278in}{0.804717in}}%
\pgfpathlineto{\pgfqpoint{2.732045in}{0.991356in}}%
\pgfpathlineto{\pgfqpoint{2.738219in}{1.182892in}}%
\pgfpathlineto{\pgfqpoint{2.745866in}{1.349437in}}%
\pgfpathlineto{\pgfqpoint{2.754971in}{1.500919in}}%
\pgfpathlineto{\pgfqpoint{2.765318in}{1.637310in}}%
\pgfpathlineto{\pgfqpoint{2.778131in}{1.775920in}}%
\pgfpathlineto{\pgfqpoint{2.790919in}{1.886972in}}%
\pgfpathlineto{\pgfqpoint{2.807519in}{2.009953in}}%
\pgfpathlineto{\pgfqpoint{2.823182in}{2.105360in}}%
\pgfpathlineto{\pgfqpoint{2.840638in}{2.197813in}}%
\pgfpathlineto{\pgfqpoint{2.861228in}{2.294551in}}%
\pgfpathlineto{\pgfqpoint{2.896286in}{2.456398in}}%
\pgfpathlineto{\pgfqpoint{2.903540in}{2.500421in}}%
\pgfpathlineto{\pgfqpoint{2.906686in}{2.532570in}}%
\pgfpathlineto{\pgfqpoint{2.907066in}{2.557447in}}%
\pgfpathlineto{\pgfqpoint{2.905198in}{2.579733in}}%
\pgfpathlineto{\pgfqpoint{2.901235in}{2.599106in}}%
\pgfpathlineto{\pgfqpoint{2.895692in}{2.615320in}}%
\pgfpathlineto{\pgfqpoint{2.888155in}{2.630442in}}%
\pgfpathlineto{\pgfqpoint{2.878805in}{2.644177in}}%
\pgfpathlineto{\pgfqpoint{2.867967in}{2.656401in}}%
\pgfpathlineto{\pgfqpoint{2.854222in}{2.668597in}}%
\pgfpathlineto{\pgfqpoint{2.837548in}{2.680319in}}%
\pgfpathlineto{\pgfqpoint{2.818043in}{2.691323in}}%
\pgfpathlineto{\pgfqpoint{2.793783in}{2.702329in}}%
\pgfpathlineto{\pgfqpoint{2.764761in}{2.712874in}}%
\pgfpathlineto{\pgfqpoint{2.731038in}{2.722708in}}%
\pgfpathlineto{\pgfqpoint{2.690539in}{2.732133in}}%
\pgfpathlineto{\pgfqpoint{2.643291in}{2.740826in}}%
\pgfpathlineto{\pgfqpoint{2.587173in}{2.748862in}}%
\pgfpathlineto{\pgfqpoint{2.520039in}{2.756121in}}%
\pgfpathlineto{\pgfqpoint{2.441910in}{2.762217in}}%
\pgfpathlineto{\pgfqpoint{2.352815in}{2.766842in}}%
\pgfpathlineto{\pgfqpoint{2.259306in}{2.769491in}}%
\pgfpathlineto{\pgfqpoint{2.152720in}{2.770418in}}%
\pgfpathlineto{\pgfqpoint{2.043960in}{2.769152in}}%
\pgfpathlineto{\pgfqpoint{1.930889in}{2.765585in}}%
\pgfpathlineto{\pgfqpoint{1.828764in}{2.760042in}}%
\pgfpathlineto{\pgfqpoint{1.724586in}{2.752093in}}%
\pgfpathlineto{\pgfqpoint{1.640107in}{2.743198in}}%
\pgfpathlineto{\pgfqpoint{1.562326in}{2.732802in}}%
\pgfpathlineto{\pgfqpoint{1.484801in}{2.720261in}}%
\pgfpathlineto{\pgfqpoint{1.480740in}{2.718577in}}%
\pgfpathlineto{\pgfqpoint{1.476869in}{2.716386in}}%
\pgfpathlineto{\pgfqpoint{1.399951in}{2.699589in}}%
\pgfpathlineto{\pgfqpoint{1.346981in}{2.685510in}}%
\pgfpathlineto{\pgfqpoint{1.298718in}{2.670433in}}%
\pgfpathlineto{\pgfqpoint{1.255180in}{2.654620in}}%
\pgfpathlineto{\pgfqpoint{1.212295in}{2.636631in}}%
\pgfpathlineto{\pgfqpoint{1.176275in}{2.619070in}}%
\pgfpathlineto{\pgfqpoint{1.141096in}{2.599410in}}%
\pgfpathlineto{\pgfqpoint{1.108829in}{2.578748in}}%
\pgfpathlineto{\pgfqpoint{1.090730in}{2.565005in}}%
\pgfpathlineto{\pgfqpoint{1.055192in}{2.536305in}}%
\pgfpathlineto{\pgfqpoint{1.029879in}{2.512754in}}%
\pgfpathlineto{\pgfqpoint{1.005877in}{2.487469in}}%
\pgfpathlineto{\pgfqpoint{0.983297in}{2.460522in}}%
\pgfpathlineto{\pgfqpoint{0.962223in}{2.432024in}}%
\pgfpathlineto{\pgfqpoint{0.941451in}{2.400075in}}%
\pgfpathlineto{\pgfqpoint{0.926153in}{2.372878in}}%
\pgfpathlineto{\pgfqpoint{0.907525in}{2.336330in}}%
\pgfpathlineto{\pgfqpoint{0.890783in}{2.298605in}}%
\pgfpathlineto{\pgfqpoint{0.874988in}{2.257612in}}%
\pgfpathlineto{\pgfqpoint{0.860308in}{2.213406in}}%
\pgfpathlineto{\pgfqpoint{0.846895in}{2.166054in}}%
\pgfpathlineto{\pgfqpoint{0.837275in}{2.125200in}}%
\pgfpathlineto{\pgfqpoint{0.825765in}{2.069488in}}%
\pgfpathlineto{\pgfqpoint{0.816268in}{2.013284in}}%
\pgfpathlineto{\pgfqpoint{0.802493in}{1.899890in}}%
\pgfpathlineto{\pgfqpoint{0.796258in}{1.825559in}}%
\pgfpathlineto{\pgfqpoint{0.791455in}{1.741107in}}%
\pgfpathlineto{\pgfqpoint{0.785207in}{1.522179in}}%
\pgfpathlineto{\pgfqpoint{0.785207in}{1.522179in}}%
\pgfusepath{stroke}%
\end{pgfscope}%
\begin{pgfscope}%
\pgfpathrectangle{\pgfqpoint{0.448634in}{0.402556in}}{\pgfqpoint{4.350661in}{2.489204in}} %
\pgfusepath{clip}%
\pgfsetrectcap%
\pgfsetroundjoin%
\pgfsetlinewidth{1.003750pt}%
\definecolor{currentstroke}{rgb}{0.498039,0.498039,0.498039}%
\pgfsetstrokecolor{currentstroke}%
\pgfsetdash{}{0pt}%
\pgfpathmoveto{\pgfqpoint{0.448634in}{2.896245in}}%
\pgfpathlineto{\pgfqpoint{0.448593in}{0.407043in}}%
\pgfpathlineto{\pgfqpoint{0.448593in}{0.407043in}}%
\pgfusepath{stroke}%
\end{pgfscope}%
\begin{pgfscope}%
\pgfpathrectangle{\pgfqpoint{0.448634in}{0.402556in}}{\pgfqpoint{4.350661in}{2.489204in}} %
\pgfusepath{clip}%
\pgfsetrectcap%
\pgfsetroundjoin%
\pgfsetlinewidth{1.003750pt}%
\definecolor{currentstroke}{rgb}{0.498039,0.498039,0.498039}%
\pgfsetstrokecolor{currentstroke}%
\pgfsetdash{}{0pt}%
\pgfpathmoveto{\pgfqpoint{3.429181in}{0.402624in}}%
\pgfpathlineto{\pgfqpoint{2.817917in}{0.403790in}}%
\pgfpathlineto{\pgfqpoint{2.774454in}{0.405815in}}%
\pgfpathlineto{\pgfqpoint{2.757248in}{0.408682in}}%
\pgfpathlineto{\pgfqpoint{2.749006in}{0.411825in}}%
\pgfpathlineto{\pgfqpoint{2.743465in}{0.415733in}}%
\pgfpathlineto{\pgfqpoint{2.739053in}{0.421204in}}%
\pgfpathlineto{\pgfqpoint{2.735137in}{0.430065in}}%
\pgfpathlineto{\pgfqpoint{2.731868in}{0.444506in}}%
\pgfpathlineto{\pgfqpoint{2.729413in}{0.466723in}}%
\pgfpathlineto{\pgfqpoint{2.727404in}{0.508973in}}%
\pgfpathlineto{\pgfqpoint{2.726247in}{0.586124in}}%
\pgfpathlineto{\pgfqpoint{2.726631in}{0.720539in}}%
\pgfpathlineto{\pgfqpoint{2.729429in}{0.902221in}}%
\pgfpathlineto{\pgfqpoint{2.734610in}{1.096287in}}%
\pgfpathlineto{\pgfqpoint{2.741633in}{1.272836in}}%
\pgfpathlineto{\pgfqpoint{2.750106in}{1.431848in}}%
\pgfpathlineto{\pgfqpoint{2.759931in}{1.575781in}}%
\pgfpathlineto{\pgfqpoint{2.772412in}{1.724438in}}%
\pgfpathlineto{\pgfqpoint{2.784744in}{1.840574in}}%
\pgfpathlineto{\pgfqpoint{2.800706in}{1.968694in}}%
\pgfpathlineto{\pgfqpoint{2.815746in}{2.066761in}}%
\pgfpathlineto{\pgfqpoint{2.832197in}{2.159455in}}%
\pgfpathlineto{\pgfqpoint{2.850752in}{2.251629in}}%
\pgfpathlineto{\pgfqpoint{2.897592in}{2.476737in}}%
\pgfpathlineto{\pgfqpoint{2.902464in}{2.513649in}}%
\pgfpathlineto{\pgfqpoint{2.904162in}{2.543444in}}%
\pgfpathlineto{\pgfqpoint{2.903432in}{2.565821in}}%
\pgfpathlineto{\pgfqpoint{2.900789in}{2.585491in}}%
\pgfpathlineto{\pgfqpoint{2.896584in}{2.602225in}}%
\pgfpathlineto{\pgfqpoint{2.890421in}{2.618142in}}%
\pgfpathlineto{\pgfqpoint{2.882321in}{2.632879in}}%
\pgfpathlineto{\pgfqpoint{2.872510in}{2.646188in}}%
\pgfpathlineto{\pgfqpoint{2.859642in}{2.659573in}}%
\pgfpathlineto{\pgfqpoint{2.845485in}{2.671139in}}%
\pgfpathlineto{\pgfqpoint{2.826570in}{2.683413in}}%
\pgfpathlineto{\pgfqpoint{2.804770in}{2.694682in}}%
\pgfpathlineto{\pgfqpoint{2.778180in}{2.705679in}}%
\pgfpathlineto{\pgfqpoint{2.746842in}{2.716061in}}%
\pgfpathlineto{\pgfqpoint{2.710822in}{2.725626in}}%
\pgfpathlineto{\pgfqpoint{2.668045in}{2.734683in}}%
\pgfpathlineto{\pgfqpoint{2.616383in}{2.743281in}}%
\pgfpathlineto{\pgfqpoint{2.555852in}{2.751014in}}%
\pgfpathlineto{\pgfqpoint{2.484314in}{2.757804in}}%
\pgfpathlineto{\pgfqpoint{2.401793in}{2.763294in}}%
\pgfpathlineto{\pgfqpoint{2.312665in}{2.766997in}}%
\pgfpathlineto{\pgfqpoint{2.234369in}{2.768714in}}%
\pgfpathlineto{\pgfqpoint{2.127780in}{2.769159in}}%
\pgfpathlineto{\pgfqpoint{2.019026in}{2.767391in}}%
\pgfpathlineto{\pgfqpoint{1.905969in}{2.763278in}}%
\pgfpathlineto{\pgfqpoint{1.795177in}{2.756759in}}%
\pgfpathlineto{\pgfqpoint{1.704074in}{2.748899in}}%
\pgfpathlineto{\pgfqpoint{1.621810in}{2.739641in}}%
\pgfpathlineto{\pgfqpoint{1.546254in}{2.728929in}}%
\pgfpathlineto{\pgfqpoint{1.449471in}{2.712478in}}%
\pgfpathlineto{\pgfqpoint{1.391867in}{2.699372in}}%
\pgfpathlineto{\pgfqpoint{1.338937in}{2.685096in}}%
\pgfpathlineto{\pgfqpoint{1.290725in}{2.669808in}}%
\pgfpathlineto{\pgfqpoint{1.249309in}{2.654575in}}%
\pgfpathlineto{\pgfqpoint{1.206483in}{2.636401in}}%
\pgfpathlineto{\pgfqpoint{1.170518in}{2.618695in}}%
\pgfpathlineto{\pgfqpoint{1.137332in}{2.600033in}}%
\pgfpathlineto{\pgfqpoint{1.105085in}{2.579331in}}%
\pgfpathlineto{\pgfqpoint{1.083232in}{2.563024in}}%
\pgfpathlineto{\pgfqpoint{1.054801in}{2.540060in}}%
\pgfpathlineto{\pgfqpoint{1.029341in}{2.516715in}}%
\pgfpathlineto{\pgfqpoint{1.005176in}{2.491635in}}%
\pgfpathlineto{\pgfqpoint{0.982419in}{2.464884in}}%
\pgfpathlineto{\pgfqpoint{0.961156in}{2.436570in}}%
\pgfpathlineto{\pgfqpoint{0.940174in}{2.404802in}}%
\pgfpathlineto{\pgfqpoint{0.921002in}{2.371569in}}%
\pgfpathlineto{\pgfqpoint{0.902472in}{2.334956in}}%
\pgfpathlineto{\pgfqpoint{0.885807in}{2.297186in}}%
\pgfpathlineto{\pgfqpoint{0.870069in}{2.256164in}}%
\pgfpathlineto{\pgfqpoint{0.855417in}{2.211946in}}%
\pgfpathlineto{\pgfqpoint{0.842022in}{2.164587in}}%
\pgfpathlineto{\pgfqpoint{0.821878in}{2.075436in}}%
\pgfpathlineto{\pgfqpoint{0.811528in}{2.016885in}}%
\pgfpathlineto{\pgfqpoint{0.797570in}{1.911059in}}%
\pgfpathlineto{\pgfqpoint{0.790252in}{1.834353in}}%
\pgfpathlineto{\pgfqpoint{0.784016in}{1.745029in}}%
\pgfpathlineto{\pgfqpoint{0.777222in}{1.610838in}}%
\pgfpathlineto{\pgfqpoint{0.771554in}{1.516476in}}%
\pgfpathlineto{\pgfqpoint{0.766437in}{1.467045in}}%
\pgfpathlineto{\pgfqpoint{0.760690in}{1.432832in}}%
\pgfpathlineto{\pgfqpoint{0.754521in}{1.408976in}}%
\pgfpathlineto{\pgfqpoint{0.748461in}{1.393004in}}%
\pgfpathlineto{\pgfqpoint{0.741461in}{1.380424in}}%
\pgfpathlineto{\pgfqpoint{0.733819in}{1.371612in}}%
\pgfpathlineto{\pgfqpoint{0.726322in}{1.366618in}}%
\pgfpathlineto{\pgfqpoint{0.720046in}{1.364633in}}%
\pgfpathlineto{\pgfqpoint{0.711377in}{1.364619in}}%
\pgfpathlineto{\pgfqpoint{0.702998in}{1.367218in}}%
\pgfpathlineto{\pgfqpoint{0.693399in}{1.373026in}}%
\pgfpathlineto{\pgfqpoint{0.683198in}{1.382314in}}%
\pgfpathlineto{\pgfqpoint{0.672798in}{1.395025in}}%
\pgfpathlineto{\pgfqpoint{0.661248in}{1.413099in}}%
\pgfpathlineto{\pgfqpoint{0.650220in}{1.434546in}}%
\pgfpathlineto{\pgfqpoint{0.638871in}{1.461438in}}%
\pgfpathlineto{\pgfqpoint{0.627602in}{1.493806in}}%
\pgfpathlineto{\pgfqpoint{0.616088in}{1.534013in}}%
\pgfpathlineto{\pgfqpoint{0.605373in}{1.579686in}}%
\pgfpathlineto{\pgfqpoint{0.595142in}{1.633178in}}%
\pgfpathlineto{\pgfqpoint{0.585702in}{1.694459in}}%
\pgfpathlineto{\pgfqpoint{0.577009in}{1.765954in}}%
\pgfpathlineto{\pgfqpoint{0.569419in}{1.847634in}}%
\pgfpathlineto{\pgfqpoint{0.563219in}{1.939458in}}%
\pgfpathlineto{\pgfqpoint{0.558783in}{2.038894in}}%
\pgfpathlineto{\pgfqpoint{0.556360in}{2.143401in}}%
\pgfpathlineto{\pgfqpoint{0.556145in}{2.247945in}}%
\pgfpathlineto{\pgfqpoint{0.558113in}{2.344995in}}%
\pgfpathlineto{\pgfqpoint{0.562048in}{2.431997in}}%
\pgfpathlineto{\pgfqpoint{0.567611in}{2.506397in}}%
\pgfpathlineto{\pgfqpoint{0.574423in}{2.568132in}}%
\pgfpathlineto{\pgfqpoint{0.582340in}{2.619608in}}%
\pgfpathlineto{\pgfqpoint{0.590842in}{2.660784in}}%
\pgfpathlineto{\pgfqpoint{0.599793in}{2.694086in}}%
\pgfpathlineto{\pgfqpoint{0.609332in}{2.721882in}}%
\pgfpathlineto{\pgfqpoint{0.620000in}{2.746380in}}%
\pgfpathlineto{\pgfqpoint{0.631615in}{2.767413in}}%
\pgfpathlineto{\pgfqpoint{0.643807in}{2.784927in}}%
\pgfpathlineto{\pgfqpoint{0.657674in}{2.800723in}}%
\pgfpathlineto{\pgfqpoint{0.673079in}{2.814528in}}%
\pgfpathlineto{\pgfqpoint{0.689750in}{2.826253in}}%
\pgfpathlineto{\pgfqpoint{0.707367in}{2.836008in}}%
\pgfpathlineto{\pgfqpoint{0.727710in}{2.844806in}}%
\pgfpathlineto{\pgfqpoint{0.752775in}{2.853124in}}%
\pgfpathlineto{\pgfqpoint{0.782541in}{2.860465in}}%
\pgfpathlineto{\pgfqpoint{0.819077in}{2.866974in}}%
\pgfpathlineto{\pgfqpoint{0.864489in}{2.872612in}}%
\pgfpathlineto{\pgfqpoint{0.923068in}{2.877455in}}%
\pgfpathlineto{\pgfqpoint{1.001298in}{2.881516in}}%
\pgfpathlineto{\pgfqpoint{1.112200in}{2.884842in}}%
\pgfpathlineto{\pgfqpoint{1.275335in}{2.887341in}}%
\pgfpathlineto{\pgfqpoint{1.553772in}{2.889246in}}%
\pgfpathlineto{\pgfqpoint{2.106304in}{2.890444in}}%
\pgfpathlineto{\pgfqpoint{3.335366in}{2.890569in}}%
\pgfpathlineto{\pgfqpoint{4.037996in}{2.888950in}}%
\pgfpathlineto{\pgfqpoint{4.285973in}{2.886419in}}%
\pgfpathlineto{\pgfqpoint{4.409932in}{2.883154in}}%
\pgfpathlineto{\pgfqpoint{4.485985in}{2.879137in}}%
\pgfpathlineto{\pgfqpoint{4.538021in}{2.874347in}}%
\pgfpathlineto{\pgfqpoint{4.574691in}{2.868910in}}%
\pgfpathlineto{\pgfqpoint{4.602449in}{2.862762in}}%
\pgfpathlineto{\pgfqpoint{4.625490in}{2.855409in}}%
\pgfpathlineto{\pgfqpoint{4.643721in}{2.847272in}}%
\pgfpathlineto{\pgfqpoint{4.659085in}{2.837948in}}%
\pgfpathlineto{\pgfqpoint{4.671533in}{2.827932in}}%
\pgfpathlineto{\pgfqpoint{4.682746in}{2.816161in}}%
\pgfpathlineto{\pgfqpoint{4.692498in}{2.802797in}}%
\pgfpathlineto{\pgfqpoint{4.701941in}{2.786084in}}%
\pgfpathlineto{\pgfqpoint{4.710583in}{2.765995in}}%
\pgfpathlineto{\pgfqpoint{4.718219in}{2.742695in}}%
\pgfpathlineto{\pgfqpoint{4.725284in}{2.713950in}}%
\pgfpathlineto{\pgfqpoint{4.732457in}{2.674984in}}%
\pgfpathlineto{\pgfqpoint{4.738699in}{2.628237in}}%
\pgfpathlineto{\pgfqpoint{4.744555in}{2.566372in}}%
\pgfpathlineto{\pgfqpoint{4.749529in}{2.489420in}}%
\pgfpathlineto{\pgfqpoint{4.753841in}{2.387484in}}%
\pgfpathlineto{\pgfqpoint{4.757022in}{2.258098in}}%
\pgfpathlineto{\pgfqpoint{4.758756in}{2.096314in}}%
\pgfpathlineto{\pgfqpoint{4.758503in}{1.912114in}}%
\pgfpathlineto{\pgfqpoint{4.756053in}{1.727936in}}%
\pgfpathlineto{\pgfqpoint{4.751618in}{1.558750in}}%
\pgfpathlineto{\pgfqpoint{4.745483in}{1.414550in}}%
\pgfpathlineto{\pgfqpoint{4.738386in}{1.302833in}}%
\pgfpathlineto{\pgfqpoint{4.730561in}{1.211174in}}%
\pgfpathlineto{\pgfqpoint{4.720259in}{1.119859in}}%
\pgfpathlineto{\pgfqpoint{4.710354in}{1.056145in}}%
\pgfpathlineto{\pgfqpoint{4.699876in}{1.002716in}}%
\pgfpathlineto{\pgfqpoint{4.690113in}{0.961906in}}%
\pgfpathlineto{\pgfqpoint{4.677721in}{0.919411in}}%
\pgfpathlineto{\pgfqpoint{4.664599in}{0.882530in}}%
\pgfpathlineto{\pgfqpoint{4.651210in}{0.851238in}}%
\pgfpathlineto{\pgfqpoint{4.636989in}{0.823276in}}%
\pgfpathlineto{\pgfqpoint{4.620957in}{0.796630in}}%
\pgfpathlineto{\pgfqpoint{4.604449in}{0.773502in}}%
\pgfpathlineto{\pgfqpoint{4.586354in}{0.751984in}}%
\pgfpathlineto{\pgfqpoint{4.566792in}{0.732218in}}%
\pgfpathlineto{\pgfqpoint{4.545928in}{0.714278in}}%
\pgfpathlineto{\pgfqpoint{4.522086in}{0.696891in}}%
\pgfpathlineto{\pgfqpoint{4.495247in}{0.680439in}}%
\pgfpathlineto{\pgfqpoint{4.469462in}{0.667166in}}%
\pgfpathlineto{\pgfqpoint{4.439001in}{0.653796in}}%
\pgfpathlineto{\pgfqpoint{4.405861in}{0.641637in}}%
\pgfpathlineto{\pgfqpoint{4.368014in}{0.630170in}}%
\pgfpathlineto{\pgfqpoint{4.325482in}{0.619709in}}%
\pgfpathlineto{\pgfqpoint{4.278312in}{0.610487in}}%
\pgfpathlineto{\pgfqpoint{4.226555in}{0.602673in}}%
\pgfpathlineto{\pgfqpoint{4.170259in}{0.596468in}}%
\pgfpathlineto{\pgfqpoint{4.107313in}{0.591731in}}%
\pgfpathlineto{\pgfqpoint{4.033395in}{0.588906in}}%
\pgfpathlineto{\pgfqpoint{3.963788in}{0.588448in}}%
\pgfpathlineto{\pgfqpoint{3.892021in}{0.590211in}}%
\pgfpathlineto{\pgfqpoint{3.818151in}{0.594330in}}%
\pgfpathlineto{\pgfqpoint{3.750929in}{0.600430in}}%
\pgfpathlineto{\pgfqpoint{3.701226in}{0.606824in}}%
\pgfpathlineto{\pgfqpoint{3.686145in}{0.609198in}}%
\pgfpathlineto{\pgfqpoint{3.628026in}{0.618869in}}%
\pgfpathlineto{\pgfqpoint{3.587436in}{0.627634in}}%
\pgfpathlineto{\pgfqpoint{3.568265in}{0.632160in}}%
\pgfpathlineto{\pgfqpoint{3.542748in}{0.638442in}}%
\pgfpathlineto{\pgfqpoint{3.496363in}{0.651905in}}%
\pgfpathlineto{\pgfqpoint{3.452633in}{0.666995in}}%
\pgfpathlineto{\pgfqpoint{3.409614in}{0.684540in}}%
\pgfpathlineto{\pgfqpoint{3.375643in}{0.701248in}}%
\pgfpathlineto{\pgfqpoint{3.348337in}{0.716669in}}%
\pgfpathlineto{\pgfqpoint{3.319922in}{0.735011in}}%
\pgfpathlineto{\pgfqpoint{3.292603in}{0.755417in}}%
\pgfpathlineto{\pgfqpoint{3.268254in}{0.776338in}}%
\pgfpathlineto{\pgfqpoint{3.245197in}{0.799094in}}%
\pgfpathlineto{\pgfqpoint{3.223594in}{0.823648in}}%
\pgfpathlineto{\pgfqpoint{3.203589in}{0.849913in}}%
\pgfpathlineto{\pgfqpoint{3.185238in}{0.877717in}}%
\pgfpathlineto{\pgfqpoint{3.168757in}{0.907013in}}%
\pgfpathlineto{\pgfqpoint{3.153924in}{0.937442in}}%
\pgfpathlineto{\pgfqpoint{3.139803in}{0.971094in}}%
\pgfpathlineto{\pgfqpoint{3.125255in}{1.012684in}}%
\pgfpathlineto{\pgfqpoint{3.113766in}{1.052900in}}%
\pgfpathlineto{\pgfqpoint{3.104308in}{1.093803in}}%
\pgfpathlineto{\pgfqpoint{3.095883in}{1.140099in}}%
\pgfpathlineto{\pgfqpoint{3.089287in}{1.189302in}}%
\pgfpathlineto{\pgfqpoint{3.084614in}{1.241296in}}%
\pgfpathlineto{\pgfqpoint{3.082116in}{1.293486in}}%
\pgfpathlineto{\pgfqpoint{3.081623in}{1.348241in}}%
\pgfpathlineto{\pgfqpoint{3.083351in}{1.405454in}}%
\pgfpathlineto{\pgfqpoint{3.087052in}{1.460048in}}%
\pgfpathlineto{\pgfqpoint{3.093317in}{1.519352in}}%
\pgfpathlineto{\pgfqpoint{3.101620in}{1.575805in}}%
\pgfpathlineto{\pgfqpoint{3.112163in}{1.631766in}}%
\pgfpathlineto{\pgfqpoint{3.124975in}{1.687102in}}%
\pgfpathlineto{\pgfqpoint{3.139513in}{1.739271in}}%
\pgfpathlineto{\pgfqpoint{3.155528in}{1.788221in}}%
\pgfpathlineto{\pgfqpoint{3.172784in}{1.833916in}}%
\pgfpathlineto{\pgfqpoint{3.192078in}{1.878530in}}%
\pgfpathlineto{\pgfqpoint{3.215662in}{1.926173in}}%
\pgfpathlineto{\pgfqpoint{3.237990in}{1.965966in}}%
\pgfpathlineto{\pgfqpoint{3.262091in}{2.004382in}}%
\pgfpathlineto{\pgfqpoint{3.287852in}{2.041360in}}%
\pgfpathlineto{\pgfqpoint{3.316568in}{2.078754in}}%
\pgfpathlineto{\pgfqpoint{3.351198in}{2.120071in}}%
\pgfpathlineto{\pgfqpoint{3.416495in}{2.196666in}}%
\pgfpathlineto{\pgfqpoint{3.426763in}{2.212721in}}%
\pgfpathlineto{\pgfqpoint{3.431431in}{2.223931in}}%
\pgfpathlineto{\pgfqpoint{3.432538in}{2.231261in}}%
\pgfpathlineto{\pgfqpoint{3.431062in}{2.238456in}}%
\pgfpathlineto{\pgfqpoint{3.426743in}{2.243962in}}%
\pgfpathlineto{\pgfqpoint{3.420965in}{2.247384in}}%
\pgfpathlineto{\pgfqpoint{3.412532in}{2.249773in}}%
\pgfpathlineto{\pgfqpoint{3.399523in}{2.250789in}}%
\pgfpathlineto{\pgfqpoint{3.384332in}{2.249716in}}%
\pgfpathlineto{\pgfqpoint{3.365018in}{2.246104in}}%
\pgfpathlineto{\pgfqpoint{3.341842in}{2.239323in}}%
\pgfpathlineto{\pgfqpoint{3.317151in}{2.229649in}}%
\pgfpathlineto{\pgfqpoint{3.291148in}{2.216948in}}%
\pgfpathlineto{\pgfqpoint{3.265972in}{2.202224in}}%
\pgfpathlineto{\pgfqpoint{3.239847in}{2.184327in}}%
\pgfpathlineto{\pgfqpoint{3.214814in}{2.164490in}}%
\pgfpathlineto{\pgfqpoint{3.190936in}{2.142869in}}%
\pgfpathlineto{\pgfqpoint{3.166687in}{2.117894in}}%
\pgfpathlineto{\pgfqpoint{3.143862in}{2.091220in}}%
\pgfpathlineto{\pgfqpoint{3.121100in}{2.061098in}}%
\pgfpathlineto{\pgfqpoint{3.099970in}{2.029458in}}%
\pgfpathlineto{\pgfqpoint{3.079265in}{1.994403in}}%
\pgfpathlineto{\pgfqpoint{3.059230in}{1.955913in}}%
\pgfpathlineto{\pgfqpoint{3.040070in}{1.914014in}}%
\pgfpathlineto{\pgfqpoint{3.022822in}{1.871040in}}%
\pgfpathlineto{\pgfqpoint{3.005804in}{1.822534in}}%
\pgfpathlineto{\pgfqpoint{2.990082in}{1.770816in}}%
\pgfpathlineto{\pgfqpoint{2.975725in}{1.715977in}}%
\pgfpathlineto{\pgfqpoint{2.962300in}{1.655677in}}%
\pgfpathlineto{\pgfqpoint{2.950510in}{1.592383in}}%
\pgfpathlineto{\pgfqpoint{2.940394in}{1.526183in}}%
\pgfpathlineto{\pgfqpoint{2.931750in}{1.454681in}}%
\pgfpathlineto{\pgfqpoint{2.925080in}{1.380399in}}%
\pgfpathlineto{\pgfqpoint{2.920639in}{1.305900in}}%
\pgfpathlineto{\pgfqpoint{2.918431in}{1.231271in}}%
\pgfpathlineto{\pgfqpoint{2.918530in}{1.159088in}}%
\pgfpathlineto{\pgfqpoint{2.920771in}{1.091932in}}%
\pgfpathlineto{\pgfqpoint{2.925162in}{1.027413in}}%
\pgfpathlineto{\pgfqpoint{2.931177in}{0.970581in}}%
\pgfpathlineto{\pgfqpoint{2.938744in}{0.919035in}}%
\pgfpathlineto{\pgfqpoint{2.947631in}{0.872852in}}%
\pgfpathlineto{\pgfqpoint{2.958188in}{0.829712in}}%
\pgfpathlineto{\pgfqpoint{2.969637in}{0.792109in}}%
\pgfpathlineto{\pgfqpoint{2.982421in}{0.757764in}}%
\pgfpathlineto{\pgfqpoint{2.996374in}{0.726797in}}%
\pgfpathlineto{\pgfqpoint{3.011239in}{0.699279in}}%
\pgfpathlineto{\pgfqpoint{3.026671in}{0.675197in}}%
\pgfpathlineto{\pgfqpoint{3.043753in}{0.652621in}}%
\pgfpathlineto{\pgfqpoint{3.062415in}{0.631747in}}%
\pgfpathlineto{\pgfqpoint{3.082518in}{0.612706in}}%
\pgfpathlineto{\pgfqpoint{3.103874in}{0.595542in}}%
\pgfpathlineto{\pgfqpoint{3.128180in}{0.579016in}}%
\pgfpathlineto{\pgfqpoint{3.153448in}{0.564499in}}%
\pgfpathlineto{\pgfqpoint{3.181482in}{0.550897in}}%
\pgfpathlineto{\pgfqpoint{3.214282in}{0.537593in}}%
\pgfpathlineto{\pgfqpoint{3.249758in}{0.525658in}}%
\pgfpathlineto{\pgfqpoint{3.289922in}{0.514517in}}%
\pgfpathlineto{\pgfqpoint{3.334732in}{0.504370in}}%
\pgfpathlineto{\pgfqpoint{3.386284in}{0.494947in}}%
\pgfpathlineto{\pgfqpoint{3.446710in}{0.486206in}}%
\pgfpathlineto{\pgfqpoint{3.518155in}{0.478232in}}%
\pgfpathlineto{\pgfqpoint{3.600597in}{0.471361in}}%
\pgfpathlineto{\pgfqpoint{3.696181in}{0.465666in}}%
\pgfpathlineto{\pgfqpoint{3.807056in}{0.461323in}}%
\pgfpathlineto{\pgfqpoint{3.931028in}{0.458706in}}%
\pgfpathlineto{\pgfqpoint{4.061546in}{0.458169in}}%
\pgfpathlineto{\pgfqpoint{4.185529in}{0.459841in}}%
\pgfpathlineto{\pgfqpoint{4.292073in}{0.463409in}}%
\pgfpathlineto{\pgfqpoint{4.378975in}{0.468410in}}%
\pgfpathlineto{\pgfqpoint{4.448381in}{0.474472in}}%
\pgfpathlineto{\pgfqpoint{4.504603in}{0.481490in}}%
\pgfpathlineto{\pgfqpoint{4.549775in}{0.489248in}}%
\pgfpathlineto{\pgfqpoint{4.586028in}{0.497578in}}%
\pgfpathlineto{\pgfqpoint{4.615483in}{0.506412in}}%
\pgfpathlineto{\pgfqpoint{4.640214in}{0.515949in}}%
\pgfpathlineto{\pgfqpoint{4.662170in}{0.526811in}}%
\pgfpathlineto{\pgfqpoint{4.679327in}{0.537585in}}%
\pgfpathlineto{\pgfqpoint{4.695408in}{0.550341in}}%
\pgfpathlineto{\pgfqpoint{4.708494in}{0.563453in}}%
\pgfpathlineto{\pgfqpoint{4.720195in}{0.578179in}}%
\pgfpathlineto{\pgfqpoint{4.730379in}{0.594315in}}%
\pgfpathlineto{\pgfqpoint{4.740035in}{0.613790in}}%
\pgfpathlineto{\pgfqpoint{4.748755in}{0.636584in}}%
\pgfpathlineto{\pgfqpoint{4.756335in}{0.662547in}}%
\pgfpathlineto{\pgfqpoint{4.763229in}{0.693924in}}%
\pgfpathlineto{\pgfqpoint{4.769533in}{0.733086in}}%
\pgfpathlineto{\pgfqpoint{4.775177in}{0.782445in}}%
\pgfpathlineto{\pgfqpoint{4.780207in}{0.846905in}}%
\pgfpathlineto{\pgfqpoint{4.784603in}{0.933879in}}%
\pgfpathlineto{\pgfqpoint{4.788273in}{1.053285in}}%
\pgfpathlineto{\pgfqpoint{4.791272in}{1.227494in}}%
\pgfpathlineto{\pgfqpoint{4.793571in}{1.501293in}}%
\pgfpathlineto{\pgfqpoint{4.794883in}{1.946858in}}%
\pgfpathlineto{\pgfqpoint{4.794284in}{2.499460in}}%
\pgfpathlineto{\pgfqpoint{4.792163in}{2.725962in}}%
\pgfpathlineto{\pgfqpoint{4.789427in}{2.810531in}}%
\pgfpathlineto{\pgfqpoint{4.786464in}{2.845205in}}%
\pgfpathlineto{\pgfqpoint{4.783332in}{2.862243in}}%
\pgfpathlineto{\pgfqpoint{4.779062in}{2.873649in}}%
\pgfpathlineto{\pgfqpoint{4.774896in}{2.879364in}}%
\pgfpathlineto{\pgfqpoint{4.769472in}{2.883473in}}%
\pgfpathlineto{\pgfqpoint{4.761241in}{2.886642in}}%
\pgfpathlineto{\pgfqpoint{4.748345in}{2.888875in}}%
\pgfpathlineto{\pgfqpoint{4.724457in}{2.890390in}}%
\pgfpathlineto{\pgfqpoint{4.665730in}{2.891305in}}%
\pgfpathlineto{\pgfqpoint{4.428620in}{2.891689in}}%
\pgfpathlineto{\pgfqpoint{1.072085in}{2.891701in}}%
\pgfpathlineto{\pgfqpoint{0.600041in}{2.890510in}}%
\pgfpathlineto{\pgfqpoint{0.543515in}{2.888485in}}%
\pgfpathlineto{\pgfqpoint{0.517545in}{2.885563in}}%
\pgfpathlineto{\pgfqpoint{0.502686in}{2.881825in}}%
\pgfpathlineto{\pgfqpoint{0.492659in}{2.877062in}}%
\pgfpathlineto{\pgfqpoint{0.485579in}{2.871311in}}%
\pgfpathlineto{\pgfqpoint{0.479882in}{2.863815in}}%
\pgfpathlineto{\pgfqpoint{0.474764in}{2.852859in}}%
\pgfpathlineto{\pgfqpoint{0.470700in}{2.838679in}}%
\pgfpathlineto{\pgfqpoint{0.466902in}{2.816712in}}%
\pgfpathlineto{\pgfqpoint{0.463483in}{2.782090in}}%
\pgfpathlineto{\pgfqpoint{0.460556in}{2.727433in}}%
\pgfpathlineto{\pgfqpoint{0.458022in}{2.632890in}}%
\pgfpathlineto{\pgfqpoint{0.456555in}{2.525868in}}%
\pgfpathlineto{\pgfqpoint{0.456555in}{2.525868in}}%
\pgfusepath{stroke}%
\end{pgfscope}%
\begin{pgfscope}%
\pgfpathrectangle{\pgfqpoint{0.448634in}{0.402556in}}{\pgfqpoint{4.350661in}{2.489204in}} %
\pgfusepath{clip}%
\pgfsetrectcap%
\pgfsetroundjoin%
\pgfsetlinewidth{1.003750pt}%
\definecolor{currentstroke}{rgb}{0.498039,0.498039,0.498039}%
\pgfsetstrokecolor{currentstroke}%
\pgfsetdash{}{0pt}%
\pgfpathmoveto{\pgfqpoint{0.456431in}{1.369072in}}%
\pgfpathlineto{\pgfqpoint{0.459628in}{1.117690in}}%
\pgfpathlineto{\pgfqpoint{0.463640in}{0.963430in}}%
\pgfpathlineto{\pgfqpoint{0.468429in}{0.859030in}}%
\pgfpathlineto{\pgfqpoint{0.473943in}{0.784626in}}%
\pgfpathlineto{\pgfqpoint{0.480025in}{0.730313in}}%
\pgfpathlineto{\pgfqpoint{0.486695in}{0.688697in}}%
\pgfpathlineto{\pgfqpoint{0.494170in}{0.654922in}}%
\pgfpathlineto{\pgfqpoint{0.502633in}{0.626676in}}%
\pgfpathlineto{\pgfqpoint{0.511604in}{0.604010in}}%
\pgfpathlineto{\pgfqpoint{0.521493in}{0.584688in}}%
\pgfpathlineto{\pgfqpoint{0.533277in}{0.566815in}}%
\pgfpathlineto{\pgfqpoint{0.545329in}{0.552466in}}%
\pgfpathlineto{\pgfqpoint{0.558707in}{0.539745in}}%
\pgfpathlineto{\pgfqpoint{0.575033in}{0.527401in}}%
\pgfpathlineto{\pgfqpoint{0.594330in}{0.515932in}}%
\pgfpathlineto{\pgfqpoint{0.616488in}{0.505616in}}%
\pgfpathlineto{\pgfqpoint{0.641349in}{0.496529in}}%
\pgfpathlineto{\pgfqpoint{0.670889in}{0.488074in}}%
\pgfpathlineto{\pgfqpoint{0.707195in}{0.480051in}}%
\pgfpathlineto{\pgfqpoint{0.752394in}{0.472495in}}%
\pgfpathlineto{\pgfqpoint{0.806458in}{0.465785in}}%
\pgfpathlineto{\pgfqpoint{0.873683in}{0.459726in}}%
\pgfpathlineto{\pgfqpoint{0.958391in}{0.454398in}}%
\pgfpathlineto{\pgfqpoint{1.064912in}{0.450016in}}%
\pgfpathlineto{\pgfqpoint{1.197578in}{0.446883in}}%
\pgfpathlineto{\pgfqpoint{1.354196in}{0.445475in}}%
\pgfpathlineto{\pgfqpoint{1.521694in}{0.446179in}}%
\pgfpathlineto{\pgfqpoint{1.682648in}{0.449038in}}%
\pgfpathlineto{\pgfqpoint{1.821809in}{0.453676in}}%
\pgfpathlineto{\pgfqpoint{1.936981in}{0.459668in}}%
\pgfpathlineto{\pgfqpoint{2.032489in}{0.466822in}}%
\pgfpathlineto{\pgfqpoint{2.110485in}{0.474834in}}%
\pgfpathlineto{\pgfqpoint{2.175286in}{0.483655in}}%
\pgfpathlineto{\pgfqpoint{2.229036in}{0.493096in}}%
\pgfpathlineto{\pgfqpoint{2.276009in}{0.503555in}}%
\pgfpathlineto{\pgfqpoint{2.316162in}{0.514746in}}%
\pgfpathlineto{\pgfqpoint{2.351564in}{0.526958in}}%
\pgfpathlineto{\pgfqpoint{2.382168in}{0.539891in}}%
\pgfpathlineto{\pgfqpoint{2.407989in}{0.553072in}}%
\pgfpathlineto{\pgfqpoint{2.431023in}{0.567112in}}%
\pgfpathlineto{\pgfqpoint{2.453041in}{0.583142in}}%
\pgfpathlineto{\pgfqpoint{2.472109in}{0.599672in}}%
\pgfpathlineto{\pgfqpoint{2.489911in}{0.617957in}}%
\pgfpathlineto{\pgfqpoint{2.506295in}{0.637901in}}%
\pgfpathlineto{\pgfqpoint{2.521181in}{0.659330in}}%
\pgfpathlineto{\pgfqpoint{2.535713in}{0.684134in}}%
\pgfpathlineto{\pgfqpoint{2.549552in}{0.712345in}}%
\pgfpathlineto{\pgfqpoint{2.562469in}{0.743896in}}%
\pgfpathlineto{\pgfqpoint{2.574345in}{0.778667in}}%
\pgfpathlineto{\pgfqpoint{2.585780in}{0.818904in}}%
\pgfpathlineto{\pgfqpoint{2.597042in}{0.866986in}}%
\pgfpathlineto{\pgfqpoint{2.607757in}{0.922905in}}%
\pgfpathlineto{\pgfqpoint{2.618088in}{0.989062in}}%
\pgfpathlineto{\pgfqpoint{2.628094in}{1.067887in}}%
\pgfpathlineto{\pgfqpoint{2.638281in}{1.166767in}}%
\pgfpathlineto{\pgfqpoint{2.648700in}{1.290654in}}%
\pgfpathlineto{\pgfqpoint{2.660495in}{1.459379in}}%
\pgfpathlineto{\pgfqpoint{2.675263in}{1.705229in}}%
\pgfpathlineto{\pgfqpoint{2.687904in}{1.948741in}}%
\pgfpathlineto{\pgfqpoint{2.692732in}{2.080550in}}%
\pgfpathlineto{\pgfqpoint{2.693862in}{2.167659in}}%
\pgfpathlineto{\pgfqpoint{2.692652in}{2.232359in}}%
\pgfpathlineto{\pgfqpoint{2.689584in}{2.284508in}}%
\pgfpathlineto{\pgfqpoint{2.684741in}{2.328963in}}%
\pgfpathlineto{\pgfqpoint{2.678540in}{2.365612in}}%
\pgfpathlineto{\pgfqpoint{2.671100in}{2.396823in}}%
\pgfpathlineto{\pgfqpoint{2.662157in}{2.424876in}}%
\pgfpathlineto{\pgfqpoint{2.651953in}{2.449631in}}%
\pgfpathlineto{\pgfqpoint{2.640884in}{2.471050in}}%
\pgfpathlineto{\pgfqpoint{2.628093in}{2.491172in}}%
\pgfpathlineto{\pgfqpoint{2.613666in}{2.509790in}}%
\pgfpathlineto{\pgfqpoint{2.597778in}{2.526779in}}%
\pgfpathlineto{\pgfqpoint{2.578877in}{2.543555in}}%
\pgfpathlineto{\pgfqpoint{2.558783in}{2.558409in}}%
\pgfpathlineto{\pgfqpoint{2.535823in}{2.572605in}}%
\pgfpathlineto{\pgfqpoint{2.510047in}{2.585901in}}%
\pgfpathlineto{\pgfqpoint{2.481539in}{2.598147in}}%
\pgfpathlineto{\pgfqpoint{2.448300in}{2.609942in}}%
\pgfpathlineto{\pgfqpoint{2.410340in}{2.620910in}}%
\pgfpathlineto{\pgfqpoint{2.367700in}{2.630777in}}%
\pgfpathlineto{\pgfqpoint{2.320437in}{2.639354in}}%
\pgfpathlineto{\pgfqpoint{2.268606in}{2.646499in}}%
\pgfpathlineto{\pgfqpoint{2.210091in}{2.652261in}}%
\pgfpathlineto{\pgfqpoint{2.147103in}{2.656193in}}%
\pgfpathlineto{\pgfqpoint{2.079691in}{2.658147in}}%
\pgfpathlineto{\pgfqpoint{2.010083in}{2.657956in}}%
\pgfpathlineto{\pgfqpoint{1.938331in}{2.655530in}}%
\pgfpathlineto{\pgfqpoint{1.866664in}{2.650842in}}%
\pgfpathlineto{\pgfqpoint{1.797310in}{2.644039in}}%
\pgfpathlineto{\pgfqpoint{1.732484in}{2.635474in}}%
\pgfpathlineto{\pgfqpoint{1.672222in}{2.625357in}}%
\pgfpathlineto{\pgfqpoint{1.614426in}{2.613411in}}%
\pgfpathlineto{\pgfqpoint{1.561292in}{2.600164in}}%
\pgfpathlineto{\pgfqpoint{1.512849in}{2.585863in}}%
\pgfpathlineto{\pgfqpoint{1.467041in}{2.570024in}}%
\pgfpathlineto{\pgfqpoint{1.425987in}{2.553559in}}%
\pgfpathlineto{\pgfqpoint{1.387657in}{2.535878in}}%
\pgfpathlineto{\pgfqpoint{1.352107in}{2.517107in}}%
\pgfpathlineto{\pgfqpoint{1.319380in}{2.497414in}}%
\pgfpathlineto{\pgfqpoint{1.287658in}{2.475675in}}%
\pgfpathlineto{\pgfqpoint{1.258901in}{2.453249in}}%
\pgfpathlineto{\pgfqpoint{1.231396in}{2.428856in}}%
\pgfpathlineto{\pgfqpoint{1.206910in}{2.404185in}}%
\pgfpathlineto{\pgfqpoint{1.183834in}{2.377796in}}%
\pgfpathlineto{\pgfqpoint{1.162297in}{2.349754in}}%
\pgfpathlineto{\pgfqpoint{1.142408in}{2.320162in}}%
\pgfpathlineto{\pgfqpoint{1.124243in}{2.289153in}}%
\pgfpathlineto{\pgfqpoint{1.107844in}{2.256880in}}%
\pgfpathlineto{\pgfqpoint{1.092313in}{2.221246in}}%
\pgfpathlineto{\pgfqpoint{1.078787in}{2.184556in}}%
\pgfpathlineto{\pgfqpoint{1.067223in}{2.146997in}}%
\pgfpathlineto{\pgfqpoint{1.057021in}{2.106330in}}%
\pgfpathlineto{\pgfqpoint{1.048890in}{2.065055in}}%
\pgfpathlineto{\pgfqpoint{1.042450in}{2.020865in}}%
\pgfpathlineto{\pgfqpoint{1.038163in}{1.976334in}}%
\pgfpathlineto{\pgfqpoint{1.035902in}{1.929116in}}%
\pgfpathlineto{\pgfqpoint{1.035910in}{1.881827in}}%
\pgfpathlineto{\pgfqpoint{1.038162in}{1.834608in}}%
\pgfpathlineto{\pgfqpoint{1.042652in}{1.787599in}}%
\pgfpathlineto{\pgfqpoint{1.049402in}{1.740946in}}%
\pgfpathlineto{\pgfqpoint{1.057917in}{1.697218in}}%
\pgfpathlineto{\pgfqpoint{1.068544in}{1.654101in}}%
\pgfpathlineto{\pgfqpoint{1.081337in}{1.611761in}}%
\pgfpathlineto{\pgfqpoint{1.095457in}{1.572657in}}%
\pgfpathlineto{\pgfqpoint{1.111594in}{1.534589in}}%
\pgfpathlineto{\pgfqpoint{1.128647in}{1.499876in}}%
\pgfpathlineto{\pgfqpoint{1.147510in}{1.466413in}}%
\pgfpathlineto{\pgfqpoint{1.168160in}{1.434361in}}%
\pgfpathlineto{\pgfqpoint{1.190550in}{1.403877in}}%
\pgfpathlineto{\pgfqpoint{1.214604in}{1.375102in}}%
\pgfpathlineto{\pgfqpoint{1.238576in}{1.349778in}}%
\pgfpathlineto{\pgfqpoint{1.265545in}{1.324611in}}%
\pgfpathlineto{\pgfqpoint{1.293821in}{1.301399in}}%
\pgfpathlineto{\pgfqpoint{1.323258in}{1.280160in}}%
\pgfpathlineto{\pgfqpoint{1.353707in}{1.260878in}}%
\pgfpathlineto{\pgfqpoint{1.387006in}{1.242484in}}%
\pgfpathlineto{\pgfqpoint{1.421123in}{1.226168in}}%
\pgfpathlineto{\pgfqpoint{1.457975in}{1.211038in}}%
\pgfpathlineto{\pgfqpoint{1.497519in}{1.197300in}}%
\pgfpathlineto{\pgfqpoint{1.539695in}{1.185100in}}%
\pgfpathlineto{\pgfqpoint{1.584426in}{1.174503in}}%
\pgfpathlineto{\pgfqpoint{1.635920in}{1.164678in}}%
\pgfpathlineto{\pgfqpoint{1.707057in}{1.153671in}}%
\pgfpathlineto{\pgfqpoint{1.767340in}{1.143747in}}%
\pgfpathlineto{\pgfqpoint{1.794998in}{1.137044in}}%
\pgfpathlineto{\pgfqpoint{1.813643in}{1.130253in}}%
\pgfpathlineto{\pgfqpoint{1.825393in}{1.123787in}}%
\pgfpathlineto{\pgfqpoint{1.834055in}{1.116311in}}%
\pgfpathlineto{\pgfqpoint{1.839207in}{1.108343in}}%
\pgfpathlineto{\pgfqpoint{1.841142in}{1.101245in}}%
\pgfpathlineto{\pgfqpoint{1.841017in}{1.093809in}}%
\pgfpathlineto{\pgfqpoint{1.838182in}{1.084437in}}%
\pgfpathlineto{\pgfqpoint{1.833412in}{1.076129in}}%
\pgfpathlineto{\pgfqpoint{1.825926in}{1.067118in}}%
\pgfpathlineto{\pgfqpoint{1.813875in}{1.056492in}}%
\pgfpathlineto{\pgfqpoint{1.798853in}{1.046460in}}%
\pgfpathlineto{\pgfqpoint{1.781031in}{1.037205in}}%
\pgfpathlineto{\pgfqpoint{1.758448in}{1.028177in}}%
\pgfpathlineto{\pgfqpoint{1.733196in}{1.020637in}}%
\pgfpathlineto{\pgfqpoint{1.705399in}{1.014723in}}%
\pgfpathlineto{\pgfqpoint{1.675164in}{1.010589in}}%
\pgfpathlineto{\pgfqpoint{1.642595in}{1.008402in}}%
\pgfpathlineto{\pgfqpoint{1.607794in}{1.008344in}}%
\pgfpathlineto{\pgfqpoint{1.570872in}{1.010618in}}%
\pgfpathlineto{\pgfqpoint{1.534105in}{1.015122in}}%
\pgfpathlineto{\pgfqpoint{1.495444in}{1.022189in}}%
\pgfpathlineto{\pgfqpoint{1.457153in}{1.031534in}}%
\pgfpathlineto{\pgfqpoint{1.419332in}{1.043115in}}%
\pgfpathlineto{\pgfqpoint{1.382087in}{1.056922in}}%
\pgfpathlineto{\pgfqpoint{1.347543in}{1.072017in}}%
\pgfpathlineto{\pgfqpoint{1.313727in}{1.089133in}}%
\pgfpathlineto{\pgfqpoint{1.280761in}{1.108296in}}%
\pgfpathlineto{\pgfqpoint{1.248778in}{1.129529in}}%
\pgfpathlineto{\pgfqpoint{1.219702in}{1.151410in}}%
\pgfpathlineto{\pgfqpoint{1.191742in}{1.175121in}}%
\pgfpathlineto{\pgfqpoint{1.165018in}{1.200628in}}%
\pgfpathlineto{\pgfqpoint{1.139639in}{1.227874in}}%
\pgfpathlineto{\pgfqpoint{1.115699in}{1.256776in}}%
\pgfpathlineto{\pgfqpoint{1.093273in}{1.287227in}}%
\pgfpathlineto{\pgfqpoint{1.071162in}{1.321139in}}%
\pgfpathlineto{\pgfqpoint{1.050853in}{1.356495in}}%
\pgfpathlineto{\pgfqpoint{1.032349in}{1.393127in}}%
\pgfpathlineto{\pgfqpoint{1.014700in}{1.433116in}}%
\pgfpathlineto{\pgfqpoint{0.999004in}{1.474159in}}%
\pgfpathlineto{\pgfqpoint{0.984484in}{1.518433in}}%
\pgfpathlineto{\pgfqpoint{0.971984in}{1.563508in}}%
\pgfpathlineto{\pgfqpoint{0.960915in}{1.611648in}}%
\pgfpathlineto{\pgfqpoint{0.951499in}{1.662794in}}%
\pgfpathlineto{\pgfqpoint{0.944255in}{1.714401in}}%
\pgfpathlineto{\pgfqpoint{0.938921in}{1.768816in}}%
\pgfpathlineto{\pgfqpoint{0.935845in}{1.823461in}}%
\pgfpathlineto{\pgfqpoint{0.935014in}{1.878210in}}%
\pgfpathlineto{\pgfqpoint{0.936451in}{1.932942in}}%
\pgfpathlineto{\pgfqpoint{0.939997in}{1.985053in}}%
\pgfpathlineto{\pgfqpoint{0.945755in}{2.036904in}}%
\pgfpathlineto{\pgfqpoint{0.953410in}{2.085906in}}%
\pgfpathlineto{\pgfqpoint{0.962766in}{2.131968in}}%
\pgfpathlineto{\pgfqpoint{0.974287in}{2.177381in}}%
\pgfpathlineto{\pgfqpoint{0.987331in}{2.219621in}}%
\pgfpathlineto{\pgfqpoint{1.001662in}{2.258624in}}%
\pgfpathlineto{\pgfqpoint{1.018041in}{2.296556in}}%
\pgfpathlineto{\pgfqpoint{1.035387in}{2.331077in}}%
\pgfpathlineto{\pgfqpoint{1.054631in}{2.364255in}}%
\pgfpathlineto{\pgfqpoint{1.074383in}{2.393966in}}%
\pgfpathlineto{\pgfqpoint{1.095745in}{2.422183in}}%
\pgfpathlineto{\pgfqpoint{1.118633in}{2.448786in}}%
\pgfpathlineto{\pgfqpoint{1.142935in}{2.473693in}}%
\pgfpathlineto{\pgfqpoint{1.168517in}{2.496862in}}%
\pgfpathlineto{\pgfqpoint{1.197050in}{2.519660in}}%
\pgfpathlineto{\pgfqpoint{1.226690in}{2.540526in}}%
\pgfpathlineto{\pgfqpoint{1.259205in}{2.560675in}}%
\pgfpathlineto{\pgfqpoint{1.294574in}{2.579885in}}%
\pgfpathlineto{\pgfqpoint{1.332754in}{2.597987in}}%
\pgfpathlineto{\pgfqpoint{1.373679in}{2.614867in}}%
\pgfpathlineto{\pgfqpoint{1.417280in}{2.630455in}}%
\pgfpathlineto{\pgfqpoint{1.465592in}{2.645324in}}%
\pgfpathlineto{\pgfqpoint{1.518599in}{2.659218in}}%
\pgfpathlineto{\pgfqpoint{1.576268in}{2.671945in}}%
\pgfpathlineto{\pgfqpoint{1.638556in}{2.683360in}}%
\pgfpathlineto{\pgfqpoint{1.705421in}{2.693360in}}%
\pgfpathlineto{\pgfqpoint{1.778986in}{2.702082in}}%
\pgfpathlineto{\pgfqpoint{1.857056in}{2.709096in}}%
\pgfpathlineto{\pgfqpoint{1.939591in}{2.714300in}}%
\pgfpathlineto{\pgfqpoint{2.026557in}{2.717535in}}%
\pgfpathlineto{\pgfqpoint{2.113564in}{2.718547in}}%
\pgfpathlineto{\pgfqpoint{2.198393in}{2.717330in}}%
\pgfpathlineto{\pgfqpoint{2.278824in}{2.713959in}}%
\pgfpathlineto{\pgfqpoint{2.352636in}{2.708632in}}%
\pgfpathlineto{\pgfqpoint{2.417616in}{2.701747in}}%
\pgfpathlineto{\pgfqpoint{2.473730in}{2.693673in}}%
\pgfpathlineto{\pgfqpoint{2.523101in}{2.684417in}}%
\pgfpathlineto{\pgfqpoint{2.565688in}{2.674259in}}%
\pgfpathlineto{\pgfqpoint{2.603563in}{2.662913in}}%
\pgfpathlineto{\pgfqpoint{2.634591in}{2.651378in}}%
\pgfpathlineto{\pgfqpoint{2.660862in}{2.639417in}}%
\pgfpathlineto{\pgfqpoint{2.684329in}{2.626353in}}%
\pgfpathlineto{\pgfqpoint{2.704857in}{2.612302in}}%
\pgfpathlineto{\pgfqpoint{2.722345in}{2.597516in}}%
\pgfpathlineto{\pgfqpoint{2.736786in}{2.582404in}}%
\pgfpathlineto{\pgfqpoint{2.749661in}{2.565545in}}%
\pgfpathlineto{\pgfqpoint{2.760687in}{2.547049in}}%
\pgfpathlineto{\pgfqpoint{2.769704in}{2.527178in}}%
\pgfpathlineto{\pgfqpoint{2.776725in}{2.506277in}}%
\pgfpathlineto{\pgfqpoint{2.782387in}{2.482254in}}%
\pgfpathlineto{\pgfqpoint{2.786421in}{2.455273in}}%
\pgfpathlineto{\pgfqpoint{2.788874in}{2.423043in}}%
\pgfpathlineto{\pgfqpoint{2.789447in}{2.385717in}}%
\pgfpathlineto{\pgfqpoint{2.787874in}{2.338461in}}%
\pgfpathlineto{\pgfqpoint{2.783296in}{2.273958in}}%
\pgfpathlineto{\pgfqpoint{2.773564in}{2.172511in}}%
\pgfpathlineto{\pgfqpoint{2.744488in}{1.878175in}}%
\pgfpathlineto{\pgfqpoint{2.730844in}{1.712132in}}%
\pgfpathlineto{\pgfqpoint{2.718053in}{1.528515in}}%
\pgfpathlineto{\pgfqpoint{2.703691in}{1.287623in}}%
\pgfpathlineto{\pgfqpoint{2.684065in}{0.959820in}}%
\pgfpathlineto{\pgfqpoint{2.674904in}{0.845800in}}%
\pgfpathlineto{\pgfqpoint{2.666418in}{0.766744in}}%
\pgfpathlineto{\pgfqpoint{2.657955in}{0.707799in}}%
\pgfpathlineto{\pgfqpoint{2.649145in}{0.661598in}}%
\pgfpathlineto{\pgfqpoint{2.640173in}{0.625706in}}%
\pgfpathlineto{\pgfqpoint{2.630336in}{0.595378in}}%
\pgfpathlineto{\pgfqpoint{2.620019in}{0.570684in}}%
\pgfpathlineto{\pgfqpoint{2.608682in}{0.549454in}}%
\pgfpathlineto{\pgfqpoint{2.596685in}{0.531765in}}%
\pgfpathlineto{\pgfqpoint{2.582963in}{0.515804in}}%
\pgfpathlineto{\pgfqpoint{2.567673in}{0.501833in}}%
\pgfpathlineto{\pgfqpoint{2.551109in}{0.489912in}}%
\pgfpathlineto{\pgfqpoint{2.531609in}{0.478900in}}%
\pgfpathlineto{\pgfqpoint{2.509274in}{0.469100in}}%
\pgfpathlineto{\pgfqpoint{2.484255in}{0.460598in}}%
\pgfpathlineto{\pgfqpoint{2.454569in}{0.452842in}}%
\pgfpathlineto{\pgfqpoint{2.418129in}{0.445656in}}%
\pgfpathlineto{\pgfqpoint{2.372814in}{0.439074in}}%
\pgfpathlineto{\pgfqpoint{2.316491in}{0.433202in}}%
\pgfpathlineto{\pgfqpoint{2.242678in}{0.427865in}}%
\pgfpathlineto{\pgfqpoint{2.144875in}{0.423186in}}%
\pgfpathlineto{\pgfqpoint{2.012226in}{0.419228in}}%
\pgfpathlineto{\pgfqpoint{1.827344in}{0.416092in}}%
\pgfpathlineto{\pgfqpoint{1.566311in}{0.414030in}}%
\pgfpathlineto{\pgfqpoint{1.237837in}{0.413695in}}%
\pgfpathlineto{\pgfqpoint{0.959400in}{0.415471in}}%
\pgfpathlineto{\pgfqpoint{0.785396in}{0.418626in}}%
\pgfpathlineto{\pgfqpoint{0.681040in}{0.422584in}}%
\pgfpathlineto{\pgfqpoint{0.615904in}{0.427127in}}%
\pgfpathlineto{\pgfqpoint{0.572634in}{0.432262in}}%
\pgfpathlineto{\pgfqpoint{0.542608in}{0.438032in}}%
\pgfpathlineto{\pgfqpoint{0.521569in}{0.444317in}}%
\pgfpathlineto{\pgfqpoint{0.507369in}{0.450581in}}%
\pgfpathlineto{\pgfqpoint{0.495951in}{0.457786in}}%
\pgfpathlineto{\pgfqpoint{0.487377in}{0.465422in}}%
\pgfpathlineto{\pgfqpoint{0.480041in}{0.474591in}}%
\pgfpathlineto{\pgfqpoint{0.473121in}{0.487228in}}%
\pgfpathlineto{\pgfqpoint{0.467368in}{0.503343in}}%
\pgfpathlineto{\pgfqpoint{0.462960in}{0.522597in}}%
\pgfpathlineto{\pgfqpoint{0.459133in}{0.549618in}}%
\pgfpathlineto{\pgfqpoint{0.455930in}{0.589271in}}%
\pgfpathlineto{\pgfqpoint{0.453357in}{0.651428in}}%
\pgfpathlineto{\pgfqpoint{0.451405in}{0.758439in}}%
\pgfpathlineto{\pgfqpoint{0.450035in}{0.977483in}}%
\pgfpathlineto{\pgfqpoint{0.449263in}{1.545020in}}%
\pgfpathlineto{\pgfqpoint{0.449802in}{2.762240in}}%
\pgfpathlineto{\pgfqpoint{0.451339in}{2.864274in}}%
\pgfpathlineto{\pgfqpoint{0.453304in}{2.881521in}}%
\pgfpathlineto{\pgfqpoint{0.455130in}{2.886013in}}%
\pgfpathlineto{\pgfqpoint{0.458531in}{2.888986in}}%
\pgfpathlineto{\pgfqpoint{0.464894in}{2.890540in}}%
\pgfpathlineto{\pgfqpoint{0.482275in}{2.891421in}}%
\pgfpathlineto{\pgfqpoint{0.564936in}{2.891729in}}%
\pgfpathlineto{\pgfqpoint{2.729390in}{2.891760in}}%
\pgfpathlineto{\pgfqpoint{4.789408in}{2.890892in}}%
\pgfpathlineto{\pgfqpoint{4.793630in}{2.889758in}}%
\pgfpathlineto{\pgfqpoint{4.795405in}{2.888364in}}%
\pgfpathlineto{\pgfqpoint{4.797080in}{2.881225in}}%
\pgfpathlineto{\pgfqpoint{4.797985in}{2.858852in}}%
\pgfpathlineto{\pgfqpoint{4.798028in}{2.856363in}}%
\pgfpathlineto{\pgfqpoint{4.798028in}{2.856363in}}%
\pgfusepath{stroke}%
\end{pgfscope}%
\begin{pgfscope}%
\pgfpathrectangle{\pgfqpoint{0.448634in}{0.402556in}}{\pgfqpoint{4.350661in}{2.489204in}} %
\pgfusepath{clip}%
\pgfsetrectcap%
\pgfsetroundjoin%
\pgfsetlinewidth{1.003750pt}%
\definecolor{currentstroke}{rgb}{0.498039,0.498039,0.498039}%
\pgfsetstrokecolor{currentstroke}%
\pgfsetdash{}{0pt}%
\pgfpathmoveto{\pgfqpoint{4.798840in}{2.852369in}}%
\pgfpathlineto{\pgfqpoint{4.797564in}{2.889610in}}%
\pgfpathlineto{\pgfqpoint{4.796215in}{2.891483in}}%
\pgfpathlineto{\pgfqpoint{4.787551in}{2.891760in}}%
\pgfpathlineto{\pgfqpoint{0.452128in}{2.891660in}}%
\pgfpathlineto{\pgfqpoint{0.450530in}{2.890083in}}%
\pgfpathlineto{\pgfqpoint{0.449454in}{2.882764in}}%
\pgfpathlineto{\pgfqpoint{0.448970in}{2.845432in}}%
\pgfpathlineto{\pgfqpoint{0.448743in}{2.494455in}}%
\pgfpathlineto{\pgfqpoint{0.449624in}{0.615108in}}%
\pgfpathlineto{\pgfqpoint{0.451433in}{0.510587in}}%
\pgfpathlineto{\pgfqpoint{0.453994in}{0.473375in}}%
\pgfpathlineto{\pgfqpoint{0.457407in}{0.453869in}}%
\pgfpathlineto{\pgfqpoint{0.461541in}{0.442385in}}%
\pgfpathlineto{\pgfqpoint{0.466739in}{0.434437in}}%
\pgfpathlineto{\pgfqpoint{0.473594in}{0.428350in}}%
\pgfpathlineto{\pgfqpoint{0.483492in}{0.423243in}}%
\pgfpathlineto{\pgfqpoint{0.491853in}{0.420500in}}%
\pgfpathlineto{\pgfqpoint{0.491853in}{0.420500in}}%
\pgfusepath{stroke}%
\end{pgfscope}%
\begin{pgfscope}%
\pgfpathrectangle{\pgfqpoint{0.448634in}{0.402556in}}{\pgfqpoint{4.350661in}{2.489204in}} %
\pgfusepath{clip}%
\pgfsetrectcap%
\pgfsetroundjoin%
\pgfsetlinewidth{1.003750pt}%
\definecolor{currentstroke}{rgb}{0.498039,0.498039,0.498039}%
\pgfsetstrokecolor{currentstroke}%
\pgfsetdash{}{0pt}%
\pgfpathmoveto{\pgfqpoint{2.583205in}{2.736528in}}%
\pgfpathlineto{\pgfqpoint{2.637031in}{2.727670in}}%
\pgfpathlineto{\pgfqpoint{2.681949in}{2.718174in}}%
\pgfpathlineto{\pgfqpoint{2.720074in}{2.707978in}}%
\pgfpathlineto{\pgfqpoint{2.751388in}{2.697502in}}%
\pgfpathlineto{\pgfqpoint{2.777962in}{2.686454in}}%
\pgfpathlineto{\pgfqpoint{2.799777in}{2.675219in}}%
\pgfpathlineto{\pgfqpoint{2.818755in}{2.663072in}}%
\pgfpathlineto{\pgfqpoint{2.834761in}{2.650192in}}%
\pgfpathlineto{\pgfqpoint{2.847732in}{2.636934in}}%
\pgfpathlineto{\pgfqpoint{2.857760in}{2.623836in}}%
\pgfpathlineto{\pgfqpoint{2.866245in}{2.609383in}}%
\pgfpathlineto{\pgfqpoint{2.872978in}{2.593768in}}%
\pgfpathlineto{\pgfqpoint{2.878468in}{2.574889in}}%
\pgfpathlineto{\pgfqpoint{2.881790in}{2.555353in}}%
\pgfpathlineto{\pgfqpoint{2.883376in}{2.533034in}}%
\pgfpathlineto{\pgfqpoint{2.882977in}{2.505666in}}%
\pgfpathlineto{\pgfqpoint{2.880229in}{2.473466in}}%
\pgfpathlineto{\pgfqpoint{2.874373in}{2.431688in}}%
\pgfpathlineto{\pgfqpoint{2.862975in}{2.368300in}}%
\pgfpathlineto{\pgfqpoint{2.816797in}{2.122519in}}%
\pgfpathlineto{\pgfqpoint{2.801089in}{2.022060in}}%
\pgfpathlineto{\pgfqpoint{2.786839in}{1.916276in}}%
\pgfpathlineto{\pgfqpoint{2.774124in}{1.805213in}}%
\pgfpathlineto{\pgfqpoint{2.762480in}{1.683974in}}%
\pgfpathlineto{\pgfqpoint{2.751869in}{1.550108in}}%
\pgfpathlineto{\pgfqpoint{2.742307in}{1.401159in}}%
\pgfpathlineto{\pgfqpoint{2.733735in}{1.232179in}}%
\pgfpathlineto{\pgfqpoint{2.726273in}{1.040702in}}%
\pgfpathlineto{\pgfqpoint{2.718961in}{0.791923in}}%
\pgfpathlineto{\pgfqpoint{2.712034in}{0.570530in}}%
\pgfpathlineto{\pgfqpoint{2.708008in}{0.508477in}}%
\pgfpathlineto{\pgfqpoint{2.703672in}{0.473996in}}%
\pgfpathlineto{\pgfqpoint{2.699331in}{0.454725in}}%
\pgfpathlineto{\pgfqpoint{2.694147in}{0.441042in}}%
\pgfpathlineto{\pgfqpoint{2.687776in}{0.430993in}}%
\pgfpathlineto{\pgfqpoint{2.681170in}{0.424542in}}%
\pgfpathlineto{\pgfqpoint{2.671567in}{0.418751in}}%
\pgfpathlineto{\pgfqpoint{2.659107in}{0.414356in}}%
\pgfpathlineto{\pgfqpoint{2.641970in}{0.410936in}}%
\pgfpathlineto{\pgfqpoint{2.615980in}{0.408204in}}%
\pgfpathlineto{\pgfqpoint{2.570341in}{0.405996in}}%
\pgfpathlineto{\pgfqpoint{2.483340in}{0.404421in}}%
\pgfpathlineto{\pgfqpoint{2.278861in}{0.403454in}}%
\pgfpathlineto{\pgfqpoint{1.543599in}{0.402942in}}%
\pgfpathlineto{\pgfqpoint{0.536423in}{0.403923in}}%
\pgfpathlineto{\pgfqpoint{0.471185in}{0.405627in}}%
\pgfpathlineto{\pgfqpoint{0.458228in}{0.407301in}}%
\pgfpathlineto{\pgfqpoint{0.454130in}{0.408923in}}%
\pgfpathlineto{\pgfqpoint{0.451096in}{0.412347in}}%
\pgfpathlineto{\pgfqpoint{0.449593in}{0.419578in}}%
\pgfpathlineto{\pgfqpoint{0.448903in}{0.441960in}}%
\pgfpathlineto{\pgfqpoint{0.448660in}{0.586333in}}%
\pgfpathlineto{\pgfqpoint{0.448679in}{2.891336in}}%
\pgfpathlineto{\pgfqpoint{0.448679in}{2.891336in}}%
\pgfusepath{stroke}%
\end{pgfscope}%
\begin{pgfscope}%
\pgfpathrectangle{\pgfqpoint{0.448634in}{0.402556in}}{\pgfqpoint{4.350661in}{2.489204in}} %
\pgfusepath{clip}%
\pgfsetrectcap%
\pgfsetroundjoin%
\pgfsetlinewidth{1.003750pt}%
\definecolor{currentstroke}{rgb}{0.498039,0.498039,0.498039}%
\pgfsetstrokecolor{currentstroke}%
\pgfsetdash{}{0pt}%
\pgfpathmoveto{\pgfqpoint{3.427742in}{0.402583in}}%
\pgfpathlineto{\pgfqpoint{2.779498in}{0.403654in}}%
\pgfpathlineto{\pgfqpoint{2.753448in}{0.405400in}}%
\pgfpathlineto{\pgfqpoint{2.742839in}{0.408014in}}%
\pgfpathlineto{\pgfqpoint{2.737219in}{0.411735in}}%
\pgfpathlineto{\pgfqpoint{2.733222in}{0.417586in}}%
\pgfpathlineto{\pgfqpoint{2.730300in}{0.426941in}}%
\pgfpathlineto{\pgfqpoint{2.727904in}{0.444132in}}%
\pgfpathlineto{\pgfqpoint{2.726113in}{0.478916in}}%
\pgfpathlineto{\pgfqpoint{2.724973in}{0.551088in}}%
\pgfpathlineto{\pgfqpoint{2.725338in}{0.687992in}}%
\pgfpathlineto{\pgfqpoint{2.728071in}{0.872166in}}%
\pgfpathlineto{\pgfqpoint{2.733135in}{1.071215in}}%
\pgfpathlineto{\pgfqpoint{2.739997in}{1.250264in}}%
\pgfpathlineto{\pgfqpoint{2.748898in}{1.421715in}}%
\pgfpathlineto{\pgfqpoint{2.758925in}{1.570624in}}%
\pgfpathlineto{\pgfqpoint{2.771671in}{1.724249in}}%
\pgfpathlineto{\pgfqpoint{2.783945in}{1.840393in}}%
\pgfpathlineto{\pgfqpoint{2.800730in}{1.975909in}}%
\pgfpathlineto{\pgfqpoint{2.815815in}{2.073966in}}%
\pgfpathlineto{\pgfqpoint{2.832741in}{2.169091in}}%
\pgfpathlineto{\pgfqpoint{2.852323in}{2.266102in}}%
\pgfpathlineto{\pgfqpoint{2.892461in}{2.459809in}}%
\pgfpathlineto{\pgfqpoint{2.898586in}{2.501533in}}%
\pgfpathlineto{\pgfqpoint{2.900938in}{2.531273in}}%
\pgfpathlineto{\pgfqpoint{2.900867in}{2.556154in}}%
\pgfpathlineto{\pgfqpoint{2.898554in}{2.578385in}}%
\pgfpathlineto{\pgfqpoint{2.894191in}{2.597645in}}%
\pgfpathlineto{\pgfqpoint{2.888330in}{2.613712in}}%
\pgfpathlineto{\pgfqpoint{2.880538in}{2.628665in}}%
\pgfpathlineto{\pgfqpoint{2.871005in}{2.642236in}}%
\pgfpathlineto{\pgfqpoint{2.858390in}{2.655932in}}%
\pgfpathlineto{\pgfqpoint{2.844425in}{2.667798in}}%
\pgfpathlineto{\pgfqpoint{2.827602in}{2.679240in}}%
\pgfpathlineto{\pgfqpoint{2.808000in}{2.690018in}}%
\pgfpathlineto{\pgfqpoint{2.783673in}{2.700833in}}%
\pgfpathlineto{\pgfqpoint{2.754614in}{2.711243in}}%
\pgfpathlineto{\pgfqpoint{2.720868in}{2.720976in}}%
\pgfpathlineto{\pgfqpoint{2.680356in}{2.730325in}}%
\pgfpathlineto{\pgfqpoint{2.633100in}{2.738959in}}%
\pgfpathlineto{\pgfqpoint{2.576976in}{2.746948in}}%
\pgfpathlineto{\pgfqpoint{2.509839in}{2.754161in}}%
\pgfpathlineto{\pgfqpoint{2.433879in}{2.760061in}}%
\pgfpathlineto{\pgfqpoint{2.346957in}{2.764578in}}%
\pgfpathlineto{\pgfqpoint{2.253448in}{2.767222in}}%
\pgfpathlineto{\pgfqpoint{2.149036in}{2.768103in}}%
\pgfpathlineto{\pgfqpoint{2.040277in}{2.766778in}}%
\pgfpathlineto{\pgfqpoint{1.929382in}{2.763186in}}%
\pgfpathlineto{\pgfqpoint{1.827263in}{2.757526in}}%
\pgfpathlineto{\pgfqpoint{1.725264in}{2.749583in}}%
\pgfpathlineto{\pgfqpoint{1.640797in}{2.740547in}}%
\pgfpathlineto{\pgfqpoint{1.565190in}{2.730305in}}%
\pgfpathlineto{\pgfqpoint{1.466154in}{2.714252in}}%
\pgfpathlineto{\pgfqpoint{1.406336in}{2.701132in}}%
\pgfpathlineto{\pgfqpoint{1.353313in}{2.687318in}}%
\pgfpathlineto{\pgfqpoint{1.304982in}{2.672527in}}%
\pgfpathlineto{\pgfqpoint{1.259299in}{2.656222in}}%
\pgfpathlineto{\pgfqpoint{1.216338in}{2.638471in}}%
\pgfpathlineto{\pgfqpoint{1.178254in}{2.620108in}}%
\pgfpathlineto{\pgfqpoint{1.143023in}{2.600567in}}%
\pgfpathlineto{\pgfqpoint{1.110696in}{2.580028in}}%
\pgfpathlineto{\pgfqpoint{1.092562in}{2.566375in}}%
\pgfpathlineto{\pgfqpoint{1.080009in}{2.556524in}}%
\pgfpathlineto{\pgfqpoint{1.051881in}{2.533078in}}%
\pgfpathlineto{\pgfqpoint{1.026743in}{2.509282in}}%
\pgfpathlineto{\pgfqpoint{1.002933in}{2.483759in}}%
\pgfpathlineto{\pgfqpoint{0.980560in}{2.456588in}}%
\pgfpathlineto{\pgfqpoint{0.959701in}{2.427883in}}%
\pgfpathlineto{\pgfqpoint{0.939164in}{2.395737in}}%
\pgfpathlineto{\pgfqpoint{0.919394in}{2.359982in}}%
\pgfpathlineto{\pgfqpoint{0.901446in}{2.322990in}}%
\pgfpathlineto{\pgfqpoint{0.884454in}{2.282630in}}%
\pgfpathlineto{\pgfqpoint{0.868611in}{2.238954in}}%
\pgfpathlineto{\pgfqpoint{0.854057in}{2.192043in}}%
\pgfpathlineto{\pgfqpoint{0.842947in}{2.149092in}}%
\pgfpathlineto{\pgfqpoint{0.831003in}{2.096068in}}%
\pgfpathlineto{\pgfqpoint{0.820102in}{2.037648in}}%
\pgfpathlineto{\pgfqpoint{0.812500in}{1.986111in}}%
\pgfpathlineto{\pgfqpoint{0.796969in}{1.835344in}}%
\pgfpathlineto{\pgfqpoint{0.792025in}{1.753398in}}%
\pgfpathlineto{\pgfqpoint{0.788568in}{1.658893in}}%
\pgfpathlineto{\pgfqpoint{0.785056in}{1.522047in}}%
\pgfpathlineto{\pgfqpoint{0.785056in}{1.522047in}}%
\pgfusepath{stroke}%
\end{pgfscope}%
\begin{pgfscope}%
\pgfpathrectangle{\pgfqpoint{0.448634in}{0.402556in}}{\pgfqpoint{4.350661in}{2.489204in}} %
\pgfusepath{clip}%
\pgfsetrectcap%
\pgfsetroundjoin%
\pgfsetlinewidth{1.003750pt}%
\definecolor{currentstroke}{rgb}{0.498039,0.498039,0.498039}%
\pgfsetstrokecolor{currentstroke}%
\pgfsetdash{}{0pt}%
\pgfpathmoveto{\pgfqpoint{0.456540in}{2.527429in}}%
\pgfpathlineto{\pgfqpoint{0.459326in}{2.691683in}}%
\pgfpathlineto{\pgfqpoint{0.462680in}{2.771240in}}%
\pgfpathlineto{\pgfqpoint{0.466711in}{2.815797in}}%
\pgfpathlineto{\pgfqpoint{0.470964in}{2.840195in}}%
\pgfpathlineto{\pgfqpoint{0.475204in}{2.854305in}}%
\pgfpathlineto{\pgfqpoint{0.480565in}{2.865105in}}%
\pgfpathlineto{\pgfqpoint{0.486490in}{2.872364in}}%
\pgfpathlineto{\pgfqpoint{0.493744in}{2.877826in}}%
\pgfpathlineto{\pgfqpoint{0.503875in}{2.882302in}}%
\pgfpathlineto{\pgfqpoint{0.518778in}{2.885816in}}%
\pgfpathlineto{\pgfqpoint{0.540419in}{2.888282in}}%
\pgfpathlineto{\pgfqpoint{0.579540in}{2.890078in}}%
\pgfpathlineto{\pgfqpoint{0.664371in}{2.891173in}}%
\pgfpathlineto{\pgfqpoint{0.942813in}{2.891663in}}%
\pgfpathlineto{\pgfqpoint{3.759866in}{2.891747in}}%
\pgfpathlineto{\pgfqpoint{4.717007in}{2.890626in}}%
\pgfpathlineto{\pgfqpoint{4.749588in}{2.888754in}}%
\pgfpathlineto{\pgfqpoint{4.762457in}{2.886340in}}%
\pgfpathlineto{\pgfqpoint{4.770592in}{2.882877in}}%
\pgfpathlineto{\pgfqpoint{4.775823in}{2.878455in}}%
\pgfpathlineto{\pgfqpoint{4.779732in}{2.872503in}}%
\pgfpathlineto{\pgfqpoint{4.783093in}{2.863339in}}%
\pgfpathlineto{\pgfqpoint{4.786357in}{2.846336in}}%
\pgfpathlineto{\pgfqpoint{4.788932in}{2.819120in}}%
\pgfpathlineto{\pgfqpoint{4.791141in}{2.769404in}}%
\pgfpathlineto{\pgfqpoint{4.793037in}{2.669861in}}%
\pgfpathlineto{\pgfqpoint{4.794466in}{2.458286in}}%
\pgfpathlineto{\pgfqpoint{4.794957in}{2.012719in}}%
\pgfpathlineto{\pgfqpoint{4.793513in}{1.487501in}}%
\pgfpathlineto{\pgfqpoint{4.790680in}{1.181347in}}%
\pgfpathlineto{\pgfqpoint{4.787114in}{1.007152in}}%
\pgfpathlineto{\pgfqpoint{4.782703in}{0.890272in}}%
\pgfpathlineto{\pgfqpoint{4.777518in}{0.808346in}}%
\pgfpathlineto{\pgfqpoint{4.771671in}{0.748987in}}%
\pgfpathlineto{\pgfqpoint{4.764877in}{0.702342in}}%
\pgfpathlineto{\pgfqpoint{4.757342in}{0.666023in}}%
\pgfpathlineto{\pgfqpoint{4.749267in}{0.637627in}}%
\pgfpathlineto{\pgfqpoint{4.740658in}{0.614778in}}%
\pgfpathlineto{\pgfqpoint{4.731115in}{0.595229in}}%
\pgfpathlineto{\pgfqpoint{4.719669in}{0.577071in}}%
\pgfpathlineto{\pgfqpoint{4.707884in}{0.562434in}}%
\pgfpathlineto{\pgfqpoint{4.694723in}{0.549420in}}%
\pgfpathlineto{\pgfqpoint{4.678574in}{0.536776in}}%
\pgfpathlineto{\pgfqpoint{4.661369in}{0.526103in}}%
\pgfpathlineto{\pgfqpoint{4.641406in}{0.516233in}}%
\pgfpathlineto{\pgfqpoint{4.616700in}{0.506610in}}%
\pgfpathlineto{\pgfqpoint{4.587262in}{0.497704in}}%
\pgfpathlineto{\pgfqpoint{4.553162in}{0.489749in}}%
\pgfpathlineto{\pgfqpoint{4.510168in}{0.482155in}}%
\pgfpathlineto{\pgfqpoint{4.458298in}{0.475392in}}%
\pgfpathlineto{\pgfqpoint{4.395426in}{0.469504in}}%
\pgfpathlineto{\pgfqpoint{4.317239in}{0.464493in}}%
\pgfpathlineto{\pgfqpoint{4.221584in}{0.460670in}}%
\pgfpathlineto{\pgfqpoint{4.108485in}{0.458414in}}%
\pgfpathlineto{\pgfqpoint{3.980142in}{0.458122in}}%
\pgfpathlineto{\pgfqpoint{3.849634in}{0.460057in}}%
\pgfpathlineto{\pgfqpoint{3.730042in}{0.463977in}}%
\pgfpathlineto{\pgfqpoint{3.623568in}{0.469651in}}%
\pgfpathlineto{\pgfqpoint{3.534584in}{0.476526in}}%
\pgfpathlineto{\pgfqpoint{3.458771in}{0.484522in}}%
\pgfpathlineto{\pgfqpoint{3.393988in}{0.493509in}}%
\pgfpathlineto{\pgfqpoint{3.338105in}{0.503465in}}%
\pgfpathlineto{\pgfqpoint{3.289015in}{0.514500in}}%
\pgfpathlineto{\pgfqpoint{3.246755in}{0.526312in}}%
\pgfpathlineto{\pgfqpoint{3.209252in}{0.539175in}}%
\pgfpathlineto{\pgfqpoint{3.176550in}{0.552793in}}%
\pgfpathlineto{\pgfqpoint{3.148638in}{0.566717in}}%
\pgfpathlineto{\pgfqpoint{3.123520in}{0.581571in}}%
\pgfpathlineto{\pgfqpoint{3.099409in}{0.598465in}}%
\pgfpathlineto{\pgfqpoint{3.078276in}{0.615987in}}%
\pgfpathlineto{\pgfqpoint{3.058434in}{0.635384in}}%
\pgfpathlineto{\pgfqpoint{3.040067in}{0.656596in}}%
\pgfpathlineto{\pgfqpoint{3.023299in}{0.679479in}}%
\pgfpathlineto{\pgfqpoint{3.008187in}{0.703826in}}%
\pgfpathlineto{\pgfqpoint{2.993659in}{0.731579in}}%
\pgfpathlineto{\pgfqpoint{2.980047in}{0.762744in}}%
\pgfpathlineto{\pgfqpoint{2.967590in}{0.797246in}}%
\pgfpathlineto{\pgfqpoint{2.956447in}{0.834970in}}%
\pgfpathlineto{\pgfqpoint{2.946695in}{0.875783in}}%
\pgfpathlineto{\pgfqpoint{2.937958in}{0.922004in}}%
\pgfpathlineto{\pgfqpoint{2.930529in}{0.973576in}}%
\pgfpathlineto{\pgfqpoint{2.924641in}{1.030426in}}%
\pgfpathlineto{\pgfqpoint{2.920492in}{1.092470in}}%
\pgfpathlineto{\pgfqpoint{2.918275in}{1.159627in}}%
\pgfpathlineto{\pgfqpoint{2.918197in}{1.231810in}}%
\pgfpathlineto{\pgfqpoint{2.920423in}{1.306439in}}%
\pgfpathlineto{\pgfqpoint{2.924879in}{1.380937in}}%
\pgfpathlineto{\pgfqpoint{2.931562in}{1.455217in}}%
\pgfpathlineto{\pgfqpoint{2.940217in}{1.526717in}}%
\pgfpathlineto{\pgfqpoint{2.950758in}{1.595359in}}%
\pgfpathlineto{\pgfqpoint{2.963137in}{1.661054in}}%
\pgfpathlineto{\pgfqpoint{2.976741in}{1.721301in}}%
\pgfpathlineto{\pgfqpoint{2.991960in}{1.778442in}}%
\pgfpathlineto{\pgfqpoint{3.007970in}{1.830044in}}%
\pgfpathlineto{\pgfqpoint{3.025289in}{1.878409in}}%
\pgfpathlineto{\pgfqpoint{3.043810in}{1.923450in}}%
\pgfpathlineto{\pgfqpoint{3.063393in}{1.965093in}}%
\pgfpathlineto{\pgfqpoint{3.083855in}{2.003287in}}%
\pgfpathlineto{\pgfqpoint{3.104981in}{2.038011in}}%
\pgfpathlineto{\pgfqpoint{3.126519in}{2.069289in}}%
\pgfpathlineto{\pgfqpoint{3.149691in}{2.098998in}}%
\pgfpathlineto{\pgfqpoint{3.172899in}{2.125236in}}%
\pgfpathlineto{\pgfqpoint{3.197519in}{2.149730in}}%
\pgfpathlineto{\pgfqpoint{3.223504in}{2.172301in}}%
\pgfpathlineto{\pgfqpoint{3.248930in}{2.191472in}}%
\pgfpathlineto{\pgfqpoint{3.275422in}{2.208647in}}%
\pgfpathlineto{\pgfqpoint{3.300915in}{2.222640in}}%
\pgfpathlineto{\pgfqpoint{3.327210in}{2.234525in}}%
\pgfpathlineto{\pgfqpoint{3.352150in}{2.243320in}}%
\pgfpathlineto{\pgfqpoint{3.373388in}{2.248672in}}%
\pgfpathlineto{\pgfqpoint{3.392811in}{2.251391in}}%
\pgfpathlineto{\pgfqpoint{3.408027in}{2.251398in}}%
\pgfpathlineto{\pgfqpoint{3.418753in}{2.249446in}}%
\pgfpathlineto{\pgfqpoint{3.426780in}{2.245697in}}%
\pgfpathlineto{\pgfqpoint{3.431542in}{2.240670in}}%
\pgfpathlineto{\pgfqpoint{3.433331in}{2.236160in}}%
\pgfpathlineto{\pgfqpoint{3.433513in}{2.228748in}}%
\pgfpathlineto{\pgfqpoint{3.430759in}{2.219338in}}%
\pgfpathlineto{\pgfqpoint{3.423943in}{2.206624in}}%
\pgfpathlineto{\pgfqpoint{3.411508in}{2.189333in}}%
\pgfpathlineto{\pgfqpoint{3.388871in}{2.162448in}}%
\pgfpathlineto{\pgfqpoint{3.317091in}{2.079155in}}%
\pgfpathlineto{\pgfqpoint{3.286940in}{2.039892in}}%
\pgfpathlineto{\pgfqpoint{3.261227in}{2.002870in}}%
\pgfpathlineto{\pgfqpoint{3.237182in}{1.964409in}}%
\pgfpathlineto{\pgfqpoint{3.214915in}{1.924570in}}%
\pgfpathlineto{\pgfqpoint{3.194480in}{1.883466in}}%
\pgfpathlineto{\pgfqpoint{3.175877in}{1.841239in}}%
\pgfpathlineto{\pgfqpoint{3.159093in}{1.798025in}}%
\pgfpathlineto{\pgfqpoint{3.142652in}{1.749260in}}%
\pgfpathlineto{\pgfqpoint{3.128292in}{1.699642in}}%
\pgfpathlineto{\pgfqpoint{3.116010in}{1.649299in}}%
\pgfpathlineto{\pgfqpoint{3.105195in}{1.595957in}}%
\pgfpathlineto{\pgfqpoint{3.096093in}{1.539666in}}%
\pgfpathlineto{\pgfqpoint{3.089223in}{1.482962in}}%
\pgfpathlineto{\pgfqpoint{3.084532in}{1.425966in}}%
\pgfpathlineto{\pgfqpoint{3.082023in}{1.368791in}}%
\pgfpathlineto{\pgfqpoint{3.081753in}{1.311545in}}%
\pgfpathlineto{\pgfqpoint{3.083680in}{1.256832in}}%
\pgfpathlineto{\pgfqpoint{3.087696in}{1.204766in}}%
\pgfpathlineto{\pgfqpoint{3.093605in}{1.155448in}}%
\pgfpathlineto{\pgfqpoint{3.101784in}{1.106558in}}%
\pgfpathlineto{\pgfqpoint{3.111831in}{1.060688in}}%
\pgfpathlineto{\pgfqpoint{3.123641in}{1.017976in}}%
\pgfpathlineto{\pgfqpoint{3.136183in}{0.980831in}}%
\pgfpathlineto{\pgfqpoint{3.150730in}{0.944658in}}%
\pgfpathlineto{\pgfqpoint{3.167384in}{0.909696in}}%
\pgfpathlineto{\pgfqpoint{3.186205in}{0.876205in}}%
\pgfpathlineto{\pgfqpoint{3.204647in}{0.848482in}}%
\pgfpathlineto{\pgfqpoint{3.224740in}{0.822304in}}%
\pgfpathlineto{\pgfqpoint{3.246424in}{0.797844in}}%
\pgfpathlineto{\pgfqpoint{3.269554in}{0.775185in}}%
\pgfpathlineto{\pgfqpoint{3.293967in}{0.754362in}}%
\pgfpathlineto{\pgfqpoint{3.321344in}{0.734059in}}%
\pgfpathlineto{\pgfqpoint{3.349809in}{0.715817in}}%
\pgfpathlineto{\pgfqpoint{3.381114in}{0.698426in}}%
\pgfpathlineto{\pgfqpoint{3.415206in}{0.682044in}}%
\pgfpathlineto{\pgfqpoint{3.445962in}{0.669599in}}%
\pgfpathlineto{\pgfqpoint{3.483318in}{0.656196in}}%
\pgfpathlineto{\pgfqpoint{3.527433in}{0.642636in}}%
\pgfpathlineto{\pgfqpoint{3.572027in}{0.631331in}}%
\pgfpathlineto{\pgfqpoint{3.636108in}{0.617400in}}%
\pgfpathlineto{\pgfqpoint{3.692109in}{0.608354in}}%
\pgfpathlineto{\pgfqpoint{3.765579in}{0.598919in}}%
\pgfpathlineto{\pgfqpoint{3.832834in}{0.593310in}}%
\pgfpathlineto{\pgfqpoint{3.900201in}{0.589913in}}%
\pgfpathlineto{\pgfqpoint{3.971973in}{0.588408in}}%
\pgfpathlineto{\pgfqpoint{4.041578in}{0.589120in}}%
\pgfpathlineto{\pgfqpoint{4.113314in}{0.592106in}}%
\pgfpathlineto{\pgfqpoint{4.176248in}{0.597051in}}%
\pgfpathlineto{\pgfqpoint{4.236844in}{0.604075in}}%
\pgfpathlineto{\pgfqpoint{4.288550in}{0.612318in}}%
\pgfpathlineto{\pgfqpoint{4.335647in}{0.622019in}}%
\pgfpathlineto{\pgfqpoint{4.378075in}{0.633013in}}%
\pgfpathlineto{\pgfqpoint{4.415783in}{0.645067in}}%
\pgfpathlineto{\pgfqpoint{4.448745in}{0.657841in}}%
\pgfpathlineto{\pgfqpoint{4.480995in}{0.672803in}}%
\pgfpathlineto{\pgfqpoint{4.508339in}{0.688132in}}%
\pgfpathlineto{\pgfqpoint{4.532798in}{0.704360in}}%
\pgfpathlineto{\pgfqpoint{4.554350in}{0.721201in}}%
\pgfpathlineto{\pgfqpoint{4.574720in}{0.739869in}}%
\pgfpathlineto{\pgfqpoint{4.593716in}{0.760344in}}%
\pgfpathlineto{\pgfqpoint{4.611194in}{0.782520in}}%
\pgfpathlineto{\pgfqpoint{4.627061in}{0.806230in}}%
\pgfpathlineto{\pgfqpoint{4.642404in}{0.833404in}}%
\pgfpathlineto{\pgfqpoint{4.655976in}{0.861786in}}%
\pgfpathlineto{\pgfqpoint{4.668730in}{0.893424in}}%
\pgfpathlineto{\pgfqpoint{4.681220in}{0.930592in}}%
\pgfpathlineto{\pgfqpoint{4.692419in}{0.970915in}}%
\pgfpathlineto{\pgfqpoint{4.702828in}{1.016680in}}%
\pgfpathlineto{\pgfqpoint{4.712753in}{1.070248in}}%
\pgfpathlineto{\pgfqpoint{4.721833in}{1.131600in}}%
\pgfpathlineto{\pgfqpoint{4.729597in}{1.198215in}}%
\pgfpathlineto{\pgfqpoint{4.731478in}{1.220504in}}%
\pgfpathlineto{\pgfqpoint{4.738338in}{1.302269in}}%
\pgfpathlineto{\pgfqpoint{4.745179in}{1.409015in}}%
\pgfpathlineto{\pgfqpoint{4.750548in}{1.528337in}}%
\pgfpathlineto{\pgfqpoint{4.754938in}{1.672622in}}%
\pgfpathlineto{\pgfqpoint{4.757774in}{1.836876in}}%
\pgfpathlineto{\pgfqpoint{4.758912in}{2.023560in}}%
\pgfpathlineto{\pgfqpoint{4.757847in}{2.202777in}}%
\pgfpathlineto{\pgfqpoint{4.754776in}{2.357066in}}%
\pgfpathlineto{\pgfqpoint{4.750319in}{2.473945in}}%
\pgfpathlineto{\pgfqpoint{4.744787in}{2.563328in}}%
\pgfpathlineto{\pgfqpoint{4.738474in}{2.630143in}}%
\pgfpathlineto{\pgfqpoint{4.731757in}{2.679324in}}%
\pgfpathlineto{\pgfqpoint{4.723862in}{2.720658in}}%
\pgfpathlineto{\pgfqpoint{4.716270in}{2.749228in}}%
\pgfpathlineto{\pgfqpoint{4.708112in}{2.772293in}}%
\pgfpathlineto{\pgfqpoint{4.698861in}{2.792022in}}%
\pgfpathlineto{\pgfqpoint{4.688818in}{2.808270in}}%
\pgfpathlineto{\pgfqpoint{4.678457in}{2.821022in}}%
\pgfpathlineto{\pgfqpoint{4.666724in}{2.832109in}}%
\pgfpathlineto{\pgfqpoint{4.653865in}{2.841422in}}%
\pgfpathlineto{\pgfqpoint{4.638177in}{2.850017in}}%
\pgfpathlineto{\pgfqpoint{4.619730in}{2.857495in}}%
\pgfpathlineto{\pgfqpoint{4.596552in}{2.864265in}}%
\pgfpathlineto{\pgfqpoint{4.568719in}{2.869960in}}%
\pgfpathlineto{\pgfqpoint{4.532009in}{2.875038in}}%
\pgfpathlineto{\pgfqpoint{4.484296in}{2.879254in}}%
\pgfpathlineto{\pgfqpoint{4.416937in}{2.882874in}}%
\pgfpathlineto{\pgfqpoint{4.316906in}{2.885821in}}%
\pgfpathlineto{\pgfqpoint{4.158120in}{2.888056in}}%
\pgfpathlineto{\pgfqpoint{3.866630in}{2.889694in}}%
\pgfpathlineto{\pgfqpoint{3.227083in}{2.890637in}}%
\pgfpathlineto{\pgfqpoint{1.930586in}{2.890224in}}%
\pgfpathlineto{\pgfqpoint{1.378056in}{2.888262in}}%
\pgfpathlineto{\pgfqpoint{1.138784in}{2.885393in}}%
\pgfpathlineto{\pgfqpoint{1.006129in}{2.881705in}}%
\pgfpathlineto{\pgfqpoint{0.919208in}{2.877192in}}%
\pgfpathlineto{\pgfqpoint{0.858472in}{2.871976in}}%
\pgfpathlineto{\pgfqpoint{0.813089in}{2.866047in}}%
\pgfpathlineto{\pgfqpoint{0.776605in}{2.859167in}}%
\pgfpathlineto{\pgfqpoint{0.746927in}{2.851379in}}%
\pgfpathlineto{\pgfqpoint{0.721999in}{2.842540in}}%
\pgfpathlineto{\pgfqpoint{0.701849in}{2.833183in}}%
\pgfpathlineto{\pgfqpoint{0.684492in}{2.822840in}}%
\pgfpathlineto{\pgfqpoint{0.668179in}{2.810474in}}%
\pgfpathlineto{\pgfqpoint{0.654811in}{2.797738in}}%
\pgfpathlineto{\pgfqpoint{0.641271in}{2.781575in}}%
\pgfpathlineto{\pgfqpoint{0.629412in}{2.763763in}}%
\pgfpathlineto{\pgfqpoint{0.618143in}{2.742484in}}%
\pgfpathlineto{\pgfqpoint{0.607800in}{2.717803in}}%
\pgfpathlineto{\pgfqpoint{0.598548in}{2.689880in}}%
\pgfpathlineto{\pgfqpoint{0.589854in}{2.656489in}}%
\pgfpathlineto{\pgfqpoint{0.581582in}{2.615251in}}%
\pgfpathlineto{\pgfqpoint{0.574184in}{2.566197in}}%
\pgfpathlineto{\pgfqpoint{0.567903in}{2.509402in}}%
\pgfpathlineto{\pgfqpoint{0.562547in}{2.439978in}}%
\pgfpathlineto{\pgfqpoint{0.558567in}{2.357964in}}%
\pgfpathlineto{\pgfqpoint{0.556353in}{2.265901in}}%
\pgfpathlineto{\pgfqpoint{0.556155in}{2.163846in}}%
\pgfpathlineto{\pgfqpoint{0.558147in}{2.059327in}}%
\pgfpathlineto{\pgfqpoint{0.562269in}{1.957381in}}%
\pgfpathlineto{\pgfqpoint{0.568235in}{1.863041in}}%
\pgfpathlineto{\pgfqpoint{0.572343in}{1.813480in}}%
\pgfpathlineto{\pgfqpoint{0.572343in}{1.813480in}}%
\pgfusepath{stroke}%
\end{pgfscope}%
\begin{pgfscope}%
\pgfpathrectangle{\pgfqpoint{0.448634in}{0.402556in}}{\pgfqpoint{4.350661in}{2.489204in}} %
\pgfusepath{clip}%
\pgfsetrectcap%
\pgfsetroundjoin%
\pgfsetlinewidth{1.003750pt}%
\definecolor{currentstroke}{rgb}{0.737255,0.741176,0.133333}%
\pgfsetstrokecolor{currentstroke}%
\pgfsetdash{}{0pt}%
\pgfpathmoveto{\pgfqpoint{0.448634in}{2.896245in}}%
\pgfpathlineto{\pgfqpoint{0.448593in}{0.407043in}}%
\pgfpathlineto{\pgfqpoint{0.448593in}{0.407043in}}%
\pgfusepath{stroke}%
\end{pgfscope}%
\begin{pgfscope}%
\pgfpathrectangle{\pgfqpoint{0.448634in}{0.402556in}}{\pgfqpoint{4.350661in}{2.489204in}} %
\pgfusepath{clip}%
\pgfsetrectcap%
\pgfsetroundjoin%
\pgfsetlinewidth{1.003750pt}%
\definecolor{currentstroke}{rgb}{0.737255,0.741176,0.133333}%
\pgfsetstrokecolor{currentstroke}%
\pgfsetdash{}{0pt}%
\pgfpathmoveto{\pgfqpoint{1.875515in}{2.754162in}}%
\pgfpathlineto{\pgfqpoint{1.975495in}{2.758856in}}%
\pgfpathlineto{\pgfqpoint{2.077708in}{2.761461in}}%
\pgfpathlineto{\pgfqpoint{2.184297in}{2.761935in}}%
\pgfpathlineto{\pgfqpoint{2.284349in}{2.760197in}}%
\pgfpathlineto{\pgfqpoint{2.380002in}{2.756314in}}%
\pgfpathlineto{\pgfqpoint{2.462522in}{2.750805in}}%
\pgfpathlineto{\pgfqpoint{2.536218in}{2.743679in}}%
\pgfpathlineto{\pgfqpoint{2.598884in}{2.735403in}}%
\pgfpathlineto{\pgfqpoint{2.650494in}{2.726418in}}%
\pgfpathlineto{\pgfqpoint{2.695319in}{2.716362in}}%
\pgfpathlineto{\pgfqpoint{2.731207in}{2.706174in}}%
\pgfpathlineto{\pgfqpoint{2.762362in}{2.695094in}}%
\pgfpathlineto{\pgfqpoint{2.788703in}{2.683342in}}%
\pgfpathlineto{\pgfqpoint{2.810194in}{2.671323in}}%
\pgfpathlineto{\pgfqpoint{2.826913in}{2.659683in}}%
\pgfpathlineto{\pgfqpoint{2.842442in}{2.646062in}}%
\pgfpathlineto{\pgfqpoint{2.854807in}{2.632069in}}%
\pgfpathlineto{\pgfqpoint{2.864143in}{2.618319in}}%
\pgfpathlineto{\pgfqpoint{2.871811in}{2.603281in}}%
\pgfpathlineto{\pgfqpoint{2.877666in}{2.587209in}}%
\pgfpathlineto{\pgfqpoint{2.882185in}{2.567993in}}%
\pgfpathlineto{\pgfqpoint{2.884842in}{2.545811in}}%
\pgfpathlineto{\pgfqpoint{2.885423in}{2.520938in}}%
\pgfpathlineto{\pgfqpoint{2.883779in}{2.491135in}}%
\pgfpathlineto{\pgfqpoint{2.879413in}{2.454139in}}%
\pgfpathlineto{\pgfqpoint{2.870917in}{2.402782in}}%
\pgfpathlineto{\pgfqpoint{2.852926in}{2.310461in}}%
\pgfpathlineto{\pgfqpoint{2.826351in}{2.171877in}}%
\pgfpathlineto{\pgfqpoint{2.809494in}{2.071662in}}%
\pgfpathlineto{\pgfqpoint{2.794807in}{1.971000in}}%
\pgfpathlineto{\pgfqpoint{2.781561in}{1.865045in}}%
\pgfpathlineto{\pgfqpoint{2.769287in}{1.748901in}}%
\pgfpathlineto{\pgfqpoint{2.758176in}{1.622592in}}%
\pgfpathlineto{\pgfqpoint{2.748001in}{1.481188in}}%
\pgfpathlineto{\pgfqpoint{2.738868in}{1.322223in}}%
\pgfpathlineto{\pgfqpoint{2.730892in}{1.143235in}}%
\pgfpathlineto{\pgfqpoint{2.723924in}{0.934295in}}%
\pgfpathlineto{\pgfqpoint{2.710732in}{0.511429in}}%
\pgfpathlineto{\pgfqpoint{2.706953in}{0.474351in}}%
\pgfpathlineto{\pgfqpoint{2.702755in}{0.452485in}}%
\pgfpathlineto{\pgfqpoint{2.698007in}{0.438596in}}%
\pgfpathlineto{\pgfqpoint{2.693301in}{0.430242in}}%
\pgfpathlineto{\pgfqpoint{2.687071in}{0.423327in}}%
\pgfpathlineto{\pgfqpoint{2.679613in}{0.418234in}}%
\pgfpathlineto{\pgfqpoint{2.669389in}{0.414032in}}%
\pgfpathlineto{\pgfqpoint{2.654473in}{0.410580in}}%
\pgfpathlineto{\pgfqpoint{2.630675in}{0.407785in}}%
\pgfpathlineto{\pgfqpoint{2.591563in}{0.405732in}}%
\pgfpathlineto{\pgfqpoint{2.513263in}{0.404220in}}%
\pgfpathlineto{\pgfqpoint{2.324012in}{0.403318in}}%
\pgfpathlineto{\pgfqpoint{1.536542in}{0.402847in}}%
\pgfpathlineto{\pgfqpoint{0.518489in}{0.403862in}}%
\pgfpathlineto{\pgfqpoint{0.461960in}{0.405634in}}%
\pgfpathlineto{\pgfqpoint{0.453492in}{0.407690in}}%
\pgfpathlineto{\pgfqpoint{0.450609in}{0.411243in}}%
\pgfpathlineto{\pgfqpoint{0.449348in}{0.418542in}}%
\pgfpathlineto{\pgfqpoint{0.448806in}{0.443421in}}%
\pgfpathlineto{\pgfqpoint{0.448644in}{0.664960in}}%
\pgfpathlineto{\pgfqpoint{0.448661in}{2.890308in}}%
\pgfpathlineto{\pgfqpoint{0.448661in}{2.890308in}}%
\pgfusepath{stroke}%
\end{pgfscope}%
\begin{pgfscope}%
\pgfpathrectangle{\pgfqpoint{0.448634in}{0.402556in}}{\pgfqpoint{4.350661in}{2.489204in}} %
\pgfusepath{clip}%
\pgfsetrectcap%
\pgfsetroundjoin%
\pgfsetlinewidth{1.003750pt}%
\definecolor{currentstroke}{rgb}{0.737255,0.741176,0.133333}%
\pgfsetstrokecolor{currentstroke}%
\pgfsetdash{}{0pt}%
\pgfpathmoveto{\pgfqpoint{3.577928in}{0.462532in}}%
\pgfpathlineto{\pgfqpoint{3.488947in}{0.469464in}}%
\pgfpathlineto{\pgfqpoint{3.413143in}{0.477572in}}%
\pgfpathlineto{\pgfqpoint{3.350539in}{0.486440in}}%
\pgfpathlineto{\pgfqpoint{3.296837in}{0.496234in}}%
\pgfpathlineto{\pgfqpoint{3.249928in}{0.507057in}}%
\pgfpathlineto{\pgfqpoint{3.209844in}{0.518569in}}%
\pgfpathlineto{\pgfqpoint{3.174506in}{0.531023in}}%
\pgfpathlineto{\pgfqpoint{3.143941in}{0.544078in}}%
\pgfpathlineto{\pgfqpoint{3.116156in}{0.558330in}}%
\pgfpathlineto{\pgfqpoint{3.091232in}{0.573603in}}%
\pgfpathlineto{\pgfqpoint{3.069198in}{0.589607in}}%
\pgfpathlineto{\pgfqpoint{3.048324in}{0.607530in}}%
\pgfpathlineto{\pgfqpoint{3.028823in}{0.627373in}}%
\pgfpathlineto{\pgfqpoint{3.010869in}{0.649043in}}%
\pgfpathlineto{\pgfqpoint{2.994572in}{0.672367in}}%
\pgfpathlineto{\pgfqpoint{2.979969in}{0.697116in}}%
\pgfpathlineto{\pgfqpoint{2.966016in}{0.725254in}}%
\pgfpathlineto{\pgfqpoint{2.953036in}{0.756771in}}%
\pgfpathlineto{\pgfqpoint{2.941270in}{0.791589in}}%
\pgfpathlineto{\pgfqpoint{2.930875in}{0.829591in}}%
\pgfpathlineto{\pgfqpoint{2.921937in}{0.870648in}}%
\pgfpathlineto{\pgfqpoint{2.914124in}{0.917084in}}%
\pgfpathlineto{\pgfqpoint{2.907725in}{0.968837in}}%
\pgfpathlineto{\pgfqpoint{2.902951in}{1.025823in}}%
\pgfpathlineto{\pgfqpoint{2.899973in}{1.087956in}}%
\pgfpathlineto{\pgfqpoint{2.898946in}{1.155151in}}%
\pgfpathlineto{\pgfqpoint{2.900062in}{1.227323in}}%
\pgfpathlineto{\pgfqpoint{2.903422in}{1.301896in}}%
\pgfpathlineto{\pgfqpoint{2.909135in}{1.378781in}}%
\pgfpathlineto{\pgfqpoint{2.917106in}{1.455401in}}%
\pgfpathlineto{\pgfqpoint{2.927025in}{1.529206in}}%
\pgfpathlineto{\pgfqpoint{2.938778in}{1.600125in}}%
\pgfpathlineto{\pgfqpoint{2.952285in}{1.668084in}}%
\pgfpathlineto{\pgfqpoint{2.967495in}{1.732995in}}%
\pgfpathlineto{\pgfqpoint{2.983690in}{1.792397in}}%
\pgfpathlineto{\pgfqpoint{3.000591in}{1.846279in}}%
\pgfpathlineto{\pgfqpoint{3.018777in}{1.896928in}}%
\pgfpathlineto{\pgfqpoint{3.037234in}{1.942004in}}%
\pgfpathlineto{\pgfqpoint{3.056656in}{1.983746in}}%
\pgfpathlineto{\pgfqpoint{3.076909in}{2.022085in}}%
\pgfpathlineto{\pgfqpoint{3.097806in}{2.056991in}}%
\pgfpathlineto{\pgfqpoint{3.119108in}{2.088480in}}%
\pgfpathlineto{\pgfqpoint{3.142023in}{2.118449in}}%
\pgfpathlineto{\pgfqpoint{3.166537in}{2.146713in}}%
\pgfpathlineto{\pgfqpoint{3.190927in}{2.171508in}}%
\pgfpathlineto{\pgfqpoint{3.216610in}{2.194529in}}%
\pgfpathlineto{\pgfqpoint{3.243498in}{2.215671in}}%
\pgfpathlineto{\pgfqpoint{3.271492in}{2.234844in}}%
\pgfpathlineto{\pgfqpoint{3.300483in}{2.251964in}}%
\pgfpathlineto{\pgfqpoint{3.330366in}{2.266944in}}%
\pgfpathlineto{\pgfqpoint{3.358969in}{2.278893in}}%
\pgfpathlineto{\pgfqpoint{3.388178in}{2.288736in}}%
\pgfpathlineto{\pgfqpoint{3.415773in}{2.295783in}}%
\pgfpathlineto{\pgfqpoint{3.441584in}{2.300201in}}%
\pgfpathlineto{\pgfqpoint{3.465451in}{2.302034in}}%
\pgfpathlineto{\pgfqpoint{3.485010in}{2.301361in}}%
\pgfpathlineto{\pgfqpoint{3.500056in}{2.298764in}}%
\pgfpathlineto{\pgfqpoint{3.510434in}{2.295088in}}%
\pgfpathlineto{\pgfqpoint{3.518030in}{2.290285in}}%
\pgfpathlineto{\pgfqpoint{3.522601in}{2.285001in}}%
\pgfpathlineto{\pgfqpoint{3.525082in}{2.278160in}}%
\pgfpathlineto{\pgfqpoint{3.524895in}{2.270740in}}%
\pgfpathlineto{\pgfqpoint{3.521760in}{2.261490in}}%
\pgfpathlineto{\pgfqpoint{3.515629in}{2.251230in}}%
\pgfpathlineto{\pgfqpoint{3.505306in}{2.238434in}}%
\pgfpathlineto{\pgfqpoint{3.487435in}{2.220237in}}%
\pgfpathlineto{\pgfqpoint{3.454979in}{2.190959in}}%
\pgfpathlineto{\pgfqpoint{3.381275in}{2.125064in}}%
\pgfpathlineto{\pgfqpoint{3.346676in}{2.090939in}}%
\pgfpathlineto{\pgfqpoint{3.316697in}{2.058388in}}%
\pgfpathlineto{\pgfqpoint{3.289722in}{2.025917in}}%
\pgfpathlineto{\pgfqpoint{3.264337in}{1.991811in}}%
\pgfpathlineto{\pgfqpoint{3.241959in}{1.958128in}}%
\pgfpathlineto{\pgfqpoint{3.220047in}{1.921002in}}%
\pgfpathlineto{\pgfqpoint{3.200012in}{1.882513in}}%
\pgfpathlineto{\pgfqpoint{3.181856in}{1.842822in}}%
\pgfpathlineto{\pgfqpoint{3.164685in}{1.799808in}}%
\pgfpathlineto{\pgfqpoint{3.149480in}{1.755835in}}%
\pgfpathlineto{\pgfqpoint{3.135488in}{1.708701in}}%
\pgfpathlineto{\pgfqpoint{3.122890in}{1.658460in}}%
\pgfpathlineto{\pgfqpoint{3.111842in}{1.605181in}}%
\pgfpathlineto{\pgfqpoint{3.102506in}{1.548940in}}%
\pgfpathlineto{\pgfqpoint{3.095325in}{1.492285in}}%
\pgfpathlineto{\pgfqpoint{3.090066in}{1.432853in}}%
\pgfpathlineto{\pgfqpoint{3.087085in}{1.373215in}}%
\pgfpathlineto{\pgfqpoint{3.086423in}{1.315972in}}%
\pgfpathlineto{\pgfqpoint{3.088025in}{1.258755in}}%
\pgfpathlineto{\pgfqpoint{3.091829in}{1.204172in}}%
\pgfpathlineto{\pgfqpoint{3.097735in}{1.152343in}}%
\pgfpathlineto{\pgfqpoint{3.105184in}{1.105828in}}%
\pgfpathlineto{\pgfqpoint{3.114277in}{1.062253in}}%
\pgfpathlineto{\pgfqpoint{3.124881in}{1.021720in}}%
\pgfpathlineto{\pgfqpoint{3.137541in}{0.981968in}}%
\pgfpathlineto{\pgfqpoint{3.151592in}{0.945537in}}%
\pgfpathlineto{\pgfqpoint{3.165725in}{0.914678in}}%
\pgfpathlineto{\pgfqpoint{3.181712in}{0.885025in}}%
\pgfpathlineto{\pgfqpoint{3.199513in}{0.856758in}}%
\pgfpathlineto{\pgfqpoint{3.217643in}{0.831932in}}%
\pgfpathlineto{\pgfqpoint{3.237237in}{0.808609in}}%
\pgfpathlineto{\pgfqpoint{3.258222in}{0.786926in}}%
\pgfpathlineto{\pgfqpoint{3.282126in}{0.765344in}}%
\pgfpathlineto{\pgfqpoint{3.307230in}{0.745626in}}%
\pgfpathlineto{\pgfqpoint{3.333377in}{0.727769in}}%
\pgfpathlineto{\pgfqpoint{3.362290in}{0.710474in}}%
\pgfpathlineto{\pgfqpoint{3.399754in}{0.690550in}}%
\pgfpathlineto{\pgfqpoint{3.434256in}{0.675337in}}%
\pgfpathlineto{\pgfqpoint{3.473442in}{0.660313in}}%
\pgfpathlineto{\pgfqpoint{3.515272in}{0.646640in}}%
\pgfpathlineto{\pgfqpoint{3.561797in}{0.633823in}}%
\pgfpathlineto{\pgfqpoint{3.610857in}{0.622609in}}%
\pgfpathlineto{\pgfqpoint{3.703389in}{0.607031in}}%
\pgfpathlineto{\pgfqpoint{3.776992in}{0.598751in}}%
\pgfpathlineto{\pgfqpoint{3.844273in}{0.593576in}}%
\pgfpathlineto{\pgfqpoint{3.913827in}{0.590435in}}%
\pgfpathlineto{\pgfqpoint{3.987779in}{0.589241in}}%
\pgfpathlineto{\pgfqpoint{4.059556in}{0.590330in}}%
\pgfpathlineto{\pgfqpoint{4.131278in}{0.593699in}}%
\pgfpathlineto{\pgfqpoint{4.194191in}{0.598984in}}%
\pgfpathlineto{\pgfqpoint{4.256916in}{0.606603in}}%
\pgfpathlineto{\pgfqpoint{4.308549in}{0.615415in}}%
\pgfpathlineto{\pgfqpoint{4.359832in}{0.626540in}}%
\pgfpathlineto{\pgfqpoint{4.397819in}{0.637376in}}%
\pgfpathlineto{\pgfqpoint{4.435320in}{0.650250in}}%
\pgfpathlineto{\pgfqpoint{4.461908in}{0.661261in}}%
\pgfpathlineto{\pgfqpoint{4.491900in}{0.675945in}}%
\pgfpathlineto{\pgfqpoint{4.517061in}{0.690704in}}%
\pgfpathlineto{\pgfqpoint{4.541232in}{0.707487in}}%
\pgfpathlineto{\pgfqpoint{4.562441in}{0.724889in}}%
\pgfpathlineto{\pgfqpoint{4.582363in}{0.744176in}}%
\pgfpathlineto{\pgfqpoint{4.600812in}{0.765297in}}%
\pgfpathlineto{\pgfqpoint{4.617647in}{0.788114in}}%
\pgfpathlineto{\pgfqpoint{4.632793in}{0.812431in}}%
\pgfpathlineto{\pgfqpoint{4.647375in}{0.840148in}}%
\pgfpathlineto{\pgfqpoint{4.661112in}{0.871242in}}%
\pgfpathlineto{\pgfqpoint{4.673813in}{0.905628in}}%
\pgfpathlineto{\pgfqpoint{4.686062in}{0.945550in}}%
\pgfpathlineto{\pgfqpoint{4.697519in}{0.990985in}}%
\pgfpathlineto{\pgfqpoint{4.708006in}{1.041858in}}%
\pgfpathlineto{\pgfqpoint{4.717737in}{1.100547in}}%
\pgfpathlineto{\pgfqpoint{4.726437in}{1.167010in}}%
\pgfpathlineto{\pgfqpoint{4.733977in}{1.241182in}}%
\pgfpathlineto{\pgfqpoint{4.746071in}{1.417360in}}%
\pgfpathlineto{\pgfqpoint{4.750861in}{1.529236in}}%
\pgfpathlineto{\pgfqpoint{4.755153in}{1.673525in}}%
\pgfpathlineto{\pgfqpoint{4.757894in}{1.840271in}}%
\pgfpathlineto{\pgfqpoint{4.758846in}{2.036914in}}%
\pgfpathlineto{\pgfqpoint{4.757561in}{2.218618in}}%
\pgfpathlineto{\pgfqpoint{4.754381in}{2.370413in}}%
\pgfpathlineto{\pgfqpoint{4.749872in}{2.482306in}}%
\pgfpathlineto{\pgfqpoint{4.744260in}{2.569187in}}%
\pgfpathlineto{\pgfqpoint{4.737677in}{2.635967in}}%
\pgfpathlineto{\pgfqpoint{4.730687in}{2.685098in}}%
\pgfpathlineto{\pgfqpoint{4.722983in}{2.723929in}}%
\pgfpathlineto{\pgfqpoint{4.715195in}{2.752430in}}%
\pgfpathlineto{\pgfqpoint{4.706800in}{2.775382in}}%
\pgfpathlineto{\pgfqpoint{4.697255in}{2.794925in}}%
\pgfpathlineto{\pgfqpoint{4.686870in}{2.810886in}}%
\pgfpathlineto{\pgfqpoint{4.676215in}{2.823316in}}%
\pgfpathlineto{\pgfqpoint{4.664230in}{2.834045in}}%
\pgfpathlineto{\pgfqpoint{4.651177in}{2.842999in}}%
\pgfpathlineto{\pgfqpoint{4.635341in}{2.851234in}}%
\pgfpathlineto{\pgfqpoint{4.616802in}{2.858409in}}%
\pgfpathlineto{\pgfqpoint{4.593566in}{2.864920in}}%
\pgfpathlineto{\pgfqpoint{4.563548in}{2.870766in}}%
\pgfpathlineto{\pgfqpoint{4.524644in}{2.875799in}}%
\pgfpathlineto{\pgfqpoint{4.474739in}{2.879854in}}%
\pgfpathlineto{\pgfqpoint{4.403021in}{2.883368in}}%
\pgfpathlineto{\pgfqpoint{4.292109in}{2.886305in}}%
\pgfpathlineto{\pgfqpoint{4.126794in}{2.888302in}}%
\pgfpathlineto{\pgfqpoint{3.765693in}{2.890094in}}%
\pgfpathlineto{\pgfqpoint{2.934717in}{2.890806in}}%
\pgfpathlineto{\pgfqpoint{1.694780in}{2.889876in}}%
\pgfpathlineto{\pgfqpoint{1.292348in}{2.887733in}}%
\pgfpathlineto{\pgfqpoint{1.087886in}{2.884666in}}%
\pgfpathlineto{\pgfqpoint{0.974815in}{2.881033in}}%
\pgfpathlineto{\pgfqpoint{0.898776in}{2.876678in}}%
\pgfpathlineto{\pgfqpoint{0.840239in}{2.871243in}}%
\pgfpathlineto{\pgfqpoint{0.794909in}{2.864807in}}%
\pgfpathlineto{\pgfqpoint{0.764913in}{2.858807in}}%
\pgfpathlineto{\pgfqpoint{0.737401in}{2.851349in}}%
\pgfpathlineto{\pgfqpoint{0.712531in}{2.842303in}}%
\pgfpathlineto{\pgfqpoint{0.692510in}{2.832592in}}%
\pgfpathlineto{\pgfqpoint{0.675353in}{2.821821in}}%
\pgfpathlineto{\pgfqpoint{0.659329in}{2.808972in}}%
\pgfpathlineto{\pgfqpoint{0.646293in}{2.795794in}}%
\pgfpathlineto{\pgfqpoint{0.633187in}{2.779169in}}%
\pgfpathlineto{\pgfqpoint{0.621791in}{2.760965in}}%
\pgfpathlineto{\pgfqpoint{0.611028in}{2.739346in}}%
\pgfpathlineto{\pgfqpoint{0.601198in}{2.714391in}}%
\pgfpathlineto{\pgfqpoint{0.592427in}{2.686265in}}%
\pgfpathlineto{\pgfqpoint{0.584261in}{2.652699in}}%
\pgfpathlineto{\pgfqpoint{0.576045in}{2.608897in}}%
\pgfpathlineto{\pgfqpoint{0.568846in}{2.557282in}}%
\pgfpathlineto{\pgfqpoint{0.562852in}{2.497940in}}%
\pgfpathlineto{\pgfqpoint{0.557742in}{2.423496in}}%
\pgfpathlineto{\pgfqpoint{0.554210in}{2.338963in}}%
\pgfpathlineto{\pgfqpoint{0.552416in}{2.241908in}}%
\pgfpathlineto{\pgfqpoint{0.552652in}{2.137364in}}%
\pgfpathlineto{\pgfqpoint{0.555048in}{2.032856in}}%
\pgfpathlineto{\pgfqpoint{0.559675in}{1.928447in}}%
\pgfpathlineto{\pgfqpoint{0.566208in}{1.831659in}}%
\pgfpathlineto{\pgfqpoint{0.574080in}{1.747510in}}%
\pgfpathlineto{\pgfqpoint{0.583430in}{1.671095in}}%
\pgfpathlineto{\pgfqpoint{0.593449in}{1.607403in}}%
\pgfpathlineto{\pgfqpoint{0.604436in}{1.551554in}}%
\pgfpathlineto{\pgfqpoint{0.616084in}{1.503593in}}%
\pgfpathlineto{\pgfqpoint{0.627961in}{1.463525in}}%
\pgfpathlineto{\pgfqpoint{0.640477in}{1.429052in}}%
\pgfpathlineto{\pgfqpoint{0.652251in}{1.402402in}}%
\pgfpathlineto{\pgfqpoint{0.664862in}{1.379145in}}%
\pgfpathlineto{\pgfqpoint{0.676814in}{1.361414in}}%
\pgfpathlineto{\pgfqpoint{0.687550in}{1.349075in}}%
\pgfpathlineto{\pgfqpoint{0.697984in}{1.340130in}}%
\pgfpathlineto{\pgfqpoint{0.707697in}{1.334575in}}%
\pgfpathlineto{\pgfqpoint{0.716103in}{1.332081in}}%
\pgfpathlineto{\pgfqpoint{0.724776in}{1.332051in}}%
\pgfpathlineto{\pgfqpoint{0.733089in}{1.334868in}}%
\pgfpathlineto{\pgfqpoint{0.740401in}{1.340215in}}%
\pgfpathlineto{\pgfqpoint{0.746494in}{1.347301in}}%
\pgfpathlineto{\pgfqpoint{0.752596in}{1.357587in}}%
\pgfpathlineto{\pgfqpoint{0.759026in}{1.373363in}}%
\pgfpathlineto{\pgfqpoint{0.765376in}{1.397160in}}%
\pgfpathlineto{\pgfqpoint{0.770544in}{1.426431in}}%
\pgfpathlineto{\pgfqpoint{0.774934in}{1.465933in}}%
\pgfpathlineto{\pgfqpoint{0.778489in}{1.520538in}}%
\pgfpathlineto{\pgfqpoint{0.783538in}{1.647353in}}%
\pgfpathlineto{\pgfqpoint{0.789369in}{1.766646in}}%
\pgfpathlineto{\pgfqpoint{0.796122in}{1.858418in}}%
\pgfpathlineto{\pgfqpoint{0.803915in}{1.935063in}}%
\pgfpathlineto{\pgfqpoint{0.813064in}{2.003966in}}%
\pgfpathlineto{\pgfqpoint{0.823714in}{2.067523in}}%
\pgfpathlineto{\pgfqpoint{0.835717in}{2.125659in}}%
\pgfpathlineto{\pgfqpoint{0.849414in}{2.180718in}}%
\pgfpathlineto{\pgfqpoint{0.862593in}{2.225541in}}%
\pgfpathlineto{\pgfqpoint{0.876074in}{2.264941in}}%
\pgfpathlineto{\pgfqpoint{0.892182in}{2.305774in}}%
\pgfpathlineto{\pgfqpoint{0.909244in}{2.343311in}}%
\pgfpathlineto{\pgfqpoint{0.919674in}{2.362230in}}%
\pgfpathlineto{\pgfqpoint{0.948393in}{2.414845in}}%
\pgfpathlineto{\pgfqpoint{0.969944in}{2.446110in}}%
\pgfpathlineto{\pgfqpoint{0.990343in}{2.471975in}}%
\pgfpathlineto{\pgfqpoint{1.013516in}{2.498254in}}%
\pgfpathlineto{\pgfqpoint{1.039625in}{2.524572in}}%
\pgfpathlineto{\pgfqpoint{1.067274in}{2.548752in}}%
\pgfpathlineto{\pgfqpoint{1.080022in}{2.558206in}}%
\pgfpathlineto{\pgfqpoint{1.087468in}{2.563324in}}%
\pgfpathlineto{\pgfqpoint{1.115232in}{2.582932in}}%
\pgfpathlineto{\pgfqpoint{1.145764in}{2.602041in}}%
\pgfpathlineto{\pgfqpoint{1.179083in}{2.620389in}}%
\pgfpathlineto{\pgfqpoint{1.219203in}{2.639632in}}%
\pgfpathlineto{\pgfqpoint{1.258032in}{2.655828in}}%
\pgfpathlineto{\pgfqpoint{1.303680in}{2.672256in}}%
\pgfpathlineto{\pgfqpoint{1.320549in}{2.676952in}}%
\pgfpathlineto{\pgfqpoint{1.331109in}{2.679878in}}%
\pgfpathlineto{\pgfqpoint{1.386022in}{2.695356in}}%
\pgfpathlineto{\pgfqpoint{1.441429in}{2.708326in}}%
\pgfpathlineto{\pgfqpoint{1.492916in}{2.718088in}}%
\pgfpathlineto{\pgfqpoint{1.557513in}{2.728569in}}%
\pgfpathlineto{\pgfqpoint{1.630928in}{2.738822in}}%
\pgfpathlineto{\pgfqpoint{1.711034in}{2.747746in}}%
\pgfpathlineto{\pgfqpoint{1.799972in}{2.755366in}}%
\pgfpathlineto{\pgfqpoint{1.880341in}{2.760324in}}%
\pgfpathlineto{\pgfqpoint{1.989027in}{2.765032in}}%
\pgfpathlineto{\pgfqpoint{2.091249in}{2.767235in}}%
\pgfpathlineto{\pgfqpoint{2.197838in}{2.767481in}}%
\pgfpathlineto{\pgfqpoint{2.302238in}{2.765521in}}%
\pgfpathlineto{\pgfqpoint{2.397887in}{2.761525in}}%
\pgfpathlineto{\pgfqpoint{2.482576in}{2.755807in}}%
\pgfpathlineto{\pgfqpoint{2.556269in}{2.748635in}}%
\pgfpathlineto{\pgfqpoint{2.618932in}{2.740334in}}%
\pgfpathlineto{\pgfqpoint{2.670543in}{2.731348in}}%
\pgfpathlineto{\pgfqpoint{2.713244in}{2.721837in}}%
\pgfpathlineto{\pgfqpoint{2.749158in}{2.711767in}}%
\pgfpathlineto{\pgfqpoint{2.780340in}{2.700790in}}%
\pgfpathlineto{\pgfqpoint{2.806704in}{2.689105in}}%
\pgfpathlineto{\pgfqpoint{2.828204in}{2.677107in}}%
\pgfpathlineto{\pgfqpoint{2.844901in}{2.665427in}}%
\pgfpathlineto{\pgfqpoint{2.858710in}{2.653324in}}%
\pgfpathlineto{\pgfqpoint{2.871113in}{2.639377in}}%
\pgfpathlineto{\pgfqpoint{2.880416in}{2.625599in}}%
\pgfpathlineto{\pgfqpoint{2.887952in}{2.610476in}}%
\pgfpathlineto{\pgfqpoint{2.893557in}{2.594289in}}%
\pgfpathlineto{\pgfqpoint{2.897662in}{2.574954in}}%
\pgfpathlineto{\pgfqpoint{2.899750in}{2.552693in}}%
\pgfpathlineto{\pgfqpoint{2.899665in}{2.527811in}}%
\pgfpathlineto{\pgfqpoint{2.897047in}{2.495597in}}%
\pgfpathlineto{\pgfqpoint{2.892030in}{2.461230in}}%
\pgfpathlineto{\pgfqpoint{2.881852in}{2.407724in}}%
\pgfpathlineto{\pgfqpoint{2.865882in}{2.332756in}}%
\pgfpathlineto{\pgfqpoint{2.837460in}{2.197211in}}%
\pgfpathlineto{\pgfqpoint{2.819970in}{2.102221in}}%
\pgfpathlineto{\pgfqpoint{2.804678in}{2.006735in}}%
\pgfpathlineto{\pgfqpoint{2.793105in}{1.935983in}}%
\pgfpathlineto{\pgfqpoint{2.788318in}{1.899052in}}%
\pgfpathlineto{\pgfqpoint{2.775642in}{1.787983in}}%
\pgfpathlineto{\pgfqpoint{2.764079in}{1.666734in}}%
\pgfpathlineto{\pgfqpoint{2.754354in}{1.540277in}}%
\pgfpathlineto{\pgfqpoint{2.744858in}{1.391322in}}%
\pgfpathlineto{\pgfqpoint{2.736569in}{1.224817in}}%
\pgfpathlineto{\pgfqpoint{2.729619in}{1.038298in}}%
\pgfpathlineto{\pgfqpoint{2.724372in}{0.836763in}}%
\pgfpathlineto{\pgfqpoint{2.714680in}{0.446179in}}%
\pgfpathlineto{\pgfqpoint{2.711841in}{0.429077in}}%
\pgfpathlineto{\pgfqpoint{2.708551in}{0.419888in}}%
\pgfpathlineto{\pgfqpoint{2.704422in}{0.414151in}}%
\pgfpathlineto{\pgfqpoint{2.698880in}{0.410266in}}%
\pgfpathlineto{\pgfqpoint{2.690547in}{0.407461in}}%
\pgfpathlineto{\pgfqpoint{2.675443in}{0.405326in}}%
\pgfpathlineto{\pgfqpoint{2.645017in}{0.403904in}}%
\pgfpathlineto{\pgfqpoint{2.564535in}{0.403052in}}%
\pgfpathlineto{\pgfqpoint{2.218657in}{0.402676in}}%
\pgfpathlineto{\pgfqpoint{0.458817in}{0.403264in}}%
\pgfpathlineto{\pgfqpoint{0.450132in}{0.403603in}}%
\pgfpathlineto{\pgfqpoint{0.450132in}{0.403603in}}%
\pgfusepath{stroke}%
\end{pgfscope}%
\begin{pgfscope}%
\pgfpathrectangle{\pgfqpoint{0.448634in}{0.402556in}}{\pgfqpoint{4.350661in}{2.489204in}} %
\pgfusepath{clip}%
\pgfsetrectcap%
\pgfsetroundjoin%
\pgfsetlinewidth{1.003750pt}%
\definecolor{currentstroke}{rgb}{0.737255,0.741176,0.133333}%
\pgfsetstrokecolor{currentstroke}%
\pgfsetdash{}{0pt}%
\pgfpathmoveto{\pgfqpoint{4.798836in}{2.852368in}}%
\pgfpathlineto{\pgfqpoint{4.797563in}{2.889609in}}%
\pgfpathlineto{\pgfqpoint{4.796215in}{2.891483in}}%
\pgfpathlineto{\pgfqpoint{4.787551in}{2.891760in}}%
\pgfpathlineto{\pgfqpoint{0.452128in}{2.891658in}}%
\pgfpathlineto{\pgfqpoint{0.450531in}{2.890080in}}%
\pgfpathlineto{\pgfqpoint{0.449454in}{2.882761in}}%
\pgfpathlineto{\pgfqpoint{0.448969in}{2.845430in}}%
\pgfpathlineto{\pgfqpoint{0.448743in}{2.491963in}}%
\pgfpathlineto{\pgfqpoint{0.449614in}{0.610126in}}%
\pgfpathlineto{\pgfqpoint{0.451477in}{0.505608in}}%
\pgfpathlineto{\pgfqpoint{0.451700in}{0.500636in}}%
\pgfpathlineto{\pgfqpoint{0.451700in}{0.500636in}}%
\pgfusepath{stroke}%
\end{pgfscope}%
\begin{pgfscope}%
\pgfpathrectangle{\pgfqpoint{0.448634in}{0.402556in}}{\pgfqpoint{4.350661in}{2.489204in}} %
\pgfusepath{clip}%
\pgfsetrectcap%
\pgfsetroundjoin%
\pgfsetlinewidth{1.003750pt}%
\definecolor{currentstroke}{rgb}{0.737255,0.741176,0.133333}%
\pgfsetstrokecolor{currentstroke}%
\pgfsetdash{}{0pt}%
\pgfpathmoveto{\pgfqpoint{0.445604in}{0.402803in}}%
\pgfpathlineto{\pgfqpoint{0.454230in}{0.403544in}}%
\pgfpathlineto{\pgfqpoint{0.475976in}{0.403032in}}%
\pgfpathlineto{\pgfqpoint{0.676106in}{0.402678in}}%
\pgfpathlineto{\pgfqpoint{2.572993in}{0.403112in}}%
\pgfpathlineto{\pgfqpoint{2.664342in}{0.404671in}}%
\pgfpathlineto{\pgfqpoint{2.686011in}{0.406738in}}%
\pgfpathlineto{\pgfqpoint{2.696595in}{0.409527in}}%
\pgfpathlineto{\pgfqpoint{2.702432in}{0.412817in}}%
\pgfpathlineto{\pgfqpoint{2.707100in}{0.417983in}}%
\pgfpathlineto{\pgfqpoint{2.710970in}{0.426857in}}%
\pgfpathlineto{\pgfqpoint{2.713897in}{0.441394in}}%
\pgfpathlineto{\pgfqpoint{2.716174in}{0.466142in}}%
\pgfpathlineto{\pgfqpoint{2.718266in}{0.518357in}}%
\pgfpathlineto{\pgfqpoint{2.720978in}{0.660207in}}%
\pgfpathlineto{\pgfqpoint{2.726860in}{0.941405in}}%
\pgfpathlineto{\pgfqpoint{2.733494in}{1.150358in}}%
\pgfpathlineto{\pgfqpoint{2.741353in}{1.326861in}}%
\pgfpathlineto{\pgfqpoint{2.750588in}{1.485817in}}%
\pgfpathlineto{\pgfqpoint{2.760961in}{1.627203in}}%
\pgfpathlineto{\pgfqpoint{2.773834in}{1.770814in}}%
\pgfpathlineto{\pgfqpoint{2.786777in}{1.886862in}}%
\pgfpathlineto{\pgfqpoint{2.793775in}{1.938475in}}%
\pgfpathlineto{\pgfqpoint{2.795905in}{1.945466in}}%
\pgfpathlineto{\pgfqpoint{2.810622in}{2.046121in}}%
\pgfpathlineto{\pgfqpoint{2.826806in}{2.141414in}}%
\pgfpathlineto{\pgfqpoint{2.845159in}{2.236192in}}%
\pgfpathlineto{\pgfqpoint{2.871559in}{2.359493in}}%
\pgfpathlineto{\pgfqpoint{2.889746in}{2.449205in}}%
\pgfpathlineto{\pgfqpoint{2.896575in}{2.493317in}}%
\pgfpathlineto{\pgfqpoint{2.899467in}{2.528000in}}%
\pgfpathlineto{\pgfqpoint{2.899519in}{2.552881in}}%
\pgfpathlineto{\pgfqpoint{2.897392in}{2.575137in}}%
\pgfpathlineto{\pgfqpoint{2.893247in}{2.594461in}}%
\pgfpathlineto{\pgfqpoint{2.887607in}{2.610632in}}%
\pgfpathlineto{\pgfqpoint{2.880038in}{2.625734in}}%
\pgfpathlineto{\pgfqpoint{2.870708in}{2.639488in}}%
\pgfpathlineto{\pgfqpoint{2.858282in}{2.653408in}}%
\pgfpathlineto{\pgfqpoint{2.844457in}{2.665487in}}%
\pgfpathlineto{\pgfqpoint{2.827748in}{2.677146in}}%
\pgfpathlineto{\pgfqpoint{2.808232in}{2.688124in}}%
\pgfpathlineto{\pgfqpoint{2.783976in}{2.699144in}}%
\pgfpathlineto{\pgfqpoint{2.754968in}{2.709739in}}%
\pgfpathlineto{\pgfqpoint{2.721259in}{2.719635in}}%
\pgfpathlineto{\pgfqpoint{2.680774in}{2.729137in}}%
\pgfpathlineto{\pgfqpoint{2.633536in}{2.737900in}}%
\pgfpathlineto{\pgfqpoint{2.577424in}{2.745990in}}%
\pgfpathlineto{\pgfqpoint{2.510293in}{2.753287in}}%
\pgfpathlineto{\pgfqpoint{2.434337in}{2.759257in}}%
\pgfpathlineto{\pgfqpoint{2.347418in}{2.763839in}}%
\pgfpathlineto{\pgfqpoint{2.249562in}{2.766715in}}%
\pgfpathlineto{\pgfqpoint{2.145150in}{2.767555in}}%
\pgfpathlineto{\pgfqpoint{2.038566in}{2.766228in}}%
\pgfpathlineto{\pgfqpoint{1.927672in}{2.762596in}}%
\pgfpathlineto{\pgfqpoint{1.827727in}{2.757038in}}%
\pgfpathlineto{\pgfqpoint{1.725730in}{2.749065in}}%
\pgfpathlineto{\pgfqpoint{1.641265in}{2.739996in}}%
\pgfpathlineto{\pgfqpoint{1.565663in}{2.729711in}}%
\pgfpathlineto{\pgfqpoint{1.470933in}{2.714259in}}%
\pgfpathlineto{\pgfqpoint{1.413210in}{2.701862in}}%
\pgfpathlineto{\pgfqpoint{1.360146in}{2.688257in}}%
\pgfpathlineto{\pgfqpoint{1.280324in}{2.663970in}}%
\pgfpathlineto{\pgfqpoint{1.228935in}{2.643629in}}%
\pgfpathlineto{\pgfqpoint{1.190623in}{2.625898in}}%
\pgfpathlineto{\pgfqpoint{1.155083in}{2.607106in}}%
\pgfpathlineto{\pgfqpoint{1.122396in}{2.587325in}}%
\pgfpathlineto{\pgfqpoint{1.090766in}{2.565411in}}%
\pgfpathlineto{\pgfqpoint{1.051245in}{2.534717in}}%
\pgfpathlineto{\pgfqpoint{1.024423in}{2.509348in}}%
\pgfpathlineto{\pgfqpoint{1.002292in}{2.485414in}}%
\pgfpathlineto{\pgfqpoint{0.978274in}{2.456597in}}%
\pgfpathlineto{\pgfqpoint{0.956114in}{2.425894in}}%
\pgfpathlineto{\pgfqpoint{0.937078in}{2.395592in}}%
\pgfpathlineto{\pgfqpoint{0.933531in}{2.386577in}}%
\pgfpathlineto{\pgfqpoint{0.909282in}{2.342317in}}%
\pgfpathlineto{\pgfqpoint{0.892275in}{2.304747in}}%
\pgfpathlineto{\pgfqpoint{0.876220in}{2.263887in}}%
\pgfpathlineto{\pgfqpoint{0.862105in}{2.222102in}}%
\pgfpathlineto{\pgfqpoint{0.845303in}{2.162928in}}%
\pgfpathlineto{\pgfqpoint{0.832819in}{2.110066in}}%
\pgfpathlineto{\pgfqpoint{0.821424in}{2.051770in}}%
\pgfpathlineto{\pgfqpoint{0.811329in}{1.988094in}}%
\pgfpathlineto{\pgfqpoint{0.802973in}{1.921573in}}%
\pgfpathlineto{\pgfqpoint{0.795595in}{1.844874in}}%
\pgfpathlineto{\pgfqpoint{0.789549in}{1.758030in}}%
\pgfpathlineto{\pgfqpoint{0.784641in}{1.653636in}}%
\pgfpathlineto{\pgfqpoint{0.776094in}{1.459753in}}%
\pgfpathlineto{\pgfqpoint{0.771437in}{1.417782in}}%
\pgfpathlineto{\pgfqpoint{0.766133in}{1.388545in}}%
\pgfpathlineto{\pgfqpoint{0.759715in}{1.364782in}}%
\pgfpathlineto{\pgfqpoint{0.753003in}{1.349163in}}%
\pgfpathlineto{\pgfqpoint{0.746619in}{1.339107in}}%
\pgfpathlineto{\pgfqpoint{0.740252in}{1.332345in}}%
\pgfpathlineto{\pgfqpoint{0.732684in}{1.327493in}}%
\pgfpathlineto{\pgfqpoint{0.724240in}{1.325238in}}%
\pgfpathlineto{\pgfqpoint{0.715576in}{1.325761in}}%
\pgfpathlineto{\pgfqpoint{0.707248in}{1.328582in}}%
\pgfpathlineto{\pgfqpoint{0.697638in}{1.334375in}}%
\pgfpathlineto{\pgfqpoint{0.687290in}{1.343453in}}%
\pgfpathlineto{\pgfqpoint{0.676605in}{1.355852in}}%
\pgfpathlineto{\pgfqpoint{0.666031in}{1.371651in}}%
\pgfpathlineto{\pgfqpoint{0.654295in}{1.392599in}}%
\pgfpathlineto{\pgfqpoint{0.642194in}{1.419058in}}%
\pgfpathlineto{\pgfqpoint{0.630141in}{1.451054in}}%
\pgfpathlineto{\pgfqpoint{0.618499in}{1.488580in}}%
\pgfpathlineto{\pgfqpoint{0.606358in}{1.536380in}}%
\pgfpathlineto{\pgfqpoint{0.595809in}{1.587236in}}%
\pgfpathlineto{\pgfqpoint{0.585931in}{1.645894in}}%
\pgfpathlineto{\pgfqpoint{0.577349in}{1.709859in}}%
\pgfpathlineto{\pgfqpoint{0.569193in}{1.786455in}}%
\pgfpathlineto{\pgfqpoint{0.562390in}{1.870726in}}%
\pgfpathlineto{\pgfqpoint{0.556971in}{1.965110in}}%
\pgfpathlineto{\pgfqpoint{0.553370in}{2.062099in}}%
\pgfpathlineto{\pgfqpoint{0.551524in}{2.171601in}}%
\pgfpathlineto{\pgfqpoint{0.552086in}{2.278632in}}%
\pgfpathlineto{\pgfqpoint{0.554602in}{2.375664in}}%
\pgfpathlineto{\pgfqpoint{0.559050in}{2.460140in}}%
\pgfpathlineto{\pgfqpoint{0.565052in}{2.531995in}}%
\pgfpathlineto{\pgfqpoint{0.571905in}{2.588702in}}%
\pgfpathlineto{\pgfqpoint{0.580041in}{2.637601in}}%
\pgfpathlineto{\pgfqpoint{0.589164in}{2.678602in}}%
\pgfpathlineto{\pgfqpoint{0.598957in}{2.711590in}}%
\pgfpathlineto{\pgfqpoint{0.609488in}{2.738909in}}%
\pgfpathlineto{\pgfqpoint{0.620157in}{2.760591in}}%
\pgfpathlineto{\pgfqpoint{0.631469in}{2.778862in}}%
\pgfpathlineto{\pgfqpoint{0.644509in}{2.795554in}}%
\pgfpathlineto{\pgfqpoint{0.657498in}{2.808793in}}%
\pgfpathlineto{\pgfqpoint{0.673475in}{2.821718in}}%
\pgfpathlineto{\pgfqpoint{0.690598in}{2.832561in}}%
\pgfpathlineto{\pgfqpoint{0.708569in}{2.841430in}}%
\pgfpathlineto{\pgfqpoint{0.729199in}{2.849304in}}%
\pgfpathlineto{\pgfqpoint{0.756598in}{2.857285in}}%
\pgfpathlineto{\pgfqpoint{0.786521in}{2.863742in}}%
\pgfpathlineto{\pgfqpoint{0.825322in}{2.869736in}}%
\pgfpathlineto{\pgfqpoint{0.885988in}{2.875902in}}%
\pgfpathlineto{\pgfqpoint{0.953316in}{2.880217in}}%
\pgfpathlineto{\pgfqpoint{1.042455in}{2.883573in}}%
\pgfpathlineto{\pgfqpoint{1.179476in}{2.886467in}}%
\pgfpathlineto{\pgfqpoint{1.390475in}{2.888541in}}%
\pgfpathlineto{\pgfqpoint{1.773331in}{2.890080in}}%
\pgfpathlineto{\pgfqpoint{2.667391in}{2.890819in}}%
\pgfpathlineto{\pgfqpoint{3.852945in}{2.889859in}}%
\pgfpathlineto{\pgfqpoint{4.231445in}{2.887281in}}%
\pgfpathlineto{\pgfqpoint{4.381517in}{2.884202in}}%
\pgfpathlineto{\pgfqpoint{4.468465in}{2.880396in}}%
\pgfpathlineto{\pgfqpoint{4.527055in}{2.875752in}}%
\pgfpathlineto{\pgfqpoint{4.568105in}{2.870289in}}%
\pgfpathlineto{\pgfqpoint{4.598080in}{2.864169in}}%
\pgfpathlineto{\pgfqpoint{4.621240in}{2.857316in}}%
\pgfpathlineto{\pgfqpoint{4.639658in}{2.849747in}}%
\pgfpathlineto{\pgfqpoint{4.655304in}{2.841053in}}%
\pgfpathlineto{\pgfqpoint{4.668106in}{2.831638in}}%
\pgfpathlineto{\pgfqpoint{4.679756in}{2.820438in}}%
\pgfpathlineto{\pgfqpoint{4.690016in}{2.807579in}}%
\pgfpathlineto{\pgfqpoint{4.699948in}{2.791244in}}%
\pgfpathlineto{\pgfqpoint{4.709040in}{2.771418in}}%
\pgfpathlineto{\pgfqpoint{4.717034in}{2.748278in}}%
\pgfpathlineto{\pgfqpoint{4.724468in}{2.719652in}}%
\pgfpathlineto{\pgfqpoint{4.732255in}{2.678293in}}%
\pgfpathlineto{\pgfqpoint{4.738601in}{2.631563in}}%
\pgfpathlineto{\pgfqpoint{4.744580in}{2.569714in}}%
\pgfpathlineto{\pgfqpoint{4.749496in}{2.495254in}}%
\pgfpathlineto{\pgfqpoint{4.753843in}{2.395812in}}%
\pgfpathlineto{\pgfqpoint{4.757039in}{2.268917in}}%
\pgfpathlineto{\pgfqpoint{4.758868in}{2.107134in}}%
\pgfpathlineto{\pgfqpoint{4.758782in}{1.917956in}}%
\pgfpathlineto{\pgfqpoint{4.756585in}{1.733773in}}%
\pgfpathlineto{\pgfqpoint{4.752288in}{1.562091in}}%
\pgfpathlineto{\pgfqpoint{4.746248in}{1.415396in}}%
\pgfpathlineto{\pgfqpoint{4.739256in}{1.303671in}}%
\pgfpathlineto{\pgfqpoint{4.730547in}{1.202108in}}%
\pgfpathlineto{\pgfqpoint{4.725729in}{1.157645in}}%
\pgfpathlineto{\pgfqpoint{4.716737in}{1.091232in}}%
\pgfpathlineto{\pgfqpoint{4.706704in}{1.032609in}}%
\pgfpathlineto{\pgfqpoint{4.695890in}{0.981827in}}%
\pgfpathlineto{\pgfqpoint{4.684057in}{0.936518in}}%
\pgfpathlineto{\pgfqpoint{4.671393in}{0.896768in}}%
\pgfpathlineto{\pgfqpoint{4.658249in}{0.862601in}}%
\pgfpathlineto{\pgfqpoint{4.644031in}{0.831792in}}%
\pgfpathlineto{\pgfqpoint{4.628962in}{0.804419in}}%
\pgfpathlineto{\pgfqpoint{4.613311in}{0.780523in}}%
\pgfpathlineto{\pgfqpoint{4.595996in}{0.758181in}}%
\pgfpathlineto{\pgfqpoint{4.577097in}{0.737589in}}%
\pgfpathlineto{\pgfqpoint{4.556775in}{0.718855in}}%
\pgfpathlineto{\pgfqpoint{4.535229in}{0.702003in}}%
\pgfpathlineto{\pgfqpoint{4.510761in}{0.685794in}}%
\pgfpathlineto{\pgfqpoint{4.485367in}{0.671566in}}%
\pgfpathlineto{\pgfqpoint{4.457218in}{0.658281in}}%
\pgfpathlineto{\pgfqpoint{4.426379in}{0.646100in}}%
\pgfpathlineto{\pgfqpoint{4.390858in}{0.634340in}}%
\pgfpathlineto{\pgfqpoint{4.361268in}{0.626109in}}%
\pgfpathlineto{\pgfqpoint{4.322855in}{0.617468in}}%
\pgfpathlineto{\pgfqpoint{4.275616in}{0.608711in}}%
\pgfpathlineto{\pgfqpoint{4.223809in}{0.601346in}}%
\pgfpathlineto{\pgfqpoint{4.169641in}{0.595844in}}%
\pgfpathlineto{\pgfqpoint{4.100157in}{0.591105in}}%
\pgfpathlineto{\pgfqpoint{3.980546in}{0.588453in}}%
\pgfpathlineto{\pgfqpoint{3.908771in}{0.589781in}}%
\pgfpathlineto{\pgfqpoint{3.834879in}{0.593352in}}%
\pgfpathlineto{\pgfqpoint{3.765446in}{0.599007in}}%
\pgfpathlineto{\pgfqpoint{3.702683in}{0.606248in}}%
\pgfpathlineto{\pgfqpoint{3.644510in}{0.615497in}}%
\pgfpathlineto{\pgfqpoint{3.586640in}{0.626925in}}%
\pgfpathlineto{\pgfqpoint{3.537760in}{0.639129in}}%
\pgfpathlineto{\pgfqpoint{3.493567in}{0.652349in}}%
\pgfpathlineto{\pgfqpoint{3.451996in}{0.667019in}}%
\pgfpathlineto{\pgfqpoint{3.411075in}{0.683907in}}%
\pgfpathlineto{\pgfqpoint{3.377068in}{0.700513in}}%
\pgfpathlineto{\pgfqpoint{3.355665in}{0.712738in}}%
\pgfpathlineto{\pgfqpoint{3.325016in}{0.731593in}}%
\pgfpathlineto{\pgfqpoint{3.295495in}{0.752670in}}%
\pgfpathlineto{\pgfqpoint{3.270852in}{0.773134in}}%
\pgfpathlineto{\pgfqpoint{3.247486in}{0.795476in}}%
\pgfpathlineto{\pgfqpoint{3.227108in}{0.817901in}}%
\pgfpathlineto{\pgfqpoint{3.208151in}{0.841906in}}%
\pgfpathlineto{\pgfqpoint{3.190698in}{0.867359in}}%
\pgfpathlineto{\pgfqpoint{3.173641in}{0.896220in}}%
\pgfpathlineto{\pgfqpoint{3.158389in}{0.926376in}}%
\pgfpathlineto{\pgfqpoint{3.144989in}{0.957663in}}%
\pgfpathlineto{\pgfqpoint{3.130964in}{0.996807in}}%
\pgfpathlineto{\pgfqpoint{3.116016in}{1.048816in}}%
\pgfpathlineto{\pgfqpoint{3.106283in}{1.092209in}}%
\pgfpathlineto{\pgfqpoint{3.098211in}{1.138588in}}%
\pgfpathlineto{\pgfqpoint{3.091928in}{1.187845in}}%
\pgfpathlineto{\pgfqpoint{3.087556in}{1.239873in}}%
\pgfpathlineto{\pgfqpoint{3.085195in}{1.294564in}}%
\pgfpathlineto{\pgfqpoint{3.084970in}{1.351811in}}%
\pgfpathlineto{\pgfqpoint{3.086919in}{1.409014in}}%
\pgfpathlineto{\pgfqpoint{3.091165in}{1.468553in}}%
\pgfpathlineto{\pgfqpoint{3.097662in}{1.527824in}}%
\pgfpathlineto{\pgfqpoint{3.106443in}{1.586709in}}%
\pgfpathlineto{\pgfqpoint{3.117112in}{1.642638in}}%
\pgfpathlineto{\pgfqpoint{3.129500in}{1.695529in}}%
\pgfpathlineto{\pgfqpoint{3.143469in}{1.745293in}}%
\pgfpathlineto{\pgfqpoint{3.158831in}{1.791864in}}%
\pgfpathlineto{\pgfqpoint{3.175403in}{1.835185in}}%
\pgfpathlineto{\pgfqpoint{3.194004in}{1.877413in}}%
\pgfpathlineto{\pgfqpoint{3.213551in}{1.916230in}}%
\pgfpathlineto{\pgfqpoint{3.234934in}{1.953758in}}%
\pgfpathlineto{\pgfqpoint{3.258142in}{1.989837in}}%
\pgfpathlineto{\pgfqpoint{3.260829in}{1.993753in}}%
\pgfpathlineto{\pgfqpoint{3.260829in}{1.993753in}}%
\pgfusepath{stroke}%
\end{pgfscope}%
\begin{pgfscope}%
\pgfpathrectangle{\pgfqpoint{0.448634in}{0.402556in}}{\pgfqpoint{4.350661in}{2.489204in}} %
\pgfusepath{clip}%
\pgfsetrectcap%
\pgfsetroundjoin%
\pgfsetlinewidth{1.003750pt}%
\definecolor{currentstroke}{rgb}{0.737255,0.741176,0.133333}%
\pgfsetstrokecolor{currentstroke}%
\pgfsetdash{}{0pt}%
\pgfpathmoveto{\pgfqpoint{2.763303in}{2.149319in}}%
\pgfpathlineto{\pgfqpoint{2.735963in}{1.827242in}}%
\pgfpathlineto{\pgfqpoint{2.722658in}{1.636181in}}%
\pgfpathlineto{\pgfqpoint{2.709151in}{1.405202in}}%
\pgfpathlineto{\pgfqpoint{2.688092in}{1.030110in}}%
\pgfpathlineto{\pgfqpoint{2.679352in}{0.918546in}}%
\pgfpathlineto{\pgfqpoint{2.670349in}{0.832039in}}%
\pgfpathlineto{\pgfqpoint{2.660483in}{0.758225in}}%
\pgfpathlineto{\pgfqpoint{2.650453in}{0.699603in}}%
\pgfpathlineto{\pgfqpoint{2.640893in}{0.656160in}}%
\pgfpathlineto{\pgfqpoint{2.630829in}{0.620651in}}%
\pgfpathlineto{\pgfqpoint{2.620840in}{0.593062in}}%
\pgfpathlineto{\pgfqpoint{2.609738in}{0.568820in}}%
\pgfpathlineto{\pgfqpoint{2.598936in}{0.550148in}}%
\pgfpathlineto{\pgfqpoint{2.586449in}{0.532908in}}%
\pgfpathlineto{\pgfqpoint{2.572305in}{0.517437in}}%
\pgfpathlineto{\pgfqpoint{2.556698in}{0.503931in}}%
\pgfpathlineto{\pgfqpoint{2.539924in}{0.492396in}}%
\pgfpathlineto{\pgfqpoint{2.520294in}{0.481689in}}%
\pgfpathlineto{\pgfqpoint{2.497894in}{0.472082in}}%
\pgfpathlineto{\pgfqpoint{2.470747in}{0.463040in}}%
\pgfpathlineto{\pgfqpoint{2.438901in}{0.454928in}}%
\pgfpathlineto{\pgfqpoint{2.400288in}{0.447521in}}%
\pgfpathlineto{\pgfqpoint{2.352795in}{0.440802in}}%
\pgfpathlineto{\pgfqpoint{2.294295in}{0.434836in}}%
\pgfpathlineto{\pgfqpoint{2.218307in}{0.429424in}}%
\pgfpathlineto{\pgfqpoint{2.118330in}{0.424675in}}%
\pgfpathlineto{\pgfqpoint{1.983506in}{0.420654in}}%
\pgfpathlineto{\pgfqpoint{1.798624in}{0.417503in}}%
\pgfpathlineto{\pgfqpoint{1.544117in}{0.415510in}}%
\pgfpathlineto{\pgfqpoint{1.211292in}{0.415228in}}%
\pgfpathlineto{\pgfqpoint{0.967661in}{0.417104in}}%
\pgfpathlineto{\pgfqpoint{0.787137in}{0.420610in}}%
\pgfpathlineto{\pgfqpoint{0.687136in}{0.424673in}}%
\pgfpathlineto{\pgfqpoint{0.624181in}{0.429235in}}%
\pgfpathlineto{\pgfqpoint{0.580915in}{0.434424in}}%
\pgfpathlineto{\pgfqpoint{0.550875in}{0.440111in}}%
\pgfpathlineto{\pgfqpoint{0.529775in}{0.446131in}}%
\pgfpathlineto{\pgfqpoint{0.513434in}{0.452943in}}%
\pgfpathlineto{\pgfqpoint{0.500003in}{0.461115in}}%
\pgfpathlineto{\pgfqpoint{0.489697in}{0.470248in}}%
\pgfpathlineto{\pgfqpoint{0.482370in}{0.479427in}}%
\pgfpathlineto{\pgfqpoint{0.475334in}{0.491984in}}%
\pgfpathlineto{\pgfqpoint{0.469334in}{0.507981in}}%
\pgfpathlineto{\pgfqpoint{0.464620in}{0.527140in}}%
\pgfpathlineto{\pgfqpoint{0.460442in}{0.554092in}}%
\pgfpathlineto{\pgfqpoint{0.456885in}{0.593704in}}%
\pgfpathlineto{\pgfqpoint{0.454085in}{0.653356in}}%
\pgfpathlineto{\pgfqpoint{0.451927in}{0.752891in}}%
\pgfpathlineto{\pgfqpoint{0.450409in}{0.942062in}}%
\pgfpathlineto{\pgfqpoint{0.449469in}{1.405052in}}%
\pgfpathlineto{\pgfqpoint{0.449544in}{2.579957in}}%
\pgfpathlineto{\pgfqpoint{0.450955in}{2.838825in}}%
\pgfpathlineto{\pgfqpoint{0.452801in}{2.873593in}}%
\pgfpathlineto{\pgfqpoint{0.454976in}{2.883198in}}%
\pgfpathlineto{\pgfqpoint{0.457631in}{2.887092in}}%
\pgfpathlineto{\pgfqpoint{0.461522in}{2.889247in}}%
\pgfpathlineto{\pgfqpoint{0.470113in}{2.890712in}}%
\pgfpathlineto{\pgfqpoint{0.494029in}{2.891493in}}%
\pgfpathlineto{\pgfqpoint{0.624548in}{2.891742in}}%
\pgfpathlineto{\pgfqpoint{4.779428in}{2.891527in}}%
\pgfpathlineto{\pgfqpoint{4.792415in}{2.890342in}}%
\pgfpathlineto{\pgfqpoint{4.795944in}{2.887712in}}%
\pgfpathlineto{\pgfqpoint{4.797235in}{2.880425in}}%
\pgfpathlineto{\pgfqpoint{4.798039in}{2.858045in}}%
\pgfpathlineto{\pgfqpoint{4.798039in}{2.858045in}}%
\pgfusepath{stroke}%
\end{pgfscope}%
\begin{pgfscope}%
\pgfpathrectangle{\pgfqpoint{0.448634in}{0.402556in}}{\pgfqpoint{4.350661in}{2.489204in}} %
\pgfusepath{clip}%
\pgfsetrectcap%
\pgfsetroundjoin%
\pgfsetlinewidth{1.003750pt}%
\definecolor{currentstroke}{rgb}{0.737255,0.741176,0.133333}%
\pgfsetstrokecolor{currentstroke}%
\pgfsetdash{}{0pt}%
\pgfpathmoveto{\pgfqpoint{2.791360in}{1.953249in}}%
\pgfpathlineto{\pgfqpoint{2.778019in}{1.842282in}}%
\pgfpathlineto{\pgfqpoint{2.765771in}{1.721122in}}%
\pgfpathlineto{\pgfqpoint{2.754591in}{1.587317in}}%
\pgfpathlineto{\pgfqpoint{2.744806in}{1.443379in}}%
\pgfpathlineto{\pgfqpoint{2.736049in}{1.281893in}}%
\pgfpathlineto{\pgfqpoint{2.728213in}{1.095420in}}%
\pgfpathlineto{\pgfqpoint{2.720967in}{0.866564in}}%
\pgfpathlineto{\pgfqpoint{2.711437in}{0.563085in}}%
\pgfpathlineto{\pgfqpoint{2.707400in}{0.506026in}}%
\pgfpathlineto{\pgfqpoint{2.703154in}{0.474044in}}%
\pgfpathlineto{\pgfqpoint{2.698672in}{0.454817in}}%
\pgfpathlineto{\pgfqpoint{2.693350in}{0.441204in}}%
\pgfpathlineto{\pgfqpoint{2.686870in}{0.431246in}}%
\pgfpathlineto{\pgfqpoint{2.680217in}{0.424857in}}%
\pgfpathlineto{\pgfqpoint{2.670601in}{0.419092in}}%
\pgfpathlineto{\pgfqpoint{2.658147in}{0.414669in}}%
\pgfpathlineto{\pgfqpoint{2.641019in}{0.411192in}}%
\pgfpathlineto{\pgfqpoint{2.615035in}{0.408391in}}%
\pgfpathlineto{\pgfqpoint{2.571573in}{0.406188in}}%
\pgfpathlineto{\pgfqpoint{2.488924in}{0.404557in}}%
\pgfpathlineto{\pgfqpoint{2.299673in}{0.403553in}}%
\pgfpathlineto{\pgfqpoint{1.620970in}{0.402974in}}%
\pgfpathlineto{\pgfqpoint{0.557235in}{0.403891in}}%
\pgfpathlineto{\pgfqpoint{0.476767in}{0.405666in}}%
\pgfpathlineto{\pgfqpoint{0.459436in}{0.407301in}}%
\pgfpathlineto{\pgfqpoint{0.453333in}{0.409788in}}%
\pgfpathlineto{\pgfqpoint{0.450748in}{0.413678in}}%
\pgfpathlineto{\pgfqpoint{0.449350in}{0.423470in}}%
\pgfpathlineto{\pgfqpoint{0.448804in}{0.458311in}}%
\pgfpathlineto{\pgfqpoint{0.448644in}{0.754526in}}%
\pgfpathlineto{\pgfqpoint{0.448676in}{2.890263in}}%
\pgfpathlineto{\pgfqpoint{0.448676in}{2.890263in}}%
\pgfusepath{stroke}%
\end{pgfscope}%
\begin{pgfscope}%
\pgfpathrectangle{\pgfqpoint{0.448634in}{0.402556in}}{\pgfqpoint{4.350661in}{2.489204in}} %
\pgfusepath{clip}%
\pgfsetrectcap%
\pgfsetroundjoin%
\pgfsetlinewidth{1.003750pt}%
\definecolor{currentstroke}{rgb}{0.737255,0.741176,0.133333}%
\pgfsetstrokecolor{currentstroke}%
\pgfsetdash{}{0pt}%
\pgfpathmoveto{\pgfqpoint{3.431449in}{0.402556in}}%
\pgfpathlineto{\pgfqpoint{0.449071in}{0.402556in}}%
\pgfpathlineto{\pgfqpoint{0.449071in}{0.402556in}}%
\pgfusepath{stroke}%
\end{pgfscope}%
\begin{pgfscope}%
\pgfpathrectangle{\pgfqpoint{0.448634in}{0.402556in}}{\pgfqpoint{4.350661in}{2.489204in}} %
\pgfusepath{clip}%
\pgfsetrectcap%
\pgfsetroundjoin%
\pgfsetlinewidth{1.003750pt}%
\definecolor{currentstroke}{rgb}{0.737255,0.741176,0.133333}%
\pgfsetstrokecolor{currentstroke}%
\pgfsetdash{}{0pt}%
\pgfpathmoveto{\pgfqpoint{0.456907in}{2.376067in}}%
\pgfpathlineto{\pgfqpoint{0.460548in}{2.649843in}}%
\pgfpathlineto{\pgfqpoint{0.464079in}{2.741851in}}%
\pgfpathlineto{\pgfqpoint{0.467958in}{2.791431in}}%
\pgfpathlineto{\pgfqpoint{0.472464in}{2.823366in}}%
\pgfpathlineto{\pgfqpoint{0.477082in}{2.842552in}}%
\pgfpathlineto{\pgfqpoint{0.482433in}{2.856152in}}%
\pgfpathlineto{\pgfqpoint{0.488841in}{2.866172in}}%
\pgfpathlineto{\pgfqpoint{0.495412in}{2.872674in}}%
\pgfpathlineto{\pgfqpoint{0.503013in}{2.877486in}}%
\pgfpathlineto{\pgfqpoint{0.513269in}{2.881590in}}%
\pgfpathlineto{\pgfqpoint{0.528185in}{2.885042in}}%
\pgfpathlineto{\pgfqpoint{0.551983in}{2.887843in}}%
\pgfpathlineto{\pgfqpoint{0.591101in}{2.889732in}}%
\pgfpathlineto{\pgfqpoint{0.678106in}{2.891031in}}%
\pgfpathlineto{\pgfqpoint{0.880410in}{2.891630in}}%
\pgfpathlineto{\pgfqpoint{2.459700in}{2.891752in}}%
\pgfpathlineto{\pgfqpoint{4.715513in}{2.890688in}}%
\pgfpathlineto{\pgfqpoint{4.750267in}{2.888727in}}%
\pgfpathlineto{\pgfqpoint{4.763124in}{2.886238in}}%
\pgfpathlineto{\pgfqpoint{4.771207in}{2.882625in}}%
\pgfpathlineto{\pgfqpoint{4.776343in}{2.878061in}}%
\pgfpathlineto{\pgfqpoint{4.780134in}{2.872009in}}%
\pgfpathlineto{\pgfqpoint{4.783383in}{2.862790in}}%
\pgfpathlineto{\pgfqpoint{4.786545in}{2.845761in}}%
\pgfpathlineto{\pgfqpoint{4.789205in}{2.816054in}}%
\pgfpathlineto{\pgfqpoint{4.791421in}{2.761353in}}%
\pgfpathlineto{\pgfqpoint{4.793143in}{2.656827in}}%
\pgfpathlineto{\pgfqpoint{4.794393in}{2.425336in}}%
\pgfpathlineto{\pgfqpoint{4.794729in}{1.949899in}}%
\pgfpathlineto{\pgfqpoint{4.793198in}{1.511803in}}%
\pgfpathlineto{\pgfqpoint{4.789904in}{1.190720in}}%
\pgfpathlineto{\pgfqpoint{4.786136in}{1.021511in}}%
\pgfpathlineto{\pgfqpoint{4.781664in}{0.907124in}}%
\pgfpathlineto{\pgfqpoint{4.776495in}{0.825197in}}%
\pgfpathlineto{\pgfqpoint{4.770490in}{0.763353in}}%
\pgfpathlineto{\pgfqpoint{4.763822in}{0.716684in}}%
\pgfpathlineto{\pgfqpoint{4.756520in}{0.680301in}}%
\pgfpathlineto{\pgfqpoint{4.748082in}{0.649426in}}%
\pgfpathlineto{\pgfqpoint{4.738873in}{0.624165in}}%
\pgfpathlineto{\pgfqpoint{4.728439in}{0.602339in}}%
\pgfpathlineto{\pgfqpoint{4.717138in}{0.584060in}}%
\pgfpathlineto{\pgfqpoint{4.705535in}{0.569233in}}%
\pgfpathlineto{\pgfqpoint{4.692573in}{0.555960in}}%
\pgfpathlineto{\pgfqpoint{4.676629in}{0.542979in}}%
\pgfpathlineto{\pgfqpoint{4.659586in}{0.531972in}}%
\pgfpathlineto{\pgfqpoint{4.639750in}{0.521772in}}%
\pgfpathlineto{\pgfqpoint{4.615144in}{0.511821in}}%
\pgfpathlineto{\pgfqpoint{4.585780in}{0.502604in}}%
\pgfpathlineto{\pgfqpoint{4.551736in}{0.494339in}}%
\pgfpathlineto{\pgfqpoint{4.510948in}{0.486718in}}%
\pgfpathlineto{\pgfqpoint{4.459131in}{0.479450in}}%
\pgfpathlineto{\pgfqpoint{4.398463in}{0.473285in}}%
\pgfpathlineto{\pgfqpoint{4.324641in}{0.468138in}}%
\pgfpathlineto{\pgfqpoint{4.235518in}{0.464246in}}%
\pgfpathlineto{\pgfqpoint{4.126775in}{0.461742in}}%
\pgfpathlineto{\pgfqpoint{3.994083in}{0.461001in}}%
\pgfpathlineto{\pgfqpoint{3.859219in}{0.462414in}}%
\pgfpathlineto{\pgfqpoint{3.737438in}{0.465817in}}%
\pgfpathlineto{\pgfqpoint{3.646166in}{0.470451in}}%
\pgfpathlineto{\pgfqpoint{3.554992in}{0.477163in}}%
\pgfpathlineto{\pgfqpoint{3.476987in}{0.485057in}}%
\pgfpathlineto{\pgfqpoint{3.410013in}{0.494036in}}%
\pgfpathlineto{\pgfqpoint{3.354088in}{0.503678in}}%
\pgfpathlineto{\pgfqpoint{3.304934in}{0.514340in}}%
\pgfpathlineto{\pgfqpoint{3.262583in}{0.525717in}}%
\pgfpathlineto{\pgfqpoint{3.224948in}{0.538067in}}%
\pgfpathlineto{\pgfqpoint{3.192065in}{0.551101in}}%
\pgfpathlineto{\pgfqpoint{3.161933in}{0.565413in}}%
\pgfpathlineto{\pgfqpoint{3.134647in}{0.580877in}}%
\pgfpathlineto{\pgfqpoint{3.110267in}{0.597259in}}%
\pgfpathlineto{\pgfqpoint{3.088791in}{0.614227in}}%
\pgfpathlineto{\pgfqpoint{3.068502in}{0.633010in}}%
\pgfpathlineto{\pgfqpoint{3.049574in}{0.653568in}}%
\pgfpathlineto{\pgfqpoint{3.032140in}{0.675790in}}%
\pgfpathlineto{\pgfqpoint{3.016272in}{0.699498in}}%
\pgfpathlineto{\pgfqpoint{3.000852in}{0.726615in}}%
\pgfpathlineto{\pgfqpoint{2.986230in}{0.757175in}}%
\pgfpathlineto{\pgfqpoint{2.973526in}{0.788840in}}%
\pgfpathlineto{\pgfqpoint{2.961846in}{0.823698in}}%
\pgfpathlineto{\pgfqpoint{2.950753in}{0.864059in}}%
\pgfpathlineto{\pgfqpoint{2.941187in}{0.907501in}}%
\pgfpathlineto{\pgfqpoint{2.932834in}{0.956353in}}%
\pgfpathlineto{\pgfqpoint{2.926289in}{1.008082in}}%
\pgfpathlineto{\pgfqpoint{2.921371in}{1.065052in}}%
\pgfpathlineto{\pgfqpoint{2.918422in}{1.124693in}}%
\pgfpathlineto{\pgfqpoint{2.917436in}{1.189399in}}%
\pgfpathlineto{\pgfqpoint{2.918599in}{1.256590in}}%
\pgfpathlineto{\pgfqpoint{2.921978in}{1.326177in}}%
\pgfpathlineto{\pgfqpoint{2.927718in}{1.398061in}}%
\pgfpathlineto{\pgfqpoint{2.935724in}{1.469660in}}%
\pgfpathlineto{\pgfqpoint{2.945663in}{1.538420in}}%
\pgfpathlineto{\pgfqpoint{2.957415in}{1.604265in}}%
\pgfpathlineto{\pgfqpoint{2.970882in}{1.667118in}}%
\pgfpathlineto{\pgfqpoint{2.985974in}{1.726900in}}%
\pgfpathlineto{\pgfqpoint{3.002604in}{1.783523in}}%
\pgfpathlineto{\pgfqpoint{3.019857in}{1.834598in}}%
\pgfpathlineto{\pgfqpoint{3.038287in}{1.882424in}}%
\pgfpathlineto{\pgfqpoint{3.057787in}{1.926921in}}%
\pgfpathlineto{\pgfqpoint{3.078241in}{1.968011in}}%
\pgfpathlineto{\pgfqpoint{3.099505in}{2.005627in}}%
\pgfpathlineto{\pgfqpoint{3.121400in}{2.039721in}}%
\pgfpathlineto{\pgfqpoint{3.143708in}{2.070283in}}%
\pgfpathlineto{\pgfqpoint{3.166174in}{2.097353in}}%
\pgfpathlineto{\pgfqpoint{3.188519in}{2.121023in}}%
\pgfpathlineto{\pgfqpoint{3.212190in}{2.142938in}}%
\pgfpathlineto{\pgfqpoint{3.235339in}{2.161514in}}%
\pgfpathlineto{\pgfqpoint{3.259588in}{2.178150in}}%
\pgfpathlineto{\pgfqpoint{3.282924in}{2.191518in}}%
\pgfpathlineto{\pgfqpoint{3.305108in}{2.201760in}}%
\pgfpathlineto{\pgfqpoint{3.325932in}{2.208924in}}%
\pgfpathlineto{\pgfqpoint{3.343031in}{2.212559in}}%
\pgfpathlineto{\pgfqpoint{3.356053in}{2.213358in}}%
\pgfpathlineto{\pgfqpoint{3.366806in}{2.211704in}}%
\pgfpathlineto{\pgfqpoint{3.372791in}{2.208789in}}%
\pgfpathlineto{\pgfqpoint{3.376099in}{2.205586in}}%
\pgfpathlineto{\pgfqpoint{3.378277in}{2.201308in}}%
\pgfpathlineto{\pgfqpoint{3.379000in}{2.193945in}}%
\pgfpathlineto{\pgfqpoint{3.376855in}{2.184330in}}%
\pgfpathlineto{\pgfqpoint{3.370821in}{2.171109in}}%
\pgfpathlineto{\pgfqpoint{3.359450in}{2.152884in}}%
\pgfpathlineto{\pgfqpoint{3.338470in}{2.124293in}}%
\pgfpathlineto{\pgfqpoint{3.271859in}{2.035557in}}%
\pgfpathlineto{\pgfqpoint{3.245525in}{1.995935in}}%
\pgfpathlineto{\pgfqpoint{3.222216in}{1.956885in}}%
\pgfpathlineto{\pgfqpoint{3.200754in}{1.916473in}}%
\pgfpathlineto{\pgfqpoint{3.181225in}{1.874797in}}%
\pgfpathlineto{\pgfqpoint{3.163648in}{1.831998in}}%
\pgfpathlineto{\pgfqpoint{3.147216in}{1.785907in}}%
\pgfpathlineto{\pgfqpoint{3.132827in}{1.738930in}}%
\pgfpathlineto{\pgfqpoint{3.119802in}{1.688831in}}%
\pgfpathlineto{\pgfqpoint{3.108302in}{1.635679in}}%
\pgfpathlineto{\pgfqpoint{3.098473in}{1.579547in}}%
\pgfpathlineto{\pgfqpoint{3.090472in}{1.520517in}}%
\pgfpathlineto{\pgfqpoint{3.084676in}{1.461150in}}%
\pgfpathlineto{\pgfqpoint{3.081034in}{1.401558in}}%
\pgfpathlineto{\pgfqpoint{3.079612in}{1.341844in}}%
\pgfpathlineto{\pgfqpoint{3.080418in}{1.284605in}}%
\pgfpathlineto{\pgfqpoint{3.083308in}{1.229947in}}%
\pgfpathlineto{\pgfqpoint{3.088175in}{1.177976in}}%
\pgfpathlineto{\pgfqpoint{3.094935in}{1.128802in}}%
\pgfpathlineto{\pgfqpoint{3.103469in}{1.082532in}}%
\pgfpathlineto{\pgfqpoint{3.113664in}{1.039278in}}%
\pgfpathlineto{\pgfqpoint{3.125375in}{0.999146in}}%
\pgfpathlineto{\pgfqpoint{3.138421in}{0.962230in}}%
\pgfpathlineto{\pgfqpoint{3.152583in}{0.928599in}}%
\pgfpathlineto{\pgfqpoint{3.168728in}{0.896162in}}%
\pgfpathlineto{\pgfqpoint{3.185650in}{0.867196in}}%
\pgfpathlineto{\pgfqpoint{3.204370in}{0.839719in}}%
\pgfpathlineto{\pgfqpoint{3.223316in}{0.815704in}}%
\pgfpathlineto{\pgfqpoint{3.243673in}{0.793252in}}%
\pgfpathlineto{\pgfqpoint{3.267023in}{0.770890in}}%
\pgfpathlineto{\pgfqpoint{3.291661in}{0.750417in}}%
\pgfpathlineto{\pgfqpoint{3.319263in}{0.730516in}}%
\pgfpathlineto{\pgfqpoint{3.347882in}{0.712594in}}%
\pgfpathlineto{\pgfqpoint{3.381258in}{0.694381in}}%
\pgfpathlineto{\pgfqpoint{3.415479in}{0.678352in}}%
\pgfpathlineto{\pgfqpoint{3.452406in}{0.663464in}}%
\pgfpathlineto{\pgfqpoint{3.494048in}{0.649059in}}%
\pgfpathlineto{\pgfqpoint{3.538300in}{0.636097in}}%
\pgfpathlineto{\pgfqpoint{3.587224in}{0.624130in}}%
\pgfpathlineto{\pgfqpoint{3.638656in}{0.613889in}}%
\pgfpathlineto{\pgfqpoint{3.696818in}{0.604560in}}%
\pgfpathlineto{\pgfqpoint{3.750898in}{0.598016in}}%
\pgfpathlineto{\pgfqpoint{3.818133in}{0.592120in}}%
\pgfpathlineto{\pgfqpoint{3.889831in}{0.588114in}}%
\pgfpathlineto{\pgfqpoint{3.963772in}{0.586231in}}%
\pgfpathlineto{\pgfqpoint{4.035555in}{0.586559in}}%
\pgfpathlineto{\pgfqpoint{4.107304in}{0.589145in}}%
\pgfpathlineto{\pgfqpoint{4.172438in}{0.593720in}}%
\pgfpathlineto{\pgfqpoint{4.233083in}{0.600173in}}%
\pgfpathlineto{\pgfqpoint{4.287011in}{0.608180in}}%
\pgfpathlineto{\pgfqpoint{4.336330in}{0.617793in}}%
\pgfpathlineto{\pgfqpoint{4.376730in}{0.627757in}}%
\pgfpathlineto{\pgfqpoint{4.414569in}{0.639261in}}%
\pgfpathlineto{\pgfqpoint{4.447684in}{0.651507in}}%
\pgfpathlineto{\pgfqpoint{4.478091in}{0.665035in}}%
\pgfpathlineto{\pgfqpoint{4.505699in}{0.679731in}}%
\pgfpathlineto{\pgfqpoint{4.530441in}{0.695388in}}%
\pgfpathlineto{\pgfqpoint{4.552295in}{0.711711in}}%
\pgfpathlineto{\pgfqpoint{4.572982in}{0.729915in}}%
\pgfpathlineto{\pgfqpoint{4.592294in}{0.750000in}}%
\pgfpathlineto{\pgfqpoint{4.610052in}{0.771881in}}%
\pgfpathlineto{\pgfqpoint{4.626142in}{0.795392in}}%
\pgfpathlineto{\pgfqpoint{4.640556in}{0.820288in}}%
\pgfpathlineto{\pgfqpoint{4.654361in}{0.848521in}}%
\pgfpathlineto{\pgfqpoint{4.667312in}{0.880054in}}%
\pgfpathlineto{\pgfqpoint{4.680018in}{0.917126in}}%
\pgfpathlineto{\pgfqpoint{4.692063in}{0.959752in}}%
\pgfpathlineto{\pgfqpoint{4.703171in}{1.007881in}}%
\pgfpathlineto{\pgfqpoint{4.713207in}{1.061421in}}%
\pgfpathlineto{\pgfqpoint{4.722418in}{1.122749in}}%
\pgfpathlineto{\pgfqpoint{4.730584in}{1.191814in}}%
\pgfpathlineto{\pgfqpoint{4.737495in}{1.268569in}}%
\pgfpathlineto{\pgfqpoint{4.744304in}{1.365332in}}%
\pgfpathlineto{\pgfqpoint{4.749906in}{1.477158in}}%
\pgfpathlineto{\pgfqpoint{4.754970in}{1.623905in}}%
\pgfpathlineto{\pgfqpoint{4.758231in}{1.785658in}}%
\pgfpathlineto{\pgfqpoint{4.759766in}{1.962382in}}%
\pgfpathlineto{\pgfqpoint{4.759312in}{2.156537in}}%
\pgfpathlineto{\pgfqpoint{4.756770in}{2.323287in}}%
\pgfpathlineto{\pgfqpoint{4.752920in}{2.445176in}}%
\pgfpathlineto{\pgfqpoint{4.747592in}{2.544554in}}%
\pgfpathlineto{\pgfqpoint{4.741518in}{2.616401in}}%
\pgfpathlineto{\pgfqpoint{4.734822in}{2.670619in}}%
\pgfpathlineto{\pgfqpoint{4.727592in}{2.712112in}}%
\pgfpathlineto{\pgfqpoint{4.720163in}{2.743325in}}%
\pgfpathlineto{\pgfqpoint{4.711857in}{2.768992in}}%
\pgfpathlineto{\pgfqpoint{4.703158in}{2.789049in}}%
\pgfpathlineto{\pgfqpoint{4.693598in}{2.805672in}}%
\pgfpathlineto{\pgfqpoint{4.683667in}{2.818865in}}%
\pgfpathlineto{\pgfqpoint{4.672290in}{2.830425in}}%
\pgfpathlineto{\pgfqpoint{4.659685in}{2.840180in}}%
\pgfpathlineto{\pgfqpoint{4.644184in}{2.849205in}}%
\pgfpathlineto{\pgfqpoint{4.625851in}{2.857040in}}%
\pgfpathlineto{\pgfqpoint{4.604861in}{2.863548in}}%
\pgfpathlineto{\pgfqpoint{4.577088in}{2.869609in}}%
\pgfpathlineto{\pgfqpoint{4.542570in}{2.874687in}}%
\pgfpathlineto{\pgfqpoint{4.499214in}{2.878784in}}%
\pgfpathlineto{\pgfqpoint{4.436216in}{2.882509in}}%
\pgfpathlineto{\pgfqpoint{4.344890in}{2.885485in}}%
\pgfpathlineto{\pgfqpoint{4.205684in}{2.887764in}}%
\pgfpathlineto{\pgfqpoint{3.946825in}{2.889558in}}%
\pgfpathlineto{\pgfqpoint{3.285526in}{2.890736in}}%
\pgfpathlineto{\pgfqpoint{1.934646in}{2.890412in}}%
\pgfpathlineto{\pgfqpoint{1.373413in}{2.888549in}}%
\pgfpathlineto{\pgfqpoint{1.108038in}{2.885478in}}%
\pgfpathlineto{\pgfqpoint{0.971039in}{2.881477in}}%
\pgfpathlineto{\pgfqpoint{0.899337in}{2.877562in}}%
\pgfpathlineto{\pgfqpoint{0.840776in}{2.872460in}}%
\pgfpathlineto{\pgfqpoint{0.793246in}{2.866115in}}%
\pgfpathlineto{\pgfqpoint{0.765343in}{2.860872in}}%
\pgfpathlineto{\pgfqpoint{0.737718in}{2.853979in}}%
\pgfpathlineto{\pgfqpoint{0.710597in}{2.844845in}}%
\pgfpathlineto{\pgfqpoint{0.688420in}{2.834604in}}%
\pgfpathlineto{\pgfqpoint{0.671182in}{2.824000in}}%
\pgfpathlineto{\pgfqpoint{0.655003in}{2.811416in}}%
\pgfpathlineto{\pgfqpoint{0.640117in}{2.796887in}}%
\pgfpathlineto{\pgfqpoint{0.626906in}{2.780372in}}%
\pgfpathlineto{\pgfqpoint{0.615584in}{2.762107in}}%
\pgfpathlineto{\pgfqpoint{0.604963in}{2.740395in}}%
\pgfpathlineto{\pgfqpoint{0.595325in}{2.715343in}}%
\pgfpathlineto{\pgfqpoint{0.586155in}{2.684740in}}%
\pgfpathlineto{\pgfqpoint{0.578356in}{2.651060in}}%
\pgfpathlineto{\pgfqpoint{0.570961in}{2.609604in}}%
\pgfpathlineto{\pgfqpoint{0.563015in}{2.548049in}}%
\pgfpathlineto{\pgfqpoint{0.557596in}{2.486133in}}%
\pgfpathlineto{\pgfqpoint{0.552491in}{2.396721in}}%
\pgfpathlineto{\pgfqpoint{0.549901in}{2.312146in}}%
\pgfpathlineto{\pgfqpoint{0.549281in}{2.125464in}}%
\pgfpathlineto{\pgfqpoint{0.551809in}{2.043373in}}%
\pgfpathlineto{\pgfqpoint{0.555895in}{1.946412in}}%
\pgfpathlineto{\pgfqpoint{0.563546in}{1.829753in}}%
\pgfpathlineto{\pgfqpoint{0.569913in}{1.760440in}}%
\pgfpathlineto{\pgfqpoint{0.579699in}{1.676557in}}%
\pgfpathlineto{\pgfqpoint{0.589273in}{1.612775in}}%
\pgfpathlineto{\pgfqpoint{0.600715in}{1.551944in}}%
\pgfpathlineto{\pgfqpoint{0.611917in}{1.503843in}}%
\pgfpathlineto{\pgfqpoint{0.624045in}{1.461248in}}%
\pgfpathlineto{\pgfqpoint{0.636090in}{1.426554in}}%
\pgfpathlineto{\pgfqpoint{0.648339in}{1.397397in}}%
\pgfpathlineto{\pgfqpoint{0.660415in}{1.373768in}}%
\pgfpathlineto{\pgfqpoint{0.673163in}{1.353614in}}%
\pgfpathlineto{\pgfqpoint{0.684942in}{1.338974in}}%
\pgfpathlineto{\pgfqpoint{0.696737in}{1.327983in}}%
\pgfpathlineto{\pgfqpoint{0.706185in}{1.321848in}}%
\pgfpathlineto{\pgfqpoint{0.714382in}{1.318550in}}%
\pgfpathlineto{\pgfqpoint{0.722995in}{1.317318in}}%
\pgfpathlineto{\pgfqpoint{0.731582in}{1.318706in}}%
\pgfpathlineto{\pgfqpoint{0.739490in}{1.322784in}}%
\pgfpathlineto{\pgfqpoint{0.746267in}{1.328996in}}%
\pgfpathlineto{\pgfqpoint{0.753106in}{1.338647in}}%
\pgfpathlineto{\pgfqpoint{0.759389in}{1.351720in}}%
\pgfpathlineto{\pgfqpoint{0.765505in}{1.370348in}}%
\pgfpathlineto{\pgfqpoint{0.770808in}{1.394480in}}%
\pgfpathlineto{\pgfqpoint{0.775369in}{1.426408in}}%
\pgfpathlineto{\pgfqpoint{0.779226in}{1.470991in}}%
\pgfpathlineto{\pgfqpoint{0.782290in}{1.535611in}}%
\pgfpathlineto{\pgfqpoint{0.792620in}{1.786725in}}%
\pgfpathlineto{\pgfqpoint{0.798686in}{1.866074in}}%
\pgfpathlineto{\pgfqpoint{0.807852in}{1.952557in}}%
\pgfpathlineto{\pgfqpoint{0.818726in}{2.028706in}}%
\pgfpathlineto{\pgfqpoint{0.829282in}{2.087209in}}%
\pgfpathlineto{\pgfqpoint{0.842029in}{2.145136in}}%
\pgfpathlineto{\pgfqpoint{0.855892in}{2.197545in}}%
\pgfpathlineto{\pgfqpoint{0.871354in}{2.246725in}}%
\pgfpathlineto{\pgfqpoint{0.887521in}{2.290246in}}%
\pgfpathlineto{\pgfqpoint{0.904847in}{2.330420in}}%
\pgfpathlineto{\pgfqpoint{0.916897in}{2.354036in}}%
\pgfpathlineto{\pgfqpoint{0.937145in}{2.392321in}}%
\pgfpathlineto{\pgfqpoint{0.940161in}{2.398894in}}%
\pgfpathlineto{\pgfqpoint{0.960832in}{2.430929in}}%
\pgfpathlineto{\pgfqpoint{0.981817in}{2.459513in}}%
\pgfpathlineto{\pgfqpoint{1.007589in}{2.489854in}}%
\pgfpathlineto{\pgfqpoint{1.034876in}{2.518400in}}%
\pgfpathlineto{\pgfqpoint{1.062212in}{2.543042in}}%
\pgfpathlineto{\pgfqpoint{1.080173in}{2.557013in}}%
\pgfpathlineto{\pgfqpoint{1.087645in}{2.562064in}}%
\pgfpathlineto{\pgfqpoint{1.115363in}{2.581757in}}%
\pgfpathlineto{\pgfqpoint{1.145869in}{2.600919in}}%
\pgfpathlineto{\pgfqpoint{1.179170in}{2.619310in}}%
\pgfpathlineto{\pgfqpoint{1.219264in}{2.638622in}}%
\pgfpathlineto{\pgfqpoint{1.258069in}{2.654894in}}%
\pgfpathlineto{\pgfqpoint{1.305786in}{2.672087in}}%
\pgfpathlineto{\pgfqpoint{1.326895in}{2.677933in}}%
\pgfpathlineto{\pgfqpoint{1.386003in}{2.694729in}}%
\pgfpathlineto{\pgfqpoint{1.439267in}{2.707273in}}%
\pgfpathlineto{\pgfqpoint{1.495032in}{2.717973in}}%
\pgfpathlineto{\pgfqpoint{1.555318in}{2.727816in}}%
\pgfpathlineto{\pgfqpoint{1.628722in}{2.738160in}}%
\pgfpathlineto{\pgfqpoint{1.708822in}{2.747164in}}%
\pgfpathlineto{\pgfqpoint{1.797755in}{2.754864in}}%
\pgfpathlineto{\pgfqpoint{1.897677in}{2.760928in}}%
\pgfpathlineto{\pgfqpoint{1.999852in}{2.765065in}}%
\pgfpathlineto{\pgfqpoint{2.152109in}{2.767346in}}%
\pgfpathlineto{\pgfqpoint{2.258694in}{2.766311in}}%
\pgfpathlineto{\pgfqpoint{2.356545in}{2.763216in}}%
\pgfpathlineto{\pgfqpoint{2.445626in}{2.758243in}}%
\pgfpathlineto{\pgfqpoint{2.523728in}{2.751719in}}%
\pgfpathlineto{\pgfqpoint{2.590817in}{2.743940in}}%
\pgfpathlineto{\pgfqpoint{2.646866in}{2.735296in}}%
\pgfpathlineto{\pgfqpoint{2.694007in}{2.725885in}}%
\pgfpathlineto{\pgfqpoint{2.734348in}{2.715613in}}%
\pgfpathlineto{\pgfqpoint{2.767847in}{2.704827in}}%
\pgfpathlineto{\pgfqpoint{2.794517in}{2.694086in}}%
\pgfpathlineto{\pgfqpoint{2.818397in}{2.682045in}}%
\pgfpathlineto{\pgfqpoint{2.837408in}{2.669966in}}%
\pgfpathlineto{\pgfqpoint{2.853415in}{2.657087in}}%
\pgfpathlineto{\pgfqpoint{2.866326in}{2.643756in}}%
\pgfpathlineto{\pgfqpoint{2.876215in}{2.630522in}}%
\pgfpathlineto{\pgfqpoint{2.884445in}{2.615879in}}%
\pgfpathlineto{\pgfqpoint{2.890790in}{2.600055in}}%
\pgfpathlineto{\pgfqpoint{2.895213in}{2.583394in}}%
\pgfpathlineto{\pgfqpoint{2.898120in}{2.563773in}}%
\pgfpathlineto{\pgfqpoint{2.899155in}{2.541412in}}%
\pgfpathlineto{\pgfqpoint{2.898036in}{2.514071in}}%
\pgfpathlineto{\pgfqpoint{2.894143in}{2.479516in}}%
\pgfpathlineto{\pgfqpoint{2.886703in}{2.435531in}}%
\pgfpathlineto{\pgfqpoint{2.871825in}{2.362824in}}%
\pgfpathlineto{\pgfqpoint{2.831166in}{2.166651in}}%
\pgfpathlineto{\pgfqpoint{2.814431in}{2.071482in}}%
\pgfpathlineto{\pgfqpoint{2.799865in}{1.975847in}}%
\pgfpathlineto{\pgfqpoint{2.793937in}{1.939240in}}%
\pgfpathlineto{\pgfqpoint{2.792365in}{1.932031in}}%
\pgfpathlineto{\pgfqpoint{2.779324in}{1.823529in}}%
\pgfpathlineto{\pgfqpoint{2.767356in}{1.704837in}}%
\pgfpathlineto{\pgfqpoint{2.757601in}{1.585883in}}%
\pgfpathlineto{\pgfqpoint{2.747868in}{1.444438in}}%
\pgfpathlineto{\pgfqpoint{2.739452in}{1.290410in}}%
\pgfpathlineto{\pgfqpoint{2.732138in}{1.116368in}}%
\pgfpathlineto{\pgfqpoint{2.726317in}{0.924817in}}%
\pgfpathlineto{\pgfqpoint{2.721265in}{0.685924in}}%
\pgfpathlineto{\pgfqpoint{2.716535in}{0.479400in}}%
\pgfpathlineto{\pgfqpoint{2.713788in}{0.444704in}}%
\pgfpathlineto{\pgfqpoint{2.710464in}{0.427724in}}%
\pgfpathlineto{\pgfqpoint{2.706606in}{0.418839in}}%
\pgfpathlineto{\pgfqpoint{2.702014in}{0.413581in}}%
\pgfpathlineto{\pgfqpoint{2.696250in}{0.410128in}}%
\pgfpathlineto{\pgfqpoint{2.685710in}{0.407136in}}%
\pgfpathlineto{\pgfqpoint{2.668401in}{0.405158in}}%
\pgfpathlineto{\pgfqpoint{2.631442in}{0.403789in}}%
\pgfpathlineto{\pgfqpoint{2.529205in}{0.402988in}}%
\pgfpathlineto{\pgfqpoint{2.052807in}{0.402659in}}%
\pgfpathlineto{\pgfqpoint{0.456120in}{0.403499in}}%
\pgfpathlineto{\pgfqpoint{0.449622in}{0.403674in}}%
\pgfpathlineto{\pgfqpoint{0.449622in}{0.403674in}}%
\pgfusepath{stroke}%
\end{pgfscope}%
\begin{pgfscope}%
\pgfsetrectcap%
\pgfsetmiterjoin%
\pgfsetlinewidth{0.803000pt}%
\definecolor{currentstroke}{rgb}{0.000000,0.000000,0.000000}%
\pgfsetstrokecolor{currentstroke}%
\pgfsetdash{}{0pt}%
\pgfpathmoveto{\pgfqpoint{0.448634in}{0.402556in}}%
\pgfpathlineto{\pgfqpoint{0.448634in}{2.891760in}}%
\pgfusepath{stroke}%
\end{pgfscope}%
\begin{pgfscope}%
\pgfsetrectcap%
\pgfsetmiterjoin%
\pgfsetlinewidth{0.803000pt}%
\definecolor{currentstroke}{rgb}{0.000000,0.000000,0.000000}%
\pgfsetstrokecolor{currentstroke}%
\pgfsetdash{}{0pt}%
\pgfpathmoveto{\pgfqpoint{4.799294in}{0.402556in}}%
\pgfpathlineto{\pgfqpoint{4.799294in}{2.891760in}}%
\pgfusepath{stroke}%
\end{pgfscope}%
\begin{pgfscope}%
\pgfsetrectcap%
\pgfsetmiterjoin%
\pgfsetlinewidth{0.803000pt}%
\definecolor{currentstroke}{rgb}{0.000000,0.000000,0.000000}%
\pgfsetstrokecolor{currentstroke}%
\pgfsetdash{}{0pt}%
\pgfpathmoveto{\pgfqpoint{0.448634in}{0.402556in}}%
\pgfpathlineto{\pgfqpoint{4.799294in}{0.402556in}}%
\pgfusepath{stroke}%
\end{pgfscope}%
\begin{pgfscope}%
\pgfsetrectcap%
\pgfsetmiterjoin%
\pgfsetlinewidth{0.803000pt}%
\definecolor{currentstroke}{rgb}{0.000000,0.000000,0.000000}%
\pgfsetstrokecolor{currentstroke}%
\pgfsetdash{}{0pt}%
\pgfpathmoveto{\pgfqpoint{0.448634in}{2.891760in}}%
\pgfpathlineto{\pgfqpoint{4.799294in}{2.891760in}}%
\pgfusepath{stroke}%
\end{pgfscope}%
\begin{pgfscope}%
\pgfsetbuttcap%
\pgfsetmiterjoin%
\definecolor{currentfill}{rgb}{1.000000,1.000000,1.000000}%
\pgfsetfillcolor{currentfill}%
\pgfsetfillopacity{0.500000}%
\pgfsetlinewidth{1.003750pt}%
\definecolor{currentstroke}{rgb}{0.800000,0.800000,0.800000}%
\pgfsetstrokecolor{currentstroke}%
\pgfsetstrokeopacity{0.500000}%
\pgfsetdash{}{0pt}%
\pgfpathmoveto{\pgfqpoint{3.700085in}{0.761312in}}%
\pgfpathlineto{\pgfqpoint{4.547129in}{0.761312in}}%
\pgfpathquadraticcurveto{\pgfqpoint{4.574907in}{0.761312in}}{\pgfqpoint{4.574907in}{0.789090in}}%
\pgfpathlineto{\pgfqpoint{4.574907in}{2.769646in}}%
\pgfpathquadraticcurveto{\pgfqpoint{4.574907in}{2.797424in}}{\pgfqpoint{4.547129in}{2.797424in}}%
\pgfpathlineto{\pgfqpoint{3.700085in}{2.797424in}}%
\pgfpathquadraticcurveto{\pgfqpoint{3.672307in}{2.797424in}}{\pgfqpoint{3.672307in}{2.769646in}}%
\pgfpathlineto{\pgfqpoint{3.672307in}{0.789090in}}%
\pgfpathquadraticcurveto{\pgfqpoint{3.672307in}{0.761312in}}{\pgfqpoint{3.700085in}{0.761312in}}%
\pgfpathclose%
\pgfusepath{stroke,fill}%
\end{pgfscope}%
\begin{pgfscope}%
\pgfsetrectcap%
\pgfsetroundjoin%
\pgfsetlinewidth{1.003750pt}%
\definecolor{currentstroke}{rgb}{1.000000,0.388235,0.278431}%
\pgfsetstrokecolor{currentstroke}%
\pgfsetdash{}{0pt}%
\pgfpathmoveto{\pgfqpoint{3.727863in}{2.693257in}}%
\pgfpathlineto{\pgfqpoint{3.797307in}{2.693257in}}%
\pgfusepath{stroke}%
\end{pgfscope}%
\begin{pgfscope}%
\pgftext[x=3.908418in,y=2.644646in,left,base]{\rmfamily\fontsize{10.000000}{12.000000}\selectfont \textnormal{Reference}}%
\end{pgfscope}%
\begin{pgfscope}%
\pgfsetrectcap%
\pgfsetroundjoin%
\pgfsetlinewidth{1.003750pt}%
\definecolor{currentstroke}{rgb}{0.121569,0.466667,0.705882}%
\pgfsetstrokecolor{currentstroke}%
\pgfsetdash{}{0pt}%
\pgfpathmoveto{\pgfqpoint{3.727863in}{2.493812in}}%
\pgfpathlineto{\pgfqpoint{3.797307in}{2.493812in}}%
\pgfusepath{stroke}%
\end{pgfscope}%
\begin{pgfscope}%
\pgftext[x=3.908418in,y=2.445201in,left,base]{\rmfamily\fontsize{10.000000}{12.000000}\selectfont \(\displaystyle \textnormal{tol}=10^{{-}10}\)}%
\end{pgfscope}%
\begin{pgfscope}%
\pgfsetrectcap%
\pgfsetroundjoin%
\pgfsetlinewidth{1.003750pt}%
\definecolor{currentstroke}{rgb}{1.000000,0.498039,0.054902}%
\pgfsetstrokecolor{currentstroke}%
\pgfsetdash{}{0pt}%
\pgfpathmoveto{\pgfqpoint{3.727863in}{2.294368in}}%
\pgfpathlineto{\pgfqpoint{3.797307in}{2.294368in}}%
\pgfusepath{stroke}%
\end{pgfscope}%
\begin{pgfscope}%
\pgftext[x=3.908418in,y=2.245757in,left,base]{\rmfamily\fontsize{10.000000}{12.000000}\selectfont \(\displaystyle \textnormal{tol}=10^{{-}9}\)}%
\end{pgfscope}%
\begin{pgfscope}%
\pgfsetrectcap%
\pgfsetroundjoin%
\pgfsetlinewidth{1.003750pt}%
\definecolor{currentstroke}{rgb}{0.172549,0.627451,0.172549}%
\pgfsetstrokecolor{currentstroke}%
\pgfsetdash{}{0pt}%
\pgfpathmoveto{\pgfqpoint{3.727863in}{2.094924in}}%
\pgfpathlineto{\pgfqpoint{3.797307in}{2.094924in}}%
\pgfusepath{stroke}%
\end{pgfscope}%
\begin{pgfscope}%
\pgftext[x=3.908418in,y=2.046312in,left,base]{\rmfamily\fontsize{10.000000}{12.000000}\selectfont \(\displaystyle \textnormal{tol}=10^{{-}8}\)}%
\end{pgfscope}%
\begin{pgfscope}%
\pgfsetrectcap%
\pgfsetroundjoin%
\pgfsetlinewidth{1.003750pt}%
\definecolor{currentstroke}{rgb}{0.839216,0.152941,0.156863}%
\pgfsetstrokecolor{currentstroke}%
\pgfsetdash{}{0pt}%
\pgfpathmoveto{\pgfqpoint{3.727863in}{1.895479in}}%
\pgfpathlineto{\pgfqpoint{3.797307in}{1.895479in}}%
\pgfusepath{stroke}%
\end{pgfscope}%
\begin{pgfscope}%
\pgftext[x=3.908418in,y=1.846868in,left,base]{\rmfamily\fontsize{10.000000}{12.000000}\selectfont \(\displaystyle \textnormal{tol}=10^{{-}7}\)}%
\end{pgfscope}%
\begin{pgfscope}%
\pgfsetrectcap%
\pgfsetroundjoin%
\pgfsetlinewidth{1.003750pt}%
\definecolor{currentstroke}{rgb}{0.580392,0.403922,0.741176}%
\pgfsetstrokecolor{currentstroke}%
\pgfsetdash{}{0pt}%
\pgfpathmoveto{\pgfqpoint{3.727863in}{1.696035in}}%
\pgfpathlineto{\pgfqpoint{3.797307in}{1.696035in}}%
\pgfusepath{stroke}%
\end{pgfscope}%
\begin{pgfscope}%
\pgftext[x=3.908418in,y=1.647424in,left,base]{\rmfamily\fontsize{10.000000}{12.000000}\selectfont \(\displaystyle \textnormal{tol}=10^{{-}6}\)}%
\end{pgfscope}%
\begin{pgfscope}%
\pgfsetrectcap%
\pgfsetroundjoin%
\pgfsetlinewidth{1.003750pt}%
\definecolor{currentstroke}{rgb}{0.549020,0.337255,0.294118}%
\pgfsetstrokecolor{currentstroke}%
\pgfsetdash{}{0pt}%
\pgfpathmoveto{\pgfqpoint{3.727863in}{1.496590in}}%
\pgfpathlineto{\pgfqpoint{3.797307in}{1.496590in}}%
\pgfusepath{stroke}%
\end{pgfscope}%
\begin{pgfscope}%
\pgftext[x=3.908418in,y=1.447979in,left,base]{\rmfamily\fontsize{10.000000}{12.000000}\selectfont \(\displaystyle \textnormal{tol}=10^{{-}5}\)}%
\end{pgfscope}%
\begin{pgfscope}%
\pgfsetrectcap%
\pgfsetroundjoin%
\pgfsetlinewidth{1.003750pt}%
\definecolor{currentstroke}{rgb}{0.890196,0.466667,0.760784}%
\pgfsetstrokecolor{currentstroke}%
\pgfsetdash{}{0pt}%
\pgfpathmoveto{\pgfqpoint{3.727863in}{1.297146in}}%
\pgfpathlineto{\pgfqpoint{3.797307in}{1.297146in}}%
\pgfusepath{stroke}%
\end{pgfscope}%
\begin{pgfscope}%
\pgftext[x=3.908418in,y=1.248535in,left,base]{\rmfamily\fontsize{10.000000}{12.000000}\selectfont \(\displaystyle \textnormal{tol}=10^{{-}4}\)}%
\end{pgfscope}%
\begin{pgfscope}%
\pgfsetrectcap%
\pgfsetroundjoin%
\pgfsetlinewidth{1.003750pt}%
\definecolor{currentstroke}{rgb}{0.498039,0.498039,0.498039}%
\pgfsetstrokecolor{currentstroke}%
\pgfsetdash{}{0pt}%
\pgfpathmoveto{\pgfqpoint{3.727863in}{1.097701in}}%
\pgfpathlineto{\pgfqpoint{3.797307in}{1.097701in}}%
\pgfusepath{stroke}%
\end{pgfscope}%
\begin{pgfscope}%
\pgftext[x=3.908418in,y=1.049090in,left,base]{\rmfamily\fontsize{10.000000}{12.000000}\selectfont \(\displaystyle \textnormal{tol}=10^{{-}3}\)}%
\end{pgfscope}%
\begin{pgfscope}%
\pgfsetrectcap%
\pgfsetroundjoin%
\pgfsetlinewidth{1.003750pt}%
\definecolor{currentstroke}{rgb}{0.737255,0.741176,0.133333}%
\pgfsetstrokecolor{currentstroke}%
\pgfsetdash{}{0pt}%
\pgfpathmoveto{\pgfqpoint{3.727863in}{0.898257in}}%
\pgfpathlineto{\pgfqpoint{3.797307in}{0.898257in}}%
\pgfusepath{stroke}%
\end{pgfscope}%
\begin{pgfscope}%
\pgftext[x=3.908418in,y=0.849646in,left,base]{\rmfamily\fontsize{10.000000}{12.000000}\selectfont \(\displaystyle \textnormal{tol}=10^{{-}2}\)}%
\end{pgfscope}%
\end{pgfpicture}%
\makeatother%
\endgroup%
}
    %\includegraphics[width=0.9\linewidth]{figures/lcs_figures/rkdp87.pdf}
    \caption[LCS curves found by means of the Dormand-Prince 8(7) integration
    scheme]{
        LCS curves found by means of the Dormand-Prince 8(7) integration
        scheme. The reference LCS, as shown by itself in figure
        \ref{fig:referencelcs}, is plotted on the bottom layer. Note that
        the LCS for the lowest tolerance level considered, that is,
        $\textnormal{tol}=10^{-1}$, is not included. This is because the
        corresponding $\mathcal{U}_{0}$ domain, shown in figure
        \ref{fig:u0_dom_err_dp87}, and the reference $\mathcal{U}_{0}$, shown in figure
        \ref{fig:u0_domain} are unlike one another. Here, the most immediately
        discernible dissimilarities emanate from the tolerance level
        $\textnormal{tol}=10^{-2}$. Close inspection, however, reveals
        discrepancies for all tolerance levels $\textnormal{tol}>10^{-6}$,
        particularly in the lower left corner.}
    \label{fig:lcs_rkdp87}
\end{figure}


\clearpage

