For the double gyre system considered here, the calculation of its LCSs
does not exhibit particularly sensitive dependence to
the choice of numerical integration method. However, this could be a
consequence of the LCSs present within the system being robust under the chosen
parameter values. This view is supported by the fact that the errors in the
computed LCS curves quickly flattened for sufficiently small integration time
steps or tolerance levels, while a similar effect did not manifest itself for
the computed flow maps or strain eigenvalues. At the same time, the fact that
the same strainline segments were identified as LCSs even for comparably large
errors in the strain eigenvalues, suggests that the numerical implementation
of the variational principles in order to find LCSs is, in itself, robust.

There is, however, room for research with regards to the numerical
implementation of one of the LCS existence conditions, derived from their
variational theory. The approach used for this project is not particularly
well-documented in the literature, in that, there exists independent variables
for which there does not yet exist an objective way to determine. Here, these
were chosen based upon careful inspection of the system under consideration,
and in order for the LCS curves to conform with the ones obtained in the
literature. A suggested alternative approach, which was not investigated as
part of this project, would be to utilize a sort of numerical clustering
algorithm rather than resorting to similar (to some extent) subject
considerations which were employed here.

Although no numerical integration scheme stood out as the absolute best,
using higher order methods invariably resulted in more efficient calculations,
which are less prone to numerical round-off errors. Accordingly, the use of
higher order methods is generaly advisable. Equally important as the choice of
integration scheme, however, is the choice of numerical step length or tolerance
level. Ideally, this should be selected based upon physical considerations of
the system at hand. For instance, one could identify a characteristic velocity,
then tune the step length or tolerance level such that no tracers moving with
said velocity never entirely skip a grid cell when moving from one time level
to the next. This, in order to ensure that the local instantaneous dynamics
are resolved properly. To what scale the microscopic behaviour should be
resolved depends on what scale to which detailed information regarding the flow
is sought, and, of course, the sampling frequency (spatial and temporal alike)
of discrete data samples, for most real-life applications.

On that note, when investigating transport systems for which the available
data sets are discrete, the choice of numerical \emph{interpolation} scheme
will generally also impact the calculations of LCSs. This depencence has not
been investigated as part of this project. However, similar reservations
as for the integration time step or tolerance levels are also applicable to
the spatial and temporal sampling frequencies involved in the interpolation.
Furthermore, the choice of integration scheme should be made in relation
to the interpolation method. For instance, it makes little sense to use a
\nth{5} order accurate integration scheme in tandem with a \nth{3} order
accurate interpolator. Put simply, the interaction between interpolation
and integration schemes in the analysis of discrete data sets is of great
interest for practical applications of the LCS theory, and warrants
further investigation.
