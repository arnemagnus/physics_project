\subsubsection{Suggestions for further work}

Regarding my own future work, I think that it would be natural to continue
the study of LCSs. In particular, I find the prospect of analyzing LCSs in
three-dimensional flow systems alluring. To my knowledge, the three-dimensional
formulation of the variational LCS approach has not yet been explored in the
literature. Thus, this could potentially accelerate the recommendability of
the use of LCS theory to describe systems which can not reasonably be regarded
as two-dimensional. Furthermore, under the assumption that I am able to
find a three-dimensional velocity field for which there exists a plane with a
similar double gyre field to the one which has been investigated in detail in
relation to this project, the results obtained here could prove useful
with regards to verification.

Moreover, there are a myriad of integration methods which were not investigated
here. Aside from the many singlestep and embedded methods, there also exists
an entirely different class of ODE solvers; namely, linear multistep methods.
These methods have memory, and thus function quite differently from the
ones considered for this project. Interestingly, both fixed and adaptive
stepsize linear multistep methods exist, and, by means of known recurrence
relations, one may in principle design multistep methods of arbitrary order
\parencite[chapter III]{hairer1993solving}. A study of linear multistep
integration methods can be viewed as a natural extension to the study of LCSs
in three-dimensional flows, as suggested above.





Regarding my own future work, I find the prospect of analyzing
LCSs in three-dimensional flow systems alluring. To my knowledge, the
three-dimensional formulation of the variational LCS approach has not yet
been explored in detail in the literature. Thus, this could potentially
accelerate the recommendability of the use of LCS theory to describe systems
which can not reasonably be regarded as two-dimensional. Moreover, I think
that it would be natural to continue working on LCSs, considering that I have
developed a good understanding of the underlying physical principles through the
work I have conducted so far --- not to mention the fact that I have become
quite adept regarding the use of supercomputers for scientific purposes, which
I most certainly expect will become necessary, in view of the added inherent
mathematical complexity that follows by moving from two to three spatial
dimensions.
