\mainmatter

\chapter{Introduction}
\label{cha:introduction}
When analyzing complex dynamical systems, such as nonlinear many-body problems,
the conventional approach to predicting future states by means of simulating
the trajectories of point particles is frequently insufficient. This is due
to the resulting predictions being very sensitive to small changes in time
and initial positions. One way of addressing the delicate dependence on initial
conditions is to run several different models for the same underlying physical
systems. This invariably results in larger, more well-resolved distributions
of the transported particles, at the cost of ignoring key, inherent, organizing
structures in the system.

Recently, the concept of Lagrangian coherent structures saw the light of day,
emerging from the intersection between nonlinear dynamics, that is, the
underlying mathematical principles of chaos theory, and fluid dynamics. These
provide a new framework for understanding transport phenomena in concept
fluid flow systems. Lagrangian coherent structures can be described as
time-evolving `landscapes' in a multidimensional space, which dictate flow
patterns in dynamical systems. In particular, such structures define the
interfaces of dynamically distinct, invariant regions. An invariant region is
characterized as a domain where all particle trajectories that originate within
the region, remain in it, although the region itself can move and deform with
time. So, simply put; Lagrangian coherent structures enable us to make
predictions regarding the future states of flow systems.

There are two possible ways of describing fluid flow. The Eulerian approach
is to consider the properties of a flow field at each fixed point in time and
space. An example is the concept of velocity fields, which produce the
local and instantaneous velocities of all fluid elements within the domain
under consideration. The Lagrangian point of view, on the other hand, concerns
the developing velocity of each individual particle along their paths, as they
are transported by the flow. Unlike the Eulerian perspective, the Lagrangian
mindset is objective, as in frame-invariant. That is, properties of Lagrangian
fields are unchanged by time-dependent translations and rotations of the
reference frame. For unsteady flow systems, which are more common than
steady flow systems in nature, there exists no self-evident preferred frame
of reference. This means that any transport-dictating dynamic structures
should hold for \emph{any} choice of reference frame. Furthermore, this is
the main rationale for which \emph{Lagrangian}, rather
than \emph{Eulerian}, coherent structures have been pursued.

Although the framework for detecting Lagrangian coherent structures is
mathematically valid for any number of dimensions, the focus of this project
work has been two-dimensional flow systems. Many natural processes fall
within this category, perhaps most notably the transport of debris and
contaminations, such as radioactive particles or the remnants of an oil
spill, by means of oceanic surface currents. Being able to successfully predict
where such particles will be taken by the flow enables us to isolate and
extract them before they are able to reach the coastline, thus preventing
potential humanitarian and natural calamities.

\newpage

At the heart of detecting Lagrangian coherent structures lies the advection
of fluid elements by means of a velocity field, which describes the system
under consideration. This rings true both for test cases, where the velocity
profile is known analytically, and for real-life systems, typically described
by means of discrete samples of the instantaneous velocity field. For this
project work, the topic of interest is how the detection of Lagrangian coherent
structures depends on the choice of numerical integration method used in order
to compute the aforementioned particle transport. In particular, four singlestep
methods and four embedded, adaptive step length methods, each with different
properties, were used in order to advect a collection of fluid elements
by means of an analytically known, two-dimensional, unsteady velocity field.




%\begin{framed}
%    \begin{itemize}
%        \item \sout{Lagrangian coherent structure: A structure that can separate dynamically
%            distinct invariant regions. Invariant retions: All trajectories starting
%            out within it, remains within the region, although the region itself
%        may move and deform with time.}
%    \item \sout{Lagrangian coherent structure: Landscapes in multidimensional
%        landscapes, shaping the flow patterns in dynamical systems.}
%    \item \sout{Motivation: Complex systems, i.e., many-particle nonlinear
%        systems: Need computational shortcuts}
%    \item \sout{LCS: Principally robust structures which enable us
%                        to make predictions regarding the future states of
%                    a flow system.}
%                \item \sout{Examples of real-world application:}
%                            \begin{itemize}
%                                \item \sout{Population growth}
%                                \item \sout{Predicting the advection of particles by means of oceanic currents or wind}
%    \begin{itemize}
%        \item \sout{Recent examples: The volcanic eruptions at Eyafjällajökul, and
%            at Bali (2017)}
%    \end{itemize}
%\item \sout{Predicting where the remnants of an oil spill will end up --> Where to focus
%    the rescue operation in the short and medium term}
%                            \end{itemize}
%    \end{itemize}
%\end{framed}


\vspace{\fill}

\newpage

\chapter{Theory}
\label{cha:theory}
\input{mainmatter/theory/generalode}

\subsection{The Runge-Kutta family of numerical ODE solvers}
\label{sub:the_runge_kutta_family_of_numerical_methods}

In numerical analysis, the Runge-Kutta family of methods is a popular
collection of implicit and explicit iterative methods, used in temporal
discretization in order to obtain numerical approximations of the \emph{true}
solutions of systems like \eqref{eq:ivpsystem}. The German mathematicians C.
Runge and M. W. Kutta developed the first of the family's methods at the turn
of the twentieth century \parencite[p.134]{hairer1993solving}. The general
scheme of what is now known as a Runge-Kutta method is as follows: \\

\begin{defn}
    \label{def:generalrungekutta}
    Let $s$ be an integer and $a_{1,1},a_{1,2},\ldots,a_{1,s},a_{2,1},
    a_{2,2},\ldots,a_{2,s},\ldots,a_{s,1},a_{s,2},\ldots,a_{s,s}$,
    $b_{1},b_{2},\ldots,b_{s}$ and $c_{1},c_{2},\ldots,c_{s}$ be real
    coefficients. Let $h$ be the numerical step length used in the
    temporal discretization. Then, the method
\begin{equation}
    \label{eq:generalrungekutta}
    \begin{aligned}
        k_{i} &= f\bigg(t_{n}+c_{i}h,x_{n}+
                h\sum\limits_{j=1}^{s}a_{i,j}k_{j}\bigg),\quad{}i=1,\ldots,s\\
        x_{n+1} &= x_{n} + h\sum\limits_{i=1}^{s}b_{i}k_{i}
    \end{aligned}
\end{equation}
is called an \emph{s-stage Runge-Kutta method} for the system
\eqref{eq:ivpsystem}.
\end{defn}

The main reason to include multiple stages in a Runge-Kutta method,
is to improve the numerical accuracy of the computed solutions.
The \emph{order} of a Runge-Kutta method can be defined as follows:\\

\begin{defn}
    \label{def:rungekuttaorder}
    A Runge-Kutta method, given by~\cref{eq:generalrungekutta}, is
    said to be of \emph{order} $p$ if, for sufficiently smooth systems
    \eqref{eq:ivpsystem}, the local error $e_{n}$ scales as $h^{p+1}$, that is:
    \begin{equation}
        \label{eq:rungekuttaorder}
        e_{n}=\norm{x_{n}-u_{n-1}(t_{n})} \leq Kh^{p+1},
    \end{equation}
    where $u_{n-1}(t)$ is the exact solution of the ODE in system
    \eqref{eq:ivpsystem} at time $t$, subject to the initial condition
    $u_{n-1}(t_{n-1})=x_{n-1}$, and $K$ is a numerical constant. This is true,
    if the Taylor series for the exact solution $u_{n-1}(t_{n})$ and the
    numerical solution $x_{n}$ coincide up to (and including) the term $h^p$.
\end{defn}

The \emph{global} error
\begin{equation}
    \label{eq:rungekuttaglobalorderdef}
    E_{n} = x_{n}-x(t_{n}),
\end{equation}
where $x(t)$ is the exact solution of system~\eqref{eq:ivpsystem} at time $t$,
accumulated by $n$ repeated applications of the numerical method, can be
estimated by
\begin{equation}
    \label{eq:rungekuttaglobalorderapprox}
    \abs{E_{n}} \leq C\sum\limits_{l=1}^{n}\abs{e_{l}},
\end{equation}
where $C$ is a numerical constant, depending on both the right hand side of
the ODE in system~\eqref{eq:ivpsystem} and the difference $t_{n}-t_{0}$.
Making use of~\cref{def:rungekuttaorder}, the global error can be estimated
by
\begin{gather}
    \label{eq:rungekuttaglobalorderestimate}
    \begin{aligned}
        \abs{E_{n}}&\leq C\sum\limits_{l=1}^{n}\abs{e_{l}} %
        \leq C\sum\limits_{l=1}^{n}\abs{K_{l}}\hspace{0.5ex}h^{p+1} \\
        &\leq C\hspace{0.5ex}\max\limits_{l}\big\{\abs{K_{l}}\big\}\hspace{0.5ex}n\hspace{0.5ex}h^{p+1}%
        \leq C\hspace{0.5ex}\max\limits_{l}\big\{\abs{K_{l}}\big\}\hspace{0.5ex}\frac{t_{n}-t_{0}}{h}\hspace{0.5ex}h^{p+1}\\
        &\leq \widetilde{K}h^{p},
    \end{aligned}
\end{gather}
where $\widetilde{K}$ is a numerical constant.
\Cref{eq:rungekuttaglobalorderestimate} demonstrates that, for a $p$-th
order Runge-Kutta method, the global error can be expected to scale
as $h^{p}$.

%It is easy to show that if the local error of a Runge-Kutta method is of order
%$p+1$, the global error, i.e., the total accumulated error resulting of
%applying the algorithm a number of times, is expected to scale as $h^{p}$.
%Showing this is left as an exercise for the interested reader.
%
In definition~\ref{def:generalrungekutta}, the matrix $(a_{i,j})$ is commonly
called the \emph{Runge-Kutta matrix}, while $b_{i}$ and $c_{i}$ are known as
the \emph{weights} and \emph{nodes}, respectively.  Since the 1960s, it has
been customary to represent Runge-Kutta methods, given by
\cref{eq:generalrungekutta}, symbolically, by means of mnemonic devices known
as Butcher tableaus \parencite[p.134]{hairer1993solving}. The Butcher tableau
for a general \emph{s}-stage Runge-Kutta method, introduced in definition
\ref{def:generalrungekutta}, is presented in table~\ref{tab:generalbutcher}.

\clearpage

\begin{table}[htpb]
    \centering
    \caption[Butcher tableau representation of a general $s$-stage
                Runge-Kutta method]{Butcher tableau representation of a general
                    $s$-stage Runge-Kutta method.}
    \label{tab:generalbutcher}
    \[\renewcommand{\arraystretch}{1.25}
        \begin{array}{c|cccc}
            \toprule
            c_{1} & a_{1,1} & a_{1,2} & \ldots & a_{1,s}\\
            c_{2} & a_{2,1} & a_{2,2} & \ldots & a_{2,s}\\
            \vdots & \vdots & \vdots & \ddots & \vdots \\
            c_{s} & a_{s,1} & a_{s,2} & \ldots & a_{s,s}\\
            \hline
            & b_{1} & b_{2} & \ldots & b_{s}\\
            \bottomrule
    \end{array}
\]
\end{table}

For explicit Runge-Kutta methods, the Runge-Kutta matrix $(a_{i,j})$ is lower
triangular. Similarly, for fully implicit Runge-Kutta methods, the Runge-Kutta
matrix is upper triangular. The difference between explicit and implicit
methods is outlined in~\cref{eq:exim}.
%Unlike explicit methods, implicit methods require
%the solution of a linear system at every time level, making them more
%computationally demanding than their explicit siblings. The main selling point
%of implicit methods is that they are more numerically stable than explicit
%methods. This property means that implicit methods are particularly well-suited
%for \emph{stiff} systems, i.e., physical systems with highly disparate time
%scales~\parencite[p.2]{hairer1996solving}. For such systems,
%most explicit methods are highly numerically unstable, unless the numerical step
%size is made exceptionally small, rendering most explicit methods practically
%useless. For \emph{nonstiff} systems, however, implicit methods behave similarly
%to their explicit analogues in terms of numerical accuracy and
%convergence properties.
%
%\clearpage

During the first half of the twentieth century, a substantial amount of research
was conducted in order to develop numerically robust, high-order, explicit
Runge-Kutta methods. The idea was that using such methods would mean one could
resort to larger time increments $h$ without sacrificing precision in the
computational solution. However, the number of stages $s$ grows quicker than
linearly as a function of the required order $p$. It has been proven
that, for $p\geq5$, no explicit Runge-Kutta method of order $p$ with $s=p$
stages exists \parencite[p.173]{hairer1993solving}. This is
one of the reasons for the attention shift from the latter half of the 1950s
and onwards, towards so-called \emph{embedded} Runge-Kutta methods.

The basic idea of embedded Runge-Kutta methods is that they, aside from the
numerical approximation $x_{n+1}$, yield a second approximation
$\widehat{x}_{n+1}$. The difference between the two approximations then yields
an estimate of the local error of the less precise result, which can be used for
automatic step size control~\parencite[pp.167--168]{hairer1993solving}. The
trick is to construct two independent, explicit Runge-Kutta methods which both
use the \emph{same} function evaluations. This results in practically obtaining
the two solutions for the price of one, in terms of computational complexity.
The Butcher tableau of an embedded, general, explicit Runge-Kutta method is
illustrated in~\cref{tab:generalembeddedbutcher}.

\begin{table}[htpb]
    \centering
    \caption[Butcher tableau representation of general, embedded, explicit
    Runge-Kutta methods]{Butcher tableau representation of general, embedded,
        explicit Runge-Kutta methods.}
    \label{tab:generalembeddedbutcher}
    \[\renewcommand{\arraystretch}{1.25}
    \begin{array}{c|ccccc}
    \toprule
    0 \\
    c_{2} & a_{2,1} \\
    c_{3} & a_{3,1} & a_{3,2} \\
    \vdots & \vdots & \vdots & \ddots\\
    c_{s} & a_{s,1} & a_{s,2} & \ldots & a_{s,s-1}\\
    \hline
    & b_{1} & b_{2} & \ldots & b_{s-1} & b_{s} \\
    \hline
    & \widehat{b}_{1} & \widehat{b}_{2} & \ldots & \widehat{b}_{s-1}& \widehat{b}_{s}\\
    \bottomrule
    \end{array}
\]
\end{table}

For embedded methods, the coefficients are tuned such that
\begin{subequations}
    \begin{equation}
        \label{eq:embeddedsol}
        x_{n+1} = x_{n} + h\sum\limits_{i=1}^{s}b_{i}k_{i}
    \end{equation}
is of order $p$, and
    \begin{equation}
        \label{eq:embeddedinterp}
        \widehat{x}_{n+1} = x_{n} + h\sum\limits_{i=1}^{s}\widehat{b}_{i}k_{i}
    \end{equation}
\end{subequations}
is of order $\widehat{p}$, typically with $\widehat{p} = p \pm 1$. Which
of the solutions are used to continue the numerical integration, depends on
the integration scheme in question. In the following, the solution which is
\emph{not} used to continue the integration, will be referred to as the
\emph{interpolant} solution.

%\clearpage
\subsection{The Runge-Kutta methods under consideration}
\label{sub:the_runge_kutta_methods_under_consideration}

There exists an abundance of Runge-Kutta methods. Many of them are
fine-tuned for specific constraints, such as problems of varying degrees of
stiffness. It is neither possible nor meaningful to investigate them all
in the context of general flow dynamics. For this reason, we consider two classes
of explicit Runge-Kutta methods, namely singlestep and adaptive stepsize
methods. From both classes, we include four different general-purpose ODE solvers
of varying order.

\subsubsection{Singlestep methods}
\label{ssub:singlestep_methods}

The singlestep methods under consideration are the classical, explicit
Runge-Kutta methods of orders one through to four, i.e., the \emph{Euler},
\emph{Heun}, \emph{Kutta} and \emph{classical Runge-Kutta} methods. The
Euler method is \nth{1}-order accurate, and requires a single function
evaluation of the right hand side of the ODE of system
\eqref{eq:ivpsystem} or~\eqref{eq:ivpsystemhigherdimensions} at each time step.
Its Butcher tableau representation can be found in~\cref{tab:butchereuler}.
It is the simplest explicit method for numerical integration of ordinary
differential equations. The Euler method is often used as a basis to construct
more complex methods, such as the Heun method, which is also known as the
\emph{improved Euler method} or the \emph{explicit trapezoidal rule}. The Heun
method is \nth{2}-order accurate, and requires two function evaluations at each
time step. Its Butcher tableau representation can be found in
\cref{tab:butcherrk2}.

\begin{figure}[htpb]
    \centering
    \includegraphics[width=0.8\linewidth]{figures/lcs_figures/euler.pdf}
    \caption[LCS curves found by means of the Euler integration scheme]{
        LCS curves found by means of the Euler integration scheme. The
        reference LCS, as shown by itself in figure~\ref{fig:referencelcs},
        is plotted on the bottom layer. There is a clearly visible offset
        compared to the reference, for all but the two smallest numerical step
        lengths considered.}
    \label{fig:lcs_euler}
\end{figure}

\clearpage
\begin{figure}[htpb]
    \centering
    \includegraphics[width=0.8\linewidth]{figures/lcs_figures/rk2.pdf}
    \caption[LCS curves found by means of the Heun integration scheme]{
        LCS curves found by means of the Heun integration scheme. The
        reference LCS, as shown by itself in figure~\ref{fig:referencelcs},
        is plotted on the bottom layer. There are clear discrepancies with
        regards to the reference for the two largest numerical time step
        lengths considered.}
    \label{fig:lcs_rk2}
\end{figure}


The Kutta method is \nth{3}-order accurate, and requires three function
evaluations of the right hand side of the ordinary differential
\cref{eq:ivpsystem} or ~\eqref{eq:ivpsystemhigherdimensions} at each time
step. Its Butcher tableau representation can be found in \cref{tab:butcherrk3}.
The classical Runge-Kutta method is \nth{4}-order accurate, and perhaps the most
well-known and frequently used of the four singlestep schemes discussed in this
project. One reason for its popularity is that it is exceptionally stable
numerically (of the aforementioned singlestep methods, the classical
Runge-Kutta method has the largest numerical stability domain). Another is that,
as mentioned previously, for $p\geq5$, no explicit Runge-Kutta method of order
$p$ with $s=p$ stages exist
\parencite[p.173]{hairer1993solving} -- in other words,
the required number of function evaluations grows at a disproportional rate with
the required accuracy order. For systems with right hand sides which are
computationally costly to evaluate, this means that one frequently is able to
obtain the desired numerical accuracy more effectively by using, for instance,
the classical Runge-Kutta method with a finer step length. The Butcher tableau
representation of the classical Runge-Kutta method can be found in
\cref{tab:butcherrk4}.

\begin{figure}[htpb]
    \centering
    \input{figures/lcs_figures/rk3.pgf}
    \caption[LCS curves found by means of the Kutta integration scheme]{
        LCS curves found by means of the Kutta integration scheme. The
        reference LCS, as shown by itself in figure~\ref{fig:referencelcs},
        is dashed on the top layer. There are some disparities with
        regards to the reference both for the two largest numerical
        time step lengths considered. These are most pronounced in the lower
        left corner, near the leftmost `U' shape by $y=0.4$ and the `$\cap$'
        shape near $x=1.25$.}
    \label{fig:lcs_rk3}
\end{figure}

\clearpage
\begin{figure}[htpb]
    \centering
    \includegraphics[width=0.8\linewidth]{figures/lcs_figures/rk4.pdf}
    \caption[LCS curves found by means of the classical Runge-Kutta integration scheme]{
        LCS curves found by means of the classical Runge-Kutta integration scheme. The
        reference LCS, as shown by itself in figure~\ref{fig:referencelcs},
        is plotted on the bottom layer. The only visible discrepancy belongs
        to the second largest numerical time step length considered, and is
        located in the lower left corner.}
    \label{fig:lcs_rk4}
\end{figure}


\subsubsection{Adaptive stepsize methods}
\label{ssub:adaptive_stepsize_methods}

The adaptive stepsize methods under consideration are the Bogacki-Shampine
3(2) and 5(4) methods, and the Dormand-Prince 5(4) and 8(7) methods. The digit
outside of the parentheses indicates the order of the solution which is used
to continue the integration, while the digit within the parentheses indicates
the order of the interpolant solution. Note that the concept of \emph{order}
does not translate directly from singlestep methods, as a direct consequence
of the adaptive time step. Although the \emph{local} errors of each integration
step scale as per~\cref{eq:rungekuttaorder}, the bound on the \emph{global}
(i.e., observable) error suggested in~\cref{eq:rungekuttaglobalorderestimate}
is invalid, as the time step is, in principle, different for each integration
step. Generally, lower order methods are more suitable than higher order methods
for cases where crude approximations of the solution are sufficient.
\citeauthor{bogacki1989pair} argue that their methods
outperform other methods of the same order
\parencite{bogacki1989pair,bogacki1996efficient}, a notion which, for the 5(4)
method, is supported by
\textcite[p.194]{hairer1993solving}.

Butcher tableau representations of the aforementioned adaptive stepsize methods
can be found in
\cref{tab:butcherbs32,tab:butcherbs54,tab:butcherdopri54,tab:butcherdopri87},
the latter of which has been typeset in landscape orientation for the reader's
convenience. Three of the methods, namely the Bogacki-Shampine 3(2) and 5(4)
methods, in addition to the Dormand-Prince 5(4) method, possess the so-called
\emph{First Same As Last} property. This means that the last function evaluation
of an accepted step is exactly the same as the first function evaluation of the
next step. The notions of accepted and rejected integration steps will be
elaborated upon in
\cref{sub:on_the_implementation_of_embedded_runge_kutta_methods}. The
\emph{First Same As Last} property is readily apparent from their Butcher
tableaus, where the $b$ coefficients correspond exactly with the last row of the
Runge-Kutta matrix. This property reduces the computational cost of a successive
step. Moreover, the Bogacki-Shampine 5(4) method yields \emph{two} interpolant
solutions. The details on how these were used, will be presented in
\cref{sub:on_the_implementation_of_embedded_runge_kutta_methods}.
%\clearpage
\begin{figure}[htpb]
    \centering
    \input{figures/lcs_figures/rkbs32.pgf}
    \caption[LCS curves found by means of the Bogacki-Shampine 3(2) integration
    scheme]{
        LCS curves found by means of the Bogacki-Shampine 3(2) integration
        scheme. The reference LCS, as shown in figure
        \ref{fig:referencelcs}, is dashed on the top layer. Note that
        the LCS for the lowest tolerance level considered, that is,
        $\textnormal{tol}=10^{-1}$,
        is not included. This is because the corresponding $\mathcal{U}_{0}$
        domain, shown in~\cref{fig:u0_dom_err_bs32}, and the reference
        $\mathcal{U}_{0}$, shown in~\cref{fig:u0_domain} are very
        dissimilar. The second lowest tolerance level, i.e., $\textnormal{tol}=10^{-2}$,
        is the main culprit among the remaining, as far as discrepancies are
    concerned.}
    \label{fig:lcs_rkbs32}
\end{figure}


%\clearpage
\begin{figure}[htpb]
    \centering
    \input{figures/lcs_figures/rkbs54.pgf}
    \caption[LCS curves found by means of the Bogacki-Shampine 5(4) integration
    scheme]{
        LCS curves found by means of the Bogacki-Shampine 5(4) integration
        scheme. The reference LCS, as shown by itself in figure
        \ref{fig:referencelcs}, is dashed on the top layer. Note that
        the LCS for the lowest tolerance level considered, that is,
        $\textnormal{tol}=0.1$, is not included. This is because the
        corresponding $\mathcal{U}_{0}$ domain, shown in figure
        \ref{fig:u0_dom_err_bs54}, and the reference $\mathcal{U}_{0}$, shown
        in figure~\ref{fig:u0_domain} are very dissimilar. Here, there are visible
        discrepancies for all tolerance levels $\textnormal{tol}>10^{-6}$.
        These are prominent all over the domain.}
    \label{fig:lcs_rkbs54}
\end{figure}


%\clearpage
\begin{figure}[htpb]
    \centering
    \includegraphics[width=0.9\linewidth]{figures/lcs_figures/rkdp54.pdf}
    \caption[LCS curves found by means of the Dormand-Prince 5(4) integration
    scheme]{
        LCS curves found by means of the Dormand-Prince 5(4) integration
        scheme. The reference LCS, as shown by itself in figure
        \ref{fig:referencelcs}, is plotted on the bottom layer. Note that
        the LCS for the lowest tolerance level considered, that is,
        $\textnormal{tol}=0.1$, is not included. This is because the
        corresponding $\mathcal{U}_{0}$ domain, shown in figure
        \ref{fig:u0_dp54}, and the reference $\mathcal{U}_{0}$, shown in figure
        \ref{fig:u0_domain} are dissimilar. Here, there are visible
        disparities for for all tolerance levels $\textnormal{tol}>10^{-6}$.}
    \label{fig:lcs_rkdp54}
\end{figure}


%\clearpage
\begin{figure}[htpb]
    \centering
    \input{figures/lcs_figures/rkdp87.pgf}
    %\resizebox{0.9\linewidth}{!}{\input{figures/lcs_figures/rkdp87.pgf}}
    %\includegraphics[width=0.9\linewidth]{figures/lcs_figures/rkdp87.pdf}
    \caption[LCS curves found by means of the Dormand-Prince 8(7) integration
    scheme]{
        LCS curves found by means of the Dormand-Prince 8(7) integration
        scheme. The reference LCS, as shown by itself in figure
        \ref{fig:referencelcs}, is plotted on the bottom layer. Note that
        the LCS for the lowest tolerance level considered, that is,
        $\textnormal{tol}=10^{-1}$, is not included. This is because the
        corresponding $\mathcal{U}_{0}$ domain, shown in figure
        \ref{fig:u0_dom_err_dp87}, and the reference $\mathcal{U}_{0}$, shown in figure
        \ref{fig:u0_domain} are unlike one another. Here, the most immediately
        discernible dissimilarities emanate from the tolerance level
        $\textnormal{tol}=10^{-2}$. Close inspection, however, reveals
        discrepancies for all tolerance levels $\textnormal{tol}>10^{-6}$,
        particularly in the lower left corner.}
    \label{fig:lcs_rkdp87}
\end{figure}


\clearpage



\section{The type of flow systems considered}
\label{sec:typeofflow}

We consider flow in two-dimensional dynamical systems of the form
\begin{equation}
    \label{eq:odeflowsystem}
\dot{\vct{x}} = \vct{v}(t,\vct{x}),\quad\vct{x}\in\mathcal{U},\quad{}t\in[t_{0},t_{1}],
\end{equation}
i.e., systems defined for the finite time interval $[t_{0},t_{1}]$, on an open,
bounded subset $\mathcal{U}$ of $\mathbb{R}^{2}$. In addition, the velocity
field $\vct{v}$ is assumed to be smooth in its arguments. Depending on the
exact nature of the velocity field $\vct{v}$, analytical particle trajectories,
that is, analytical solutions of system~\eqref{eq:odeflowsystem}, may or may
not be computed. The flow particles are assumed to be infinitesimal and
massless, i.e., non-interacting \emph{tracers} of the overall circulation.

Letting $\vct{x}(t;t_{0},\vct{x}_{0})$ denote the trajectory of a tracer in the
system defined by~\eqref{eq:odeflowsystem}, the flow map is defined as
\begin{equation}
    \label{eq:flowmap}
    \vct{F}_{t_{0}}^{t}(\vct{x}_{0}) = \vct{x}(t;t_{0},\vct{x}_{0}),
\end{equation}
i.e., the flow map describes the mathematical movement of the tracers from
one point in time to another. Generally, the flow map is as smooth as the
velocity field $\vct{v}$ in system \eqref{eq:odeflowsystem}
\parencite{farazmand2012computing}. For sufficiently
smooth velocity fields, the right Cauchy-Green strain tensor field is defined as
\begin{equation}
    \label{eq:cauchygreen}
    \mtrx{C}_{t_{0}}^{t}(\vct{x}_{0}) = %
    {\big(\vct{\nabla}\vct{F}_{t_{0}}^{t}(\vct{x}_{0})\big)}^{\ast} %
    \,\vct{\nabla}\vct{F}_{t_{0}}^{t}(\vct{x}_{0}),
\end{equation}
where $\vct{\nabla}\vct{F}_{t_{0}}^{t}$ denotes the Jacobian matrix of the flow
map $\vct{F}_{t_{0}}^{t}$, and the asterisk refers to the adjoint operation,
which, because the Jacobian $\vct{\nabla}\vct{F}_{t_{0}}^{t}$ is real-valued,
equates to matrix transposition. Component-wise, the Jacobian matrix of a
general vector-valued function $\vct{f}$ is defined as
\begin{equation}
    \label{eq:jacobiancomponent}
    {(\vct{\nabla}\vct{f})}_{i,j} = \pdv{f_{i}}{x_{j}},\quad%
\vct{f}=\vct{f}(\vct{x})=\begin{pmatrix}f_{1}(\vct{x}),f_{2}(\vct{x}),\ldots\end{pmatrix},
\end{equation}
which, for our two-dimensional flow, reduces to
\begin{equation}
    \label{eq:jacobiantotal}
    \renewcommand{\arraystretch}{2.5}
    \vct{\nabla}\vct{f} = \begin{pmatrix} %
        \dfrac{\partial{}f_{1}}{\partial{}x} %
                                    & \dfrac{\partial{}f_{1}}{\partial{}y}\\
        \dfrac{\partial{}f_{2}}{\partial{}x} %
                                    & \dfrac{\partial{}f_{2}}{\partial{}y}%
                        \end{pmatrix}.
\end{equation}

By construction, the Cauchy-Green strain tensor $\mtrx{C}_{t_{0}}^{t}$ is
symmetric and positive definite. Thus, it has two real, positive eigenvalues
and orthogonal, real eigenvectors. Its eigenvalues $\lambda_{i}$ and
corresponding unit eigenvectors $\vct{\xi}_{i}$ are defined by
\begin{equation}
    \label{eq:cauchygreencharacteristics}
    \begin{gathered}
        \mtrx{C}_{t_{0}}^{t}(\vct{x}_{0})\vct{\xi}_{i}(\vct{x}_{0}) %
            = \lambda_{i}\vct{\xi}_{i}(\vct{x}_{0}),%
            \quad\norm{\vct{\xi}_{i}(\vct{x}_{0})}=1,\quad{}i=1,2,\\
        0<\lambda_{1}(\vct{x}_{0})\leq\lambda_{2}(\vct{x}_{0}),
    \end{gathered}
\end{equation}
where, for the sake of notational transparency, the dependence of $\lambda_{i}$
and $\vct{\xi}_{i}$ on $t_{0}$ and $t$ has been suppressed. The geometric
interpretation of equation~\eqref{eq:cauchygreencharacteristics} is that
a fluid element undergoes the most stretching along the $\vct{\xi}_{2}$ axis,
and the least along the $\vct{\xi}_{1}$ axis. This concept is shown in figure
\ref{fig:stretch_and_strain}.

\begin{figure}[htpb]
    \centering
    \def\svgwidth{0.8\linewidth}
    \input{figures/stretch_and_strain.pdf_tex}
    \caption[Geometric interpretation of the eigenvectors of the Cauchy-Green
        strain tensor]{Geometric interpretation of the eigenvectors of the
        Cauchy-Green strain tensor. The central unit cell is stretched and
        deformed under the flow map $\vct{F}_{t_{0}}^{t}(\vct{x}_{0})$. The
        local stretching is the largest in the direction of $\vct{\xi}_{2}$, the
        eigenvector which corresponds to the largest eigenvalue $\lambda_{2}$
        of the Cauchy-Green strain tensor, cf.\ equation
        \eqref{eq:cauchygreencharacteristics}. Along the $\vct{\xi}_{2}$ and
        $\vct{\xi}_{1}$ axes, the stretch factors are $\sqrt{\lambda_{2}}$ and
        $\sqrt{\lambda_{1}}$, respectively.}
    \label{fig:stretch_and_strain}
\end{figure}


Because the stretch factors along the $\vct{\xi}_{1}$ and $\vct{\xi}_{2}$ axes
are given by the square root of the corresponding eigenvalues, for
incompressible flow, the eigenvalues satisfy
\begin{equation}
    \label{eq:cauchygreenincomprlambda}
    \lambda_{1}(\vct{x}_{0})\lambda_{2}(\vct{x}_{0})=1%
        \quad\forall\hskip0.5em\vct{x}_{0}\in\mathcal{U}
\end{equation}
where, for our case, incompressibility is equivalent to the velocity field
$\vct{v}$ being divergence-free (i.e., $\vct{\nabla}\vdot\vct{v}\equiv0$).
%This is due to the tracer particles being massless, per definiton.

\section[Definition of Lagrangian Coherent Structures for two-dimensional flows ]
{Definition of Lagrangian coherent structures for\\\phantom{2.3 }two-dimensional flows}
\label{sec:definition_of_lcss_for_two-dimensional_flow}
Lagrangian coherent structures (henceforth abbreviated to LCSs) can be described
as time-evolving surfaces which shape coherent trajectory patterns in dynamical
systems, defined over a finite time interval \parencite{haller2010variational}.
There are three main types of LCSs, namely \emph{elliptic},
\emph{hyperbolic} and \emph{parabolic}. Roughly speaking, parabolic structures
outline cores of jet-like trajectories, elliptic structures describe vortex
boundaries, whereas hyperbolic structures illustrate overall attractive or
repelling manifolds. As such, hyperbolic LCSs practically act as organizing
centers of observable tracer patterns \parencite{onu2015lcstool}. Because
hyperbolic LCSs provide the most readily applicable insight in terms of
forecasting flow in e.g.\ oceanic currents, such structures have been the focus
of this project.

\clearpage

\subsection{Hyperbolic LCSs}
\label{sub:hyperbolic_lcss}

The identifacion of LCSs for reliable forecasting requires sufficiency and necessity
conditions, supported by mathematical theorems. \textcite{haller2010variational}
derived a variational LCS theory based on the Cauchy-Green strain tensor,
defined by equation~\eqref{eq:cauchygreen}, from which the aforementioned
conditions follow. The immediately relevant parts of Haller's theory
are quoted in
\cref{def:normalrepellence,def:repellinglcs,def:attractinglcs,def:hyperboliclcs}
\parencite{haller2010variational}.\\

\begin{defn}
    \label{def:normalrepellence}
    A \emph{normally repelling material line} over the time interval
    $[t_{0},t_{0}+T]$ is a compact material line segment $\mathcal{M}(t)$
    which is overall repelling, and on which the normal repulsion rate
    is greater than the tangential repulsion rate.
\end{defn}

A \emph{material line} is a smooth curve $\mathcal{M}(t_{0})$ at time $t_{0}$,
which is advected by the flow map, given by equation~\eqref{eq:flowmap}, into
the dynamic material line
$\mathcal{M}(t)=\vct{F}_{t_{0}}^{t}\mathcal{M}(t_{0})$. The required
\emph{compactness} of the material line segment signifies that, in some sense,
it must be topologically well-behaved. That the material line is
\emph{overall repelling} means that nearby trajectories are repelled from,
rather than attracted to, the material line. Lastly, requiring that the
\emph{normal repulsion rate} is greater than the
\emph{tangential repulsion rate} means that nearby trajectories are in fact
driven away from the material line, rather than being stretched
\emph{along with} it due to shear stress.
\\
\begin{defn}
    \label{def:repellinglcs}
    A \emph{repelling LCS} over the time interval $[t_{0},t_{0}+T]$ is a
    normally repelling material line $\mathcal{M}(t_{0})$ whose normal repulsion
    admits a pointwise non-degenerate maximum relative to any nearby material
    line $\widehat{\mathcal{M}}(t_{0})$.\\
\end{defn}
\begin{defn}
    \label{def:attractinglcs}
    An \emph{attracting LCS}  over the time interval $[t_{0},t_{0}+T]$ is
    defined as a repelling LCS over the \emph{backward} time interval
    $[t_{0}+T,t_{0}]$.\\
\end{defn}
\begin{defn}
    \label{def:hyperboliclcs}
    A \emph{hyperbolic} LCS over the time interval $[t_{0},t_{0}+T]$ is a
    \emph{repelling} or \emph{attracting} LCS over the same time interval.
\end{defn}

Note that the above definitions associate LCSs with the time interval $I$ over
which the dynamical system under consideration is known, or, at the very least,
where information regarding the behaviour of tracers, is sought. Generally,
LCSs obtained over a time interval $I$ do not exist over different time
intervals \parencite{farazmand2012computing}.

For two-dimensional flow, the above definitions can be summarized as a set of
mathematical existence criteria, based on the Cauchy-Green strain tensor,
\parencite{haller2010variational,farazmand2011erratum}. These are given in
\cref{thm:lcsexistence}:

\clearpage

\begin{thm}[Sufficient and necessary conditions for LCSs in two-dimensional
    flows]
    \label{thm:lcsexistence}
    Consider a compact material line $\mathcal{M}(t)\subset\mathcal{U}$ evolving
    over the time interval $[t_{0},t_{0}+T]$. $\mathcal{M}(t)$ is a repelling
    LCS over $[t_{0},t_{0}+T]$ if and only if all the following hold for all
    initial conditions $\vct{x}_{0}\in\mathcal{M}(t_{0})$:
    \begin{subequations}
        \label{eq:lcsexistence}
        \begin{align}
            \label{eq:lcsexistence1}
            &\lambda_{1}(\vct{x}_{0})\neq\lambda_{2}(\vct{x}_{0})>1\\
            \label{eq:lcsexistence2}
            &\big\langle\vct{\xi}_{2}(\vct{x}_{0}),%
            \mtrx{H}_{\lambda_{2}}(\vct{x}_{0})\vct{\xi}_{2}(\vct{x}_{0})\big\rangle<0\\
            \label{eq:lcsexistence3}
            &\vct{\xi}_{2}(\vct{x}_{0})\perp\mathcal{M}(t_{0})\\
            \label{eq:lcsexistence4}
            &\big\langle\vct{\nabla}\lambda_{2}(\vct{x}_{0}),%
                \vct{\xi}_{2}(\vct{x}_{0})\big\rangle=0
        \end{align}
    \end{subequations}
\end{thm}
In \cref{thm:lcsexistence}, $\langle\cdot,\cdot\rangle$ denotes the Euclidean
inner product, and $\mtrx{H}_{\lambda_{2}}$ denotes the Hessian matrix of the
set of largest eigenvalues of the Cauchy-Green strain tensor. Component-wise,
the Hessian matrix of a general, smooth, scalar-valued function $f$ is defined
as
\begin{equation}
    \label{eq:hessiancomponent}
    {(\mtrx{H}_{f})}_{i,j} = \pdv[2]{f}{x_{i}}{x_{j}},
\end{equation}
which, for our two-dimensional flow, reduces to
\begin{equation}
    \label{eq:hessiantotal}
    \renewcommand{\arraystretch}{2.5}
    \mtrx{H}_{f} = \begin{pmatrix}%
    \dfrac{\partial{}^{2}f}{\partial{x}^{2}} & %
                \dfrac{\partial{}^{2}f}{\partial{x}\partial{y}} \\
                \dfrac{\partial{}^{2}f}{\partial{y}\partial{x}} & %
                \dfrac{\partial{}^{2}f}{\partial{y}^{2}} \\
            \end{pmatrix}
\end{equation}
Condition~\eqref{eq:lcsexistence1} ensures that the normal repulsion rate is
larger than the tangential stretch due to shear strain along the LCS, as per
\cref{def:normalrepellence}. Conditions~\eqref{eq:lcsexistence3} and
\eqref{eq:lcsexistence4} suffice to ensure that the normal repulsion rate
attains a local extremum along the LCS, relative to all nearby material lines.
Lastly, condition~\eqref{eq:lcsexistence2} forces this to be a strict
local apex.

\section{FTLE fields as predictors for LCSs}
\label{sec:ftle_fields_as_predictors_for_lcss}

Finite-time Lyapunov exponent (hereafter abbreviated to FTLE) fields provide
a measure of the extent to which particles which start out close to eachother,
are separated in a given time interval. Mathematically, Lyapunov exponents
quantify the asymptotic divergence or convergence of trajectories of
tracers which start out infinitesimally close to each other
\parencite[pp.328--330]{strogatz2014nonlinear}. The FTLE field
is intrinsically linked to the Cauchy-Green strain tensor. When a
two-dimensional system is evolved from time $t_{0}$ to $t_{0}+T$, the FTLE field
is defined as
\begin{equation}
    \label{eq:ftledef}
    \sigma(\vct{x}_{0}) = \frac{\ln\big(\lambda_{2}(\vct{x}_{0})\big)}{2\abs{T}},
\end{equation}
\clearpage
where $\lambda_{2}(\vct{x}_{0})$ is the largest eigenvalue of the Cauchy-Green
strain tensor. The absolute value of the integration time $T$ is taken because
one generally can consider the evolution of a system in either temporal
direction. Analogously to \cref{def:attractinglcs}, the attraction of nearby
trajectories can be estimated by FTLEs found by means of integration backwards
in time.

\textcite{shadden2005definition} in fact \emph{define} hyperbolic LCSs as ridges
of the FTLE field. By \emph{ridges}, they mean gradient lines which are
orthogonal to the direction of minimum curvature.
\textcite{haller2010variational} showed, by means of explicit examples, that the
sole use of the FTLE field for LCS detection is prone to both false positives
and false negatives, even for conceptually simple flows. Furthermore, the FTLE
field is generally less well-resolved than the LCSs obtained from Haller's
variational formalism. For these reasons, the FTLE field can generally be used
as a first approximation in terms of where one can reasonably expect LCSs
to be found, but it does not represent the blueprint for the actual LCSs of a
system in general.



\vspace{\fill}

\newpage

\chapter{Method}
\label{cha:method}
In order to investigate the dependence of LCS identification by means of
the variational approach as presented in \cref{sub:hyperbolic_lcss} on
the choice of numerical integration method, cf.\ \cref{sec:solvingsystems},
a system which has been studied extensively in the literature, was chosen.
The system, an unsteady double gyre, has been used frequently as a test case
for locating LCSs from different indicators
\parencite{farazmand2012computing,shadden2005definition}. As a result, the
LCSs the system exhibit are well known.

\section{The double gyre model}
\label{sec:the_double_gyre_model}

The double gyre model is defined as a pair of counter-rotating gyres, with a
time-periodic perturbation. The perturbation can be interpreted as a solid, as
in impenetrable, wall which oscillates periodically, which causes the gyres
to periodically contract and expand. In terms of the cartesian coordinate vector
$\vct{x}=(x,y)$, the system can be expressed mathematically as

\begin{equation}
    \label{eq:doublegyre}
    \renewcommand{\arraystretch}{2.5}
    \dot{\vct{x}} =\vct{v}(t,\vct{x})= \pi{}A\begin{pmatrix}%
        -\sin\big(\pi{}f(t,x)\big)\cos(\pi{}y)\\
        \cos\big(\pi{}f(t,x)\big)\sin(\pi{}y)\dfrac{\partial{}f(t,x)}{\partial{}x}
    \end{pmatrix}
\end{equation}

where

\begin{equation}
    \label{eq:doublegyrefuns}
    \begin{gathered}
        f(t,x) = a(t)x^{2} + b(t)x\\
        a(t) = \epsilon\sin(\omega{}t)\\
        b(t) = 1-2\epsilon\sin(\omega{}t)
    \end{gathered}
\end{equation}

and the parameters $A$, $\epsilon$ and $\omega$ dictate the nature of the
flow pattern. As in the literature, the parameter values

\begin{equation}
    \label{eq:doublegyreparams}
    \begin{gathered}
        A = 0.1\\
        \epsilon=0.1\\
        \omega=\frac{2\pi}{10}
    \end{gathered}
\end{equation}

were used \parencite{farazmand2012computing,shadden2005definition}. Moreover,
the starting time was $t_{0}=0$, and the integration time was $T=20$, i.e.,
forcing two periods of motion, cf.
\eqref{eq:doublegyreparams}.

Note that the velocity field $\vct{v}(t,\vct{x})$ in equation
\eqref{eq:doublegyre} can be expressed in terms of a scalar stream function:

\begin{equation}
    \label{eq:doublegyrestreamfun}
    \renewcommand{\arraystretch}{2.5}
    \begin{gathered}
        \psi(t,\vct{x}) = A\sin\big(\pi{}f(t,x)\big)\sin(\pi{}y) \\
        \vct{v}(t,\vct{x}) = \begin{pmatrix}%
            -\dfrac{\partial{}\psi}{\partial{}y} \\
            \dfrac{\partial{}\psi}{\partial{}x}
        \end{pmatrix}
    \end{gathered}
\end{equation}

which means that the velocity field is divergence-free by construction:

\begin{equation}
    \label{eq:doublegyreincompr}
    \vct{\nabla}\vdot\vct{v}(t,\vct{x}) = -\pdv[2]{\psi}{x}{y} %
                                        + \pdv[2]{\psi}{y}{x} = 0
\end{equation}

where the latter equality follows from Schwartz' theorem of mixed partial
derivatives, as the stream function is smooth. This means that we expect the
property given in equation~\eqref{eq:cauchygreenincomprlambda} to hold for the
double gyre flow.



\section{Advecting a set of initial conditions}
\label{sec:advecting_a_set_of_initial_conditions}

The variational model is based upon the advection of non-interacting tracers,
cf. \cref{sec:typeofflow}, by the velocity field defined in equation
\eqref{eq:doublegyre}. The system has no known analytical solution for the
tracer trajectories. Thus, it must be solved numerically, by means of some
numerical integration method, e.g.\ a Runge-Kutta method, cf.\
\cref{sub:the_runge_kutta_family_of_numerical_methods}. With the main focus
of this project being the dependence on LCSs on the chosen integration method,
the advection was performed using all of the numerical integrators introduced
in \cref{sub:the_runge_kutta_methods_under_consideration}.

\subsection{Generating a set of initial conditions}
\label{sub:generating_a_set_of_initial_conditions}
The computational domain $\mathcal{U}=[0\hspace{1ex}2]\times[0\hspace{1ex}1]$
was discretized by a set of linearly spaced tracers, with $1000\times500$ grid
points, effectively creating a nearly uniform grid of approximate spacing
$\Delta{x}\simeq\Delta{y}\simeq0.002$. Tracers were placed on, and within, the
domain boundaries of $\mathcal{U}$. The grid was extended artificially,
with an additional two rows or columns appended to all of the domain edges,
with the same grid spacing as the \emph{main} grid. This was done in order to
ensure that the dynamics at the domain boundaries were included in the analysis
to follow. The extended grid thus had a total of $1004\times504$ grid points.
The construction of the grid is illustrated in figure~\ref{fig:initialgrid}.

\vfill{}

\begin{figure}[htpb]
    \centering
    \def\svgwidth{0.8\linewidth}{\input{figures/initial_grid.pdf_tex}}
    \caption[Illustration of the set of initial conditions]
        {Illustration of the set of initial conditions.
                Dark grey blobs signify the main tracers, i.e., the tracers
                which discretize the computational domain
            $[0\hspace{1ex}2]\times[0\hspace{1ex}1]$. These were linearly
        spaced in either direction, with twice as many points in the $x$-
        direction as the $y$-direction, in order to generate an approximately
        equidistant grid. Light grey blobs
        signify the artificially extended grid, i.e., tracers starting
        originating outside of the computational domain. These were used in
        order to properly encapsulate the dynamics at the domain boundaries,
        in the analysis to follow.}
    \label{fig:initialgrid}
\end{figure}


In order to increase the resolution of the Cauchy-Green strain tensor,
it is necessary to increase the accuracy with which one computes the
Jacobian of the flow map, cf.\ equation \eqref{eq:cauchygreen}. This was done
by advecting a set of auxiliary tracer points surrounding each main point. To
each tracer point $\vct{x}_{j}=(x_{j},y_{j})$, neighboring points defined as

\begin{equation}
    \label{eq:auxgrid}
    \begin{gathered}
        \vct{x}_{j}^{r} = (x_{j}+\delta{x},y_{j}),%
                \quad\vct{x}_{j}^{l} = (x_{j}-\delta{x},y_{j})\\
                \vct{x}_{j}^{u} = (x_{j},y_{j}+\delta{y}),%
                \quad\vct{x}_{j}^{l} = (x_{j},y_{j}-\delta{y})\\
\end{gathered}
\end{equation}

where $\delta{x}$ and $\delta{y}$ are increments smaller than the grid spacings
$\Delta{x}\simeq\Delta{y}$, were assigned. Even though this effectively means
that five times as many particles have to be advected, the resulting accuracy
in computing the Jacobian by means of the auxiliary tracers is determined
by the independent variables $\delta{x}$ and $\delta{y}$, theoretically allowing
for arbitrary precision. The concept of the auxiliary tracers is illustrated in
figure~\ref{fig:auxiliarygrid}.

\begin{figure}[htpb]
    \centering
    \resizebox{0.8\linewidth}{!}{\input{figures/aux_grid.pdf_tex}}
    \caption[Illustration of the concept of auxiliary tracers]
    {Illustration of the concept of auxilary tracers, used in order to
    compute the Jacobian of the flow map, and by extension, the Cauchy-Green
    strain tensor field, cf.\ equation~\eqref{eq:cauchygreen}, more accurately.
    Grey blobs represent the original tracers, whereas white blobs represent
    the auxiliary ones.}
    \label{fig:auxiliarygrid}
\end{figure}


Because of the limited number of decimal digits which can be represented by
floating point numbers, however, there is a strict lower limit to which it makes
sense to lower vaues of the degrees of freedom $\delta{x}$ and $\delta{y}$. In
particular, the smallest number which can be resolved by double-precision
floating point numbers is of the order $10^{-16}$. When decreasing the auxiliary
grid spacing, the increase in precision is quickly offset by the fact that
one automatically gets allocated a smaller number of decimal digits with
which one can calculate the discrete approximation of the derivatives involved
in the Jacobian. This is due to the double gyre velocity field, cf.\
equation \eqref{eq:doublegyre}, being well-behaved, leading most tracers which
are close together initially to follow very similar trajectories, often
ending up with a separation distance comparable to the initial offset. For this
reason, the auxiliary grid spacing $\delta{x}=\delta{y}=10^{-5}$ was chosen
--- three orders of magnitude smaller than the original grid spacing, ensuring
that the derivatives in the Jacobian are far more well-resolved than for the
main tracers, while also leaving approximately 10 decimal digits for which
there can be a difference in the final positions of the auxiliary tracers.

\subsection{On the choice of numerical step lengths and tolerance levels}
\label{sub:on_the_choice_of_numerical_step_lengths_and_tolerance_levels}

For the fixed stepsize integrators, step lengths of $10^{-1}$ through to
$10^{-5}$ were used. For a step length of $10^{-5}$, the total number of
integration steps required in order to advect the system from $t_{0}=0$ to
$t=20$ is of order $10^{6}$. Because the inherent accuracy of double precision
floating-point numbers is of order $10^{-16}$, as mentioned previously, the
total floating point error expected to arise when performing the integration
for a step length of $10^{-5}$ is of order $10^{-10}$. The least accurate of
the fixed stepsize integrators under consideration, the Euler method, is a
\nth{1} order accurate globally, meaning that its local error is of \nth{2}
order in the time step, cf.\
\cref{def:rungekuttaorder}. Thus, we expect that the local error of the Euler
method to be of order $10^{-10}$, i.e., the same order of the accumulated
floating-point errors. Reducing the time step further necessarily leads to
an increase in the accumulated floating-point errors, meaning that we cannot
reasonably expect more accurate results for the Euler method --- at the very
least, a time step of $10^{-5}$ appears to be a point after which there is
little to be gained in terms of numerical accuracy for the Euler method by
lowering the time step further. For the other fixed stepsize integrators, which
are of higher order, we expect this breaking point to occur for a (somewhat)
larger time step.

While the above logic does not translate directly for the adaptive stepsize
integrators, empirical tests indicate that for the Bogacki-Shampine integrators,
as well as the Dormand-Prince 5(4) integrator, the accumulated floating point
errors caught up to the required tolerance level at some point between the
levels $10^{-10}$ and $10^{-11}$, while the Dormand-Prince 8(7) integrator held
its ground until about $10^{-13}$. For this reason, tolerance levels of
$10^{-1}$ through to $10^{-10}$ were used for the adaptive stepsize integrators.

As previously mentioned, reference solutions must be obtained by means of a
high order fixed stepsize method with a small step length, alternatively a high
order adaptive stepsize method with a small tolerance level. In this case, the
latter approach was chosen, and the solution obtained via the Dormand-Prince
8(7) integrator with a numerical tolerance of $10^{-12}$ was used as the
reference.





\vspace{\fill}

\newpage

%\chapter{Results and Discussion}
%\label{cha:results_and_discussion}
%\section{The LCS curves obtained using the different schemes}
\label{sec:the_lcs_curves_obtained_using_the_different_schemes}

Here, the repelling LCS curves found by means of the variety of
numerical integration schemes under consideration, are presented. All of the
LCS curves obtained for a given integrator, for all numerical time step lengths
or tolerance levels are included in one figure, where the reference LCS, shown
in figure~\ref{fig:referencelcs}, is also included in order to facilitate
visual comparison. The idea is that any LCS curve that deviates from the
reference, will reveal itself by not conforming perfectly. The LCS curves
resulting from the singlestep methods are presented in
\cref{fig:lcs_euler,fig:lcs_rk2,fig:lcs_rk3,fig:lcs_rk4}, whereas the ones
found by virtue of the embedded, that is, adaptive stepsize, methods, are
shown in~\cref{fig:lcs_rkbs32,fig:lcs_rkbs54,fig:lcs_rkdp54,fig:lcs_rkdp87}.

\subsection{LCS curves stemming from singlestep methods}
\label{sub:lcs_curves_stemming_from_singlestep_methods}



\begin{figure}[htpb]
    \centering
    \includegraphics[width=0.8\linewidth]{figures/lcs_figures/euler.pdf}
    \caption[LCS curves found by means of the Euler integration scheme]{
        LCS curves found by means of the Euler integration scheme. The
        reference LCS, as shown by itself in figure~\ref{fig:referencelcs},
        is plotted on the bottom layer. There is a clearly visible offset
        compared to the reference, for all but the two smallest numerical step
        lengths considered.}
    \label{fig:lcs_euler}
\end{figure}

\begin{figure}[htpb]
    \centering
    \includegraphics[width=0.8\linewidth]{figures/lcs_figures/rk2.pdf}
    \caption[LCS curves found by means of the Heun integration scheme]{
        LCS curves found by means of the Heun integration scheme. The
        reference LCS, as shown by itself in figure~\ref{fig:referencelcs},
        is plotted on the bottom layer. There are clear discrepancies with
        regards to the reference for the two largest numerical time step
        lengths considered.}
    \label{fig:lcs_rk2}
\end{figure}

\begin{figure}[htpb]
    \centering
    \input{figures/lcs_figures/rk3.pgf}
    \caption[LCS curves found by means of the Kutta integration scheme]{
        LCS curves found by means of the Kutta integration scheme. The
        reference LCS, as shown by itself in figure~\ref{fig:referencelcs},
        is dashed on the top layer. There are some disparities with
        regards to the reference both for the two largest numerical
        time step lengths considered. These are most pronounced in the lower
        left corner, near the leftmost `U' shape by $y=0.4$ and the `$\cap$'
        shape near $x=1.25$.}
    \label{fig:lcs_rk3}
\end{figure}

\begin{figure}[htpb]
    \centering
    \includegraphics[width=0.8\linewidth]{figures/lcs_figures/rk4.pdf}
    \caption[LCS curves found by means of the classical Runge-Kutta integration scheme]{
        LCS curves found by means of the classical Runge-Kutta integration scheme. The
        reference LCS, as shown by itself in figure~\ref{fig:referencelcs},
        is plotted on the bottom layer. The only visible discrepancy belongs
        to the second largest numerical time step length considered, and is
        located in the lower left corner.}
    \label{fig:lcs_rk4}
\end{figure}


The LCS curves obtained from the Euler scheme using relatively large step
lengths are the most obviously incorrect ones, in comparison to the reference.
There appears to be very little separating the performance of the other
singlestep methods. Interestingly, the Heun scheme appears to yield more
accurate results than the Kutta scheme for all the considered step lengths.
This is somewhat unexpected, because, as mentioned in
\cref{sub:the_runge_kutta_methods_under_consideration}, the Heun scheme is
\nth{2}-order accurate in the time step, whereas the Kutta scheme is
\nth{3}-order accurate. On a less surprising note, seeing as it is \nth{4}-order
accurate, the classical Runge-Kutta scheme produces very accurate curves for
all step lengths.

\subsection{LCS curves stemming from adaptive stepsize methods}
\label{sub:lcs_curves_stemming_from_adaptive_stepsize_methods}

Note that, although numerical tolerance levels of $10^{-1}$ through to
$10^{-10}$ were investigated, the figures presented here do not show LCS curves
for $\textnormal{tol}=10^{-1}$. This is due to the fact that, for all
embedded methods, none of the resulting domains $\mathcal{U}_{0}$ bore
a great amount of resemblance to the reference. This resulted in very disparate
behaviour of strainlines, to the extent that the resulting strain systems were
too dissimilar to the reference to warrant any meaningful comparisons. The
erroneous $\mathcal{U}_{0}$ domains are shown in~\cref{fig:u0_dom_errs}
(compare to~\cref{fig:u0_domain}).
%In fact, most likely because of the large density of points in these domains,
%keeping track of the neighbors for each individual strainline and its
%intersections with the lines in $\mathcal{L}$ turned out to be an unconquerable
%task. It is known that storing the necessary data requires in excess of 28 GB
%of memory, at which point even the NTNU supercomputer, Vilje, proved
%insufficient.

Notably, all of the embedded integration methods result in LCSs which are
dissimilar from the reference for the tolerance level $\textnormal{tol}=10^{-2}$,
where only the curve obtained by the Dormand-Prince 8(7) scheme is anything
alike the reference. More curiously, the Bogacki-Shampine 3(2) method appears
to outperform its theoretically more accurate sibling --- the
Bogacki-Shampine 5(4) method --- even for relatively
large tolerance levels. This is visible in the
vicinity of the `$\cup$' shape near $y=0.4$ in~\cref{fig:lcs_rkbs32,fig:lcs_rkbs54},
for instance. Furthermore, the Dormand-Prince 5(4) method appears to yield more
accurate LCS curves in general, than the Bogacki-Shampine 5(4) method. This
is somewhat surprising, given that the latter scheme is supposedly more accurate
than other methods of similar order, as mentioned in
\cref{sub:the_runge_kutta_methods_under_consideration}. Lastly, the
Dormand-Prince 8(7) method appears to produce very accurate results for all
tolerance levels smaller than $10^{-2}$, and, as previously mentioned, gives
the most correct LCS curve for the tolerance level $10^{-2}$
--- which is to be expected, as it is the highest order method among the ones
considered, after all.
\vspace{\fill}

\begin{figure}[htpb]
    \centering
    \begin{subfigure}[b]{0.475\textwidth}
        \centering
        \includegraphics{figures/domain_figures/rkbs32_err_half_width.png}
        \caption[]{{\small Bogacki-Shampine 3(2)}}
        \label{fig:u0_dom_err_bs32}
    \end{subfigure}
    \begin{subfigure}[b]{0.475\textwidth}
        \centering
        \includegraphics{figures/domain_figures/rkbs54_err_half_width.png}
        \caption[]{{\small Bogacki-Shampine 5(4)}}
        \label{fig:u0_dom_err_bs54}
    \end{subfigure}

    \begin{subfigure}[b]{0.475\textwidth}
        \centering
        \includegraphics{figures/domain_figures/rkdp54_err_half_width.png}
        \caption[]{{\small Dormand-Prince 5(4)}}
        \label{fig:u0_dom_err_dp54}
    \end{subfigure}
    \begin{subfigure}[b]{0.475\textwidth}
        \centering
        \includegraphics{figures/domain_figures/rkdp87_err_half_width.png}
        \caption[]{{\small Dormand-Prince 8(7)}}
        \label{fig:u0_dom_err_dp87}
    \end{subfigure}
    \caption[The $\mathcal{U}_{0}$ domains obtained with the adaptive stepsize
    integration schemes, with numerical tolerance level
    $\textnormal{tol}=10^{-1}$]{
        The $\mathcal{U}_{0}$ domains obtained with the adaptive stepsize
        integration schemes, with numerical tolerance level
        $\textnormal{tol}=10^{-1}$. Out of the four schemes considered, only the
    Dormand-Prince 8(7) method is remotely close to reproducing the
    reference domain, shown in~\ref{fig:u0_domain}. In any case, the
    differences are sufficiently large to alter the resulting strain
    system beyond recognition, rendering LCS comparisons with the reference
    moot.}
    \label{fig:u0_dom_errs}
\end{figure}



\begin{figure}[htpb]
    \centering
    \input{figures/lcs_figures/rkbs32.pgf}
    \caption[LCS curves found by means of the Bogacki-Shampine 3(2) integration
    scheme]{
        LCS curves found by means of the Bogacki-Shampine 3(2) integration
        scheme. The reference LCS, as shown in figure
        \ref{fig:referencelcs}, is dashed on the top layer. Note that
        the LCS for the lowest tolerance level considered, that is,
        $\textnormal{tol}=10^{-1}$,
        is not included. This is because the corresponding $\mathcal{U}_{0}$
        domain, shown in~\cref{fig:u0_dom_err_bs32}, and the reference
        $\mathcal{U}_{0}$, shown in~\cref{fig:u0_domain} are very
        dissimilar. The second lowest tolerance level, i.e., $\textnormal{tol}=10^{-2}$,
        is the main culprit among the remaining, as far as discrepancies are
    concerned.}
    \label{fig:lcs_rkbs32}
\end{figure}

%\clearpage
\begin{figure}[htpb]
    \centering
    \input{figures/lcs_figures/rkbs54.pgf}
    \caption[LCS curves found by means of the Bogacki-Shampine 5(4) integration
    scheme]{
        LCS curves found by means of the Bogacki-Shampine 5(4) integration
        scheme. The reference LCS, as shown by itself in figure
        \ref{fig:referencelcs}, is dashed on the top layer. Note that
        the LCS for the lowest tolerance level considered, that is,
        $\textnormal{tol}=0.1$, is not included. This is because the
        corresponding $\mathcal{U}_{0}$ domain, shown in figure
        \ref{fig:u0_dom_err_bs54}, and the reference $\mathcal{U}_{0}$, shown
        in figure~\ref{fig:u0_domain} are very dissimilar. Here, there are visible
        discrepancies for all tolerance levels $\textnormal{tol}>10^{-6}$.
        These are prominent all over the domain.}
    \label{fig:lcs_rkbs54}
\end{figure}

\begin{figure}[htpb]
    \centering
    \includegraphics[width=0.9\linewidth]{figures/lcs_figures/rkdp54.pdf}
    \caption[LCS curves found by means of the Dormand-Prince 5(4) integration
    scheme]{
        LCS curves found by means of the Dormand-Prince 5(4) integration
        scheme. The reference LCS, as shown by itself in figure
        \ref{fig:referencelcs}, is plotted on the bottom layer. Note that
        the LCS for the lowest tolerance level considered, that is,
        $\textnormal{tol}=0.1$, is not included. This is because the
        corresponding $\mathcal{U}_{0}$ domain, shown in figure
        \ref{fig:u0_dp54}, and the reference $\mathcal{U}_{0}$, shown in figure
        \ref{fig:u0_domain} are dissimilar. Here, there are visible
        disparities for for all tolerance levels $\textnormal{tol}>10^{-6}$.}
    \label{fig:lcs_rkdp54}
\end{figure}


\begin{figure}[htpb]
    \centering
    \input{figures/lcs_figures/rkdp87.pgf}
    %\resizebox{0.9\linewidth}{!}{\input{figures/lcs_figures/rkdp87.pgf}}
    %\includegraphics[width=0.9\linewidth]{figures/lcs_figures/rkdp87.pdf}
    \caption[LCS curves found by means of the Dormand-Prince 8(7) integration
    scheme]{
        LCS curves found by means of the Dormand-Prince 8(7) integration
        scheme. The reference LCS, as shown by itself in figure
        \ref{fig:referencelcs}, is plotted on the bottom layer. Note that
        the LCS for the lowest tolerance level considered, that is,
        $\textnormal{tol}=10^{-1}$, is not included. This is because the
        corresponding $\mathcal{U}_{0}$ domain, shown in figure
        \ref{fig:u0_dom_err_dp87}, and the reference $\mathcal{U}_{0}$, shown in figure
        \ref{fig:u0_domain} are unlike one another. Here, the most immediately
        discernible dissimilarities emanate from the tolerance level
        $\textnormal{tol}=10^{-2}$. Close inspection, however, reveals
        discrepancies for all tolerance levels $\textnormal{tol}>10^{-6}$,
        particularly in the lower left corner.}
    \label{fig:lcs_rkdp87}
\end{figure}





\section{Measures of error}
\label{sec:measures_of_error}

Here, the measures of error introduced in~\cref{sec:estimation_of_errors} for
the different numerical integration schemes considered, are presented. For
each different type of error, two figures are included. One in which the error
of the singlestep methods is shown as a function of the integration step length,
and one in which the error of \emph{all} the integration methods is shown
as a function of the number of times right hand side of the ordinary
differential equation system, that is, the velocity field given by
\cref{eq:doublegyre}, was evaluated for each step length or tolerance level.
The reason for which only the singlestep errors are included in the first set of
figures, is that they are the only methods where the error is expected to scale
as a power of $h$, per \cref{def:rungekuttaorder}. This is naturally not the
case for the adaptive stepsize methods, where the step length varies.

More pertinent information can be found in the second set of figures, where
a well-suited integration method is characterized by generating small
errors at a small cost in terms of the required number of function calls,
in this case, the number of times the velocity field had to be evaluated.
Unlike the singlestep integrators, the advection of each tracer by means
of an adaptive stepsize method in principle requires a different number of
integration steps, and thus function evaluations. Thus, the presented number of
function evaluations is the \emph{average} across all of the advected tracers,
including the number of evaluations associated with rejected trial integration
steps. The number of function evaluations for each step of the considered
Runge-Kutta methods can be found as the number of rows in the Runge-Kutta
matrices of
\cref{tab:butchereuler,tab:butcherrk2,tab:butcherrk3,tab:butcherrk4,%
tab:butcherbs32,tab:butcherbs54,tab:butcherdopri54,tab:butcherdopri87}.
\subsection{Computed deviations in the flow maps}
\label{sub:computed_deviations_in_the_flow_maps}


Figure~\ref{fig:flowmap_err_fixed} indicates that the $\rmsd$ of the flow maps,
as defined in equation~\eqref{eq:rmsdflowmap} follow the expected power laws
of the numerical errors for the various singlestep integrators. This is
as expected, because the flow maps are found by direct application of the
numerical integrators. In addition, the flow map $\rmsd$ seem to exhibit
boundedness from below by the counteracting accumulated floating-point
arithmetic error, as hypothesized in
\cref{sub:on_the_choice_of_numerical_step_lengths_and_tolerance_levels}.
This effect is most prominent in the error curves of the Kutta and
classical Runge-Kutta schemes, which inflect upwards for sufficiently small
step lengths. Based on this evidence, the chosen step lengths appear to be
appropriate. Although neither the $\rmsd$ of the Euler scheme nor that of the
Heun scheme appears to have reached their respective inflection points, the
trends nevertheless indicate that there would not be much to gain in terms
of accuracy before the accumulated flating-point errors outweigh the inherent
precisions of the two schemes.

Figure~\ref{fig:flowmap_err_both} indicates that a peak in numerical accuracy
did not materialize for the adaptive stepsize methods, for the considered
numerical tolerance levels. The figure also reveals that the Bogacki-Shampine
3(2) scheme is the `cheapest' in terms of the required number of function
evaluations  when only a very crude approximation is needed, that is, using a
very high numerical tolerance level. Furthermore, the its higher order sibling
is able to keep up with the Dormand-Prince 8(7) method regarding efficiency,
until a mean of approximately $600$ function evaluations. This complies with the
notion that the Bogacki-Shampine 5(4) method is highly optimized and practically
behaves as an even higher order order method, as mentioned in
\cref{sub:the_runge_kutta_methods_under_consideration}.
Interestingly, there does not seem to be a direct correspondence between the
$\rmsd$ of the flow maps, and that of the computed LCS curves, presented in
\cref{sub:computed_deviations_in_the_lcs_curves}.

\input{mainmatter/results/figures/error_figures/flowmap_error_fixed_steplength.tex}

\input{mainmatter/results/figures/error_figures/flowmap_error_both.tex}

\subsection{Computed deviations in the strain eigenvalues and -vectors}
\label{sub:computed_deviations_in_the_strain_eigenvalues_and_vectors}

For brevity, only the $\rmsd$ of $\lambda_{2}$ and the direction of
$\vct{\xi}_{2}$, are included here. The latter is justified because of
the orthogonality of the eigenvectors of the Cauchy-Green strain tensor, which
means that the $\rmsd$ of the directions of both eigenvectors \emph{must} be
identical. Furthermore, the $\rmsd$ of $\lambda_{1}$ scales similarly to
that of $\lambda_{2}$, both as a function of numerical step length (for the
singlestep methods) and the number of function evaluations. Additionally, the
numerical reformulation of the existence theorem for hyperbolic LCSs, given in
\cref{eq:numericalexistence}, exhibits a more sensitive dependence to
$\lambda_{2}$ than $\lambda_{1}$. This is explicitly observable from
conditions~\eqref{eq:numericalexistence2} and~\eqref{eq:numericalexistence4},
and favors $\lambda_{2}$ as the most crucial eigenvalue.

Figure~\ref{fig:lmbd2_err_fixed} indicates that the $\rmsd$ of $\lambda_{2}$,
as defined in~\cref{eq:rmsdlmbd}, follows the anticipated scalings with the
step lengths for the various singlestep integrators. Interestingly, the $\rmsd$
dependence on the numerical step length is completely analogous to that of the
$\rmsd$ of the flow maps, as presented in
\cref{sub:computed_deviations_in_the_flow_maps}. Furthermore, inspection of
figure~\ref{fig:lmbd2_err_both} reveals that the $\rmsd$ of the $\lambda_{2}$
scales similarly to the $\rmsd$ of the flow maps for all of the integration
schemes, differing only by a numerical prefactor. This makes sense, seeing as
the calculation of the Cauchy-Green strain tensor field is based upon
finite differencing applied to the flow map, as described in
\cref{sec:calculating_the_cauchy_green_strain_tensor}.

\input{mainmatter/results/figures/error_figures/lmbd2_error_fixed_steplength.tex}

\input{mainmatter/results/figures/error_figures/lmbd2_error_both.tex}

\vspace{\fill}
\newpage

\Cref{fig:xi2_err_fixed} reveals that the $\rmsd$ of the eigenvector direction,
as defined in~\cref{eq:rmsddirection}, follows the anticipated scalings with
the step lengths for the various singlestep integrators to some degree. The
$\rmsd$ dependence on the numerical step length is not entirely analogous to
that of the $\rmsd$ of $\lambda_{2}$. Notably, the error in the eigenvector
directions is two to three orders of magnitude smaller than the error
in $\lambda_{2}$, for any given step size. This can be understood as a
consequence of using the auxiliary tracers in order to compute the strain
eigenvectors, as described in
\cref{sec:calculating_the_cauchy_green_strain_tensor}, which, by construction,
is more accurate than using the main tracers (which are used to
compute the strain eigenvalues). \Cref{fig:xi2_err_both} shows that the $\rmsd$ of
the eigenvector direction behaves qualitatively like that of the
$\rmsd$ of $\lambda_{2}$ (see~\cref{fig:lmbd2_err_both}) as a function
of the required number of function evaluations for each integration scheme.
Like for the singlestep methods, the $\rmsd$ of the eigenvector direction for
the embedded methods is generally between two and three orders of magnitude
smaller than the $\rmsd$ of $\lambda_{2}$ for any given number of function
evaluations.

%Figure~\ref{fig:xi2_err_fixed} reveals that the $\rmsd$ of the eigenvector
%direction, as defined in~\cref{eq:rmsddirection}, is negligible for
%all numerical step lengths. In fact, the numerical errors are comparable
%to, or even smaller than, the machine epsilon of double-precision floating-point
%numbers, which, as mentioned in
%\cref{sub:generating_a_set_of_initial_conditions}, is of order $10^{-16}$
%\parencite{ieee2008standard}. Because these errors are so small, the agreement
%with the anticipated scalings with the stepsize should not necessarily be taken
%as conclusive evidence of general behaviour. Figure~\ref{fig:xi2_err_both},
%shows that this is in fact the case for the adaptive stepsize methods too,
%aside from the tolerance level $\textnormal{tol}=10^{-1}$, which corresponds to
%the fewest function evaluations for each scheme. As has already been
%established in figure \ref{fig:u0_dom_errs}, for that tolerance level, none of
%the integrators yield $\mathcal{U}_{0}$ domains with reasonable degrees of
%resemblance to the reference, which is shown in figure~\ref{fig:u0_domain}.
%
From \cref{fig:lmbd2_err_both,fig:xi2_err_both}, it is clear that the
$\rmsd$s of $\lambda_{2}$ and the eigenvector direction are both quite large
for the tolerance level $\textnormal{tol}=10^{-1}$ (corresponding to the
smallest number of function evaluations) for all of the embedded methods.
This seems reasonable, given the erroneous $\mathcal{U}_{0}$ domains
obtained for that tolerance level, shown in~\cref{fig:u0_dom_errs} (compare
to~\cref{fig:u0_domain}). Moreover, the sharp turns which are present in
some of the computed LCS curves, perhaps most prominently in
\cref{fig:lcs_rkbs54,fig:lcs_rkdp54} for the tolerance level
$\textnormal{tol}=10^{-2}$, are likely a consequence of the error in the
eigenvalues and -vectors being sufficient for the corresponding
$\mathcal{U}_{0}$ domains to extend to regions in which there are very
strong local orientational discontinuities in the $\vct{\xi}_{1}$-field.
There, the special-purpose linear interpolation introduced in
\cref{sub:a_framework_for_computing_smooth_strainlines} is evidently
insufficient for computing smooth strainlines.

%Aside from the least accurate tolerance level for the embedded integrators,
%though, the error in eigenvector direction is comparable to, or smaller than,
%double-precision machine epsilon. Thus, we may conclude that the eigenvector
%directions are computed correctly for just about any tolerance level and
%numerical time step length. This can be understood as a consequence of the use
%of the auxiliary tracers in order to compute the strain eigenvectors, as
%described in~\cref{sec:calculating_the_cauchy_green_strain_tensor}, which, by
%construction, is more accurate than using the main tracers. Moreover, these
%results imply that the error in the resulting LCS curves are strongly driven by
%the error in the computed eigenvalues. The sharp turns in the computed LCS
%curves for the tolerance level $10^{-2}$, which are particularly visible
%in~\cref{fig:lcs_rkbs54,fig:lcs_rkdp54}, are likely a consequence of the
%corresponding $\mathcal{U}_{0}$ domains extending to regions in which
%there are very strong local orientational discontinuities in the
%$\vct{\xi}_{1}$-field, for which the special-purpose linear interpolation
%introduced in~\cref{sub:a_framework_for_computing_smooth_strainlines} is
%evidently insufficient.
%
\input{mainmatter/results/figures/error_figures/xi2_error_fixed_steplength.tex}


\input{mainmatter/results/figures/error_figures/xi2_error_both.tex}

%\clearpage

\subsection{Computed deviations in the LCS curves}
\label{sub:computed_deviations_in_the_lcs_curves}

\input{mainmatter/results/figures/error_figures/lcs_error_fixed_steplength.tex}

\input{mainmatter/results/figures/error_figures/lcs_error_both.tex}

Figure~\ref{fig:lcs_err_fixed} indicates that the $\rmsd$ of the LCS curves,
as defined in equation~\eqref{eq:rmsdlcs}, does not follow the expected power
laws of the numerical errors for the various singlestep integrators. The error
quickly flattens for all integrators except the Euler scheme, which agrees well
with the LCS curves presented in~\cref{fig:lcs_rk2,fig:lcs_rk3,fig:lcs_rk4},
where there are no visible discrepancies with regards to the reference LCS curve
for step lengths smaller than $10^{-2}$. Moreover, the fact that the $\rmsd$
of the curves obtained by means of the Euler method appears to have flattened
at a higher level than the rest of the integrators implies that the Euler
method will never result in LCS curves with similar degrees of accuracy
as for the other schemes.

Interestingly, a similar pattern is apparent in figure
\ref{fig:lcs_err_both}, in that, the $\rmsd$ of the higher-order methods
decreases steadily, until a certain point where it suddenly jumps to the same
steady level as the (high order) singlestep methods. The most reasonable
explanation is that when the number of function evaluations reaches some
threshold, the accumulated floating-point arithmetic errors gain a larger
influence than the inherent precision of the integrators. By similar logic, the
sudden drop in $\rmsd$ for the high order singlestep methods with numerical
step length $h=10^{-5}$, apparent in both
\cref{fig:lcs_err_fixed,fig:lcs_err_both}, seems to be an artifact of the random
nature of the accumulated floating-point errors, moreso than a sudden
re-manifestation of the inherent integrator accuracy.




%\section{General remarks}
\label{sec:general_remarks}

First off, the computed LCS used as the reference, shown in figure
\ref{fig:referencelcs}, is made up of \emph{seven} different strainline segments.
The LCS presented in the article by \textcite{farazmand2012computing} is claimed
to consist of a \emph{single} strainline segment. Comparing the two curves
visually, however, indicates that the resulting LCSs are similar. Likewise, the
domain $\mathcal{U}_{0}$, shown in figure~\ref{fig:u0_domain}, strongly
resembles the one found by \citeauthor{farazmand2012computing}. Nevertheless,
the total number of points in the domain computed here is approximately two
percent larger than what \citeauthor{farazmand2012computing} found. These
discrepancies could originate from different conventions in terms of generating
the grid of tracers. Notably, \citeauthor{farazmand2012computing} fail to
provide a description of their approach.

Overall, the use of a variational framework for computing LCSs appears to
produce robust and consistent LCS curves for all of the numerical integration
schemes considered here, subject to usage of a sufficiently small numerical
step length or tolerance level. This indicates that calculations of this kind
are not particularly sensitive to integration method. Note that the
$\mathcal{U}_{0}$ domains obtained by means of the adaptive stepsize methods
for $\textnormal{tol}=10^{-1}$, shown in figure~\ref{fig:u0_dom_errs}, differ
greatly to the \emph{correct} domain, which correlates well with the computed
$\rmsd$ of the flow maps, shown in figure~\ref{fig:flowmap_err_both}. In
particular, the error for the aforementioned tolerance level is of order $1$,
which is comparable to the extent of the numerical domain, that is,
$[0\hspace{1ex}2]\times[0\hspace{1ex}1]$. Naturally, this leads to a drastically
different system.

The above observation implies that the most crucial part of the computation is
advecting the tracers accurately. As mentioned in
\cref{sub:computed_deviations_in_the_strain_eigenvalues_and_vectors}, the error
in the computed strain eigenvalues scales like the error in the computed flow
map. This is to be expected, as the eigenvalues are found by applying finite
differences to the \emph{main} tracers. The eigenvalues are essential in terms
of identifying LCSs, as can be seen in the numerical reformulation of the
LCS existence theorem, which is given in equation~\eqref{eq:numericalexistence}.
The strain eigendirections are important too, of course --- however, the
error in the computed strain eigendirections is consistently negligible.
This is most likely due to them being computed based on the denser
\emph{auxiliary} set of tracers, which per construction results in more accurate
finite differences.

The double gyre model considered in this project is clearly not representative
of a generic system, in terms of exact numerical step lengths or tolerance
levels which are sufficient in order to obtain correct LCSs with a certain
degree of confidence. It does, however, indicate that these quantities should
be chosen based on the considered system. For a fixed timestep integration
scheme, any single integration time step should not be so large that \emph{too}
much detail in the local and instantaneous velocity profile is glossed over.
Similar logic applies when adaptive stepsize methods are used, although it
may be more difficult to enforce, depending on how the step length
update is implemented. One possibility in terms of choosing the time step, is to
find a characteristic velocity for the system, and choose the time step small
enough so that a tracer moving with the characteristic velocity never skips over
a grid cell entirely, when moving from one time level to the next.

When computing transport based on discrete data sets, such as snapshots of the
instantaneous velocity profiles in oceanic currents, spatial and temporal
interpolation becomes necessary. Together with the inherent precision of the
measurement data, the choice of interpolation scheme(s) set a lower bound
in terms of the accucacy with which tracers can be advected. For such cases,
the interaction between the integration and interpolation schemes could
be critical --- both in terms of computation time and memory requirements,
aside from the numerical precision. Independently of the scales at which
well-resolved LCS information is sought in this kind of system, the
aforementioned effects warrant further investigation.


\section{On the incompressibility of the velocity field}
\label{sec:on_the_incompressibility_of_the_velocity_field}

As explained in~\cref{sec:the_double_gyre_model}, the double gyre velocity field
is incompressible per construction. Thus, per equation
\eqref{eq:cauchygreenincomprlambda}, the product of the strain eigenvalues
should equal one. Regardless of the integration method used, however, this
property was lost after approximately five units of time. This is not
particularly surprising, because this property only really holds for
infinitesimal fluid elements. Seeing as there is no assurance that neighboring
tracers will remain nearby under the flow map, the finite difference
approximation to the local stretch and strain could practically be rendered
invalid. This issue is expected to be most prominent in regions of high local
repulsion, which are precisely where accuracy is most imperative.

Thus, one should not really expect this property to hold numerically, especially
for an indefinite time. However, sample tests indicate that the
incompressibility property was preserved for longer when using a denser
grid of tracers. In fact, the strain eigenvalue product was found to be unity
beyond 20 units of time, that is, the integration time used in this project,
for an initial grid spacing of $10^{-12}$. However, due to the inherent
inaccuracy of double-precision floating-point numbers, this would leave very
few significant digits with which one could perform finite differencing, as
mentioned in
\cref{sub:on_the_choice_of_numerical_step_lengths_and_tolerance_levels}.
In the end, the grid spacing $\Delta{x}\simeq\Delta{y}\simeq0.02$ was chosen,
because it was the resolution at which repelling LCS curves for the double
gyre system have previously been described in the literature, cf.
\textcite{farazmand2012computing}.

For further analysis, grid spacings of approximate order $10^{-7}$--$10^{-8}$
could be considered, when working with double-precision floating-point numbers.
This would probably result in the incompressibility property being preserved
for longer, while also leaving up to 7 or 8 significant digits for finite
differencing. An entirely different approach, as suggested by
\textcite{onu2015lcstool}, is to simply \emph{define} the smaller strain
eigenvalue as  the reciprocal of the larger one, that is,
$\lambda_{1}\equiv\lambda_{2}^{-1}$. This was not done here, because the
flow incompressibility has no obvious practical consequences for LCSs.
In addition, subject to the accuracy of the numerical integration scheme,
no tracer which starts out within the computational domain
$[0\hspace{1ex}2]\times[0\hspace{1ex}1]$ ever leave it, due to the normal
component of the velocity field being zero along the domain edges. This
follows from the definition of the associated velocity field, given in
equation~\eqref{eq:doublegyre}.


\section{Concerning the numerical representation of tracers}
\label{sec:concerning_the_numerical_representation_of_tracers}

One way of increasing the numerical accuracy in the flow map, would be
to use higher precision floating-point numbers to represent the tracer
coordinates. The main drawback of making such a change, and using e.g.
quadruple-precision floating-point numbers, is that these are, to this date,
not fully implemented in conventional hardware --- even that of the NTNU's
supercomputer, Vilje. Accordingly, quadruple-precision floating-point
arithmetic is several orders of magnitude slower than that of double-precision
floating-point numbers. Coupled with the increase in required memory
(quadruple-precision numbers are stored using twice as many bits as
double-precision numbers), one would generally be forced to perform several
advections of smaller sets of tracers, calculating a piecewise representation
of the overall flow map. However, should this prove impossible, using higher
precision floating-point numbers would lead to a more granular representation
of the flow map. Such an approach is perhaps most sensible for systems with
velocity fields that are well-behaved, or even largely spatially invariant.
Then again, in such cases, calculating LCSs have little practical relevance,
seeing as the overall behaviour of the flow system could be estimated purely by
advecting a smaller set of tracers, or even by simple inspection.

Rather than generating and advecting a fixed, large amount of tracer particles,
another possible approach would be to use a set of fewer `base' tracers, and
making use of an \emph{adaptive multigrid method}. A practical implementation
could involve the dynamic introduction of increasingly finer grids in regions
where the local velocities would have the largest Euclidean norm, for instance.
Such grids, however, would have to move \emph{with} the flow, as the
benefit over simply increasing the initial tracer density would diminish
otherwise. The main reason why this was not implemented for this project,
aside from it not being used in the literature, is that it in all likelyhood
leads to inconsistencies when applying finite differences. That is, unless some
sort of interpolation scheme was applied to the flow map. Seeing as this
project is centered around integration methods, this idea was scrapped.
Regardless, this technique seems promising --- at least on paper --- which
warrants further investigation.
%
%\vspace{\fill}


\section{About the computation of strainlines}
\label{sec:about_the_computation_of_strainlines}

\subsection{The special linear interpolation of strain eigendirections}
\label{sub:the_special_linear_interpolation_of_strain_eigendirections}

\subsection{The use of a single numerical integrator}
\label{sub:the_use_of_a_single_numerical_integrator}






\begin{framed}
    \begin{itemize}
        \item \sout{Approach does produce a consistent LCS picture for all numerical integrators considered, provided sensible
            integration steps or tolerance levels are chosen. This indicates that the calculation of this type
        of transport barrier is not particularly sensitive to the integration method.}
    \item \sout{Most crucial part: Compute advection correctly. The error of the computed strain eigenvalues scales
                like the advection error, while the error in the strain eigendirections is negligible for all time
                steps. The computed LCSs seem to exhibit sensitive dependence on the calculation of the eigenvalues,
            which is not particularly surprising, considering the LCS conditions of eq. 3.13 }
                \begin{itemize}
                    \item \sout{The precision of the strain eigendirections is likely a direct consequence of the
                        auxiliary tracers in their computation.}
                \end{itemize}
            \item \sout{The number of points in the $\mathcal{U}$ domain is different to the one obtained by
                Haller et  al, by approximately $3\%$. This, however, is likely related to how the
            grid of tracers was set up. In their paper, Farazmand and Haller do not provide details.}
        \item \sout{Notably, the reference LCS as shown in figure (.) consists of seven different strainline
                segments, unlike the structure found by Haller et al., which is claimed to be a single
            coherent strainline.}

        \item \sout{The time lenghts or tolerance levels should be chosen based on the system under consideration.
            In particular, an individual time step should not be so large that too much detail in the local,
        instantaneous velocity field is glossed over.}
    \item \sout{Possible approach in order to refine the computations: Use fewer `base' tracers, and an \emph{adaptive multigrid method}. Main con: May lead to inconsistent centered differencing when approaching
        the Jacobian etc. of the flow map}
        \item \sout{Regarding the completely different domains obtained via the embedded methods for the highest tolerance level:
            Check if this corresponds th the error in the flow map larger than some threshold. If (likely) so, this is
        further evidence that one should exhert great effort in computing the flow map correctly.}
    \item \sout{In terms of application to discrete velocity data sets, where both spatial and temporal interpolation
                    may be necessary, the interaction between interpolation and integration scheme is likely to
                    have a great overall impact. In particular, the underlying interpolation scheme sets a lower accuracy
                    bound for the entire advection process, practically enforcing restrictions on the numerical step
                    lengths or tolerance levels which can be considered sensible. This is also based on the considered
                velocity field.}
            \item \sout{Regarding the incompressibility of the velocity field}
                    \begin{itemize}
                        \item \sout{Incompressibility property is conserved until approximately $t=5$ units. Generally can't expect
                            this property to hold numerically over time.}
                        \item \sout{There is no assurance that neighboring tracers will remain nearby after the advection,
                                leaving the finite difference approximation of the local strain and stretch invalid. This
                                issue is expected to be most prominent in regions of high repulsion, which are precisely
                            where accuracy is imperative.}
                    \end{itemize}
        \item Parameter choices
            \begin{itemize}
                \item Numerical step lengths and tolerance levels
                \item The use of RK4 for all strainline iterations
                    \begin{itemize}
                        \item Alternative: Use a high order adaptive step method, paying close attention to the strainline curvature
                    \end{itemize}
                \item \sout{Represent tracer positions by means of higher precision floating-point numbers}
                        \begin{itemize}
                            \item \sout{Yields more accurate finite difference approximations}
                            \item \sout{Requires more memory, i.e., less tracers can be advected, leading to
                                        a more granular flow map representation. Perhaps most relevant for
                                        well-behaved velocity fields, for which the LCS approach is not really
                                        practically relevant --- why not simply advect a few particles and see where they
                                    end up?}
                        \end{itemize}
            \end{itemize}
        \item The identification process of local strain maximizing strainlines
            \begin{itemize}
                \item The cutting of strainline tails
                \item Lines in $\mathcal{L}$ $\rightarrow$ not necessarily a robust approach for general flows
                \item Alternative: Clustering algorithm
                \item Approach where both the strainline lengths and avg $\lambda_{2}$ are taken into consideration,
                    i.e., selecting a longer line with slighly smaller avg $\lambda_{2}$ over a shorter line with larger.
            \end{itemize}
        \item The special linear interpolation used for the eigenvectors
            \begin{itemize}
                \item Possible alternative: Higher order interpolation
            \end{itemize}
            \item Chose mean error, rather than max, because:
                \begin{itemize}
                    \item In terms of flow map: Error should be limited from above, by the domain extent, as the normal component
                        of the velocity at the boundaries is zero
                        \item Generally no way of telling where the maximum error occurs
                \end{itemize}
                \item Regarding the use of $\rmsd$:
                    \begin{itemize}
                        \item A practical application of empirical standard deviation
                            \item The distribution of errors across the domain is unknown. From the Central Limit thm., however,
                                the mean error is nearly distributed as a Gaussian.
                    \end{itemize}
    \end{itemize}
\end{framed}




\chapter{Results}
\label{cha:results}
\section{The LCS curves obtained using the different schemes}
\label{sec:the_lcs_curves_obtained_using_the_different_schemes}

Here, the repelling LCS curves found by means of the variety of
numerical integration schemes under consideration, are presented. All of the
LCS curves obtained for a given integrator, for all numerical time step lengths
or tolerance levels are included in one figure, where the reference LCS, shown
in figure~\ref{fig:referencelcs}, is also included in order to facilitate
visual comparison. The idea is that any LCS curve that deviates from the
reference, will reveal itself by not conforming perfectly. The LCS curves
resulting from the singlestep methods are presented in
\cref{fig:lcs_euler,fig:lcs_rk2,fig:lcs_rk3,fig:lcs_rk4}, whereas the ones
found by virtue of the embedded, that is, adaptive stepsize, methods, are
shown in~\cref{fig:lcs_rkbs32,fig:lcs_rkbs54,fig:lcs_rkdp54,fig:lcs_rkdp87}.

\subsection{LCS curves stemming from singlestep methods}
\label{sub:lcs_curves_stemming_from_singlestep_methods}



\begin{figure}[htpb]
    \centering
    \includegraphics[width=0.8\linewidth]{figures/lcs_figures/euler.pdf}
    \caption[LCS curves found by means of the Euler integration scheme]{
        LCS curves found by means of the Euler integration scheme. The
        reference LCS, as shown by itself in figure~\ref{fig:referencelcs},
        is plotted on the bottom layer. There is a clearly visible offset
        compared to the reference, for all but the two smallest numerical step
        lengths considered.}
    \label{fig:lcs_euler}
\end{figure}

\begin{figure}[htpb]
    \centering
    \includegraphics[width=0.8\linewidth]{figures/lcs_figures/rk2.pdf}
    \caption[LCS curves found by means of the Heun integration scheme]{
        LCS curves found by means of the Heun integration scheme. The
        reference LCS, as shown by itself in figure~\ref{fig:referencelcs},
        is plotted on the bottom layer. There are clear discrepancies with
        regards to the reference for the two largest numerical time step
        lengths considered.}
    \label{fig:lcs_rk2}
\end{figure}

\begin{figure}[htpb]
    \centering
    \input{figures/lcs_figures/rk3.pgf}
    \caption[LCS curves found by means of the Kutta integration scheme]{
        LCS curves found by means of the Kutta integration scheme. The
        reference LCS, as shown by itself in figure~\ref{fig:referencelcs},
        is dashed on the top layer. There are some disparities with
        regards to the reference both for the two largest numerical
        time step lengths considered. These are most pronounced in the lower
        left corner, near the leftmost `U' shape by $y=0.4$ and the `$\cap$'
        shape near $x=1.25$.}
    \label{fig:lcs_rk3}
\end{figure}

\begin{figure}[htpb]
    \centering
    \includegraphics[width=0.8\linewidth]{figures/lcs_figures/rk4.pdf}
    \caption[LCS curves found by means of the classical Runge-Kutta integration scheme]{
        LCS curves found by means of the classical Runge-Kutta integration scheme. The
        reference LCS, as shown by itself in figure~\ref{fig:referencelcs},
        is plotted on the bottom layer. The only visible discrepancy belongs
        to the second largest numerical time step length considered, and is
        located in the lower left corner.}
    \label{fig:lcs_rk4}
\end{figure}


The LCS curves obtained from the Euler scheme using relatively large step
lengths are the most obviously incorrect ones, in comparison to the reference.
There appears to be very little separating the performance of the other
singlestep methods. Interestingly, the Heun scheme appears to yield more
accurate results than the Kutta scheme for all the considered step lengths.
This is somewhat unexpected, because, as mentioned in
\cref{sub:the_runge_kutta_methods_under_consideration}, the Heun scheme is
\nth{2}-order accurate in the time step, whereas the Kutta scheme is
\nth{3}-order accurate. On a less surprising note, seeing as it is \nth{4}-order
accurate, the classical Runge-Kutta scheme produces very accurate curves for
all step lengths.

\subsection{LCS curves stemming from adaptive stepsize methods}
\label{sub:lcs_curves_stemming_from_adaptive_stepsize_methods}

Note that, although numerical tolerance levels of $10^{-1}$ through to
$10^{-10}$ were investigated, the figures presented here do not show LCS curves
for $\textnormal{tol}=10^{-1}$. This is due to the fact that, for all
embedded methods, none of the resulting domains $\mathcal{U}_{0}$ bore
a great amount of resemblance to the reference. This resulted in very disparate
behaviour of strainlines, to the extent that the resulting strain systems were
too dissimilar to the reference to warrant any meaningful comparisons. The
erroneous $\mathcal{U}_{0}$ domains are shown in~\cref{fig:u0_dom_errs}
(compare to~\cref{fig:u0_domain}).
%In fact, most likely because of the large density of points in these domains,
%keeping track of the neighbors for each individual strainline and its
%intersections with the lines in $\mathcal{L}$ turned out to be an unconquerable
%task. It is known that storing the necessary data requires in excess of 28 GB
%of memory, at which point even the NTNU supercomputer, Vilje, proved
%insufficient.

Notably, all of the embedded integration methods result in LCSs which are
dissimilar from the reference for the tolerance level $\textnormal{tol}=10^{-2}$,
where only the curve obtained by the Dormand-Prince 8(7) scheme is anything
alike the reference. More curiously, the Bogacki-Shampine 3(2) method appears
to outperform its theoretically more accurate sibling --- the
Bogacki-Shampine 5(4) method --- even for relatively
large tolerance levels. This is visible in the
vicinity of the `$\cup$' shape near $y=0.4$ in~\cref{fig:lcs_rkbs32,fig:lcs_rkbs54},
for instance. Furthermore, the Dormand-Prince 5(4) method appears to yield more
accurate LCS curves in general, than the Bogacki-Shampine 5(4) method. This
is somewhat surprising, given that the latter scheme is supposedly more accurate
than other methods of similar order, as mentioned in
\cref{sub:the_runge_kutta_methods_under_consideration}. Lastly, the
Dormand-Prince 8(7) method appears to produce very accurate results for all
tolerance levels smaller than $10^{-2}$, and, as previously mentioned, gives
the most correct LCS curve for the tolerance level $10^{-2}$
--- which is to be expected, as it is the highest order method among the ones
considered, after all.
\vspace{\fill}

\begin{figure}[htpb]
    \centering
    \begin{subfigure}[b]{0.475\textwidth}
        \centering
        \includegraphics{figures/domain_figures/rkbs32_err_half_width.png}
        \caption[]{{\small Bogacki-Shampine 3(2)}}
        \label{fig:u0_dom_err_bs32}
    \end{subfigure}
    \begin{subfigure}[b]{0.475\textwidth}
        \centering
        \includegraphics{figures/domain_figures/rkbs54_err_half_width.png}
        \caption[]{{\small Bogacki-Shampine 5(4)}}
        \label{fig:u0_dom_err_bs54}
    \end{subfigure}

    \begin{subfigure}[b]{0.475\textwidth}
        \centering
        \includegraphics{figures/domain_figures/rkdp54_err_half_width.png}
        \caption[]{{\small Dormand-Prince 5(4)}}
        \label{fig:u0_dom_err_dp54}
    \end{subfigure}
    \begin{subfigure}[b]{0.475\textwidth}
        \centering
        \includegraphics{figures/domain_figures/rkdp87_err_half_width.png}
        \caption[]{{\small Dormand-Prince 8(7)}}
        \label{fig:u0_dom_err_dp87}
    \end{subfigure}
    \caption[The $\mathcal{U}_{0}$ domains obtained with the adaptive stepsize
    integration schemes, with numerical tolerance level
    $\textnormal{tol}=10^{-1}$]{
        The $\mathcal{U}_{0}$ domains obtained with the adaptive stepsize
        integration schemes, with numerical tolerance level
        $\textnormal{tol}=10^{-1}$. Out of the four schemes considered, only the
    Dormand-Prince 8(7) method is remotely close to reproducing the
    reference domain, shown in~\ref{fig:u0_domain}. In any case, the
    differences are sufficiently large to alter the resulting strain
    system beyond recognition, rendering LCS comparisons with the reference
    moot.}
    \label{fig:u0_dom_errs}
\end{figure}



\begin{figure}[htpb]
    \centering
    \input{figures/lcs_figures/rkbs32.pgf}
    \caption[LCS curves found by means of the Bogacki-Shampine 3(2) integration
    scheme]{
        LCS curves found by means of the Bogacki-Shampine 3(2) integration
        scheme. The reference LCS, as shown in figure
        \ref{fig:referencelcs}, is dashed on the top layer. Note that
        the LCS for the lowest tolerance level considered, that is,
        $\textnormal{tol}=10^{-1}$,
        is not included. This is because the corresponding $\mathcal{U}_{0}$
        domain, shown in~\cref{fig:u0_dom_err_bs32}, and the reference
        $\mathcal{U}_{0}$, shown in~\cref{fig:u0_domain} are very
        dissimilar. The second lowest tolerance level, i.e., $\textnormal{tol}=10^{-2}$,
        is the main culprit among the remaining, as far as discrepancies are
    concerned.}
    \label{fig:lcs_rkbs32}
\end{figure}

%\clearpage
\begin{figure}[htpb]
    \centering
    \input{figures/lcs_figures/rkbs54.pgf}
    \caption[LCS curves found by means of the Bogacki-Shampine 5(4) integration
    scheme]{
        LCS curves found by means of the Bogacki-Shampine 5(4) integration
        scheme. The reference LCS, as shown by itself in figure
        \ref{fig:referencelcs}, is dashed on the top layer. Note that
        the LCS for the lowest tolerance level considered, that is,
        $\textnormal{tol}=0.1$, is not included. This is because the
        corresponding $\mathcal{U}_{0}$ domain, shown in figure
        \ref{fig:u0_dom_err_bs54}, and the reference $\mathcal{U}_{0}$, shown
        in figure~\ref{fig:u0_domain} are very dissimilar. Here, there are visible
        discrepancies for all tolerance levels $\textnormal{tol}>10^{-6}$.
        These are prominent all over the domain.}
    \label{fig:lcs_rkbs54}
\end{figure}

\begin{figure}[htpb]
    \centering
    \includegraphics[width=0.9\linewidth]{figures/lcs_figures/rkdp54.pdf}
    \caption[LCS curves found by means of the Dormand-Prince 5(4) integration
    scheme]{
        LCS curves found by means of the Dormand-Prince 5(4) integration
        scheme. The reference LCS, as shown by itself in figure
        \ref{fig:referencelcs}, is plotted on the bottom layer. Note that
        the LCS for the lowest tolerance level considered, that is,
        $\textnormal{tol}=0.1$, is not included. This is because the
        corresponding $\mathcal{U}_{0}$ domain, shown in figure
        \ref{fig:u0_dp54}, and the reference $\mathcal{U}_{0}$, shown in figure
        \ref{fig:u0_domain} are dissimilar. Here, there are visible
        disparities for for all tolerance levels $\textnormal{tol}>10^{-6}$.}
    \label{fig:lcs_rkdp54}
\end{figure}


\begin{figure}[htpb]
    \centering
    \input{figures/lcs_figures/rkdp87.pgf}
    %\resizebox{0.9\linewidth}{!}{\input{figures/lcs_figures/rkdp87.pgf}}
    %\includegraphics[width=0.9\linewidth]{figures/lcs_figures/rkdp87.pdf}
    \caption[LCS curves found by means of the Dormand-Prince 8(7) integration
    scheme]{
        LCS curves found by means of the Dormand-Prince 8(7) integration
        scheme. The reference LCS, as shown by itself in figure
        \ref{fig:referencelcs}, is plotted on the bottom layer. Note that
        the LCS for the lowest tolerance level considered, that is,
        $\textnormal{tol}=10^{-1}$, is not included. This is because the
        corresponding $\mathcal{U}_{0}$ domain, shown in figure
        \ref{fig:u0_dom_err_dp87}, and the reference $\mathcal{U}_{0}$, shown in figure
        \ref{fig:u0_domain} are unlike one another. Here, the most immediately
        discernible dissimilarities emanate from the tolerance level
        $\textnormal{tol}=10^{-2}$. Close inspection, however, reveals
        discrepancies for all tolerance levels $\textnormal{tol}>10^{-6}$,
        particularly in the lower left corner.}
    \label{fig:lcs_rkdp87}
\end{figure}





\section{Measures of error}
\label{sec:measures_of_error}

Here, the measures of error introduced in~\cref{sec:estimation_of_errors} for
the different numerical integration schemes considered, are presented. For
each different type of error, two figures are included. One in which the error
of the singlestep methods is shown as a function of the integration step length,
and one in which the error of \emph{all} the integration methods is shown
as a function of the number of times right hand side of the ordinary
differential equation system, that is, the velocity field given by
\cref{eq:doublegyre}, was evaluated for each step length or tolerance level.
The reason for which only the singlestep errors are included in the first set of
figures, is that they are the only methods where the error is expected to scale
as a power of $h$, per \cref{def:rungekuttaorder}. This is naturally not the
case for the adaptive stepsize methods, where the step length varies.

More pertinent information can be found in the second set of figures, where
a well-suited integration method is characterized by generating small
errors at a small cost in terms of the required number of function calls,
in this case, the number of times the velocity field had to be evaluated.
Unlike the singlestep integrators, the advection of each tracer by means
of an adaptive stepsize method in principle requires a different number of
integration steps, and thus function evaluations. Thus, the presented number of
function evaluations is the \emph{average} across all of the advected tracers,
including the number of evaluations associated with rejected trial integration
steps. The number of function evaluations for each step of the considered
Runge-Kutta methods can be found as the number of rows in the Runge-Kutta
matrices of
\cref{tab:butchereuler,tab:butcherrk2,tab:butcherrk3,tab:butcherrk4,%
tab:butcherbs32,tab:butcherbs54,tab:butcherdopri54,tab:butcherdopri87}.
\subsection{Computed deviations in the flow maps}
\label{sub:computed_deviations_in_the_flow_maps}


Figure~\ref{fig:flowmap_err_fixed} indicates that the $\rmsd$ of the flow maps,
as defined in equation~\eqref{eq:rmsdflowmap} follow the expected power laws
of the numerical errors for the various singlestep integrators. This is
as expected, because the flow maps are found by direct application of the
numerical integrators. In addition, the flow map $\rmsd$ seem to exhibit
boundedness from below by the counteracting accumulated floating-point
arithmetic error, as hypothesized in
\cref{sub:on_the_choice_of_numerical_step_lengths_and_tolerance_levels}.
This effect is most prominent in the error curves of the Kutta and
classical Runge-Kutta schemes, which inflect upwards for sufficiently small
step lengths. Based on this evidence, the chosen step lengths appear to be
appropriate. Although neither the $\rmsd$ of the Euler scheme nor that of the
Heun scheme appears to have reached their respective inflection points, the
trends nevertheless indicate that there would not be much to gain in terms
of accuracy before the accumulated flating-point errors outweigh the inherent
precisions of the two schemes.

Figure~\ref{fig:flowmap_err_both} indicates that a peak in numerical accuracy
did not materialize for the adaptive stepsize methods, for the considered
numerical tolerance levels. The figure also reveals that the Bogacki-Shampine
3(2) scheme is the `cheapest' in terms of the required number of function
evaluations  when only a very crude approximation is needed, that is, using a
very high numerical tolerance level. Furthermore, the its higher order sibling
is able to keep up with the Dormand-Prince 8(7) method regarding efficiency,
until a mean of approximately $600$ function evaluations. This complies with the
notion that the Bogacki-Shampine 5(4) method is highly optimized and practically
behaves as an even higher order order method, as mentioned in
\cref{sub:the_runge_kutta_methods_under_consideration}.
Interestingly, there does not seem to be a direct correspondence between the
$\rmsd$ of the flow maps, and that of the computed LCS curves, presented in
\cref{sub:computed_deviations_in_the_lcs_curves}.

\input{mainmatter/results/figures/error_figures/flowmap_error_fixed_steplength.tex}

\input{mainmatter/results/figures/error_figures/flowmap_error_both.tex}

\subsection{Computed deviations in the strain eigenvalues and -vectors}
\label{sub:computed_deviations_in_the_strain_eigenvalues_and_vectors}

For brevity, only the $\rmsd$ of $\lambda_{2}$ and the direction of
$\vct{\xi}_{2}$, are included here. The latter is justified because of
the orthogonality of the eigenvectors of the Cauchy-Green strain tensor, which
means that the $\rmsd$ of the directions of both eigenvectors \emph{must} be
identical. Furthermore, the $\rmsd$ of $\lambda_{1}$ scales similarly to
that of $\lambda_{2}$, both as a function of numerical step length (for the
singlestep methods) and the number of function evaluations. Additionally, the
numerical reformulation of the existence theorem for hyperbolic LCSs, given in
\cref{eq:numericalexistence}, exhibits a more sensitive dependence to
$\lambda_{2}$ than $\lambda_{1}$. This is explicitly observable from
conditions~\eqref{eq:numericalexistence2} and~\eqref{eq:numericalexistence4},
and favors $\lambda_{2}$ as the most crucial eigenvalue.

Figure~\ref{fig:lmbd2_err_fixed} indicates that the $\rmsd$ of $\lambda_{2}$,
as defined in~\cref{eq:rmsdlmbd}, follows the anticipated scalings with the
step lengths for the various singlestep integrators. Interestingly, the $\rmsd$
dependence on the numerical step length is completely analogous to that of the
$\rmsd$ of the flow maps, as presented in
\cref{sub:computed_deviations_in_the_flow_maps}. Furthermore, inspection of
figure~\ref{fig:lmbd2_err_both} reveals that the $\rmsd$ of the $\lambda_{2}$
scales similarly to the $\rmsd$ of the flow maps for all of the integration
schemes, differing only by a numerical prefactor. This makes sense, seeing as
the calculation of the Cauchy-Green strain tensor field is based upon
finite differencing applied to the flow map, as described in
\cref{sec:calculating_the_cauchy_green_strain_tensor}.

\input{mainmatter/results/figures/error_figures/lmbd2_error_fixed_steplength.tex}

\input{mainmatter/results/figures/error_figures/lmbd2_error_both.tex}

\vspace{\fill}
\newpage

\Cref{fig:xi2_err_fixed} reveals that the $\rmsd$ of the eigenvector direction,
as defined in~\cref{eq:rmsddirection}, follows the anticipated scalings with
the step lengths for the various singlestep integrators to some degree. The
$\rmsd$ dependence on the numerical step length is not entirely analogous to
that of the $\rmsd$ of $\lambda_{2}$. Notably, the error in the eigenvector
directions is two to three orders of magnitude smaller than the error
in $\lambda_{2}$, for any given step size. This can be understood as a
consequence of using the auxiliary tracers in order to compute the strain
eigenvectors, as described in
\cref{sec:calculating_the_cauchy_green_strain_tensor}, which, by construction,
is more accurate than using the main tracers (which are used to
compute the strain eigenvalues). \Cref{fig:xi2_err_both} shows that the $\rmsd$ of
the eigenvector direction behaves qualitatively like that of the
$\rmsd$ of $\lambda_{2}$ (see~\cref{fig:lmbd2_err_both}) as a function
of the required number of function evaluations for each integration scheme.
Like for the singlestep methods, the $\rmsd$ of the eigenvector direction for
the embedded methods is generally between two and three orders of magnitude
smaller than the $\rmsd$ of $\lambda_{2}$ for any given number of function
evaluations.

%Figure~\ref{fig:xi2_err_fixed} reveals that the $\rmsd$ of the eigenvector
%direction, as defined in~\cref{eq:rmsddirection}, is negligible for
%all numerical step lengths. In fact, the numerical errors are comparable
%to, or even smaller than, the machine epsilon of double-precision floating-point
%numbers, which, as mentioned in
%\cref{sub:generating_a_set_of_initial_conditions}, is of order $10^{-16}$
%\parencite{ieee2008standard}. Because these errors are so small, the agreement
%with the anticipated scalings with the stepsize should not necessarily be taken
%as conclusive evidence of general behaviour. Figure~\ref{fig:xi2_err_both},
%shows that this is in fact the case for the adaptive stepsize methods too,
%aside from the tolerance level $\textnormal{tol}=10^{-1}$, which corresponds to
%the fewest function evaluations for each scheme. As has already been
%established in figure \ref{fig:u0_dom_errs}, for that tolerance level, none of
%the integrators yield $\mathcal{U}_{0}$ domains with reasonable degrees of
%resemblance to the reference, which is shown in figure~\ref{fig:u0_domain}.
%
From \cref{fig:lmbd2_err_both,fig:xi2_err_both}, it is clear that the
$\rmsd$s of $\lambda_{2}$ and the eigenvector direction are both quite large
for the tolerance level $\textnormal{tol}=10^{-1}$ (corresponding to the
smallest number of function evaluations) for all of the embedded methods.
This seems reasonable, given the erroneous $\mathcal{U}_{0}$ domains
obtained for that tolerance level, shown in~\cref{fig:u0_dom_errs} (compare
to~\cref{fig:u0_domain}). Moreover, the sharp turns which are present in
some of the computed LCS curves, perhaps most prominently in
\cref{fig:lcs_rkbs54,fig:lcs_rkdp54} for the tolerance level
$\textnormal{tol}=10^{-2}$, are likely a consequence of the error in the
eigenvalues and -vectors being sufficient for the corresponding
$\mathcal{U}_{0}$ domains to extend to regions in which there are very
strong local orientational discontinuities in the $\vct{\xi}_{1}$-field.
There, the special-purpose linear interpolation introduced in
\cref{sub:a_framework_for_computing_smooth_strainlines} is evidently
insufficient for computing smooth strainlines.

%Aside from the least accurate tolerance level for the embedded integrators,
%though, the error in eigenvector direction is comparable to, or smaller than,
%double-precision machine epsilon. Thus, we may conclude that the eigenvector
%directions are computed correctly for just about any tolerance level and
%numerical time step length. This can be understood as a consequence of the use
%of the auxiliary tracers in order to compute the strain eigenvectors, as
%described in~\cref{sec:calculating_the_cauchy_green_strain_tensor}, which, by
%construction, is more accurate than using the main tracers. Moreover, these
%results imply that the error in the resulting LCS curves are strongly driven by
%the error in the computed eigenvalues. The sharp turns in the computed LCS
%curves for the tolerance level $10^{-2}$, which are particularly visible
%in~\cref{fig:lcs_rkbs54,fig:lcs_rkdp54}, are likely a consequence of the
%corresponding $\mathcal{U}_{0}$ domains extending to regions in which
%there are very strong local orientational discontinuities in the
%$\vct{\xi}_{1}$-field, for which the special-purpose linear interpolation
%introduced in~\cref{sub:a_framework_for_computing_smooth_strainlines} is
%evidently insufficient.
%
\input{mainmatter/results/figures/error_figures/xi2_error_fixed_steplength.tex}


\input{mainmatter/results/figures/error_figures/xi2_error_both.tex}

%\clearpage

\subsection{Computed deviations in the LCS curves}
\label{sub:computed_deviations_in_the_lcs_curves}

\input{mainmatter/results/figures/error_figures/lcs_error_fixed_steplength.tex}

\input{mainmatter/results/figures/error_figures/lcs_error_both.tex}

Figure~\ref{fig:lcs_err_fixed} indicates that the $\rmsd$ of the LCS curves,
as defined in equation~\eqref{eq:rmsdlcs}, does not follow the expected power
laws of the numerical errors for the various singlestep integrators. The error
quickly flattens for all integrators except the Euler scheme, which agrees well
with the LCS curves presented in~\cref{fig:lcs_rk2,fig:lcs_rk3,fig:lcs_rk4},
where there are no visible discrepancies with regards to the reference LCS curve
for step lengths smaller than $10^{-2}$. Moreover, the fact that the $\rmsd$
of the curves obtained by means of the Euler method appears to have flattened
at a higher level than the rest of the integrators implies that the Euler
method will never result in LCS curves with similar degrees of accuracy
as for the other schemes.

Interestingly, a similar pattern is apparent in figure
\ref{fig:lcs_err_both}, in that, the $\rmsd$ of the higher-order methods
decreases steadily, until a certain point where it suddenly jumps to the same
steady level as the (high order) singlestep methods. The most reasonable
explanation is that when the number of function evaluations reaches some
threshold, the accumulated floating-point arithmetic errors gain a larger
influence than the inherent precision of the integrators. By similar logic, the
sudden drop in $\rmsd$ for the high order singlestep methods with numerical
step length $h=10^{-5}$, apparent in both
\cref{fig:lcs_err_fixed,fig:lcs_err_both}, seems to be an artifact of the random
nature of the accumulated floating-point errors, moreso than a sudden
re-manifestation of the inherent integrator accuracy.





%\vspace{\fill}

%\newpage
\addtocontents{toc}{\vspace{\fill}\protect\pagebreak}
\chapter{Discussion}
\label{cha:discussion}
\section{General remarks}
\label{sec:general_remarks}

First off, the computed LCS used as the reference, shown in figure
\ref{fig:referencelcs}, is made up of \emph{seven} different strainline segments.
The LCS presented in the article by \textcite{farazmand2012computing} is claimed
to consist of a \emph{single} strainline segment. Comparing the two curves
visually, however, indicates that the resulting LCSs are similar. Likewise, the
domain $\mathcal{U}_{0}$, shown in figure~\ref{fig:u0_domain}, strongly
resembles the one found by \citeauthor{farazmand2012computing}. Nevertheless,
the total number of points in the domain computed here is approximately two
percent larger than what \citeauthor{farazmand2012computing} found. These
discrepancies could originate from different conventions in terms of generating
the grid of tracers. Notably, \citeauthor{farazmand2012computing} fail to
provide a description of their approach.

Overall, the use of a variational framework for computing LCSs appears to
produce robust and consistent LCS curves for all of the numerical integration
schemes considered here, subject to usage of a sufficiently small numerical
step length or tolerance level. This indicates that calculations of this kind
are not particularly sensitive to integration method. Note that the
$\mathcal{U}_{0}$ domains obtained by means of the adaptive stepsize methods
for $\textnormal{tol}=10^{-1}$, shown in figure~\ref{fig:u0_dom_errs}, differ
greatly to the \emph{correct} domain, which correlates well with the computed
$\rmsd$ of the flow maps, shown in figure~\ref{fig:flowmap_err_both}. In
particular, the error for the aforementioned tolerance level is of order $1$,
which is comparable to the extent of the numerical domain, that is,
$[0\hspace{1ex}2]\times[0\hspace{1ex}1]$. Naturally, this leads to a drastically
different system.

The above observation implies that the most crucial part of the computation is
advecting the tracers accurately. As mentioned in
\cref{sub:computed_deviations_in_the_strain_eigenvalues_and_vectors}, the error
in the computed strain eigenvalues scales like the error in the computed flow
map. This is to be expected, as the eigenvalues are found by applying finite
differences to the \emph{main} tracers. The eigenvalues are essential in terms
of identifying LCSs, as can be seen in the numerical reformulation of the
LCS existence theorem, which is given in equation~\eqref{eq:numericalexistence}.
The strain eigendirections are important too, of course --- however, the
error in the computed strain eigendirections is consistently negligible.
This is most likely due to them being computed based on the denser
\emph{auxiliary} set of tracers, which per construction results in more accurate
finite differences.

The double gyre model considered in this project is clearly not representative
of a generic system, in terms of exact numerical step lengths or tolerance
levels which are sufficient in order to obtain correct LCSs with a certain
degree of confidence. It does, however, indicate that these quantities should
be chosen based on the considered system. For a fixed timestep integration
scheme, any single integration time step should not be so large that \emph{too}
much detail in the local and instantaneous velocity profile is glossed over.
Similar logic applies when adaptive stepsize methods are used, although it
may be more difficult to enforce, depending on how the step length
update is implemented. One possibility in terms of choosing the time step, is to
find a characteristic velocity for the system, and choose the time step small
enough so that a tracer moving with the characteristic velocity never skips over
a grid cell entirely, when moving from one time level to the next.

When computing transport based on discrete data sets, such as snapshots of the
instantaneous velocity profiles in oceanic currents, spatial and temporal
interpolation becomes necessary. Together with the inherent precision of the
measurement data, the choice of interpolation scheme(s) set a lower bound
in terms of the accucacy with which tracers can be advected. For such cases,
the interaction between the integration and interpolation schemes could
be critical --- both in terms of computation time and memory requirements,
aside from the numerical precision. Independently of the scales at which
well-resolved LCS information is sought in this kind of system, the
aforementioned effects warrant further investigation.


\section{On the incompressibility of the velocity field}
\label{sec:on_the_incompressibility_of_the_velocity_field}

As explained in~\cref{sec:the_double_gyre_model}, the double gyre velocity field
is incompressible per construction. Thus, per equation
\eqref{eq:cauchygreenincomprlambda}, the product of the strain eigenvalues
should equal one. Regardless of the integration method used, however, this
property was lost after approximately five units of time. This is not
particularly surprising, because this property only really holds for
infinitesimal fluid elements. Seeing as there is no assurance that neighboring
tracers will remain nearby under the flow map, the finite difference
approximation to the local stretch and strain could practically be rendered
invalid. This issue is expected to be most prominent in regions of high local
repulsion, which are precisely where accuracy is most imperative.

Thus, one should not really expect this property to hold numerically, especially
for an indefinite time. However, sample tests indicate that the
incompressibility property was preserved for longer when using a denser
grid of tracers. In fact, the strain eigenvalue product was found to be unity
beyond 20 units of time, that is, the integration time used in this project,
for an initial grid spacing of $10^{-12}$. However, due to the inherent
inaccuracy of double-precision floating-point numbers, this would leave very
few significant digits with which one could perform finite differencing, as
mentioned in
\cref{sub:on_the_choice_of_numerical_step_lengths_and_tolerance_levels}.
In the end, the grid spacing $\Delta{x}\simeq\Delta{y}\simeq0.02$ was chosen,
because it was the resolution at which repelling LCS curves for the double
gyre system have previously been described in the literature, cf.
\textcite{farazmand2012computing}.

For further analysis, grid spacings of approximate order $10^{-7}$--$10^{-8}$
could be considered, when working with double-precision floating-point numbers.
This would probably result in the incompressibility property being preserved
for longer, while also leaving up to 7 or 8 significant digits for finite
differencing. An entirely different approach, as suggested by
\textcite{onu2015lcstool}, is to simply \emph{define} the smaller strain
eigenvalue as  the reciprocal of the larger one, that is,
$\lambda_{1}\equiv\lambda_{2}^{-1}$. This was not done here, because the
flow incompressibility has no obvious practical consequences for LCSs.
In addition, subject to the accuracy of the numerical integration scheme,
no tracer which starts out within the computational domain
$[0\hspace{1ex}2]\times[0\hspace{1ex}1]$ ever leave it, due to the normal
component of the velocity field being zero along the domain edges. This
follows from the definition of the associated velocity field, given in
equation~\eqref{eq:doublegyre}.


\section{Concerning the numerical representation of tracers}
\label{sec:concerning_the_numerical_representation_of_tracers}

One way of increasing the numerical accuracy in the flow map, would be
to use higher precision floating-point numbers to represent the tracer
coordinates. The main drawback of making such a change, and using e.g.
quadruple-precision floating-point numbers, is that these are, to this date,
not fully implemented in conventional hardware --- even that of the NTNU's
supercomputer, Vilje. Accordingly, quadruple-precision floating-point
arithmetic is several orders of magnitude slower than that of double-precision
floating-point numbers. Coupled with the increase in required memory
(quadruple-precision numbers are stored using twice as many bits as
double-precision numbers), one would generally be forced to perform several
advections of smaller sets of tracers, calculating a piecewise representation
of the overall flow map. However, should this prove impossible, using higher
precision floating-point numbers would lead to a more granular representation
of the flow map. Such an approach is perhaps most sensible for systems with
velocity fields that are well-behaved, or even largely spatially invariant.
Then again, in such cases, calculating LCSs have little practical relevance,
seeing as the overall behaviour of the flow system could be estimated purely by
advecting a smaller set of tracers, or even by simple inspection.

Rather than generating and advecting a fixed, large amount of tracer particles,
another possible approach would be to use a set of fewer `base' tracers, and
making use of an \emph{adaptive multigrid method}. A practical implementation
could involve the dynamic introduction of increasingly finer grids in regions
where the local velocities would have the largest Euclidean norm, for instance.
Such grids, however, would have to move \emph{with} the flow, as the
benefit over simply increasing the initial tracer density would diminish
otherwise. The main reason why this was not implemented for this project,
aside from it not being used in the literature, is that it in all likelyhood
leads to inconsistencies when applying finite differences. That is, unless some
sort of interpolation scheme was applied to the flow map. Seeing as this
project is centered around integration methods, this idea was scrapped.
Regardless, this technique seems promising --- at least on paper --- which
warrants further investigation.
%
%\vspace{\fill}


\section{About the computation of strainlines}
\label{sec:about_the_computation_of_strainlines}

\subsection{The special linear interpolation of strain eigendirections}
\label{sub:the_special_linear_interpolation_of_strain_eigendirections}

\subsection{The use of a single numerical integrator}
\label{sub:the_use_of_a_single_numerical_integrator}






\begin{framed}
    \begin{itemize}
        \item \sout{Approach does produce a consistent LCS picture for all numerical integrators considered, provided sensible
            integration steps or tolerance levels are chosen. This indicates that the calculation of this type
        of transport barrier is not particularly sensitive to the integration method.}
    \item \sout{Most crucial part: Compute advection correctly. The error of the computed strain eigenvalues scales
                like the advection error, while the error in the strain eigendirections is negligible for all time
                steps. The computed LCSs seem to exhibit sensitive dependence on the calculation of the eigenvalues,
            which is not particularly surprising, considering the LCS conditions of eq. 3.13 }
                \begin{itemize}
                    \item \sout{The precision of the strain eigendirections is likely a direct consequence of the
                        auxiliary tracers in their computation.}
                \end{itemize}
            \item \sout{The number of points in the $\mathcal{U}$ domain is different to the one obtained by
                Haller et  al, by approximately $3\%$. This, however, is likely related to how the
            grid of tracers was set up. In their paper, Farazmand and Haller do not provide details.}
        \item \sout{Notably, the reference LCS as shown in figure (.) consists of seven different strainline
                segments, unlike the structure found by Haller et al., which is claimed to be a single
            coherent strainline.}

        \item \sout{The time lenghts or tolerance levels should be chosen based on the system under consideration.
            In particular, an individual time step should not be so large that too much detail in the local,
        instantaneous velocity field is glossed over.}
    \item \sout{Possible approach in order to refine the computations: Use fewer `base' tracers, and an \emph{adaptive multigrid method}. Main con: May lead to inconsistent centered differencing when approaching
        the Jacobian etc. of the flow map}
        \item \sout{Regarding the completely different domains obtained via the embedded methods for the highest tolerance level:
            Check if this corresponds th the error in the flow map larger than some threshold. If (likely) so, this is
        further evidence that one should exhert great effort in computing the flow map correctly.}
    \item \sout{In terms of application to discrete velocity data sets, where both spatial and temporal interpolation
                    may be necessary, the interaction between interpolation and integration scheme is likely to
                    have a great overall impact. In particular, the underlying interpolation scheme sets a lower accuracy
                    bound for the entire advection process, practically enforcing restrictions on the numerical step
                    lengths or tolerance levels which can be considered sensible. This is also based on the considered
                velocity field.}
            \item \sout{Regarding the incompressibility of the velocity field}
                    \begin{itemize}
                        \item \sout{Incompressibility property is conserved until approximately $t=5$ units. Generally can't expect
                            this property to hold numerically over time.}
                        \item \sout{There is no assurance that neighboring tracers will remain nearby after the advection,
                                leaving the finite difference approximation of the local strain and stretch invalid. This
                                issue is expected to be most prominent in regions of high repulsion, which are precisely
                            where accuracy is imperative.}
                    \end{itemize}
        \item Parameter choices
            \begin{itemize}
                \item Numerical step lengths and tolerance levels
                \item The use of RK4 for all strainline iterations
                    \begin{itemize}
                        \item Alternative: Use a high order adaptive step method, paying close attention to the strainline curvature
                    \end{itemize}
                \item \sout{Represent tracer positions by means of higher precision floating-point numbers}
                        \begin{itemize}
                            \item \sout{Yields more accurate finite difference approximations}
                            \item \sout{Requires more memory, i.e., less tracers can be advected, leading to
                                        a more granular flow map representation. Perhaps most relevant for
                                        well-behaved velocity fields, for which the LCS approach is not really
                                        practically relevant --- why not simply advect a few particles and see where they
                                    end up?}
                        \end{itemize}
            \end{itemize}
        \item The identification process of local strain maximizing strainlines
            \begin{itemize}
                \item The cutting of strainline tails
                \item Lines in $\mathcal{L}$ $\rightarrow$ not necessarily a robust approach for general flows
                \item Alternative: Clustering algorithm
                \item Approach where both the strainline lengths and avg $\lambda_{2}$ are taken into consideration,
                    i.e., selecting a longer line with slighly smaller avg $\lambda_{2}$ over a shorter line with larger.
            \end{itemize}
        \item The special linear interpolation used for the eigenvectors
            \begin{itemize}
                \item Possible alternative: Higher order interpolation
            \end{itemize}
            \item Chose mean error, rather than max, because:
                \begin{itemize}
                    \item In terms of flow map: Error should be limited from above, by the domain extent, as the normal component
                        of the velocity at the boundaries is zero
                        \item Generally no way of telling where the maximum error occurs
                \end{itemize}
                \item Regarding the use of $\rmsd$:
                    \begin{itemize}
                        \item A practical application of empirical standard deviation
                            \item The distribution of errors across the domain is unknown. From the Central Limit thm., however,
                                the mean error is nearly distributed as a Gaussian.
                    \end{itemize}
    \end{itemize}
\end{framed}




\clearpage

\chapter{Conclusions}
\label{cha:conclusions}
For the double gyre system considered here, the calculation of its LCSs
does not exhibit particularly sensitive dependence to
the choice of numerical integration method. However, this could be a
consequence of the LCSs present within the system being robust under the chosen
parameter values. This view is supported by the fact that the errors in the
computed LCS curves quickly flattened for sufficiently small integration time
steps or tolerance levels, while a similar effect did not manifest itself for
the computed flow maps or strain eigenvalues. At the same time, the fact that
the same strainline segments were identified as LCSs even for comparably large
errors in the strain eigenvalues, suggests that the numerical implementation
of the variational principles in order to find LCSs is, in itself, robust.

There is, however, room for research with regards to the numerical
implementation of one of the LCS existence conditions, derived from their
variational theory. The approach used for this project is not particularly
well-documented in the literature, in that, there exists independent variables
for which there does not yet exist an objective way to determine. Here, these
were chosen based upon careful inspection of the system under consideration,
and in order for the LCS curves to conform with the ones obtained in the
literature. A suggested alternative approach, which was not investigated as
part of this project, would be to utilize a sort of numerical clustering
algorithm rather than resorting to similar (to some extent) subject
considerations which were employed here.

Although no numerical integration scheme stood out as the absolute best,
using higher order methods invariably resulted in more efficient calculations,
which are less prone to numerical round-off errors. Accordingly, the use of
higher order methods is generaly advisable. Equally important as the choice of
integration scheme, however, is the choice of numerical step length or tolerance
level. Ideally, this should be selected based upon physical considerations of
the system at hand. For instance, one could identify a characteristic velocity,
then tune the step length or tolerance level such that no tracers moving with
said velocity never entirely skip a grid cell when moving from one time level
to the next. This, in order to ensure that the local instantaneous dynamics
are resolved properly. To what scale the microscopic behaviour should be
resolved depends on what scale to which detailed information regarding the flow
is sought, and, of course, the sampling frequency (spatial and temporal alike)
of discrete data samples, for most real-life applications.

On that note, when investigating transport systems for which the available
data sets are discrete, the choice of numerical \emph{interpolation} scheme
will generally also impact the calculations of LCSs. This depencence has not
been investigated as part of this project. However, similar reservations
as for the integration time step or tolerance levels are also applicable to
the spatial and temporal sampling frequencies involved in the interpolation.
Furthermore, the choice of integration scheme should be made in relation
to the interpolation method. For instance, it makes little sense to use a
\nth{5} order accurate integration scheme in tandem with a \nth{3} order
accurate interpolator. Put simply, the interaction between interpolation
and integration schemes in the analysis of discrete data sets is of great
interest for practical applications of the LCS theory, and warrants
further investigation.

\clearpage


\begin{figure}[htpb]
    \centering
    \def\svgwidth{0.8\linewidth}
    \input{figures/falsepositives.pdf_tex}
    \caption[Illustration of how the offset of false LCS segments was computed]%
    {Illustration of how the offset of false LCS segments was computed.
    An LCS segment, denoted by $\widetilde{\gamma}_{0}$ in the figure, is
    compared with the reference LCS, labelled $\gamma_{0}$. Each part of the
    curve LCS curve, which is farther away from all points on the reference LCS
    than a pre-set length $l_{\textnormal{noise}}=0.01$ is flagged as a
    false positive. The area $\mathcal{A}$ between the reference LCS and the
    curve segments identified as false positives, is estimated by means of the
    midpoint rule. In the case of false negatives, $\mathcal{A}$ denotes the
    minimal area between segment of the reference LCS and the approximated LCS,
    for each part of the reference LCS which is not present in the
approximation.}
    \label{fig:fp_fn_principle}
\end{figure}



