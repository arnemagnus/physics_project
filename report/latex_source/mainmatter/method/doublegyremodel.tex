In order to investigate the dependence of LCS identification by means of
the variational approach as presented in \cref{sub:hyperbolic_lcss} on
the choice of numerical integration method, outlined in
\cref{sec:solvingsystems}, a system which has been studied extensively in the
literature, was chosen. The system, an unsteady double gyre, has been used
frequently as a test case for locating LCSs from different indicators
\parencite{farazmand2012computing,shadden2005definition}. As a result, the
LCSs the system exhibit are well known.

\section{The double gyre model}
\label{sec:the_double_gyre_model}

The double gyre model is defined as a pair of counter-rotating gyres, with a
time-periodic perturbation. The perturbation can be interpreted as a solid, as
in impenetrable, wall which oscillates periodically, causing the gyres
to contract and expand peridically. In terms of the cartesian coordinate vector
$\vct{x}=(x,y)$, the system can be expressed mathematically as

\begin{equation}
    \label{eq:doublegyre}
    \renewcommand{\arraystretch}{2.5}
    \dot{\vct{x}} =\vct{v}(t,\vct{x})= \pi{}A\begin{pmatrix}%
        -\sin\big(\pi{}f(t,x)\big)\cos(\pi{}y)\\
        \cos\big(\pi{}f(t,x)\big)\sin(\pi{}y)\dfrac{\partial{}f(t,x)}{\partial{}x}
    \end{pmatrix}
\end{equation}

where

\begin{equation}
    \label{eq:doublegyrefuns}
    \begin{gathered}
        f(t,x) = a(t)x^{2} + b(t)x\\
        a(t) = \epsilon\sin(\omega{}t)\\
        b(t) = 1-2\epsilon\sin(\omega{}t)
    \end{gathered}
\end{equation}

and the parameters $A$, $\epsilon$ and $\omega$ dictate the nature of the
flow pattern. As in the literature, the parameter values

\begin{equation}
    \label{eq:doublegyreparams}
    \begin{gathered}
        A = 0.1\\
        \epsilon=0.1\\
        \omega=\frac{2\pi}{10}
    \end{gathered}
\end{equation}

were used \parencite{farazmand2012computing,shadden2005definition}. Moreover,
the starting time was $t_{0}=0$, and the integration time was $T=20$, i.e.,
forcing two periods of motion, per
\eqref{eq:doublegyreparams}.

Note that the velocity field $\vct{v}(t,\vct{x})$ in equation
\eqref{eq:doublegyre} can be expressed in terms of a scalar stream function:

\begin{equation}
    \label{eq:doublegyrestreamfun}
    \renewcommand{\arraystretch}{2.5}
    \begin{gathered}
        \psi(t,\vct{x}) = A\sin\big(\pi{}f(t,x)\big)\sin(\pi{}y) \\
        \vct{v}(t,\vct{x}) = \begin{pmatrix}%
            -\dfrac{\partial{}\psi}{\partial{}y} \\
            \dfrac{\partial{}\psi}{\partial{}x}
        \end{pmatrix}
    \end{gathered}
\end{equation}

which means that the velocity field is divergence-free by construction:

\begin{equation}
    \label{eq:doublegyreincompr}
    \vct{\nabla}\vdot\vct{v}(t,\vct{x}) = -\pdv[2]{\psi}{x}{y} %
                                        + \pdv[2]{\psi}{y}{x} = 0,
\end{equation}

where the latter equality follows from Schwartz' theorem of mixed partial
derivatives, as the stream function is smooth. This means that we expect the
property given in equation~\eqref{eq:cauchygreenincomprlambda} to hold for the
double gyre flow.
