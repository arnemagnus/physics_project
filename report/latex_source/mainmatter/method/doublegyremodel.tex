In order to investigate how LCS identification by means of
the variational approach presented in \cref{sub:hyperbolic_lcss} depends
on the choice of numerical integration method, outlined in
\cref{sec:solvingsystems}, a system which has been studied extensively in the
literature, was chosen. The system, an unsteady double gyre, has been used
frequently as a test case for locating LCSs from different indicators
\parencite{farazmand2012computing,shadden2005definition}. As a result, the
LCSs the system exhibit are well documented.

\section{The double gyre model}
\label{sec:the_double_gyre_model}

The double gyre model consists of a pair of counter-rotating vortices, with a
time-periodic perturbation. The perturbation can be interpreted as a wall which
oscillates periodically, causing the vortices to contract and expand. In terms
of the Cartesian coordinate vector $\vct{x}=(x,y)$, the system can be expressed
mathematically as
\begin{equation}
    \label{eq:doublegyre}
    \renewcommand{\arraystretch}{2.5}
    \dot{\vct{x}} =\vct{v}(t,\vct{x})= \pi{}A\begin{pmatrix}%
        -\sin\big(\pi{}f(t,x)\big)\cos(\pi{}y)\\
        \cos\big(\pi{}f(t,x)\big)\sin(\pi{}y)\dfrac{\partial{}f(t,x)}{\partial{}x}
    \end{pmatrix}
\end{equation}
where
\begin{equation}
    \label{eq:doublegyrefuns}
    \begin{gathered}
        f(t,x) = a(t)x^{2} + b(t)x\\
        a(t) = \epsilon\sin(\omega{}t)\\
        b(t) = 1-2\epsilon\sin(\omega{}t)
    \end{gathered}
\end{equation}
and the parameters $A$, $\epsilon$ and $\omega$ dictate the nature of the
flow pattern. The domain of interest,
$\mathcal{U}=[0,\hspace{0.5ex}2]\times[0,\hspace{0.5ex}1]$, is characterized by
the orthogonal component of the velocity field being zero at its boundaries.
Thus, any trajectory which starts out within $\mathcal{U}$, remains there
for all time.
The parameter values
\begin{equation}
    \label{eq:doublegyreparams}
    \begin{gathered}
        A = 0.1\\
        \epsilon=0.1\\
        \omega=\frac{2\pi}{10}
    \end{gathered}
\end{equation}
were used, as has been common in literature, in particular in
the articles by \textcite{farazmand2012computing} and
\textcite{shadden2005definition}.
%There are other articles in which different
%parameter values are used, such as the one by \textcite{onu2015lcstool}.
The starting time $t_{0}=0$ and the integration time $T=20$ forces
the velocity field to undergo two periods of motion, as
$\omega=2\pi/10$, per~\cref{eq:doublegyreparams}.
%forcing two periods of motion, per
%\eqref{eq:doublegyreparams}.

\clearpage
Note that the velocity field $\vct{v}(t,\vct{x})$ in~\cref{eq:doublegyre} can
be expressed in terms of a scalar stream function:
\begin{equation}
    \label{eq:doublegyrestreamfun}
    \renewcommand{\arraystretch}{2.5}
    \begin{gathered}
        \psi(t,\vct{x}) = A\sin\big(\pi{}f(t,x)\big)\sin(\pi{}y),\\
        \vct{v}(t,\vct{x}) = \begin{pmatrix}%
            -\dfrac{\partial{}\psi}{\partial{}y} \\
            \dfrac{\partial{}\psi}{\partial{}x}
        \end{pmatrix}.
    \end{gathered}
\end{equation}
This means that the velocity field is divergence-free by construction:
\begin{equation}
    \label{eq:doublegyreincompr}
    \vct{\nabla}\vdot\vct{v}(t,\vct{x}) = -\pdv[2]{\psi}{x}{y} %
                                        + \pdv[2]{\psi}{y}{x} = 0,
\end{equation}
where the latter equality follows from the smoothness of the stream function.
In particular, since it is `infinitely' continuously differentiable, partial
differentiation of \emph{any} order of the stream function is commutative
\parencite[p.689]{adams2010calculus}. Moreover, this means that we expect the
property given in~\cref{eq:cauchygreenincomprlambda}, that is,
the product of the eigenvalues of the Cauchy-Green strain tensor to be unity,
to hold for the double gyre flow.
