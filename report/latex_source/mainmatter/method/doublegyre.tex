In order to investigate the dependence of LCS identification by means of
the variational approach as presented in \cref{sub:hyperbolic_lcss} on
the choice of numerical integration method, cf.\ \cref{sec:solvingsystems},
a system which has been studied extensively in the literature, was chosen.
The system, an unsteady double gyre, has been used frequently as a test case
for locating LCSs from different indicators
\parencite{farazmand2012computing,shadden2005definition}. As a result, the
LCSs the system exhibit are well known.

\section{The double gyre model}
\label{sec:the_double_gyre_model}

The double gyre model is defined as a pair of counter-rotating gyres, with a
time-periodic perturbation. The perturbation can be interpreted as a solid, as
in impenetrable, wall which oscillates periodically, which causes the gyres
to periodically contract and expand. In terms of the cartesian coordinate vector
$\vct{x}=(x,y)$, the system can be expressed mathematically as

\begin{equation}
    \label{eq:doublegyre}
    \renewcommand{\arraystretch}{2.5}
    \dot{\vct{x}} =\vct{v}(t,\vct{x})= \pi{}A\begin{pmatrix}%
        -\sin\big(\pi{}f(t,x)\big)\cos(\pi{}y)\\
        \cos\big(\pi{}f(t,x)\big)\sin(\pi{}y)\dfrac{\partial{}f(t,x)}{\partial{}x}
    \end{pmatrix}
\end{equation}

where

\begin{equation}
    \label{eq:doublegyrefuns}
    \begin{gathered}
        f(t,x) = a(t)x^{2} + b(t)x\\
        a(t) = \epsilon\sin(\omega{}t)\\
        b(t) = 1-2\epsilon\sin(\omega{}t)
    \end{gathered}
\end{equation}

and the parameters $A$, $\epsilon$ and $\omega$ dictate the nature of the
flow pattern. As in the literature, the parameter values

\begin{equation}
    \label{eq:doublegyreparams}
    \begin{gathered}
        A = 0.1\\
        \epsilon=0.1\\
        \omega=\frac{2\pi}{10}
    \end{gathered}
\end{equation}

were used \parencite{farazmand2012computing,shadden2005definition}. Moreover,
the starting time was $t_{0}=0$, and the integration time was $T=20$, i.e.,
forcing two periods of motion, cf.
\eqref{eq:doublegyreparams}.

Note that the velocity field $\vct{v}(t,\vct{x})$ in equation
\eqref{eq:doublegyre} can be expressed in terms of a scalar stream function:

\begin{equation}
    \label{eq:doublegyrestreamfun}
    \renewcommand{\arraystretch}{2.5}
    \begin{gathered}
        \psi(t,\vct{x}) = A\sin\big(\pi{}f(t,x)\big)\sin(\pi{}y) \\
        \vct{v}(t,\vct{x}) = \begin{pmatrix}%
            -\dfrac{\partial{}\psi}{\partial{}y} \\
            \dfrac{\partial{}\psi}{\partial{}x}
        \end{pmatrix}
    \end{gathered}
\end{equation}

which means that the velocity field is divergence-free by construction:

\begin{equation}
    \label{eq:doublegyreincompr}
    \vct{\nabla}\vdot\vct{v}(t,\vct{x}) = -\pdv[2]{\psi}{x}{y} %
                                        + \pdv[2]{\psi}{y}{x} = 0
\end{equation}

where the latter equality follows from Schwartz' theorem of mixed partial
derivatives, as the stream function is smooth. This means that we expect the
property given in equation~\eqref{eq:cauchygreenincomprlambda} to hold for the
double gyre flow.



\section{Advecting a set of initial conditions}
\label{sec:advecting_a_set_of_initial_conditions}

The variational model is based upon the advection of non-interacting tracers,
cf. \cref{sec:typeofflow}, by the velocity field defined in equation
\eqref{eq:doublegyre}. The system has no known analytical solution for the
tracer trajectories. Thus, it must be solved numerically, by means of some
numerical integration method, e.g.\ a Runge-Kutta method, cf.\
\cref{sub:the_runge_kutta_family_of_numerical_methods}. With the main focus
of this project being the dependence on LCSs on the chosen integration method,
the advection was performed using all of the numerical integrators introduced
in \cref{sub:the_runge_kutta_methods_under_consideration}.

\subsection{Generating a set of initial conditions}
\label{sub:generating_a_set_of_initial_conditions}
The computational domain $\mathcal{U}=[0\hspace{1ex}2]\times[0\hspace{1ex}1]$
was discretized by a set of linearly spaced tracers, with $1000\times500$ grid
points, effectively creating a nearly uniform grid of approximate spacing
$\Delta{x}\simeq\Delta{y}\simeq0.002$. Tracers were placed on, and within, the
domain boundaries of $\mathcal{U}$. The grid was extended artificially,
with an additional two rows or columns appended to all of the domain edges,
with the same grid spacing as the \emph{main} grid. This was done in order to
ensure that the dynamics at the domain boundaries were included in the analysis
to follow. The extended grid thus had a total of $1004\times504$ grid points.
The construction of the grid is illustrated in figure~\ref{fig:initialgrid}.

\vfill{}

\begin{figure}[htpb]
    \centering
    \def\svgwidth{0.8\linewidth}{\input{figures/initial_grid.pdf_tex}}
    \caption[Illustration of the set of initial conditions]
        {Illustration of the set of initial conditions.
                Dark grey blobs signify the main tracers, i.e., the tracers
                which discretize the computational domain
            $[0\hspace{1ex}2]\times[0\hspace{1ex}1]$. These were linearly
        spaced in either direction, with twice as many points in the $x$-
        direction as the $y$-direction, in order to generate an approximately
        equidistant grid. Light grey blobs
        signify the artificially extended grid, i.e., tracers starting
        originating outside of the computational domain. These were used in
        order to properly encapsulate the dynamics at the domain boundaries,
        in the analysis to follow.}
    \label{fig:initialgrid}
\end{figure}


In order to increase the resolution of the Cauchy-Green strain tensor,
it is necessary to increase the accuracy with which one computes the
Jacobian of the flow map, cf.\ equation \eqref{eq:cauchygreen}. This was done
by advecting a set of auxiliary tracer points surrounding each main point. To
each tracer point $\vct{x}_{j}=(x_{j},y_{j})$, neighboring points defined as

\begin{equation}
    \label{eq:auxgrid}
    \begin{gathered}
        \vct{x}_{j}^{r} = (x_{j}+\delta{x},y_{j}),%
                \quad\vct{x}_{j}^{l} = (x_{j}-\delta{x},y_{j})\\
                \vct{x}_{j}^{u} = (x_{j},y_{j}+\delta{y}),%
                \quad\vct{x}_{j}^{l} = (x_{j},y_{j}-\delta{y})\\
\end{gathered}
\end{equation}

where $\delta{x}$ and $\delta{y}$ are increments smaller than the grid spacings
$\Delta{x}\simeq\Delta{y}$, were assigned. Even though this effectively means
that five times as many particles have to be advected, the resulting accuracy
in computing the Jacobian by means of the auxiliary tracers is determined
by the independent variables $\delta{x}$ and $\delta{y}$, theoretically allowing
for arbitrary precision. The concept of the auxiliary tracers is illustrated in
figure~\ref{fig:auxiliarygrid}.

\begin{figure}[htpb]
    \centering
    \def\svgwidth{0.8\linewidth}{\input{figures/aux_grid.pdf_tex}}
    \caption[Illustration of the concept of auxiliary tracers]
    {Illustration of the concept of auxilary tracers, used in order to
    compute the Jacobian of the flow map, and by extension, the Cauchy-Green
    strain tensor field, cf.\ equation~\eqref{eq:cauchygreen}, more accurately.
    Grey blobs represent the original tracers, whereas white blobs represent
    the auxiliary ones.}
    \label{fig:auxiliarygrid}
\end{figure}


Because of the limited number of decimal digits which can be represented by
floating point numbers, however, there is a strict lower limit to which it makes
sense to lower vaues of the degrees of freedom $\delta{x}$ and $\delta{y}$. In
particular, the smallest number which can be resolved by double-precision
floating point numbers is of the order $10^{-16}$. When decreasing the auxiliary
grid spacing, the increase in precision is quickly offset by the fact that
one automatically gets allocated a smaller number of decimal digits with
which one can calculate the discrete approximation of the derivatives involved
in the Jacobian. This is due to the double gyre velocity field, cf.\
equation \eqref{eq:doublegyre}, being well-behaved, leading most tracers which
are close together initially to follow very similar trajectories, often
ending up with a separation distance comparable to the initial offset. For this
reason, the auxiliary grid spacing $\delta{x}=\delta{y}=10^{-5}$ was chosen
--- three orders of magnitude smaller than the original grid spacing, ensuring
that the derivatives in the Jacobian are far more well-resolved than for the
main tracers, while also leaving approximately 10 decimal digits for which
there can be a difference in the final positions of the auxiliary tracers.

\subsection{On the choice of numerical step lengths and tolerance levels}
\label{sub:on_the_choice_of_numerical_step_lengths_and_tolerance_levels}

For the fixed stepsize integrators, step lengths of $10^{-1}$ through to
$10^{-5}$ were used. For a step length of $10^{-5}$, the total number of
integration steps required in order to advect the system from $t_{0}=0$ to
$t=20$ is of order $10^{6}$. Because the inherent accuracy of double precision
floating-point numbers is of order $10^{-16}$, as mentioned previously, the
total floating point error expected to arise when performing the integration
for a step length of $10^{-5}$ is of order $10^{-10}$. The least accurate of
the fixed stepsize integrators under consideration, the Euler method, is a
\nth{1} order accurate globally, meaning that its local error is of \nth{2}
order in the time step, cf.\
\cref{def:rungekuttaorder}. Thus, we expect that the local error of the Euler
method to be of order $10^{-10}$, i.e., the same order of the accumulated
floating-point errors. Reducing the time step further necessarily leads to
an increase in the accumulated floating-point errors, meaning that we cannot
reasonably expect more accurate results for the Euler method --- at the very
least, a time step of $10^{-5}$ appears to be a point after which there is
little to be gained in terms of numerical accuracy for the Euler method by
lowering the time step further. For the other fixed stepsize integrators, which
are of higher order, we expect this breaking point to occur for a (somewhat)
larger time step.

While the above logic does not translate directly for the adaptive stepsize
integrators, empirical tests indicate that for the Bogacki-Shampine integrators,
as well as the Dormand-Prince 5(4) integrator, the accumulated floating point
errors caught up to the required tolerance level at some point between the
levels $10^{-10}$ and $10^{-11}$, while the Dormand-Prince 8(7) integrator held
its ground until about $10^{-13}$. For this reason, tolerance levels of
$10^{-1}$ through to $10^{-10}$ were used for the adaptive stepsize integrators.

As previously mentioned, reference solutions must be obtained by means of a
high order fixed stepsize method with a small step length, alternatively a high
order adaptive stepsize method with a small tolerance level. In this case, the
latter approach was chosen, and the solution obtained via the Dormand-Prince
8(7) integrator with a numerical tolerance of $10^{-12}$ was used as the
reference.


