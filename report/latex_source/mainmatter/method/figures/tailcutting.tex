\begin{figure}[htpb]
    \centering
    \def\svgwidth{0.8\linewidth}
    \begin{figure}[htpb]
    \centering
    \def\svgwidth{0.8\linewidth}
    \begin{figure}[htpb]
    \centering
    \def\svgwidth{0.8\linewidth}
    \begin{figure}[htpb]
    \centering
    \def\svgwidth{0.8\linewidth}
    \input{figures/tailcutting.pdf_tex}
    \caption[Illustration of the concept of strainline tail end cutting]%
    {Illustration of the concept of strainline tail end cutting. The
    filled, gray ellipsis constitutes a fictitious set of points $\mathcal{U}_{0}$,
    for which the LCS existence conditions expressed in equations
    \eqref{eq:numericalexistence1} and~\eqref{eq:numericalexistence2} hold. A
    strainline, starting at (a), is integrated in both directions in
    pseudotime. In both directions, the strainline integration is eventually
    stopped due to repeated failure of at least one of the two aforementioned
    LCS existence conditions, over a pre-set length $l_{\textnormal{f}}=0.2$.
    The tails at either end, i.e., the parts of the strainline which exited
    the $\mathcal{U}_{0}$ and did not return, are indicated by dashed curves at
    (b) and (c). These were cut, leaving the solid curve as the part of
    the strainline which was considered as an LCS candidate. Although parts
    of \emph{this} curve may reside outside of the $\mathcal{U}_{0}$ domain, these segments are all
\emph{shorter} than $l_{\textnormal{f}}$.}
    \label{fig:tailcutting}
\end{figure}

    \caption[Illustration of the concept of strainline tail end cutting]%
    {Illustration of the concept of strainline tail end cutting. The
    filled, gray ellipsis constitutes a fictitious set of points $\mathcal{U}_{0}$,
    for which the LCS existence conditions expressed in equations
    \eqref{eq:numericalexistence1} and~\eqref{eq:numericalexistence2} hold. A
    strainline, starting at (a), is integrated in both directions in
    pseudotime. In both directions, the strainline integration is eventually
    stopped due to repeated failure of at least one of the two aforementioned
    LCS existence conditions, over a pre-set length $l_{\textnormal{f}}=0.2$.
    The tails at either end, i.e., the parts of the strainline which exited
    the $\mathcal{U}_{0}$ and did not return, are indicated by dashed curves at
    (b) and (c). These were cut, leaving the solid curve as the part of
    the strainline which was considered as an LCS candidate. Although parts
    of \emph{this} curve may reside outside of the $\mathcal{U}_{0}$ domain, these segments are all
\emph{shorter} than $l_{\textnormal{f}}$.}
    \label{fig:tailcutting}
\end{figure}

    \caption[Illustration of the concept of strainline tail end cutting]%
    {Illustration of the concept of strainline tail end cutting. The
    filled, gray ellipsis constitutes a fictitious set of points $\mathcal{U}_{0}$,
    for which the LCS existence conditions expressed in equations
    \eqref{eq:numericalexistence1} and~\eqref{eq:numericalexistence2} hold. A
    strainline, starting at (a), is integrated in both directions in
    pseudotime. In both directions, the strainline integration is eventually
    stopped due to repeated failure of at least one of the two aforementioned
    LCS existence conditions, over a pre-set length $l_{\textnormal{f}}=0.2$.
    The tails at either end, i.e., the parts of the strainline which exited
    the $\mathcal{U}_{0}$ and did not return, are indicated by dashed curves at
    (b) and (c). These were cut, leaving the solid curve as the part of
    the strainline which was considered as an LCS candidate. Although parts
    of \emph{this} curve may reside outside of the $\mathcal{U}_{0}$ domain, these segments are all
\emph{shorter} than $l_{\textnormal{f}}$.}
    \label{fig:tailcutting}
\end{figure}

    \caption[Illustration of the concept of strainline tail end cutting]%
    {Illustration of the concept of strainline tail end cutting. The
    filled, gray ellipsis constitutes a fictitious set of points $\mathcal{U}_{0}$,
    for which the LCS existence conditions expressed in equations
    \eqref{eq:numericalexistence1} and~\eqref{eq:numericalexistence2} hold. A
    strainline, starting at (a), is integrated in both directions in
    pseudotime. In both directions, the strainline integration is eventually
    stopped due to repeated failure of at least one of the two aforementioned
    LCS existence conditions, over a pre-set length $l_{\textnormal{f}}=0.2$.
    The tails at either end, i.e., the parts of the strainline which exited
    the $\mathcal{U}_{0}$ and did not return, are indicated by dashed curves at
    (b) and (c). These were cut, leaving the solid curve as the part of
    the strainline which was considered as an LCS candidate. Although parts
    of \emph{this} curve may reside outside of the $\mathcal{U}_{0}$ domain, these segments are all
\emph{shorter} than $l_{\textnormal{f}}$.}
    \label{fig:tailcutting}
\end{figure}
