\begin{figure}[htpb]
    \centering
    \def\svgwidth{0.8\linewidth}
    \input{figures/falsepositives.pdf_tex}
    \caption[Illustration of how the offset of false LCS segments was computed]%
    {Illustration of how the offset of false LCS segments was computed.
        An LCS segment, denoted by $\widetilde{\gamma}_{0}$ in the figure, is
        compared with the reference LCS, labelled $\gamma_{0}$. Each part
        the approximated LCS curve, which is farther away from all points on
        the reference LCS than a pre-set length $l_{\textnormal{noise}}=0.01$
        is flagged as a false positive. The area $\mathcal{A}$ between
        the reference LCS and the curve segments identified as false positives,
        is estimated by means of the midpoint rule; that is, for each point on
        a false positive segment of the approximated LCS, the smallest
        distance separating it and the reference LCS is weighted with $\Delta$,
        the step length used in the numerical integration of strainlines.
        Because the $\vct{\xi}_{1}$-field is normalized to unit length,
        $\Delta$ equals the distance between consecutive points in the
        parametrizations of both the approximated and the reference LCS curves,
        per equation~\eqref{eq:strainlineode}. In the case of false negatives,
        $\mathcal{A}$ denotes the minimal area between the reference LCS curve
        and the approximaated LCS curve, for each part of the reference LCS
        which is not present in the approximation; that is, each point on the
        reference LCS curve which is farther away from all points on the
        approximate curve than $l_{\textnormal{noise}}$.}
    \label{fig:fp_fn_principle}
\end{figure}

