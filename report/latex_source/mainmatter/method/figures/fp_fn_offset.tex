\begin{figure}[htpb]
    \centering
    \def\svgwidth{0.8\linewidth}
    \input{figures/falsepositives.pdf_tex}
    \caption[Illustration of how the offset of false LCS segments was computed]%
    {Illustration of how the offset of false LCS segments was computed.
        An LCS segment, denoted by $\widetilde{\gamma}_{0}$ in the figure, is
        compared with the reference LCS, labelled $\gamma_{0}$. Each part of
        $\widetilde{\gamma}_{0}$, which is farther away from all points on $\gamma_{0}$
        than a pre-set length $l_{\textnormal{noise}}=0.01$ is flagged as a
        false positive. The area $\mathcal{A}$ between $\gamma_{0}$ and the
        curve segments identified as false positives, is estimated by means of the
        midpoint rule. In the case of false negatives, $\mathcal{A}$ denotes the
        minimal area between the curves $\gamma_{0}$ and $\widetilde{\gamma}_{0}$,
        for each part of the reference LCS which is not present in the
    approximation.}
    \label{fig:fp_fn_principle}
\end{figure}

