\section{Advecting a set of initial conditions}
\label{sec:advecting_a_set_of_initial_conditions}

The variational model for computing LCSs is based upon the advection of
non-interacting tracers, as described in \cref{sec:typeofflow}, by the
velocity field defined by
\cref{eq:doublegyre,eq:doublegyrefuns,eq:doublegyreparams}. To our knowledge,
the system has no analytical solution for the tracer trajectories. Thus, it must
be solved numerically, by means of some numerical integration method, e.g.\ a
Runge-Kutta method, a family of numerical ODE methods which is outlined in
\cref{sub:the_runge_kutta_family_of_numerical_methods}. With the main focus of
this project being the dependence on LCSs detection upon the chosen integration
method, the advection was performed using all of the numerical integrators
introduced in \cref{sub:the_runge_kutta_methods_under_consideration}.

\subsection{Generating a set of initial conditions}
\label{sub:generating_a_set_of_initial_conditions}
The computational domain $\mathcal{U}=[0,\hspace{0.5ex}2]\times[0,\hspace{0.5ex}1]$
was discretized by a set of equidistant \emph{principal} tracers, with
$1000\times500$ grid points, effectively creating a nearly uniform grid of
approximate spacing $\Delta{x}\simeq\Delta{y}\simeq0.002$, as tracers were
placed on and within the domain boundaries of $\mathcal{U}$. The grid was
extended outside of $\mathcal{U}$, with an additional two rows or columns
appended to all of the domain edges, with the same grid spacing as the
\emph{principal} grid. This was done in order to ensure that the dynamics at the
domain boundaries were included in the analysis to follow. The velocity field
outside of $\mathcal{U}$ was also defined by
\cref{eq:doublegyre,eq:doublegyrefuns,eq:doublegyreparams}. The
\emph{extended} grid thus had a total of $1004\times504$ grid points.
The construction of the grid is illustrated in figure~\ref{fig:initialgrid}.
\clearpage
\begin{figure}[htpb]
    \centering
    \def\svgwidth{0.8\linewidth}{\input{figures/initial_grid.pdf_tex}}
    \caption[Illustration of the set of initial conditions]
        {Illustration of the set of initial conditions.
                Dark grey blobs signify the main tracers, i.e., the tracers
                which discretize the computational domain
            $[0\hspace{1ex}2]\times[0\hspace{1ex}1]$. These were linearly
        spaced in either direction, with twice as many points in the $x$-
        direction as the $y$-direction, in order to generate an approximately
        equidistant grid. Light grey blobs
        signify the artificially extended grid, i.e., tracers starting
        originating outside of the computational domain. These were used in
        order to properly encapsulate the dynamics at the domain boundaries,
        in the analysis to follow.}
    \label{fig:initialgrid}
\end{figure}


In order to increase the precision of the Cauchy-Green strain tensor,
it is necessary to increase the accuracy with which one computes the
Jacobian of the flow map, as their accuracies are intrinsically linked; which
follows from~\cref{eq:cauchygreen}.
This was done by advecting a set of four auxiliary tracers surrounding each
\emph{main} tracer, that is, tracers originating at the principal or extended
grid points. To each tracer point $\vct{x}_{i,j}=(x_{i},y_{j})$, neighboring
points defined as
\begin{equation}
    \label{eq:auxgrid}
    \begin{gathered}
        \vct{x}_{i,j}^{r} = (x_{i}+\delta{x},y_{j}),%
                \quad\vct{x}_{i,j}^{l} = (x_{i}-\delta{x},y_{j})\\
                \vct{x}_{i,j}^{u} = (x_{i},y_{j}+\delta{y}),%
                \quad\vct{x}_{i,j}^{d} = (x_{i},y_{j}-\delta{y})\\
\end{gathered}
\end{equation}
were assigned, where $\delta{x}$ and $\delta{y}$ are increments smaller than
the grid spacings $\Delta{x}\simeq\Delta{y}$. Even though this effectively means
that five times as many tracers have to be advected, the resulting accuracy in
computing the Jacobian of the flow map, by means of the auxiliary tracers, is
determined by the independent variables $\delta{x}$ and $\delta{y}$. This, in
principle, allows for much higher precision than what would be obtained by
simply advecting five times as many \emph{equidistant} tracers. The concept
of the auxiliary tracers is illustrated in figure~\ref{fig:auxiliarygrid}.

\begin{figure}[htpb]
    \centering
    \def\svgwidth{0.8\linewidth}{\input{figures/aux_grid.pdf_tex}}
    \caption[Illustration of the concept of auxiliary tracers]
    {Illustration of the concept of auxilary tracers, used in order to
    compute the Jacobian of the flow map, and by extension, the Cauchy-Green
    strain tensor field, cf.\ equation~\eqref{eq:cauchygreen}, more accurately.
    Grey blobs represent the original tracers, whereas white blobs represent
    the auxiliary ones.}
    \label{fig:auxiliarygrid}
\end{figure}


Because of the limited number of decimal digits which can be accuractely
represented by floating-point numbers, however, there is a strict lower limit
to which it makes sense to lower the values of $\delta{x}$ and $\delta{y}$. In
particular, because spatial extent of the computational domain is of order $1$,
so too will all of the tracer coordinates be. Thus, the so-called machine
epsilon of double-precision floating-point numbers --- which, somewhat
simplified, can be described as the smallest positive number $\varepsilon$ for
which $1+\varepsilon$ is not rounded down to $1$ --- is the dominant source
of floating-point errors with regards to the flow map, moreso than the inherent
precision, that is, the smallest positive number which can be
\emph{represented} as a double-precision floating-point number. Per the
IEEE standard for floating-point arithmetic, the double-precision machine
epsilon is of order $10^{-16}$, whereas the smallest positive number
which can be represented in double precision is of order $10^{-308}$
\parencite{ieee2008standard}.

So, because of the scale of the numerical domain, we cannot reasonably expect
to accurately represent the flow maps beyond 15 decimal digits. In order to
approximate the Jacobian of the flow map, finite differencing was applied,
which will be described in greater detail in
\cref{sec:calculating_the_cauchy_green_strain_tensor}. Note, however, that
taking the finite difference between the endpoint coordinates of neighboring
tracers is generally expected to yield small decimal numbers, because the
velocity field, given by
\cref{eq:doublegyre,eq:doublegyrefuns,eq:doublegyreparams}, is reasonably
well-behaved. Say, for instance, that the difference is of order $10^{-10}$.
Because, as previously mentioned, we can only really expect to accurately
resolve the flow map to the \nth{15} decimal digit, this leaves only 5
significant digits. With the parameter choices specified in
\cref{eq:doublegyreparams}, the velocity field leads most tracers which
are intially close to follow very similar trajectories, often ending up
with a separation distance comparable to the initial offset. For this reason,
the auxiliary grid spacing $\delta{x}=\delta{y}=10^{-5}$ was chosen --- two
orders of magnitude less than the original spacing --- ensuring that the
derivatives in the Jacobian are far more well-resolved than for the main
tracers alone, while also leaving up to 10 significant decimal digits for
which the difference in the final positions of the auxiliary tracers could be
resolved.
\clearpage
%In particular, the smallest number which can be resolved as a double-precision
%floating-point number is of the order $10^{-16}$. When decreasing the
%auxiliary grid spacing, the increase in precision is quickly offset by the fact
%that one automatically gets allocated a smaller number of decimal digits with
%which one calculate the discrete approximation of the derivatives involved
%in the Jacobian. This is due to the double gyre velocity field, being
%reasonably well-behaved, leading most tracers which are initially close to
%follow very similar trajectories, often ending up with a separation distance
%comparable to the initial offset. For this reason, the auxiliary grid spacing
%$\delta{x}=\delta{y}=10^{-5}$ was chosen --- three orders of magnitude smaller
%than the original grid spacing, ensuring that the derivatives in the Jacobian
%are far more well-resolved than for the main tracers, while also leaving
%approximately $10$ decimal digits for which there can be a difference in the
%final positions of the auxiliary tracers.
%
\subsection{On the choice of numerical step lengths and tolerance levels}
\label{sub:on_the_choice_of_numerical_step_lengths_and_tolerance_levels}

For the fixed stepsize integrators, step lengths of $10^{-1}$ through to
$10^{-5}$ were used. The reason even smaller step lengths were not considered
is that the accumulated numerical round-off errors, mentioned in
\cref{sub:generating_a_set_of_initial_conditions}, for the advection
of tracers by means of the highest order methods, that is, Kutta's method and
the classical Runge-Kutta method (cf.\ \cref{tab:butcherrk3,tab:butcherrk4}),
became large enough to counteract the inherent accuracies of the methods,
for time steps smaller than $10^{-4}$ and $10^{-3}$, respectively. Further
details will be presented in \cref{cha:results}. We expect a similar development
for both the Euler and Heun methods (cf.\
\cref{tab:butchereuler,tab:butcherrk2}), albeit for even smaller time steps,
due to these methods being of lower order accuracy.
%Furthermore, rather
%than using e.g.\ the Euler method with an even smaller time step, one would in
%practice simply use a higher-order method, in order to obtain similar or better
%accuracy even for larger time steps.
%
%the following: For a step length of $10^{-5}$, the total number of
%integration steps required in order to take the system from $t=0$ to $t=20$
%is of order $10^{6}$. As previously mentioned, the inherent accuracy of
%double-precision floating point numbers is of order $10^{-16}$. Thus, the
%total floating point error expected to arise when integrating with a step
%length of $10^{-5}$ is of order $10^{-10}$.
%
%The least accurate of the fixed stepsize integrators integrators under
%consideration, the Euler method, is presented in \cref{tab:butchereuler}. It is
%\nth{1} order accurate globally, meaning that its local error is of \nth{2}
%order, per \cref{def:rungekuttaorder}. Thus, we expect that the local error of
%the Euler method, for a step length of $10^{-5}$, is of order $10^{-10}$, that
%is, the same order as the accumulated floating-point errors. Reducing the time
%step further necessarily leads to an increase in the accumulated floating-point
%errors, meaning that we cannot reasonably expect to resolve the positions
%from one step to the next more accurately, for the Euler method. At the very
%least, a time step of $10^{-5}$ appears to represent a point after which
%there is little to be gained in terms of increased numerical accuracy for the
%Euler method. For the other fixed stepsize integrators, which are of higher
%order, we expect this breaking point to occur for a somewhat larger time step.
%

As previously mentioned, the concept of \emph{order} does not translate
directly from the singlestep integrators to the adaptive stepsize methods, by
virtue of the dynamical step length. The step length adjustment procedure
considered here, which will be outlined in detail in
\cref{sub:on_the_implementation_of_embedded_runge_kutta_methods}, involves
the use of numerical tolerances regarding the difference between the interpolant
and locally extrapolated solutions. Empirical tests indicate that for both of the
Bogacki-Shampine integrators, as well as for the Dormand-Prince 5(4) integrator,
the accumulated floating-point errors caught up to the required tolerance level
at some point between the levels $10^{-10}$ and $10^{-11}$, while the
Dormand-Prince 8(7) integrator held its ground until a tolerance level of about
$10^{-13}$. For this reason, tolerance levels of $10^{-1}$ through to $10^{-10}$
were used for the adaptive stepsize integrators. Furthermore, as to our
knowledge, no analytical solution exists for the double gyre system, a numerical
solution is needed as the reference. Following the discussion above, the
solution obtained via the Dormand-Prince 8(7) integrator with a numerical
tolerance level of $10^{-12}$ was used for this purpose.

With the addition of the aforementioned auxiliary tracers (see
\cref{eq:auxgrid,fig:auxiliarygrid}), the total number of tracers which were
advected became of order $2.5$ million. In order to accelerate the computational
process, the advection was parallellized by means of MPI and run on NTNU's
supercomputer, Vilje. The choice of MPI over alternative multiprocessing tools
was mainly motivated by MPI facilitating access to multiple nodes within the
Vilje cluster. Because the tracers are independent, the parallelization
consisted of distributing an approximately even amount of tracers across all
ranks, whereupon each rank advected its allocated tracers. In the end, all of
the tracer end positions was collected by the designated main process
(that is, $\textnormal{rank}=0$). The associated speedup was crucial for this
project --- for example, advection using the Euler method and a time step of
$10^{-5}$ required in excess of $1000$ CPU hours; an insurmountable feat for
most computers, including the author's own personal laptop.
\clearpage
\subsection{On the implementation of embedded Runge-Kutta methods}
\label{sub:on_the_implementation_of_embedded_runge_kutta_methods}

In order to implement automatic step size control, the procedure suggested
by \textcite[pp.167--168]{hairer1993solving} was followed
closely. A starting step size of $h=0.1$ was used throughout. For the first
solution step, the embedded integration method, as described in
\cref{sub:the_runge_kutta_family_of_numerical_methods}, yields the two
approximations $x_{1}$ and $\widehat{x}_{1}$, from which the difference
$x_{1}-\widehat{x}_{1}$ can be used as an estimate of the error of the less
precise result. The idea is to enforce the error of the numerical solution to
satisfy componentwise:
\begin{equation}
    \label{eq:embeddederror}
    \abs{x_{1,i}-\widehat{x}_{1,i}} \leq \scem_{i},\quad{}%
    \scem_{i}=\atol_{i}+\max\big(\abs{x_{0,i}},\abs{x_{1,i}}\big)\cdot{}\rtol_{i},
\end{equation}
where $\atol_{i}$ and $\rtol_{i}$ are the desired absolute and relative
numerical tolerances, prescribed by the user. For this project, $\atol_{i}$ was
always set equal to $\rtol_{i}$.

As a measure of the numerical error,
\begin{equation}
    \label{eq:embeddednumericalerror}
    \errem = \sqrt{\frac{1}{n}\sum\limits_{i=1}^{n}%
    {\bigg(\frac{x_{1,i}-\widehat{x}_{1,i}}{\scem_{i}}\bigg)}^{2}}
\end{equation}
is used. Then, $\errem$ is compared to $1$ in order to find an optimal step
size. From~\cref{def:rungekuttaorder}, it follows that $\errem\approx{}Kh^{q+1}$,
where $q=\min(p,\widehat{p}\,)$. With $1\approx{}Kh_{\mathrm{opt}}^{q+1}$,
one finds the optimal step size
\begin{equation}
    \label{eq:embeddedoptimalstepsize}
    h_{\mathrm{opt}}=h\cdot\Big(\frac{1}{\errem}\Big)^{\frac{1}{q+1}}.
\end{equation}

If $\errem\leq1$, the solution step is accepted, the time variable $t$ is
increased by $h$, and the step length is modified according to
\cref{eq:embeddedoptimalstepsize,eq:embeddedstepsizeadjustment}. Which of the
two approximations $x_{n+1}$ or $\widehat{x}_{n+1}$ are used to continue the
integration varies depending on the embedded method in question. Continuing the
integration with the higher order result is commonly referred to as
\emph{local extrapolation}. If $\errem>1$, the solution step is rejected, the
time variable $t$ remains unaltered, and the step length is decreased before
another attempted step. The procedure for updating the step length can be
summarized as follows:
\begin{equation}
    \label{eq:embeddedstepsizeadjustment}
h_{\mathrm{new}}=\begin{cases}%
    \min(\facem_{\mathrm{max}}\cdot{}h,\facem\cdot{}h_{\mathrm{opt}}),&
\textnormal{if the solution step is accepted}\\
\facem\cdot{}h_{\mathrm{opt}},&\textnormal{if the solution step is rejected}%
\end{cases}
\end{equation}
where $\facem$ and $\facem_{\mathrm{max}}$ are numerical safety factors,
intended to prevent increasing the step size \emph{too} much. For this project,
$\facem=0.8$ and $\facem_{\mathrm{max}}=2.0$ were used throughout.

All of the embedded methods considered here are tuned in order to minimize
the error of the higher order result; accordingly, local extrapolation was
used throughout. Regarding the Bogacki-Shampine 5(4) method, which yields
\emph{two} independent \nth{4}-order interpolant solutions, the procedure was
slightly altered. First, the error of the \nth{5}-order solution with regards to
the interpolant corresponding to the first row of $\widehat{b}$-coefficients in
\cref{tab:butcherbs54} was computed. If this error was larger than unity,
the attempted step was rejected, and the same error was used in order to
update the time step, per
\cref{eq:embeddedoptimalstepsize,eq:embeddedstepsizeadjustment}. In the
opposite case, a new error estimate of the \nth{5}-order solution was computed
relative to the interpolant corresponding to the second row of
$\widehat{b}$-coefficients in \cref{tab:butcherbs54}. If this was larger than
unity, the attempted step was rejected, whereas the step was accepted if the
error was smaller than or equal to unity. In either case, this \emph{second}
error estimate was then used in order to update the time step.
