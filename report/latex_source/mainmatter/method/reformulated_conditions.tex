\section{Identifying LCS candidates numerically}
\label{sec:identifying_lcs_candidates_numerically}

\textcite{farazmand2012computing} found the identification of the
zeros of the inner product in equation~\eqref{eq:lcsexistence4} of
\cref{thm:lcsexistence} to be numerically sensitive. For this reason, they
suggest a reformulated set of conditions which make for more robust numerical
implementation, as follows:

\begin{subequations}
    \label{eq:numericalexistence}
    \begin{align}
        \label{eq:numericalexistence1}
        &\lambda_{1}(\vct{x}_{0})\neq\lambda_{2}(\vct{x}_{0})>1\\
        \label{eq:numericalexistence2}
        &\big\langle\vct{\xi}_{2}(\vct{x}_{0}),%
            \mtrx{H}_{\lambda_{2}}(\vct{x}_{0})%
            \vct{\xi}_{2}(\vct{x}_{0})\big\rangle \leq 0\\
        \label{eq:numericalexistence3}
        &\vct{\xi}_{1}(\vct{x}_{0})\parallel\mathcal{M}(t_{0})\\
        \label{eq:numericalexistence4}
        &\begin{gathered}
            \overline{\lambda}_{2},\ \textnormal{the average of}%
            \ \lambda_{2}\ \textnormal{over a curve}\ \gamma,\ %
            \textnormal{is maximal on}\ \mathcal{M}(t_{0})\\ %
            \hspace{-1.3em}\textnormal{among all}\ \textnormal{nearby curves}\ %
            \gamma\ \textnormal{satisfying}\ \gamma\parallel\vct{\xi}_{1}(\vct{x}_{0})
        \end{gathered}
    \end{align}
\end{subequations}

The relaxation of condition~\eqref{eq:lcsexistence2} in~\cref{thm:lcsexistence}
from strict inequality to condition~\eqref{eq:numericalexistence2}, which
also allows equality, means that LCSs are allowed to have finite thickness.
However, the set of criterions~\eqref{eq:numericalexistence} enforce that
all LCSs have uniquely defined local orientations. Conditions
\eqref{eq:lcsexistence3} and~\eqref{eq:numericalexistence3} are equivalent,
due to the orthogonality of the eigenvectors $\vct{\xi}_{1}(\vct{x}_{0})$ and
$\vct{\xi}_{2}(\vct{x}_{0})$, although the form~\eqref{eq:numericalexistence3}
turns out to be more advantageous for use in computations.

Because of the artificial extension of the main computational grid, cf.\
\cref{sub:generating_a_set_of_initial_conditions}, the Cauchy-Green strain
tensor could be calculated for the innermost of the padded rows and columns
by means of the same centered difference methods as described in
\cref{sec:calculating_the_cauchy_green_strain_tensor}. Thus, a similar
centered differencing approach was used in order to approximate the Hessian
matrices of the set of eigenvalues $\lambda_{2}(\vct{x}_{0})$ for the tracers
in the entire computational domain.

\subsection{A framework for computing smooth strainlines}
\label{sub:a_framework_for_computing_smooth_strainlines}

Per condition~\eqref{eq:numericalexistence3}, hyperbolic LCSs are composed
of material curves tangent to the $\vct{\xi}_{1}(\vct{x}_{0})$ vector field,
i.e., the eigenvector field associated with the smaller eigenvalue field
$\lambda_{1}(\vct{x}_{0})$ of the Cauchy-Green strain tensor field
$\mtrx{C}_{t_{0}}^{t}(\vct{x}_{0})$. The tensor lines tangent to the
$\vct{\xi}_{1}$-field will be referred to as \emph{strainlines}
in the following, a term coined by \textcite{farazmand2012computing}. Aside from
points within $\mathcal{U}$ which exhibit repeated eigenvalues and thus
oriental discontinuities in both eigenvector fields, strainlines can be computed
as smooth trajectories of the ordinary differential equation

\begin{equation}
    \label{eq:strainlinebasicode}
\vct{r}'=\vct{\xi}_{1}(\vct{r}),\quad\vct{r}\in\mathcal{U},%
    \quad\norm{\vct{\xi}_{1}(\vct{r})}=1.
\end{equation}

As pointed out by \textcite{onu2015lcstool}, the orientational discontinuities
of the $\vct{\xi}_{1}$-field are removable, through careful monitoring and
local reorientation. This process can be described in terms of three steps.
First, the nearest neighboring grid points are identified. Then, oriental
discontinuities inbetween the grid elements are identified by inspecting the
inner product of the $\vct{\xi}_{1}$ vectors of adjacent grid points. Rotations
exceeding $90\si{\degree}$ between pairs of neighboring vectors are labelled
as oriental discontinuities, which are corrected prior to the linear
interpolation by flipping the corresponding vectors by $180\si{\degree}$.
In the end, linear interpolation is used within the grid element.

Furthermore, should the $\vct{\xi}_{1}$-vector obtained from the local special-
purpose linear interpolation outlined above prove to be rotated by more than
$90\si{\degree}$ relative to the $\vct{\xi}_{1}$-vector from the previous level
of pseudotime used in the numerical integration of system
\eqref{eq:strainlinebasicode}, it would be flipped $180\si{\degree}$, by
the same logic as in the special linear interpolation. The entire process
of the special-purpose local linear interpolation method is outlined in
figure \ref{fig:locallinearinterp}.

\begin{figure}[htpb]
    \centering
    \def\svgwidth{0.8\linewidth}
    \begin{figure}[htpb]
    \centering
    \def\svgwidth{0.8\linewidth}
    \begin{figure}[htpb]
    \centering
    \def\svgwidth{0.8\linewidth}
    \input{figures/linearspecialinterp.pdf_tex}
    \caption[Illustration of the special linear interpolation used
    for the $\vct{\xi}_{1}$ eigenvector field]
    {Illustration of the special-purpose linear interpolation used to compute
        trajectories of the $\vct{\xi}_{1}$ eigenvector field. At point
        \textbf{(a)}, there is an orientational discontinuity at the lower
        right grid point, which is corrected by rotating the corresponding
        vector by $180\si{\degree}$ prior to linear interpolation. At point
        \textbf{(b)}, there is no orientational discontinuity. Lastly,
        at point \textbf{(c)}, the interpolated vector must be flipped due
    to the overall orientation of the trajectory.}
    \label{fig:locallinearinterp}
\end{figure}

    \caption[Illustration of the special linear interpolation used
    for the $\vct{\xi}_{1}$ eigenvector field]
    {Illustration of the special-purpose linear interpolation used to compute
        trajectories of the $\vct{\xi}_{1}$ eigenvector field. At point
        \textbf{(a)}, there is an orientational discontinuity at the lower
        right grid point, which is corrected by rotating the corresponding
        vector by $180\si{\degree}$ prior to linear interpolation. At point
        \textbf{(b)}, there is no orientational discontinuity. Lastly,
        at point \textbf{(c)}, the interpolated vector must be flipped due
    to the overall orientation of the trajectory.}
    \label{fig:locallinearinterp}
\end{figure}

    \caption[Illustration of the special linear interpolation used
    for the $\vct{\xi}_{1}$ eigenvector field]
    {Illustration of the special-purpose linear interpolation used to compute
        trajectories of the $\vct{\xi}_{1}$ eigenvector field. At point
        \textbf{(a)}, there is an orientational discontinuity at the lower
        right grid point, which is corrected by rotating the corresponding
        vector by $180\si{\degree}$ prior to linear interpolation. At point
        \textbf{(b)}, there is no orientational discontinuity. Lastly,
        at point \textbf{(c)}, the interpolated vector must be flipped due
    to the overall orientation of the trajectory.}
    \label{fig:locallinearinterp}
\end{figure}


So, in order to compute globally smooth strainlines, equation
\eqref{eq:strainlinebasicode} is altered in the following way:

\begin{equation}
    \label{eq:strainlineode}
    \vct{r}'(s)=\sgn\Big(\big\langle\vct{f}\big(\vct{r}(s)\big),%
    \vct{r}'(s-\Delta)\big\rangle\Big)\ \vct{f}\big(\vct{r}'(s)\big),
\end{equation}

where $\vct{f}$ denotes the special-purpose local linear interpolation of
the $\vct{\xi}_{1}$ field, as outlined above and in figure
\ref{fig:locallinearinterp}, $\Delta$ is the numerical step length
used in the numerical integration, while signum function is defined as

\begin{equation}
    \label{eq:signumfunction}
\sgn(x)=\begin{cases}1,&\textnormal{for}\ x>0\\
        0,&\textnormal{for}\ x=0\\
        -1,&\textnormal{for}\ x<0\end{cases}
\end{equation}

\subsection{Extracting hyperbolic LCSs from strainlines}
\label{sub:extracting_hyperbolic_lcss_from_strainlines}

If a material line $\mathcal{M}(t_{0})$ lies within a strainline, it
automatically fulfills condition~\eqref{eq:numericalexistence3}.
The segments of the strainlines on which the remaining conditions
\eqref{eq:numericalexistence1},~\eqref{eq:numericalexistence2} and
\eqref{eq:numericalexistence4} are satisfied comprises the set of hyperbolic
LCSs in the flow over time interval $[t_{0},t_{0}+T]$.
\textcite{farazmand2012computing} suggest that, in order to identify this
set of LCSs, one should start by identifying the subdomain
$\mathcal{U}_{0}\subset\mathcal{U}$ on which the conditions
\eqref{eq:numericalexistence1} and~\eqref{eq:numericalexistence2} are satisfied,
and then integrate the system given by equation~\eqref{eq:strainlineode} from
initial conditions within $\mathcal{U}_{0}$ to construct strainlines.
Generally, the integration proceeds until each strainline reaches the domain
boundaries of $\mathcal{U}$, or reaches a degenerate point of the original
$\vct{\xi}_{1}$ vector field.

The degenerate points of the $\vct{\xi}_{1}$ vector field is, as the name
implies, the set of points in $\mathcal{U}$ for which the eigenvalues
$\lambda_{1}(\vct{x}_{0})$ and $\lambda_{2}(\vct{x}_{0})$ are equal, leaving
the strain eigenvector field $\vct{\xi}_{1}$ undiscernible. As a computational
measure of this degeneracy, the scalar field defined as

\begin{equation}
    \label{eq:alphafield}
    \alpha(\vct{x}_{0})=\Big(\dfrac{%
                        \lambda_{2}(\vct{x}_{0})-\lambda_{1}(\vct{x}_{0})}%
                    {\lambda_{2}(\vct{x}_{0})+\lambda_{1}(\vct{x}_{0})}\Big)^{2}
\end{equation}

was used. For points $\vct{x}$ which did not coincide with the grid points
$\vct{x}_{i,j}$, the values $\lambda_{1}(\vct{x})$ and $\lambda_{2}(\vct{x})$
were found by means of regular linear interpolation. Wherever the value of
$\alpha(\vct{x})$ decreased below the predefined threshold of $10^{-6}$, the
point $\vct{x}$ was flagged as degenerate, thus stopping the strainline
integration.

Frequently, only a segment of any given strainline will qualify as a hyperbolic
LCS\@. Hence, the integration of any strainline can be stopped when it reaches
a point at which one of the conditions~\eqref{eq:numericalexistence1} or
\eqref{eq:numericalexistence2} fails. Doing so uncritically, however, opens up
the possibility of stopping a strainline which only exited the $\mathcal{U}_{0}$
domain due to numerical noise. In order to avoid such unwanted failures,
the approach of \textcite{farazmand2012computing} was followed, where
strainline integration is only stopped if one of the LCS conditions fail
repeatedly over the pre-set length $l_{f}=0.2$ of the strainline.

Now, having located the strainline pieces which satisfy conditions
\eqref{eq:numericalexistence1} and~\eqref{eq:numericalexistence2}, the next
step is imposing condition~\eqref{eq:numericalexistence4}, i.e., identifying the
strainline segments that are local maxima of the averaged maximum strain. The
suggested approach of \textcite{farazmand2012computing} is to define a set
$\mathcal{L}$ of uniformly spaced horizontal and vertical lines within the
domain $\mathcal{U}_{0}$, then comparing the values of
$\overline{\lambda}_{2}(\gamma_{0})$, the average of $\lambda_{2}$ on the curve
$\gamma_{0}$, at the neighboring intersections of all sufficiently close
strainline segments along each of the lines in $\mathcal{L}$. Intersections
between strainlines and the lines in $\mathcal{L}$ are to be found through
linear interpolation.

\begin{figure}[htpb]
    \centering
    \def\svgwidth{0.8\linewidth}
    \begin{figure}[htpb]
    \centering
    \def\svgwidth{0.8\linewidth}
    \begin{figure}[htpb]
    \centering
    \def\svgwidth{0.8\linewidth}
    \input{figures/neighborlcs.pdf_tex}
    \caption[The identification process of strainlines which
    are local strain maximizers]{Illustration of the identification process of
        strainlines which are local strain maximizers.
        Local maxima of $\overline{\lambda}_{2}(\gamma)$, the average of
        $\lambda_{2}$ over a curve $\gamma$, are found along a set $\mathcal{L}$
        of horizontal and vertical lines. The strainline segment
    $\gamma_{0}$ is identified as a local maximizer if it contains a locally
maximal $\overline{\lambda}_{2}$ value along at least one of the lines in
$\mathcal{L}$. The dashed ellipses indicate the regions outside which
adjacent intersections are excluded from the local maximization process.
The half-width of the ellipses is given as a pre-select threshold
$l_{\textnormal{nb}}=0.2$. Intersections of sufficiently close strainlines,
that is, strainlines whose intersection(s) fall within the dashed
ellipses, are included in the local comparison process for a given intersection
between a line in $\mathcal{L}$ and the strainline segment $\gamma_{0}$.}
    \label{fig:neighborlcs}
\end{figure}

    \caption[The identification process of strainlines which
    are local strain maximizers]{Illustration of the identification process of
        strainlines which are local strain maximizers.
        Local maxima of $\overline{\lambda}_{2}(\gamma)$, the average of
        $\lambda_{2}$ over a curve $\gamma$, are found along a set $\mathcal{L}$
        of horizontal and vertical lines. The strainline segment
    $\gamma_{0}$ is identified as a local maximizer if it contains a locally
maximal $\overline{\lambda}_{2}$ value along at least one of the lines in
$\mathcal{L}$. The dashed ellipses indicate the regions outside which
adjacent intersections are excluded from the local maximization process.
The half-width of the ellipses is given as a pre-select threshold
$l_{\textnormal{nb}}=0.2$. Intersections of sufficiently close strainlines,
that is, strainlines whose intersection(s) fall within the dashed
ellipses, are included in the local comparison process for a given intersection
between a line in $\mathcal{L}$ and the strainline segment $\gamma_{0}$.}
    \label{fig:neighborlcs}
\end{figure}

    \caption[The identification process of strainlines which
    are local strain maximizers]{Illustration of the identification process of
        strainlines which are local strain maximizers.
        Local maxima of $\overline{\lambda}_{2}(\gamma)$, the average of
        $\lambda_{2}$ over a curve $\gamma$, are found along a set $\mathcal{L}$
        of horizontal and vertical lines. The strainline segment
    $\gamma_{0}$ is identified as a local maximizer if it contains a locally
maximal $\overline{\lambda}_{2}$ value along at least one of the lines in
$\mathcal{L}$. The dashed ellipses indicate the regions outside which
adjacent intersections are excluded from the local maximization process.
The half-width of the ellipses is given as a pre-select threshold
$l_{\textnormal{nb}}=0.2$. Intersections of sufficiently close strainlines,
that is, strainlines whose intersection(s) fall within the dashed
ellipses, are included in the local comparison process for a given intersection
between a line in $\mathcal{L}$ and the strainline segment $\gamma_{0}$.}
    \label{fig:neighborlcs}
\end{figure}


Should a strainline segment prove to be a local maximizer along at least one
of the lines in $\mathcal{L}$, the strainline segment is labelled as a LCS\@.
Adjacent intersections who are separated by a distance larger than a preselect
threshold are excluded from the local maximization process. The process is
illustrated in figure~\ref{fig:neighborlcs}. What
\textcite{farazmand2012computing} fail to do, however, is describing an
objective and robust way of selecting the constituent lines of $\mathcal{L}$.
