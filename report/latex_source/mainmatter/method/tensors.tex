\section{Calculating the Cauchy-Green strain tensor}
\label{sec:calculating_the_cauchy_green_strain_tensor}

Making use of the auxiliary tracer points, as outlined in
\cref{sub:generating_a_set_of_initial_conditions}, the Jacobian of the flow
map was approximated by means of centered differences as
\begin{equation}
    \label{eq:auxjacobian}
    \vct{\nabla}\vct{F}_{t_{0}}^{t}(\vct{x}_{i,j}) \approx%
    \begin{pmatrix}\dfrac{\vct{F}_{t_{0}}^{t}(\vct{x}_{i,j}^{r})%
        -\vct{F}_{t_{0}}^{t}(\vct{x}_{i,j}^{l})}{2\delta{}x}, &
        \dfrac{\vct{F}_{t_{0}}^{t}(\vct{x}_{i,j}^{u})%
        -\vct{F}_{t_{0}}^{t}(\vct{x}_{i,j}^{d})}{2\delta{}y}
    \end{pmatrix}.
\end{equation}
The Cauchy-Green strain tensor was then calculated as per
\cref{eq:cauchygreen}. \textcite{farazmand2012computing} bring up the point
originally raised by \textcite{lekien2010computation}, that using only the
auxiliary tracers in the calculation of the eigenvalues and -vectors of the
Cauchy-Green strain tensor is not sufficient for measuring the amount of local
repulsion or attraction of the material lines. Their justification is that the
set of auxiliary tracers around a main tracer $\vct{x}_{i,j}$ tends to stay on
the same side of an LCS. This is due to the small initial separations
$\delta{}x$ and $\delta{}y$, resulting in the auxiliary tracers undergoing less
stretching than the main tracers which lay on opposite sides of repelling LCSs.
Thus, the magnitudes of eigen\emph{values} stemming from finite differencing
applied to the auxiliary tracers are generally underestimated. On the other
hand, the eigen\emph{vectors} are computed more accurately than they would be
by use of the main tracers alone.

Because of the tendency of finite differencing applied to the auxiliary tracers
to underestimate the magnitudes of the eigenvalues, an analogous approximation
of the Jacobian of the flow map was also made by means of the main tracers, as
\begin{equation}
    \label{eq:mainjacobian}
    \widetilde{\vct{\nabla}\vct{F}}\hspace{0ex}_{t_{0}}^{t}(\vct{x}_{i,j}) \approx%
    \begin{pmatrix}\dfrac{\vct{F}_{t_{0}}^{t}(\vct{x}_{i+1,j})%
        -\vct{F}_{t_{0}}^{t}(\vct{x}_{i-1,j})}{2\Delta{}x}, &
        \dfrac{\vct{F}_{t_{0}}^{t}(\vct{x}_{i,j+1})%
        -\vct{F}_{t_{0}}^{t}(\vct{x}_{i,j-1})}{2\Delta{}y}
    \end{pmatrix}.
\end{equation}
The extended grid, as outlined in
\cref{sub:generating_a_set_of_initial_conditions}, allowed for a consistent
centered differencing approximation for all the main tracers. This also
includes the first set of extended rows and columns, where the latter play an
important role in the analysis to follow --- more on that in
\cref{sec:identifying_lcs_candidates_numerically}. The numerical approximation
of the eigen\emph{values} of the Cauchy-Green strain tensor were thus calculated
by means of the main tracers, using~\cref{eq:mainjacobian}, while
the approximation of the corresponding eigen\emph{vectors} were calculated by
means of the auxiliary tracers, that is, using~\cref{eq:auxjacobian}.
\clearpage
