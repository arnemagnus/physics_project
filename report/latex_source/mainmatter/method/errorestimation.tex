\newpage
\section{Estimation of errors}
\label{sec:estimation_of_errors}

Aside from the qualitiative visual comparison between the obtained LCSs and
the reference LCS, a set of numerical errors was also computed, with a view to
use them in order to explain any visual discrepancies. To this end, the error
in the flow map, given by equation~\eqref{eq:flowmap}, was computed as the
root mean square deviation (hereafter abbreviated as RMSD) with regards to the
reference, as follows:
\begin{equation}
    \label{eq:rmsdflowmap}
    \rmsd_{\textnormal{flow map}} = \sqrt{\frac{1}{\widetilde{N}}%
        \sum\limits_{\vct{x}_{0}\,\in\,\widetilde{\mathcal{U}}}%
\big(\widehat{\vct{F}}{}_{t_{0}}^{t}(\vct{x}_{0})-\vct{F}_{t_{0}}^{t}{(\vct{x}_{0})}\big)^{2}},
\end{equation}
where the summation is over all of the tracers in the computational domain
$\widetilde{\mathcal{U}}$, including the tracers which constitute the artificially extended
domain \emph{and} the auxiliary tracers, as outlined in
\cref{sub:generating_a_set_of_initial_conditions}. $\widetilde{N}$ is the
total number of advected tracers, and
$\widehat{\vct{F}}{}_{t_{0}}^{t}(\vct{x}_{0})$ is the flow map approximated by
the numerical integrator in question.

The RMSD in both sets of eigen\emph{values}, $\lambda_{1}(\vct{x}_{0})$ and
$\lambda_{2}(\vct{x}_{0})$ of the Cauchy-Green strain tensor field, as given
by equation \eqref{eq:cauchygreencharacteristics} was also computed completely
analogously:
\begin{equation}
    \label{eq:rmsdlmbd}
    \rmsd_{\textnormal{eigenvalue}} = \sqrt{\frac{1}{N}%
    \sum\limits_{\vct{x}_{0}\,\in\,\mathcal{U}}%
\big(\widehat{\lambda}(\vct{x}_{0})-\lambda(\vct{x}_{0})\big)^{2}},
\end{equation}
where the summation is over all tracers within the computational domain
$\mathcal{U}$, for which the Hessians of the largest eigenvalue were computed,
as per equation~\eqref{eq:numericalexistence}. $N$ is the corresponding
number of tracers. $\widehat{\lambda}(\vct{x}_{0})$ is the approximation
of the true eigenvalue $\lambda(\vct{x}_{0})$ located at $\vct{x}_{0}$, obtained
by means of the numerical integrator in question.

The best way of estimating the error in the computed eigen\emph{vectors} of the
Cauchy-Green Strain tensor field is by means of their orientation, because
the eigenvectors themselves are normalized to unit length. Since
the iteration of strainlines was performed in both directions of pseudotime,
as described in \cref{sub:a_framework_for_computing_smooth_strainlines},
a $180\si{\degree}$ shift in direction would not influence the LCS calculation
in any way. Thus, the eigenvector directions were computed as azimuthal angles
in the interval $[0\si{\degree},180\si{\degree})$. The RMSD for the direction of
the eigenvector fields was then calculated as follows:
\begin{equation}
    \label{eq:rmsddirection}
    \begin{gathered}
        \vct{\xi}(\vct{x}_{0}) = \begin{pmatrix}\xi_{x}(\vct{x}_{0}),&%
            \xi_{y}(\vct{x}_{0})
        \end{pmatrix},\quad\phi(\vct{x}_{0}) = \arctan{\bigg(\frac{\xi_{y}(\vct{x}_{0})}{\xi_{x}(\vct{x}_{0})}\bigg)},\\
        \rmsd_{\textnormal{eigenvector direction}} = \sqrt{\frac{1}{N}%
    \sum\limits_{\vct{x}_{0}\,\in\,\mathcal{U}}%
{\big(\widehat{\phi}{(\vct{x}_{0})}-\phi{(\vct{x}_{0})}\big)}^{2}},
    \end{gathered}
\end{equation}
where the conventions for $\mathcal{U}$ and $N$ are the same as for the
RMSD of the eigenvalues. $\widehat{\phi}(\vct{x}_{0})$ denotes the
azimuthal angle of the eigenvector located at $\vct{x}_{0}$, found by means of
the numerical integrator in question.
\clearpage

Regarding the error in the computed LCS curves, false positives and false
negatives should be treated separately, as they would influence the overall
flow patterns in the system in different ways. Because numerical noise
could result in LCS curve segments being erroneously identified as either,
a lower numerical threshold $l_{\textnormal{noise}}=0.01$, that is,
ten times the numerical integration step used in order to compute the
constituent strainlines, was used. Any point on a LCS curve which was farther
away from all points on the reference LCS than $l_{\textnormal{noise}}$ was
identified as a false positive. Likewise, any point on the reference LCS curve
which was farther away from all points on the LCS curve under consideration
than $l_{\textnormal{noise}}$ was flagged as a false negative. In order to
estimate the accumulated offset of the false LCS segments, the midpoint
rule was used;
\begin{equation}
    \label{eq:midpointfalselcs}
    \textnormal{Offset}_{\textnormal{false LCSs}} = %
    \sum\limits_{\substack{\vct{x}\,\in\,\gamma\\\vct{x}\,\notin\,\widehat{\gamma}}}%
\min\limits_{\widehat{\vct{x}}\,\in\,\widehat{\gamma}}%
\norm{\vct{x}-\widehat{\vct{x}}}\cdot\Delta,
\end{equation}
where the sum is over all points with a greater offset than the aforementioned
$l_{\textnormal{noise}}$, and $\Delta=10^{-3}$ is the numerical integration step
length used in order to compute strainlines. Concerning false positives,
$\widehat{\gamma}$ denotes the reference LCS curve, and $\gamma$ the
approximation under consideration. For false negatives, the roles are reversed.

Lastly, in order to obtain a quantitative measure of the offset in the
LCS curve segments which comply with the reference LCS (that is, to the
numerical threshold $l_{\textnormal{noise}}$), the root mean square
of the distance separating each point on the LCS curve with the nearest point
on the reference, was calculated:
\begin{equation}
    \label{eq:rmsdlcs}
    \rmsd_{\textnormal{LCS}} = \sqrt{%
        \frac{1}{\breve{N}}\sum\limits_{\vct{x}\,\in\,\gamma}%
    \min\limits_{\widehat{\vct{x}}\,\in\,\widehat{\gamma}}{\norm{\vct{x}-\widehat{\vct{x}}}}^{2}},
\end{equation}
where the term $\rmsd$ is used somewhat loosely, in order to conform with
notation used for the other measures of error introduced in this section.
$\breve{N}$ is the number of points which constitute
$\widehat{\gamma}$, the LCS curve found by means of the numerical
integrator in question, which are within the distance $l_{\textnormal{noise}}$
of any point on $\gamma$, the reference LCS curve. The idea
is that LCS curves need not necessarily start from, or end at, the same points
in space. For this reason, a robust way of estimating the error
of a given LCS curve is the summation of the smallest distances
separating each point on the curve from the reference.
