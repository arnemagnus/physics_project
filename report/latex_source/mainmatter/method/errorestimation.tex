\newpage
\section{Estimation of errors}
\label{sec:estimation_of_errors}

Aside from the qualitiative visual comparison between the obtained LCSs and
the reference LCS, a set of numerical errors was also computed, with a view to
use them in order to explain any visual discrepancies. To this end, the error
in the flow map (see~\cref{eq:flowmap}), was computed as the
root mean square deviation (hereafter abbreviated to RMSD) with regards to the
reference, as follows:
\begin{equation}
    \label{eq:rmsdflowmap}
    \rmsd_{\textnormal{flow map}} = \sqrt{\frac{1}{\widetilde{N}}%
        \sum\limits_{\vct{x}_{0}\,\in\,\widetilde{\mathcal{U}}}%
\big(\widehat{\vct{F}}{}_{t_{0}}^{t}(\vct{x}_{0})-\vct{F}_{t_{0}}^{t}{(\vct{x}_{0})}\big)^{2}},
\end{equation}
where the summation is over all of the tracers in the extended computational domain
$\widetilde{\mathcal{U}}$, that is, both the main and auxiliary tracers which
originate from points in the extended grid, as outlined in
\cref{sub:generating_a_set_of_initial_conditions}. $\widetilde{N}$ is the
total number of advected tracers,
$\widehat{\vct{F}}{}_{t_{0}}^{t}(\vct{x}_{0})$ is the flow map approximated by
the numerical integrator in question, and $\vct{F}_{t_{0}}^{t}$ is the
reference flow map.

The RMSD in both sets of eigenvalues, $\lambda_{1}(\vct{x}_{0})$ and
$\lambda_{2}(\vct{x}_{0})$ of the Cauchy-Green strain tensor field, as given
by~\cref{eq:cauchygreencharacteristics} was also computed analogously:
\begin{equation}
    \label{eq:rmsdlmbd}
    \rmsd_{\textnormal{eigenvalue}} = \sqrt{\frac{1}{N}%
        \sum\limits_{\vct{x}_{0}\,\in\,\mathcal{U}^{\prime}}%
\big(\widehat{\lambda}(\vct{x}_{0})-\lambda(\vct{x}_{0})\big)^{2}},
\end{equation}
where the summation is over all main tracers located in (or at the boundary
of) the domain $\mathcal{U}^{\prime}$, the borders of which are governed by the
first set of rows and columns in the extended grid. $N$ is the
corresponding total number of main tracers in $\mathcal{U}^{\prime}$,
$\widehat{\lambda}(\vct{x}_{0})$ is the approximation of
the eigenvalue located at $\vct{x}_{0}$, obtained
by means of the numerical integrator in question, and $\lambda(\vct{x}_{0})$
is the reference eigenvalue located at $\vct{x}_{0}$.

Because the eigenvectors of the Cauchy-Green strain tensor field are normalized
to unit length, a reasonable way of estimating the error in the computed
eigenvectors is by considering their orientation. Thus, the eigenvector
directions were computed as azimuthal angles. The RMSD for the direction of
the eigenvector fields was then calculated as follows:
\begin{equation}
    \label{eq:rmsddirection}
    \begin{gathered}
        \vct{\xi}(\vct{x}_{0}) = \begin{pmatrix}\xi_{x}(\vct{x}_{0}),&%
            \xi_{y}(\vct{x}_{0})
        \end{pmatrix},\quad\phi(\vct{x}_{0}) = \arctan{\bigg(\frac{\xi_{y}(\vct{x}_{0})}{\xi_{x}(\vct{x}_{0})}\bigg)},\\
        \rmsd_{\textnormal{eigenvector direction}} = \sqrt{\frac{1}{N}%
    \sum\limits_{\vct{x}_{0}\,\in\,\mathcal{U}}%
{\big(\widehat{\phi}{(\vct{x}_{0})}-\phi{(\vct{x}_{0})}\big)}^{2}},
    \end{gathered}
\end{equation}
where the conventions for $\mathcal{U}$ and $N$ are the same as for the
RMSD of the eigenvalues, given in~\cref{eq:rmsdlmbd}.
$\widehat{\phi}(\vct{x}_{0})$ denotes the azimuthal angle of the eigenvector
located at $\vct{x}_{0}$, found by means of
the numerical integrator in question, and $\phi(\vct{x}_{0})$ denotes the
angle of the reference eigenvector located at $\vct{x}_{0}$.
%\clearpage

Regarding the error in the computed LCS curves, false positives, that is,
any parts of the computed LCS curves which are not present in
the reference curve, and false negatives, i.e., any parts of the reference curve
which is not present in the computed LCS curves, should be treated separately.
This is because they would influence the predictions of overall flow patterns
in the system in different ways. Numerical noise could result in LCS curve
segments being erroneously identified as either a false positive or a false
negative. In order to reduce this sort of error, a lower numerical threshold
$l_{\textnormal{noise}}=0.01$, that is, ten times the numerical integration
step $\Delta=10^{-3}$ used in order to compute the constituent strainlines, was
used in identifying false positives and negatives; any point on an LCS curve
which was farther away from all points on the reference LCS than
$l_{\textnormal{noise}}$ was identified as a false positive.
Similarly, any point on the reference LCS curve which was farther away from all
points on the LCS curve under consideration than $l_{\textnormal{noise}}$ was
flagged as a false negative.

Let $\widehat{\gamma}$ represent the parametrization of the computed LCS approximation
under consideration, and $\gamma$ the parametrization of the
reference LCS. In order to estimate the offset of the false LCS segments, the
area between these segments and the reference LCS curve was approximated by the
midpoint rule for numerical integration;
\begin{equation}
    \label{eq:midpointfalselcs}
    \textnormal{Offset}_{\textnormal{false LCSs}} = %
    \sum\limits_{\substack{\vct{x}\,\in\,\gamma\\\vct{x}\,\notin\,\widehat{\gamma}}}%
\min\limits_{\widehat{\vct{x}}\,\in\,\widehat{\gamma}}%
\norm{\widehat{\vct{x}}-\vct{x}}\cdot\Delta,
\end{equation}
where the sum is over all points with a greater offset than the aforementioned
$l_{\textnormal{noise}}$. Regarding false negatives, the roles of
$\gamma$ and $\widehat{\gamma}$ are reversed. Per~\cref{eq:strainlineode},
$\Delta$, the previously mentioned numerical integration step length used in
order to compute strainlines, equals the distance separating consecutive
points in the parametrization of the LCS curves. This is due to the eigenvectors
$\vct{\xi}_{1}$ being normalized to unit length. The calculation of this offset
is illustrated in figure~\ref{fig:fp_fn_principle}.

\begin{figure}[htpb]
    \centering
    \def\svgwidth{0.8\linewidth}
    \input{figures/falsepositives.pdf_tex}
    \caption[Illustration of how the offset of false LCS segments was computed]%
    {Illustration of how the offset of false LCS segments was computed.
        A segment of the reference LCS is denoted $\gamma$. Two segments of
        an LCS approximation are denoted $\widetilde{\gamma}_{0}$ and
        $\widehat{\gamma}_{0}$, respectively. When one or more such segments
        are within a pre-set length $l_{\textnormal{noise}}=0.01$ of
        $\gamma$ --- where the borders of this region are indicated by
        dashed lines --- any other segments which are further away are labelled
        as false positives. Accordingly, $\widehat{\gamma}_{0}$ is recognized
        as a false positive in the above figure. Then, the shaded area
        $\mathcal{A}$ between the reference LCS and the curve segments
        identified as false positives is estimated by means of the midpoint rule.
        That is, for each point on a false positive segment of the approximated
        LCS, the smallest distance separating it and the reference is weighted
        with $\Delta$, the step length used in the numerical integration of
        strainlines. Because the $\vct{\xi}_{1}$-field is normalized to unit
        length, $\Delta$ equals the distance between consecutive points in
        the parametrizations of both the approximated and the reference LCS
        curves, per~\cref{eq:strainlineode}. In the case of false negatives,
        no LCS curve segments fall within $l_{\textnormal{noise}}$, in which
        case $\mathcal{A}$ denotes the minimal area between the reference
        LCS curve and the approximated LCS curve, for each point of the
        reference LCS segment which is not found in the approximation.
        In the figure, this would correspond to the absence of
        $\widetilde{\gamma}_{0}$, if $\widehat{\gamma}_{0}$ was then the
    part of the LCS approximation closest to $\gamma$.}
%
%        In the opposite case
%        An LCS segment, denoted by $\widetilde{\gamma}_{0}$ in the figure, is
%        compared with the reference LCS, labelled $\gamma_{0}$. Each part
%        the approximated LCS curve, which is farther away from all points on
%        the reference LCS than a pre-set length $l_{\textnormal{noise}}=0.01$
%        is flagged as a false positive. The area $\mathcal{A}$ between
%        the reference LCS and the curve segments identified as false positives,
%        is estimated by means of the midpoint rule; that is, for each point on
%        a false positive segment of the approximated LCS, the smallest
%        distance separating it and the reference LCS is weighted with $\Delta$,
%        the step length used in the numerical integration of strainlines.
%        Because the $\vct{\xi}_{1}$-field is normalized to unit length,
%        $\Delta$ equals the distance between consecutive points in the
%        parametrizations of both the approximated and the reference LCS curves,
%       % per equation~\eqref{eq:strainlineode}. In the case of false negatives,
%        $\mathcal{A}$ denotes the minimal area between the reference LCS curve
%        and the approximated LCS curve, for each part of the reference LCS
%        which is not present in the approximation; that is, each point on the
%        reference LCS curve which is farther away from all points on the
%        approximate curve than $l_{\textnormal{noise}}$.}
    \label{fig:fp_fn_principle}
\end{figure}


%\clearpage
Lastly, in order to obtain a quantitative measure of the offset in the
LCS curve segments which comply with the reference LCS (that is, to the
numerical threshold $l_{\textnormal{noise}}$), the RMS distance separating each
point on the LCS curve with the nearest point on the reference, was calculated:
\begin{equation}
    \label{eq:rmsdlcs}
    \rmsd_{\textnormal{LCS}} = \sqrt{%
        \frac{1}{\breve{N}}\sum\limits_{\vct{x}\,\in\,\gamma}%
    \min\limits_{\widehat{\vct{x}}\,\in\,\widehat{\gamma}}{\norm{\widehat{\vct{x}}-\vct{x}}}^{2}},
\end{equation}
where the term $\rmsd$ is used somewhat loosely, in order to conform with
notation used for the other measures of error introduced in this section.
$\breve{N}$ is the number of points which constitute
$\widehat{\gamma}$, the LCS curve found by means of the numerical
integrator in question, which are within the distance $l_{\textnormal{noise}}$
of any point on $\gamma$, the reference LCS curve. The idea
is that LCS curves need not necessarily start from nor end at the same points
in space. Thus, a robust way of estimating the error
of a given LCS curve is the summation of the smallest distances
separating each point on it from the reference.
