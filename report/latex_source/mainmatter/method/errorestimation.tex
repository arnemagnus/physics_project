\newpage
\section{Estimation of errors}
\label{sec:estimation_of_errors}

Aside from the qualitiative visual comparison between the obtained LCSs and
the reference LCS, a set of numerical errors was also computed, with a view to
use them in order to explain any visual discrepancies. To this end, the error
in the flow map, given by equation~\eqref{eq:flowmap}, was computed as the
root mean square deviation (hereafter abbreviated as RMSD) with regards to the
reference, as follows:

\begin{equation}
    \label{eq:rmsdflowmap}
    \rmsd_{\textnormal{flow map}} = \sqrt{\frac{1}{N}%
    \sum\limits_{\vct{x}_{0}\,\in\,\mathcal{U}}%
\big(\widehat{\vct{F}}{}_{t_{0}}^{t}(\vct{x}_{0})-\vct{F}_{t_{0}}^{t}{(\vct{x}_{0})}\big)^{2}},
\end{equation}

where the summation is over all of the tracers in the computational domain
$\mathcal{U}_{0}$, $N$ is the total number of tracers, and
$\widehat{\vct{F}}{}_{t_{0}}^{t}(\vct{x}_{0})$ is the flow map approximated by
the numerical integrator in question. The RMSD in both sets of
eigen\emph{values}, $\lambda_{1}(\vct{x}_{0})$ and $\lambda_{2}(\vct{x}_{0})$ of the Cauchy-Green
strain tensor field, as given by equation
\eqref{eq:cauchygreencharacteristics} was also computed completely
analogously.

The best way of estimating the error in the computed eigen\emph{vectors} of the
Cauchy-Green Strain tensor field is by means of their orientation, because
the eigenvectors themselves are normalized to unit length. Since
the iteration of strainlines was performed in both directions of pseudotime,
as described in \cref{sub:a_framework_for_computing_smooth_strainlines},
a $180\si{\degree}$ shift in direction would not influence the LCS calculation
in any way. Thus, the eigenvector directions were computed as azimuthal angles
in the interval $[0\si{\degree},180\si{\degree})$. The RMSD for the direction of
the eigenvector fields was then calculated as follows:

\begin{equation}
    \label{eq:rmsddirection}
    \begin{gathered}
        \vct{\xi}(\vct{x}_{0}) = \begin{pmatrix}\xi_{x}(\vct{x}_{0}),&%
            \xi_{y}(\vct{x}_{0})
        \end{pmatrix},\quad\phi(\vct{x}_{0}) = \arctan{\bigg(\frac{\xi_{y}(\vct{x}_{0})}{\xi_{x}(\vct{x}_{0})}\bigg)},\\
        \rmsd_{\textnormal{eigenvector direction}} = \sqrt{\frac{1}{N}%
    \sum\limits_{\vct{x}_{0}\,\in\,\mathcal{U}}%
{\big(\widehat{\phi}{(\vct{x}_{0})}-\phi{(\vct{x}_{0})}\big)}^{2}},
    \end{gathered}
\end{equation}

where, once again, the summation is over all of the tracers in the computational
domain $\mathcal{U}_{0}$, $N$ is the total number of tracers, while
$\widehat{\phi}(\vct{x}_{0})$ denotes the azimuthal angle of the eigenvector
located at $\vct{x}_{0}$, found by means of the numerical integrator in
question.

Lastly, in order to obtain a quantitative measure of the compliance between
the obtained LCSs and the reference LCS, the root mean square of the distance
separating each point on the LCS curve with the nearest point on the reference
LCS curve, was used:

\begin{equation}
    \label{eq:rmsdlcs}
    \rmsd_{\textnormal{LCS}} = \sqrt{%
        \frac{1}{\widetilde{N}}\sum\limits_{\vct{\widehat{x}}\,\in\,\widehat{\gamma}}%
\min\limits_{\vct{x}\,\in\,\gamma}{\norm{\vct{\widehat{x}}_{0}-\vct{x}}}^{2}},
\end{equation}

where $\widehat{\gamma}$ is the LCS curve, found by means of the numerical
integrator in question, and $\gamma$ is the reference LCS curve. The idea
is that LCS curves need not necessarily start from, or end at, the same points
in space. For this reason, the most robust way of estimating the error
of a given LCS curve simply becomes the summation of the smallest distances
separating each point on the curve from the reference.
