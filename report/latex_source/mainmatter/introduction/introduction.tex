When analyzing complex dynamical systems, such as nonlinear many-body problems,
the conventional approach to predicting future states by means of simulating
the trajectories of point particles is frequently insufficient. This is due
to the resulting predictions being very sensitive to small changes in time
and initial positions. One way of addressing the delicate dependence on initial
conditions is to run several different models for the same underlying physical
systems. This invariably results in larger, more well-resolved distributions
of the transported particles, at the cost of ignoring key, inherent, organizing
structures in the system.

Recently, the concept of Lagrangian coherent structures saw the light of day,
emerging from the intersection between nonlinear dynamics, that is, the
underlying mathematical principles of chaos theory, and fluid dynamics. These
provide a new framework for understanding transport phenomena in concept
fluid flow systems. Lagrangian coherent structures can be described as
time-evolving `landscapes' in a multidimensional space, which dictate flow
patterns in dynamical systems. In particular, such structures define the
interfaces of dynamically distinct, invariant regions. An invariant region is
characterized as a domain where all particle trajectories that originate within
the region, remain in it, although the region itself can move and deform with
time. So, simply put; Lagrangian coherent structures enable us to make
predictions regarding the future states of flow systems.

There are two possible ways of describing fluid flow. The Eulerian approach
is to consider the properties of a flow field at each fixed point in time and
space. An example is the concept of velocity fields, which produce the
local and instantaneous velocities of all fluid elements within the domain
under consideration. The Lagrangian point of view, on the other hand, concerns
the developing velocity of each individual particle along their paths, as they
are transported by the flow. Unlike the Eulerian perspective, the Lagrangian
mindset is objective, as in frame-invariant. That is, properties of Lagrangian
fields are unchanged by time-dependent translations and rotations of the
reference frame. For unsteady flow systems, which are more common than
steady flow systems in nature, there exists no self-evident preferred frame
of reference. This means that any transport-dictating dynamic structures
should hold for \emph{any} choice of reference frame. Furthermore, this is
the main rationale for which \emph{Lagrangian}, rather
than \emph{Eulerian}, coherent structures have been pursued.

Although the framework for detecting Lagrangian coherent structures is
mathematically valid for any number of dimensions, the focus of this project
work has been two-dimensional flow systems. Many natural processes fall
within this category, perhaps most notably the transport of debris and
contaminations, such as radioactive particles or the remnants of an oil
spill, by means of oceanic surface currents. Being able to successfully predict
where such particles will be taken by the flow enables us to isolate and
extract them before they are able to reach the coastline, thus preventing
potential humanitarian and natural calamities.

\newpage

At the heart of detecting Lagrangian coherent structures lies the advection
of fluid elements by means of a velocity field, which describes the system
under consideration. This rings true both for test cases, where the velocity
profile is known analytically, and for real-life systems, typically described
by means of discrete samples of the instantaneous velocity field. For this
project work, the topic of interest is how the detection of Lagrangian coherent
structures depends on the choice of numerical integration method used in order
to compute the aforementioned particle transport. In particular, four singlestep
methods and four embedded, adaptive step length methods, each with different
properties, were used in order to advect a collection of fluid elements
by means of an analytically known, two-dimensional, unsteady velocity field.




%\begin{framed}
%    \begin{itemize}
%        \item \sout{Lagrangian coherent structure: A structure that can separate dynamically
%            distinct invariant regions. Invariant retions: All trajectories starting
%            out within it, remains within the region, although the region itself
%        may move and deform with time.}
%    \item \sout{Lagrangian coherent structure: Landscapes in multidimensional
%        landscapes, shaping the flow patterns in dynamical systems.}
%    \item \sout{Motivation: Complex systems, i.e., many-particle nonlinear
%        systems: Need computational shortcuts}
%    \item \sout{LCS: Principally robust structures which enable us
%                        to make predictions regarding the future states of
%                    a flow system.}
%                \item \sout{Examples of real-world application:}
%                            \begin{itemize}
%                                \item \sout{Population growth}
%                                \item \sout{Predicting the advection of particles by means of oceanic currents or wind}
%    \begin{itemize}
%        \item \sout{Recent examples: The volcanic eruptions at Eyafjällajökul, and
%            at Bali (2017)}
%    \end{itemize}
%\item \sout{Predicting where the remnants of an oil spill will end up --> Where to focus
%    the rescue operation in the short and medium term}
%                            \end{itemize}
%    \end{itemize}
%\end{framed}
