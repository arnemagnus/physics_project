When analyzing complex dynamical systems, such as the nonlinear many-body
problems arising in transport phenomena by virtue of oceanic currents or
atmospheric winds, the conventional approach to predicting future states
by means of simulating the trajectories of phase points is frequently
insufficient. This is due to the resulting predictions being very sensitive to
small changes in time and initial positions. One way of addressing the delicate
dependence on initial conditions is to run different models for the same
underlying physical systems, of increasing spatial and temporal resolution. This
sort of approach is made possible because the fundamental dynamics are known ---
yet, for complex transport systems, the computational cost quickly grows beyond
the available resources, in terms of computation time or memory. For many
practical purposes, however, microscopic details matter little in comparison to
the overarching trends in the system, which means that a less ambitious
approach, merely aiming to understand the macroscopics of the transport
phenomenon, is often justifiable.

At the turn of the millennium, the concept of Lagrangian coherent structures
saw the light of day, emerging from the intersection between nonlinear dynamics,
that is, the underlying mathematical principles of chaos theory, and fluid
dynamics \parencite{haller2000lagrangian}. These provide a new framework for
understanding transport phenomena in conceptual fluid flow systems. Lagrangian
coherent structures can be described as time-evolving `landscapes' in a
multidimensional space, which dictate macroscopical flow patterns in dynamical
systems. In particular, such structures define the interfaces of dynamically
distinct, invariant regions. An invariant region in fluid dynamics is
characterized as a domain where all particle trajectories that originate within
the region, remain in it, although the region itself can move and deform with
time. So, simply put; Lagrangian coherent structures enable us to make
predictions regarding the future states of flow systems.

There are two possible perspectives regarding the description of fluid flow.
The Eulerian approach is to consider the properties of a flow field at a set of
fixed points in time and space. An example is the concept of velocity fields,
which produce the local and instantaneous velocities at all points within their
domains. The Lagrangian point of view, on the other hand, concerns the
developing velocity of each fluid element along their paths, as they are
transported by the flow. Unlike the Eulerian perspective, the Lagrangian
mindset is objective, as in frame-invariant. That is, properties of Lagrangian
fields are unchanged by time-dependent translations and rotations of the
reference frame. For unsteady flow systems, which are more common than steady
flow systems in nature, there exists no self-evident preferred frame of
reference. Thus, any transport-dictating dynamical structures should
hold for \emph{any} choice of reference frame. This is the main
rationale for which \emph{Lagrangian}, rather than \emph{Eulerian}, coherent
structures have been pursued.

A generic flow system can be described as a structure whose state depends on
flowing streams of energy, material or information. Conventional examples of
flow systems include the transport of physical properties such as pressure,
temperature or matter in fluids, and the transport of charge in electrical
currents. A large variety of phenomena can reasonably be modelled as flow
systems, such as the classical harmonic oscillator, or the interaction between
predator and prey in closed systems
\parencite[parts I--II]{strogatz2014nonlinear}. In doing so, valuable pieces
of insight can be obtained from well-understood properties of generic flows.
In recent years, analyses based upon Lagrangian coherent structures have been
conducted for a variety of naturally occuring phenomena which are not
commonly considered as flow systems. Two prominent examples are how
\textcite{olascoaga2008tracing} used Lagrangian coherent structures in order
to forecast the development of toxic algae in the ocean, and
\textcite{ali2007lagrangian} used Lagrangian coherent structures to predict
the formation and stability of human crowd patterns. As these examples
emphasize, the theory of Lagrangian coherent structures is applicable to a
wide range of systems.

Although the framework for detecting Lagrangian coherent structures is
mathematically valid for any number of dimensions, the focus of this project
work has been two-dimensional flow systems. Many natural processes can
reasonably be described as two-dimensional, perhaps most notably the transport
of debris and contaminations, such as garbage patches or the remnants of
an oil spill, by means of oceanic surface currents. Being able to successfully
predict where such particles will be taken by the flow could enable us to
isolate them and accelerate the cleanup process before the particles are able
to reach the coastline, thus mitigating potential humanitarian and natural
calamities.

%\newpage
%
At the heart of detecting Lagrangian coherent structures lies the advection
of (generalized) fluid elements by means of a velocity field, which describes
the system under consideration. This is true both for test cases, where the
velocity profile is known analytically, and for real-life systems, typically
described by means of some model for the instantaneous velocity field. For this
project work, the topic of interest is how the detection of Lagrangian coherent
structures depends on the choice of numerical integration method used in order
to compute the aforementioned particle transport. In particular, four singlestep
methods and four embedded, adaptive step length methods, each with different
properties, were used in order to advect a collection of fluid elements by means
of an analytically known, two-dimensional, unsteady velocity field.

This thesis is structured based on the idea that readers possessing
at least an undergraduate level of knowledge of physics and mathematics, in
addition to a rudimentary understanding of programming, should be able to
understand and repeat the numerical investigations which have been conducted.
To this end, the immediately forthcoming chapter contains a description of the
various numerical integration schemes that were utilized, in addition to a
generic yet brief mathematical description of the kind of flow systems
considered and the Lagrangian coherent structures situated therein, the latter
based on variational theory. In the ensuing chapter, we present a transport
model frequently used in literature --- whose Lagrangian coherent structures
are documented --- in addition to a description of how the variational
principles of Lagrangian coherent structure detection were implemented
numerically. Lastly, the results are presented and discussed, before the
conclusions of the project as a whole are drawn.


%\begin{framed}
%    \begin{itemize}
%        \item \sout{Lagrangian coherent structure: A structure that can separate dynamically
%            distinct invariant regions. Invariant retions: All trajectories starting
%            out within it, remains within the region, although the region itself
%        may move and deform with time.}
%    \item \sout{Lagrangian coherent structure: Landscapes in multidimensional
%        landscapes, shaping the flow patterns in dynamical systems.}
%    \item \sout{Motivation: Complex systems, i.e., many-particle nonlinear
%        systems: Need computational shortcuts}
%    \item \sout{LCS: Principally robust structures which enable us
%                        to make predictions regarding the future states of
%                    a flow system.}
%                \item \sout{Examples of real-world application:}
%                            \begin{itemize}
%                                \item \sout{Population growth}
%                                \item \sout{Predicting the advection of particles by means of oceanic currents or wind}
%    \begin{itemize}
%        \item \sout{Recent examples: The volcanic eruptions at Eyafjällajökul, and
%            at Bali (2017)}
%    \end{itemize}
%\item \sout{Predicting where the remnants of an oil spill will end up --> Where to focus
%    the rescue operation in the short and medium term}
%                            \end{itemize}
%    \end{itemize}
%\end{framed}
