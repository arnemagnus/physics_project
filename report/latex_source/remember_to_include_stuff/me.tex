\chapter{Introduction}
Only interested in hyperbolic LCSs as they are the only ones relevant for
transport barriers.

\section{Complex systems --> Need shortcuts}

\section{Intuitively, what is an LCS?}

\section{LCS definition}

\section{Different types of LCSs}

\section{Hyperbolic LCSs}
--> Connect to application

\chapter{Theory}
\section{Solving ODE systems}
---> General ODE systems
---> Numerical integrators dump
--> Interpolation necessary for discrete systems

\section{Flowmaps}
---> Introduce system and limitations
--> Introduce the concept of a flow map

\section{LCS definition}
--> Different kinds of LCSs (hyperbolic, elliptic and parabolic, cf. LCS tool)
--> More mathematical definitions? Ask Th{\"o}r

\section{FTLE as LCS predictor}
--> Prone to false positives and negatives
--> Definition somewhat arbitrary (what is a ridge)?
--> Strogatz' motivation as a simple explanation of why we consider it at all?

\section{Identify hyperbolic LCS from variational theory}
--> Mathematically involved.

\chapter{Tool!}

\section{Adveksjon}
--> Si noe om system, glatte vektorfelt/hastighetsfelt
--> Integrasjonsteknikker

\section{CG tensors}
--> Auxiliary grid
--> Extended grids i fire retninger
--> Beregn CG tensors
--> Centered differencing, consistently for all main particles
--> Har med gitterpunkter på utsiden av hoveddomenet for å inkludere
    diskontinuitet i oppførsel i hastighetsfeltet

\section{Eigenvalues/Eigenvectors}
--> Auxiliary grid
--> Laplacian, extended grid layer 2 for centered differencing

\section{Identify AB domain}
--> Klargjør måten vi tolket Laplacian på

\section{Compute strainlines}
--> Define G0 along vertical and horizontal lin
--> Avoid redundant computations of trajectories
--> Integrate forwards and backwards
----> (Notice that strainlines "fall out" of AB domain, likely due to num. error)
--> Special linear interpolation with local direction correction
--> Higher order spline interpolations are inappropriate because of oriental
--> discontinuities (in case of vectors) and great variance (in case of evals)
--> Linear spline interpolation without orientation fix caused random turns
--> at discontinuities.
--> Stop criteria
--> Alpha scaling introduced by Haller gave unpredictable leaps
-----> After linear interp?
--> Used just one integrator here, because [...]
--> Choice of integration step (needs test!)
--> Note that this step is very sensitive to the flow map details,
--> components in the strain tensors down to the $10^-15$ level.
--> LCS results sensitive to continious failure length, needed to increase
-----> it in order to replicate results from Haller due to different AB domain

\section{Identify intersections}
--> Which lines and why (maximize intersections with as few lines as possible)
--> Include all intersetions between a strainline and a vert / horz linear

\section{Identify neighbors}
--> Neighbor length essential for LCS results

\section{Select LCSs}
---> Identify LCS as local maxima of $\lambda_{2}$ which are also long enough
--> Needs at least one neighbor other than itself
--> \textbf{Cut tail of strainlines which exit AB domain and do not return}
--> That part is no LCS!
--> Parts/sections of strainlines may qualify as LCSs

\chapter{Experiments}
--> What did we try and why?
