\section{Weeks 34--36}
\label{sec:34-36}

\subsection{Preliminaries}
\label{sec:preliminaries}
I've spent the first few weeks of the project work studying the
dynamics, i.e., the transport properties,  of an analytically known
velocity field. The velocity field is a simplified
dynamical model of a periodic, two dimensional double gyre system,
first described by \textcite{shadden2005definition}. It is
defined on the domain $[0,2]\times[0,1]$, and described mathematically
by the Lagrange stream function

\begin{equation}
  \label{eq:analytical_field_stream_function}
  \psi(\vect{x},t) = A\sin\big(\pi{}f(x,t)\big)\sin(\pi{}y)
\end{equation}

where 

\begin{subequations}
  \label{eq:analytical_vel_field_parameter_f}
  \begin{align}
    f(x,t) &= a(t)x^{2} + b(t)x \\
    \label{eq:analytical_velocity_field_parameter_a}
    a(t) &= \epsilon \sin(\omega{t}) \\
    \label{eq:analytical_velocity_field_parameter_b}
    b(t) &= 1 - 2\epsilon \sin(\omega{}t)
  \end{align}
\end{subequations}

and the parameters $A$, $\epsilon$ and $\omega$ adjust the properties
of the system.

The stream function description only applies to incompressible, i.e.,
divergence-free, two-dimensional flows. Per definition, the velocity
field relates to the stream function in the following way:

\begin{subequations}
  \begin{align}
    \label{eq:general_velocity_field}
    \vect{V} &= \mqty[u \\ v] \\
    \label{eq:general_velocity_field_x_component}
    u &= - \pdv{\psi}{y} \\
    \label{eq:general_velocity_field_y_component}
    v &= \pdv{\psi}{x}
  \end{align}
\end{subequations}

which, for the stream function given by eqn.
\eqref{eq:analytical_field_stream_function}, yields

\begin{subequations}
  \begin{align}
    u &= - \pi{}A\sin\big(\pi{}f(x,t)\big)\cos(\pi{}y) \\
    v &= \pi{}A\cos\big(\pi{}f(x,t)\big)\sin(\pi{}y)\pdv{f(x,t)}{x}
  \end{align}
\end{subequations}

with $f(x,t)$ is given by eqn. 
\eqref{eq:analytical_vel_field_parameter_f}.

\subsection{Considerations regarding simulation of
                    the system dynamics}
\label{sec:initial_simulation_considerations}
To begin with, i.e., at time $t=0$, I covered the domain
$[0,2]\times[0,1]$ with a mesh of equidistant fluid elements,
approximated as point masses on the grid points. Because the velocity
field is known analytically, simulating how the fluid elements are
transported in time amounted to solving a set of independent ordinary
differential eqn.s (ODEs) in time. A myriad of numerical solution
techniques for ODE systems exist, which is reflected in my project
problem theme, namely ``Sensitivity to numerical integration scheme in
calculation of transport barriers''. I simulated the transport of the
fluid elements in a set amount of time using a variety of solution
schemes, with intention to estimate how the precision of the fluid
elements' trajectories is affected by the choice of numerical scheme.

Put simply, all numerical integration schemes have their strengths
and weaknesses. Simple schemes involve few computational steps,
at the cost of precision. Similarly, schemes with a high degree of
precision are more complex, and typically require a larger number
of intermediary calculations in order to arrive at a solution.
Basically, my project work is centered around the tradeoff between
the precision of numerical solutions of dynamic systems, and the
time we have to wait for the calculations to finish.

Naturally, devising a way to estimate the degree of precision of the
numerical solutions produced by the different integration schemes is
crucial. One glaring hindrance lies in the fact that the fluid
elements' trajectories can be expressed analytically, but only in
terms of integrals that can only be evaluated numerically. Thus,
simply using a solution from a higher order numerical scheme -- using
a small time step (for fixed step size integrators) or low tolerance
levels (for adaptive step size integrators) -- as reference strikes
me as the most practical approach.

\subsection{Numerical integrators implemented so far}
\label{sec:initial_numerical_integrators}
So far, I've implemented the four most common single-step explicit
Runge-Kutta methods. They are summarized in table 
\ref{tab:initial_explicit_integrators}. 

\begin{table}[h]
  \centering
  \caption{The most common explicit Runge-Kutta integrators, and their
           precision, here indicated by their leading error terms.}
  \label{tab:initial_explicit_integrators}
  \begin{tabular}[h]{l|c}
    \hline
    (Explicit) integrator & Leading error term \\
    \hline
    Euler's method & $\mathcal{O}(\Delta{}t)$ \\
    Heun's method & $\mathcal{O}(\Delta{}t^{2})$ \\
    Kutta's method & $\mathcal{O}(\Delta{}t^{3})$ \\
    RK4 method & $\mathcal{O}(\Delta{}t^{4})$ \\
    \hline
  \end{tabular}
\end{table}



Høyere ordens numerisk metode må brukes som referanse, ettersom
posisjonen til enhver partikkel ikke kan uttrykkes analytisk (på lukket
form). Adaptiv-tidssteg-metoder krever en viss finesse, ettersom hver
enkelt bane vil trenge sitt eget tidssteg osv.




\subsection{Error estimation, various integrators}

Using the RK4 method with timestep $\Delta{}t=0.001$ as reference, I
have simulated the fluid flow until $t=5$ with the integration schemes
named in table \ref{tab:initial_explicit_integrators} for time step
sizes $\Delta{}t\in\{0.1,0.01,0.001\}$ and computed the error in the
FTLE field. The errors of each direct solver are presented in
\crefrange{tab:euler-lyapunov-error-rk4-ref}{tab:RK4-lyapunov-error-rk4-ref}. 

\begin{tcolorbox}
  Introduce the concept of a (Finite-Time) Lyapunov Exponent field
  in an introductory chapter.

  Explicitly provide the Runge-Kutta methods, e.g. Euler, Heun, Kutta
  and RK4, in the theory section. State the leading error term.

  Investigate array slicing and utilization of adaptive timestep
  integrators as reference solution.

  Compare numerical error of the different integrators with the
  reference solution -- e.g. logarithmic plot of the error as function
  of the time step, with reference slopes $\sim{}h^{n}$.
\end{tcolorbox}

\parencite{dormand1986reconsideration}

\begin{table}[h]
  \caption{Euler, $t = 5$}
  \label{tab:euler-lyapunov-error-rk4-ref}
  \centering
  \begin{tabular}[h]{l|c|c|c}
    \hline
    Euler \textbackslash\hspace{0.5em} $\Delta{t}$& 0.1 & 0.01 & 0.001 \\
    \hline
    Avg. abs. err. & $5 \cdot 10^{-3}$ & $4 \cdot 10^{-4} $ & $4 \cdot 10^{-5}$ \\
    Max. abs. err. & $7 \cdot 10^{-2}$ & $1 \cdot 10^{-2}$ & $1 \cdot 10^{-3} $ \\
    \hline
  \end{tabular}
\end{table}

\begin{table}[h]
  \caption{Heun, $t = 5$}
  \label{tab:heun-lyapunov-error-rk4-ref}
  \centering
  \begin{tabular}[h]{l|c|c|c}
    \hline
    Heun \textbackslash\hspace{0.5em} $\Delta{t}$& 0.1 & 0.01 & 0.001 \\
    \hline
    Avg. abs. err. & $1 \cdot 10^{-4}$ & $1 \cdot 10^{-6} $ & $1 \cdot 10^{-8}$ \\
    Max. abs. err. & $5 \cdot 10^{-3}$ & $5 \cdot 10^{-5}$ & $5 \cdot 10^{-7} $ \\
    \hline
  \end{tabular}
\end{table}

\begin{table}[h]
  \caption{Kutta, $t = 5$}
  \label{tab:kutta-lyapunov-error-rk4-ref}
  \centering
  \begin{tabular}[h]{l|c|c|c}
    \hline
    Kutta \textbackslash\hspace{0.5em} $\Delta{t}$& 0.1 & 0.01 & 0.001 \\
    \hline
    Avg. abs. err. & $4 \cdot 10^{-6}$ & $4 \cdot 10^{-9} $ & $4 \cdot 10^{-12}$ \\
    Max. abs. err. & $2 \cdot 10^{-4}$ & $2 \cdot 10^{-7}$ & $3 \cdot 10^{-10} $ \\
    \hline
  \end{tabular}
\end{table}

\begin{table}[h]
  \caption{RK4, $t = 5$}
  \label{tab:RK4-lyapunov-error-rk4-ref}
  \centering
  \begin{tabular}[h]{l|c|c}
    \hline
    RK4 \textbackslash\hspace{0.5em} $\Delta{t}$& 0.1 & 0.01 \\
    \hline
    Avg. abs. err. & $7 \cdot 10^{-8}$ & $7 \cdot 10^{-12} $ \\
    Max. abs. err. & $4 \cdot 10^{-6}$ & $4 \cdot 10^{-10}$ \\
    \hline
  \end{tabular}
\end{table}

